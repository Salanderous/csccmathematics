\documentclass{ximera}


\graphicspath{
  {./}
  {ximeraTutorial/}
  {basicPhilosophy/}
}

\newcommand{\mooculus}{\textsf{\textbf{MOOC}\textnormal{\textsf{ULUS}}}}

\usepackage{tkz-euclide}\usepackage{tikz}
\usepackage{tikz-cd}
\usetikzlibrary{arrows}
\tikzset{>=stealth,commutative diagrams/.cd,
  arrow style=tikz,diagrams={>=stealth}} %% cool arrow head
\tikzset{shorten <>/.style={ shorten >=#1, shorten <=#1 } } %% allows shorter vectors

\usetikzlibrary{backgrounds} %% for boxes around graphs
\usetikzlibrary{shapes,positioning}  %% Clouds and stars
\usetikzlibrary{matrix} %% for matrix
\usepgfplotslibrary{polar} %% for polar plots
\usepgfplotslibrary{fillbetween} %% to shade area between curves in TikZ
\usetkzobj{all}
\usepackage[makeroom]{cancel} %% for strike outs
%\usepackage{mathtools} %% for pretty underbrace % Breaks Ximera
%\usepackage{multicol}
\usepackage{pgffor} %% required for integral for loops



%% http://tex.stackexchange.com/questions/66490/drawing-a-tikz-arc-specifying-the-center
%% Draws beach ball
\tikzset{pics/carc/.style args={#1:#2:#3}{code={\draw[pic actions] (#1:#3) arc(#1:#2:#3);}}}



\usepackage{array}
\setlength{\extrarowheight}{+.1cm}
\newdimen\digitwidth
\settowidth\digitwidth{9}
\def\divrule#1#2{
\noalign{\moveright#1\digitwidth
\vbox{\hrule width#2\digitwidth}}}






\DeclareMathOperator{\arccot}{arccot}
\DeclareMathOperator{\arcsec}{arcsec}
\DeclareMathOperator{\arccsc}{arccsc}

















%%This is to help with formatting on future title pages.
\newenvironment{sectionOutcomes}{}{}


\author{Jim Talamo}
\license{Creative Commons 3.0 By-bC}


\outcome{}


\begin{document}
\begin{exercise}

Determine whether the series:

\[
\sum_{k=1}^{\infty} \frac{(-1)^k}{k^2+5k+7}
\]

converges absolutely, converges conditionally, or diverges by completing the exercise that follows.

We begin by checking for:

\begin{multipleChoice}
\choice[correct]{absolute convergence.  If the series converges, we still need to determine if $\sum_{k=1}^{\infty} |a_k|$ converges in order to classify whether the convergence is absolute or conditional.}
\choice{conditional convergence.  The Alternating Series Test is easy to apply here.}
\end{multipleChoice}

Thus, we will check whether the series:

\[
\sum_{k=1}^{\infty} \left|  \frac{(-1)^k}{k^2+5k+7} \right| = \sum_{k=1}^{\infty} \frac{1}{k^2+5k+7}
\]

converges.  Now that the summand is positive, we can use more tests!  Since the summand is a rational expression in $k$, the test to use is the:
\begin{multipleChoice}
\choice{Ratio Test}
\choice{Root Test}
\choice[correct]{Limit Comparison Test}
\end{multipleChoice}

\begin{exercise}
Which of the following would be a good series to use as a point of comparison?

\begin{multipleChoice}
\choice{$\sum_{k=1}^{\infty} \frac{1}{k}$}
\choice[correct]{$\sum_{k=1}^{\infty} \frac{1}{k^2}$}
\choice{$\sum_{k=1}^{\infty} \frac{1}{k^3}$}
\choice{$\sum_{k=1}^{\infty} \left(\frac{1}{2}\right)^k$}
\end{multipleChoice}

So, let $b_n = \frac{1}{n^2}$.  Then, $L = \lim_{n \to \infty} \frac{a_n}{b_n} = \answer{1}$, so by the Limit Comparison Test, both series either converge or both series diverge.

Now, the chosen series is:
\begin{multipleChoice}
\choice{is a geometric series with $|r|<1$.  It converges.}
\choice{is a geometric series with $|r|\geq1$.  It diverges.}
\choice[correct]{is a $p$-series with $p>1$.  It converges.}
\choice{is a $p$-series with $p \leq 1$.  It diverges.}
\end{multipleChoice}

Hence:

\begin{multipleChoice}
\choice[correct]{$\sum_{k=1}^{\infty} \frac{1}{k^2+5k+7}$ converges.}
\choice{$\sum_{k=1}^{\infty} \frac{1}{k^2+5k+7}$ diverges.}
\end{multipleChoice}

So, the original series $\sum_{k=1}^{\infty} \frac{(-1)^k}{k^2+5k+7} $:
\begin{multipleChoice}
\choice[correct]{converges absolutely.}
\choice{converges absolutely but might diverge.}
\end{multipleChoice}

\begin{exercise}
Note that we could have applied the Alternating Series Test to check for convergence first.  Indeed the series is alternating since we can write it in the form $\sum_{k=1}^{\infty} (-1)^k a_k$, where $a_k = \answer{\frac{1}{k^2+5k+7}}$. Also $a_n$ is decreasing and $\lim_{n \to \infty} a_n = \answer{0}$, so:

\begin{multipleChoice}
\choice[correct]{the series converges by the Alternating Series Test.}
\choice{the series diverges by the Alternating Series Test.}
\choice{the Alternating Series Test is inconclusive.}
\end{multipleChoice}

If we were only asked to check if the series converges, this would have been an easier route, but since we are asked to classify the convergence, it's better here to check for absolute convergence first.
\end{exercise}

\end{exercise}
\end{exercise}
\end{document}
