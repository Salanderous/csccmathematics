\documentclass{ximera}


\graphicspath{
  {./}
  {ximeraTutorial/}
  {basicPhilosophy/}
}

\newcommand{\mooculus}{\textsf{\textbf{MOOC}\textnormal{\textsf{ULUS}}}}

\usepackage{tkz-euclide}\usepackage{tikz}
\usepackage{tikz-cd}
\usetikzlibrary{arrows}
\tikzset{>=stealth,commutative diagrams/.cd,
  arrow style=tikz,diagrams={>=stealth}} %% cool arrow head
\tikzset{shorten <>/.style={ shorten >=#1, shorten <=#1 } } %% allows shorter vectors

\usetikzlibrary{backgrounds} %% for boxes around graphs
\usetikzlibrary{shapes,positioning}  %% Clouds and stars
\usetikzlibrary{matrix} %% for matrix
\usepgfplotslibrary{polar} %% for polar plots
\usepgfplotslibrary{fillbetween} %% to shade area between curves in TikZ
\usetkzobj{all}
\usepackage[makeroom]{cancel} %% for strike outs
%\usepackage{mathtools} %% for pretty underbrace % Breaks Ximera
%\usepackage{multicol}
\usepackage{pgffor} %% required for integral for loops



%% http://tex.stackexchange.com/questions/66490/drawing-a-tikz-arc-specifying-the-center
%% Draws beach ball
\tikzset{pics/carc/.style args={#1:#2:#3}{code={\draw[pic actions] (#1:#3) arc(#1:#2:#3);}}}



\usepackage{array}
\setlength{\extrarowheight}{+.1cm}
\newdimen\digitwidth
\settowidth\digitwidth{9}
\def\divrule#1#2{
\noalign{\moveright#1\digitwidth
\vbox{\hrule width#2\digitwidth}}}






\DeclareMathOperator{\arccot}{arccot}
\DeclareMathOperator{\arcsec}{arcsec}
\DeclareMathOperator{\arccsc}{arccsc}

















%%This is to help with formatting on future title pages.
\newenvironment{sectionOutcomes}{}{}


\author{Jim Talamo}
\license{Creative Commons 3.0 By-bC}


\outcome{}


\begin{document}
\begin{exercise}

Consider the series:

\[
\sum_{k=1}^{\infty} \frac{(-1)^k}{k^p}
\] 

Select the option below that describes all of the $p$-values for which the series converges:

\begin{multipleChoice}
\choice{$p \geq1$}
\choice{$p > 1$}
\choice{$p =1$}
\choice{$0 < p \leq 1$}
\choice{$0 < p < 1$}
\choice[correct]{$p >0$}
\choice{There are no $p$-values for which the series converges.}
\choice{The series converges for all $p$-values.}
\end{multipleChoice}

Select the option below that describes all of the $p$-values for which the series converges \emph{absolutely}:

\begin{multipleChoice}
\choice{$p \geq1$}
\choice[correct]{$p > 1$}
\choice{$p =1$}
\choice{$0 < p \leq 1$}
\choice{$0 < p < 1$}
\choice{$p >0$}
\choice{There are no $p$-values for which the series converges absolutely.}
\choice{The series converges absolutely for all $p$-values.}
\end{multipleChoice}

Select the option below that describes all of the $p$-values for which the series converges \emph{conditionally}:

\begin{multipleChoice}
\choice{$p \geq1$}
\choice{$p > 1$}
\choice{$p =1$}
\choice[correct]{$0 < p \leq 1$}
\choice{$0 < p < 1$}
\choice{$p >0$}
\choice{There are no $p$-values for which the series converges conditionally.}
\choice{The series converges conditionallyfor all $p$-values.}
\end{multipleChoice}

\begin{hint}
The alternating series test can be used for the first part.  For the remaining two parts, remember that we say a series $\sum_{k=1}^{\infty} a_k$:

\begin{itemize}
\item \emph{converges absolutely} if $\sum_{k=1}^{\infty} |a_k|$ converges.
\item  \emph{converges conditionally} if $\sum_{k=1}^{\infty} a_k$ converges but $\sum_{k=1}^{\infty} |a_k|$ diverges.
\end{itemize}

Use this as well as the results for $p$-series.
\end{hint}

\end{exercise}
\end{document}