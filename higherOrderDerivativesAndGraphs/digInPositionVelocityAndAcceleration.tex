\documentclass{ximera}


\graphicspath{
  {./}
  {ximeraTutorial/}
  {basicPhilosophy/}
}

\newcommand{\mooculus}{\textsf{\textbf{MOOC}\textnormal{\textsf{ULUS}}}}

\usepackage{tkz-euclide}\usepackage{tikz}
\usepackage{tikz-cd}
\usetikzlibrary{arrows}
\tikzset{>=stealth,commutative diagrams/.cd,
  arrow style=tikz,diagrams={>=stealth}} %% cool arrow head
\tikzset{shorten <>/.style={ shorten >=#1, shorten <=#1 } } %% allows shorter vectors

\usetikzlibrary{backgrounds} %% for boxes around graphs
\usetikzlibrary{shapes,positioning}  %% Clouds and stars
\usetikzlibrary{matrix} %% for matrix
\usepgfplotslibrary{polar} %% for polar plots
\usepgfplotslibrary{fillbetween} %% to shade area between curves in TikZ
\usetkzobj{all}
\usepackage[makeroom]{cancel} %% for strike outs
%\usepackage{mathtools} %% for pretty underbrace % Breaks Ximera
%\usepackage{multicol}
\usepackage{pgffor} %% required for integral for loops



%% http://tex.stackexchange.com/questions/66490/drawing-a-tikz-arc-specifying-the-center
%% Draws beach ball
\tikzset{pics/carc/.style args={#1:#2:#3}{code={\draw[pic actions] (#1:#3) arc(#1:#2:#3);}}}



\usepackage{array}
\setlength{\extrarowheight}{+.1cm}
\newdimen\digitwidth
\settowidth\digitwidth{9}
\def\divrule#1#2{
\noalign{\moveright#1\digitwidth
\vbox{\hrule width#2\digitwidth}}}






\DeclareMathOperator{\arccot}{arccot}
\DeclareMathOperator{\arcsec}{arcsec}
\DeclareMathOperator{\arccsc}{arccsc}

















%%This is to help with formatting on future title pages.
\newenvironment{sectionOutcomes}{}{}


\title[Dig-In:]{Position, velocity, and acceleration}

\outcome{Interpret the second derivative of a position function as acceleration.}
\outcome{Calculate higher order derivatives.}


\begin{document}
\begin{abstract}
  Here we discuss how position, velocity, and acceleration relate to
  higher derivatives.
\end{abstract}
\maketitle

Studying functions and their derivatives might seem somewhat
abstract. However, consider this passage from a physics book:
\begin{quote}
  Assuming acceleration $a$ is constant, we may write velocity and
  position as
  \begin{align*}
    v(t) &= v_0 + at,\\
    x(t) &= x_0 + v_0 t + (1/2) a t^2,
  \end{align*}
  where $a$ is the (constant) acceleration, $v_0$ is the velocity at
  time zero, and $x_0$ is the position at time zero.
\end{quote}
These equations model the position and velocity of any object with
constant acceleration. In particular these equations can be used to
model the motion of a falling object, since the acceleration due to
gravity is constant.

Calculus allows us to see the connection between these
equations. First note that the derivative of the formula for position with respect to time, is the formula for velocity with respect to
time.
\[
x'(t) = v_0 + at = v(t).
\]
Moreover, the derivative of formula for velocity with respect to time,
is simply $a$, the acceleration.

\begin{question}
  A ball has been tossed at time $t=0$.  Suppose that $s(t)$ represents the position of a ball at time $t$.
  Recalling that the acceleration for $t>0$ is only due to
  gravity, and knowing that the acceleration due to gravity is
  $-9.8~\mathrm{m}/\mathrm{s}^2$, what is $s''(t)$?
  \begin{prompt}
    \[
  s''(t) = \answer[given]{-9.8}
  \]
  \end{prompt}
\end{question}



\begin{example}
You recently took a road trip from Columbus Ohio to Urbana-Champaign
Illinois. The distance traveled from Columbus Ohio is roughly modeled
by:
\[
s(t) = 36t^2 -4.8t^3 \qquad\text{(miles West of Columbus)}
\]
where $t$ is measured in hours, and is between $0$ and $6$. Find a
formula for your acceleration.
  \begin{explanation}
    Here we simply need to find the second derivative:
    \[
    s'(t) = \answer[given]{72 t- 14.4 t^2} \qquad{and}\qquad s''(t) = \answer[given]{72 - 28.8t}
    \]
    Hence our acceleration is $72-28t~\text{miles/hour}^2$.
  \end{explanation}
\end{example}


\end{document}
