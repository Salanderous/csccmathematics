\documentclass{ximera}


\graphicspath{
  {./}
  {ximeraTutorial/}
  {basicPhilosophy/}
}

\newcommand{\mooculus}{\textsf{\textbf{MOOC}\textnormal{\textsf{ULUS}}}}

\usepackage{tkz-euclide}\usepackage{tikz}
\usepackage{tikz-cd}
\usetikzlibrary{arrows}
\tikzset{>=stealth,commutative diagrams/.cd,
  arrow style=tikz,diagrams={>=stealth}} %% cool arrow head
\tikzset{shorten <>/.style={ shorten >=#1, shorten <=#1 } } %% allows shorter vectors

\usetikzlibrary{backgrounds} %% for boxes around graphs
\usetikzlibrary{shapes,positioning}  %% Clouds and stars
\usetikzlibrary{matrix} %% for matrix
\usepgfplotslibrary{polar} %% for polar plots
\usepgfplotslibrary{fillbetween} %% to shade area between curves in TikZ
\usetkzobj{all}
\usepackage[makeroom]{cancel} %% for strike outs
%\usepackage{mathtools} %% for pretty underbrace % Breaks Ximera
%\usepackage{multicol}
\usepackage{pgffor} %% required for integral for loops



%% http://tex.stackexchange.com/questions/66490/drawing-a-tikz-arc-specifying-the-center
%% Draws beach ball
\tikzset{pics/carc/.style args={#1:#2:#3}{code={\draw[pic actions] (#1:#3) arc(#1:#2:#3);}}}



\usepackage{array}
\setlength{\extrarowheight}{+.1cm}
\newdimen\digitwidth
\settowidth\digitwidth{9}
\def\divrule#1#2{
\noalign{\moveright#1\digitwidth
\vbox{\hrule width#2\digitwidth}}}






\DeclareMathOperator{\arccot}{arccot}
\DeclareMathOperator{\arcsec}{arcsec}
\DeclareMathOperator{\arccsc}{arccsc}

















%%This is to help with formatting on future title pages.
\newenvironment{sectionOutcomes}{}{}


\outcome{Compute average rate of change}
\outcome{Compute instantaneous rate of change as the limit of the average rate of change}

\author{Nela Lakos \and Kyle Parsons}

\begin{document}
\begin{exercise}

An oil tank is to be drained for cleaning.  There are $V$ gallons of oil left in the tank $t$ minutes after the draining has begun, where 
\[
V(t) = 45(60-t)^2.
\]

The average rate at which the oil drains in the time interval $\left[0,15\right]$ is
\[
AR = \answer{-4725}\text{gal/min}.
\]

The average rate at which the oil drains in the time interval $\left[10,15\right]$ is
\[
AR = \answer{-4275}\text{gal/min}.
\]

The rate at which the oil drains 15 minutes after draining has begun is
\[
R = \answer{-4050}\text{gal/min}.
\]

The average rate at which the oil drains during the time interval $\left[15,15+h\right]$ for $0<h<1$ or the interval $\left[15+h,15\right]$ for $-1<h<0$ is
\[
AR(h) = \answer{-4050 + 45h}\text{gal/min}.
\]

The limit as $h$ goes to zero of the above average rate is
\[
\lim_{h\to0}AR(h) = \answer{-4050}\text{gal/min}.
\]

\end{exercise}
\end{document}