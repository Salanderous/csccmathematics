\documentclass{ximera}


\graphicspath{
  {./}
  {ximeraTutorial/}
  {basicPhilosophy/}
}

\newcommand{\mooculus}{\textsf{\textbf{MOOC}\textnormal{\textsf{ULUS}}}}

\usepackage{tkz-euclide}\usepackage{tikz}
\usepackage{tikz-cd}
\usetikzlibrary{arrows}
\tikzset{>=stealth,commutative diagrams/.cd,
  arrow style=tikz,diagrams={>=stealth}} %% cool arrow head
\tikzset{shorten <>/.style={ shorten >=#1, shorten <=#1 } } %% allows shorter vectors

\usetikzlibrary{backgrounds} %% for boxes around graphs
\usetikzlibrary{shapes,positioning}  %% Clouds and stars
\usetikzlibrary{matrix} %% for matrix
\usepgfplotslibrary{polar} %% for polar plots
\usepgfplotslibrary{fillbetween} %% to shade area between curves in TikZ
\usetkzobj{all}
\usepackage[makeroom]{cancel} %% for strike outs
%\usepackage{mathtools} %% for pretty underbrace % Breaks Ximera
%\usepackage{multicol}
\usepackage{pgffor} %% required for integral for loops



%% http://tex.stackexchange.com/questions/66490/drawing-a-tikz-arc-specifying-the-center
%% Draws beach ball
\tikzset{pics/carc/.style args={#1:#2:#3}{code={\draw[pic actions] (#1:#3) arc(#1:#2:#3);}}}



\usepackage{array}
\setlength{\extrarowheight}{+.1cm}
\newdimen\digitwidth
\settowidth\digitwidth{9}
\def\divrule#1#2{
\noalign{\moveright#1\digitwidth
\vbox{\hrule width#2\digitwidth}}}






\DeclareMathOperator{\arccot}{arccot}
\DeclareMathOperator{\arcsec}{arcsec}
\DeclareMathOperator{\arccsc}{arccsc}

















%%This is to help with formatting on future title pages.
\newenvironment{sectionOutcomes}{}{}


\author{Jim Talamo}
\license{Creative Commons 3.0 By-bC}


\outcome{Compute a Taylor Polynomial}


\begin{document}
\begin{exercise}
Find the first, second, third, and fourth degree Taylor polynomials centered at $x=2$ for the function $f(x) = \ln(5-2x)$.

\begin{align*}
p_1(x) &= \answer{-2(x-2)} \\
p_2(x) &= \answer{-2(x-2)-2(x-2)^2} \\
p_3(x) &= \answer{-2(x-2)-2(x-2)^2-\frac{8}{3}(x-2)^3} \\
p_4(x) &= \answer{-2(x-2)-2(x-2)^2-\frac{8}{3}(x-2)^3-4(x-2)^4}
\end{align*}

\begin{hint}
Make sure that your answer is a polynomial in powers of $(x-2)$!

The relationship between the coefficients of the Taylor Polynomial to the derivatives of the function that it approximates for $k=0,1,2, 3, 4$ is given by:

\[
a_k = \frac{f^{(k)}(c)}{k!}
\]
where $c$ is the $x$-value at which the series is centered.  Here, $c=\answer{2}$.  

\begin{question}
Complete the table below:

\begin{tabular}{|c|c|c|c|}
\hline
$k$ \quad & \quad \quad $f^{(k)}(x)$  \quad \quad & \quad \quad $f^{(k)}(2)$ \quad \quad & \quad \quad $a_k = \frac{f^{(k)}(2)}{k!}$ \quad \quad \\
\hline 
$0$ \quad & \quad \quad $\answer{\ln(5-2x)}$  \quad \quad & \quad \quad $\answer{0}$ \quad \quad  & \quad \quad $\answer{0}$ \quad \quad \\
\hline
$1$ \quad & \quad \quad $\answer{-2(5-2x)^{-1}}$ \quad \quad & \quad \quad $\answer{-2}$ \quad \quad & \quad \quad  $\answer{-2}$ \quad \quad  \\
\hline
$2$ \quad & \quad \quad $\answer{-4(5-2x)^{-2}}$ \quad \quad & \quad \quad $\answer{-4}$ \quad \quad & \quad \quad  $\answer{-2}$ \quad \quad  \\
\hline
$3$ \quad & \quad \quad $\answer{-16(5-2x)^{-3}}$ \quad \quad & \quad \quad $\answer{-16}$ \quad \quad & \quad \quad  $\answer{-\frac{8}{3}}$ \quad \quad  \\
\hline
$4$ \quad & \quad \quad $\answer{-96(5-2x)^{-4}}$ \quad \quad & \quad \quad $\answer{-96}$ \quad \quad & \quad \quad  $\answer{-4}$ \quad \quad  \\
\hline
\end{tabular}

Hence, the fourth degree Taylor polynomial for $f(x) =\sqrt{2x-1}$ is:

\begin{align*}
p_4(x) &=a_0+a_1(x-2)+a_2(x-2)^2+a_3(x-2)^3 \\
&= \answer{0}+\answer{-2}(x-2)+\answer{-2}(x-2)^2+\answer{-\frac{8}{3}}(x-2)^3+\answer{-4}(x-2)^4\\
\end{align*}

From this, we can easily write down the first, second, and third degree Taylor polynomials at $x=2$:
\begin{align*}
p_1(x) &= \answer{-2(x-2)} \\
p_2(x) &= \answer{-2(x-2)-2(x-2)^2} \\
p_3(x) &= \answer{-2(x-2)-2(x-2)^2-\frac{8}{3}(x-2)^3}
\end{align*}

\end{question}

\end{hint}

\begin{exercise}
The Taylor polynomials are used to approximate the value of a function near the center.  As long as we are ``close enough" (which we will formalize in a later section), the higher order polynomials should give better approximations.  Suppose that we want to approximate $\ln(.6)$.

The \emph{exact} value is found using the function, provided we use the correct $x$-value.  Since $f(x) = \ln(5-2x)$, the $x$-value that gives $\ln(.6)$ is $x=\answer{2.2}$. 

\begin{hint}
To find this, just set $5-2x = 2.2$.
\end{hint}

Thus, the \emph{exact} answer to 6 decimal places is $\ln(.6) = \answer[tolerance= .000002]{-.510826}$

We can approximate $\ln(5-2x)$ by substituting $x=2.2$ into the Taylor polynomials.  In fact:

\begin{exercise}
The first degree Taylor polynomial approximation to 6 decimal places is: $\ln(.6) \approx p_1(2.2) = \answer[tolerance=.000001]{-.400000}$.

The second degree Taylor polynomial approximation to 6 decimal places is: $\ln(.6) \approx p_2(2.2) = \answer[tolerance=.000001]{-.480000}$.

The third degree Taylor polynomial approximation to 6 decimal places is: $\ln(.6) \approx p_3(2.2) = \answer[tolerance=.000001]{-.501333}$.

The fourth degree Taylor polynomial approximation to 6 decimal places is: $\ln(.6) \approx p_4(2.2) = \answer[tolerance=.000001]{-.507733}$.

Do the higher order polynomials provide more accurate results?

\begin{multipleChoice}
\choice[correct]{Yes}
\choice{No}
\end{multipleChoice}
\end{exercise}
\end{exercise}

\end{exercise}
\end{document}
