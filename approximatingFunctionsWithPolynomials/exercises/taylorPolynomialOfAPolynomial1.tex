\documentclass{ximera}


\graphicspath{
  {./}
  {ximeraTutorial/}
  {basicPhilosophy/}
}

\newcommand{\mooculus}{\textsf{\textbf{MOOC}\textnormal{\textsf{ULUS}}}}

\usepackage{tkz-euclide}\usepackage{tikz}
\usepackage{tikz-cd}
\usetikzlibrary{arrows}
\tikzset{>=stealth,commutative diagrams/.cd,
  arrow style=tikz,diagrams={>=stealth}} %% cool arrow head
\tikzset{shorten <>/.style={ shorten >=#1, shorten <=#1 } } %% allows shorter vectors

\usetikzlibrary{backgrounds} %% for boxes around graphs
\usetikzlibrary{shapes,positioning}  %% Clouds and stars
\usetikzlibrary{matrix} %% for matrix
\usepgfplotslibrary{polar} %% for polar plots
\usepgfplotslibrary{fillbetween} %% to shade area between curves in TikZ
\usetkzobj{all}
\usepackage[makeroom]{cancel} %% for strike outs
%\usepackage{mathtools} %% for pretty underbrace % Breaks Ximera
%\usepackage{multicol}
\usepackage{pgffor} %% required for integral for loops



%% http://tex.stackexchange.com/questions/66490/drawing-a-tikz-arc-specifying-the-center
%% Draws beach ball
\tikzset{pics/carc/.style args={#1:#2:#3}{code={\draw[pic actions] (#1:#3) arc(#1:#2:#3);}}}



\usepackage{array}
\setlength{\extrarowheight}{+.1cm}
\newdimen\digitwidth
\settowidth\digitwidth{9}
\def\divrule#1#2{
\noalign{\moveright#1\digitwidth
\vbox{\hrule width#2\digitwidth}}}






\DeclareMathOperator{\arccot}{arccot}
\DeclareMathOperator{\arcsec}{arcsec}
\DeclareMathOperator{\arccsc}{arccsc}

















%%This is to help with formatting on future title pages.
\newenvironment{sectionOutcomes}{}{}


\author{Jim Talamo}
\license{Creative Commons 3.0 By-bC}


\outcome{Understand the relationship between a Taylor Polynomial and a polynomial}


\begin{document}
\begin{exercise}
In this exercise, we find Taylor polynomials of polynomials.

Suppose that $f(x) = 4+2x^2-7x^3$ and we want to find the third degree Taylor polynomial for $f(x)$ centered at $x=0$.  

Start by completing the table below:

\begin{tabular}{|c|c|c|c|}
\hline
$k$ \quad & \quad \quad $f^{(k)}(x)$  \quad \quad & \quad \quad $f^{(k)}(0)$ \quad \quad & \quad \quad $a_k = \frac{f^{(k)}(0)}{k!}$ \quad \quad \\
\hline 
$0$ \quad & \quad \quad $4+2x^2-7x^3$  \quad \quad & \quad \quad $\answer{4}$ \quad \quad  & \quad \quad $\answer{4}$ \quad \quad \\
\hline
$1$ \quad & \quad \quad $\answer{4x-21x^2}$ \quad \quad & \quad \quad $\answer{0}$ \quad \quad & \quad \quad  $\answer{0}$ \quad \quad  \\
\hline
$2$ \quad & \quad \quad $\answer{4-42x}$ \quad \quad & \quad \quad $\answer{4}$ \quad \quad & \quad \quad  $\answer{2}$ \quad \quad  \\
\hline
$3$ \quad & \quad \quad $\answer{-42}$ \quad \quad & \quad \quad $\answer{-42}$ \quad \quad & \quad \quad  $\answer{-7}$ \quad \quad  \\
\hline
\end{tabular}

\begin{exercise}
Hence, the third degree Taylor polynomial for $f(x) = 4+2x^2-7x^3$ centered at $x=0$ is:

\begin{align*}
p_3(x) &=a_0+a_1x+a_2x^2+a_3x^3 \\
&= \answer{4}+\answer{0}x+\answer{2}x^2+\answer{-7}x^3\\
\end{align*}

The result here should not be too surprising; if we are looking for a Taylor polynomial centered at $x=0$, the Taylor polynomial will be a polynomial in powers of $x$.  Since our original function was already a polynomial in powers of $x$, the Taylor polynomial is exactly the original function!
\end{exercise}

%%%%%%%%%%%%%%%%%%%%%%%%%%
\begin{exercise}
Now, let's spice things up a bit.  Suppose that $f(x) = 4+2x^2-7x^3$ and we want to find the third degree Taylor polynomial for $f(x)$ centered at $x=1$.  

Should we write the Taylor polynomial centered at $x=1$ as $p_3(x) = 4+2x^2-7x^3$? 
\begin{multipleChoice}
\choice{Yes; this is already a polynomial.}
\choice[correct]{No; we must write a polynomial in powers of $x-1$.}
\end{multipleChoice}

Complete the table below:

\begin{tabular}{|c|c|c|c|}
\hline
$k$ \quad & \quad \quad $f^{(k)}(x)$  \quad \quad & \quad \quad $f^{(k)}(1)$ \quad \quad & \quad \quad $a_k = \frac{f^{(k)}(1)}{k!}$ \quad \quad \\
\hline 
$0$ \quad & \quad \quad $4+2x^2-7x^3$  \quad \quad & \quad \quad $\answer{-1}$ \quad \quad  & \quad \quad $\answer{-1}$ \quad \quad \\
\hline
$1$ \quad & \quad \quad $\answer{4x-21x^2}$ \quad \quad & \quad \quad $\answer{-17}$ \quad \quad & \quad \quad  $\answer{-17}$ \quad \quad  \\
\hline
$2$ \quad & \quad \quad $\answer{4-42x}$ \quad \quad & \quad \quad $\answer{-38}$ \quad \quad & \quad \quad  $\answer{-19}$ \quad \quad  \\
\hline
$3$ \quad & \quad \quad $\answer{-42}$ \quad \quad & \quad \quad $\answer{-42}$ \quad \quad & \quad \quad  $\answer{-7}$ \quad \quad  \\
\hline
\end{tabular}

\begin{exercise}
Hence, the third degree Taylor polynomial for $f(x) = 4+2x^2-7x^3$ centered at $x=1$ is:

\begin{align*}
p_3(x) &=a_0+a_1(x-1)+a_2(x-1)^2+a_3(x-1)^3 \\
&= \answer{-1}+\answer{-17}(x-1)+\answer{-19}(x-1)^2+\answer{-7}(x-1)^3\\
\end{align*}

To explore an important fact, first note that this polynomial is a polynomial of degree:

\begin{multipleChoice}
\choice{1}
\choice{2}
\choice[correct]{3}
\choice{4}
\end{multipleChoice}

Let's explore whether this is a different polynomial than the one with which we started or is simply a rearrangement of the terms.  Expand the powers of $x-1$ below:

\begin{align*}
p_3(x) &=-1-17(x-1)-19(x-1)^2-7(x-1)^3\\
&= -1-17(x-1)-19(\answer{x^2-2x+1})-7(\answer{x^3-3x^2+3x-1}) \\
\end{align*}

Now, distribute the coefficients, and collect all of the like powers of $x$ to find:

\[
p_3(x) = \answer{4}+ \answer{0}x+ \answer{2}x^2+ \answer{-7}x^3
\]

Is this the same polynomial as $f(x) = 4+2x^2-7x^3$?

\begin{multipleChoice}
\choice[correct]{Yes}
\choice{No}
\end{multipleChoice}

The new polynomial is just the old one rearranged in powers of $x-1$.
\end{exercise}
\end{exercise}
%%%%


\end{exercise}
\end{document}
