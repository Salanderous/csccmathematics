\documentclass{ximera}


\graphicspath{
  {./}
  {ximeraTutorial/}
  {basicPhilosophy/}
}

\newcommand{\mooculus}{\textsf{\textbf{MOOC}\textnormal{\textsf{ULUS}}}}

\usepackage{tkz-euclide}\usepackage{tikz}
\usepackage{tikz-cd}
\usetikzlibrary{arrows}
\tikzset{>=stealth,commutative diagrams/.cd,
  arrow style=tikz,diagrams={>=stealth}} %% cool arrow head
\tikzset{shorten <>/.style={ shorten >=#1, shorten <=#1 } } %% allows shorter vectors

\usetikzlibrary{backgrounds} %% for boxes around graphs
\usetikzlibrary{shapes,positioning}  %% Clouds and stars
\usetikzlibrary{matrix} %% for matrix
\usepgfplotslibrary{polar} %% for polar plots
\usepgfplotslibrary{fillbetween} %% to shade area between curves in TikZ
\usetkzobj{all}
\usepackage[makeroom]{cancel} %% for strike outs
%\usepackage{mathtools} %% for pretty underbrace % Breaks Ximera
%\usepackage{multicol}
\usepackage{pgffor} %% required for integral for loops



%% http://tex.stackexchange.com/questions/66490/drawing-a-tikz-arc-specifying-the-center
%% Draws beach ball
\tikzset{pics/carc/.style args={#1:#2:#3}{code={\draw[pic actions] (#1:#3) arc(#1:#2:#3);}}}



\usepackage{array}
\setlength{\extrarowheight}{+.1cm}
\newdimen\digitwidth
\settowidth\digitwidth{9}
\def\divrule#1#2{
\noalign{\moveright#1\digitwidth
\vbox{\hrule width#2\digitwidth}}}






\DeclareMathOperator{\arccot}{arccot}
\DeclareMathOperator{\arcsec}{arcsec}
\DeclareMathOperator{\arccsc}{arccsc}

















%%This is to help with formatting on future title pages.
\newenvironment{sectionOutcomes}{}{}


\author{Jim Talamo}
\license{Creative Commons 3.0 By-bC}


\outcome{Understand the relationship between a Taylor Polynomials and the function it represents}


\begin{document}
\begin{exercise}
Suppose that $f(x)$ is an infinitely differentiable function at $x=17$ and that the \emph{fifth} degree Taylor polynomial of $f(x)$ centered at $x=-4$ is:

\[
p_5(x) = 1+3(x+4)-2(x+4)^2+4(x+4)^3+8(x+4)^5
\]

\begin{exercise}
Is there enough information to determine $f''(-4)$?

\begin{multipleChoice}
\choice[correct]{Yes}
\choice{No}
\end{multipleChoice}
In fact, $f''(-4) = \answer{-4}$.

\begin{hint}
Since the coefficients of the Taylor Polynomial and the values of the derivatives of the function are related via the formula:

\[
a_k = \frac{f^{(k)}(c)}{k!}
\]
We have that $c=\answer{-4}$, $a_3 = \answer{7}$ and that $f'''(0)$ can be found using this formula with $k=
\answer{2}$.   
\end{hint}

\end{exercise}

%%%%%%%%%%%%%%%%%%%%%%%%

\begin{exercise}
Is there enough information to determine $f^{(4)}(-4)$?

\begin{multipleChoice}
\choice[correct]{Yes}
\choice{No}
\end{multipleChoice}
In fact, $f^{(4)}(-4) = \answer{0}$.

\begin{hint}
Since the coefficients of the Taylor Polynomial and the values of the derivatives of the function are related via the formula:

\[
a_k = \frac{f^{(k)}(c)}{k!}
\]
We have that $c=\answer{-4}$, $a_4 = \answer{0}$ and that $f^{(4)}(0)$ can be found using this formula with $k=
\answer{4}$.   
\end{hint}

\end{exercise}

%%%%%%%%%%%%%%%%%%%%%%%%
\begin{exercise}
Is there enough information to determine $f^{(6)}(-4)$?

\begin{multipleChoice}
\choice{Yes}
\choice[correct]{No}
\end{multipleChoice}


\begin{hint}
Since the coefficients of the Taylor Polynomial and the values of the derivatives of the function are related via the formula:

\[
a_k = \frac{f^{(k)}(c)}{k!}
\]
We have that $c=\answer{-4}$ and would need information about $a_6$, which would be the coefficient of $(x+4)^6$.  We do not have this information because we only have a \emph{fifth} degree Taylor Polynomial!
\end{hint}

\end{exercise}


%%%%%%%%%%%%%%%%%%%%%%%%
Which of the following could be computed at $x=17$ by using the Taylor Polynomial?

\begin{selectAll}
\choice[correct]{$f(-4)$}
\choice[correct]{$f'(-4)$}
\choice[correct]{$f''(-4)$}
\choice[correct]{$f'''(-4)$}
\choice[correct]{$f^{(4)}(-4)$}
\choice[correct]{$f^{(5)}(-4)$}
\choice{$f^{(6)}(-4)$}
\end{selectAll}

%%%%%%%%%%%%%%%%%%%%%%%%

\begin{exercise}
Is there enough information to determine $f''(-4.1)$?

\begin{multipleChoice}
\choice{Yes}
\choice[correct]{No}
\end{multipleChoice}

\begin{exercise}
Since the coefficients of the Taylor Polynomial and the values of the derivatives of the function are related via the formula:

\[
a_k = \frac{f^{(k)}(c)}{k!}
\]
We have that $c=\answer{-4}$ so we only can find an exact value of $f''(-4)$ by using the Taylor Polynomial!  

(It's certainly true that we could \emph{approximate} $f''(8)$ by using this Taylor Polynomial, though without more information, it's impossible to determine how good of an approximation this would be)

\end{exercise}
\end{exercise}


\end{exercise}
\end{document}
