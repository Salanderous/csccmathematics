\documentclass{ximera}


\graphicspath{
  {./}
  {ximeraTutorial/}
  {basicPhilosophy/}
}

\newcommand{\mooculus}{\textsf{\textbf{MOOC}\textnormal{\textsf{ULUS}}}}

\usepackage{tkz-euclide}\usepackage{tikz}
\usepackage{tikz-cd}
\usetikzlibrary{arrows}
\tikzset{>=stealth,commutative diagrams/.cd,
  arrow style=tikz,diagrams={>=stealth}} %% cool arrow head
\tikzset{shorten <>/.style={ shorten >=#1, shorten <=#1 } } %% allows shorter vectors

\usetikzlibrary{backgrounds} %% for boxes around graphs
\usetikzlibrary{shapes,positioning}  %% Clouds and stars
\usetikzlibrary{matrix} %% for matrix
\usepgfplotslibrary{polar} %% for polar plots
\usepgfplotslibrary{fillbetween} %% to shade area between curves in TikZ
\usetkzobj{all}
\usepackage[makeroom]{cancel} %% for strike outs
%\usepackage{mathtools} %% for pretty underbrace % Breaks Ximera
%\usepackage{multicol}
\usepackage{pgffor} %% required for integral for loops



%% http://tex.stackexchange.com/questions/66490/drawing-a-tikz-arc-specifying-the-center
%% Draws beach ball
\tikzset{pics/carc/.style args={#1:#2:#3}{code={\draw[pic actions] (#1:#3) arc(#1:#2:#3);}}}



\usepackage{array}
\setlength{\extrarowheight}{+.1cm}
\newdimen\digitwidth
\settowidth\digitwidth{9}
\def\divrule#1#2{
\noalign{\moveright#1\digitwidth
\vbox{\hrule width#2\digitwidth}}}






\DeclareMathOperator{\arccot}{arccot}
\DeclareMathOperator{\arcsec}{arcsec}
\DeclareMathOperator{\arccsc}{arccsc}

















%%This is to help with formatting on future title pages.
\newenvironment{sectionOutcomes}{}{}


\author{Jim Talamo}
\license{Creative Commons 3.0 By-bC}


\outcome{Compute a Taylor Polynomial}


\begin{document}
\begin{exercise}
Find the third degree Taylor polynomial centered at $x=1$ for the function $f(x) = \sqrt{2x-1}$.


Since we want a Taylor Polynomial centered at $x=1$, we want to look for a polynomial in powers of $x-1$.  Since we want a third degree Taylor polynomial, this means we are looking for a polynomial of the form:

\begin{multipleChoice}
\choice{$p_3(x) = a_0+a_1x+a_2x^2+a_3x^3$}
\choice[correct]{$p_3(x) =  a_0+a_1(x-1)+a_2(x-1)^2+a_3(x-1)^3$}
\end{multipleChoice} 

The coefficients are found using the requirements:

\begin{selectAll}
\choice[correct]{$f(1) = p_3(1)$}
\choice[correct]{$f'(1) = p_3'(1)$}
\choice[correct]{$f''(1) = p_3''(1)$}
\choice[correct]{$f'''(1) = p_3'''(1)$}
\choice{$f^{(4)}(1) = p_3^{(4)}(1)$}
\choice{$f^{(5)}(1) = p_3^{(5)}(1)$}
\end{selectAll}

These requirements are used to establish a formula that relates the coefficients of the Taylor Polynomial to the derivatives of the function that it approximates for each $k=0,1,2,\ldots,n$:

\[
a_k = \frac{f^{(k)}(c)}{k!}
\]
where $c$ is the $x$-value at which the series is centered.  Here, $c=\answer{1}$.  

\begin{exercise}
Complete the table below:

\begin{tabular}{|c|c|c|c|}
\hline
$k$ \quad & \quad \quad $f^{(k)}(x)$  \quad \quad & \quad \quad $f^{(k)}(1)$ \quad \quad & \quad \quad $a_k = \frac{f^{(k)}(1)}{k!}$ \quad \quad \\
\hline 
$0$ \quad & \quad \quad $\answer{(2x-1)^{1/2}}$  \quad \quad & \quad \quad $\answer{1}$ \quad \quad  & \quad \quad $\answer{1}$ \quad \quad \\
\hline
$1$ \quad & \quad \quad $\answer{(2x-1)^{-1/2}}$ \quad \quad & \quad \quad $\answer{1}$ \quad \quad & \quad \quad  $\answer{1}$ \quad \quad  \\
\hline
$2$ \quad & \quad \quad $\answer{-(2x-1)^{-3/2}}$ \quad \quad & \quad \quad $\answer{-1}$ \quad \quad & \quad \quad  $\answer{-\frac{1}{2}}$ \quad \quad  \\
\hline
$3$ \quad & \quad \quad $\answer{3(2x-1)^{-5/2}}$ \quad \quad & \quad \quad $\answer{3}$ \quad \quad & \quad \quad  $\answer{\frac{1}{2}}$ \quad \quad  \\
\hline
\end{tabular}

Hence, the third degree Taylor polynomial for $f(x) =\sqrt{2x-1}$ is:

\begin{align*}
p_3(x) &=a_0+a_1(x-1)+a_2(x-1)^2+a_3(x-1)^3 \\
&= \answer{1}+\answer{1}(x-1)+\answer{-\frac{1}{2}}(x-1)^2+\answer{\frac{1}{2}}(x-1)^3\\
\end{align*}

\end{exercise}

\begin{exercise}
The Taylor polynomials are used to approximate the value of a function near the center.  As long as we are ``close enough" (which we will formalize in a later section), the higher order polynomials should give better approximations.  Suppose that we want to approximate $\sqrt{1.2}$.

The \emph{exact} value is found using the function, provided we use the correct $x$-value.  Since $f(x) = \sqrt{2x-1}$, the $x$-value that gives $\sqrt{1.2}$ is $x=\answer{1.1}$. 

\begin{hint}
To find this, just set $2x-1 = 1.2$.
\end{hint}

Thus, the \emph{exact} answer to 6 decimal places is $\sqrt{1.2} = \answer[tolerance= .000001]{1.095445}$

We can use the third degree Taylor polynomial to write down the first and second degree Taylor polynomials.  We can then substitute $x=1.1$ into these to approximate $\sqrt{1.2}$.  In fact:

\begin{exercise}
The first degree Taylor polynomial is $p_1(x) = \answer{1}+\answer{1}(x-1)$ so to 6 decimal places: $\sqrt{1.2} \approx p_1(1.1) = \answer[tolerance=.000001]{1.100000}$.

The second degree Taylor polynomial is $p_2(x) = \answer{1}+\answer{1}(x-1)+\answer{-\frac{1}{2}}(x-1)^2$ so to 6 decimal places: $\sqrt{1.2} \approx p_2(1.1) = \answer[tolerance=.000001]{1.095000}$.

The third degree Taylor polynomial is $p_3(x) =1+(x-1)-\frac{1}{2}(x-1)^2+\frac{1}{2}(x-1)^3$ so to 6 decimal places: $\sqrt{1.2} \approx p_3(1.1) = \answer[tolerance=.000001]{1.095500}$.

Do the higher order polynomials provide more accurate results?

\begin{multipleChoice}
\choice[correct]{Yes}
\choice{No}
\end{multipleChoice}
\end{exercise}
\end{exercise}

\end{exercise}
\end{document}
