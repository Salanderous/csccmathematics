\documentclass{ximera}


\graphicspath{
  {./}
  {ximeraTutorial/}
  {basicPhilosophy/}
}

\newcommand{\mooculus}{\textsf{\textbf{MOOC}\textnormal{\textsf{ULUS}}}}

\usepackage{tkz-euclide}\usepackage{tikz}
\usepackage{tikz-cd}
\usetikzlibrary{arrows}
\tikzset{>=stealth,commutative diagrams/.cd,
  arrow style=tikz,diagrams={>=stealth}} %% cool arrow head
\tikzset{shorten <>/.style={ shorten >=#1, shorten <=#1 } } %% allows shorter vectors

\usetikzlibrary{backgrounds} %% for boxes around graphs
\usetikzlibrary{shapes,positioning}  %% Clouds and stars
\usetikzlibrary{matrix} %% for matrix
\usepgfplotslibrary{polar} %% for polar plots
\usepgfplotslibrary{fillbetween} %% to shade area between curves in TikZ
\usetkzobj{all}
\usepackage[makeroom]{cancel} %% for strike outs
%\usepackage{mathtools} %% for pretty underbrace % Breaks Ximera
%\usepackage{multicol}
\usepackage{pgffor} %% required for integral for loops



%% http://tex.stackexchange.com/questions/66490/drawing-a-tikz-arc-specifying-the-center
%% Draws beach ball
\tikzset{pics/carc/.style args={#1:#2:#3}{code={\draw[pic actions] (#1:#3) arc(#1:#2:#3);}}}



\usepackage{array}
\setlength{\extrarowheight}{+.1cm}
\newdimen\digitwidth
\settowidth\digitwidth{9}
\def\divrule#1#2{
\noalign{\moveright#1\digitwidth
\vbox{\hrule width#2\digitwidth}}}






\DeclareMathOperator{\arccot}{arccot}
\DeclareMathOperator{\arcsec}{arcsec}
\DeclareMathOperator{\arccsc}{arccsc}

















%%This is to help with formatting on future title pages.
\newenvironment{sectionOutcomes}{}{}


\author{Jim Talamo}
\license{Creative Commons 3.0 By-bC}


\outcome{Understand the relationship between a Taylor Polynomial and a polynomial}


\begin{document}
\begin{exercise}
In this exercise, we find Taylor polynomials of polynomials.

Suppose that $f(x) = 2-4(x-1)+6(x-1)^2$ and we want to find the second degree Taylor polynomial for $f(x)$ centered at $x=0$.  Before working the exercise, think about what the answer should be.

\begin{exercise}
One way to determine what the Taylor polynomial should be is by using the definition.  Start by completing the table below:

\begin{tabular}{|c|c|c|c|}
\hline
$k$ \quad & \quad \quad $f^{(k)}(x)$  \quad \quad & \quad \quad $f^{(k)}(0)$ \quad \quad & \quad \quad $a_k = \frac{f^{(k)}(0)}{k!}$ \quad \quad \\
\hline 
$0$ \quad & \quad \quad $2-4(x-1)+6(x-1)^2$  \quad \quad & \quad \quad $\answer{12}$ \quad \quad  & \quad \quad $\answer{12}$ \quad \quad \\
\hline
$1$ \quad & \quad \quad $\answer{-4+12(x-1)}$ \quad \quad & \quad \quad $\answer{-16}$ \quad \quad & \quad \quad  $\answer{-16}$ \quad \quad  \\
\hline
$2$ \quad & \quad \quad $\answer{12}$ \quad \quad & \quad \quad $\answer{12}$ \quad \quad & \quad \quad  $\answer{6}$ \quad \quad  \\
\hline
\end{tabular}

\begin{exercise}
Hence, the second degree Taylor polynomial for  $f(x) = 2-4(x-1)+6(x-1)^2$ centered at $x=0$ is:

\begin{align*}
p_2(x) &=a_0+a_1x+a_2x^2+a_3x^3 \\
&= \answer{12}+\answer{-16}x+\answer{6}x^2\\
\end{align*}

\end{exercise}
\end{exercise}


%%%%%%%%%%%%%%%%%%%%%%%%%%
\begin{exercise}

Recall that conceptually, we think of a Taylor polynomial at $x=0$ as a polynomial in powers of $x-0$ (or $x$) that approximates the original function.  There is an easy way to obtain such a polynomial here without performing any calculus.

First note that $f(x) = 2-4(x-1)+6(x-1)^2$ is a polynomial of degree:

\begin{multipleChoice}
\choice{1}
\choice[correct]{2}
\choice{3}
\choice{4}
\end{multipleChoice}

Expand the powers of $x-1$ below:

\begin{align*}
p_2(x) &=2-4(x-1)+6(x-1)^2\\
&= 2-4(x-1)+6(\answer{x^2-2x+1})\\
\end{align*}

Now, distribute the coefficients, and collect all of the like powers of $x$ to find:

\[
p_2(x) = \answer{12}+ \answer{-16}x+ \answer{6}x^2
\]

Does this agree with the previous result?

\begin{multipleChoice}
\choice[correct]{Yes}
\choice{No}
\end{multipleChoice}



\begin{exercise}
Suppose that $f(x) = x^2(2x-1)(x^2+1)$.  Give the degree 4 Taylor polynomial for $f(x)$ centered at $x=0$.

\[
p_4(x) = \answer{-x^2+2x^3-x^4}
\]

\begin{hint}
This is a polynomial of degree $\answer{5}$.  Expanding it we find:

\[
f(x) = \answer{0}+\answer{0}x+\answer{-1}x^2+\answer{2}x^3+\answer{-1}x^4+\answer{2}x^5
\]

Now, note that we are asked to find the degree four Taylor polynomial here!
\end{hint}

\end{exercise}

%%%%

\end{exercise}
\end{exercise}
\end{document}
