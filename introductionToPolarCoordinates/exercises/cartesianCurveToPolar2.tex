\documentclass{ximera}


\graphicspath{
  {./}
  {ximeraTutorial/}
  {basicPhilosophy/}
}

\newcommand{\mooculus}{\textsf{\textbf{MOOC}\textnormal{\textsf{ULUS}}}}

\usepackage{tkz-euclide}\usepackage{tikz}
\usepackage{tikz-cd}
\usetikzlibrary{arrows}
\tikzset{>=stealth,commutative diagrams/.cd,
  arrow style=tikz,diagrams={>=stealth}} %% cool arrow head
\tikzset{shorten <>/.style={ shorten >=#1, shorten <=#1 } } %% allows shorter vectors

\usetikzlibrary{backgrounds} %% for boxes around graphs
\usetikzlibrary{shapes,positioning}  %% Clouds and stars
\usetikzlibrary{matrix} %% for matrix
\usepgfplotslibrary{polar} %% for polar plots
\usepgfplotslibrary{fillbetween} %% to shade area between curves in TikZ
\usetkzobj{all}
\usepackage[makeroom]{cancel} %% for strike outs
%\usepackage{mathtools} %% for pretty underbrace % Breaks Ximera
%\usepackage{multicol}
\usepackage{pgffor} %% required for integral for loops



%% http://tex.stackexchange.com/questions/66490/drawing-a-tikz-arc-specifying-the-center
%% Draws beach ball
\tikzset{pics/carc/.style args={#1:#2:#3}{code={\draw[pic actions] (#1:#3) arc(#1:#2:#3);}}}



\usepackage{array}
\setlength{\extrarowheight}{+.1cm}
\newdimen\digitwidth
\settowidth\digitwidth{9}
\def\divrule#1#2{
\noalign{\moveright#1\digitwidth
\vbox{\hrule width#2\digitwidth}}}






\DeclareMathOperator{\arccot}{arccot}
\DeclareMathOperator{\arcsec}{arcsec}
\DeclareMathOperator{\arccsc}{arccsc}

















%%This is to help with formatting on future title pages.
\newenvironment{sectionOutcomes}{}{}


%\outcome{Find tangent lines to parametric curves}
\author{Jim Talamo }

\begin{document}
\begin{exercise}

Express the parabola $y= x^2$ using polar coordinates:

\[
r= \answer{\sec(\theta)\tan(\theta)}
\]

\begin{hint}
Start by using:

\begin{multipleChoice}
\choice[correct]{$x = r\cos(\theta), y= r \sin(\theta)$}
\choice{$x = r\sin(\theta), y= r \cos(\theta)$}
\end{multipleChoice}

We thus find:

\begin{align*}
y &= x^2 \\
\left( \answer{r \sin(\theta)} \right) &= \left( \answer{r \cos(\theta)} \right)^2
\end{align*}

Now, solve for $r$ by moving all terms with an $r$ to the lefthand side:

\[
\answer{r \sin(\theta) - r^2 \cos^2(\theta)} =0
\]

Factor out an $r$:

\[
r \answer{ \sin(\theta) -r  \cos^2(\theta)} = 0
\]
So, either $r=0$ or $ \sin(\theta) -r  \cos^2(\theta) = 0$.  We can solve for $r$ for the latter equation:

\begin{align*}
\sin(\theta) -r  \cos^2(\theta) &= 0 \\
r &= \answer{\sec(\theta)\tan(\theta)}
\end{align*}
Note that there is a $\theta$ value for which $r=0$; in fact, when $\theta = \answer{0}$, $r=0$, so this does include the case when $r=0$.

\end{hint}

\end{exercise}
\end{document}