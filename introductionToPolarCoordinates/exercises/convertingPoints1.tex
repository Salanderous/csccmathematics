\documentclass{ximera}


\graphicspath{
  {./}
  {ximeraTutorial/}
  {basicPhilosophy/}
}

\newcommand{\mooculus}{\textsf{\textbf{MOOC}\textnormal{\textsf{ULUS}}}}

\usepackage{tkz-euclide}\usepackage{tikz}
\usepackage{tikz-cd}
\usetikzlibrary{arrows}
\tikzset{>=stealth,commutative diagrams/.cd,
  arrow style=tikz,diagrams={>=stealth}} %% cool arrow head
\tikzset{shorten <>/.style={ shorten >=#1, shorten <=#1 } } %% allows shorter vectors

\usetikzlibrary{backgrounds} %% for boxes around graphs
\usetikzlibrary{shapes,positioning}  %% Clouds and stars
\usetikzlibrary{matrix} %% for matrix
\usepgfplotslibrary{polar} %% for polar plots
\usepgfplotslibrary{fillbetween} %% to shade area between curves in TikZ
\usetkzobj{all}
\usepackage[makeroom]{cancel} %% for strike outs
%\usepackage{mathtools} %% for pretty underbrace % Breaks Ximera
%\usepackage{multicol}
\usepackage{pgffor} %% required for integral for loops



%% http://tex.stackexchange.com/questions/66490/drawing-a-tikz-arc-specifying-the-center
%% Draws beach ball
\tikzset{pics/carc/.style args={#1:#2:#3}{code={\draw[pic actions] (#1:#3) arc(#1:#2:#3);}}}



\usepackage{array}
\setlength{\extrarowheight}{+.1cm}
\newdimen\digitwidth
\settowidth\digitwidth{9}
\def\divrule#1#2{
\noalign{\moveright#1\digitwidth
\vbox{\hrule width#2\digitwidth}}}






\DeclareMathOperator{\arccot}{arccot}
\DeclareMathOperator{\arcsec}{arcsec}
\DeclareMathOperator{\arccsc}{arccsc}

















%%This is to help with formatting on future title pages.
\newenvironment{sectionOutcomes}{}{}


%\outcome{Find tangent lines to parametric curves}
\author{Jim Talamo and Jason Miller}

\begin{document}
\begin{exercise}

Express the following polar coordinates in Cartesian coordinates: 

If $\left(r,\theta\right) = \left( 6, \frac{3\pi}{2}\right)$, then in Cartesian coordinates $(x,y) = \left( \answer{0}, \answer{-6 }\right)$.

If $\left(r,\theta\right) = \left(3, \frac{7 \pi}{4}\right)$, then in Cartesian coordinates $(x,y) = \left( \answer{\frac{3\sqrt{2}}{2}}, \answer{\frac{-3\sqrt{2}}{2}} \right)$

\begin{hint}
Use the relationships $x=r\cos(\theta)$ and $y=r\sin(\theta)$.
\end{hint}

\end{exercise}

\begin{exercise}

Express the following Cartesian coordinates in polar coordinates.  In your answer, use $r>0$ and $0 \leq \theta < 2\pi$. 


If $(x,y) =\left(2, -2 \right)$, then in polar coordinates $(r,\theta) =\left( \answer{ 2\sqrt{2}}, \answer{ \frac{7\pi}{4 } } \right)$. 

If $(x,y) =\left(-4, -4\sqrt{3} \right)$, then in polar coordinates $(r,\theta) =\left( \answer{ 8 } , \answer{  \frac{4\pi}{3}  } \right)$. 

\begin{hint}
Use the relationships $r^2=x^2+y^2$ and $\tan\left(\theta\right) = \frac{y}{x}$.  Note that for the angle, there are generally two options; you can determine which one to use by examining the quadrant in which the point $(x,y)$ lies.

\end{hint}
\end{exercise}
\end{document}
