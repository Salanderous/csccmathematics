\documentclass{ximera}


\graphicspath{
  {./}
  {ximeraTutorial/}
  {basicPhilosophy/}
}

\newcommand{\mooculus}{\textsf{\textbf{MOOC}\textnormal{\textsf{ULUS}}}}

\usepackage{tkz-euclide}\usepackage{tikz}
\usepackage{tikz-cd}
\usetikzlibrary{arrows}
\tikzset{>=stealth,commutative diagrams/.cd,
  arrow style=tikz,diagrams={>=stealth}} %% cool arrow head
\tikzset{shorten <>/.style={ shorten >=#1, shorten <=#1 } } %% allows shorter vectors

\usetikzlibrary{backgrounds} %% for boxes around graphs
\usetikzlibrary{shapes,positioning}  %% Clouds and stars
\usetikzlibrary{matrix} %% for matrix
\usepgfplotslibrary{polar} %% for polar plots
\usepgfplotslibrary{fillbetween} %% to shade area between curves in TikZ
\usetkzobj{all}
\usepackage[makeroom]{cancel} %% for strike outs
%\usepackage{mathtools} %% for pretty underbrace % Breaks Ximera
%\usepackage{multicol}
\usepackage{pgffor} %% required for integral for loops



%% http://tex.stackexchange.com/questions/66490/drawing-a-tikz-arc-specifying-the-center
%% Draws beach ball
\tikzset{pics/carc/.style args={#1:#2:#3}{code={\draw[pic actions] (#1:#3) arc(#1:#2:#3);}}}



\usepackage{array}
\setlength{\extrarowheight}{+.1cm}
\newdimen\digitwidth
\settowidth\digitwidth{9}
\def\divrule#1#2{
\noalign{\moveright#1\digitwidth
\vbox{\hrule width#2\digitwidth}}}






\DeclareMathOperator{\arccot}{arccot}
\DeclareMathOperator{\arcsec}{arcsec}
\DeclareMathOperator{\arccsc}{arccsc}

















%%This is to help with formatting on future title pages.
\newenvironment{sectionOutcomes}{}{}


%\outcome{Find tangent lines to parametric curves}
\author{Jim Talamo }

\begin{document}
\begin{exercise}

Select all of the following that provide an alternate description for the polar coordinates $(r, \theta) = \left(3,\frac{\pi}{3}\right)$:

\begin{selectAll}
\choice[correct]{$(r, \theta) = \left(3,-\frac{5\pi}{3} \right)$}
\choice{$(r, \theta) = \left(-3,\frac{5\pi}{3} \right)$}
\choice[correct]{$(r, \theta) = \left(-3,-\frac{2\pi}{3} \right)$}
\choice{$(r, \theta) = \left(-3,-\frac{5\pi}{3} \right)$}
\choice[correct]{$(r, \theta) = \left(3,\frac{7\pi}{3} \right)$}
\choice[correct]{$(r, \theta) = \left(-3,-\frac{2\pi}{3} \right)$}
\end{selectAll}

\begin{hint}
One way to do this is to convert all of the points to Cartesian coordinates.  A better way is to remember that to graph a point in polar coordinates:

\begin{itemize}
\item If $r>0$, start along the positive $x$-axis.
\item If $r<0$, start along the negative $x$-axis.
\item If $\theta>0$, rotate counterclockwise.
\item If $\theta<0$, rotate clockwise.
\end{itemize}
\end{hint}
\end{exercise}
\end{document}