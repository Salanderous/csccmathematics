\documentclass{ximera}


\graphicspath{
  {./}
  {ximeraTutorial/}
  {basicPhilosophy/}
}

\newcommand{\mooculus}{\textsf{\textbf{MOOC}\textnormal{\textsf{ULUS}}}}

\usepackage{tkz-euclide}\usepackage{tikz}
\usepackage{tikz-cd}
\usetikzlibrary{arrows}
\tikzset{>=stealth,commutative diagrams/.cd,
  arrow style=tikz,diagrams={>=stealth}} %% cool arrow head
\tikzset{shorten <>/.style={ shorten >=#1, shorten <=#1 } } %% allows shorter vectors

\usetikzlibrary{backgrounds} %% for boxes around graphs
\usetikzlibrary{shapes,positioning}  %% Clouds and stars
\usetikzlibrary{matrix} %% for matrix
\usepgfplotslibrary{polar} %% for polar plots
\usepgfplotslibrary{fillbetween} %% to shade area between curves in TikZ
\usetkzobj{all}
\usepackage[makeroom]{cancel} %% for strike outs
%\usepackage{mathtools} %% for pretty underbrace % Breaks Ximera
%\usepackage{multicol}
\usepackage{pgffor} %% required for integral for loops



%% http://tex.stackexchange.com/questions/66490/drawing-a-tikz-arc-specifying-the-center
%% Draws beach ball
\tikzset{pics/carc/.style args={#1:#2:#3}{code={\draw[pic actions] (#1:#3) arc(#1:#2:#3);}}}



\usepackage{array}
\setlength{\extrarowheight}{+.1cm}
\newdimen\digitwidth
\settowidth\digitwidth{9}
\def\divrule#1#2{
\noalign{\moveright#1\digitwidth
\vbox{\hrule width#2\digitwidth}}}






\DeclareMathOperator{\arccot}{arccot}
\DeclareMathOperator{\arcsec}{arcsec}
\DeclareMathOperator{\arccsc}{arccsc}

















%%This is to help with formatting on future title pages.
\newenvironment{sectionOutcomes}{}{}


\author{Jim Talamo}
\license{Creative Commons 3.0 By-NC}


\outcome{Think about Partial Fraction Decomposition}


\begin{document}
\begin{exercise}
Consider the indefinite integral: 

\[
\int \frac{2x-4}{x^3-4x} dx.
\]

If we want to integrate this, partial fraction decomposition would be a good technique to try.  However, we could first note that we could factor and simplify.  Indeed:

\[
\frac{2x-4}{x^3-4x} = \frac{2(x-2)}{x(x+2)(x-2)} = \answer{\frac{2}{x(x+2)}}
\]  

The partial fraction decomposition for this expression is:

\[
\frac{2}{x(x+2)} = \answer{\frac{A}{x}+\frac{B}{x+2}}
\]
(Use $A$ and $B$ for the constants and list the terms in the order written in the factored form)

\begin{exercise}
Solving for these constants, we find: $A = \answer{1}$ and $B=\answer{-1}$ and we conclude:

\[
\frac{2}{x(x+2)} = \answer{\frac{1}{x}-\frac{1}{x+2}}
\]


\begin{exercise}
Suppose we did not notice that there was simplifying that could be done first.  In this case, we would write:

\[
\frac{2x-4}{x^3-4x} = \frac{2x-4}{x(x+2)(x-2)} 
\]  
Without simplifying, what is the correct general partial fraction decomposition of the above expression?

\[
\frac{2x-4}{x^3-4x} = \answer{\frac{A}{x}+\frac{B}{x+2} +\frac{C}{x-2}}
\]  
(Use $A$, $B$, and $C$ for the constants and list the terms in the order written in the factored form)

\begin{exercise}

Multiplying both sides by $x^3-4x$ gives:

\[
2x-4 = \answer{(x+2)(x-2)}A+\answer{x(x-2)}B +\answer{x(x+2)}C
\]  

\begin{exercise}
Setting $x=0$ gives $A=\answer{1}$.

Setting $x=2$ gives $C=\answer{0}$.

Setting $x=-2$ gives $B=\answer{-1}$.

So, the partial fraction decomposition for this expression is:

\[
\frac{2x-4}{x(x+2)(x-2)}  = \answer{\frac{1}{x}+\frac{-1}{x+2}}
\]


\begin{multipleChoice}
\choice[correct]{The results are the same}
\choice{The results are not the same; we broke math!}
\end{multipleChoice}

\begin{exercise}
Using the above results, we find:

 \[
\int \frac{2x-4}{x^3-4x} dx = \answer{\ln|x|-\ln|x+2|+C}
\]
(Use $C$ for the constant of integration)


\end{exercise}
\end{exercise}
\end{exercise}
\end{exercise}
\end{exercise}
\end{exercise}
\end{document}
