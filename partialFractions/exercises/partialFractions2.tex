\documentclass{ximera}


\graphicspath{
  {./}
  {ximeraTutorial/}
  {basicPhilosophy/}
}

\newcommand{\mooculus}{\textsf{\textbf{MOOC}\textnormal{\textsf{ULUS}}}}

\usepackage{tkz-euclide}\usepackage{tikz}
\usepackage{tikz-cd}
\usetikzlibrary{arrows}
\tikzset{>=stealth,commutative diagrams/.cd,
  arrow style=tikz,diagrams={>=stealth}} %% cool arrow head
\tikzset{shorten <>/.style={ shorten >=#1, shorten <=#1 } } %% allows shorter vectors

\usetikzlibrary{backgrounds} %% for boxes around graphs
\usetikzlibrary{shapes,positioning}  %% Clouds and stars
\usetikzlibrary{matrix} %% for matrix
\usepgfplotslibrary{polar} %% for polar plots
\usepgfplotslibrary{fillbetween} %% to shade area between curves in TikZ
\usetkzobj{all}
\usepackage[makeroom]{cancel} %% for strike outs
%\usepackage{mathtools} %% for pretty underbrace % Breaks Ximera
%\usepackage{multicol}
\usepackage{pgffor} %% required for integral for loops



%% http://tex.stackexchange.com/questions/66490/drawing-a-tikz-arc-specifying-the-center
%% Draws beach ball
\tikzset{pics/carc/.style args={#1:#2:#3}{code={\draw[pic actions] (#1:#3) arc(#1:#2:#3);}}}



\usepackage{array}
\setlength{\extrarowheight}{+.1cm}
\newdimen\digitwidth
\settowidth\digitwidth{9}
\def\divrule#1#2{
\noalign{\moveright#1\digitwidth
\vbox{\hrule width#2\digitwidth}}}






\DeclareMathOperator{\arccot}{arccot}
\DeclareMathOperator{\arcsec}{arcsec}
\DeclareMathOperator{\arccsc}{arccsc}

















%%This is to help with formatting on future title pages.
\newenvironment{sectionOutcomes}{}{}


\author{Jason Miller}
\license{Creative Commons 3.0 By-NC}


\outcome{}


\begin{document}
\begin{exercise}
Use the method of partial fractions to determine the integral.
\[
\int \frac{x^{2}+7x+1}{(x^{2}-2x+1)(x+2)} dx
\]

Note that the degree of the numerator is smaller than the degree of the denominator so we do not need 
to use long division. 

First we see if we can factor the denominator. 

In this case we can factor and we get:

\[
(x^{2}-2x+1)(x+2)=\answer{(x-1)^{2}}(x+2)
\]

\begin{exercise} 

We notice that the bottom has a repeated linear factor. That means, for some constants $A$, $B$ and $C$, we have:

\[
\frac{x^{2}+7x+1}{(x-1)^{2}(x+2)}= \frac{A}{x+2} + \frac{B}{x-1} +\frac{C}{\answer{(x-1)^{2}}}
\]

We need to determine the three constants. 

We clear denominators by multiplying both sides of the above equation by $\answer{(x-1)^{2}(x+2)}$. 

This gives us 

\[
x^{2}+7x+1=A\answer{ (x-1)^{2}} + B\answer{ (x-1)(x+2)} +  C\answer{ (x+2)}
\]

We can determine the unknown coefficients $A$ and $B$ in two ways. 
We could expand the right hand side and then identify coefficients of like terms on the left and right.
Instead, for this case, we try certain convenient values of $x$ to simplify the above equation and then solve for the coefficients. 

Let us try plugging in $x=1$ into the above equation. 

Then we get 
\[
\answer{9 }= C\answer{3}
\]

This tells us that $C=\answer{ 3}$. 


In a similar fashion we can plug in $x=-2$. 

Then we have 

\[
\answer{-9}=A\answer{9}
\]

Solving, we obtain $A=\answer{-1}$. 

We still need to determine $B$ but there is no $x$ value which will make the $B$ term above nonzero but make the other two terms zero. 
However we already know $A$ and $C$ so we can just use any convenient $x$ value.

We try $x=0$. 

Then we obtain $B=2$. 

\begin{exercise}
This means our original integral can be rewritten as 

\[
\int \frac{x^{2}+7x+1}{(x-1)^{2}(x+2)} \d x= \int \frac{-1}{x+2} dx + \int \frac{2}{x-1} dx + \int \frac{ 3  }{(x-1)^{2}}   dx
\]

 Finally computing the antiderivative we get 

\[
\int \frac{x^{2}+7x+1}{(x-1)^{2}(x+2)} dx= \answer{ -1\ln|x+2|+2\ln|x-1| -\frac{3}{x-1}+ C }
\]
(Use $C$ for the constant of integration)




\end{exercise}
\end{exercise}
\end{exercise}
\end{document}
