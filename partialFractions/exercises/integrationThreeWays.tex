\documentclass{ximera}


\graphicspath{
  {./}
  {ximeraTutorial/}
  {basicPhilosophy/}
}

\newcommand{\mooculus}{\textsf{\textbf{MOOC}\textnormal{\textsf{ULUS}}}}

\usepackage{tkz-euclide}\usepackage{tikz}
\usepackage{tikz-cd}
\usetikzlibrary{arrows}
\tikzset{>=stealth,commutative diagrams/.cd,
  arrow style=tikz,diagrams={>=stealth}} %% cool arrow head
\tikzset{shorten <>/.style={ shorten >=#1, shorten <=#1 } } %% allows shorter vectors

\usetikzlibrary{backgrounds} %% for boxes around graphs
\usetikzlibrary{shapes,positioning}  %% Clouds and stars
\usetikzlibrary{matrix} %% for matrix
\usepgfplotslibrary{polar} %% for polar plots
\usepgfplotslibrary{fillbetween} %% to shade area between curves in TikZ
\usetkzobj{all}
\usepackage[makeroom]{cancel} %% for strike outs
%\usepackage{mathtools} %% for pretty underbrace % Breaks Ximera
%\usepackage{multicol}
\usepackage{pgffor} %% required for integral for loops



%% http://tex.stackexchange.com/questions/66490/drawing-a-tikz-arc-specifying-the-center
%% Draws beach ball
\tikzset{pics/carc/.style args={#1:#2:#3}{code={\draw[pic actions] (#1:#3) arc(#1:#2:#3);}}}



\usepackage{array}
\setlength{\extrarowheight}{+.1cm}
\newdimen\digitwidth
\settowidth\digitwidth{9}
\def\divrule#1#2{
\noalign{\moveright#1\digitwidth
\vbox{\hrule width#2\digitwidth}}}






\DeclareMathOperator{\arccot}{arccot}
\DeclareMathOperator{\arcsec}{arcsec}
\DeclareMathOperator{\arccsc}{arccsc}

















%%This is to help with formatting on future title pages.
\newenvironment{sectionOutcomes}{}{}


\author{Jim Talamo}
\license{Creative Commons 3.0 By-NC}


\outcome{Compute antiderivatives using 3 different integration techniques}


\begin{document}
\begin{exercise}
Consider the indefinite integral 

\[
\int \frac{2x}{x^2-4} dx.
\]

This integral could be evaluated with a substitution, a trigonometric substitution, or by using partial fraction decomposition.

%%%%%%%%u-sub%%%%%%%%%%%%
\begin{exercise}
If we want to evaluate the indefinite integral by using a substitution, let $u= \answer{x^2-4}$ so $du = \answer{2x} dx$.

Making this substitution, we have:

\[
\int \frac{2x}{x^2-4} dx = \int \answer{\frac{1}{u}} du
\]
\begin{exercise}
Evaluating gives:
\[
\int\frac{1}{u} du = \answer{\ln|u|+C}
\]
(type the antiderivative in terms of $u$ and use $C$ for the constant of integration)

and reversing the substitution gives:

\[
\int \frac{2x}{x^2-4} dx = \answer{\ln|x^2-4|+C}
\]
(type the antiderivative in terms of $x$ and use $C$ for the constant of integration)
\end{exercise}
\end{exercise}
%%%%%%%%%%%   Trig Sub  %%%%%%%%%%%%
\begin{exercise}
If we want to use a trigonometric substitution, we should set $u=\answer{2} \sec(\theta)$.  Then, $du = \answer{2 \sec(\theta)\tan(\theta)} d\theta$, and after substituting into the original integral and simplifying:

\[
\int \frac{2x}{x^2-4} dx = \int \answer{\frac{2 \sec^2(\theta)}{\tan(\theta)}} d\theta
\]
(type your answer in terms of $\sec(\theta)$ and $\tan(\theta)$.

\begin{exercise}
We can evaluate this integral and find:

\[
 \int \frac{2 \sec^2(\theta)}{\tan(\theta)} d\theta = \answer{2 \ln|\tan(\theta)|+C}
\]
(Use $C$ for the constant of integration and make sure that your antiderivative is in terms of $\theta$)

\begin{exercise}

To reverse the substitution, we must write $\tan(\theta)$ in terms of $x$.  In fact, we find:

\[
\tan(\theta) = \answer{\frac{\sqrt{x^2-4}}{2}}
\]

\begin{exercise}
We can now express the antiderivatives in terms of $x$ and find:

\[
\int \frac{2x}{x^2-4} dx =  2 \ln|\tan(\theta)|+C = \answer{2 \ln \left| \frac{\sqrt{x^2-4}}{2} \right| +C}
\]

\begin{exercise}
Using the properties of logarithms, we can show

\begin{align*}
2 \ln \left| \frac{ \sqrt{x^2-4} }{2} \right| & = 2 \ln \left|(x^2-4)^{1/2} \right| - 2 \ln(2) \\
& = 2 \cdot \frac{1}{2} \ln \left| x^2-4 \right| - 2 \ln(2) \textrm{ since } \ln\left(a^b\right) = b \ln(a) \\
& =  \ln \left|x^2-4 \right| - 2 \ln(2)
\end{align*}
which differs from the antiderivative found using the $u$-substitution method by a constant!

\end{exercise}
\end{exercise}
\end{exercise}
\end{exercise}
\end{exercise}

%%%%%%%%%% Partial Fractions  %%%%%%%
\begin{exercise}
If we want to use partial fraction decomposition to evaluate the integral, we find:

\[
\frac{2x}{x^2-4} = \frac{2x}{(x-2)(x+2)} = \frac{\answer{1}}{x-2}+\frac{\answer{1}}{x+2}
\]
\begin{hint}
The general partial fraction decomposition is:
\[
\frac{2x}{x^2-4} = \frac{2x}{(x-2)(x+2)} = \frac{A}{x-2}+\frac{B}{x+2}
\]
Solve for the constants and type your answers above!
\end{hint}

\begin{exercise}
We can now evaluate the integral to obtain:

\[
\int \frac{2x}{x^2-4} dx = \int \frac{1}{x-2}+\frac{1}{x+2} dx = \answer{\ln|x-2| + \ln|x+2|+C}
\]
 
(Use $C$ for the constant of integration)

\begin{exercise}
Use the properties of logarithms, we find:

\[ \ln|x-2| + \ln|x+2| = \ln \left( |x-2| \cdot |x+2|  \right) = \ln \left| x^2-4  \right| \]
\end{exercise}
\end{exercise}

\begin{exercise}
Are the results of the three procedures equivalent?
\begin{multipleChoice}
\choice[correct]{Yes}
\choice{No}
\end{multipleChoice}
There is often more than one technique that can be used to find antiderivatives, but this exercise illustrates that some may be easier than others!
\end{exercise}
\end{exercise}
\end{exercise}
\end{document}
