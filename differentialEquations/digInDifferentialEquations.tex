\documentclass{ximera}


\graphicspath{
  {./}
  {ximeraTutorial/}
  {basicPhilosophy/}
}

\newcommand{\mooculus}{\textsf{\textbf{MOOC}\textnormal{\textsf{ULUS}}}}

\usepackage{tkz-euclide}\usepackage{tikz}
\usepackage{tikz-cd}
\usetikzlibrary{arrows}
\tikzset{>=stealth,commutative diagrams/.cd,
  arrow style=tikz,diagrams={>=stealth}} %% cool arrow head
\tikzset{shorten <>/.style={ shorten >=#1, shorten <=#1 } } %% allows shorter vectors

\usetikzlibrary{backgrounds} %% for boxes around graphs
\usetikzlibrary{shapes,positioning}  %% Clouds and stars
\usetikzlibrary{matrix} %% for matrix
\usepgfplotslibrary{polar} %% for polar plots
\usepgfplotslibrary{fillbetween} %% to shade area between curves in TikZ
\usetkzobj{all}
\usepackage[makeroom]{cancel} %% for strike outs
%\usepackage{mathtools} %% for pretty underbrace % Breaks Ximera
%\usepackage{multicol}
\usepackage{pgffor} %% required for integral for loops



%% http://tex.stackexchange.com/questions/66490/drawing-a-tikz-arc-specifying-the-center
%% Draws beach ball
\tikzset{pics/carc/.style args={#1:#2:#3}{code={\draw[pic actions] (#1:#3) arc(#1:#2:#3);}}}



\usepackage{array}
\setlength{\extrarowheight}{+.1cm}
\newdimen\digitwidth
\settowidth\digitwidth{9}
\def\divrule#1#2{
\noalign{\moveright#1\digitwidth
\vbox{\hrule width#2\digitwidth}}}






\DeclareMathOperator{\arccot}{arccot}
\DeclareMathOperator{\arcsec}{arcsec}
\DeclareMathOperator{\arccsc}{arccsc}

















%%This is to help with formatting on future title pages.
\newenvironment{sectionOutcomes}{}{}



\outcome{Identify a differential equation.}
\outcome{Verify a solution to a differential equation.}
\outcome{Compute a general solution to a differential equation via integration.}
\outcome{Solve initial value problems.}
\outcome{Determine the order of a differential equation.}


\title[Dig-In:]{Differential equations}

\begin{document}
\begin{abstract}
  Differential equations show you relationships between rates of
  functions.
\end{abstract}
\maketitle

A \textit{differential equation}\index{differential equation} is
simply an equation with a derivative in it. Here is an example:
\[
a\cdot f''(x) + b\cdot f'(x) + c\cdot f(x) = g(x). 
\]
\begin{question}
  What is a differential equation?
  \begin{multipleChoice}
    \choice{An equation that you take the derivative of.}
    \choice[correct]{An equation that relates the rate of a function to other values.}
    \choice{It is a formula for the slope of a tangent line at a given point.}  
  \end{multipleChoice}
\end{question}

When a mathematician solves a differential equation, they are finding
\textit{functions} satisfying the equation.
\begin{question}
  Which of the following functions solve the differential equation
  \[
  f^{(4)}(x) = f(x)?
  \]
  \begin{hint}
    Remember, $f^{(4)}$ is the fourth derivative of $f$.
  \end{hint}
  \begin{selectAll}
    \choice[correct]{$f(x) = \sin(x)$}
    \choice{$f(x) = x^2$}
    \choice[correct]{$f(x) = e^x$}
    \choice[correct]{$f(x) = e^{-x}$}
    \choice{$f(x) = \tan(x)$}
  \end{selectAll}
  \begin{feedback}
    It turns out that the complete solution to this differential equation
    is $c_1\sin(x)+c_2\cos(x)+c_3e^x+c_4e^{-x}$.  In other words, every
    solution to this differential equation can be written in this form.
    You should check that these are all solutions (for example $f(x) =
    \sin(x)+3\cos(x)-7e^x+\pi e^{-x}$ is a solution).  Proving that these
    are \textbf{all} of the solutions is beyond the scope of this course.
  \end{feedback}
\end{question}


The differential equation above is an example of a fourth \textbf{order}
differential equation, because the highest derivative in the equation
is a ``fourth'' derivative. In general the highest derivative in a
differential equation is the order.


Differential equations are one of the most practical objects of
mathematical study.  They appear constantly in every field of science
and engineering.  They are a powerful way to model many diverse
situations.





\section{Modeling with differential equations}

Setting up differential equations is a skill to be acquired. However,
you can try your hand with our next question.

\begin{question}
  Imagine that a glass of water has initial temperature $5^\circ
  \unit{C}$, and that the ambient temperature is $22^\circ \unit{C}$.
  The water will warm up over time.  Assume that the rate of change in
  the temperature of the water is directly proportional to the
  difference between the current water temperature and the ambient
  temperature.  Which of the following differential equations must be
  satisfied by the function $H(t)$ which measures the temperature of
  the water with respect to time?
  \begin{multipleChoice}
    \choice{$y' = 5+\frac{y}{22}$}
    \choice[correct]{$y' = k(22-y)$ for some $k>0$}
    \choice{$y' = k(y-22)$ for some $k>0$}
    \choice{$y' = k(5-y)$ for some $k>0$}
    \choice{$y' = k(y-5)$ for some $k>0$}
  \end{multipleChoice}
  \begin{hint}
    This is just a straight translation job.  ``The rate of change in
    the temperature of the water'' is $y'$.  ``Directly proportional
    to'' means that it is equal to some constant (say $k$) times
    whatever it is proportional to.  ``The difference between the
    current water temperature and the ambient temperature" is either
    $22-y$ or $y-22$, since $y$ is the temperature of the water and
    $22$ is the ambient temperature.  Think about which we should
    choose before looking at the next hint.  Will it be $y'=k(22-y)$
    or $y'=k(y-22)$ where $k>0$?
  \end{hint}
  \begin{hint}
    Since the temperature of the water is increasing over time, we
    want $y'>0$.  Since the temperature will be increasing (and it is
    reasonable to assume it never surpasses the ambient temperature!)
    $22-y$ is positive.  So we can conclude that $y' = k(22-y)$ for
    some $k>0$.
  \end{hint}
  
  \begin{feedback}
    The differential equation does not involve the number $5$.  If we
    wanted to incorporate that piece of data into our model we could
    ask ``Which solution(s) to this differential equation satisfy
    $H(0) = 5$?''  This is known as an \textbf{initial value problem}.  %%%
  \end{feedback}
\end{question}


Sometimes the rates in question are constant. 



\begin{question}
  One can approximate the force of gravity as constant near the Earth.
  So the acceleration of a falling object is a constant $g>0$.  If
  $h(t)$ is the height of an object at time $t$, which differential
  equation must $h$ satisfy?
  \begin{multipleChoice}
    \choice{$y' = g$}
    \choice{$y=-g$}
    \choice[correct]{$y''=-g$}
    \choice{$y'' = g$}
    \choice{$y'' = -gy$}
  \end{multipleChoice}
  
  \begin{hint}
    The acceleration of an object is the second derivative of its
    position, so the differential equation should say the second
    derivative, $y''$, is constant.  Should it be a positive of
    negative constant?
  \end{hint}
  \begin{hint}
    A falling object will fall quicker and quicker, so the second
    derivative of its height should be negative.  Thus $y''=-g$ is the
    correct answer.
  \end{hint}	
\end{question}

\section{Initial value problems}


We have already seen, and solved, a particular kind of differential
equation in this course.  Namely a solution $F$ to the differential
equation $y' = f(x)$ is just an antiderivative of $f$!  We know the
``general solution'' of this differential equation is just $F(x)+C$,
as long as the domain of $f$ is an interval.  We can use this idea to
solve differential equations of the form $y^{(n)} = f(x)$, by just
repeatedly integrating and solving for ``$C$'' when we can.

\begin{example}
  Find the general solution to the differential equation $y''(x) = x$.
  \begin{explanation}
    Since $y''(x) = x$, we know that
    \begin{align*}
      y'(x) &= \int x dx\\
      &= \answer[given]{\frac{x^2}{2}} + C_1.
    \end{align*}
    This further implies that
    \[
    y(x) = \int \frac{x^2}{2} + C_1 dx,
    \]
    so we must have that
    \[
    y(x) = \answer[given]{\frac{x^3}{6}}+C_1x+C_2,
    \]
    for some constant $C_2$.  This is the general solution of the
    differential equation, in fact, \textit{every} solution of this
    differential equation is of the form $y(x) =
    \frac{1}{6}x^3+C_1x+C_2$.
  \end{explanation}
\end{example}

We can use the general solution to give specific solutions.

\begin{example}
  Find the solution to the differential equation $y''(x) = x$ that
  passes through the points $(0,1)$ and $(1,2)$.
  \begin{explanation}
    From our work above we know that
    \[
    y(x) = \frac{x^3}{6}+C_1x+C_2.
    \]
    To find the particular solution we are interested in, we solve the
    system of equations
    \begin{align*}
      \answer[given]{1}= y(\answer[given]{0}) &= \frac{1}{6}\cdot 0^3+C_1\cdot 0+C_2,\\
      \answer[given]{2}= y(\answer[given]{1}) &= \frac{1}{6}\cdot 1^3+C_1\cdot 1+C_2.
    \end{align*}
    The first line tells us that $C_2 = \answer[given]{1}$, and now the second line is
    \[
    2= \frac{1}{6}+C_1+1.
    \]
    and so $C_1 = \answer[given]{\frac{5}{6}}$. Our solution is now
    \[
    y(x) = \answer[given]{\frac{x^3}{6}+\frac{5 x}{6}+1}.
    \]
  \end{explanation}
\end{example}



\end{document}
