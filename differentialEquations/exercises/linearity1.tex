\documentclass{ximera}

\graphicspath{
  {./}
  {ximeraTutorial/}
  {basicPhilosophy/}
}

\newcommand{\mooculus}{\textsf{\textbf{MOOC}\textnormal{\textsf{ULUS}}}}

\usepackage{tkz-euclide}\usepackage{tikz}
\usepackage{tikz-cd}
\usetikzlibrary{arrows}
\tikzset{>=stealth,commutative diagrams/.cd,
  arrow style=tikz,diagrams={>=stealth}} %% cool arrow head
\tikzset{shorten <>/.style={ shorten >=#1, shorten <=#1 } } %% allows shorter vectors

\usetikzlibrary{backgrounds} %% for boxes around graphs
\usetikzlibrary{shapes,positioning}  %% Clouds and stars
\usetikzlibrary{matrix} %% for matrix
\usepgfplotslibrary{polar} %% for polar plots
\usepgfplotslibrary{fillbetween} %% to shade area between curves in TikZ
\usetkzobj{all}
\usepackage[makeroom]{cancel} %% for strike outs
%\usepackage{mathtools} %% for pretty underbrace % Breaks Ximera
%\usepackage{multicol}
\usepackage{pgffor} %% required for integral for loops



%% http://tex.stackexchange.com/questions/66490/drawing-a-tikz-arc-specifying-the-center
%% Draws beach ball
\tikzset{pics/carc/.style args={#1:#2:#3}{code={\draw[pic actions] (#1:#3) arc(#1:#2:#3);}}}



\usepackage{array}
\setlength{\extrarowheight}{+.1cm}
\newdimen\digitwidth
\settowidth\digitwidth{9}
\def\divrule#1#2{
\noalign{\moveright#1\digitwidth
\vbox{\hrule width#2\digitwidth}}}






\DeclareMathOperator{\arccot}{arccot}
\DeclareMathOperator{\arcsec}{arcsec}
\DeclareMathOperator{\arccsc}{arccsc}

















%%This is to help with formatting on future title pages.
\newenvironment{sectionOutcomes}{}{}

\author{Jim Talamo}
\license{Creative Commons 3.0 By-NC}
\outcome{Verify solutions to differential equations}
\begin{document}


\begin{exercise}
Select all of the following differential equations below that are \emph{linear}

\begin{selectAll}
\choice[correct]{$\frac{dy}{dx}+xy=3$}
\choice{$y\frac{dy}{dx}+y=3x$}
\choice[correct]{$\frac{dy}{dx}+2x=e^x$}
\choice{$\frac{dy}{dx}+2x=e^y$}
\choice[correct]{$x^2\frac{d^2y}{dx^2}+2xy=\arctan(2x^2)$}
\choice{$\frac{dy}{dx}+\sqrt{y}=3x$}
\end{selectAll}

\begin{hint}
Recall that an $n$-th order differential equation written in the form $f(x,y(x),y'(x), \ldots y^{(n)}(x))=0$ is \emph{linear} if $f$ is linear in $y$ and its derivatives. 

For first order equations, this means an equation is linear if it can be written in the form:

\[
\frac{dy}{dx}+p(x)y(x)=q(x)
\]


For second order equations, this means an equation is linear if it can be written in the form:

\[
\frac{d^2y}{dx^2}+p(x)\frac{dy}{dx}+q(x)y(x)=r(x)
\]

Note that the functions of $x$ that serve as the ``coefficients'' of $y$ and its derivatives can be anything!
\end{hint}
\end{exercise}

\end{document}