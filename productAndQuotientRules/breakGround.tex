\documentclass{ximera}


\graphicspath{
  {./}
  {ximeraTutorial/}
  {basicPhilosophy/}
}

\newcommand{\mooculus}{\textsf{\textbf{MOOC}\textnormal{\textsf{ULUS}}}}

\usepackage{tkz-euclide}\usepackage{tikz}
\usepackage{tikz-cd}
\usetikzlibrary{arrows}
\tikzset{>=stealth,commutative diagrams/.cd,
  arrow style=tikz,diagrams={>=stealth}} %% cool arrow head
\tikzset{shorten <>/.style={ shorten >=#1, shorten <=#1 } } %% allows shorter vectors

\usetikzlibrary{backgrounds} %% for boxes around graphs
\usetikzlibrary{shapes,positioning}  %% Clouds and stars
\usetikzlibrary{matrix} %% for matrix
\usepgfplotslibrary{polar} %% for polar plots
\usepgfplotslibrary{fillbetween} %% to shade area between curves in TikZ
\usetkzobj{all}
\usepackage[makeroom]{cancel} %% for strike outs
%\usepackage{mathtools} %% for pretty underbrace % Breaks Ximera
%\usepackage{multicol}
\usepackage{pgffor} %% required for integral for loops



%% http://tex.stackexchange.com/questions/66490/drawing-a-tikz-arc-specifying-the-center
%% Draws beach ball
\tikzset{pics/carc/.style args={#1:#2:#3}{code={\draw[pic actions] (#1:#3) arc(#1:#2:#3);}}}



\usepackage{array}
\setlength{\extrarowheight}{+.1cm}
\newdimen\digitwidth
\settowidth\digitwidth{9}
\def\divrule#1#2{
\noalign{\moveright#1\digitwidth
\vbox{\hrule width#2\digitwidth}}}






\DeclareMathOperator{\arccot}{arccot}
\DeclareMathOperator{\arcsec}{arcsec}
\DeclareMathOperator{\arccsc}{arccsc}

















%%This is to help with formatting on future title pages.
\newenvironment{sectionOutcomes}{}{}


\outcome{Explain why the product rule is not given by multiplying the
	derivatives of the products.}
\outcome{Apply the sum rule repeatedly to find the derivative of a product.}
\outcome{Relate the sum rule, the constant multiple rule, and the product rule.}


\title[Break-Ground:]{Derivatives of products are tricky}

\begin{document}
\begin{abstract}
Two young mathematicians discuss derivatives of products and products
of derivatives.
\end{abstract}
\maketitle

Check out this dialogue between two calculus students (based on a true
story):

\begin{dialogue}
\item[Devyn] Hey Riley, remember the sum rule for derivatives?
\item[Riley] You know I do.
\item[Devyn] What do you think that the ``product rule'' will be?
\item[Riley] Let's give this a spin:
  \[
  \frac{d}{dx} \left(f(x)\cdot g(x)\right) = f'(x) \cdot g'(x)?
  \]
\item[Devyn] Hmmm, let's give this theory an acid test. Let's try
  \[
  f(x) = x^2+1\qquad\text{and}\qquad g(x) = x^3-3x
  \]
  Now
  \begin{align*}
    f'(x)g'(x) &= (2x)(3x^2-3)\\
    &= 6x^3-6x.
  \end{align*}
\item[Riley] On the other hand,
  \begin{align*}
    f(x)g(x) &= (x^2+1)(x^3-3x)\\
    &=x^5-3x^3+x^3-3x\\
    &=x^5-2x^3-3x.
  \end{align*} 
\item[Devyn] And so, 
  \[
  \frac{d}{dx} \left(f(x) \cdot g(x)\right) = 5x^4-6x^2-3.
  \]
\item[Riley] Wow. Hmmm. It looks like our guess was incorrect.
\item[Devyn] I've got a feeling that the so-called ``product rule''
  might be a bit tricky.
\end{dialogue}

\begin{problem}
  Above, our intrepid young mathematicians guess that the ``product rule'' might be:
  \[
  \frac{d}{dx} \left(f(x)\cdot g(x)\right) = f'(x) \cdot g'(x)?
  \]
  Does this \textbf{ever} hold true?
  \begin{freeResponse}
    Answers will vary. A partial answer is that this will hold when
    either $f(x)$ or $g(x)$ are zero, or when both are constants.
  \end{freeResponse}
\end{problem}



%\input{../leveledQuestions.tex}


\end{document}
