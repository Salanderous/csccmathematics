\documentclass{ximera}


\graphicspath{
  {./}
  {ximeraTutorial/}
  {basicPhilosophy/}
}

\newcommand{\mooculus}{\textsf{\textbf{MOOC}\textnormal{\textsf{ULUS}}}}

\usepackage{tkz-euclide}\usepackage{tikz}
\usepackage{tikz-cd}
\usetikzlibrary{arrows}
\tikzset{>=stealth,commutative diagrams/.cd,
  arrow style=tikz,diagrams={>=stealth}} %% cool arrow head
\tikzset{shorten <>/.style={ shorten >=#1, shorten <=#1 } } %% allows shorter vectors

\usetikzlibrary{backgrounds} %% for boxes around graphs
\usetikzlibrary{shapes,positioning}  %% Clouds and stars
\usetikzlibrary{matrix} %% for matrix
\usepgfplotslibrary{polar} %% for polar plots
\usepgfplotslibrary{fillbetween} %% to shade area between curves in TikZ
\usetkzobj{all}
\usepackage[makeroom]{cancel} %% for strike outs
%\usepackage{mathtools} %% for pretty underbrace % Breaks Ximera
%\usepackage{multicol}
\usepackage{pgffor} %% required for integral for loops



%% http://tex.stackexchange.com/questions/66490/drawing-a-tikz-arc-specifying-the-center
%% Draws beach ball
\tikzset{pics/carc/.style args={#1:#2:#3}{code={\draw[pic actions] (#1:#3) arc(#1:#2:#3);}}}



\usepackage{array}
\setlength{\extrarowheight}{+.1cm}
\newdimen\digitwidth
\settowidth\digitwidth{9}
\def\divrule#1#2{
\noalign{\moveright#1\digitwidth
\vbox{\hrule width#2\digitwidth}}}






\DeclareMathOperator{\arccot}{arccot}
\DeclareMathOperator{\arcsec}{arcsec}
\DeclareMathOperator{\arccsc}{arccsc}

















%%This is to help with formatting on future title pages.
\newenvironment{sectionOutcomes}{}{}


\author{Jason Miller}
\license{Creative Commons 3.0 By-bC}


\outcome{}

\begin{document}
\begin{exercise}


Consider the polar curve $r=\cos(\theta)$. The graph is show below. 




\begin{image}  
  \begin{tikzpicture}  
    \begin{axis}[  
        xmin=-1,  
        xmax=1.2,  
        ymin=-1,  
        ymax=1,  
        axis lines=center,  
        xlabel=$x$,  
        ylabel=$y$,  
        every axis y label/.style={at=(current axis.above origin),anchor=south},  
        every axis x label/.style={at=(current axis.right of origin),anchor=west},  
      ]  
      \addplot[data cs=polar,penColor,domain=0:360,samples=360,smooth, thick] (x,{cos(x)});
            \end{axis}  
  \end{tikzpicture}  
\end{image} 

We want to find all the horizontal and vertical tangent lines to this curve.

First we find the $\theta$ values where $r=\cos(\theta)$ has a horizontal tangent line and give the equation of the line. 

Since our curve can be generated by letting $\theta$ vary from over the interval $[0, \pi)$, we can assume the $\theta$ values lie in $[0, \pi)$. List the $\theta$ values in order from smallest to largest below: 


When $\theta=\answer{\frac{\pi}{4}}$ the equation of the tangent line is $\answer{y=\frac{1}{2}  }$ 

and when $\theta=\answer{\frac{3\pi}{4}}$ the equation of the tangent line is $\answer{y=-\frac{1}{2}}$. 




\begin{hint}

Consider the formulas for changing from polar coordinates to Cartesian coordinates:

\begin{align*}
x&=r\cos(\theta) \\
y&=r\sin(\theta)
\end{align*}

Using the equation of our curve $r=\cos(\theta)$, we can substitute for $r$ to obtain:

\begin{align*}
x&=\answer{\cos(\theta)\cos(\theta)} \\
y&=\answer{\cos(\theta)\sin(\theta) }
\end{align*}

Note that we have expressed both the $x$ and $y$ coordinates of the points of the curve in terms of functions of a single parameter $\theta$.

Since we have a parametric description of our curve in terms of $\theta$, 
we can use the chain rule to express $\frac{dy}{dx}$ in terms of $\theta$ to get $\frac{dy}{dx}=\frac{ dy/d\theta}{dx/d\theta}$.

Calculating we get

\begin{align*}
\frac{dx}{\theta}&=\answer{ -2\cos(\theta)\sin(\theta)  } \\
\frac{dy}{\theta}&=\answer{ \cos^2(\theta)-\sin^2(\theta)    }
\end{align*} 

Thus we find $\frac{dy}{dx}=\answer{ \frac{ \cos^2(\theta)-\sin^2(\theta)}{-2\cos(\theta)\sin(\theta)}}$. 

The tangent line is horizontal when the slope is $0$. Therefore we need to find where $\frac{dy}{dx}$ is equal to $0$.

Setting the numerator of $\frac{dy}{dx}$ equal to $0$ gives us $\cos^2(\theta)-\sin^2(\theta)=0$. 

We can rewrite this equation in terms of tangent as $\answer{ \tan^2(\theta)=1 }$. 

Taking square roots of both sides this becomes two equations

\begin{align*}
 \tan(\theta)&=1 \\
\tan(\theta)&=-1
\end{align*}

Notice that our curve $r=\cos(\theta)$ can be generated by letting $\theta$ vary over the interval $[0, \pi)$. 

Therefore we need to solve these equations for $\theta$ values in  $[0, \pi)$. 

The only solution for $\tan(\theta)=1$ in this interval is $\theta=\answer{\frac{\pi}{4}}$ and the only solution 
for $\tan(\theta)=-1$ in this interval is $\theta=\answer{\frac{3\pi}{4}}$. 



\end{hint}

\begin{exercise}



Now we want to find the $\theta$ values where $r=\cos(\theta)$ has a vertical tangent line and give the equation of the line. 

Since our curve can be generated by letting $\theta$ vary from over the interval $[0, \pi)$, we can assume the $\theta$ values lie in $[0, \pi)$. List the $\theta$ values in order below: 

When $\theta=\answer{0}$ the equation of the tangent line is $\answer{x=1 }$ 

and when $\theta=\answer{ \frac{\pi}{2}}$ the equation of the tangent line is $\answer{x=0}$. 





\begin{hint}


In the hint for exercise 1 we derived a parametric description of our curve in terms of $\theta$: 

\begin{align*}
\frac{dx}{\theta}&=\answer{ -2\cos(\theta)\sin(\theta)  } \\
\frac{dy}{\theta}&=\answer{ \cos^2(\theta)-\sin^2(\theta)    }
\end{align*} 

Where we calculated $\frac{dy}{dx}=\answer{ \frac{ \cos^2(\theta)-\sin^2(\theta)}{-2\cos(\theta)\sin(\theta)}}$. 

The tangent line is vertical when the slope becomes infinite. We can check for this by looking for where the denominator of $\frac{dy}{dx}$ is equal to $0$.

Setting the denominator equal to $0$ gives us $-2\cos(\theta)\sin(\theta)=0$. 

Setting $\cos(\theta)=0$ gives us $\theta=\answer{ \frac{\pi}{2}}$ and setting $\sin(\theta)=0$ gives us 
$\theta=\answer{ 0}$ (use $\theta$ values that lie in $[0, \pi)$). 



\end{hint}

\end{exercise}
\end{exercise}
\end{document}
