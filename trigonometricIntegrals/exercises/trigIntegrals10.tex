\documentclass{ximera}


\graphicspath{
  {./}
  {ximeraTutorial/}
  {basicPhilosophy/}
}

\newcommand{\mooculus}{\textsf{\textbf{MOOC}\textnormal{\textsf{ULUS}}}}

\usepackage{tkz-euclide}\usepackage{tikz}
\usepackage{tikz-cd}
\usetikzlibrary{arrows}
\tikzset{>=stealth,commutative diagrams/.cd,
  arrow style=tikz,diagrams={>=stealth}} %% cool arrow head
\tikzset{shorten <>/.style={ shorten >=#1, shorten <=#1 } } %% allows shorter vectors

\usetikzlibrary{backgrounds} %% for boxes around graphs
\usetikzlibrary{shapes,positioning}  %% Clouds and stars
\usetikzlibrary{matrix} %% for matrix
\usepgfplotslibrary{polar} %% for polar plots
\usepgfplotslibrary{fillbetween} %% to shade area between curves in TikZ
\usetkzobj{all}
\usepackage[makeroom]{cancel} %% for strike outs
%\usepackage{mathtools} %% for pretty underbrace % Breaks Ximera
%\usepackage{multicol}
\usepackage{pgffor} %% required for integral for loops



%% http://tex.stackexchange.com/questions/66490/drawing-a-tikz-arc-specifying-the-center
%% Draws beach ball
\tikzset{pics/carc/.style args={#1:#2:#3}{code={\draw[pic actions] (#1:#3) arc(#1:#2:#3);}}}



\usepackage{array}
\setlength{\extrarowheight}{+.1cm}
\newdimen\digitwidth
\settowidth\digitwidth{9}
\def\divrule#1#2{
\noalign{\moveright#1\digitwidth
\vbox{\hrule width#2\digitwidth}}}






\DeclareMathOperator{\arccot}{arccot}
\DeclareMathOperator{\arcsec}{arcsec}
\DeclareMathOperator{\arccsc}{arccsc}

















%%This is to help with formatting on future title pages.
\newenvironment{sectionOutcomes}{}{}


\author{Jason Miller}
\license{Creative Commons 3.0 By-NC}


\outcome{ Recognize the patterns that appear in trigonometric integrals and use appropriate substitutions 
to compute them.}


\begin{document}
\begin{exercise}
Determine the integral

\[
\int \sec^{2}(x) \tan(x) dx
\]

We solve this integral in two different ways


First let $u=\tan(x)$. Then $du=\answer{ \sec^{2}(x)}$. 

Then the integral in terms of $u$ is
\[
\int \answer{u} du
\]

Integrating and going back to $x$ we get:
\[
\int \sec^{2}(x)\tan(x) dx= \answer{  \frac{ \tan^{2}(x)}{2}  + C}
\]
Use $C$ for the constant of integration. 

\begin{exercise}

Now let us do a different substitution to find the integral. Let $u=\sec(x)$. Then $du=\answer{ \sec(x) \tan(x)} dx$

Then the integral in terms of $u$ becomes:

\[
\int \answer{u} du 
\]

Integrating and going back to $x$ gives us:

\[
\int \sec^{2}(x) \tan(x) dx= \answer{  \frac{ \sec^{2}(x) }{2}+ C}
\]
Use $C$ for the constant of integration. 

Note that the answer you obtained from the second substitution seems to differ from the answer we obtained using the first substitution. Recall there is a theorem that all antiderivatives 
of a given function can only differ by a constant. Can you explain this apparent discrepancy?
\end{exercise}



\end{exercise}
\end{document}
