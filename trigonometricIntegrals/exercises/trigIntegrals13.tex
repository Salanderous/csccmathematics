\documentclass{ximera}


\graphicspath{
  {./}
  {ximeraTutorial/}
  {basicPhilosophy/}
}

\newcommand{\mooculus}{\textsf{\textbf{MOOC}\textnormal{\textsf{ULUS}}}}

\usepackage{tkz-euclide}\usepackage{tikz}
\usepackage{tikz-cd}
\usetikzlibrary{arrows}
\tikzset{>=stealth,commutative diagrams/.cd,
  arrow style=tikz,diagrams={>=stealth}} %% cool arrow head
\tikzset{shorten <>/.style={ shorten >=#1, shorten <=#1 } } %% allows shorter vectors

\usetikzlibrary{backgrounds} %% for boxes around graphs
\usetikzlibrary{shapes,positioning}  %% Clouds and stars
\usetikzlibrary{matrix} %% for matrix
\usepgfplotslibrary{polar} %% for polar plots
\usepgfplotslibrary{fillbetween} %% to shade area between curves in TikZ
\usetkzobj{all}
\usepackage[makeroom]{cancel} %% for strike outs
%\usepackage{mathtools} %% for pretty underbrace % Breaks Ximera
%\usepackage{multicol}
\usepackage{pgffor} %% required for integral for loops



%% http://tex.stackexchange.com/questions/66490/drawing-a-tikz-arc-specifying-the-center
%% Draws beach ball
\tikzset{pics/carc/.style args={#1:#2:#3}{code={\draw[pic actions] (#1:#3) arc(#1:#2:#3);}}}



\usepackage{array}
\setlength{\extrarowheight}{+.1cm}
\newdimen\digitwidth
\settowidth\digitwidth{9}
\def\divrule#1#2{
\noalign{\moveright#1\digitwidth
\vbox{\hrule width#2\digitwidth}}}






\DeclareMathOperator{\arccot}{arccot}
\DeclareMathOperator{\arcsec}{arcsec}
\DeclareMathOperator{\arccsc}{arccsc}

















%%This is to help with formatting on future title pages.
\newenvironment{sectionOutcomes}{}{}


\author{Jason Miller}
\license{Creative Commons 3.0 By-NC}


\outcome{Recognize the patterns that appear in trigonometric integrals
  and use appropriate substitutions to compute them.}


\begin{document}
\begin{exercise}
Using the substitution $u= \csc(x)$, determine the integral:

\[
\int \cot^{3}(x) \csc^{2}(x) dx=\answer{- \frac{1}{4}\csc^{4}(x) + \frac{1}{2}\csc^{2}(x) +C }                  
\]         
(Use $C$ for the constant of integration)

Using the substitution $u= \cot(x)$, determine the integral:

\[
\int \cot^{3}(x) \csc^{2}(x) dx=\answer{-\frac{1}{4} \cot^4(x) +C }                  
\]         
(Use $C$ for the constant of integration)

\begin{multipleChoice}
\choice{These results are inconsistent with each other}
\choice[correct]{While these results look different, we can use trigonometric identities to show they differ by a constant}
\end{multipleChoice}

\begin{exercise}
Indeed, since both techniques work, we know that the antiderivatives must differ by a constant.  If we use the identity $\cot^2(x) = \csc^2(x)-1$, we find:

\begin{align*}
-\frac{1}{4} \cot^4(x) & = -\frac{1}{4} \left(\csc^2(x)-1\right)^2 \\
&= -\frac{1}{4}\left( \answer{\csc^4(x) - 2 \csc^2(x) +1}\right)  \textrm{ (expand the above expression) } \\
&= \answer{-\frac{1}{4}} \csc^4(x) + \answer{\frac{1}{2}} \csc^2(x) +\answer{-\frac{1}{4}}  \textrm{ (expand the above expression) } \\
\end{align*}
which differs from the answer to the first part by the constant $\answer{-\frac{1}{4}}$.

\end{exercise}
\end{exercise}

\end{document}
