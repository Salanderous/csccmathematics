\documentclass{ximera}


\graphicspath{
  {./}
  {ximeraTutorial/}
  {basicPhilosophy/}
}

\newcommand{\mooculus}{\textsf{\textbf{MOOC}\textnormal{\textsf{ULUS}}}}

\usepackage{tkz-euclide}\usepackage{tikz}
\usepackage{tikz-cd}
\usetikzlibrary{arrows}
\tikzset{>=stealth,commutative diagrams/.cd,
  arrow style=tikz,diagrams={>=stealth}} %% cool arrow head
\tikzset{shorten <>/.style={ shorten >=#1, shorten <=#1 } } %% allows shorter vectors

\usetikzlibrary{backgrounds} %% for boxes around graphs
\usetikzlibrary{shapes,positioning}  %% Clouds and stars
\usetikzlibrary{matrix} %% for matrix
\usepgfplotslibrary{polar} %% for polar plots
\usepgfplotslibrary{fillbetween} %% to shade area between curves in TikZ
\usetkzobj{all}
\usepackage[makeroom]{cancel} %% for strike outs
%\usepackage{mathtools} %% for pretty underbrace % Breaks Ximera
%\usepackage{multicol}
\usepackage{pgffor} %% required for integral for loops



%% http://tex.stackexchange.com/questions/66490/drawing-a-tikz-arc-specifying-the-center
%% Draws beach ball
\tikzset{pics/carc/.style args={#1:#2:#3}{code={\draw[pic actions] (#1:#3) arc(#1:#2:#3);}}}



\usepackage{array}
\setlength{\extrarowheight}{+.1cm}
\newdimen\digitwidth
\settowidth\digitwidth{9}
\def\divrule#1#2{
\noalign{\moveright#1\digitwidth
\vbox{\hrule width#2\digitwidth}}}






\DeclareMathOperator{\arccot}{arccot}
\DeclareMathOperator{\arcsec}{arcsec}
\DeclareMathOperator{\arccsc}{arccsc}

















%%This is to help with formatting on future title pages.
\newenvironment{sectionOutcomes}{}{}


\author{Jim Talamo}
\license{Creative Commons 3.0 By-NC}


\outcome{Set up an integral with respect to $x$.}

\begin{document}

\begin{exercise}
Consider the region bounded by $y=e^x, y=2e^{-x}+1$, $x=0$, and $x=a$.  

What is the largest value of $a$ for which the area of the region can be computed as a single integral with respect to $x$? 

$a= \begin{prompt} \answer{\ln(2)} \end{prompt}$



\begin{hint}	

\begin{exercise}

How many integrals with respect to $x$ are needed to find the area below? $\answer{1}$
	\begin{image}
	\begin{tikzpicture}
		\begin{axis}[
			domain=-0.5:2.5, ymax=3.5,xmax=1.3, ymin=-0.25, xmin=-0.25,
			axis lines =center, xlabel=$x$, ylabel=$y$,
            		every axis y label/.style={at=(current axis.above origin),anchor=south},
            		every axis x label/.style={at=(current axis.right of origin),anchor=west},
            		axis on top,
            		]
                      
            	\addplot [draw=penColor,very thick,smooth,samples=800] {exp(x)};
            	\addplot [draw=penColor2,very thick,smooth] {2*exp(-x)+1};
	
	
		\addplot [name path=A,domain=0:.4,draw=none] {exp(x)};   
            	\addplot [name path=B,domain=0:.4,draw=none] {2*exp(-x)+1};
		\addplot [fillp] fill between[of=A and B];
		
		\node at (axis cs:0.3,1.1) [penColor] {$y=e^{x}$};
            	\node at (axis cs:0.5,2.7) [penColor2] {$y=2e^{-x}+1$};
                       
            	\end{axis}
	\end{tikzpicture}
	\end{image}
\end{exercise}

\begin{exercise}
	
How many integrals with respect to $x$ are needed to find the area below? $\answer{2}$
	\begin{image}
	\begin{tikzpicture}
		\begin{axis}[
			domain=-0.5:2.5, ymax=3.5,xmax=1.3, ymin=-0.25, xmin=-0.25,
			axis lines =center, xlabel=$x$, ylabel=$y$,
            		every axis y label/.style={at=(current axis.above origin),anchor=south},
            		every axis x label/.style={at=(current axis.right of origin),anchor=west},
            		axis on top,
            		]
                      
            	\addplot [draw=penColor,very thick,smooth,samples=800] {exp(x)};
            	\addplot [draw=penColor2,very thick,smooth] {2*exp(-x)+1};
	
	
		\addplot [name path=A,domain=0:1.2,draw=none] {exp(x)};   
            	\addplot [name path=B,domain=0:1.2,draw=none] {2*exp(-x)+1};
		\addplot [fillp] fill between[of=A and B];
		
		\node at (axis cs:0.3,1.1) [penColor] {$y=e^{x}$};
            	\node at (axis cs:0.5,2.7) [penColor2] {$y=2e^{-x}+1$};
                       
            	\end{axis}
	\end{tikzpicture}
	\end{image}

\end{exercise}

\begin{exercise}	
We can thus find $a$ by setting $e^x = \answer{2e^{-x}+1}$ and solving for $x$!

Now, let $r=e^x$.  Then, $2e^{-x} = \frac{2}{e^x} = \frac{2}{\answer{r}}$ (type an answer in terms of $r$).

The equation to solve is:

\[
\answer{r}=\frac{2}{\answer{r}}+1.
\]	

Multiply both sides by $r$ to obtain the quadratic equation $\answer{r^2-r-2}=0$.  

Solving this for $r$ gives $r=\answer{-1}$ and $r=\answer{2}$ (type the smaller answer first)

Now use the fact that $r=e^x$ and solve for $x$.  You should find that only one of the roots for $r$ produces a real value for $x$!

\end{exercise}	

	\end{hint}
\end{exercise}


\end{document}
