\documentclass{ximera}

\graphicspath{
  {./}
  {ximeraTutorial/}
  {basicPhilosophy/}
}

\newcommand{\mooculus}{\textsf{\textbf{MOOC}\textnormal{\textsf{ULUS}}}}

\usepackage{tkz-euclide}\usepackage{tikz}
\usepackage{tikz-cd}
\usetikzlibrary{arrows}
\tikzset{>=stealth,commutative diagrams/.cd,
  arrow style=tikz,diagrams={>=stealth}} %% cool arrow head
\tikzset{shorten <>/.style={ shorten >=#1, shorten <=#1 } } %% allows shorter vectors

\usetikzlibrary{backgrounds} %% for boxes around graphs
\usetikzlibrary{shapes,positioning}  %% Clouds and stars
\usetikzlibrary{matrix} %% for matrix
\usepgfplotslibrary{polar} %% for polar plots
\usepgfplotslibrary{fillbetween} %% to shade area between curves in TikZ
\usetkzobj{all}
\usepackage[makeroom]{cancel} %% for strike outs
%\usepackage{mathtools} %% for pretty underbrace % Breaks Ximera
%\usepackage{multicol}
\usepackage{pgffor} %% required for integral for loops



%% http://tex.stackexchange.com/questions/66490/drawing-a-tikz-arc-specifying-the-center
%% Draws beach ball
\tikzset{pics/carc/.style args={#1:#2:#3}{code={\draw[pic actions] (#1:#3) arc(#1:#2:#3);}}}



\usepackage{array}
\setlength{\extrarowheight}{+.1cm}
\newdimen\digitwidth
\settowidth\digitwidth{9}
\def\divrule#1#2{
\noalign{\moveright#1\digitwidth
\vbox{\hrule width#2\digitwidth}}}






\DeclareMathOperator{\arccot}{arccot}
\DeclareMathOperator{\arcsec}{arcsec}
\DeclareMathOperator{\arccsc}{arccsc}

















%%This is to help with formatting on future title pages.
\newenvironment{sectionOutcomes}{}{}

\author{Jim Talamo and Bart Snapp}
\license{Creative Commons 3.0 By-NC}
\outcome{Find area as an integral with respect to $y$.}
\begin{document}

\begin{exercise}
The following is a guided exercise to find the area of the region bounded by the functions $y = x-3$, $y =
\sqrt{x-1}$ and the $x$-axis:
\begin{image}
\begin{tikzpicture}
	\begin{axis}[
            domain=0:5.5, ymax=2.8,xmax=5.5, ymin=0, xmin=0,
            axis lines =center, xlabel=$x$, ylabel=$y$,
            every axis y label/.style={at=(current axis.above origin),anchor=south},
            every axis x label/.style={at=(current axis.right of origin),anchor=west},
            axis on top,
          ]
          \addplot [ fill = fillp, smooth, samples=100, domain=(0:2)] ({1+x^2},{x}) \closedcycle;
          \addplot [draw=none,fill=background,domain=0:5.2] {x-3} \closedcycle;   
          \addplot [very thick, penColor2, smooth, samples=100, domain=(0:3)] ({1+x^2},{x});
          \addplot [draw=penColor,very thick,smooth] {x-3};
          
          \node at (axis cs:2,1.5) [penColor2] {$y=\sqrt{x-1}$};
          \node at (axis cs:4.5,0.7) [penColor] {$y=x-3$};
        \end{axis}
\end{tikzpicture}
\end{image}

\begin{exercise}
In order to express this area as a single integral, we should:

\begin{multipleChoice}
\choice{integrate with respect to $x$.}
\choice[correct]{integrate with respect to $y$.}
\end{multipleChoice}


\begin{exercise}
Since we are computing the area by using an integral with respect to $y$ , we must use 
\begin{multipleChoice}
\choice{vertical slices.}
\choice[correct]{horizontal slices.}
\end{multipleChoice}


\begin{exercise}
Since we want to integrate with respect to $y$, we need to describe the curves as functions of $y$, the limits of integration must be $y$ limits, and we must find $h$ in terms of $y$.

\begin{image}
\begin{tikzpicture}
	\begin{axis}[
            domain=0:5.5, ymax=2.5,xmax=5.5, ymin=0, xmin=0,
            axis lines =center, xlabel=$x$, ylabel=$y$,
            every axis y label/.style={at=(current axis.above origin),anchor=south},
            every axis x label/.style={at=(current axis.right of origin),anchor=west},
            axis on top,
          ]
          \addplot [ fill = fillp, smooth, samples=100, domain=(0:2)] ({1+x^2},{x}) \closedcycle;
          \addplot [draw=none,fill=background,domain=0:5.2] {x-3} \closedcycle;   
          \addplot [very thick, penColor2, smooth, samples=100, domain=(0:3)] ({1+x^2},{x});
          \addplot [draw=penColor,very thick,smooth] {x-3};
          
          \node at (axis cs:2,1.5) [penColor2] {$y=\sqrt{x-1}$};
          \node at (axis cs:4.5,0.7) [penColor] {$y=x-3$};

	        \addplot [draw=penColor, fill = gray!50] plot coordinates {(2,1) (2,1.1) (4, 1.1) (4,1) (2, 1)};

          \draw[decoration={brace,raise=.2cm},decorate,thin] (axis cs:2,1)--(axis cs:2,1.1);
          \node at (axis cs:1.5,1.05) {$\Delta y$};

          \draw[decoration={brace,mirror,raise=.2cm},decorate,thin] (axis cs:2,1)--(axis cs:4,1);
          \node at (axis cs:3,.7) {$h$};
        \end{axis}
\end{tikzpicture}
\end{image}

For the curve described by $y=\sqrt{x-1}$, $x= \begin{prompt} \answer{y^2+1} \end{prompt}$.

For the curve described by $y=x-3$, $x= \begin{prompt} \answer{y+3} \end{prompt}$.

The lower limit of integration is $y= \begin{prompt} \answer{0} \end{prompt}$ and the upper limit of integration is $y= \begin{prompt} \answer{2} \end{prompt}$


\begin{hint}
 The upper limit is the $y$-value where the curves intersect: 
 
   \begin{align*}
    y+3 &= y^2 +1\\
    y^2-y-2 &= 0\\
    y &= -1 \text{ or }\answer[given]{2}.
  \end{align*}
  From the picture, note that $y=-1$ is not relevant for this problem!
\end{hint}

We must now express $h$ in terms of the variable of integration!  Since $h$ is a horizontal distance: 

\begin{multipleChoice}
\choice{$h=y_{top}-y_{bot}$} 
\choice[correct]{$h=x_{right}-x_{left}$}.
\end{multipleChoice}

\begin{exercise}

The function used to determine the rightmost $x$-value, $x_{right}$ is:
\begin{multipleChoice}
\choice[correct]{$x_{right}=y+3$}
\choice{$x_{left}=y^2+1$}
\end{multipleChoice}

The function used to determine  the leftmost $x$-value, $x_{left}$ is:
\begin{multipleChoice}
\choice{$x_{right}=y+3$}
\choice[correct]{$x_{left}=y^2+1$}
\end{multipleChoice}

The height $h$ of the rectangle is thus $h=x_{right}-x_{left} = \answer{(y+3)-(y^2+1)}$.


\begin{exercise}

Thus, the area is given by:
  \[
 \int_{y=c}^{y=d} h dy =  \int_{y=\answer{0}}^{y=\answer{2}} \answer{(y+3) - (y^2+1)} dy = \answer{\frac{10}{3}}.
  \]
  \begin{hint}
    \begin{align*}
      \int_0^2 (y+3) - (y^2+1) dy &= \int_0^2 -y^2+y+2 dy\\
      &=\bigg[\answer{\frac{-y^3}{3} + \frac{y^2}{2}+2y}\bigg]_0^2\\
      &=\answer{\frac{10}{3}}.
    \end{align*}
  \end{hint}

\end{exercise}
\end{exercise}
\end{exercise}
\end{exercise}
\end{exercise}
\end{exercise}
\end{document}