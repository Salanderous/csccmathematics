\documentclass{ximera}


\graphicspath{
  {./}
  {ximeraTutorial/}
  {basicPhilosophy/}
}

\newcommand{\mooculus}{\textsf{\textbf{MOOC}\textnormal{\textsf{ULUS}}}}

\usepackage{tkz-euclide}\usepackage{tikz}
\usepackage{tikz-cd}
\usetikzlibrary{arrows}
\tikzset{>=stealth,commutative diagrams/.cd,
  arrow style=tikz,diagrams={>=stealth}} %% cool arrow head
\tikzset{shorten <>/.style={ shorten >=#1, shorten <=#1 } } %% allows shorter vectors

\usetikzlibrary{backgrounds} %% for boxes around graphs
\usetikzlibrary{shapes,positioning}  %% Clouds and stars
\usetikzlibrary{matrix} %% for matrix
\usepgfplotslibrary{polar} %% for polar plots
\usepgfplotslibrary{fillbetween} %% to shade area between curves in TikZ
\usetkzobj{all}
\usepackage[makeroom]{cancel} %% for strike outs
%\usepackage{mathtools} %% for pretty underbrace % Breaks Ximera
%\usepackage{multicol}
\usepackage{pgffor} %% required for integral for loops



%% http://tex.stackexchange.com/questions/66490/drawing-a-tikz-arc-specifying-the-center
%% Draws beach ball
\tikzset{pics/carc/.style args={#1:#2:#3}{code={\draw[pic actions] (#1:#3) arc(#1:#2:#3);}}}



\usepackage{array}
\setlength{\extrarowheight}{+.1cm}
\newdimen\digitwidth
\settowidth\digitwidth{9}
\def\divrule#1#2{
\noalign{\moveright#1\digitwidth
\vbox{\hrule width#2\digitwidth}}}






\DeclareMathOperator{\arccot}{arccot}
\DeclareMathOperator{\arcsec}{arcsec}
\DeclareMathOperator{\arccsc}{arccsc}

















%%This is to help with formatting on future title pages.
\newenvironment{sectionOutcomes}{}{}


%\outcome{Given a velocity function, calculate displacement and distance traveled.}
%\outcome{Given a velocity function, find the position function.}
%\outcome{Given an acceleration function, find the velocity function.}
%\outcome{Understand the difference between displacement and distance traveled.}
%\outcome{Understand the relationship between position, velocity and acceleration.}

\author{Nela Lakos \and Kyle Parsons}

\begin{document}
\begin{exercise}

The graph of a function $f$ on the interval $[1,9]$ is given below.

\begin{image}
  \begin{tikzpicture}
    \begin{axis}[
        xmin=-0.3,xmax=10.3,ymin=-0.3,ymax=6.3,
        clip=true,
        unit vector ratio*=1 1 1,
        axis lines=center,
        grid = major,
        ytick={-1,0,...,36},
        xtick={0,1,...,10},
        xlabel=$x$, ylabel=$y$,
        every axis y label/.style={at=(current axis.above origin),anchor=south},
        every axis x label/.style={at=(current axis.right of origin),anchor=west},
      ]
      \addplot[ultra thick,penColor,domain=1:5,samples=2] {x+1};    
      \addplot[ultra thick,penColor,domain=5:9,samples=2] {11-x};
        
      \node at (axis cs:1.5,5.5) {$y=f(x)$};
      \end{axis}`
  \end{tikzpicture}
\end{image}

Using geometry we can evaluate $\int_1^9f(x) dx$ as
\[
\int_1^9f(x) dx = \answer{32}.
\]
With this, we can calculate $\bar{f}$, the average value of $f$ on $[1,9]$.
\[
 \bar{f}=\answer{4}.
 \]

The rectangle with base $[1,9]$ that has area $\int_1^9f(x) dx$ has height
\[
h=\answer{4}.
\]

\begin{image}
  \begin{tikzpicture}
    \begin{axis}[
        xmin=-0.3,xmax=10.3,ymin=-0.3,ymax=6.3,
        clip=true,
        unit vector ratio*=1 1 1,
        axis lines=center,
        grid = major,
        ytick={2,4,6},
        xtick={0,1,...,10},
        xlabel=$x$, ylabel=$y$,
        every axis y label/.style={at=(current axis.above origin),anchor=south},
        every axis x label/.style={at=(current axis.right of origin),anchor=west},
      ]
      \addplot[ultra thick,penColor,domain=1:5,samples=2] {x+1};    
      \addplot[ultra thick,penColor,domain=5:9,samples=2] {11-x};
      \filldraw[penColor2,opacity=0.3] (axis cs:1,0) rectangle (axis cs:9,4);
        
      \node at (axis cs:1.5,5.5) {$y=f(x)$};
      \end{axis}`
  \end{tikzpicture}
\end{image}

The points in $[1,9]$ where $f(c) = \bar{f}$, from left to right, are
\[
c_{1}=\answer{3}
\]
 and
 \[ 
 c_{2}=\answer{7} .
 \]
 

\end{exercise}
\end{document}