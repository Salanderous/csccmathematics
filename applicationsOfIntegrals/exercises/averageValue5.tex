\documentclass{ximera}

\graphicspath{
  {./}
  {ximeraTutorial/}
  {basicPhilosophy/}
}

\newcommand{\mooculus}{\textsf{\textbf{MOOC}\textnormal{\textsf{ULUS}}}}

\usepackage{tkz-euclide}\usepackage{tikz}
\usepackage{tikz-cd}
\usetikzlibrary{arrows}
\tikzset{>=stealth,commutative diagrams/.cd,
  arrow style=tikz,diagrams={>=stealth}} %% cool arrow head
\tikzset{shorten <>/.style={ shorten >=#1, shorten <=#1 } } %% allows shorter vectors

\usetikzlibrary{backgrounds} %% for boxes around graphs
\usetikzlibrary{shapes,positioning}  %% Clouds and stars
\usetikzlibrary{matrix} %% for matrix
\usepgfplotslibrary{polar} %% for polar plots
\usepgfplotslibrary{fillbetween} %% to shade area between curves in TikZ
\usetkzobj{all}
\usepackage[makeroom]{cancel} %% for strike outs
%\usepackage{mathtools} %% for pretty underbrace % Breaks Ximera
%\usepackage{multicol}
\usepackage{pgffor} %% required for integral for loops



%% http://tex.stackexchange.com/questions/66490/drawing-a-tikz-arc-specifying-the-center
%% Draws beach ball
\tikzset{pics/carc/.style args={#1:#2:#3}{code={\draw[pic actions] (#1:#3) arc(#1:#2:#3);}}}



\usepackage{array}
\setlength{\extrarowheight}{+.1cm}
\newdimen\digitwidth
\settowidth\digitwidth{9}
\def\divrule#1#2{
\noalign{\moveright#1\digitwidth
\vbox{\hrule width#2\digitwidth}}}






\DeclareMathOperator{\arccot}{arccot}
\DeclareMathOperator{\arcsec}{arcsec}
\DeclareMathOperator{\arccsc}{arccsc}

















%%This is to help with formatting on future title pages.
\newenvironment{sectionOutcomes}{}{}

\author{Steven Gubkin\and nela Lakos}
\license{Creative Commons 3.0 By-NC}
\begin{document}
\begin{exercise}

Let $f$ be a function.  Consider the function $A(t) = \frac{1}{t} \int_0^t f(x) dx $, which gives the average of $f$ on the interval $[0,t]$, for $t>0$.  

Suppose that $A$ has a local maximum at $t=b$, then $A'(b) = \answer{0}$ (answer with a number).  Now, the formula for $A'(t)$ is
\begin{align*}
A'(t) &= \answer{\frac{f(t)}{t}} - \answer{\frac{1}{t^2}}\int_0^tf(x)d x\\
&= \frac{1}{t}\left(\answer{f(t)}-A(t)\right)
\end{align*}
Now since $A'(b) = \answer{0}$ (answer with a number), we can conclude that $A(b) = \answer{f(b)}$ (answer with an expression in terms of $f$).



Suppose $f(x)  = x(1-x)$.  For what value of $t$ is $A(t)$ maximized?
\begin{hint}
You have to find a critical point of $A$.
This means that we have to solve the equation

\[
A'(x)=0.
\]
That is equivalent to solving 
\[
f(x)=A(x).
\]
\end{hint}
\begin{hint}
let's solve the equation
\[
f(x)=A(x).
\]
\[
 x(1-x)=\frac{1}{x} \int_0^x f(t) dt
\]
\[
 x-x^2=\frac{1}{x} \int_0^x (t-t^2) dt
\]
\[
 x-x^2= \frac{x}{2}-\frac{x^2}{3}
\]
Divide by $x$.
\[
 1-x= \frac{1}{2}-\frac{x}{3}
\]
Solve for $x$.
How do we know that the function $A$ has a maximum at this point?
\end{hint}
\[
t = \answer{\frac{3}{4}}
\]

\end{exercise}
\end{document}