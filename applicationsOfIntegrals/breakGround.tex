\documentclass{ximera}


\graphicspath{
  {./}
  {ximeraTutorial/}
  {basicPhilosophy/}
}

\newcommand{\mooculus}{\textsf{\textbf{MOOC}\textnormal{\textsf{ULUS}}}}

\usepackage{tkz-euclide}\usepackage{tikz}
\usepackage{tikz-cd}
\usetikzlibrary{arrows}
\tikzset{>=stealth,commutative diagrams/.cd,
  arrow style=tikz,diagrams={>=stealth}} %% cool arrow head
\tikzset{shorten <>/.style={ shorten >=#1, shorten <=#1 } } %% allows shorter vectors

\usetikzlibrary{backgrounds} %% for boxes around graphs
\usetikzlibrary{shapes,positioning}  %% Clouds and stars
\usetikzlibrary{matrix} %% for matrix
\usepgfplotslibrary{polar} %% for polar plots
\usepgfplotslibrary{fillbetween} %% to shade area between curves in TikZ
\usetkzobj{all}
\usepackage[makeroom]{cancel} %% for strike outs
%\usepackage{mathtools} %% for pretty underbrace % Breaks Ximera
%\usepackage{multicol}
\usepackage{pgffor} %% required for integral for loops



%% http://tex.stackexchange.com/questions/66490/drawing-a-tikz-arc-specifying-the-center
%% Draws beach ball
\tikzset{pics/carc/.style args={#1:#2:#3}{code={\draw[pic actions] (#1:#3) arc(#1:#2:#3);}}}



\usepackage{array}
\setlength{\extrarowheight}{+.1cm}
\newdimen\digitwidth
\settowidth\digitwidth{9}
\def\divrule#1#2{
\noalign{\moveright#1\digitwidth
\vbox{\hrule width#2\digitwidth}}}






\DeclareMathOperator{\arccot}{arccot}
\DeclareMathOperator{\arcsec}{arcsec}
\DeclareMathOperator{\arccsc}{arccsc}

















%%This is to help with formatting on future title pages.
\newenvironment{sectionOutcomes}{}{}


\outcome{Understand the relationship between position, velocity and acceleration.}


\title[Break-Ground:]{What could it represent?}

\begin{document}
\begin{abstract}
Two young mathematicians discuss whether integrals are defined properly.
\end{abstract}
\maketitle


Check out this dialogue between two calculus students (based on a true
story):

\begin{dialogue}
\item[Devyn] Riley, I like integrals.
\item[Riley] I feel fancy when I make an integral sign.
\item[Devyn] I know! An integral computes the signed area between a curve
  $y=f(x)$ and the $x$-axis. But why \textit{signed} area? Maybe we should
  just compute plain old area.
\item[Riley] Makes sense to me!
\item[Deyvn] Unless\dots maybe there are other applications where
  ``signed'' area makes more sense.
\end{dialogue}

One really great way to think about integrals is that they
``accumulate rates.''

\begin{problem}
  Write down as many examples of ``rates'' and ``accumulated rates'' as you can. For example:
  \begin{quote}
    $5$ miles per hour is a rate, and $5$ miles is then an accumulated rate. 
  \end{quote}
  \begin{freeResponse}
  \end{freeResponse}
\end{problem}

%\input{../leveledQuestions.tex}


\end{document}
