\documentclass{ximera}


\graphicspath{
  {./}
  {ximeraTutorial/}
  {basicPhilosophy/}
}

\newcommand{\mooculus}{\textsf{\textbf{MOOC}\textnormal{\textsf{ULUS}}}}

\usepackage{tkz-euclide}\usepackage{tikz}
\usepackage{tikz-cd}
\usetikzlibrary{arrows}
\tikzset{>=stealth,commutative diagrams/.cd,
  arrow style=tikz,diagrams={>=stealth}} %% cool arrow head
\tikzset{shorten <>/.style={ shorten >=#1, shorten <=#1 } } %% allows shorter vectors

\usetikzlibrary{backgrounds} %% for boxes around graphs
\usetikzlibrary{shapes,positioning}  %% Clouds and stars
\usetikzlibrary{matrix} %% for matrix
\usepgfplotslibrary{polar} %% for polar plots
\usepgfplotslibrary{fillbetween} %% to shade area between curves in TikZ
\usetkzobj{all}
\usepackage[makeroom]{cancel} %% for strike outs
%\usepackage{mathtools} %% for pretty underbrace % Breaks Ximera
%\usepackage{multicol}
\usepackage{pgffor} %% required for integral for loops



%% http://tex.stackexchange.com/questions/66490/drawing-a-tikz-arc-specifying-the-center
%% Draws beach ball
\tikzset{pics/carc/.style args={#1:#2:#3}{code={\draw[pic actions] (#1:#3) arc(#1:#2:#3);}}}



\usepackage{array}
\setlength{\extrarowheight}{+.1cm}
\newdimen\digitwidth
\settowidth\digitwidth{9}
\def\divrule#1#2{
\noalign{\moveright#1\digitwidth
\vbox{\hrule width#2\digitwidth}}}






\DeclareMathOperator{\arccot}{arccot}
\DeclareMathOperator{\arcsec}{arcsec}
\DeclareMathOperator{\arccsc}{arccsc}

















%%This is to help with formatting on future title pages.
\newenvironment{sectionOutcomes}{}{}


\outcome{Understand the relationship between limits and vertical asymptotes.}

\author{Nela Lakos \and Kyle Parsons}

\begin{document}
\begin{exercise}

Select True if the statement is \textbf{always} true; otherwise, select False.

Let $f$ be a one-to-one function and $f^{-1}$ its inverse.  If the point $(2,5)$ lies on the graph of $f$, then the point $(5,2)$ lies on the graph of $f^{-1}$.

\begin{multipleChoice}
\choice[correct]{True}
\choice{False}
\end{multipleChoice}

\begin{feedback}
The graph of $f$ contains all points of the form $\left(a,f(a)\right)$ whereas the graph of $f^{-1}$ contains all points of the form $\left(f(a),f^{-1}(f(a))\right)=(f(a),a)$.
\end{feedback}

\begin{exercise}

\[
\sin^{-1}(\pi) = 0
\]

\begin{multipleChoice}
\choice{True}
\choice[correct]{false}
\end{multipleChoice}

\begin{feedback}
The range of $\sin$ is $\left[-1,1\right]$ and so the domain of $\sin^{-1}$ is $\left[-1,1\right]$.  We see then than $\pi$ is not in the domain of $\sin^{-1}$ so $\sin^{-1}(\pi)$ does not exist.
\end{feedback}

\end{exercise}
\end{exercise}
\end{document}