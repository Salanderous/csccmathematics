\documentclass{ximera}


\graphicspath{
  {./}
  {ximeraTutorial/}
  {basicPhilosophy/}
}

\newcommand{\mooculus}{\textsf{\textbf{MOOC}\textnormal{\textsf{ULUS}}}}

\usepackage{tkz-euclide}\usepackage{tikz}
\usepackage{tikz-cd}
\usetikzlibrary{arrows}
\tikzset{>=stealth,commutative diagrams/.cd,
  arrow style=tikz,diagrams={>=stealth}} %% cool arrow head
\tikzset{shorten <>/.style={ shorten >=#1, shorten <=#1 } } %% allows shorter vectors

\usetikzlibrary{backgrounds} %% for boxes around graphs
\usetikzlibrary{shapes,positioning}  %% Clouds and stars
\usetikzlibrary{matrix} %% for matrix
\usepgfplotslibrary{polar} %% for polar plots
\usepgfplotslibrary{fillbetween} %% to shade area between curves in TikZ
\usetkzobj{all}
\usepackage[makeroom]{cancel} %% for strike outs
%\usepackage{mathtools} %% for pretty underbrace % Breaks Ximera
%\usepackage{multicol}
\usepackage{pgffor} %% required for integral for loops



%% http://tex.stackexchange.com/questions/66490/drawing-a-tikz-arc-specifying-the-center
%% Draws beach ball
\tikzset{pics/carc/.style args={#1:#2:#3}{code={\draw[pic actions] (#1:#3) arc(#1:#2:#3);}}}



\usepackage{array}
\setlength{\extrarowheight}{+.1cm}
\newdimen\digitwidth
\settowidth\digitwidth{9}
\def\divrule#1#2{
\noalign{\moveright#1\digitwidth
\vbox{\hrule width#2\digitwidth}}}






\DeclareMathOperator{\arccot}{arccot}
\DeclareMathOperator{\arcsec}{arcsec}
\DeclareMathOperator{\arccsc}{arccsc}

















%%This is to help with formatting on future title pages.
\newenvironment{sectionOutcomes}{}{}


\author{Jason Miller}
\license{Creative Commons 3.0 By-NC}


\outcome{}


\begin{document}
\begin{exercise}
Determine the length of the curve $y=x^{2}$ on the interval $[0,1/2]$. 

We must set up an integral to determine the length. 

\[
\int_{\answer{0}}^{\answer{\frac{1}{2}}} \answer{ \sqrt{1+4x^{2}}} \d x
\]

\begin{exercise}

In order to determine the integral, we try a trig substitution.
Since we have a $1+4x^{2}$ term, we should use $x=\answer{\frac{1}{2}\tan(\theta)}$. 

This means that $\d x= \answer{ \frac{1}{2}\sec^{2}(\theta)} \d \theta$. 

We have two possible options. We could determine the antiderivative in terms of $\theta$ 
and then convert back to $x$ and then evaluate using our given bounds. 
Another option is that we could transform the $x$ bounds on our original integral to $\theta$ bounds. 

Using our substitution, we can convert our bounds.

When $x=0$, $\theta=\answer{0}$. 

When $x=\frac{1}{2}$, $\theta=\answer{\frac{\pi}{4}}$. 



\begin{exercise}


Now we can express our original definite integral as a trigonometric integral in terms of the variable $\theta$. 

\[
\int_{0}^{\frac{1}{2}} \sqrt{1+4x^{2}} \d x= \int_{\answer{0}}^{\answer{\frac{\pi}{4}}}  \answer{\frac{\sec^{3}(\theta)}{2}}   \d \theta
\]


First we calculate the indefinite integral $\int \sec^{3}(\theta) \d \theta$. Recall that this integral was investigated at the end of the section on trigonometric integrals. 
We derived the following formula using integration by parts:

\[
\int \sec^{3}(\theta) \d \theta=\frac{\sec(\theta)\tan(\theta)}{2} + \frac{ \ln| \sec(\theta) + \tan(\theta) | }{2} + C
\]

\begin{exercise}
Now evaluating our integral in terms of $\theta$ we have that the length of the given curve is given by

\[
\int_{0}^{\frac{1}{2}} \sqrt{1+4x^{2}} \d x= \int_{0}^{\frac{\pi}{4}}  \frac{\sec^{3}(\theta)}{2}   \d \theta =\answer{\frac{\sqrt{2}}{4} + \frac{1}{4}\ln(\sqrt{2}+1) }
\]




\end{exercise}
\end{exercise}

\end{exercise}

\end{exercise}
\end{document}
