\documentclass{ximera}


\graphicspath{
  {./}
  {ximeraTutorial/}
  {basicPhilosophy/}
}

\newcommand{\mooculus}{\textsf{\textbf{MOOC}\textnormal{\textsf{ULUS}}}}

\usepackage{tkz-euclide}\usepackage{tikz}
\usepackage{tikz-cd}
\usetikzlibrary{arrows}
\tikzset{>=stealth,commutative diagrams/.cd,
  arrow style=tikz,diagrams={>=stealth}} %% cool arrow head
\tikzset{shorten <>/.style={ shorten >=#1, shorten <=#1 } } %% allows shorter vectors

\usetikzlibrary{backgrounds} %% for boxes around graphs
\usetikzlibrary{shapes,positioning}  %% Clouds and stars
\usetikzlibrary{matrix} %% for matrix
\usepgfplotslibrary{polar} %% for polar plots
\usepgfplotslibrary{fillbetween} %% to shade area between curves in TikZ
\usetkzobj{all}
\usepackage[makeroom]{cancel} %% for strike outs
%\usepackage{mathtools} %% for pretty underbrace % Breaks Ximera
%\usepackage{multicol}
\usepackage{pgffor} %% required for integral for loops



%% http://tex.stackexchange.com/questions/66490/drawing-a-tikz-arc-specifying-the-center
%% Draws beach ball
\tikzset{pics/carc/.style args={#1:#2:#3}{code={\draw[pic actions] (#1:#3) arc(#1:#2:#3);}}}



\usepackage{array}
\setlength{\extrarowheight}{+.1cm}
\newdimen\digitwidth
\settowidth\digitwidth{9}
\def\divrule#1#2{
\noalign{\moveright#1\digitwidth
\vbox{\hrule width#2\digitwidth}}}






\DeclareMathOperator{\arccot}{arccot}
\DeclareMathOperator{\arcsec}{arcsec}
\DeclareMathOperator{\arccsc}{arccsc}

















%%This is to help with formatting on future title pages.
\newenvironment{sectionOutcomes}{}{}


\author{Jason Miller}
\license{Creative Commons 3.0 By-NC}


\outcome{}


\begin{document}
\begin{exercise}
Use trig substitution to determine the integral.
\[
\int \frac{1}{(1+x^2)^2} dx
\]

First we must determine the appropriate substitution. 

Since we have a $1+x^2$ term, we should use $x=\answer{\tan(\theta)}$. 

This means that $dx= \answer{ \sec^{2}(\theta)} d\theta$. 

\begin{exercise}
Using the trig identity $1+\tan^2(\theta)=\sec^{2}(\theta)$, we can rewrite $1+x^2$ as $\answer{\sec^{2}(\theta)}$. 

Now we can express our original integral as a trigonometric integral in terms of the variable $\theta$. 

\begin{image}
  \begin{tikzpicture}
        \node at (0,0) {
          $\int \frac{1}{(1+x^2)^2} dx   = \int \underbrace{\frac{1}{(\sec^2(\theta))^2}}\left(  \overbrace{ \sec^{2}(\theta) d\theta } \right)$};
        \node at (.4,-.8) {\small{The integrand in}};
        \node at (.4,-1.1) {\small{terms of $\theta$.}};
        
        \node at (2.5,1.2) {\small{This is $dx$ in terms}};
        \node at (2.5,.9) {\small{of $\theta$ and $d\theta$.}};        
      \end{tikzpicture}
  \end{image}

Simplifying the integrand gives

\[
\int \frac{1}{(1+x^2)^2} dx  = \int \cos^2(\theta) d\theta.
\]

The antiderivative in terms of $\theta$ is 

\[
\answer{ \frac{1}{2}\theta + \frac{1}{2}\sin(\theta)\cos(\theta) + C}
\]
(Use $C$ for the constant of integration)

\begin{hint}
The following trig identity will be useful.
\begin{align}
\cos^{2}(\theta)&=\frac{1+\cos(2\theta)}{2} \\
\end{align}
\end{hint}

Now we really want the antiderivative in terms of $x$ so we need to reverse the substitution to find $\frac{1}{2}\theta + \frac{1}{4} \sin(2 \theta) + C$

\begin{exercise}
Using the original substitution $x=\tan(\theta)$, we draw the right triangle below.

    \begin{image}
      \begin{tikzpicture}
        \coordinate (C) at (0,2);
        \coordinate (D) at (4,2);
        \coordinate (E) at (4,4);
        \tkzMarkRightAngle(C,D,E)
        \tkzMarkAngle(D,C,E)
        \draw[decoration={brace,mirror,raise=.2cm},decorate,thin] (0,2)--(4,2);
        \draw[decoration={brace,mirror,raise=.2cm},decorate,thin] (4,2)--(4,4);
        \draw[decoration={brace,raise=.2cm},decorate,thin] (0,2)--(4,4);
        \draw[very thick] (D)--(E)--(C)--cycle;
        \node at (2,2-.7) {$1$}; %% adj
        \node[anchor=west] at (4+.3,3) {$x$}; %% opp 
        \node at (2-.3,3+.5) {$?$}; %% hyp
        \node at (1.2,2.3) {$\theta$};
      \end{tikzpicture}
    \end{image}

Then using the Pythagorean Theorem we find that the
 length of the hypotenuse=$\answer{\sqrt{1+x^2}}$. 

Using the triangle, we now convert each of the expressions in our antiderivative that was in terms of $\theta$ to be in terms of $x$.  Notice that $\sin(2 \theta)$ cannot be found directly from the triangle yet.  Using the identity

\[
\sin(2\theta)=2\sin(\theta)\cos(\theta),
\]

we find that 

\[ \int \frac{1}{ (1+x^2)^2} dx = \frac{1}{2}\theta + \frac{1}{2} \sin(\theta) \cos(\theta) + C.\]  

Using the triangle, note that: 
\begin{align}
\theta&=\answer{\arctan(x)} \\
\sin(\theta)&=\answer{\frac{x}{\sqrt{1+x^{2}}}} \\
 \cos(\theta)&=\answer{\frac{1}{\sqrt{1+x^{2}}}} 
\end{align}

We can now write the antiderivatives of the original function in terms of $x$.

\[
\int \frac{1}{ (1+x^2)^2} dx= \answer{ \frac{1}{2}\arctan(x) + \frac{x}{2(1+x^{2})} + C }
\]
(Use $C$ for the constant of integration)

\end{exercise}

\end{exercise}

\end{exercise}
\end{document}
