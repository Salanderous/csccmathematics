\documentclass{ximera}


\graphicspath{
  {./}
  {ximeraTutorial/}
  {basicPhilosophy/}
}

\newcommand{\mooculus}{\textsf{\textbf{MOOC}\textnormal{\textsf{ULUS}}}}

\usepackage{tkz-euclide}\usepackage{tikz}
\usepackage{tikz-cd}
\usetikzlibrary{arrows}
\tikzset{>=stealth,commutative diagrams/.cd,
  arrow style=tikz,diagrams={>=stealth}} %% cool arrow head
\tikzset{shorten <>/.style={ shorten >=#1, shorten <=#1 } } %% allows shorter vectors

\usetikzlibrary{backgrounds} %% for boxes around graphs
\usetikzlibrary{shapes,positioning}  %% Clouds and stars
\usetikzlibrary{matrix} %% for matrix
\usepgfplotslibrary{polar} %% for polar plots
\usepgfplotslibrary{fillbetween} %% to shade area between curves in TikZ
\usetkzobj{all}
\usepackage[makeroom]{cancel} %% for strike outs
%\usepackage{mathtools} %% for pretty underbrace % Breaks Ximera
%\usepackage{multicol}
\usepackage{pgffor} %% required for integral for loops



%% http://tex.stackexchange.com/questions/66490/drawing-a-tikz-arc-specifying-the-center
%% Draws beach ball
\tikzset{pics/carc/.style args={#1:#2:#3}{code={\draw[pic actions] (#1:#3) arc(#1:#2:#3);}}}



\usepackage{array}
\setlength{\extrarowheight}{+.1cm}
\newdimen\digitwidth
\settowidth\digitwidth{9}
\def\divrule#1#2{
\noalign{\moveright#1\digitwidth
\vbox{\hrule width#2\digitwidth}}}






\DeclareMathOperator{\arccot}{arccot}
\DeclareMathOperator{\arcsec}{arcsec}
\DeclareMathOperator{\arccsc}{arccsc}

















%%This is to help with formatting on future title pages.
\newenvironment{sectionOutcomes}{}{}


\author{Jason Miller}
\license{Creative Commons 3.0 By-NC}


\outcome{}


\begin{document}
\begin{exercise}
Determine the integral using trig substitution. 
\[
\int \frac{1}{\sqrt{1-x^{2}}} dx
\]

We should use the trig substitution $x=\answer{ \sin(\theta)}$. 

Hence $dx=\answer{ \cos(\theta) } d\theta$.

Expressing our integral in terms of $\theta$ gives:

\begin{exercise}

\[
\int \frac{1}{\sqrt{1-x^{2}}} dx=\int \answer{ 1  }  d\theta 
\]


and the antiderivative in terms of $\theta$ is 
\[
\int d\theta = \answer{ \theta + C}
\]
(use $C$ for the constant of integration)

\begin{exercise}
Switching back to the variable $x$, our original integral is:

\[
\int \frac{1}{\sqrt{1-x^{2}}} dx= \answer{ \arcsin(x) + C }
\]
(Use $C$ for the constant of integration)

\begin{feedback}
Note that since:
\[
\frac{d}{dx} \arcsin(x)=\frac{1}{\sqrt{1-x^{2}}}
\]
we could have obtained this result by simply writing down the corresponding antidifferentiation formula.  Still, it does not hurt to emphasize no matter how we choose solve a problem (provided that we do it correctly), we will obtain the same result!

\end{feedback}
\end{exercise}
\end{exercise}
\end{exercise}
\end{document}
