\documentclass{ximera}


\graphicspath{
  {./}
  {ximeraTutorial/}
  {basicPhilosophy/}
}

\newcommand{\mooculus}{\textsf{\textbf{MOOC}\textnormal{\textsf{ULUS}}}}

\usepackage{tkz-euclide}\usepackage{tikz}
\usepackage{tikz-cd}
\usetikzlibrary{arrows}
\tikzset{>=stealth,commutative diagrams/.cd,
  arrow style=tikz,diagrams={>=stealth}} %% cool arrow head
\tikzset{shorten <>/.style={ shorten >=#1, shorten <=#1 } } %% allows shorter vectors

\usetikzlibrary{backgrounds} %% for boxes around graphs
\usetikzlibrary{shapes,positioning}  %% Clouds and stars
\usetikzlibrary{matrix} %% for matrix
\usepgfplotslibrary{polar} %% for polar plots
\usepgfplotslibrary{fillbetween} %% to shade area between curves in TikZ
\usetkzobj{all}
\usepackage[makeroom]{cancel} %% for strike outs
%\usepackage{mathtools} %% for pretty underbrace % Breaks Ximera
%\usepackage{multicol}
\usepackage{pgffor} %% required for integral for loops



%% http://tex.stackexchange.com/questions/66490/drawing-a-tikz-arc-specifying-the-center
%% Draws beach ball
\tikzset{pics/carc/.style args={#1:#2:#3}{code={\draw[pic actions] (#1:#3) arc(#1:#2:#3);}}}



\usepackage{array}
\setlength{\extrarowheight}{+.1cm}
\newdimen\digitwidth
\settowidth\digitwidth{9}
\def\divrule#1#2{
\noalign{\moveright#1\digitwidth
\vbox{\hrule width#2\digitwidth}}}






\DeclareMathOperator{\arccot}{arccot}
\DeclareMathOperator{\arcsec}{arcsec}
\DeclareMathOperator{\arccsc}{arccsc}

















%%This is to help with formatting on future title pages.
\newenvironment{sectionOutcomes}{}{}


\author{Jason Miller}
\license{Creative Commons 3.0 By-NC}


\outcome{}


\begin{document}
\begin{exercise}
Determine the integral.
\[
\int x\arctan(x) dx
\]


The integrand is a product where one factor is easy to integrate and the other factor is easy to differentiate. Since nothing else seems to be helpful, we try integration by parts. 

Let $u=\answer{ \arctan(x)}$ and $dv= \answer{  x } dx$. 

Then 
\[
  du= \answer{ \frac{1}{1+x^{2}}} \text{ and }  v=\answer{ \frac{x^{2}}{2} }
\]

\begin{exercise}

Using integration by parts, our original integral becomes

\[
\int x\arctan(x) dx=\answer{\frac{x^{2}\arctan(x)}{2}} - \int \answer{\frac{x^{2}}{2(1+x^{2})}} dx
\]

\begin{exercise}

The second integral $\int \frac{x^{2}}{2(1+x^{2})} dx$ can be done with a trig substitution. 

Let $x=\answer{ \tan(\theta)}$. Then $dx=\answer{ \sec^{2}(\theta)} d\theta$.

The integral in terms of $\theta$ is

\[
\int \frac{x^{2}}{2(1+x^{2})} dx =\int \answer{ \frac{\tan^{2}(\theta)}{2}} d\theta
\]

\begin{exercise}
The antiderivative in terms of $\theta$ is

\[
\int \frac{\tan^{2}(\theta)}{2} d\theta=\answer{\frac{1}{2}\left( \tan(\theta) - \theta \right) + C }
\]
(Use $C$ for the constant of integration)

\begin{exercise}
Going back to our original variable $x$ we have 

\[
\int \frac{x^{2}}{2(1+x^{2})} dx=\answer{\frac{1}{2}\left( x-\arctan(x) \right) + C }
\]
(Use $C$ for the constant of integration)

\begin{exercise}

Now we can combine this with the result we obtained from integration by parts to get


\[
\int x\arctan(x) dx=\answer{ \frac{x^{2}\arctan(x)}{2}-\frac{x}{2}+\frac{\arctan(x)}{2} + C}
\]
(Use $C$ for the constant of integration)



\end{exercise}
\end{exercise}

\end{exercise}
\end{exercise}
\end{exercise}
\end{exercise}
\end{document}
