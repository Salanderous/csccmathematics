\documentclass{ximera}


\graphicspath{
  {./}
  {ximeraTutorial/}
  {basicPhilosophy/}
}

\newcommand{\mooculus}{\textsf{\textbf{MOOC}\textnormal{\textsf{ULUS}}}}

\usepackage{tkz-euclide}\usepackage{tikz}
\usepackage{tikz-cd}
\usetikzlibrary{arrows}
\tikzset{>=stealth,commutative diagrams/.cd,
  arrow style=tikz,diagrams={>=stealth}} %% cool arrow head
\tikzset{shorten <>/.style={ shorten >=#1, shorten <=#1 } } %% allows shorter vectors

\usetikzlibrary{backgrounds} %% for boxes around graphs
\usetikzlibrary{shapes,positioning}  %% Clouds and stars
\usetikzlibrary{matrix} %% for matrix
\usepgfplotslibrary{polar} %% for polar plots
\usepgfplotslibrary{fillbetween} %% to shade area between curves in TikZ
\usetkzobj{all}
\usepackage[makeroom]{cancel} %% for strike outs
%\usepackage{mathtools} %% for pretty underbrace % Breaks Ximera
%\usepackage{multicol}
\usepackage{pgffor} %% required for integral for loops



%% http://tex.stackexchange.com/questions/66490/drawing-a-tikz-arc-specifying-the-center
%% Draws beach ball
\tikzset{pics/carc/.style args={#1:#2:#3}{code={\draw[pic actions] (#1:#3) arc(#1:#2:#3);}}}



\usepackage{array}
\setlength{\extrarowheight}{+.1cm}
\newdimen\digitwidth
\settowidth\digitwidth{9}
\def\divrule#1#2{
\noalign{\moveright#1\digitwidth
\vbox{\hrule width#2\digitwidth}}}






\DeclareMathOperator{\arccot}{arccot}
\DeclareMathOperator{\arcsec}{arcsec}
\DeclareMathOperator{\arccsc}{arccsc}

















%%This is to help with formatting on future title pages.
\newenvironment{sectionOutcomes}{}{}

\author{Jim Talamo}

\outcome{Define dot products two ways.}
\outcome{Explore the utility of both the magnitude-angle and component formulation of the dot product.}
\outcome{Use dot products to compute the angle between vectors.}
\outcome{Use the dot product to define orthogonality.}

\title[Dig-In:]{The Dot Product}

\begin{document}
\begin{abstract}
The dot product is an important operation between vectors that captures geometric information. 
\end{abstract}
\maketitle


\section{The dot product}

We have already seen how to add vectors and how to multiply vectors by
scalars.  As it turns out, there is not a single useful way to define ``multiplication'' of vectors, but there are several types of products defined for two vectors that have intrinsic meaning.  One such example is the \emph{dot product}, which we motivate using the example below.

\begin{model}
Imagine the following scenario.

A person is trying to drag a table from one side of a room to the other across a carpeted floor.  While moving, the table is dragged only, not lifted.  In order to cause the table to move, the person applies a force, which is directed along the person's arms. We can consider two scenarios - one in which the force applied is mostly in the direction that the table moves, and one in which only a small part of the force is in the direction of motion.  
 \begin{image}
    \begin{tikzpicture}
        \begin{axis}[ymax=1.5,xmax=2.8, ymin=-.8, xmin=-2.8,
            unit vector ratio*=1 1 1,
            axis lines=none
          ]
                      \addplot[very thick,penColor] plot coordinates {(0,0)(0,.5)(2,.5)(2,0)(0,0)};
          \addplot[very thick,penColor2,->] plot coordinates {(0,.5) (-2,1)};
          \addplot[very thick,penColor5,->] plot coordinates {(0,.5) (-2,.5)};
          \node[above] at (axis cs:-.8, .8) [penColor2] {force};
                    \node[left] at (axis cs:-.5, .3) [penColor5] {direction};
                    \node[left] at (axis cs:1.4, .25) [penColor] {Table};
         \node[left] at (axis cs:2.8, -.5) [black] {\footnotesize Most of the force is in the direction of motion.};
                      \addplot[very thin,penColor] plot coordinates {(0,-1)};
        \end{axis}
    \end{tikzpicture}
  \end{image}

 \begin{image}
    \begin{tikzpicture}
        \begin{axis}[ymax=1.8,xmax=2.8, ymin=-.8, xmin=-2.8,
            unit vector ratio*=1 1 1,
            axis lines=none
          ]
                      \addplot[very thick,penColor] plot coordinates {(0,0)(0,.5)(2,.5)(2,0)(0,0)};
          \addplot[very thick,penColor2,->] plot coordinates {(0,.5) (-1.5,1.8)};
          \addplot[very thick,penColor5,->] plot coordinates {(0,.5) (-2,.5)};
          \node[above] at (axis cs:-.5, 1.2) [penColor2] {force};
                    \node[left] at (axis cs:-.5, .3) [penColor5] {direction};
                    \node[left] at (axis cs:1.4, .25) [penColor] {Table};
         \node[left] at (axis cs:2.8, -.5) [black] {\footnotesize Some of the force is in the direction of motion.};
                      \addplot[very thin,penColor] plot coordinates {(0,-1)};
        \end{axis}
    \end{tikzpicture}
  \end{image}

Since there is frictional force to overcome in order to move the table, work is done when the table is moved.  The person trying to move the table will notice that much more force is necessary to apply in the scenario when less of the applied force is in the direction of motion.  In fact, a result from physics ensures that the work done is given by

\[
\textrm{Work} = \left<\textrm{ component of force in the direction of motion }\right> \cdot \left<\textrm{ distance }\right>.
\]

By denoting the force by $\overset{\boldsymbol{\rightharpoonup}}{\mathbf{F}}$ and the displacement vector by $\overset{\boldsymbol{\rightharpoonup}}{\mathbf{d}}$, and letting $\theta$ be the angle between them, we note that the component of $\overset{\boldsymbol{\rightharpoonup}}{\mathbf{F}}$ in the direction of motion is $|\overset{\boldsymbol{\rightharpoonup}}{\mathbf{F}}|\cos(\theta)$, so 

\[
W = |\overset{\boldsymbol{\rightharpoonup}}{\mathbf{F}}| \cos(\theta) \cdot |\overset{\boldsymbol{\rightharpoonup}}{\mathbf{d}}| = |\overset{\boldsymbol{\rightharpoonup}}{\mathbf{F}}||\overset{\boldsymbol{\rightharpoonup}}{\mathbf{d}}|\cos(\theta)
\]

Let's now return to our original scenarios.  The same amount of work in both scenarios is done when dragging the table across the room and in both scenarios, the angle lies between $0$ and $90^{\circ}$.  Letting $\theta_1$ be the angle in the first scenario and $\theta_2$ be the angle in the second one, note $\theta_1 <\theta_2$.  Letting $F_1$ and $F_2$ be the force required to apply in each scenario, we now have 

\[
|\overset{\boldsymbol{\rightharpoonup}}{\mathbf{F_1}}|\cancel{|\overset{\boldsymbol{\rightharpoonup}}{\mathbf{d}}|}\cos(\theta_1) = |\overset{\boldsymbol{\rightharpoonup}}{\mathbf{F_2}}|\cancel{|\overset{\boldsymbol{\rightharpoonup}}{\mathbf{d}}|}\cos(\theta_2)
\]
Since $\cos(\theta)$ is \wordChoice{\choice{increasing}\choice[correct]{decreasing}} for $0\leq \theta\leq \frac{\pi}{2}$, in order for the above equality to hold,  we must have $|\overset{\boldsymbol{\rightharpoonup}}{\mathbf{F_1}}|$ \wordChoice{\choice[correct]{$<$}\choice{$=$}\choice{$>$}} $|\overset{\boldsymbol{\rightharpoonup}}{\mathbf{F_2}}|$.
\end{model}



\section{Two definitions of the dot product}
The above scenario illustrates a quantity that is fundamentally important in physics, but it is useful in other instances as well.  We can extract the mathematical essence of the above example as follows.  

\begin{quote}
Given two vectors $\overset{\boldsymbol{\rightharpoonup}}{\mathbf{u}}$ and $\overset{\boldsymbol{\rightharpoonup}}{\mathbf{v}}$, the quantity $|\overset{\boldsymbol{\rightharpoonup}}{\mathbf{u}}||\overset{\boldsymbol{\rightharpoonup}}{\mathbf{v}}|\cos(\theta)$ is important.
\end{quote}

Since this quantity is important, we dignify it with a definition.

\begin{definition} (Magnitude-Angle Formulation of the Dot Product)

Let $\overset{\boldsymbol{\rightharpoonup}}{\mathbf{u}}$ and $\overset{\boldsymbol{\rightharpoonup}}{\mathbf{v}}$ be nonzero vectors and let $\theta$ be the angle between them.  We define the \textbf{dot product} of $\overset{\boldsymbol{\rightharpoonup}}{\mathbf{u}}$ and $\overset{\boldsymbol{\rightharpoonup}}{\mathbf{v}}$, denoted by $\overset{\boldsymbol{\rightharpoonup}}{\mathbf{u}} \bullet \overset{\boldsymbol{\rightharpoonup}}{\mathbf{v}}$ by

\[
\overset{\boldsymbol{\rightharpoonup}}{\mathbf{u}} \bullet \overset{\boldsymbol{\rightharpoonup}}{\mathbf{v}} = |\overset{\boldsymbol{\rightharpoonup}}{\mathbf{u}}||\overset{\boldsymbol{\rightharpoonup}}{\mathbf{v}}|\cos(\theta).
\]

In the instance when one of the vectors is $\overset{\boldsymbol{\rightharpoonup}}{\mathbf{0}}$, we define $\overset{\boldsymbol{\rightharpoonup}}{\mathbf{u}} \bullet \overset{\boldsymbol{\rightharpoonup}}{\mathbf{v}} = 0$.
\end{definition}




Given the magnitude and angles made by two vectors in $\mathbb{R}^2$, it is straightforward to compute, but we want to work vectors in higher dimensions, and we therefore want to find a quick way to compute this quantity using the components of $\overset{\boldsymbol{\rightharpoonup}}{\mathbf{u}}$ and $\overset{\boldsymbol{\rightharpoonup}}{\mathbf{v}}$.  Thankfully, there's a good way to do this.

\begin{theorem} (Component Formulation of the Dot Product)

Suppose that $\overset{\boldsymbol{\rightharpoonup}}{\mathbf{u}} = \left< u_1,u_2,\ldots,u_n \right>$ and $\overset{\boldsymbol{\rightharpoonup}}{\mathbf{v}} = \left< v_1,v_2,\ldots,v_n \right>$ are vectors in $\mathbb{R}^n$.  We define the \textbf{dot product} of $\overset{\boldsymbol{\rightharpoonup}}{\mathbf{u}}$ and $\overset{\boldsymbol{\rightharpoonup}}{\mathbf{v}}$, denoted by $\overset{\boldsymbol{\rightharpoonup}}{\mathbf{u}} \bullet \overset{\boldsymbol{\rightharpoonup}}{\mathbf{v}}$ by

\[
\overset{\boldsymbol{\rightharpoonup}}{\mathbf{u}} \bullet \overset{\boldsymbol{\rightharpoonup}}{\mathbf{v}} = u_1v_1 +u_2v_2+ \ldots + u_nv_n = \sum_{k=1}^n u_kv_k.
\]

That is, to compute the dot product, we multiply the corresponding components together and add them, and we do this for as many components as we have.
\end{theorem}

While this may seem intimidating at first, we usually have in mind that $n=2$ or $3$, and we can unpack the formula in these cases.  

\begin{itemize}
\item In $\mathbb{R}^2$, we have $\overset{\boldsymbol{\rightharpoonup}}{\mathbf{u}} \bullet \overset{\boldsymbol{\rightharpoonup}}{\mathbf{v}} = u_1v_1+u_2v_2$.
\item In $\mathbb{R}^3$, we have $\overset{\boldsymbol{\rightharpoonup}}{\mathbf{u}} \bullet\overset{\boldsymbol{\rightharpoonup}}{\mathbf{v}} = u_1v_1+u_2v_2+u_3v_3$.
\end{itemize}

Some texts start with the above theorem as the definition of the dot product, and show that our definition can be derived from it.  What is really important is that we have two equivalent ways to express the dot product.  Both can be useful, as we will see in many examples to follow.

\begin{remark}
An interesting observation is that the dot product gives information about how ``aligned'' two vectors are.  Intuitively, vectors that are more closely aligned with each other have an interior angle $\theta$ closer to $0$ than two vectors that are not aligned.  This means that $\cos(\theta)$ will be closer to $1$. From the first definition, this means that the more aligned $\overset{\boldsymbol{\rightharpoonup}}{\mathbf{u}}$ and $\overset{\boldsymbol{\rightharpoonup}}{\mathbf{v}}$ are, the closer the quantity $\overset{\boldsymbol{\rightharpoonup}}{\mathbf{u}} \bullet \overset{\boldsymbol{\rightharpoonup}}{\mathbf{v}}$ will be to $|\overset{\boldsymbol{\rightharpoonup}}{\mathbf{u}}||\overset{\boldsymbol{\rightharpoonup}}{\mathbf{v}}|$.
\end{remark}




It might (and likely \emph{should}) be entirely unclear at this point why the above definition and theorem are consistent with each other.  The appendix to this section establishes this in more detail, but here's an example that demonstrates their equivalence in the context of a specific pair of vectors.

\begin{example}
Suppose that $\overset{\boldsymbol{\rightharpoonup}}{\mathbf{u}}$ and $\overset{\boldsymbol{\rightharpoonup}}{\mathbf{v}}$ are two dimensional vectors.  $\overset{\boldsymbol{\rightharpoonup}}{\mathbf{u}}$ has magnitude $1$ and makes an angle of $\frac{\pi}{6}$ with the positive $x$-axis, and  $overset{\boldsymbol{\rightharpoonup}}{\mathbf{v}}$ has magnitude $2$ and makes an angle of $\frac{\pi}{3}$ with the positive $x$-axis. The vectors are shown below.

\begin{image}
\begin{tikzpicture}

\begin{axis}
  [
  domain=0:6, ymax=1.9,xmax=1.9, ymin=-.4, xmin=-.4,
  axis lines=center, xlabel=$x$, ylabel=$y$,
  xtick={.866,1.1},
  xticklabels={$u_1$,$v_1$},
  ytick={.5,1.732},
  yticklabels={$u_2$,$v_2$},
  every axis y label/.style={at=(current axis.above origin),anchor=south},
  every axis x label/.style={at=(current axis.right of origin),anchor=west},
  axis on top,
  typeset ticklabels with strut,
  ]

          \addplot[very thick,penColor,->] plot coordinates {(0,0) (.866,.5)};
          \addplot[very thick,penColor2,->] plot coordinates {(0,0) (1.1,1.732)}; %x-comp exaggerated to produce a better image
  
  \node at (axis cs:.533,.15) [penColor] {$\overset{\boldsymbol{\rightharpoonup}}{\mathbf{u}}$};
  \node at (axis cs:.433,1.1) [penColor2] {$overset{\boldsymbol{\rightharpoonup}}{\mathbf{v}}$};
  \node at (axis cs:.3,.3) [penColor] {$\theta$};
\end{axis}

\end{tikzpicture}
\end{image}



\end{example}
































\end{document}
