\documentclass{ximera}


\graphicspath{
  {./}
  {ximeraTutorial/}
  {basicPhilosophy/}
}

\newcommand{\mooculus}{\textsf{\textbf{MOOC}\textnormal{\textsf{ULUS}}}}

\usepackage{tkz-euclide}\usepackage{tikz}
\usepackage{tikz-cd}
\usetikzlibrary{arrows}
\tikzset{>=stealth,commutative diagrams/.cd,
  arrow style=tikz,diagrams={>=stealth}} %% cool arrow head
\tikzset{shorten <>/.style={ shorten >=#1, shorten <=#1 } } %% allows shorter vectors

\usetikzlibrary{backgrounds} %% for boxes around graphs
\usetikzlibrary{shapes,positioning}  %% Clouds and stars
\usetikzlibrary{matrix} %% for matrix
\usepgfplotslibrary{polar} %% for polar plots
\usepgfplotslibrary{fillbetween} %% to shade area between curves in TikZ
\usetkzobj{all}
\usepackage[makeroom]{cancel} %% for strike outs
%\usepackage{mathtools} %% for pretty underbrace % Breaks Ximera
%\usepackage{multicol}
\usepackage{pgffor} %% required for integral for loops



%% http://tex.stackexchange.com/questions/66490/drawing-a-tikz-arc-specifying-the-center
%% Draws beach ball
\tikzset{pics/carc/.style args={#1:#2:#3}{code={\draw[pic actions] (#1:#3) arc(#1:#2:#3);}}}



\usepackage{array}
\setlength{\extrarowheight}{+.1cm}
\newdimen\digitwidth
\settowidth\digitwidth{9}
\def\divrule#1#2{
\noalign{\moveright#1\digitwidth
\vbox{\hrule width#2\digitwidth}}}






\DeclareMathOperator{\arccot}{arccot}
\DeclareMathOperator{\arcsec}{arcsec}
\DeclareMathOperator{\arccsc}{arccsc}

















%%This is to help with formatting on future title pages.
\newenvironment{sectionOutcomes}{}{}

%\pgfplotsset{compat=1.13}  
\author{Jim Talamo}
\license{Creative Commons 3.0 By-bC}
\begin{document}
\begin{exercise}
Consider the power series $f(x) = \sum_{k=0}^{\infty}
\frac{2^k}{5k-1}(x-1)^{k}$.

We want to determine the radius and interval of convergence for this
power series.

First, we use the Ratio Test to determine the radius of convergence.

To do this, we'll think of the power series as a sum of functions of
$x$ by writing:

\[
\sum_{k=0}^{\infty} \frac{2^k}{5k-1}(x-1)^{k} = \sum_{k=0}^{\infty} b_k(x)
\]

We need to determine the limit $L(x) = \lim_{k \to \infty} \left| \frac{b_{k+1}(x)}{b_k(x)}\right|$, where we have explicitly indicated here that this limit likely depends on the $x$-value we choose. 

We calculate $b_{k+1}(x)=\answer{\frac{2^{k+1}}{5k+4}(x-1)^{k+1}}$ and $b_k(x)=\answer{\frac{2^k}{5k-1}(x-1)^k}$. 

\begin{exercise}

Simplifying the ratio $\left|\frac{b_{k+1}}{b_k}\right|$ gives us $\left|\frac{b_{k+1}}{b_k}\right|=\answer{\frac{2(5k-1)}{5k+4}} |x-1|$.


\begin{exercise}

Since the $|x-1|$ term does not depend on $k$, we can factor it out of the limit and then calculate:

\[
L(x) = |x-1| \lim_{k \to \infty}   \frac{2(5k-1)}{5k+4}  =\answer{2} |x-1|
\]

\begin{exercise}

Recall the Ratio Test tells us that a series $\sum^{\infty}_{k=1} b_k$ converges if $L <1$ where $L=\lim_{k \to \infty}\left| \frac{b_{k+1}}{b_k}\right|$. 

Thus, in order to determine the set of $x$ for which our power series converges, we need to determine those $x$-values that satisfy the inequality $2|x-1| <1$. 

Rewriting this inequality we obtain $|x-1|< \answer{\frac{1}{2}}$. 

Thus, the  power series $\sum_{k=0}^{\infty} \frac{2^k}{5k-1}(x-1)^{k}$ has radius of convergence $\answer{\frac{1}{2}}$.

\begin{exercise}

Now, recall that we can interpret an inequality of the form:

\[
|x-c|<R
\]

as the set of all points that are at most $R$ units from $x=c$.  Here, $R=\answer{\frac{1}{2}}$ and $c=1$.

\begin{image}
\begin{tikzpicture}

\begin{axis}
	[
	domain=-6:3,
	axis x line = middle,
	axis line style=-,
	axis y line = none,
	xmin=-6.5,
	xmax=3.5,
	ymin=-1,
	ymax=1,
	xtick={-1.5},
	xticklabels={},
	]
	
	\draw[very thick,penColor] (-3,0) -- (0,0);
	\draw[->,very thick,penColor2] (-3,0) -- (-6.5,0);
	\draw[->,very thick,penColor2] (0,0) -- (3.5,0);
%Vertical LInes%%%
	\draw[->,thick,black,dashed] (-3,-.5) -- (-3,.3);
	\draw[->,thick,black,dashed] (0,-.5) -- (0,.3);
	
%arrows and ROC
	\draw[->,thick,penColor4] (-1.4,.15) -- (0,.15);
	\draw[->,thick,penColor4] (-1.6,.15) -- (-3,.15);
      	\node[penColor4] at (-.75,.22) {\small{$1/2$}};
        \node[penColor4] at (-2.3,.22) {\small{$1/2$}};
	
	\node at (-3,0) [scale=.5,shape=circle, fill=white, draw=penColor] {};
	%\node at (-1.5,0) [scale=.5,shape=circle, fill=penColor] {};
	\node at (-1.5,-.05) [below, penColor] {$1$};
	\node at (0,0) [scale=.5,shape=circle, fill=white, draw=penColor] {};
	\addplot[draw=none] coordinates {(-4.5,0)};
	\node at (-1.5,0) [scale=.5,shape=circle, fill=black, draw=penColor] {};
	\addplot[draw=none] coordinates {(-4.5,0)};


%	\node[excl] at (1.5,0) [scale=.5,shape=circle, fill=penColor2] {};

%%BRACES%%%%%%%
	\draw [penColor2,thick,decoration={brace,mirror,raise=2em},decorate] 
        (axis cs:-6.5,0) --
        (axis cs:-3,0);
    \draw [penColor,thick,decoration={brace,mirror,raise=2em},decorate] 
        (axis cs:-3,0) --
        (axis cs:0,0);
    \draw [penColor2,thick,decoration={brace,mirror,raise=2em},decorate] 
        (axis cs:0,0) --
        (axis cs:3.5,0);
        
 %%%text%%%
    \node[penColor2] at (-5,-.4) {diverges};
    \node[penColor] at (-1.5,-.41) {converges};
    \node[penColor2] at (2,-.4) {diverges};

    \node[black] at (-1.5,.5) {We need further analysis to check };
        \node[black] at (-1.5,.4) {for convergence at the endpoints.};
\end{axis}

\end{tikzpicture}
\end{image}

The lefthand endpoint of the interval of convergence is $x=\answer{\frac{1}{2}}$ and the righthand endpoint is $x= \answer{\frac{3}{2}}$.  The open interval of convergence is thus $\left(\answer{\frac{1}{2}},\answer{\frac{3}{2}}\right)$.

\begin{hint}
The lefthand endpoint will be $c-R$ ($R$ units left of $x=c$) while the righthand one will be $c+R$ ($R$ units right of $x=c$) 
\end{hint}

\end{exercise}
\end{exercise}
\end{exercise}
\end{exercise}
\end{exercise}
\end{document}
