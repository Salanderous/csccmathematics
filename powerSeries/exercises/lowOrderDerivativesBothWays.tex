\documentclass{ximera}


\graphicspath{
  {./}
  {ximeraTutorial/}
  {basicPhilosophy/}
}

\newcommand{\mooculus}{\textsf{\textbf{MOOC}\textnormal{\textsf{ULUS}}}}

\usepackage{tkz-euclide}\usepackage{tikz}
\usepackage{tikz-cd}
\usetikzlibrary{arrows}
\tikzset{>=stealth,commutative diagrams/.cd,
  arrow style=tikz,diagrams={>=stealth}} %% cool arrow head
\tikzset{shorten <>/.style={ shorten >=#1, shorten <=#1 } } %% allows shorter vectors

\usetikzlibrary{backgrounds} %% for boxes around graphs
\usetikzlibrary{shapes,positioning}  %% Clouds and stars
\usetikzlibrary{matrix} %% for matrix
\usepgfplotslibrary{polar} %% for polar plots
\usepgfplotslibrary{fillbetween} %% to shade area between curves in TikZ
\usetkzobj{all}
\usepackage[makeroom]{cancel} %% for strike outs
%\usepackage{mathtools} %% for pretty underbrace % Breaks Ximera
%\usepackage{multicol}
\usepackage{pgffor} %% required for integral for loops



%% http://tex.stackexchange.com/questions/66490/drawing-a-tikz-arc-specifying-the-center
%% Draws beach ball
\tikzset{pics/carc/.style args={#1:#2:#3}{code={\draw[pic actions] (#1:#3) arc(#1:#2:#3);}}}



\usepackage{array}
\setlength{\extrarowheight}{+.1cm}
\newdimen\digitwidth
\settowidth\digitwidth{9}
\def\divrule#1#2{
\noalign{\moveright#1\digitwidth
\vbox{\hrule width#2\digitwidth}}}






\DeclareMathOperator{\arccot}{arccot}
\DeclareMathOperator{\arcsec}{arcsec}
\DeclareMathOperator{\arccsc}{arccsc}

















%%This is to help with formatting on future title pages.
\newenvironment{sectionOutcomes}{}{}


\author{Jim Talamo}
\license{Creative Commons 3.0 By-bC}


\outcome{}


\begin{document}
\begin{exercise}
This exercises demonstrates two ways to find $f''(0)$ for the function $f(x) = \sum_{k=0}^{\infty} \frac{3^{k+1}}{k+2}x^k$.  

One way to do this is to write out a few terms in the function that the series represents, then explicitly compute the derivative in question.  

First, note that:

\[
f(x) = \answer{\frac{3}{2}}+\answer{3}x+\answer{\frac{27}{4}}x^2+\answer{\frac{81}{5}}x^3+\ldots
\]

\begin{exercise}
We can now compute the sum of the first several terms in the series $f''(x)$ represents:

\begin{align*}
f(x) &= \frac{3}{2}+3x+\frac{27}{4}x^2+\frac{81}{5}x^3+\ldots \\
f'(x) &= \answer{3}+\answer{\frac{27}{2}}x+\answer{\frac{243}{5}}x^2+\ldots \\
f''(x) &= \answer{\frac{27}{2}}+\answer{\frac{486}{5}}x+\ldots \\
\end{align*}

Thus, $f''(0) = \answer{\frac{27}{2}}$.
\end{exercise}

\begin{exercise}
Another way to do this is to use the relationship between the coefficients of the power series and the derivatives of the function it represents.

\[
\textrm{If } f(x) = \sum_{k=0}^{\infty} a_k(x-c)^k, \textrm{ then: } a_n = \frac{f^{(n)}(c)}{n!}
\]

Here, $c=\answer{0}$.  In order to find $f''(0)$, we should use $n=\answer{2}$.

By definition, $a_2$ is:

\begin{multipleChoice}
\choice{Always the coefficient obtained by plugging in $k=2$.}
\choice[correct]{The coefficient in front of $(x-c)^2$.}
\end{multipleChoice}

In this case, we find $x^k = x^2$ when $k=\answer{2}$, so:

\[
a_2 =  \frac{3^{k+1}}{k+2} \bigg|_{k=2} = \answer{\frac{27}{4}}
\]

\begin{exercise}
We now use the formula $a_n = \frac{f^{(n)}(c)}{n!}$ to find:

\begin{align*}
a_2 &= \frac{f''(0)}{2!} \\
\answer{\frac{27}{4}} &= \frac{f''(0)}{\answer{2}}
\end{align*}


Thus, $f''(0) = \frac{27}{2}$.
\end{exercise}
\end{exercise}

As food for thought, which method is easier?  Which method would be preferable is we instead had to compute $f^{(2018)}(0)$?
You always have the option of using either method discussed here, but it is often easier to compute lower order derivatives by simply writing out a few terms in the series for $f(x)$ then explicitly differentiating them!

\end{exercise}
\end{document}
