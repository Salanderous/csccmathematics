\documentclass{ximera}


\graphicspath{
  {./}
  {ximeraTutorial/}
  {basicPhilosophy/}
}

\newcommand{\mooculus}{\textsf{\textbf{MOOC}\textnormal{\textsf{ULUS}}}}

\usepackage{tkz-euclide}\usepackage{tikz}
\usepackage{tikz-cd}
\usetikzlibrary{arrows}
\tikzset{>=stealth,commutative diagrams/.cd,
  arrow style=tikz,diagrams={>=stealth}} %% cool arrow head
\tikzset{shorten <>/.style={ shorten >=#1, shorten <=#1 } } %% allows shorter vectors

\usetikzlibrary{backgrounds} %% for boxes around graphs
\usetikzlibrary{shapes,positioning}  %% Clouds and stars
\usetikzlibrary{matrix} %% for matrix
\usepgfplotslibrary{polar} %% for polar plots
\usepgfplotslibrary{fillbetween} %% to shade area between curves in TikZ
\usetkzobj{all}
\usepackage[makeroom]{cancel} %% for strike outs
%\usepackage{mathtools} %% for pretty underbrace % Breaks Ximera
%\usepackage{multicol}
\usepackage{pgffor} %% required for integral for loops



%% http://tex.stackexchange.com/questions/66490/drawing-a-tikz-arc-specifying-the-center
%% Draws beach ball
\tikzset{pics/carc/.style args={#1:#2:#3}{code={\draw[pic actions] (#1:#3) arc(#1:#2:#3);}}}



\usepackage{array}
\setlength{\extrarowheight}{+.1cm}
\newdimen\digitwidth
\settowidth\digitwidth{9}
\def\divrule#1#2{
\noalign{\moveright#1\digitwidth
\vbox{\hrule width#2\digitwidth}}}






\DeclareMathOperator{\arccot}{arccot}
\DeclareMathOperator{\arcsec}{arcsec}
\DeclareMathOperator{\arccsc}{arccsc}

















%%This is to help with formatting on future title pages.
\newenvironment{sectionOutcomes}{}{}


\author{Jim Talamo}
\license{Creative Commons 3.0 By-bC}


\outcome{}


\begin{document}
\begin{exercise}
Consider the power series $f(x) = \sum_{k=0}^{\infty} \frac{2^k}{k!}x^k$.

We want to determine the radius and interval of convergence for this power series. 

The radius of convergence is $\answer{\infty}$.
(Type a number or use $\infty$ if appropriate)

\begin{hint}
First, we use the Ratio Test to determine the radius of convergence. 

To do this, we'll think of the power series as a sum of functions of $x$ by writing: 

\[
\sum_{k=0}^{\infty} \frac{2^k}{k!}x^{k} = \sum_{k=0}^{\infty} b_k(x)
\]

We need to determine the limit $L(x) = \lim_{k \to \infty} \left| \frac{b_{k+1}(x)}{b_k(x)}\right|$, where we have explicitly indicated here that this limit likely depends on the $x$-value we choose. 

We calculate $b_{k+1}(x)=\answer{\frac{2^{k+1}}{(k+1)!}x^{k+1}}$ and $b_k(x)=\answer{ \frac{2^k}{k!}x^k}$. 

\begin{question}

Simplifying the ratio $\left|\frac{b_{k+1}}{b_k}\right|$ gives us $\left|\frac{b_{k+1}}{b_k}\right|=\left|\answer{\frac{2}{k+1} x} \right|$.


\begin{question}

Since the $2x$ factor does not depend on $k$, we can factor it out of the limit and then calculate:

\[
L(x) = 2|x| \lim_{k \to \infty}   \frac{1}{k+1}  =\answer{ 0}
\]

\begin{question}

Recall the Ratio Test tells us that a series $\sum^{\infty}_{k=1} b_k$ converges if $L <1$ where $L=\lim_{k \to \infty}\left| \frac{b_{k+1}}{b_k}\right|$. 

Thus, in order to determine the set of $x$ for which our power series converges, we need to determine those $x$-values that satisfy the inequality $L(x) <1$. Here:

\begin{multipleChoice}
\choice{$L(x)$ is not less than one for any real $x$-values.}
\choice[correct]{$L(x)=0$, so $L(x)<1$ for all real $x$-values.}
\end{multipleChoice}

Thus, the series converges for all $x$-values, and the radius of convergence is infinite.
 
 \end{question}
 \end{question}
 \end{question}
\end{hint}

\begin{exercise}
The interval of convergence is:
\begin{multipleChoice}
\choice{A finite interval}
\choice[correct]{$(-\infty,\infty)$}
\end{multipleChoice}

\end{exercise}
\end{exercise}
\end{document}
