\documentclass{ximera}


\graphicspath{
  {./}
  {ximeraTutorial/}
  {basicPhilosophy/}
}

\newcommand{\mooculus}{\textsf{\textbf{MOOC}\textnormal{\textsf{ULUS}}}}

\usepackage{tkz-euclide}\usepackage{tikz}
\usepackage{tikz-cd}
\usetikzlibrary{arrows}
\tikzset{>=stealth,commutative diagrams/.cd,
  arrow style=tikz,diagrams={>=stealth}} %% cool arrow head
\tikzset{shorten <>/.style={ shorten >=#1, shorten <=#1 } } %% allows shorter vectors

\usetikzlibrary{backgrounds} %% for boxes around graphs
\usetikzlibrary{shapes,positioning}  %% Clouds and stars
\usetikzlibrary{matrix} %% for matrix
\usepgfplotslibrary{polar} %% for polar plots
\usepgfplotslibrary{fillbetween} %% to shade area between curves in TikZ
\usetkzobj{all}
\usepackage[makeroom]{cancel} %% for strike outs
%\usepackage{mathtools} %% for pretty underbrace % Breaks Ximera
%\usepackage{multicol}
\usepackage{pgffor} %% required for integral for loops



%% http://tex.stackexchange.com/questions/66490/drawing-a-tikz-arc-specifying-the-center
%% Draws beach ball
\tikzset{pics/carc/.style args={#1:#2:#3}{code={\draw[pic actions] (#1:#3) arc(#1:#2:#3);}}}



\usepackage{array}
\setlength{\extrarowheight}{+.1cm}
\newdimen\digitwidth
\settowidth\digitwidth{9}
\def\divrule#1#2{
\noalign{\moveright#1\digitwidth
\vbox{\hrule width#2\digitwidth}}}






\DeclareMathOperator{\arccot}{arccot}
\DeclareMathOperator{\arcsec}{arcsec}
\DeclareMathOperator{\arccsc}{arccsc}

















%%This is to help with formatting on future title pages.
\newenvironment{sectionOutcomes}{}{}


\author{Jim Talamo}
\license{Creative Commons 3.0 By-bC}


\outcome{}


\begin{document}
\begin{exercise}
Suppose that $f(x) = \sum_{k=1}^{\infty} \frac{40}{k}x^{2k-2}$.  This exercise asks you to find several values of the function and its derivatives at the center of the series.

\begin{exercise}
Find $f(0)$.

\[
f(0) = \answer{40}
\]

\begin{hint}
Write out a few terms of $f(x)$ and evaluate the expression at $x=0$.
\end{hint}
\end{exercise}

%%%%%%%%%%%%%%%%%%%%

\begin{exercise}
Find $f''(0)$.

\[
f''(0) = \answer{40}
\]

\begin{hint}
Is it easier to compute the series represented by $f''(x)$ explicitly or use the relationship between the coefficients of the power series and the derivatives of the function $f(x)$?
\end{hint}
\end{exercise}

%%%%%%%%%%%%%%%%%%%%

\begin{exercise}
Find $f^{(54)}(0)$.

\[
f^{(54)}(0) = \answer{\frac{10}{7} \cdot 54!}
\]

\begin{hint}
Is it easier to compute the series represented by $f^{(54)}(x)$ explicitly or use the relationship between the coefficients of the power series and the derivatives of the function $f(x)$?

If you want to use the formula:

\[
a_n = \frac{f^{(n)}(c)}{n!}
\]
then $c=\answer{0}$ and $n=\answer{54}$. 

In order to find $a_{54}$, note that $a_{54}$ is the coefficient of $x^{54}$.  
\end{hint}
\end{exercise}

What can you take away from this exercise?
\end{exercise}

\end{document}
