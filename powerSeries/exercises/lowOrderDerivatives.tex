\documentclass{ximera}


\graphicspath{
  {./}
  {ximeraTutorial/}
  {basicPhilosophy/}
}

\newcommand{\mooculus}{\textsf{\textbf{MOOC}\textnormal{\textsf{ULUS}}}}

\usepackage{tkz-euclide}\usepackage{tikz}
\usepackage{tikz-cd}
\usetikzlibrary{arrows}
\tikzset{>=stealth,commutative diagrams/.cd,
  arrow style=tikz,diagrams={>=stealth}} %% cool arrow head
\tikzset{shorten <>/.style={ shorten >=#1, shorten <=#1 } } %% allows shorter vectors

\usetikzlibrary{backgrounds} %% for boxes around graphs
\usetikzlibrary{shapes,positioning}  %% Clouds and stars
\usetikzlibrary{matrix} %% for matrix
\usepgfplotslibrary{polar} %% for polar plots
\usepgfplotslibrary{fillbetween} %% to shade area between curves in TikZ
\usetkzobj{all}
\usepackage[makeroom]{cancel} %% for strike outs
%\usepackage{mathtools} %% for pretty underbrace % Breaks Ximera
%\usepackage{multicol}
\usepackage{pgffor} %% required for integral for loops



%% http://tex.stackexchange.com/questions/66490/drawing-a-tikz-arc-specifying-the-center
%% Draws beach ball
\tikzset{pics/carc/.style args={#1:#2:#3}{code={\draw[pic actions] (#1:#3) arc(#1:#2:#3);}}}



\usepackage{array}
\setlength{\extrarowheight}{+.1cm}
\newdimen\digitwidth
\settowidth\digitwidth{9}
\def\divrule#1#2{
\noalign{\moveright#1\digitwidth
\vbox{\hrule width#2\digitwidth}}}






\DeclareMathOperator{\arccot}{arccot}
\DeclareMathOperator{\arcsec}{arcsec}
\DeclareMathOperator{\arccsc}{arccsc}

















%%This is to help with formatting on future title pages.
\newenvironment{sectionOutcomes}{}{}


\author{Jim Talamo}
\license{Creative Commons 3.0 By-bC}


\outcome{}


\begin{document}
\begin{exercise}
Let $f(x) = \sum_{k=0}^{\infty} \frac{4^k}{k!}(x-1)^{k+1}$.  Find $f(1)$, $f'(1)$, and $f''(1)$.

\[
f(1) = \answer{0} \qquad f'(1) = \answer{1} \qquad f''(1) = \answer{8}
\]

\begin{hint}
In order to work this, write out a few terms in the function that the summation notation represents:

\[
f(x) = \answer{0}+\answer{1} \cdot (x-1)+\answer{4}  \cdot (x-1)^2+\answer{8}  \cdot (x-1)^3+ \ldots
\]

This can be used to find that $f(1) = \answer{0}$ and can also be used to write down the first few derivatives.

\begin{question}
By differentiating $f(x)=(x-1)+4(x-1)^2+8(x-1)^3+\ldots$ , we find:

\[
f'(x) = \answer{1}+\answer{8} \cdot (x-1)+\answer{24}  \cdot (x-1)^2+ \ldots
\]
Thus, $f'(1) = \answer{1}$.

\begin{question}
By differentiating $f'(x)=1+8(x-1)+24(x-1)^2+\ldots$ , we find:

\[
f''(x) = \answer{8}+\answer{48} \cdot (x-1)+ \ldots
\]
Thus, $f''(1) = \answer{8}$.

\end{question}
\end{question}



\end{hint}
\end{exercise}
\end{document}
