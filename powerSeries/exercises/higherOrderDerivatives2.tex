\documentclass{ximera}


\graphicspath{
  {./}
  {ximeraTutorial/}
  {basicPhilosophy/}
}

\newcommand{\mooculus}{\textsf{\textbf{MOOC}\textnormal{\textsf{ULUS}}}}

\usepackage{tkz-euclide}\usepackage{tikz}
\usepackage{tikz-cd}
\usetikzlibrary{arrows}
\tikzset{>=stealth,commutative diagrams/.cd,
  arrow style=tikz,diagrams={>=stealth}} %% cool arrow head
\tikzset{shorten <>/.style={ shorten >=#1, shorten <=#1 } } %% allows shorter vectors

\usetikzlibrary{backgrounds} %% for boxes around graphs
\usetikzlibrary{shapes,positioning}  %% Clouds and stars
\usetikzlibrary{matrix} %% for matrix
\usepgfplotslibrary{polar} %% for polar plots
\usepgfplotslibrary{fillbetween} %% to shade area between curves in TikZ
\usetkzobj{all}
\usepackage[makeroom]{cancel} %% for strike outs
%\usepackage{mathtools} %% for pretty underbrace % Breaks Ximera
%\usepackage{multicol}
\usepackage{pgffor} %% required for integral for loops



%% http://tex.stackexchange.com/questions/66490/drawing-a-tikz-arc-specifying-the-center
%% Draws beach ball
\tikzset{pics/carc/.style args={#1:#2:#3}{code={\draw[pic actions] (#1:#3) arc(#1:#2:#3);}}}



\usepackage{array}
\setlength{\extrarowheight}{+.1cm}
\newdimen\digitwidth
\settowidth\digitwidth{9}
\def\divrule#1#2{
\noalign{\moveright#1\digitwidth
\vbox{\hrule width#2\digitwidth}}}






\DeclareMathOperator{\arccot}{arccot}
\DeclareMathOperator{\arcsec}{arcsec}
\DeclareMathOperator{\arccsc}{arccsc}

















%%This is to help with formatting on future title pages.
\newenvironment{sectionOutcomes}{}{}


\author{Jim Talamo}
\license{Creative Commons 3.0 By-bC}


\outcome{}


\begin{document}
\begin{exercise}
Suppose that $f(x) = \sum_{k=0}^{\infty} \frac{80}{k!}(x+1)^{2k}$.

\begin{exercise}
Find $f^{(80)}(-1)$.

\[
f^{(80)}(-1) = \answer{\frac{80 \cdot 80!}{40!}}
\]
\end{exercise}

\begin{hint}
A good way to proceed is to use the relationship between the coefficients of the power series and the derivatives of the function it represents.

\[
\textrm{If } f(x) = \sum_{k=0}^{\infty} a_k(x-c)^k, \textrm{ then: } a_n = \frac{f^{(n)}(c)}{n!}
\]

Here, $c=\answer{-1}$.  In order to find $f^{(80)}(-1)$, we should use $n=\answer{80}$.

The coefficient in question is thus $a_{\answer{80}}$.  

\begin{question}
By definition $a_{80}$ is:

\begin{multipleChoice}
\choice{Always the coefficient obtained by plugging in $k=80$.}
\choice[correct]{The coefficient in front of $(x-c)^{80}$.}
\end{multipleChoice}

In this case, we find $(x+1)^{2k}=(x+1)^{80}$ when $k=\answer{40}$, so:

\[
a_{80} =  \frac{80}{k!} \bigg|_{k=\answer{40}} = \frac{\answer{80}}{\answer{40!}}
\]

\begin{question}
We now use the formula $a_n = \frac{f^{(n)}(c)}{n!}$, with $n=80$ (as found earlier) to find:

\begin{align*}
a_{80} &= \frac{f^{(80)}(-1)}{80!} \\
\answer{\frac{80}{40!}} &= \frac{f^{(80)}(-1)}{\left(\answer{80}\right)!}
\end{align*}

\begin{question}
Thus, $f^{(80)}(-1) = \answer{\frac{80}{40!}} \cdot \left(\answer{80}\right)! $

\end{question}
\end{question}
\end{question}

\end{hint}

\end{exercise}
\end{document}
