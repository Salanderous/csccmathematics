\documentclass{ximera}


\graphicspath{
  {./}
  {ximeraTutorial/}
  {basicPhilosophy/}
}

\newcommand{\mooculus}{\textsf{\textbf{MOOC}\textnormal{\textsf{ULUS}}}}

\usepackage{tkz-euclide}\usepackage{tikz}
\usepackage{tikz-cd}
\usetikzlibrary{arrows}
\tikzset{>=stealth,commutative diagrams/.cd,
  arrow style=tikz,diagrams={>=stealth}} %% cool arrow head
\tikzset{shorten <>/.style={ shorten >=#1, shorten <=#1 } } %% allows shorter vectors

\usetikzlibrary{backgrounds} %% for boxes around graphs
\usetikzlibrary{shapes,positioning}  %% Clouds and stars
\usetikzlibrary{matrix} %% for matrix
\usepgfplotslibrary{polar} %% for polar plots
\usepgfplotslibrary{fillbetween} %% to shade area between curves in TikZ
\usetkzobj{all}
\usepackage[makeroom]{cancel} %% for strike outs
%\usepackage{mathtools} %% for pretty underbrace % Breaks Ximera
%\usepackage{multicol}
\usepackage{pgffor} %% required for integral for loops



%% http://tex.stackexchange.com/questions/66490/drawing-a-tikz-arc-specifying-the-center
%% Draws beach ball
\tikzset{pics/carc/.style args={#1:#2:#3}{code={\draw[pic actions] (#1:#3) arc(#1:#2:#3);}}}



\usepackage{array}
\setlength{\extrarowheight}{+.1cm}
\newdimen\digitwidth
\settowidth\digitwidth{9}
\def\divrule#1#2{
\noalign{\moveright#1\digitwidth
\vbox{\hrule width#2\digitwidth}}}






\DeclareMathOperator{\arccot}{arccot}
\DeclareMathOperator{\arcsec}{arcsec}
\DeclareMathOperator{\arccsc}{arccsc}

















%%This is to help with formatting on future title pages.
\newenvironment{sectionOutcomes}{}{}


\outcome{}

\title[Break-Ground:]{A secret of the definite integral}

\begin{document}
\begin{abstract}
Two young mathematicians discuss what calculus is all about.
\end{abstract}
\maketitle


Check out this dialogue between two calculus students (based on a true
story):

\begin{dialogue}
\item[Devyn] Ah. So now we have a connection between derivatives and
  integrals.
\item[Riley] Right, the derivative of the accumulation function is the
  ``inside'' function.
\item[Devyn] So how do we use this to compute area?
\end{dialogue}

Sometimes it helps to think about the most basic examples. Consider
\[
\int_2^5 4 dt
\]
We know (by geometry) that this computes the area of a $3\times 4$
rectangle which equals $12$. On the other hand, if we consider the
accumulation function
\[
F(x) = \int_2^x 4 dt,
\]
we see that
\[
F(5) = \int_2^5 4 dt.
\]
\begin{problem}
  What is $F(2)$?
  \begin{prompt}
    \[
    F(2) = \answer{0}
    \]
  \end{prompt}
\end{problem}

\begin{problem}
  On the other hand, the First Fundamental Theorem of Calculus says that if
  \[
  F(x) = \int_2^x 4 dt,
  \]
  then $F'(x) = 4$. Armed with this knowledge, and the fact that $F(2)
  = 0$, what must $F(x)$ be?
  \begin{prompt}
    \[
    F(x) = \answer{4x-8}
    \]
  \end{prompt}
\end{problem}





%\input{../leveledQuestions.tex}

\end{document}
