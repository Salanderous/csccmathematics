\documentclass{ximera}


\graphicspath{
  {./}
  {ximeraTutorial/}
  {basicPhilosophy/}
}

\newcommand{\mooculus}{\textsf{\textbf{MOOC}\textnormal{\textsf{ULUS}}}}

\usepackage{tkz-euclide}\usepackage{tikz}
\usepackage{tikz-cd}
\usetikzlibrary{arrows}
\tikzset{>=stealth,commutative diagrams/.cd,
  arrow style=tikz,diagrams={>=stealth}} %% cool arrow head
\tikzset{shorten <>/.style={ shorten >=#1, shorten <=#1 } } %% allows shorter vectors

\usetikzlibrary{backgrounds} %% for boxes around graphs
\usetikzlibrary{shapes,positioning}  %% Clouds and stars
\usetikzlibrary{matrix} %% for matrix
\usepgfplotslibrary{polar} %% for polar plots
\usepgfplotslibrary{fillbetween} %% to shade area between curves in TikZ
\usetkzobj{all}
\usepackage[makeroom]{cancel} %% for strike outs
%\usepackage{mathtools} %% for pretty underbrace % Breaks Ximera
%\usepackage{multicol}
\usepackage{pgffor} %% required for integral for loops



%% http://tex.stackexchange.com/questions/66490/drawing-a-tikz-arc-specifying-the-center
%% Draws beach ball
\tikzset{pics/carc/.style args={#1:#2:#3}{code={\draw[pic actions] (#1:#3) arc(#1:#2:#3);}}}



\usepackage{array}
\setlength{\extrarowheight}{+.1cm}
\newdimen\digitwidth
\settowidth\digitwidth{9}
\def\divrule#1#2{
\noalign{\moveright#1\digitwidth
\vbox{\hrule width#2\digitwidth}}}






\DeclareMathOperator{\arccot}{arccot}
\DeclareMathOperator{\arcsec}{arcsec}
\DeclareMathOperator{\arccsc}{arccsc}

















%%This is to help with formatting on future title pages.
\newenvironment{sectionOutcomes}{}{}


%\outcome{Find tangent lines to parametric curves}
\author{Jim Talamo}

\begin{document}
\begin{exercise}
The following mini-project has been written to synthesize concepts from you calculus journey, including:

\begin{itemize}
\item Related rates
\item Tangent lines
\item Area between curves
\item Volumes of solids of revolution
\end{itemize}

Completing this assignment successfully will be worth an additional 5 points to be added to your midterm subtotal score.  These points should be considered purely as \emph{bonus}.

Consider the curve $C$ defined by the parametric equations $x(t) = t^2$ and $y(t) = t$ for $t \geq 0$.  Eliminate the parameter to find a description of the curve in terms of $x$ and $y$ only.

 A description of the curve in terms of $x$ and $y$ only is $y = \answer{\sqrt{x}}$.

\begin{exercise}
The slope $m_{tan}$ of the tangent line to $C$ when $t=a$ is:

\[
m_{tan} = \answer{\frac{1}{2a}}
\]
(give your answer in terms of $a$).

The point $(x,y)$ on the curve when $t=a$ is:

\[
(x,y) = \left(\answer{a^2},\answer{a}\right)
\]
(give your answer in terms of $a$).
\end{exercise}
 
 The equation of the tangent line to the curve when $t=a$ is:
 
 \[
x+ \answer{-2a}y= \answer{-a^2}
 \] 
 (give your answer in terms of $a$).
 
\end{exercise}
\end{document}
