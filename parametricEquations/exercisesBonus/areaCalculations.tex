\documentclass{ximera}


\graphicspath{
  {./}
  {ximeraTutorial/}
  {basicPhilosophy/}
}

\newcommand{\mooculus}{\textsf{\textbf{MOOC}\textnormal{\textsf{ULUS}}}}

\usepackage{tkz-euclide}\usepackage{tikz}
\usepackage{tikz-cd}
\usetikzlibrary{arrows}
\tikzset{>=stealth,commutative diagrams/.cd,
  arrow style=tikz,diagrams={>=stealth}} %% cool arrow head
\tikzset{shorten <>/.style={ shorten >=#1, shorten <=#1 } } %% allows shorter vectors

\usetikzlibrary{backgrounds} %% for boxes around graphs
\usetikzlibrary{shapes,positioning}  %% Clouds and stars
\usetikzlibrary{matrix} %% for matrix
\usepgfplotslibrary{polar} %% for polar plots
\usepgfplotslibrary{fillbetween} %% to shade area between curves in TikZ
\usetkzobj{all}
\usepackage[makeroom]{cancel} %% for strike outs
%\usepackage{mathtools} %% for pretty underbrace % Breaks Ximera
%\usepackage{multicol}
\usepackage{pgffor} %% required for integral for loops



%% http://tex.stackexchange.com/questions/66490/drawing-a-tikz-arc-specifying-the-center
%% Draws beach ball
\tikzset{pics/carc/.style args={#1:#2:#3}{code={\draw[pic actions] (#1:#3) arc(#1:#2:#3);}}}



\usepackage{array}
\setlength{\extrarowheight}{+.1cm}
\newdimen\digitwidth
\settowidth\digitwidth{9}
\def\divrule#1#2{
\noalign{\moveright#1\digitwidth
\vbox{\hrule width#2\digitwidth}}}






\DeclareMathOperator{\arccot}{arccot}
\DeclareMathOperator{\arcsec}{arcsec}
\DeclareMathOperator{\arccsc}{arccsc}

















%%This is to help with formatting on future title pages.
\newenvironment{sectionOutcomes}{}{}


%\outcome{Find tangent lines to parametric curves}
\author{Jim Talamo}

\begin{document}
\begin{exercise}
Recall that the curve $C$ is described by the parametric equations $x(t)=t^2$ and $y=t$ for $t \geq 0$.  Take $t$ to be measured in seconds.  From the Desmos worksheet you created, you can make some qualitative observations about the area of $R(a)$ and the rate at which the area is changing.  To make more quantitative observations, we first need to find the area.

For any time $t=a$, an integral with respect to $y$ that gives the area is

\[
\textrm{Area}(R(a)) = \int_{y= 0}^{y = \answer{a}} \answer{y^2-2ay+a^2} dy
\]
(type your answer in terms of $a$).

\begin{hint}
To find the upper limit of integration, use the parametric equations to find $y$ when $t=a$.  To find the integrand, recall that the area is found by computing the integral $\int_{y=c}^{y=d} \left(x_{right}-x_{left} \right) dy$.
\end{hint}

Evaluating this integral gives

\[
\textrm{Area}(R(a)) = \answer{\frac{1}{3}a^3}
\]

\begin{exercise}
Go back to your Desmos worksheet and set $a=2$, then give responses to the following.

\begin{itemize}
\item When $a=2$, the area of $R(a)$ is $\answer{\frac{8}{3}}$.
\item When $a=2$, the rate at which the area of $R(a)$ is changing is $\answer{4} \mathrm{units^2/sec}$.
\end{itemize}

\end{exercise}

\end{exercise}
\end{document}
