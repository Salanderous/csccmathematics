\documentclass{ximera}


\graphicspath{
  {./}
  {ximeraTutorial/}
  {basicPhilosophy/}
}

\newcommand{\mooculus}{\textsf{\textbf{MOOC}\textnormal{\textsf{ULUS}}}}

\usepackage{tkz-euclide}\usepackage{tikz}
\usepackage{tikz-cd}
\usetikzlibrary{arrows}
\tikzset{>=stealth,commutative diagrams/.cd,
  arrow style=tikz,diagrams={>=stealth}} %% cool arrow head
\tikzset{shorten <>/.style={ shorten >=#1, shorten <=#1 } } %% allows shorter vectors

\usetikzlibrary{backgrounds} %% for boxes around graphs
\usetikzlibrary{shapes,positioning}  %% Clouds and stars
\usetikzlibrary{matrix} %% for matrix
\usepgfplotslibrary{polar} %% for polar plots
\usepgfplotslibrary{fillbetween} %% to shade area between curves in TikZ
\usetkzobj{all}
\usepackage[makeroom]{cancel} %% for strike outs
%\usepackage{mathtools} %% for pretty underbrace % Breaks Ximera
%\usepackage{multicol}
\usepackage{pgffor} %% required for integral for loops



%% http://tex.stackexchange.com/questions/66490/drawing-a-tikz-arc-specifying-the-center
%% Draws beach ball
\tikzset{pics/carc/.style args={#1:#2:#3}{code={\draw[pic actions] (#1:#3) arc(#1:#2:#3);}}}



\usepackage{array}
\setlength{\extrarowheight}{+.1cm}
\newdimen\digitwidth
\settowidth\digitwidth{9}
\def\divrule#1#2{
\noalign{\moveright#1\digitwidth
\vbox{\hrule width#2\digitwidth}}}






\DeclareMathOperator{\arccot}{arccot}
\DeclareMathOperator{\arcsec}{arcsec}
\DeclareMathOperator{\arccsc}{arccsc}

















%%This is to help with formatting on future title pages.
\newenvironment{sectionOutcomes}{}{}


\author{Jim Talamo and Jason Miller}
\license{Creative Commons 3.0 By-bC}


\outcome{}








\begin{document}
\begin{exercise}
The parametric equations below describe a curve $C$: 

\[
\begin{cases}
x(t) = t^2\\
y(t) = t^3
\end{cases} , \textrm{ for } t \geq 0
\]

Suppose that we want to find the equation of a tangent line to the curve at a given point $(x,y)$.  

Consider the point $(x,y) = (4,8)$.  

First, how can we verify if this point lies on the curve generated by our parametric equations? 

\begin{multipleChoice}
\choice{We need to see if $x=4$ for some $t$-value and $y=8$ for some potentially different $t$-value.}
\choice[correct]{We need to determine if there is a common $t$-value for which both $x(t) = 4$ and $y(t)=8$ simultaneously.} 
\end{multipleChoice}

In this case, we find when $t=\answer{2}$, $x(t) = 4$ and $y(t) =8$. 

Now that we now that $(4,8)$ lies on our curve, let's find the equation of the tangent line at this point.  We can proceed two ways:
%%%%%%%%%%%%%%%%%%%%%%%%%%%%%
\begin{exercise}
We can work directly with the parametric description of the curve.

Recall that point-slope form of the equation of a line is given by $y-y_{0}=m_{tan}(x-x_{0})$. Here $m$ is the slope of the line and $(x_{0}, y_{0})$ is a point that lies on the line. 

We need to find the slope of the line through the point $(4,8)$. Since our curve is given parametrically, we use $\frac{dy}{dx}=\frac{dy/dt}{dx/dt}$ to determine the slope:

\[ \frac{dy}{dx}=\answer{\frac{3}{2}t} \] 

When $t=2$, we find $\frac{dy}{dx} \bigg|_{t=2}= 3$.  

Thus, using the point-slope form of a line:

\begin{align*}
y-y(2)& = m_{tan}\left(x-x(2)\right) \\
y-\left(\answer{8}\right)& =\answer{3} \cdot \left(x-\answer{4}\right) \\
\end{align*}

Thus the tangent line to our curve at the point $(4, 8)$ is $y=\answer{3x-4}$. 

\begin{exercise}
Follow the instructions in the earlier exercise to draw a picture of this on Desmos. In an unused line, type ``$y=3x-4$'' and verify that this is indeed the tangent line at $(4,8)$.
\end{exercise}
\end{exercise}
%%%%%%%%%%%%%%%%%%%%%%%%%%%%%
\begin{exercise}
Of course, we can also find a description of our curve in Cartesian coordinates, then find the tangent line.  Notice that given our parametric equations that $x^3=\answer{t^6}$ and that $y^2=\answer{t^6}$.

(give answers in terms of $t$)

Thus the the $x$-coordinate and $y$-coordinate of any point on our given curve satisfies $y^{\answer{2}}=x^{\answer{3}}$. 

We will use implicit differentiation since the above equation does not represent a function.  Differentiating both sides with respect to $x$ gives:

\begin{align*}
\left(\answer{2y}\right)\frac{dy}{dx} &= \answer{3x^2} \\
\frac{dy}{dx} &= \answer{\frac{3x^2}{2y}}
\end{align*}

\begin{exercise}
Now we can calculate the derivative at the point $(4,8)$ to find:

\[
\frac{dy}{dx}\bigg|_{(x,y)=(4,8)} = \answer{3}.
\]

Thus, using the point-slope form of a line:

\begin{align*}
y-y_0& = m_{tan}\left(x-x_0\right) \\
y-\left(\answer{8}\right)& =\answer{3} \cdot \left(x-\answer{4}\right). \\
\end{align*}

Thus, the tangent line to our curve at the point $(4, 8)$ is $y=\answer{3x-4}$. 

Do your answers using both methods agree?

\begin{multipleChoice}
\choice[correct]{Yes}
\choice{No}
\end{multipleChoice}

\begin{feedback}[correct]
Note that while either method will produce the tangent line, the second method requires that we are able to eliminate the parameter and will often involve implicit differentiation.  It is usually more convenient to work directly with the parametric description of the curve.
\end{feedback}

\end{exercise}
\end{exercise}


\end{exercise}


\end{document}
