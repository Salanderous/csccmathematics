\documentclass{ximera}


\graphicspath{
  {./}
  {ximeraTutorial/}
  {basicPhilosophy/}
}

\newcommand{\mooculus}{\textsf{\textbf{MOOC}\textnormal{\textsf{ULUS}}}}

\usepackage{tkz-euclide}\usepackage{tikz}
\usepackage{tikz-cd}
\usetikzlibrary{arrows}
\tikzset{>=stealth,commutative diagrams/.cd,
  arrow style=tikz,diagrams={>=stealth}} %% cool arrow head
\tikzset{shorten <>/.style={ shorten >=#1, shorten <=#1 } } %% allows shorter vectors

\usetikzlibrary{backgrounds} %% for boxes around graphs
\usetikzlibrary{shapes,positioning}  %% Clouds and stars
\usetikzlibrary{matrix} %% for matrix
\usepgfplotslibrary{polar} %% for polar plots
\usepgfplotslibrary{fillbetween} %% to shade area between curves in TikZ
\usetkzobj{all}
\usepackage[makeroom]{cancel} %% for strike outs
%\usepackage{mathtools} %% for pretty underbrace % Breaks Ximera
%\usepackage{multicol}
\usepackage{pgffor} %% required for integral for loops



%% http://tex.stackexchange.com/questions/66490/drawing-a-tikz-arc-specifying-the-center
%% Draws beach ball
\tikzset{pics/carc/.style args={#1:#2:#3}{code={\draw[pic actions] (#1:#3) arc(#1:#2:#3);}}}



\usepackage{array}
\setlength{\extrarowheight}{+.1cm}
\newdimen\digitwidth
\settowidth\digitwidth{9}
\def\divrule#1#2{
\noalign{\moveright#1\digitwidth
\vbox{\hrule width#2\digitwidth}}}






\DeclareMathOperator{\arccot}{arccot}
\DeclareMathOperator{\arcsec}{arcsec}
\DeclareMathOperator{\arccsc}{arccsc}

















%%This is to help with formatting on future title pages.
\newenvironment{sectionOutcomes}{}{}



%\outcome{Find tangent lines to parametric curves}
\author{Jim Talamo and Alex Beckwith}

\begin{document}
\begin{exercise}

Suppose that a curve $C$ is defined by the parametric equations:
\[
\begin{cases}
x(t) &= 2\cos(t)+\sin(t) \\
y(t) &= 4\cos(t)
\end{cases}
\]
Find the derivative $\frac{dy}{dx}$ in terms of $t$.

\[
\frac{dy}{dx} = \frac{\answer{-4\sin(t)}}{\answer{-2\sin(t)+\cos(t)}}
\]

\begin{hint}
Using the chain rule allows us to write $\frac{dy}{dx} = \frac{dy/dt}{dx/dt}$.
\end{hint}

%%%Horizontal Tangent Lines%%%%%%
\begin{exercise}
How many horizontal tangent lines does the curve $C$ have?
\begin{multipleChoice}
\choice{$0$}
\choice{$1$}
\choice[correct]{$2$}
\choice{$3$}
\end{multipleChoice}

The horizontal tangent lines occur when $t=\answer{0}$ and $t=\answer{\pi}$.

\begin{exercise}
When $t=0$, the horizontal tangent line is $y=\answer{4}$.

When $t=\pi$, the horizontal tangent line is $y=\answer{-4}$.
\end{exercise}
\end{exercise}

%%%vertical Tangent Lines%%%%%%
\begin{exercise}
Next we want to find all tangent lines to $C$ with slope 2. 

The values of $t$ for which $\frac{dy}{dx} =2$ are $t=\answer{\frac{\pi}{2}}$ and $t=\answer{\frac{3\pi}{2}}$.

(type the smaller $t$-value first)

\begin{exercise}
When $t=\frac{\pi}{2}$, the Cartesian equation of the tangent line is $y=\answer{2}x+\answer{-2}$.

When $t=\frac{3\pi}{2}$, the Cartesian equation of the tangent line is $y=\answer{2}x+\answer{2}$.
\end{exercise}
\end{exercise}


\end{exercise}


\end{document}