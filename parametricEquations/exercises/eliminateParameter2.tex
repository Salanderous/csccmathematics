\documentclass{ximera}


\graphicspath{
  {./}
  {ximeraTutorial/}
  {basicPhilosophy/}
}

\newcommand{\mooculus}{\textsf{\textbf{MOOC}\textnormal{\textsf{ULUS}}}}

\usepackage{tkz-euclide}\usepackage{tikz}
\usepackage{tikz-cd}
\usetikzlibrary{arrows}
\tikzset{>=stealth,commutative diagrams/.cd,
  arrow style=tikz,diagrams={>=stealth}} %% cool arrow head
\tikzset{shorten <>/.style={ shorten >=#1, shorten <=#1 } } %% allows shorter vectors

\usetikzlibrary{backgrounds} %% for boxes around graphs
\usetikzlibrary{shapes,positioning}  %% Clouds and stars
\usetikzlibrary{matrix} %% for matrix
\usepgfplotslibrary{polar} %% for polar plots
\usepgfplotslibrary{fillbetween} %% to shade area between curves in TikZ
\usetkzobj{all}
\usepackage[makeroom]{cancel} %% for strike outs
%\usepackage{mathtools} %% for pretty underbrace % Breaks Ximera
%\usepackage{multicol}
\usepackage{pgffor} %% required for integral for loops



%% http://tex.stackexchange.com/questions/66490/drawing-a-tikz-arc-specifying-the-center
%% Draws beach ball
\tikzset{pics/carc/.style args={#1:#2:#3}{code={\draw[pic actions] (#1:#3) arc(#1:#2:#3);}}}



\usepackage{array}
\setlength{\extrarowheight}{+.1cm}
\newdimen\digitwidth
\settowidth\digitwidth{9}
\def\divrule#1#2{
\noalign{\moveright#1\digitwidth
\vbox{\hrule width#2\digitwidth}}}






\DeclareMathOperator{\arccot}{arccot}
\DeclareMathOperator{\arcsec}{arcsec}
\DeclareMathOperator{\arccsc}{arccsc}

















%%This is to help with formatting on future title pages.
\newenvironment{sectionOutcomes}{}{}


%\outcome{Find tangent lines to parametric curves}
\author{Jim Talamo and Nick Hemleben}





\begin{document}
\begin{exercise}
Eliminate the parameter and find a Cartesian representation of the given curve.

\[
\begin{cases}
x(t) =  3 \sin (t) \\
y(t) =  3 \cos(t)+3
\end{cases}
, 0 \leq t \leq 2\pi
\]

The curve is a:

\begin{multipleChoice}
\choice{line}
\choice{parabola}
\choice[correct]{circle}
\choice{square}
\end{multipleChoice}

The radius of the circle is $\answer{3}$, and the equation that describes it is:

\[
 \answer{ x^2 + (y-3)^2 } = 9
\]
(Write out an expression in terms of $x$ and $y$.)
\begin{hint}
Use the identity $\sin^2(t)+\cos^2(t) =1$ to eliminate the parameter.  

From the above, we find:

\begin{align*}
\sin(t) &= \answer{\frac{1}{3}x} \\ 
\cos(t) &= \answer{\frac{1}{3}(y-3)}
\end{align*}

Thus:

\[
\sin^2(t) + \cos^2(t) = \left(\answer{\frac{1}{3}x}\right)^2+ \left(\answer{\frac{1}{3}(y-3)}\right)^2 =1
\]
\end{hint}

As $t$ increases, the circle is traversed in which direction?
\begin{multipleChoice}
\choice[correct]{clockwise}
\choice{counterclockwise}
\end{multipleChoice}
\end{exercise}

\end{document}