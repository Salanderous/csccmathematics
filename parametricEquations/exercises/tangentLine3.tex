\documentclass{ximera}


\graphicspath{
  {./}
  {ximeraTutorial/}
  {basicPhilosophy/}
}

\newcommand{\mooculus}{\textsf{\textbf{MOOC}\textnormal{\textsf{ULUS}}}}

\usepackage{tkz-euclide}\usepackage{tikz}
\usepackage{tikz-cd}
\usetikzlibrary{arrows}
\tikzset{>=stealth,commutative diagrams/.cd,
  arrow style=tikz,diagrams={>=stealth}} %% cool arrow head
\tikzset{shorten <>/.style={ shorten >=#1, shorten <=#1 } } %% allows shorter vectors

\usetikzlibrary{backgrounds} %% for boxes around graphs
\usetikzlibrary{shapes,positioning}  %% Clouds and stars
\usetikzlibrary{matrix} %% for matrix
\usepgfplotslibrary{polar} %% for polar plots
\usepgfplotslibrary{fillbetween} %% to shade area between curves in TikZ
\usetkzobj{all}
\usepackage[makeroom]{cancel} %% for strike outs
%\usepackage{mathtools} %% for pretty underbrace % Breaks Ximera
%\usepackage{multicol}
\usepackage{pgffor} %% required for integral for loops



%% http://tex.stackexchange.com/questions/66490/drawing-a-tikz-arc-specifying-the-center
%% Draws beach ball
\tikzset{pics/carc/.style args={#1:#2:#3}{code={\draw[pic actions] (#1:#3) arc(#1:#2:#3);}}}



\usepackage{array}
\setlength{\extrarowheight}{+.1cm}
\newdimen\digitwidth
\settowidth\digitwidth{9}
\def\divrule#1#2{
\noalign{\moveright#1\digitwidth
\vbox{\hrule width#2\digitwidth}}}






\DeclareMathOperator{\arccot}{arccot}
\DeclareMathOperator{\arcsec}{arcsec}
\DeclareMathOperator{\arccsc}{arccsc}

















%%This is to help with formatting on future title pages.
\newenvironment{sectionOutcomes}{}{}


%\outcome{Find tangent lines to parametric curves}
\author{Jim Talamo and Alex Beckwith}

\begin{document}
\begin{exercise}

Suppose that a curve $C$ is defined by the parametric equations for $0 \leq t < 2\pi$:
\[
\begin{cases}
x(t) &= \sin(t)+\cos(t) \\
y(t) &= \sin(t)-\cos(t)
\end{cases}
\]

Find the derivative $\frac{dy}{dx}$ in terms of $t$.

\[
\frac{dy}{dx} = \frac{\answer{\sin(t)+\cos(t)}}{\answer{\cos(t)-\sin(t)}}
\]

\begin{hint}
Using the chain rule allows us to write $\frac{dy}{dx} = \frac{dy/dt}{dx/dt}$.
\end{hint}

%%%Vertical Tangent Lines%%%%%%
\begin{exercise}
How many vertical tangent lines does the curve $C$ have for $0 \leq t \leq 2\pi$?
\begin{multipleChoice}
\choice{$0$}
\choice{$1$}
\choice[correct]{$2$}
\choice{$3$}
\end{multipleChoice}

The vertical tangent lines occur when $t=\answer{\frac{\pi}{4}}$ and $t=\answer{\frac{5\pi}{4}}$.

\begin{exercise}
When $t=\frac{\pi}{4}$, the vertical tangent line is $x=\answer{\sqrt{2}}$ and occurs at the point $(x,y) = \left(\answer{\sqrt{2}},\answer{0}\right)$.

When $t=\frac{5\pi}{4}$, the vertical tangent line is $x=\answer{-\sqrt{2}}$ and occurs at the point $(x,y) = \left(\answer{-\sqrt{2}},\answer{0}\right)$.
\end{exercise}
\end{exercise}

%%%Horizontal Tangent Lines%%%%%%
\begin{exercise}
How many horizontal tangent lines does the curve $C$ have?
\begin{multipleChoice}
\choice{$0$}
\choice{$1$}
\choice[correct]{$2$}
\choice{$3$}
\end{multipleChoice}

The horizontal tangent lines occur when $t=\answer{\frac{3\pi}{4}}$ and $t= \answer{\frac{7\pi}{4}}$.

\begin{exercise}
When $t=\frac{3\pi}{4}$, the horizontal tangent line is $y=\answer{\sqrt{2}}$ and occurs at the point $(x,y) = \left(\answer{0},\answer{\sqrt{2}}\right)$.

When $t=\frac{7\pi}{4}$, the horizontal tangent line is $y=\answer{-\sqrt{2}}$ and occurs at the point $(x,y) = \left(\answer{0},\answer{-\sqrt{2}}\right)$.
\end{exercise}
\end{exercise}

%%%%%GIVEN SLOPE%%%%%%%%%%%%%%%%%%
\begin{exercise}

How many distinct tangent lines of slope $1$ does the curve $C$ have?
\begin{multipleChoice}
\choice{$0$}
\choice{$1$}
\choice[correct]{$2$}
\choice{$3$}
\end{multipleChoice} 

The tangent lines have slope $1$ when $t= \answer{0}$ and $t=\answer{\pi}$.

(type the smaller $t$-value first)

\begin{exercise}
When $t=0$, the tangent line is $y=\answer{x-2}$.

When $t=\pi$, the tangent line is $y=\answer{x+2}$.
\end{exercise}
 
\begin{hint}
Since we have the slope, we can find all $t$-values for which this occurs by setting $\frac{dy}{dx} = 1$:

\[
\frac{dy}{dx} = 1 = \frac{\sin(t)+\cos(t)}{\cos(t)-\sin(t)}
\]
Multiplying both sides by $\cos(t)-\sin(t)$ gives:

\[
\answer{\cos(t)-\sin(t)} =\sin(t)+\cos(t)
\]
from which we find that $2\sin(t)=0$.  Solving for $t$ gives $t=\answer{0}$ and $t=\answer{\pi}$

(type the smaller $t$-value first)

\begin{question}
When $t=0$, we find that $x(0) = \answer{1}$ and $y(0) = \answer{-1}$.  The equation of the tangent line is thus:

\begin{align*}
y-y(0) &= m_{tan}(x-x(0)) \\
y-\left(\answer{-1}\right) &= \answer{1} \cdot \left[x- \left(\answer{1}\right) \right]\\
y&= \answer{x-2}
\end{align*}

\end{question}

\begin{question}
When $t=\pi$, we find that $x(\pi) = \answer{-1}$ and $y(\pi) = \answer{1}$.  The equation of the tangent line is thus:

\begin{align*}
y-y(\pi) &= m_{tan}(x-x(\pi)) \\
y-\left(\answer{1}\right) &= \answer{1} \cdot \left[x- \left(\answer{-1}\right) \right]\\
y&= \answer{x+2}
\end{align*}

\end{question}


\end{hint}
\end{exercise}
\end{exercise}
\end{document}