\documentclass{ximera}


\graphicspath{
  {./}
  {ximeraTutorial/}
  {basicPhilosophy/}
}

\newcommand{\mooculus}{\textsf{\textbf{MOOC}\textnormal{\textsf{ULUS}}}}

\usepackage{tkz-euclide}\usepackage{tikz}
\usepackage{tikz-cd}
\usetikzlibrary{arrows}
\tikzset{>=stealth,commutative diagrams/.cd,
  arrow style=tikz,diagrams={>=stealth}} %% cool arrow head
\tikzset{shorten <>/.style={ shorten >=#1, shorten <=#1 } } %% allows shorter vectors

\usetikzlibrary{backgrounds} %% for boxes around graphs
\usetikzlibrary{shapes,positioning}  %% Clouds and stars
\usetikzlibrary{matrix} %% for matrix
\usepgfplotslibrary{polar} %% for polar plots
\usepgfplotslibrary{fillbetween} %% to shade area between curves in TikZ
\usetkzobj{all}
\usepackage[makeroom]{cancel} %% for strike outs
%\usepackage{mathtools} %% for pretty underbrace % Breaks Ximera
%\usepackage{multicol}
\usepackage{pgffor} %% required for integral for loops



%% http://tex.stackexchange.com/questions/66490/drawing-a-tikz-arc-specifying-the-center
%% Draws beach ball
\tikzset{pics/carc/.style args={#1:#2:#3}{code={\draw[pic actions] (#1:#3) arc(#1:#2:#3);}}}



\usepackage{array}
\setlength{\extrarowheight}{+.1cm}
\newdimen\digitwidth
\settowidth\digitwidth{9}
\def\divrule#1#2{
\noalign{\moveright#1\digitwidth
\vbox{\hrule width#2\digitwidth}}}






\DeclareMathOperator{\arccot}{arccot}
\DeclareMathOperator{\arcsec}{arcsec}
\DeclareMathOperator{\arccsc}{arccsc}

















%%This is to help with formatting on future title pages.
\newenvironment{sectionOutcomes}{}{}


\author{Jim Talamo}
\license{Creative Commons 3.0 By-bC}


\outcome{}


\begin{document}
\begin{exercise}
Consider the series $\sum_{k=1}^{\infty} \left(\frac{2}{k}\right)^k$.

\begin{multipleChoice}
\choice[correct]{The Root Test applies to this series.}
\choice{The Root Test does not apply to this series.}
\end{multipleChoice}

Why is the Root Test preferable to the Ratio Test here?
\begin{multipleChoice}
\choice[correct]{The Root Test will allow for a more efficient solution since the summand involves the $k$-th power of an expression.}
\choice{The assumptions for the Root Test are not met.}
\end{multipleChoice}

To use the Root Test, note that the sequence $\{a_n\}_{n=1}$ is given by the rule:

\[
a_n = \answer{\left(\frac{2}{n}\right)^n}
\]

Thus, to use the Root Test, we must compute the limit:

\[
L = \lim_{n \to \infty} \sqrt[n]{\answer{\left(\frac{2}{n}\right)^n}} 
\]

Evaluating this limit, we find:

\[
L = \answer{0}
\]

\begin{hint}
Simplifying: $ \sqrt[n]{\left(\frac{2}{n}\right)^n} = \answer{ \frac{2}{n}}$
\end{hint}

Hence, the Root Test:
\begin{multipleChoice}
\choice[correct]{guarantees that the series converges.}
\choice{guarantees that the series converges to $0$.}
\choice{guarantees that the series diverges.}
\choice{is inconclusive.}
\end{multipleChoice}

\end{exercise}
\end{document}
