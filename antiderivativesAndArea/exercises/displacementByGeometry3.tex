\documentclass{ximera}


\graphicspath{
  {./}
  {ximeraTutorial/}
  {basicPhilosophy/}
}

\newcommand{\mooculus}{\textsf{\textbf{MOOC}\textnormal{\textsf{ULUS}}}}

\usepackage{tkz-euclide}\usepackage{tikz}
\usepackage{tikz-cd}
\usetikzlibrary{arrows}
\tikzset{>=stealth,commutative diagrams/.cd,
  arrow style=tikz,diagrams={>=stealth}} %% cool arrow head
\tikzset{shorten <>/.style={ shorten >=#1, shorten <=#1 } } %% allows shorter vectors

\usetikzlibrary{backgrounds} %% for boxes around graphs
\usetikzlibrary{shapes,positioning}  %% Clouds and stars
\usetikzlibrary{matrix} %% for matrix
\usepgfplotslibrary{polar} %% for polar plots
\usepgfplotslibrary{fillbetween} %% to shade area between curves in TikZ
\usetkzobj{all}
\usepackage[makeroom]{cancel} %% for strike outs
%\usepackage{mathtools} %% for pretty underbrace % Breaks Ximera
%\usepackage{multicol}
\usepackage{pgffor} %% required for integral for loops



%% http://tex.stackexchange.com/questions/66490/drawing-a-tikz-arc-specifying-the-center
%% Draws beach ball
\tikzset{pics/carc/.style args={#1:#2:#3}{code={\draw[pic actions] (#1:#3) arc(#1:#2:#3);}}}



\usepackage{array}
\setlength{\extrarowheight}{+.1cm}
\newdimen\digitwidth
\settowidth\digitwidth{9}
\def\divrule#1#2{
\noalign{\moveright#1\digitwidth
\vbox{\hrule width#2\digitwidth}}}






\DeclareMathOperator{\arccot}{arccot}
\DeclareMathOperator{\arcsec}{arcsec}
\DeclareMathOperator{\arccsc}{arccsc}

















%%This is to help with formatting on future title pages.
\newenvironment{sectionOutcomes}{}{}



\outcome{Interpert the product of rate and time as area.}
\outcome{Approximate position from velocity.}
\outcome{Recognize Riemann sums.}

\author{Nela Lakos \and Kyle Parsons}

\begin{document}
\begin{exercise}

A function $v$ describes the velocity of a car (in mi/hr) moving along a straight highway for a 4 hour interval.  The graph of $v$ is given below.

\begin{image}
  \begin{tikzpicture}
    \begin{axis}[
        xmin=-0.1,xmax=4.1,ymin=-0.8,ymax=25.8,
        clip=true,
        unit vector ratio*=8 1 1,
        axis lines=center,
        grid = major,
        ytick={0,5,...,36},
        xtick={0,1,...,6},
        xlabel=$t$, ylabel=$v$,
        every axis y label/.style={at=(current axis.above origin),anchor=south},
        every axis x label/.style={at=(current axis.right of origin),anchor=west},
      ]
      \addplot[ultra thick,penColor,domain=0:1] plot{20*x};     
      \addplot[ultra thick,penColor,domain=1:4] plot{20};
             
      %\node at (axis cs:4.5,30) {$v(t)$};
      \end{axis}`
  \end{tikzpicture}
\end{image}

Using geometry, we can find the displacement of the car over the 4 hour period as
\[
s(4)-s(0) = \answer{70}\text{mi}.
\]

Using a Left Riemann sum with $n=2$ subintervals, we can approximate this displacement as
\[
s(4)-s(0)\approx\answer{40}\text{mi}.
\]

Using a Right Riemann sum with $n=4$ subintervals, we can approximate this displacement as
\[
s(4)-s(0)\approx\answer{80}\text{mi}.
\]

Finally, using a Midpoint Riemann sum with $n=2$ subintervals, we can approximate this displacement as
\[
s(4)-s(0)\approx\answer{80}\text{mi}.
\]

Using geometry, we can find an expression for the displacement up to time $t$ for $1\leq t\leq4$ as
\[
s(t)-s(0) = \answer{10+20(t-1)}\text{mi}.
\]

\end{exercise}
\end{document}