\documentclass{ximera}


\graphicspath{
  {./}
  {ximeraTutorial/}
  {basicPhilosophy/}
}

\newcommand{\mooculus}{\textsf{\textbf{MOOC}\textnormal{\textsf{ULUS}}}}

\usepackage{tkz-euclide}\usepackage{tikz}
\usepackage{tikz-cd}
\usetikzlibrary{arrows}
\tikzset{>=stealth,commutative diagrams/.cd,
  arrow style=tikz,diagrams={>=stealth}} %% cool arrow head
\tikzset{shorten <>/.style={ shorten >=#1, shorten <=#1 } } %% allows shorter vectors

\usetikzlibrary{backgrounds} %% for boxes around graphs
\usetikzlibrary{shapes,positioning}  %% Clouds and stars
\usetikzlibrary{matrix} %% for matrix
\usepgfplotslibrary{polar} %% for polar plots
\usepgfplotslibrary{fillbetween} %% to shade area between curves in TikZ
\usetkzobj{all}
\usepackage[makeroom]{cancel} %% for strike outs
%\usepackage{mathtools} %% for pretty underbrace % Breaks Ximera
%\usepackage{multicol}
\usepackage{pgffor} %% required for integral for loops



%% http://tex.stackexchange.com/questions/66490/drawing-a-tikz-arc-specifying-the-center
%% Draws beach ball
\tikzset{pics/carc/.style args={#1:#2:#3}{code={\draw[pic actions] (#1:#3) arc(#1:#2:#3);}}}



\usepackage{array}
\setlength{\extrarowheight}{+.1cm}
\newdimen\digitwidth
\settowidth\digitwidth{9}
\def\divrule#1#2{
\noalign{\moveright#1\digitwidth
\vbox{\hrule width#2\digitwidth}}}






\DeclareMathOperator{\arccot}{arccot}
\DeclareMathOperator{\arcsec}{arcsec}
\DeclareMathOperator{\arccsc}{arccsc}

















%%This is to help with formatting on future title pages.
\newenvironment{sectionOutcomes}{}{}


\outcome{}

% BADBAD This is not done

% Harvard calculus left the mean value theorem out.

% How is this related to the "racetrack principle"?

\title[Break-Ground:]{Meaning of multiplication}

\begin{document}
\begin{abstract}
A dialogue where students discuss multiplication.
\end{abstract}
\maketitle

Check out this dialogue between two calculus students (based on a true story):

\begin{dialogue}
\item[Devyn] Hey Riley, I was reading about the history of mathematics.
\item[Riley] Really? Tell me about it!
\item[Devyn] Apparently, back in the day, mathematicians worried about writing things like:
  \[
  3\times 4 + 5
  \]
\item[Riley] Why? What's the matter here?
\item[Devyn] Well, they thought of $3\times 4$ as an
  \textbf{area}. But $5$ was thought of as a \textbf{length}. Apparently they worried whether it made sense to add ``areas'' and ``lengths.''
\item[Riley] Hmmm. We don't seem to worry about that now. I wonder why?
\end{dialogue}

\begin{problem}
  What are some ways to interpret $3\times 4$?
  \begin{freeResponse}
    Answer will vary.
  \end{freeResponse}
\end{problem}

%\input{../leveledQuestions.tex}

\end{document}
