\documentclass{ximera}


\graphicspath{
  {./}
  {ximeraTutorial/}
  {basicPhilosophy/}
}

\newcommand{\mooculus}{\textsf{\textbf{MOOC}\textnormal{\textsf{ULUS}}}}

\usepackage{tkz-euclide}\usepackage{tikz}
\usepackage{tikz-cd}
\usetikzlibrary{arrows}
\tikzset{>=stealth,commutative diagrams/.cd,
  arrow style=tikz,diagrams={>=stealth}} %% cool arrow head
\tikzset{shorten <>/.style={ shorten >=#1, shorten <=#1 } } %% allows shorter vectors

\usetikzlibrary{backgrounds} %% for boxes around graphs
\usetikzlibrary{shapes,positioning}  %% Clouds and stars
\usetikzlibrary{matrix} %% for matrix
\usepgfplotslibrary{polar} %% for polar plots
\usepgfplotslibrary{fillbetween} %% to shade area between curves in TikZ
\usetkzobj{all}
\usepackage[makeroom]{cancel} %% for strike outs
%\usepackage{mathtools} %% for pretty underbrace % Breaks Ximera
%\usepackage{multicol}
\usepackage{pgffor} %% required for integral for loops



%% http://tex.stackexchange.com/questions/66490/drawing-a-tikz-arc-specifying-the-center
%% Draws beach ball
\tikzset{pics/carc/.style args={#1:#2:#3}{code={\draw[pic actions] (#1:#3) arc(#1:#2:#3);}}}



\usepackage{array}
\setlength{\extrarowheight}{+.1cm}
\newdimen\digitwidth
\settowidth\digitwidth{9}
\def\divrule#1#2{
\noalign{\moveright#1\digitwidth
\vbox{\hrule width#2\digitwidth}}}






\DeclareMathOperator{\arccot}{arccot}
\DeclareMathOperator{\arcsec}{arcsec}
\DeclareMathOperator{\arccsc}{arccsc}

















%%This is to help with formatting on future title pages.
\newenvironment{sectionOutcomes}{}{}


%\outcome{Given a velocity function, calculate displacement and distance traveled.}
%\outcome{Given a velocity function, find the position function.}
%\outcome{Given an acceleration function, find the velocity function.}
%\outcome{Understand the difference between displacement and distance traveled.}
%\outcome{Understand the relationship between position, velocity and acceleration.}

\author{Nela Lakos \and Kyle Parsons}

\begin{document}
\begin{exercise}

The object is at the origin when it starts moving along the straight line. The velocity of an object is given by the function
\[
v(t) = 
\begin{cases}
t-2 & 0\leq t\leq4\\
2\cos\left(\frac{\pi t}{2}\right) & 4<t\leq8
\end{cases}.
\]
(a) Find $s(t)$, the position of the object at a time $t$, for $0\leq t\leq8$.
\begin{hint}
Solve the IVP:
\[
s'(t) = v(t)
\]

\[
s(0) = 0
\]
\end{hint}
\begin{hint}
Since the function $v$ is a piece-wise defined function, we will first solve the following IVP for $0\leq t\leq4$:
\[
s'(t) =t-2
\]

\[
s(0) = 0
\]
\end{hint}
\begin{hint}
For $0\leq t\leq4$:
\[
s(t) =\answer{\frac{1}{2}}t^2-\answer{2}t+\answer{0}
\]
\end{hint}
\begin{hint}
Now we have to solve the IVP for $4\leq t\leq8$:
\[
s'(t) =2\cos\left(\frac{\pi t}{2}\right)
\]

But what is the initial condition here?
$s(0)=0$ does not apply, because $4\leq t\leq8$.
But we know that the solution, $s$ has to be a differentiable function.  Therefore, $s$ is continuous. So, it is , in particular, continuous at $t=4$.
Therefore, using the expression for $s(t)$ for $t$ in $[0,4]$, we get that

 $s(4)= \lim_{t\to 4^{-}}s(t)=
\frac{1}{2}(16)-2(4)=0$
\end{hint}
\begin{hint}
For $0\leq t\leq4$:
\[
s(t) =\answer{\frac{1}{2}}t^2-\answer{2}t+\answer{0}
\]
\end{hint}
\begin{hint}
Now we have to solve the IVP for $4\leq t\leq8$:
\[
s'(t) =2\cos\left(\frac{\pi t}{2}\right)
\]
\[
s(4) =0
\]
So, 
\[
s(t) =\answer{\frac{4}{\pi}}\sin\left(\frac{\pi t}{2}\right) +C,
\]
and
\[
s(4) =0.
\]
If we plug in $t=4$, we get
\[
s(4) =\answer{\frac{4}{\pi}}\sin\left(\frac{\pi (4)}{2}\right) +C=C,
\]
Therefore,
\[
C =\answer{0}.
\]
\end{hint}
SOLUTION:

If $ 0<t\leq4$
\[
s(t) = \answer{\frac{1}{2}}t^2-\answer{2}t ;  
\]
if $ 4<t\leq8$
\[
s(t)=\answer{\frac{4}{\pi}}\sin\left(\frac{\pi t}{2}\right).
\]

\end{exercise}
\end{document}