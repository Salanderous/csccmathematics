\documentclass{ximera}


\graphicspath{
  {./}
  {ximeraTutorial/}
  {basicPhilosophy/}
}

\newcommand{\mooculus}{\textsf{\textbf{MOOC}\textnormal{\textsf{ULUS}}}}

\usepackage{tkz-euclide}\usepackage{tikz}
\usepackage{tikz-cd}
\usetikzlibrary{arrows}
\tikzset{>=stealth,commutative diagrams/.cd,
  arrow style=tikz,diagrams={>=stealth}} %% cool arrow head
\tikzset{shorten <>/.style={ shorten >=#1, shorten <=#1 } } %% allows shorter vectors

\usetikzlibrary{backgrounds} %% for boxes around graphs
\usetikzlibrary{shapes,positioning}  %% Clouds and stars
\usetikzlibrary{matrix} %% for matrix
\usepgfplotslibrary{polar} %% for polar plots
\usepgfplotslibrary{fillbetween} %% to shade area between curves in TikZ
\usetkzobj{all}
\usepackage[makeroom]{cancel} %% for strike outs
%\usepackage{mathtools} %% for pretty underbrace % Breaks Ximera
%\usepackage{multicol}
\usepackage{pgffor} %% required for integral for loops



%% http://tex.stackexchange.com/questions/66490/drawing-a-tikz-arc-specifying-the-center
%% Draws beach ball
\tikzset{pics/carc/.style args={#1:#2:#3}{code={\draw[pic actions] (#1:#3) arc(#1:#2:#3);}}}



\usepackage{array}
\setlength{\extrarowheight}{+.1cm}
\newdimen\digitwidth
\settowidth\digitwidth{9}
\def\divrule#1#2{
\noalign{\moveright#1\digitwidth
\vbox{\hrule width#2\digitwidth}}}






\DeclareMathOperator{\arccot}{arccot}
\DeclareMathOperator{\arcsec}{arcsec}
\DeclareMathOperator{\arccsc}{arccsc}

















%%This is to help with formatting on future title pages.
\newenvironment{sectionOutcomes}{}{}


\title[Dig-In:]{Falling objects}

\begin{document}
\begin{abstract}
  We study a special type of differential equation.
\end{abstract}
\maketitle

A \textbf{differential equation} is simply an equation with a derivative in it like this:
\[
f'(x) = k f(x).
\]
When a mathematician solves a differential equation, they are finding
a \textit{function} that satisfies the equation.


Consider a falling object. Recall that the acceleration due to gravity is about $-9.8$
m/s$^2$. Since the first derivative of the velocity function 
is the acceleration and the
second derivative of a position function 
is the acceleration, we have the differential equations
\begin{align*}
v'(t) &=  -9.8,\\
s''(t) &=  -9.8.
\end{align*}
From these simple equations, we can derive expressions for the velocity
and for the position of the object using antiderivatives.


\begin{example}
A ball is tossed into the air with an initial velocity of $15$
m/s. What is the velocity of the ball after 1 second? How about after
2 seconds?
\begin{explanation}
Knowing that the acceleration due to gravity is $-9.8$ m/s$^2$, we write
\[
v'(t) = \answer[given]{-9.8}.
\]
To solve this differential equation, take the antiderivative of both sides
\begin{align*}
\int v'(t) dt &= \int \answer[given]{-9.8} dt\\
v(t) &= \answer[given]{-9.8 t} + C.
\end{align*}
 Since it is
tossed up with an initial velocity of $15$ m/s, 
\[
\answer[given]{15} = v(0) = -9.8\cdot 0 + C,
\]
and we see that $C=\answer[given]{15}$. Therefore, $C$ represents the initial velocity of the ball.
Hence $v(t) = -9.8t + 15$. 

Now when $t=1$,
$v(1) = 5.2$ m/s, and the ball is rising, and at $t=2$, $v(2) = -4.6$ m/s,
and the ball is falling.
\end{explanation}
\end{example}

Now let's do a similar problem, but instead of finding the velocity,
we will find the position.

\begin{example}
A ball is tossed into the air with an initial velocity of $15$ m/s
from a height of 2 meters. When does the ball hit the ground?
\begin{explanation}
Knowing that the acceleration due to gravity is $-9.8$ m/s$^2$, we write
\[
v'(t) = \answer[given]{-9.8}.
\]
Start by taking the antiderivative of both sides of the equation
\begin{align*}
\int v'(t) dt &= \int \answer[given]{-9.8} dt\\
v(t) &= \answer[given]{-9.8 t} + C.
\end{align*}
Here $C$ represents the initial velocity of the ball. Since it is
tossed up with an initial velocity of $15$ m/s, $C = 15$ and 
\[
v(t) = -9.8t + 15.
\]
Since the velocity is the derivative of the position, we can write, alternatively
\[
s'(t) = -9.8t + 15.
\]
Now let's take the antiderivative again. 
\begin{align*}
\int s'(t) dt &= \int \answer[given]{-9.8t +15} dt\\
s(t) &= \answer[given]{\frac{-9.8t^2}{2} + 15t} + D.
\end{align*}
Since we know the initial height was $2$ meters, write
\[
2 = s(0) =  \frac{-9.8\cdot 0^2}{2} + 15\cdot 0 + D.
\]
Hence $s(t) = \frac{-9.8t^2}{2} + 15t + 2$. We need to know when the
ball hits the ground, this is when $s(t)=0$. Solving the equation
\[
\frac{-9.8t^2}{2} + 15t + 2 = 0
\]
we find two solutions $t\approx -0.1$ and $t\approx 3.2$. Discarding
the negative solution, we see the ball will hit the ground after
approximately $3.2$ seconds.
\end{explanation}
\end{example}

The power of calculus is that it frees us from rote memorization of
formulas and enables us to derive what we need.

\end{document}
