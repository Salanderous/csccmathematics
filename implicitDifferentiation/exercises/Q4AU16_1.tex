\documentclass{ximera}


\graphicspath{
  {./}
  {ximeraTutorial/}
  {basicPhilosophy/}
}

\newcommand{\mooculus}{\textsf{\textbf{MOOC}\textnormal{\textsf{ULUS}}}}

\usepackage{tkz-euclide}\usepackage{tikz}
\usepackage{tikz-cd}
\usetikzlibrary{arrows}
\tikzset{>=stealth,commutative diagrams/.cd,
  arrow style=tikz,diagrams={>=stealth}} %% cool arrow head
\tikzset{shorten <>/.style={ shorten >=#1, shorten <=#1 } } %% allows shorter vectors

\usetikzlibrary{backgrounds} %% for boxes around graphs
\usetikzlibrary{shapes,positioning}  %% Clouds and stars
\usetikzlibrary{matrix} %% for matrix
\usepgfplotslibrary{polar} %% for polar plots
\usepgfplotslibrary{fillbetween} %% to shade area between curves in TikZ
\usetkzobj{all}
\usepackage[makeroom]{cancel} %% for strike outs
%\usepackage{mathtools} %% for pretty underbrace % Breaks Ximera
%\usepackage{multicol}
\usepackage{pgffor} %% required for integral for loops



%% http://tex.stackexchange.com/questions/66490/drawing-a-tikz-arc-specifying-the-center
%% Draws beach ball
\tikzset{pics/carc/.style args={#1:#2:#3}{code={\draw[pic actions] (#1:#3) arc(#1:#2:#3);}}}



\usepackage{array}
\setlength{\extrarowheight}{+.1cm}
\newdimen\digitwidth
\settowidth\digitwidth{9}
\def\divrule#1#2{
\noalign{\moveright#1\digitwidth
\vbox{\hrule width#2\digitwidth}}}






\DeclareMathOperator{\arccot}{arccot}
\DeclareMathOperator{\arcsec}{arcsec}
\DeclareMathOperator{\arccsc}{arccsc}

















%%This is to help with formatting on future title pages.
\newenvironment{sectionOutcomes}{}{}


\begin{document}

\begin{exercise}

\outcome{Implicitly differentiate expressions.}
\outcome{Find the equation of the tangent line for curves that are not plots of functions.}

The curve defined by 
\[
x^2+xy+y^2=7
\]
is called a ``rotated ellipse.'' This is not a misnomer, as its graph below demonstrates.
\begin{image}
  \begin{tikzpicture}
    \begin{axis}[
            domain=-4:4, xmin =-4,xmax=4,ymax=4,ymin=-4,
            samples=1000,
            width=4in,
            height=4in,
            xtick={-4,-3,...,4},
            ytick={-4,-3,...,4},
            axis lines=center, xlabel=$x$, ylabel=$y$,
            every axis y label/.style={at=(current axis.above origin),anchor=south},
            every axis x label/.style={at=(current axis.right of origin),anchor=west},
            axis on top,
      ]
      \addplot [ultra thick,penColor,smooth,domain=0:2*pi] ({-sqrt(7)*cos(deg(x))+sqrt(7/3)*sin(deg(x))},{sqrt(7)*cos(deg(x))+sqrt(7/3)*sin(deg(x))});
    \end{axis}
  \end{tikzpicture}
\end{image}
Use implicit differentiation to find the derivative $\frac{dy}{dx}$.
\[
\frac{dy}{dx}=\answer{-\frac{2x+y}{2y+x}}
\]
\begin{exercise}
Using your answer, find the function $\ell$ whose graph is the line tangent to the curve at the point $(1,2)$.
\[
\ell(x)=\answer{-\frac{4}{5}(x-1)+2}
\]
\end{exercise}
\end{exercise}
\end{document}
