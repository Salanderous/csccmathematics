\documentclass{ximera}

\graphicspath{
  {./}
  {ximeraTutorial/}
  {basicPhilosophy/}
}

\newcommand{\mooculus}{\textsf{\textbf{MOOC}\textnormal{\textsf{ULUS}}}}

\usepackage{tkz-euclide}\usepackage{tikz}
\usepackage{tikz-cd}
\usetikzlibrary{arrows}
\tikzset{>=stealth,commutative diagrams/.cd,
  arrow style=tikz,diagrams={>=stealth}} %% cool arrow head
\tikzset{shorten <>/.style={ shorten >=#1, shorten <=#1 } } %% allows shorter vectors

\usetikzlibrary{backgrounds} %% for boxes around graphs
\usetikzlibrary{shapes,positioning}  %% Clouds and stars
\usetikzlibrary{matrix} %% for matrix
\usepgfplotslibrary{polar} %% for polar plots
\usepgfplotslibrary{fillbetween} %% to shade area between curves in TikZ
\usetkzobj{all}
\usepackage[makeroom]{cancel} %% for strike outs
%\usepackage{mathtools} %% for pretty underbrace % Breaks Ximera
%\usepackage{multicol}
\usepackage{pgffor} %% required for integral for loops



%% http://tex.stackexchange.com/questions/66490/drawing-a-tikz-arc-specifying-the-center
%% Draws beach ball
\tikzset{pics/carc/.style args={#1:#2:#3}{code={\draw[pic actions] (#1:#3) arc(#1:#2:#3);}}}



\usepackage{array}
\setlength{\extrarowheight}{+.1cm}
\newdimen\digitwidth
\settowidth\digitwidth{9}
\def\divrule#1#2{
\noalign{\moveright#1\digitwidth
\vbox{\hrule width#2\digitwidth}}}






\DeclareMathOperator{\arccot}{arccot}
\DeclareMathOperator{\arcsec}{arcsec}
\DeclareMathOperator{\arccsc}{arccsc}

















%%This is to help with formatting on future title pages.
\newenvironment{sectionOutcomes}{}{}

\author{Steven Gubkin}
\license{Creative Commons 3.0 By-NC}

\outcome{Implicitly differentiate expressions}
\outcome{Find the equation of the tangent line for curves that are not plots of functions}
\outcome{Understand how changing the variable changes how we take the derivative}
\outcome{Understand the derivatives of expressions that are not functions or not solved for y}
%\outcome{Use implicit differentiation to demonstrate the power rule for rational exponents}

\begin{document}
\begin{exercise}

Consider the curve  $y^2 = x^2(x+1)$, whose graph is given below:

\begin{image}
\begin{tikzpicture}
	\begin{axis}[
            xmin=-2,xmax=2,ymin=-2,ymax=2,
            axis lines=center,
	   ticks=none,
            xlabel=$x$, ylabel=$y$,
            every axis y label/.style={at=(current axis.above origin),anchor=south},
            every axis x label/.style={at=(current axis.right of origin),anchor=west},
          ]        
          \addplot [very thick, penColor, smooth, samples=100, domain=(-2:2)] ({x^2-1},{x^3-x});
        \end{axis}
\end{tikzpicture}
\end{image}

Use implicit differentiation to write $\frac{dy}{dx}$ in terms of both $x$ and $y$.

\[
\frac{dy}{dx} = \answer{\frac{3x^2+2x}{2y}}
\]

The equation of the tangent line to this curve at the point $(3,6)$ is

\[
y = \answer{\frac{11}{4}(x-3)+6}
\]

A difficult point to analyze on this graph is the point $(0,0)$, since
it looks like two intersecting lines there.  We would like to know the
equation of these two ``tangent lines.''  Our formula for
$\frac{dy}{dx}$ provides no help, initially, since we get
$\frac{0}{0}$ when we plug $(0,0)$ into that formula.

Since the formula only breaks down at the point $(0,0)$, a way forward
is to try to take a limit of slopes as $(x,y)$ approaches $(0,0)$
along the curve.

We can solve for the ``top part'' of the curve as $y =
\sqrt{x^2(x+1)}$.  The other half of the curve is $y =
-\sqrt{x^2(x+1)}$.

Using $y = \sqrt{x^2(x+1)}$, we see that for points in the first
quadrant of the plane, we can write $\frac{dy}{dx}$ as a function of
$x$ as
\[
\frac{dy}{dx} = \answer{\frac{3x^2+2x}{2\sqrt{x^2(x+1)}}}
\]
and
\[
\lim_{x \to 0^+} \frac{dy}{dx} = 1
\]
Thus equations of the two ``tangent lines'' are $y=x$ and $y=-x$.

We could have also arrived at this result more intuitively in the following way:

In the equation $y^2 = x^2+x^3$, when $x$ is very close to $0$, all of
the terms will be very small, but the term $x^3$ will be small even
compared to $x^2$.  So the curve should look a lot like the curve $y^2
= x^2$ close to $(0,0)$.  But $y^2 = x^2$ is equivalent to
$(y-x)(y+x)=0$, which is exactly the pair of lines $y=x$ and $y=-x$.
\end{exercise}
\end{document}
