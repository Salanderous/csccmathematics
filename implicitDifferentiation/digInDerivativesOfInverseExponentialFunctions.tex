\documentclass{ximera}


\graphicspath{
  {./}
  {ximeraTutorial/}
  {basicPhilosophy/}
}

\newcommand{\mooculus}{\textsf{\textbf{MOOC}\textnormal{\textsf{ULUS}}}}

\usepackage{tkz-euclide}\usepackage{tikz}
\usepackage{tikz-cd}
\usetikzlibrary{arrows}
\tikzset{>=stealth,commutative diagrams/.cd,
  arrow style=tikz,diagrams={>=stealth}} %% cool arrow head
\tikzset{shorten <>/.style={ shorten >=#1, shorten <=#1 } } %% allows shorter vectors

\usetikzlibrary{backgrounds} %% for boxes around graphs
\usetikzlibrary{shapes,positioning}  %% Clouds and stars
\usetikzlibrary{matrix} %% for matrix
\usepgfplotslibrary{polar} %% for polar plots
\usepgfplotslibrary{fillbetween} %% to shade area between curves in TikZ
\usetkzobj{all}
\usepackage[makeroom]{cancel} %% for strike outs
%\usepackage{mathtools} %% for pretty underbrace % Breaks Ximera
%\usepackage{multicol}
\usepackage{pgffor} %% required for integral for loops



%% http://tex.stackexchange.com/questions/66490/drawing-a-tikz-arc-specifying-the-center
%% Draws beach ball
\tikzset{pics/carc/.style args={#1:#2:#3}{code={\draw[pic actions] (#1:#3) arc(#1:#2:#3);}}}



\usepackage{array}
\setlength{\extrarowheight}{+.1cm}
\newdimen\digitwidth
\settowidth\digitwidth{9}
\def\divrule#1#2{
\noalign{\moveright#1\digitwidth
\vbox{\hrule width#2\digitwidth}}}






\DeclareMathOperator{\arccot}{arccot}
\DeclareMathOperator{\arcsec}{arcsec}
\DeclareMathOperator{\arccsc}{arccsc}

















%%This is to help with formatting on future title pages.
\newenvironment{sectionOutcomes}{}{}



\outcome{Find derivatives of inverse functions in general.}
\outcome{Understand how the derivative of an inverse function relates to the original derivative.}
\outcome{Explain the formula for the derivative of the natural log function.}
\outcome{Calculate derivatives of logs in any base.}

\title[Dig-In:]{Derivatives of inverse exponential functions}

\begin{document}
\begin{abstract}
  We derive the derivatives of inverse exponential functions using
  implicit differentiation.
\end{abstract}
\maketitle


Geometrically, there is a close relationship between the plots of
$e^x$ and $\ln(x)$, they are reflections of each other over the line
$y=x$:
\begin{image}
\begin{tikzpicture}
	\begin{axis}[
            xmin=-6,xmax=6,ymin=-6,ymax=6,
            axis lines=center,
            xlabel=$x$, ylabel=$y$,
            every axis y label/.style={at=(current axis.above origin),anchor=south},
            every axis x label/.style={at=(current axis.right of origin),anchor=west},
          ]        
          \addplot [very thick, penColor, smooth, domain=(-6:6)] {e^x};
          \addplot [very thick, penColor2, samples=100, smooth, domain=(.002:6)] {ln(x)};
          \addplot [dashed, textColor, domain=(-6:6)] {x};
          \node at (axis cs:-2,1) [penColor] {$e^x$};
          \node at (axis cs:1,-2) [penColor2] {$\ln(x)$};
        \end{axis}
\end{tikzpicture}
%% \caption{A plot of $e^x$ and $\ln(x)$. Since they are inverse
%%   functions, they are reflections of each other across the line $y=x$.}
%% \label{plot:e^x lnx}
\end{image}
One may suspect that we can use the fact that $\frac{d}{dx} e^x = e^x$, to
deduce the derivative of $\ln(x)$.  We will use implicit
differentiation to exploit this relationship computationally.

\begin{theorem}[The Derivative of the Natural Logrithm]\index{derivative!of the natural logarithm}
\[
\frac{d}{dx} \ln(x) = \frac{1}{x}.
\]
\begin{explanation}
 %% To start, note that the Inverse Function Theorem assures us that this
  %% derivative actually exists.
  Recall
\[
\ln(x) = y \qquad\text{exactly when}\qquad e^y = x\qquad\text{and}\qquad x>\answer[given]{0}.
\]
Hence
\begin{align*}
e^y &= x\\
\frac{d}{dx} e^y &= \frac{d}{dx} x &\text{Differentiate both sides.}\\
e^y \frac{dy}{dx} &= 1 &\text{Implicit differentiation.}\\
\frac{dy}{dx} &= \frac{1}{e^y} = \answer[given]{\frac{1}{x}}&\text{Solve for $\frac{dy}{dx}$.}
\end{align*}
Since $y=\ln(x)$, $\frac{d}{dx} \ln(x) = \answer[given]{\frac{1}{x}}$.
\end{explanation}
\end{theorem}

\begin{question}
  Compute:
  \[
  \frac{d}{dx} \left(-\ln(\cos(x))\right)
  \begin{prompt}
    = \answer{\tan(x)}
  \end{prompt}
  \]
\end{question}



From the derivative of the natural logarithm, we can deduce another fact:

\begin{theorem}[The derivative of any logarithm]
  Let $b$ be a positive real number. Then
  \[
  \frac{d}{dx} \log_b(x) = \frac{1}{x\ln(b)}.
  \]
  \begin{explanation}
    Here we need to remember that
    \[
    \log_b(x) = \frac{\ln(x)}{\answer[given]{\ln(b)}}.
    \]
    So we may write
    \begin{align*}
      \frac{d}{dx} \log_b(x) &= \frac{d}{dx} \frac{\ln(x)}{\answer[given]{\ln(b)}}\\
      &= \answer[given]{\frac{1}{x\ln(b)}}.
    \end{align*}
  \end{explanation}
\end{theorem}

\begin{question}
  Compute:
  \[
  \frac{d}{dx} \log_7(x)
  \begin{prompt}
    = \answer{1/(x \ln(7))}
  \end{prompt}
  \]
\end{question}


We can also compute the derivative of an arbitrary exponential
function.

\begin{theorem}[The derivative of an exponential function]
  \[
  \frac{d}{dx} a^x = a^x\cdot \ln(a).
  \]
  \begin{explanation}
    Here we need to be slightly sneaky. Note
    \[
    a^x = e^{\ln(a^x)} = e^{x\ln(a)}.
    \]
    So we may write
    \begin{align*}
      \frac{d}{dx} a^x &= \frac{d}{dx} e^{x\ln(a)}\\
      &= e^{x\ln(a)} \cdot \answer[given]{\ln(a)}\\
      &= \answer[given]{a^x\cdot \ln(a)}.
    \end{align*}
  \end{explanation}
\end{theorem}

\begin{question}
  Compute:
  \[
  \frac{d}{dx} 7^x
  \begin{prompt}
    = \answer{7^x \ln(7)}
  \end{prompt}
  \]
\end{question}


\end{document}
