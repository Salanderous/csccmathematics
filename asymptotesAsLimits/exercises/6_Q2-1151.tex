\documentclass{ximera}


\graphicspath{
  {./}
  {ximeraTutorial/}
  {basicPhilosophy/}
}

\newcommand{\mooculus}{\textsf{\textbf{MOOC}\textnormal{\textsf{ULUS}}}}

\usepackage{tkz-euclide}\usepackage{tikz}
\usepackage{tikz-cd}
\usetikzlibrary{arrows}
\tikzset{>=stealth,commutative diagrams/.cd,
  arrow style=tikz,diagrams={>=stealth}} %% cool arrow head
\tikzset{shorten <>/.style={ shorten >=#1, shorten <=#1 } } %% allows shorter vectors

\usetikzlibrary{backgrounds} %% for boxes around graphs
\usetikzlibrary{shapes,positioning}  %% Clouds and stars
\usetikzlibrary{matrix} %% for matrix
\usepgfplotslibrary{polar} %% for polar plots
\usepgfplotslibrary{fillbetween} %% to shade area between curves in TikZ
\usetkzobj{all}
\usepackage[makeroom]{cancel} %% for strike outs
%\usepackage{mathtools} %% for pretty underbrace % Breaks Ximera
%\usepackage{multicol}
\usepackage{pgffor} %% required for integral for loops



%% http://tex.stackexchange.com/questions/66490/drawing-a-tikz-arc-specifying-the-center
%% Draws beach ball
\tikzset{pics/carc/.style args={#1:#2:#3}{code={\draw[pic actions] (#1:#3) arc(#1:#2:#3);}}}



\usepackage{array}
\setlength{\extrarowheight}{+.1cm}
\newdimen\digitwidth
\settowidth\digitwidth{9}
\def\divrule#1#2{
\noalign{\moveright#1\digitwidth
\vbox{\hrule width#2\digitwidth}}}






\DeclareMathOperator{\arccot}{arccot}
\DeclareMathOperator{\arcsec}{arcsec}
\DeclareMathOperator{\arccsc}{arccsc}

















%%This is to help with formatting on future title pages.
\newenvironment{sectionOutcomes}{}{}


\outcome{Discuss what it means for a limit to equal $\infty$.}
\outcome{Understand the relationship between limits and vertical asymptotes.}
\outcome{Evaluate the limit as $x$ approaches a point where there is a vertical asymptote.}
\outcome{Recognize when a limit is indicating there is a vertical asymptote.}

\author{Nela Lakos \and Kyle Parsons}

\begin{document}
Let 
\[
f(x)=\begin{cases}
\frac{x^2-x-12}{x+3} & \text{if $x<4$ and $x\ne -3$}\\
5 & \text{if $x=-3$}\\
\frac{x}{x-4} & \text{if $x>4$}.
\end{cases}
\]
Determine if the following limits exist. If they exist, compute them \textbf{analytically} using the \textbf{limit laws} and \textbf{techniques for computing limits}. If a limit \textbf{does not exist}, write `DNE' and explain why.  Do not use a table of values, a graph, or L'H\^opital's rule to justify your answer.
\begin{exercise}
  \[
  \lim_{x\to -3}f(x)=\answer{-7}
  \]
\end{exercise}
\begin{exercise}
  \[
  \lim_{x\to 4^{-}}f(x)=\answer{0}
  \]
\end{exercise}
\begin{exercise}
  \[
  \lim_{x\to 4^+}f(x)=\answer{\infty}
  \]
  \begin{exercise}
    Let's give an explanation as to why this is true. When $x$ is
    greater than $4$,
    \[
    f(x) = \answer{\frac{x}{x-4}}.
    \]
    \begin{exercise}As $x$ approaches $4$ from the right, $x-4$ is \wordChoice{\choice[correct]{positive}\choice{negative}\choice{zero}} and approaching zero while at the same time, $x$ approaches $\answer{4}$. \begin{exercise} Therefore, as $x$ approaches $4$ from the right, $f(x)$ is positive and grows \wordChoice{\choice[correct]{arbitrarily large}\choice{arbitrarily small}}. \begin{exercise} Hence, 
    \[
    \lim_{x\to 4}f(x)=\answer{DNE}.
    \]
    \end{exercise}
    \end{exercise}
    \end{exercise}
    \end{exercise}   
  \begin{exercise}
  \[
  \lim_{x\to-\infty}f(x)=\answer{-\infty}
  \]
  \end{exercise} 
  \begin{exercise}
  \[
  \lim_{x\to\infty}f(x)=\answer{1}
  \]
  \end{exercise}
  \begin{exercise}
    List the vertical asymptotes of $f$ from least to greatest: $x=\answer{4}$.
  \end{exercise}
  \begin{exercise}
    List all the horizontal asymptotes of $f$ from least to greatest: $y=\answer{1}$.
  \end{exercise}
\end{exercise}
\end{document}
