\documentclass{ximera}


\graphicspath{
  {./}
  {ximeraTutorial/}
  {basicPhilosophy/}
}

\newcommand{\mooculus}{\textsf{\textbf{MOOC}\textnormal{\textsf{ULUS}}}}

\usepackage{tkz-euclide}\usepackage{tikz}
\usepackage{tikz-cd}
\usetikzlibrary{arrows}
\tikzset{>=stealth,commutative diagrams/.cd,
  arrow style=tikz,diagrams={>=stealth}} %% cool arrow head
\tikzset{shorten <>/.style={ shorten >=#1, shorten <=#1 } } %% allows shorter vectors

\usetikzlibrary{backgrounds} %% for boxes around graphs
\usetikzlibrary{shapes,positioning}  %% Clouds and stars
\usetikzlibrary{matrix} %% for matrix
\usepgfplotslibrary{polar} %% for polar plots
\usepgfplotslibrary{fillbetween} %% to shade area between curves in TikZ
\usetkzobj{all}
\usepackage[makeroom]{cancel} %% for strike outs
%\usepackage{mathtools} %% for pretty underbrace % Breaks Ximera
%\usepackage{multicol}
\usepackage{pgffor} %% required for integral for loops



%% http://tex.stackexchange.com/questions/66490/drawing-a-tikz-arc-specifying-the-center
%% Draws beach ball
\tikzset{pics/carc/.style args={#1:#2:#3}{code={\draw[pic actions] (#1:#3) arc(#1:#2:#3);}}}



\usepackage{array}
\setlength{\extrarowheight}{+.1cm}
\newdimen\digitwidth
\settowidth\digitwidth{9}
\def\divrule#1#2{
\noalign{\moveright#1\digitwidth
\vbox{\hrule width#2\digitwidth}}}






\DeclareMathOperator{\arccot}{arccot}
\DeclareMathOperator{\arcsec}{arcsec}
\DeclareMathOperator{\arccsc}{arccsc}

















%%This is to help with formatting on future title pages.
\newenvironment{sectionOutcomes}{}{}


\outcome{Discuss what it means for a limit to equal $\infty$.}
\outcome{Understand the relationship between limits and vertical asymptotes.}
\outcome{Evaluate the limit as $x$ approaches a point where there is a vertical asymptote.}
\outcome{Recognize when a limit is indicating there is a vertical asymptote.}

\author{Nela Lakos \and Kyle Parsons}

\begin{document}
\begin{exercise}

Let $f$ be defined by
\[
f(x) = 
\begin{cases} 
\frac{x-2}{x^2-3x+2} & \text{if } x\neq 2 \text{ and } x\neq 1 \\
C & \text{if } x=2
   \end{cases}
\]
where $C$ is some constant.

Determine if the following limits exist and if they do, compute them analytically using the limit laws and techniques for computing limits.  Possible answers are any number, $\infty$, $-\infty$, and $DNE$.

\begin{exercise}
\[
\lim_{x\to1^-}f(x) = \answer{-\infty}
\]
\begin{exercise}
\[
\lim_{x\to2}f(x) = \answer{1}
\]
\end{exercise}
\end{exercise}

\begin{exercise}
In order for $f$ to be continuous at 2, we must have
\[
\lim_{x\to\answer{2}}f(x) = f\left(\answer{2}\right) = C
\]
And so we know that $C=\answer{1}$ if $f$ is continuous.
\end{exercise}

\begin{exercise}
By what we calculated above, the only vertical asymptote of $f$ is $x=\answer1$.
\end{exercise}

\begin{exercise}
\begin{align*}
\lim_{x\to-\infty}f(x) &= \answer{0}\\
\lim_{x\to\infty}f(x) &= \answer{0}
\end{align*}
So the only horizontal asymptote of $f$ is $y=\answer{0}$.
\end{exercise}

\end{exercise}
\end{document}