\documentclass{ximera}


\graphicspath{
  {./}
  {ximeraTutorial/}
  {basicPhilosophy/}
}

\newcommand{\mooculus}{\textsf{\textbf{MOOC}\textnormal{\textsf{ULUS}}}}

\usepackage{tkz-euclide}\usepackage{tikz}
\usepackage{tikz-cd}
\usetikzlibrary{arrows}
\tikzset{>=stealth,commutative diagrams/.cd,
  arrow style=tikz,diagrams={>=stealth}} %% cool arrow head
\tikzset{shorten <>/.style={ shorten >=#1, shorten <=#1 } } %% allows shorter vectors

\usetikzlibrary{backgrounds} %% for boxes around graphs
\usetikzlibrary{shapes,positioning}  %% Clouds and stars
\usetikzlibrary{matrix} %% for matrix
\usepgfplotslibrary{polar} %% for polar plots
\usepgfplotslibrary{fillbetween} %% to shade area between curves in TikZ
\usetkzobj{all}
\usepackage[makeroom]{cancel} %% for strike outs
%\usepackage{mathtools} %% for pretty underbrace % Breaks Ximera
%\usepackage{multicol}
\usepackage{pgffor} %% required for integral for loops



%% http://tex.stackexchange.com/questions/66490/drawing-a-tikz-arc-specifying-the-center
%% Draws beach ball
\tikzset{pics/carc/.style args={#1:#2:#3}{code={\draw[pic actions] (#1:#3) arc(#1:#2:#3);}}}



\usepackage{array}
\setlength{\extrarowheight}{+.1cm}
\newdimen\digitwidth
\settowidth\digitwidth{9}
\def\divrule#1#2{
\noalign{\moveright#1\digitwidth
\vbox{\hrule width#2\digitwidth}}}






\DeclareMathOperator{\arccot}{arccot}
\DeclareMathOperator{\arcsec}{arcsec}
\DeclareMathOperator{\arccsc}{arccsc}

















%%This is to help with formatting on future title pages.
\newenvironment{sectionOutcomes}{}{}


\author{Jim Talamo}
\license{Creative Commons 3.0 By-bC}


\outcome{}


\begin{document}
\begin{exercise}
For the function $f(x) = \frac{x}{1-x^2}$, we will find the fifth degree Taylor polynomial centered at $x=0$ two ways.  First note that we can construct this polynomial by using the Taylor Series centered at $x=0$ for a known function and the rules for sums, products, and compositions.

Which function below has a Taylor Series that would be helpful to start?
\begin{multipleChoice}
\choice{$\sin(x)$}
\choice{$\cos(x)$}
\choice{$e^x$}
\choice[correct]{$\frac{1}{1-x}$}
\end{multipleChoice}






%%%%%%%%%%%%%%%%%%%%%%
\begin{exercise}
\begin{exercise}
Let's write out several terms in the series for $\frac{1}{1-x}$:

\[
\frac{1}{1-x} = \answer{1} + \answer{1}x+\answer{1}x^2+\answer{1}x^3+\answer{1}x^4 + \ldots
\]
(We may not have exhibited enough terms, but we can always exhibit more if the above is not sufficient.)

\begin{exercise}
We can thus use the rules for composition to find the first several terms in the series for this function:

\begin{align*}
\frac{1}{1-x} &= 1+x+x^2+x^3 +x^4+ \ldots \\
 \frac{1}{1-(x^2)} &= 1+(\answer{x^2})+(\answer{x^2})^2+(\answer{x^2})^3 +(\answer{x^2})^4 + \ldots \\
&= \answer{1+x^2+x^4+x^6+x^8} + \ldots \textrm{ (simplify your answer from the line above)} \\
\end{align*}

\begin{exercise}
Now, using the rule for products:

\[
f(x) = \frac{x}{1-x^2} = x \cdot \frac{1}{1-x^2} = x \cdot (1+x^2+x^4+x^6+x^8 + \ldots) = \answer{x+x^3+x^5+x^7+x^9} + \ldots
\]

Thus, the fifth degree Taylor polynomial centered at $x=0$ for $f(x) = \frac{x}{1-x^2}$ is:

\[
p_5(x) =  \answer{x+x^3+x^5}
\] 

\begin{hint}
Recall that to find the fifth degree Taylor polynomial, you should write out the sum of the terms up to and including $x^5$.
\end{hint}

\end{exercise}
\end{exercise}
\end{exercise}
%%%%%%%%%%%%%%%%%%%%%%

%%%%%%%%%%%%%%%%%%%%%%
\begin{exercise}
We can also work in summation notation.  First, note that the Taylor series centered at $x=0$ in summation notation for $\frac{1}{1-x}$ is:

\begin{multipleChoice}
\choice{$\sum_{k=0}^{\infty} \frac{1}{k!}x^k$}
\choice{$\sum_{k=0}^{\infty} \frac{(-1)^k}{(2k)!}x^{2k}$}
\choice{$\sum_{k=0}^{\infty} \frac{(-1)^k}{(2k+1)!}x^{2k+1}$}
\choice[correct]{$\sum_{k=0}^{\infty} x^k$}
\choice{None of these}
\end{multipleChoice}


We can thus use the rules for composition to find the first several terms in the series for this function:

\begin{align*}
\frac{1}{1-x} &= \sum_{k=0}^{\infty} x^k  \\
 \frac{1}{1-(x^2)} &=  \sum_{k=0}^{\infty} \left(\answer{x^2}\right)^k \\
&=  \sum_{k=0}^{\infty}x^{\answer{2k}}  \textrm{ (simplify your answer from the line above)} \\
\end{align*}

\begin{exercise}
Now, using the rule for products:

\[
f(x) = \frac{x}{1-x^2} = x \cdot \frac{1}{1-x^2} = x \cdot  \sum_{k=0}^{\infty}x^{2k}  = \sum_{k=0}^{\infty}x^{\answer{2k+1}}
\]

To find the fifth degree Taylor polynomial centered at $x=0$ for $f(x) = \frac{x}{1-x^2}$, write out the series above up to and including $x^5$.  The result is:   

\[
p_5(x) =  \answer{x+x^3+x^5}
\] 

Does this match the result before?

\begin{multipleChoice}
\choice{No}
\choice[correct]{Yes}
\end{multipleChoice}

\end{exercise}
\end{exercise}
%%%%%%%%%%%%%%%%%%%%%%

Note that either option produces the same result; working in summation notation is just another way to \emph{notate} the exact same procedure.  You always have the option of working with the first several explicit terms in a Taylor series or working with the series in summation notation (unless you're specifically prompted to work using summation notation).  

Generally, when we only need the sum of the first several terms, it's
a better idea to work with the explicit sum of the first several terms,
whereas when we will need many terms (such as when we compute higher
order derivatives of the function at the center of the series), it's
best to work in summation notation because we can write out as many
terms as we need easily.

\end{exercise}
\end{exercise}
\end{document}
