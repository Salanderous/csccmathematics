\documentclass{ximera}


\graphicspath{
  {./}
  {ximeraTutorial/}
  {basicPhilosophy/}
}

\newcommand{\mooculus}{\textsf{\textbf{MOOC}\textnormal{\textsf{ULUS}}}}

\usepackage{tkz-euclide}\usepackage{tikz}
\usepackage{tikz-cd}
\usetikzlibrary{arrows}
\tikzset{>=stealth,commutative diagrams/.cd,
  arrow style=tikz,diagrams={>=stealth}} %% cool arrow head
\tikzset{shorten <>/.style={ shorten >=#1, shorten <=#1 } } %% allows shorter vectors

\usetikzlibrary{backgrounds} %% for boxes around graphs
\usetikzlibrary{shapes,positioning}  %% Clouds and stars
\usetikzlibrary{matrix} %% for matrix
\usepgfplotslibrary{polar} %% for polar plots
\usepgfplotslibrary{fillbetween} %% to shade area between curves in TikZ
\usetkzobj{all}
\usepackage[makeroom]{cancel} %% for strike outs
%\usepackage{mathtools} %% for pretty underbrace % Breaks Ximera
%\usepackage{multicol}
\usepackage{pgffor} %% required for integral for loops



%% http://tex.stackexchange.com/questions/66490/drawing-a-tikz-arc-specifying-the-center
%% Draws beach ball
\tikzset{pics/carc/.style args={#1:#2:#3}{code={\draw[pic actions] (#1:#3) arc(#1:#2:#3);}}}



\usepackage{array}
\setlength{\extrarowheight}{+.1cm}
\newdimen\digitwidth
\settowidth\digitwidth{9}
\def\divrule#1#2{
\noalign{\moveright#1\digitwidth
\vbox{\hrule width#2\digitwidth}}}






\DeclareMathOperator{\arccot}{arccot}
\DeclareMathOperator{\arcsec}{arcsec}
\DeclareMathOperator{\arccsc}{arccsc}

















%%This is to help with formatting on future title pages.
\newenvironment{sectionOutcomes}{}{}


\author{Jim Talamo}
\license{Creative Commons 3.0 By-bC}


\outcome{}


\begin{document}
\begin{exercise}
For the function $f(x) = \frac{2x^2}{1-4x^2}$, follow the steps indicated below to find $f^{(62)}(0)$.

First, let's exhibit the Taylor series centered at $x=\answer{0}$ for $f(x)$ by using known series as well as the rules for sums, products, and compositions.

Note that we can write:

\[
 \frac{2x^2}{1-4x^2} =  2x^2 \cdot \frac{1}{1-4x^2}
\]

We thus have a product.  

The Taylor series centered at $x=0$ for $2x^2$ is:
\begin{multipleChoice}
\choice[correct]{$2x^2$ since this is already a polynomial centered at $x=0$.}
\choice{something that requires more computation than what the above step suggests.}
\end{multipleChoice}

For the Taylor series centered at $x=0$ of $ \frac{1}{1-4x^2}$, the more efficient way to compute it is:
\begin{multipleChoice}
\choice[correct]{Use the known series for $\frac{1}{1-x}$ and the rule for composing series.}
\choice{Use the definition to find the first several coefficients and look for a pattern.}
\end{multipleChoice}

The Taylor series for $\frac{1}{1-x}$ centered at $x=0$ is:

\[
\frac{1}{1-x} = \sum_{k=0}^{\infty} \answer{1}x^k.
\]
We can thus find the series for  $ \frac{1}{1-4x^2}$ by substituting $4x^2$ in for $x$ in the series above:

\[
\frac{1}{1-4x^2} = \sum_{k=0}^{\infty} \left(\answer{4x^2}\right)^k.
\]

Simplifying this gives:

\[
\frac{1}{1-4x^2} = \sum_{k=0}^{\infty} 4^{\answer{k}} \cdot x^{\answer{2k}}.
\]

\begin{exercise}
The series in summation notation for $f(x) = \frac{2x^2}{1-4x^2}$ is thus:

\begin{align*}
  \frac{2x^2}{1-4x^2} &=2x^2 \cdot \frac{1}{1-4x^2}\\
  &= 2x^2 \cdot  \sum_{k=0}^{\infty} 4^k x^{2k}
\end{align*}

We eventually need to find a way to take $62$ derivatives of this.  In order to do this, we should:
\begin{selectAll}
\choice{Use the fact that $62$ derivatives of $2x^2$ gives $0$, so the derivative of the product above is $0$.}
\choice[correct]{Simplify the above then use the relationship between the coefficients of the Taylor series and the derivatives of the function it represents.}
\end{selectAll}

Indeed, simplifying the above gives:

\begin{align*}
  \frac{2x^2}{1-4x^2} &=  2x^2 \cdot\sum_{k=0}^{\infty} 4^k \cdot  x^{2k}\\
  &= \sum_{k=0}^{\infty} 2\cdot 4^k \cdot  x^{\answer{2k+2}}.
\end{align*}


\begin{exercise}
Now, we can use the relationship that if $f(x) = \sum_{k=0}^{\infty} a_k(x-c)^k$, then $a_n = \frac{f^{(n)}(c)}{n!}$.

Here, $c= \answer{0}$, and to find $f^{(62)}(0)$, we have $n=\answer{62}$. 

We now need to find $a_{62}$.

\begin{exercise}
First, note that $a_{62}$ is:
\begin{multipleChoice}
\choice{found by plugging $k=62$ into the summand, so $a_{62} = 2\cdot4^{62}$.}
\choice{found by plugging $k=62$ into the summand, so $a_{62} = 2\cdot4^{62}x^{126}$.}
\choice[correct]{the coefficient of $x^{62}$.}
\end{multipleChoice}

To find this, note that $x^{2k+2} = x^{62}$ when $k=\answer{30}$.  

Thus:

\[
a_{62} = 2 \cdot 4^{k} \bigg|_{k=30} = 2 \cdot 4^{\answer{30}}.
\]

and the formula $a_{62} = \frac{f^{(62)}(0)}{62!}$ gives that:

\[
f^{(62)}(0) = \answer{2 \cdot 4^{30} \cdot 62!}
\]
\end{exercise}
\end{exercise}
\end{exercise}
\end{exercise}

\end{document}
