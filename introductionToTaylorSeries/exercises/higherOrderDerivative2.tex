\documentclass{ximera}


\graphicspath{
  {./}
  {ximeraTutorial/}
  {basicPhilosophy/}
}

\newcommand{\mooculus}{\textsf{\textbf{MOOC}\textnormal{\textsf{ULUS}}}}

\usepackage{tkz-euclide}\usepackage{tikz}
\usepackage{tikz-cd}
\usetikzlibrary{arrows}
\tikzset{>=stealth,commutative diagrams/.cd,
  arrow style=tikz,diagrams={>=stealth}} %% cool arrow head
\tikzset{shorten <>/.style={ shorten >=#1, shorten <=#1 } } %% allows shorter vectors

\usetikzlibrary{backgrounds} %% for boxes around graphs
\usetikzlibrary{shapes,positioning}  %% Clouds and stars
\usetikzlibrary{matrix} %% for matrix
\usepgfplotslibrary{polar} %% for polar plots
\usepgfplotslibrary{fillbetween} %% to shade area between curves in TikZ
\usetkzobj{all}
\usepackage[makeroom]{cancel} %% for strike outs
%\usepackage{mathtools} %% for pretty underbrace % Breaks Ximera
%\usepackage{multicol}
\usepackage{pgffor} %% required for integral for loops



%% http://tex.stackexchange.com/questions/66490/drawing-a-tikz-arc-specifying-the-center
%% Draws beach ball
\tikzset{pics/carc/.style args={#1:#2:#3}{code={\draw[pic actions] (#1:#3) arc(#1:#2:#3);}}}



\usepackage{array}
\setlength{\extrarowheight}{+.1cm}
\newdimen\digitwidth
\settowidth\digitwidth{9}
\def\divrule#1#2{
\noalign{\moveright#1\digitwidth
\vbox{\hrule width#2\digitwidth}}}






\DeclareMathOperator{\arccot}{arccot}
\DeclareMathOperator{\arcsec}{arcsec}
\DeclareMathOperator{\arccsc}{arccsc}

















%%This is to help with formatting on future title pages.
\newenvironment{sectionOutcomes}{}{}


\author{Jim Talamo}
\license{Creative Commons 3.0 By-bC}


\outcome{}


\begin{document}
\begin{exercise}
Given that the Taylor series centered at $x=0$ for $\ln(1-x)$ is $\sum_{k=1}^{\infty} -\frac{1}{k}x^k$, find $f^{(28)}(0)$ for the function $f(x) = x\ln(1+2x)$.

\[
f^{(28)}(0) =\answer{\frac{2^{27}\cdot (28)!}{27}}
\]

(If you're stuck, please use the hint)

\begin{hint}
First, let's exhibit the Taylor series centered at $x=\answer{0}$ for $f(x)$ by using known series as well as the rules for sums, products, and compositions.

Note that we can write:

\[
x\ln(1+2x) =  x \cdot \ln(1+2x)
\]

We thus have a product.  

The Taylor series centered at $x=0$ for $x$ is:
\begin{multipleChoice}
\choice[correct]{$x$ since this is already a polynomial centered at $x=0$.}
\choice{something that requires more computation than what the above step suggests.}
\end{multipleChoice}

For the Taylor series centered at $x=0$ of $\ln(1+2x)$, the more efficient way to compute it is:
\begin{multipleChoice}
\choice[correct]{Use the known series for $\ln(1-x)$ and the rule for composing series.}
\choice{Use the definition to find the first several coefficients and look for a pattern.}
\end{multipleChoice}

We can thus find the series for  $\ln(1+2x)$ by substituting $-2x$ in for $x$ (since $1+2x = 1- (\answer{-2x})$ ) in the given series for $\ln(1-x)$:

\[
\ln(1+2x) = \sum_{k=1}^{\infty} -\frac{1}{k}\left(\answer{-2x}\right)^k.
\]

Simplifying this gives:

\[
\ln(1+2x) = \sum_{k=1}^{\infty} -\frac{1}{k} \cdot (-2)^{\answer{k}} \cdot x^{\answer{k}}.
\]

\begin{question}
To simplify further, note that by using the laws of exponents, $(-2)^{k} = (-1 \cdot 2)^{k} = (-1)^{k} \cdot 2^k$. Thus, we can write:

\[
\ln(1+2x) = \sum_{k=1}^{\infty} -\frac{1}{k} \cdot (-2)^{2k} \cdot x^{2k}=\sum_{k=1}^{\infty} -\frac{(-1)^k \cdot 2^k}{k} \cdot x^k.
\]

\begin{question}
Using the fact that $-(-1)^k = (-1)(-1)^k = (-1)^1(-1)^k = (-1)^{\answer{k+1}}$, we have:

\[
\ln(1+2x) = \sum_{k=1}^{\infty} \frac{(-1)^{k+1} \cdot 2^k}{k} \cdot x^k.
\]

The series in summation notation for $f(x) =x\ln(1+2x)$ is thus:

\[
x\ln(1+2x) = x \cdot   \sum_{k=1}^{\infty}  \frac{(-1)^{k+1} \cdot 2^k}{k} \cdot x^k.
\]

We eventually need to find a way to take $28$ derivatives of this.  In order to do this, we should:
\begin{selectAll}
\choice{Use the fact that $28$ derivatives of $x$ gives $0$, so the derivative of the product above is $0$.}
\choice[correct]{Simplify the above then use the relationship between the coefficients of the Taylor series and the derivatives of the function it represents.}
\end{selectAll}

Indeed, simplifying the above gives:

\[
x\ln(1+2x) =  \sum_{k=1}^{\infty}  \frac{(-1)^{k+1} \cdot 2^k}{k} \cdot x^{\answer{k+1}}.
\]


\begin{question}
Now, we can use the relationship that if $f(x) = \sum_{k=1}^{\infty} a_k(x-c)^k$, then $a_n = \frac{f^{(n)}(c)}{n!}$.

Here, $c= \answer{0}$, and to find $f^{(28)}(0)$, we have $n=\answer{28}$. 

We now need to find $a_{28}$.

\begin{question}
First, note that $a_{28}$ is:
\begin{multipleChoice}
\choice{found by plugging $k=28$ into the summand, so $a_{28} = -\frac{2^{28}}{28} \cdot x^{29}$.}
\choice{found by plugging $k=28$ into the summand, so $a_{28} =   -\frac{2^{28}}{28} $.}
\choice[correct]{the coefficient of $x^{28}$.}
\end{multipleChoice}

To find this, note that $x^{k+1} = x^{28}$ when $k=\answer{27}$.  

\begin{question}
Thus:

\[
a_{28} =  \frac{(-1)^{k+1} \cdot 2^k}{k}  \bigg|_{k=27} = \answer{\frac{2^{27}}{27}}
\]

and the formula $a_{28} = \frac{f^{(28)}(0)}{28!}$ gives that:

\[
f^{(28)}(0) =\answer{\frac{2^{27}\cdot (28)!}{27}}
\]
\end{question}


\end{question}

\end{question}
\end{question}
\end{question}
\end{hint}
\end{exercise}
\end{document}
