\documentclass{ximera}


\graphicspath{
  {./}
  {ximeraTutorial/}
  {basicPhilosophy/}
}

\newcommand{\mooculus}{\textsf{\textbf{MOOC}\textnormal{\textsf{ULUS}}}}

\usepackage{tkz-euclide}\usepackage{tikz}
\usepackage{tikz-cd}
\usetikzlibrary{arrows}
\tikzset{>=stealth,commutative diagrams/.cd,
  arrow style=tikz,diagrams={>=stealth}} %% cool arrow head
\tikzset{shorten <>/.style={ shorten >=#1, shorten <=#1 } } %% allows shorter vectors

\usetikzlibrary{backgrounds} %% for boxes around graphs
\usetikzlibrary{shapes,positioning}  %% Clouds and stars
\usetikzlibrary{matrix} %% for matrix
\usepgfplotslibrary{polar} %% for polar plots
\usepgfplotslibrary{fillbetween} %% to shade area between curves in TikZ
\usetkzobj{all}
\usepackage[makeroom]{cancel} %% for strike outs
%\usepackage{mathtools} %% for pretty underbrace % Breaks Ximera
%\usepackage{multicol}
\usepackage{pgffor} %% required for integral for loops



%% http://tex.stackexchange.com/questions/66490/drawing-a-tikz-arc-specifying-the-center
%% Draws beach ball
\tikzset{pics/carc/.style args={#1:#2:#3}{code={\draw[pic actions] (#1:#3) arc(#1:#2:#3);}}}



\usepackage{array}
\setlength{\extrarowheight}{+.1cm}
\newdimen\digitwidth
\settowidth\digitwidth{9}
\def\divrule#1#2{
\noalign{\moveright#1\digitwidth
\vbox{\hrule width#2\digitwidth}}}






\DeclareMathOperator{\arccot}{arccot}
\DeclareMathOperator{\arcsec}{arcsec}
\DeclareMathOperator{\arccsc}{arccsc}

















%%This is to help with formatting on future title pages.
\newenvironment{sectionOutcomes}{}{}


\author{ Jason Miller}
\license{Creative Commons 3.0 By-NC}


\outcome{Find the bounded area between two curves}
\outcome{Find volume of solid described using cross sections}

\begin{document}
\begin{exercise}

The region bounded by the curves $y=x-1$, $y=\ln(x)$, and $y=1$ is revolved about the line $y=1$. 

To use the Washer Method to set up an integral or sum of integrals that would give the volume of the solid: 

  \begin{multipleChoice}
    \choice[correct]{we should integrate with respect to $x$.}
    \choice{we should integrate with respect to $y$.}
  \end{multipleChoice}

How many integrals will we need to express the volume of the solid using the Washer Method: $\answer{2}$

\begin{exercise}

Express the volume of the solid using the Washer Method method: 
\[
\int_{1}^{\answer{2}} \answer{ \pi \left( 1 - \ln(x) \right)^{2} - \pi \left( 2-x \right)^{2} } dx + \int_{\answer{2}}^{\answer{e}} \answer{ \pi \left( 1- \ln(x) \right)^{2} } dx
\]
\end{exercise}
\end{exercise}

\begin{exercise}

To use the Shell Method to set up an integral or sum of integrals that would give the volume of the solid: 

  \begin{multipleChoice}
    \choice{we should integrate with respect to $x$.}
    \choice[correct]{we should integrate with respect to $y$.}
  \end{multipleChoice}

How many integrals will we need to express the volume of the solid using the Shell Method: $\answer{1}$. 


\begin{exercise} 
The integral that gives the volume of $S$ is: 
\[
\int_{\answer{0}}^{\answer{1}} \answer{ 2\pi \left( 1-y \right) \left( e^{y}-y-1 \right)} dy
\] 

\begin{hint}
You should notice that the curves $y=\ln(x)$ and $y=x-1$ intersect when $x=1$ and the corresponding $y$-value will be the lower limit of integration.
\end{hint}
\end{exercise}
\end{exercise}
\end{document}

