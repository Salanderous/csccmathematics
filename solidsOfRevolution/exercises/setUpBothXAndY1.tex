\documentclass{ximera}


\graphicspath{
  {./}
  {ximeraTutorial/}
  {basicPhilosophy/}
}

\newcommand{\mooculus}{\textsf{\textbf{MOOC}\textnormal{\textsf{ULUS}}}}

\usepackage{tkz-euclide}\usepackage{tikz}
\usepackage{tikz-cd}
\usetikzlibrary{arrows}
\tikzset{>=stealth,commutative diagrams/.cd,
  arrow style=tikz,diagrams={>=stealth}} %% cool arrow head
\tikzset{shorten <>/.style={ shorten >=#1, shorten <=#1 } } %% allows shorter vectors

\usetikzlibrary{backgrounds} %% for boxes around graphs
\usetikzlibrary{shapes,positioning}  %% Clouds and stars
\usetikzlibrary{matrix} %% for matrix
\usepgfplotslibrary{polar} %% for polar plots
\usepgfplotslibrary{fillbetween} %% to shade area between curves in TikZ
\usetkzobj{all}
\usepackage[makeroom]{cancel} %% for strike outs
%\usepackage{mathtools} %% for pretty underbrace % Breaks Ximera
%\usepackage{multicol}
\usepackage{pgffor} %% required for integral for loops



%% http://tex.stackexchange.com/questions/66490/drawing-a-tikz-arc-specifying-the-center
%% Draws beach ball
\tikzset{pics/carc/.style args={#1:#2:#3}{code={\draw[pic actions] (#1:#3) arc(#1:#2:#3);}}}



\usepackage{array}
\setlength{\extrarowheight}{+.1cm}
\newdimen\digitwidth
\settowidth\digitwidth{9}
\def\divrule#1#2{
\noalign{\moveright#1\digitwidth
\vbox{\hrule width#2\digitwidth}}}






\DeclareMathOperator{\arccot}{arccot}
\DeclareMathOperator{\arcsec}{arcsec}
\DeclareMathOperator{\arccsc}{arccsc}

















%%This is to help with formatting on future title pages.
\newenvironment{sectionOutcomes}{}{}


\author{Jim Talamo and Jason Miller}
\license{Creative Commons 3.0 By-NC}


\outcome{Use both Washer and Shell Method to set up a volume integral}


\begin{document}
\begin{exercise}

The region $R$ lies in the first quadrant and is bounded by the curves $y=x$, $y=-x^{2}+6$ and $x=0$.  A solid is formed by revolving $R$ about the line $x=4$. 

To set up an integral or sum of integrals with respect to $x$ that would give the volume of the solid, which method should be used to find the volume?

\begin{multipleChoice}
\choice[correct]{Shell Method}
\choice{Washer Method}
\end{multipleChoice}

How many integrals with respect to $x$ will we need to express the volume of the solid? $\answer{1}$. 


\begin{exercise} 
The integral that gives the volume of the solid is: 
\[
V=\int_{\answer{0}}^{\answer{2}} \answer{ 2\pi \left( 4-x \right) \left(-x^{2}+6 -x \right)} dx
\] 


\end{exercise}


To set up an integral or sum of integrals with respect to $y$ that would give the volume of the solid, which method should be used to find the volume?

\begin{multipleChoice}
\choice{Shell Method}
\choice[correct]{Washer Method}
\end{multipleChoice}

How many integrals with respect to $y$ will we need to express the volume of the solid? $\answer{2}$. 

\begin{exercise}

The sum of integrals that gives the volume of the solid is: 
\[
V= \int_{\answer{0}}^{\answer{2}} \answer{ \pi 16- \pi \left( 4-y \right)^{2} } dy + \int_{\answer{2}}^{\answer{6}} \answer{ \pi 16 - \pi \left(4 -\sqrt{6-y} \right)^{2}} dy
\]

\end{exercise}

\end{exercise}
\end{document}
