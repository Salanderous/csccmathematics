\documentclass{ximera}

\graphicspath{
  {./}
  {ximeraTutorial/}
  {basicPhilosophy/}
}

\newcommand{\mooculus}{\textsf{\textbf{MOOC}\textnormal{\textsf{ULUS}}}}

\usepackage{tkz-euclide}\usepackage{tikz}
\usepackage{tikz-cd}
\usetikzlibrary{arrows}
\tikzset{>=stealth,commutative diagrams/.cd,
  arrow style=tikz,diagrams={>=stealth}} %% cool arrow head
\tikzset{shorten <>/.style={ shorten >=#1, shorten <=#1 } } %% allows shorter vectors

\usetikzlibrary{backgrounds} %% for boxes around graphs
\usetikzlibrary{shapes,positioning}  %% Clouds and stars
\usetikzlibrary{matrix} %% for matrix
\usepgfplotslibrary{polar} %% for polar plots
\usepgfplotslibrary{fillbetween} %% to shade area between curves in TikZ
\usetkzobj{all}
\usepackage[makeroom]{cancel} %% for strike outs
%\usepackage{mathtools} %% for pretty underbrace % Breaks Ximera
%\usepackage{multicol}
\usepackage{pgffor} %% required for integral for loops



%% http://tex.stackexchange.com/questions/66490/drawing-a-tikz-arc-specifying-the-center
%% Draws beach ball
\tikzset{pics/carc/.style args={#1:#2:#3}{code={\draw[pic actions] (#1:#3) arc(#1:#2:#3);}}}



\usepackage{array}
\setlength{\extrarowheight}{+.1cm}
\newdimen\digitwidth
\settowidth\digitwidth{9}
\def\divrule#1#2{
\noalign{\moveright#1\digitwidth
\vbox{\hrule width#2\digitwidth}}}






\DeclareMathOperator{\arccot}{arccot}
\DeclareMathOperator{\arcsec}{arcsec}
\DeclareMathOperator{\arccsc}{arccsc}

















%%This is to help with formatting on future title pages.
\newenvironment{sectionOutcomes}{}{}

\author{Jim Talamo}
\license{Creative Commons 3.0 By-NC}
\outcome{Set up and evaluate a volume integral using the Washer Method}
\begin{document}
\begin{exercise}

The region $R$ bounded by $y=\sin(x)$, $y=\cos(x)$, $x=-\frac{3\pi}{4}$ and $x=\frac{\pi}{4}$. is shown below:
	
\begin{image}
\begin{tikzpicture}
\begin{axis}[
            domain=-10:10, ymax=1.4,xmax=1.3, ymin=-1.4, xmin=-2.8,
            axis lines =center, xlabel=$x$, ylabel=$y$,
            every axis y label/.style={at=(current axis.above origin),anchor=south},
            every axis x label/.style={at=(current axis.right of origin),anchor=west},
            axis on top,
          ]
          
\addplot [draw=penColor,very thick,smooth,samples=100] {sin(deg(x))};
\addplot [draw=penColor2,very thick,smooth,samples=100] {cos(deg(x))};

           
\addplot [name path=A,domain=-3*pi/4:pi/4,draw=none,samples=100] {sin(deg(x))};   
\addplot [name path=B,domain=-3*pi/4:pi/4,draw=none,samples=100] {cos(deg(x))};
\addplot [fillp] fill between[of=A and B];

	\addplot [draw=penColor5,very thick,dashed] coordinates {(-10,-1.01)(10,-1.01)};
       
          
          \node at (axis cs:2,-2) [penColor] {$y=2x-3$};
          \node at (axis cs:2,7) [penColor2] {$y=\frac{9}{x}$};
          \node at (axis cs:.6,-1.2) [penColor5] {$y=-1$};
        \end{axis}
\end{tikzpicture}
\end{image}

\begin{exercise}
How many integrals are needed to express the volume of the solid of revolution using the Washer Method? 
\begin{multipleChoice}
\choice[correct]{One integral is needed.}
\choice{Two integrals are needed.}
\choice{More than two integrals are needed.}
\end{multipleChoice}

In order to use the Washer Method, we should:
\begin{multipleChoice}
\choice[correct]{integrate with respect to $x$.}
\choice{integrate with respect to $y$.}
\end{multipleChoice} 

An integral that gives the volume of the solid of revolution is:

\[
V=\int_{x=\answer{-3 \pi/4}}^{x=\answer{\pi/4}} \answer{\pi \left((\cos(x)+1)^2-(\sin(x)+1)^2\right)} dx 
\]

\begin{exercise}
Evaluating the integral gives that the volume of the solid of revolution is $\answer{4 \pi \sqrt{2}}$ cubic units.
\end{exercise}
\end{exercise}
\end{exercise}
\end{document}