\documentclass{ximera}

\graphicspath{
  {./}
  {ximeraTutorial/}
  {basicPhilosophy/}
}

\newcommand{\mooculus}{\textsf{\textbf{MOOC}\textnormal{\textsf{ULUS}}}}

\usepackage{tkz-euclide}\usepackage{tikz}
\usepackage{tikz-cd}
\usetikzlibrary{arrows}
\tikzset{>=stealth,commutative diagrams/.cd,
  arrow style=tikz,diagrams={>=stealth}} %% cool arrow head
\tikzset{shorten <>/.style={ shorten >=#1, shorten <=#1 } } %% allows shorter vectors

\usetikzlibrary{backgrounds} %% for boxes around graphs
\usetikzlibrary{shapes,positioning}  %% Clouds and stars
\usetikzlibrary{matrix} %% for matrix
\usepgfplotslibrary{polar} %% for polar plots
\usepgfplotslibrary{fillbetween} %% to shade area between curves in TikZ
\usetkzobj{all}
\usepackage[makeroom]{cancel} %% for strike outs
%\usepackage{mathtools} %% for pretty underbrace % Breaks Ximera
%\usepackage{multicol}
\usepackage{pgffor} %% required for integral for loops



%% http://tex.stackexchange.com/questions/66490/drawing-a-tikz-arc-specifying-the-center
%% Draws beach ball
\tikzset{pics/carc/.style args={#1:#2:#3}{code={\draw[pic actions] (#1:#3) arc(#1:#2:#3);}}}



\usepackage{array}
\setlength{\extrarowheight}{+.1cm}
\newdimen\digitwidth
\settowidth\digitwidth{9}
\def\divrule#1#2{
\noalign{\moveright#1\digitwidth
\vbox{\hrule width#2\digitwidth}}}






\DeclareMathOperator{\arccot}{arccot}
\DeclareMathOperator{\arcsec}{arcsec}
\DeclareMathOperator{\arccsc}{arccsc}

















%%This is to help with formatting on future title pages.
\newenvironment{sectionOutcomes}{}{}

\author{Bart Snapp}
\license{Creative Commons 3.0 By-NC}
\outcome{Set up and evaluate a volume integral using the Washer Method}
\begin{document}

\begin{exercise} 
 Find the volume of the object generated by rotating the area bounded
  by $x^2$, the line $x= 1$, and $x=3$around the line $x=3$.
  
The volume of the solid of revolution is $\answer[given]{12\pi}$ cubic units.

\begin{hint}
The solid of revolution is shown below:

  \begin{image}
    \begin{tikzpicture}[
        declare function = {f(\x) = pow(\x,2);} ]
      \begin{axis}[
          xmin =0,xmax=6,ymax=10,ymin=-1,
          axis lines=center, xlabel=$x$, ylabel=$y$,
          every axis y label/.style={at=(current axis.above origin),anchor=south},
          every axis x label/.style={at=(current axis.right of origin),anchor=west},
          axis on top,
          clip=false,
        ]
        \addplot [fill1,fill=fill1,domain=1:3, smooth] {f(x)}\closedcycle;
        \addplot [fill1,fill=fill1,domain=3:5, smooth] {f(-x+6)}\closedcycle;

        \addplot [penColor,very thick,domain=1:3, smooth] {f(x)};
        \addplot [penColor,very thick,domain=3:5, smooth] {f(-x+6)};

        \draw[penColor,very thick,fill=fill1] (axis cs:3,0) ellipse (200 and 10);

        \draw[penColor,very thick] (axis cs:3,1) ellipse (200 and 10);

        \addplot [penColor,very thick] plot coordinates {(1,0) (1,1)};

        \addplot [penColor,very thick] plot coordinates {(5,0) (5,1)};
        
        \draw[penColor,very thick] (axis cs:3,4) ellipse (100 and 5);
      \end{axis}
    \end{tikzpicture}
  \end{image}
  
  
    To start, note that our solid consists of \textit{two} pieces, a
    cylinder:
  \begin{image}
    \begin{tikzpicture}[
        declare function = {f(\x) = pow(\x,2);} ]
      \begin{axis}[
          xmin =0,xmax=6,ymax=10,ymin=-1,
          axis lines=center, xlabel=$x$, ylabel=$y$,
          every axis y label/.style={at=(current axis.above origin),anchor=south},
          every axis x label/.style={at=(current axis.right of origin),anchor=west},
          axis on top,
          clip=false,
        ]
        \addplot [draw=none, fill=fill1,very thick] plot coordinates {(1,0) (1,1) (5,1) (5,0)}\closedcycle;
        \draw[penColor,very thick,fill=fill1] (axis cs:3,0) ellipse (200 and 10);

        \draw[penColor,very thick,fill=fill1] (axis cs:3,1) ellipse (200 and 10);

        \addplot [penColor,very thick] plot coordinates {(1,0) (1,1)};

        \addplot [penColor,very thick] plot coordinates {(5,0) (5,1)};
        
      \end{axis}
    \end{tikzpicture}
  \end{image}
  and a pointy-shape:
  \begin{image}
    \begin{tikzpicture}[
        declare function = {f(\x) = pow(\x,2);} ]
      \begin{axis}[
          xmin =0,xmax=6,ymax=10,ymin=-1,
          axis lines=center, xlabel=$x$, ylabel=$y$,
          every axis y label/.style={at=(current axis.above origin),anchor=south},
          every axis x label/.style={at=(current axis.right of origin),anchor=west},
          axis on top,
          clip=false,
        ]
        \addplot [fill1,fill=fill1,domain=1:3, smooth] {f(x)}\closedcycle;
        \addplot [fill1,fill=fill1,domain=3:5, smooth] {f(-x+6)}\closedcycle;

        

        \addplot [penColor,very thick,domain=1:3, smooth] {f(x)};
        \addplot [penColor,very thick,domain=3:5, smooth] {f(-x+6)};


        \draw[draw=none,very thick,fill=fill1] (axis cs:3,1) ellipse (200 and 10);
        
        \draw[penColor,very thick] (axis cs:3,4) ellipse (100 and 5);

        \addplot [draw=white,fill=white,domain=1:5, smooth] {(-sqrt(4-(x-3)^2)+2)/2}\closedcycle;

        \draw[penColor,very thick] (axis cs:3,1) ellipse (200 and 10);
      \end{axis}
    \end{tikzpicture}
  \end{image}
  The volume of the cylinder is
  \[
  \text{Volume of cylinder}=\answer[given]{4\pi}
  \]
  since the radius is $2$ and the height is $1$.  On the other hand,
  to compute the volume of the pointy-shape, let's consider the
  following cross-sections:
      \begin{image}
    \begin{tikzpicture}[
        declare function = {f(\x) = pow(\x,2);} ]
      \begin{axis}[
          xmin =0,xmax=6,ymax=10,ymin=-1,
          axis lines=center, xlabel=$x$, ylabel=$y$,
          every axis y label/.style={at=(current axis.above origin),anchor=south},
          every axis x label/.style={at=(current axis.right of origin),anchor=west},
          axis on top,
          clip=false,
        ]
%        \addplot [fill1,fill=fill1,domain=1:3, smooth] {f(x)}\closedcycle;
%        \addplot [fill1,fill=fill1,domain=3:5, smooth] {f(-x+6)}\closedcycle;

        \addplot [penColor,very thick,domain=1:3, smooth] {f(x)};
        \addplot [penColor,very thick,domain=3:5, smooth] {f(-x+6)};
        
        \draw[penColor,very thick] (axis cs:3,1) ellipse (200 and 10);
         
        \draw[penColor,very thick,fill=fill1] (axis cs:3,3.6) ellipse (100 and 5);
        \draw[penColor,very thick,fill=fill1] (axis cs:3,4) ellipse (100 and 5);
         
        \draw[decoration={brace,mirror,raise=.1cm},decorate,thin] (axis cs:4,3.4)--(axis cs:4,4.2);  
        \node[anchor=west] at (axis cs:4.2,3.8) {$\Delta y$};
      \end{axis}
    \end{tikzpicture}
      \end{image}
      Where each radius of the cross-section is given (in terms of $y$) by
      \[
      \text{radius} = \answer[given]{3-\sqrt{y}}.
      \]
      Hence
      \[
      \Delta V = \pi (3-\sqrt{y})^2 \Delta y
      \]
      and so the volume of the pointy-shape is
      \begin{align*}
        \int_{\answer[given]{1}}^{\answer[given]{9}} \pi (3-\sqrt{y})^2 dy
        &= \pi \int_{\answer[given]{1}}^{\answer[given]{9}} 9-6\sqrt{y}+y dy\\
        &=\pi\cdot \bigg[\answer[given]{9y-4y^{3/2} + y^2/2}\bigg]_{\answer[given]{1}}^{\answer[given]{9}}\\
        &=\pi\cdot \answer[given]{8}.
      \end{align*}
      Thus the volume of the complete object is $\answer[given]{12\pi}$.
\end{hint}

\end{exercise}

\end{document}