\documentclass{ximera}

\graphicspath{
  {./}
  {ximeraTutorial/}
  {basicPhilosophy/}
}

\newcommand{\mooculus}{\textsf{\textbf{MOOC}\textnormal{\textsf{ULUS}}}}

\usepackage{tkz-euclide}\usepackage{tikz}
\usepackage{tikz-cd}
\usetikzlibrary{arrows}
\tikzset{>=stealth,commutative diagrams/.cd,
  arrow style=tikz,diagrams={>=stealth}} %% cool arrow head
\tikzset{shorten <>/.style={ shorten >=#1, shorten <=#1 } } %% allows shorter vectors

\usetikzlibrary{backgrounds} %% for boxes around graphs
\usetikzlibrary{shapes,positioning}  %% Clouds and stars
\usetikzlibrary{matrix} %% for matrix
\usepgfplotslibrary{polar} %% for polar plots
\usepgfplotslibrary{fillbetween} %% to shade area between curves in TikZ
\usetkzobj{all}
\usepackage[makeroom]{cancel} %% for strike outs
%\usepackage{mathtools} %% for pretty underbrace % Breaks Ximera
%\usepackage{multicol}
\usepackage{pgffor} %% required for integral for loops



%% http://tex.stackexchange.com/questions/66490/drawing-a-tikz-arc-specifying-the-center
%% Draws beach ball
\tikzset{pics/carc/.style args={#1:#2:#3}{code={\draw[pic actions] (#1:#3) arc(#1:#2:#3);}}}



\usepackage{array}
\setlength{\extrarowheight}{+.1cm}
\newdimen\digitwidth
\settowidth\digitwidth{9}
\def\divrule#1#2{
\noalign{\moveright#1\digitwidth
\vbox{\hrule width#2\digitwidth}}}






\DeclareMathOperator{\arccot}{arccot}
\DeclareMathOperator{\arcsec}{arcsec}
\DeclareMathOperator{\arccsc}{arccsc}

















%%This is to help with formatting on future title pages.
\newenvironment{sectionOutcomes}{}{}

\author{Jim Talamo}
\license{Creative Commons 3.0 By-NC}
\outcome{Think about Washer and Shell Method conceptually}
\begin{document}


\begin{exercise}
 The region bounded by $y=x$, $y=2x$, and $x=6$ is revolved about the line $x=9$.  If an integral or sum of integrals with respect to $x$ is used to compute the volume of the solid, which method should be used?
 
\begin{multipleChoice}
\choice{Washer Method}
\choice[correct]{Shell Method} 
\end{multipleChoice}

\end{exercise}

\begin{exercise}
 The region bounded by $y=\frac{1}{x+1}$, $x=0$, and $x=2$ is revolved about the line $y=-1$.  If the Washer Method is used to calculate the volume or the resulting solid, we must:
 
\begin{multipleChoice}
\choice[correct]{integrate with respect to $x$.}
\choice{integrate with respect to $y$.} 
\end{multipleChoice}

\end{exercise}

\begin{exercise}
 The region $R$ is bounded by $x=9-y^2$ and $x=-2$.  The Shell Method can be used to set up an integral with respect to $y$ that computes the resulting solid is $R$ is revolved about:
 
 
\begin{multipleChoice}
\choice{$x=10$}
\choice[correct]{$y=10$} 
\end{multipleChoice}

\end{exercise}

\begin{exercise}
 The region bounded by $y=3x$, $y=8-2x$, and $y=0$ is revolved about the line $x=-2$.  Which method should be used to express the volume with a single integral?
 
\begin{multipleChoice}
\choice[correct]{Washer Method}
\choice{Shell Method} 
\choice{both methods require a single integral}
\choice{both methods require more than one integral}
\end{multipleChoice}

\end{exercise}

\end{document}