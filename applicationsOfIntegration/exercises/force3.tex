\documentclass{ximera}


\graphicspath{
  {./}
  {ximeraTutorial/}
  {basicPhilosophy/}
}

\newcommand{\mooculus}{\textsf{\textbf{MOOC}\textnormal{\textsf{ULUS}}}}

\usepackage{tkz-euclide}\usepackage{tikz}
\usepackage{tikz-cd}
\usetikzlibrary{arrows}
\tikzset{>=stealth,commutative diagrams/.cd,
  arrow style=tikz,diagrams={>=stealth}} %% cool arrow head
\tikzset{shorten <>/.style={ shorten >=#1, shorten <=#1 } } %% allows shorter vectors

\usetikzlibrary{backgrounds} %% for boxes around graphs
\usetikzlibrary{shapes,positioning}  %% Clouds and stars
\usetikzlibrary{matrix} %% for matrix
\usepgfplotslibrary{polar} %% for polar plots
\usepgfplotslibrary{fillbetween} %% to shade area between curves in TikZ
\usetkzobj{all}
\usepackage[makeroom]{cancel} %% for strike outs
%\usepackage{mathtools} %% for pretty underbrace % Breaks Ximera
%\usepackage{multicol}
\usepackage{pgffor} %% required for integral for loops



%% http://tex.stackexchange.com/questions/66490/drawing-a-tikz-arc-specifying-the-center
%% Draws beach ball
\tikzset{pics/carc/.style args={#1:#2:#3}{code={\draw[pic actions] (#1:#3) arc(#1:#2:#3);}}}



\usepackage{array}
\setlength{\extrarowheight}{+.1cm}
\newdimen\digitwidth
\settowidth\digitwidth{9}
\def\divrule#1#2{
\noalign{\moveright#1\digitwidth
\vbox{\hrule width#2\digitwidth}}}






\DeclareMathOperator{\arccot}{arccot}
\DeclareMathOperator{\arcsec}{arcsec}
\DeclareMathOperator{\arccsc}{arccsc}

















%%This is to help with formatting on future title pages.
\newenvironment{sectionOutcomes}{}{}


\author{Jim Talamo}
\license{Creative Commons 3.0 By-NC}


\outcome{Use integrals to compute work done by a variable force}
\outcome{Develop problem solving skills}

\begin{document}
\begin{exercise}

A force described by $F(x) = e^{2x}$ is applied to move a particle from $x=0$ to $x=4$.  Let $a$ denote the $x$-value for which the work required to move the particle from $x=0$ to $x=a$ is half of the total work required to move the particle from $x=0$ to $x=4$.

Without doing any calculations, we should expect
\begin{multipleChoice}
\choice{$ a < 2$.}
\choice{$a=2$.}
\choice[correct]{$a > 2$.}
\end{multipleChoice}

\begin{exercise}
Calculate $a$.

The correct value of $a$ is $a=\answer{\frac{1}{2} \ln\left(\frac{1}{2}e^8+\frac{1}{2}\right)}$, which to 3 decimal places is: $\answer[tolerance=.01]{3.654}$.

\begin{hint}
The total work required is given by the integral:

\[
W = \int_{\answer{0}}^{\answer{4}} e^{2x} dx
\]

The total work required to move the particle to $x=a$ is given by the integral:

\[
W = \int_{\answer{0}}^{\answer{a}} e^{2x} dx
\]
(type your answer in terms of $a$)

\begin{question}
We can find the value for $a$ by solving the equation:

\[
\int_0^a e^{2x} dx = \answer{\frac{1}{2}} \int_0^4 e^{2x} dx
\]
\end{question}
\end{hint}

\end{exercise}
\end{exercise}
\end{document}
