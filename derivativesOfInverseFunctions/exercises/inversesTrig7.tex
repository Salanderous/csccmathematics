\documentclass{ximera}


\graphicspath{
  {./}
  {ximeraTutorial/}
  {basicPhilosophy/}
}

\newcommand{\mooculus}{\textsf{\textbf{MOOC}\textnormal{\textsf{ULUS}}}}

\usepackage{tkz-euclide}\usepackage{tikz}
\usepackage{tikz-cd}
\usetikzlibrary{arrows}
\tikzset{>=stealth,commutative diagrams/.cd,
  arrow style=tikz,diagrams={>=stealth}} %% cool arrow head
\tikzset{shorten <>/.style={ shorten >=#1, shorten <=#1 } } %% allows shorter vectors

\usetikzlibrary{backgrounds} %% for boxes around graphs
\usetikzlibrary{shapes,positioning}  %% Clouds and stars
\usetikzlibrary{matrix} %% for matrix
\usepgfplotslibrary{polar} %% for polar plots
\usepgfplotslibrary{fillbetween} %% to shade area between curves in TikZ
\usetkzobj{all}
\usepackage[makeroom]{cancel} %% for strike outs
%\usepackage{mathtools} %% for pretty underbrace % Breaks Ximera
%\usepackage{multicol}
\usepackage{pgffor} %% required for integral for loops



%% http://tex.stackexchange.com/questions/66490/drawing-a-tikz-arc-specifying-the-center
%% Draws beach ball
\tikzset{pics/carc/.style args={#1:#2:#3}{code={\draw[pic actions] (#1:#3) arc(#1:#2:#3);}}}



\usepackage{array}
\setlength{\extrarowheight}{+.1cm}
\newdimen\digitwidth
\settowidth\digitwidth{9}
\def\divrule#1#2{
\noalign{\moveright#1\digitwidth
\vbox{\hrule width#2\digitwidth}}}






\DeclareMathOperator{\arccot}{arccot}
\DeclareMathOperator{\arcsec}{arcsec}
\DeclareMathOperator{\arccsc}{arccsc}

















%%This is to help with formatting on future title pages.
\newenvironment{sectionOutcomes}{}{}


\outcome{Recall the meaning and properties of inverse trigonometric functions.}

\author{Nela Lakos \and Kyle Parsons}

\begin{document}
\begin{exercise}

The exact value of $\sin(\tan^{-1}(5))$ is
\[
\sin(\tan^{-1}(5)) = \answer{\frac{5}{\sqrt{26}}}.
\]

Find the derivative.
\[
\frac{d}{dx}\left[\vphantom{\frac{d}{dx}}\sin(\tan^{-1}(x))\right] = \answer{\cos(\arctan(x))\frac{1}{1+x^2}}
\]

Assuming $x>0$ we can use a right triangle to simplify $\sin(\tan^{-1}(x))$ as
\[
\sin(\tan^{-1}(x)) = \answer{\frac{x}{\sqrt{1+x^2}}}.
\]

Using the above expression to take the derivative we have
\[
\frac{d}{dx}\left[\vphantom{\frac{d}{dx}}\sin(\tan^{-1}(x))\right] = \frac{d}{dx}\left[\vphantom{\frac{d}{dx}}\answer{\frac{x}{\sqrt{1+x^2}}}\right] = \answer{\frac{1}{(1+x^2)^{\frac{3}{2}}}}.
\]

In order to see that the two expressions for the derivative of $\sin(\tan^{-1}(x))$ are the same, simplify $\cos(\tan^{-1}(x))$ as
\[
\cos(\tan^{-1}(x)) = \answer{\frac{1}{\sqrt{1+x^2}}}.
\]

\end{exercise}
\end{document}