\documentclass{ximera}


\graphicspath{
  {./}
  {ximeraTutorial/}
  {basicPhilosophy/}
}

\newcommand{\mooculus}{\textsf{\textbf{MOOC}\textnormal{\textsf{ULUS}}}}

\usepackage{tkz-euclide}\usepackage{tikz}
\usepackage{tikz-cd}
\usetikzlibrary{arrows}
\tikzset{>=stealth,commutative diagrams/.cd,
  arrow style=tikz,diagrams={>=stealth}} %% cool arrow head
\tikzset{shorten <>/.style={ shorten >=#1, shorten <=#1 } } %% allows shorter vectors

\usetikzlibrary{backgrounds} %% for boxes around graphs
\usetikzlibrary{shapes,positioning}  %% Clouds and stars
\usetikzlibrary{matrix} %% for matrix
\usepgfplotslibrary{polar} %% for polar plots
\usepgfplotslibrary{fillbetween} %% to shade area between curves in TikZ
\usetkzobj{all}
\usepackage[makeroom]{cancel} %% for strike outs
%\usepackage{mathtools} %% for pretty underbrace % Breaks Ximera
%\usepackage{multicol}
\usepackage{pgffor} %% required for integral for loops



%% http://tex.stackexchange.com/questions/66490/drawing-a-tikz-arc-specifying-the-center
%% Draws beach ball
\tikzset{pics/carc/.style args={#1:#2:#3}{code={\draw[pic actions] (#1:#3) arc(#1:#2:#3);}}}



\usepackage{array}
\setlength{\extrarowheight}{+.1cm}
\newdimen\digitwidth
\settowidth\digitwidth{9}
\def\divrule#1#2{
\noalign{\moveright#1\digitwidth
\vbox{\hrule width#2\digitwidth}}}






\DeclareMathOperator{\arccot}{arccot}
\DeclareMathOperator{\arcsec}{arcsec}
\DeclareMathOperator{\arccsc}{arccsc}

















%%This is to help with formatting on future title pages.
\newenvironment{sectionOutcomes}{}{}


\outcome{Recall the meaning and properties of inverse trigonometric functions.}
\outcome{Understand how the derivative of an inverse function relates to the
original derivative.}

\title[Break-Ground:]{We can figure it out}

\begin{document}
\begin{abstract}
Two young mathematicians discuss the derivative of inverse functions.
\end{abstract}
\maketitle

Check out this dialogue between two calculus students (based on a true
story):

\begin{dialogue}
\item[Devyn] Riley, I have a calculus question.
\item[Riley] Hit me with it.
\item[Devyn] What's the derivative of $\arctan(x)$?
\item[Riley] Hmmm\dots we haven't talked about that yet in our class.
\item[Devyn] I know! But maybe we can figure it out.
\item[Riley] Well
  \[
  \arctan(x) = \tan^{-1}(x)
  \]
  and now we can use the chain rule to take its derivative
  \begin{align*}
    \ddx \tan^{-1}(x) &= -\tan^{-2}(x)\sec^2(x)\\
    &= -\frac{\cos^2x}{\sin^2x}\cdot \frac{1}{\cos^2x}\\
    &= \frac{-1}{\sin^2x}\\
    &= -\csc^2x
  \end{align*}
\item[Devyn] But is this right?
\end{dialogue}

Let's see if we can figure out if Devyn and Riley are correct. Start by looking at a plot of $\theta = \arctan(x)$:

\begin{image}
\begin{tikzpicture}
	\begin{axis}[
            xmin=-6,xmax=6,ymin=-2,ymax=2,
            axis lines=center,
            ytick={0, -1.57,1.57},
            width=9in,
            height=2.5in,
            yticklabels={$0$, $-\pi/2$,$\pi/2$},
            xtick={0},
            unit vector ratio*=1 1 1,
            xlabel=$x$, ylabel=$\theta$,
            every axis y label/.style={at=(current axis.above origin),anchor=south},
            every axis x label/.style={at=(current axis.right of origin),anchor=west},
          ]        
          \addplot [very thick, penColor3!20!penColor2, samples=100,smooth, domain=(-6:6)] {atan(x)*pi/180};
          \addplot [textColor,dashed] plot coordinates {(-6,-1.57) (6,-1.57)};
          \addplot [textColor,dashed] plot coordinates {(-6,1.57) (6,1.57)};
        \end{axis}
\end{tikzpicture}
%% \caption{Here we see a plot of $\arctan(y)$, the inverse function of
%% $\tan(\theta)$ when it is restricted to the interval $(-\pi/2,\pi/2)$.}
\end{image}

\begin{problem}
\author{Nela Lakos}
  Let $f(x) = \arctan(x)$. Use the plot  to determine the  intervals(s) where the function $f$ is increasing.
  
From the graph it seems that the function $f$ is increasing on the interval
  \[
 \left(\answer[given]{-\infty},\answer[given]{\infty}\right)
  \]
\end{problem}



%\begin{problem}
%  Let $f(x) = \arctan(x)$. Use the plot above to determine the behavior of the derivative of $f$ as $x$ gets very large.  If the limit does not exist, enter ``DNE''.
%  \[
%  \lim_{x\to \infty} f'(x)
 % \begin{prompt}
  %  = \answer{0}
  %\end{prompt}
 % \]
%\end{problem}

On the other hand,

\begin{problem}
 What is the sign of $f'(x)= - \csc^2 (x)$ on the interval $(0,\pi)$?
 \begin{problem}
   \begin{multipleChoice}
	\choice{positive}
	\choice[correct]{negative}
   \end{multipleChoice}
 \end{problem}
\end{problem}


\begin{problem}
Complete the sentence below:


 Since the sign of $f'(x)= - \csc^2 (x)$ on the interval $(0,\pi)$ is
 \begin{multipleChoice}
	\choice{positive}
	\choice[correct]{negative}
\end{multipleChoice}
the function $f$ must be
\begin{multipleChoice}
	\choice{increasing on the interval $(0, \pi)$}
	\choice[correct]{decreasing on the interval $(0, \pi)$}
\end{multipleChoice}
\end{problem}

\begin{problem}

In light of the problems above, is it possible that
\[
\ddx \arctan(x) = -\csc^2(x)?
\]
\begin{multipleChoice}
	\choice{yes}
	\choice[correct]{no}
\end{multipleChoice}
\end{problem}

\begin{problem}
	When our friends wrote $\arctan(x) = \tan^{-1}(x)$, what do they think the ``$-1$'' represents?  Are they correct?
	\begin{freeResponse}
		Riley thinks that we can use the power rule on the $-1$, which tells us that the students are using $-1$ as an exponent for the tangent function.  However, in the case of inverse functions such as $\arctan(x)$, the $-1$ is not an exponent.
	\end{freeResponse}
\end{problem}


% This gets at what the notation sin^{-1} x means, and what the inverse function theorem is saying

% Contrast it with a geometric interpretation that since the inverse is the fxn flipped over the line y=x, we should have this info once we know f'

% becaue we want to actually have two answers that both seem reasonable and demand that they student resolve the contradiction in mathematics

%%\input{../leveledQuestions.tex}

\end{document}
