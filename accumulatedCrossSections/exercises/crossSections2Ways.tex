\documentclass{ximera}


\graphicspath{
  {./}
  {ximeraTutorial/}
  {basicPhilosophy/}
}

\newcommand{\mooculus}{\textsf{\textbf{MOOC}\textnormal{\textsf{ULUS}}}}

\usepackage{tkz-euclide}\usepackage{tikz}
\usepackage{tikz-cd}
\usetikzlibrary{arrows}
\tikzset{>=stealth,commutative diagrams/.cd,
  arrow style=tikz,diagrams={>=stealth}} %% cool arrow head
\tikzset{shorten <>/.style={ shorten >=#1, shorten <=#1 } } %% allows shorter vectors

\usetikzlibrary{backgrounds} %% for boxes around graphs
\usetikzlibrary{shapes,positioning}  %% Clouds and stars
\usetikzlibrary{matrix} %% for matrix
\usepgfplotslibrary{polar} %% for polar plots
\usepgfplotslibrary{fillbetween} %% to shade area between curves in TikZ
\usetkzobj{all}
\usepackage[makeroom]{cancel} %% for strike outs
%\usepackage{mathtools} %% for pretty underbrace % Breaks Ximera
%\usepackage{multicol}
\usepackage{pgffor} %% required for integral for loops



%% http://tex.stackexchange.com/questions/66490/drawing-a-tikz-arc-specifying-the-center
%% Draws beach ball
\tikzset{pics/carc/.style args={#1:#2:#3}{code={\draw[pic actions] (#1:#3) arc(#1:#2:#3);}}}



\usepackage{array}
\setlength{\extrarowheight}{+.1cm}
\newdimen\digitwidth
\settowidth\digitwidth{9}
\def\divrule#1#2{
\noalign{\moveright#1\digitwidth
\vbox{\hrule width#2\digitwidth}}}






\DeclareMathOperator{\arccot}{arccot}
\DeclareMathOperator{\arcsec}{arcsec}
\DeclareMathOperator{\arccsc}{arccsc}

















%%This is to help with formatting on future title pages.
\newenvironment{sectionOutcomes}{}{}


\author{Jim Talamo}
\license{Creative Commons 3.0 By-NC}


\outcome{Set up an integral or sum of integrals that gives the volume of a solid with known cross sections}
\outcome{Explore how different orientation of cross sections changes the set up of an integral or ssum of integrals}

\begin{document}
\begin{exercise}


The base of a solid is the region bounded by $y=8-x^2$ and the $y=2x$. This exercise explores the effect of changing the orientation of the cross-sections. 

\begin{exercise}

Suppose that the cross-sections through the solid taken \emph{perpendicular} to the $x$-axis are squares. 

            \begin{image}
            \begin{tikzpicture}
            	\begin{axis}[
            		domain=-10:10, ymax=9.9,xmax=3.9, ymin=-9.9, xmin=-4.9,
            		axis lines =center, xlabel=$x$, ylabel=$y$,
            		every axis y label/.style={at=(current axis.above origin),anchor=south},
            		every axis x label/.style={at=(current axis.right of origin),anchor=west},
            		axis on top,
            		]
                      
            	\addplot [draw=penColor,very thick,smooth] {2*x};
            	\addplot [draw=penColor2,very thick,smooth] {8-x^2};
                       
            	\addplot [name path=A,domain=-4:2,draw=none] {2*x};   
            	\addplot [name path=B,domain=-4:2,draw=none] {8-x^2};
            	\addplot [fillp] fill between[of=A and B];
                      
                      
            	\node at (axis cs:1.8,8.5) [penColor2] {$y=8-x^2$};
	        \node at (axis cs:-.7,-3.6) [penColor] {$y=2x$};
            	
            	\end{axis}
            \end{tikzpicture}
            \end{image}

	







\begin{exercise}
To express the volume as an integral, we must:
\begin{multipleChoice}
\choice[correct]{integrate with respect to $x$.}
\choice{integrate with respect to $y$.}
\end{multipleChoice}

How many integrals are needed to express the volume of the solid? $\answer{1}$
\begin{exercise}
Now, set up an integral to compute the volume of this solid then evaluate it to fine the volume.

An integral that gives the volume of the solid is:

\[
	V= \int_{x=\answer{-4}}^{x=\answer{2}}
	\answer{((8-x^2)-(2x))^2 } dx
	\]

Evaluating this integral gives that the volume of the solid is $\answer{1296/5}$ cubic units.
	\end{exercise}
	\end{exercise}
	\end{exercise}

\begin{exercise}

Suppose that the cross-sections through the solid taken \emph{parallel} to the $x$-axis are squares. 

\begin{exercise}
To express the volume as an integral, we must:
\begin{multipleChoice}
\choice{integrate with respect to $x$.}
\choice[correct]{integrate with respect to $y$.}
\end{multipleChoice}

How many integrals are needed to express the volume of the solid? $\answer{2}$

\begin{exercise}

  \begin{image}
            \begin{tikzpicture}
            	\begin{axis}[
            		domain=-10:10, ymax=9.9,xmax=3.9, ymin=-9.9, xmin=-4.9,
            		axis lines =center, xlabel=$x$, ylabel=$y$,
            		every axis y label/.style={at=(current axis.above origin),anchor=south},
            		every axis x label/.style={at=(current axis.right of origin),anchor=west},
            		axis on top,
            		]
                      
            	\addplot [draw=penColor,very thick,smooth] {2*x};
            	\addplot [draw=penColor2,very thick,smooth,domain=0:10] {8-x^2};
	        \addplot [draw=black!50!green,very thick,smooth,domain=-10:0] {8-x^2};
                       
            	\addplot [name path=A,domain=-4:2,draw=none] {2*x};   
            	\addplot [name path=B,domain=-4:2,draw=none] {8-x^2};
            	\addplot [fillp] fill between[of=A and B];
                      
                      
            	\node at (axis cs:1.8,8.5) [penColor2] {$x=\sqrt{8-y}$};
	       \node at (axis cs:-3.2,6.5) [black!50!green] {$x=-\sqrt{8-y}$};
	        \node at (axis cs:-2.1,-6.6) [penColor] {$x = \frac{y}{2}$};
            	
            	\end{axis}
            \end{tikzpicture}
            \end{image}



Now, set up a sum of integrals to compute the volume of this solid.

A sum of integrals that gives the volume of the solid is:

\[
	V= \int_{y=\answer{-8}}^{y=\answer{4}} \answer{(y/2-(-\sqrt{8-y}) )^2 } dy + \int_{y=\answer{4}}^{y=8} \answer{(\sqrt{8-y}-(-\sqrt{8-y}) )^2 } dy
	\]

\begin{exercise}
Evaluating these integrals gives:

$ \int_{y=-8}^{y=4} (y/2 + \sqrt{8-y} )^2 dy = \answer{1048/15}$

$ \int_{y=4}^{y=8} (2\sqrt{8-y})^2 dy = \answer{32}$
 
so that the volume of the solid is $\answer{1528/15}$ cubic units.

\end{exercise}

	\end{exercise}
	\end{exercise}
	\end{exercise}


\end{exercise}

\end{document}
