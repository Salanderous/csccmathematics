\documentclass{ximera}


\graphicspath{
  {./}
  {ximeraTutorial/}
  {basicPhilosophy/}
}

\newcommand{\mooculus}{\textsf{\textbf{MOOC}\textnormal{\textsf{ULUS}}}}

\usepackage{tkz-euclide}\usepackage{tikz}
\usepackage{tikz-cd}
\usetikzlibrary{arrows}
\tikzset{>=stealth,commutative diagrams/.cd,
  arrow style=tikz,diagrams={>=stealth}} %% cool arrow head
\tikzset{shorten <>/.style={ shorten >=#1, shorten <=#1 } } %% allows shorter vectors

\usetikzlibrary{backgrounds} %% for boxes around graphs
\usetikzlibrary{shapes,positioning}  %% Clouds and stars
\usetikzlibrary{matrix} %% for matrix
\usepgfplotslibrary{polar} %% for polar plots
\usepgfplotslibrary{fillbetween} %% to shade area between curves in TikZ
\usetkzobj{all}
\usepackage[makeroom]{cancel} %% for strike outs
%\usepackage{mathtools} %% for pretty underbrace % Breaks Ximera
%\usepackage{multicol}
\usepackage{pgffor} %% required for integral for loops



%% http://tex.stackexchange.com/questions/66490/drawing-a-tikz-arc-specifying-the-center
%% Draws beach ball
\tikzset{pics/carc/.style args={#1:#2:#3}{code={\draw[pic actions] (#1:#3) arc(#1:#2:#3);}}}



\usepackage{array}
\setlength{\extrarowheight}{+.1cm}
\newdimen\digitwidth
\settowidth\digitwidth{9}
\def\divrule#1#2{
\noalign{\moveright#1\digitwidth
\vbox{\hrule width#2\digitwidth}}}






\DeclareMathOperator{\arccot}{arccot}
\DeclareMathOperator{\arcsec}{arcsec}
\DeclareMathOperator{\arccsc}{arccsc}

















%%This is to help with formatting on future title pages.
\newenvironment{sectionOutcomes}{}{}


\author{Jim Talamo}
\license{Creative Commons 3.0 By-NC}


\outcome{Set up an integral that gives the area of a region}
\outcome{Set up an integral that gives the volume of a solid whose cross sections are the region}

\begin{document}


The region $R$ is bounded by $y=3-x^2$ and $y=x+1$. 





\begin{exercise}
By integrating with respect to $y$, how many integrals are needed to express the area of $R$? $\answer{2}$

\begin{exercise}
The area of the region can be found by evaluating:

\[
	A = \int_{-1}^{\answer{2}} \answer{y-1+\sqrt{3-y}} dy  +\int_{\answer{2}}^{\answer{3}} \answer{2\sqrt{3-y}} dy 
\]
	
\end{exercise}
\end{exercise}


\begin{exercise}
The base of a certain solid is the region $R$.  Cross sections through the solid taken parallel to the $y$-axis are semicircles. 


To express the volume of a solid using a definite integral, we should:
\begin{multipleChoice}
\choice[correct]{integrate with respect to $x$.}
\choice{integrate with respect to $y$.}
\end{multipleChoice}

An integral that expresses the volume of the solid is:

\[
	V= \int_{x=\answer{-2}}^{x=\answer{1}}\answer{\frac{\pi}{8}(2-x-x^2)^2} dx
\]
		
		



\end{exercise}
\end{document}
