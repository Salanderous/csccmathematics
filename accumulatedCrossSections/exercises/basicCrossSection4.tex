\documentclass{ximera}


\graphicspath{
  {./}
  {ximeraTutorial/}
  {basicPhilosophy/}
}

\newcommand{\mooculus}{\textsf{\textbf{MOOC}\textnormal{\textsf{ULUS}}}}

\usepackage{tkz-euclide}\usepackage{tikz}
\usepackage{tikz-cd}
\usetikzlibrary{arrows}
\tikzset{>=stealth,commutative diagrams/.cd,
  arrow style=tikz,diagrams={>=stealth}} %% cool arrow head
\tikzset{shorten <>/.style={ shorten >=#1, shorten <=#1 } } %% allows shorter vectors

\usetikzlibrary{backgrounds} %% for boxes around graphs
\usetikzlibrary{shapes,positioning}  %% Clouds and stars
\usetikzlibrary{matrix} %% for matrix
\usepgfplotslibrary{polar} %% for polar plots
\usepgfplotslibrary{fillbetween} %% to shade area between curves in TikZ
\usetkzobj{all}
\usepackage[makeroom]{cancel} %% for strike outs
%\usepackage{mathtools} %% for pretty underbrace % Breaks Ximera
%\usepackage{multicol}
\usepackage{pgffor} %% required for integral for loops



%% http://tex.stackexchange.com/questions/66490/drawing-a-tikz-arc-specifying-the-center
%% Draws beach ball
\tikzset{pics/carc/.style args={#1:#2:#3}{code={\draw[pic actions] (#1:#3) arc(#1:#2:#3);}}}



\usepackage{array}
\setlength{\extrarowheight}{+.1cm}
\newdimen\digitwidth
\settowidth\digitwidth{9}
\def\divrule#1#2{
\noalign{\moveright#1\digitwidth
\vbox{\hrule width#2\digitwidth}}}






\DeclareMathOperator{\arccot}{arccot}
\DeclareMathOperator{\arcsec}{arcsec}
\DeclareMathOperator{\arccsc}{arccsc}

















%%This is to help with formatting on future title pages.
\newenvironment{sectionOutcomes}{}{}


\author{Jim Talamo and Alex Beckwith}
\license{Creative Commons 3.0 By-NC}


\outcome{Set up an integral or sum of integrals that gives the volume of a solid with known cross sections}

\begin{document}
\begin{exercise}


The base of a solid is the region in the first quadrant by $y=\cos(x)$ and the $x$-axis. The cross-sections through the solid taken perpendicular to the $y$-axis are isosceles right triangles with one horizontal leg in the $xy$-plane and a vertical leg above the $x$-axis. 

            \begin{image}
            \begin{tikzpicture}
            	\begin{axis}[
            		domain=-0.5:2, ymax=1.4,xmax=1.9, ymin=-0.4, xmin=-0.4,
            		axis lines =center, xlabel=$x$, ylabel=$y$,
            		every axis y label/.style={at=(current axis.above origin),anchor=south},
            		every axis x label/.style={at=(current axis.right of origin),anchor=west},
            		axis on top,
            		]
                      
            	\addplot [draw=penColor,very thick,smooth] {cos(deg(x))};
            	\addplot [draw=penColor2,very thick,smooth] {0};
                       
            	\addplot [name path=A,domain=0:1.57,draw=none] {cos(deg(x))};   
            	\addplot [name path=B,domain=0:1.57,draw=none] {0};
            	\addplot [fillp] fill between[of=A and B];
                      
                      
            	\node at (axis cs:1.3,.7) [penColor] {$y=\cos(x)$};
            	
            	\end{axis}
            \end{tikzpicture}
            \end{image}

	


\begin{exercise}
To express the volume as an integral, we must:
\begin{multipleChoice}
\choice{integrate with respect to $x$.}
\choice[correct]{integrate with respect to $y$.}
\end{multipleChoice}

\begin{exercise}
Now, set up an integral to compute the volume of this solid, then evaluate it to find the volume of the solid.

An integral that gives the volume of the solid is:

\[
	V= \int_{y=\answer{0}}^{y=\answer{1}}
	\answer{\frac{1}{2} (\arccos(y))^2 } dy
	\]

	\end{exercise}
	\end{exercise}
	\end{exercise}
\end{document}