\documentclass[10pt,handout,twocolumn,twoside,wordchoicegiven]{xourse}


\graphicspath{
  {./}
  {ximeraTutorial/}
  {basicPhilosophy/}
}

\newcommand{\mooculus}{\textsf{\textbf{MOOC}\textnormal{\textsf{ULUS}}}}

\usepackage{tkz-euclide}\usepackage{tikz}
\usepackage{tikz-cd}
\usetikzlibrary{arrows}
\tikzset{>=stealth,commutative diagrams/.cd,
  arrow style=tikz,diagrams={>=stealth}} %% cool arrow head
\tikzset{shorten <>/.style={ shorten >=#1, shorten <=#1 } } %% allows shorter vectors

\usetikzlibrary{backgrounds} %% for boxes around graphs
\usetikzlibrary{shapes,positioning}  %% Clouds and stars
\usetikzlibrary{matrix} %% for matrix
\usepgfplotslibrary{polar} %% for polar plots
\usepgfplotslibrary{fillbetween} %% to shade area between curves in TikZ
\usetkzobj{all}
\usepackage[makeroom]{cancel} %% for strike outs
%\usepackage{mathtools} %% for pretty underbrace % Breaks Ximera
%\usepackage{multicol}
\usepackage{pgffor} %% required for integral for loops



%% http://tex.stackexchange.com/questions/66490/drawing-a-tikz-arc-specifying-the-center
%% Draws beach ball
\tikzset{pics/carc/.style args={#1:#2:#3}{code={\draw[pic actions] (#1:#3) arc(#1:#2:#3);}}}



\usepackage{array}
\setlength{\extrarowheight}{+.1cm}
\newdimen\digitwidth
\settowidth\digitwidth{9}
\def\divrule#1#2{
\noalign{\moveright#1\digitwidth
\vbox{\hrule width#2\digitwidth}}}






\DeclareMathOperator{\arccot}{arccot}
\DeclareMathOperator{\arcsec}{arcsec}
\DeclareMathOperator{\arccsc}{arccsc}

















%%This is to help with formatting on future title pages.
\newenvironment{sectionOutcomes}{}{}


\pdfOnly{\usepackage{printStyles/lulu1}}

\logo{logos/calculus1Logo/logo.png}

\title{CSCC Calculus 1}
\begin{document}
\maketitle

\setcounter{tocdepth}{2}
\begin{onlineOnly}
%% How to use
%\chapterstyle
\activity{ximeraTutorial/titlePage.tex}
%\sectionstyle
\activity{ximeraTutorial/howToUseXimera.tex}
\activity{ximeraTutorial/howIsMyWorkScored.tex}
\end{onlineOnly}

%\part{Functions, limits, and continuity}
%\part{Content for the First Exam}


%% Understanding Functions
\chapterstyle
\activity{understandingFunctions/titlePage.tex}
\sectionstyle
\activity{understandingFunctions/breakGround.tex}
\activity{understandingFunctions/digInForEachInputExactlyOneOutput.tex}
\activity{understandingFunctions/digInCompositionOfFunctions.tex}
\activity{understandingFunctions/digInInversesOfFunctions.tex}




%% Review of famous functions
\chapterstyle
\activity{reviewOfFamousFunctions/titlePage.tex}
\sectionstyle
\activity{reviewOfFamousFunctions/breakGround.tex}
\activity{reviewOfFamousFunctions/digInPolynomialFunctions.tex}
\activity{reviewOfFamousFunctions/digInRationalFunctions.tex}
\activity{reviewOfFamousFunctions/digInTrigonometricFunctions.tex}
\activity{reviewOfFamousFunctions/digInExponentialAndLogarithmeticFunctions.tex}




%% What is a limit
\chapterstyle
\activity{whatIsALimit/titlePage.tex}
\sectionstyle
\activity{whatIsALimit/breakGround.tex}
\activity{whatIsALimit/digInWhatIsALimit.tex}
\activity{whatIsALimit/digInContinuity.tex}





%% Limit Laws
\chapterstyle
\activity{limitLaws/titlePage.tex}
\sectionstyle
\activity{limitLaws/breakGround.tex}
\activity{limitLaws/digInLimitLaws.tex}
\activity{limitLaws/digInTheSqueezeTheorem.tex}





%% Indeterminant forms
\chapterstyle
\activity{indeterminateForms/titlePage.tex}
\sectionstyle
\activity{indeterminateForms/breakGround.tex}
\activity{indeterminateForms/digInLimitsOfTheFormZeroOverZero.tex}
\activity{indeterminateForms/digInLimitsOfTheFormNonZeroOverZero.tex}




%%%%%%%%%%%%%%%%%%%%%%%%%%%%%%%%%%%%%%%%%%%%%%%%%%%%%%%%%%%%%%%%%%%%%%%%%%%%%%
%\activity{samplePractice/samplePractice.tex}
%%%%%%%%%%%%%%%%%%%%%%%%%%%%%%%%%%%%%%%%%%%%%%%%%%%%%%%%%%%%%%%%%%%%%%%%%%%%%%






%% Using limits to detect asymptotes
\chapterstyle
\activity{asymptotesAsLimits/titlePage.tex}
\sectionstyle
\activity{asymptotesAsLimits/breakGround.tex}
\activity{asymptotesAsLimits/digInVerticalAsymptotes.tex}
\activity{asymptotesAsLimits/digInHorizontalAsymptotes.tex}
\activity{asymptotesAsLimits/digInSlantAsymptotes.tex}





%% The intermediate value theorem
\chapterstyle
\activity{continuity/titlePage.tex}
\sectionstyle
\activity{continuity/breakGround.tex}
\activity{continuity/digInContinuityOfPiecewiseFunctions.tex}
\activity{continuity/digInTheIntermediateValueTheorem.tex}


%%%%%%%%%%%%%%%%%%%%%%%%%%%%%%%%%%%%%%%%%%%%%%%%%%%%%%%%%%%%%%%%%%%%%%%%%%%%%%
%\activity{exams/calculus1/examReviews/examOneReview.tex}
%%%%%%%%%%%%%%%%%%%%%%%%%%%%%%%%%%%%%%%%%%%%%%%%%%%%%%%%%%%%%%%%%%%%%%%%%%%%%%






%\part{Content for the Second Exam}






%% An application of limits
\chapterstyle
\activity{anApplicationOfLimits/titlePage.tex}
\sectionstyle
\activity{anApplicationOfLimits/breakGround.tex}
\activity{anApplicationOfLimits/digInInstantaneousVelocity.tex}




%\part{Derivatives}




%% Definition of the derivative
\chapterstyle
\activity{definitionOfTheDerivative/titlePage.tex}
\sectionstyle
\activity{definitionOfTheDerivative/breakGround.tex}
\activity{definitionOfTheDerivative/digInTheDerivativeViaLimits.tex}




%% The derivative as a function
\chapterstyle
\activity{derivativeAsAFunction/titlePage.tex}
\sectionstyle
\activity{derivativeAsAFunction/breakGround.tex}
\activity{derivativeAsAFunction/digInTheDerivativeAsAFunction.tex}
\activity{derivativeAsAFunction/digInDifferentiabilityImpliesContinuity.tex}





%% Rules of differentiation
\chapterstyle
\activity{rulesOfDifferentiation/titlePage.tex}
\sectionstyle
\activity{rulesOfDifferentiation/breakGround.tex}
\activity{rulesOfDifferentiation/digInBasicRulesOfDifferentiation.tex}
\activity{rulesOfDifferentiation/digInTheDerivativeOfEToTheX.tex}
\activity{rulesOfDifferentiation/digInTheDerivativeOfSine.tex}





%% Product and quotient rules
\chapterstyle
\activity{productAndQuotientRules/titlePage.tex}
\sectionstyle
\activity{productAndQuotientRules/breakGround.tex}
\activity{productAndQuotientRules/digInProductRuleAndQuotientRule.tex}






%% The chain rule
\chapterstyle
\activity{chainRule/titlePage.tex}
\sectionstyle
\activity{chainRule/breakGround.tex}
\activity{chainRule/digInChainRule.tex}
\activity{chainRule/digInDerivativesOfTrigonometricFunctions.tex}





%% Higher order derivatives and graphs %%%%%CHANGE
\chapterstyle
\activity{higherOrderDerivativesAndGraphs/titlePage.tex}
\sectionstyle
\activity{higherOrderDerivativesAndGraphs/breakGround.tex}
\activity{higherOrderDerivativesAndGraphs/digInHigherOrderDerivativesAndGraphs.tex}
\activity{higherOrderDerivativesAndGraphs/digInConcavity.tex}
\activity{higherOrderDerivativesAndGraphs/digInPositionVelocityAndAcceleration.tex}




%% Implicit Differentiation %%%%%CHANGE
\chapterstyle
\activity{implicitDifferentiation/titlePage.tex}
\sectionstyle
\activity{implicitDifferentiation/breakGround.tex}
\activity{implicitDifferentiation/digInImplicitDifferentiation.tex}
%% Derivatives of inverse functions %%%%%CHANGE
\activity{implicitDifferentiation/digInDerivativesOfInverseExponentialFunctions.tex}
%%\activity{derivativesOfInverseFunctions/digInDerivativesOfInverseTrigonometricFunctions.tex}
%%\activity{derivativesOfInverseFunctions/digInInverseFunctionTheorem.tex}





%% Logarithmic differentiation %%%%%CHANGE
\chapterstyle
\activity{logarithmicDifferentiation/titlePage.tex}
\sectionstyle
\activity{logarithmicDifferentiation/breakGround.tex}
\activity{logarithmicDifferentiation/digInLogarithmicDifferentiation.tex}





\chapterstyle %%%%%CHANGE
\activity{derivativesOfInverseFunctions/titlePage.tex}
\sectionstyle
\activity{derivativesOfInverseFunctions/breakGround.tex}
\activity{derivativesOfInverseFunctions/digInDerivativesOfInverseTrigonometricFunctions.tex}
\activity{derivativesOfInverseFunctions/digInInverseFunctionTheorem.tex}





%% %% More than one rate %%%%%CHANGE
\chapterstyle
\activity{moreThanOneRate/titlePage.tex}
\sectionstyle
\activity{moreThanOneRate/breakGround.tex}
\activity{moreThanOneRate/digInMoreThanOneRate.tex}






%% %% Applied related rates %%%%%CHANGE
\chapterstyle
\activity{appliedRelatedRates/titlePage.tex}
\sectionstyle
\activity{appliedRelatedRates/breakGround.tex}
\activity{appliedRelatedRates/digInAppliedRelatedRates.tex}





%% Maximums and Minimums  %%%CHANGE
\chapterstyle
\activity{maximumsAndMinimums/titlePage.tex}
\sectionstyle
\activity{maximumsAndMinimums/breakGround.tex}
\activity{maximumsAndMinimums/digInMaximumsAndMinimums.tex}







%%%%%%%%%%%%%%%%%%%%%%%%%%%%%%%%%%%%%%%%%%%
%\activity{exams/calculus1/examReviews/examTwoReview.tex}
%%%%%%%%%%%%%%%%%%%%%%%%%%%%%%%%%%%%%%%%%%%





%\part{Content for the Third Exam}






%% Concepts of graphing functions %%%CHANGE
\chapterstyle
\activity{conceptsOfGraphingFunctions/titlePage.tex}
\sectionstyle
\activity{conceptsOfGraphingFunctions/breakGround.tex}
\activity{conceptsOfGraphingFunctions/digInConceptsOfGraphingFunctions.tex}






%% Computations for graphing functions %%%CHANGE
\chapterstyle
\activity{computationsForGraphingFunctions/titlePage.tex}
\sectionstyle
\activity{computationsForGraphingFunctions/breakGround.tex}
\activity{computationsForGraphingFunctions/digInComputationsForGraphingFunctions.tex}





%% %% Mean Value Theorem  %%%CHANGE
\chapterstyle
\activity{meanValueTheorem/titlePage.tex}
\sectionstyle
\activity{meanValueTheorem/breakGround.tex}
\activity{meanValueTheorem/digInExtremeValueTheorem.tex}
\activity{meanValueTheorem/digInMeanValueTheorem.tex}





%% Linear Approximation
\chapterstyle
\activity{linearApproximation/titlePage.tex}
\sectionstyle
\activity{linearApproximation/breakGround.tex}
\activity{linearApproximation/digInLinearApproximation.tex}
\activity{linearApproximation/digInExplanationOfTheProductAndChainRules.tex}





%% %%Optimization
\chapterstyle
\activity{optimization/titlePage.tex}
\sectionstyle
\activity{optimization/breakGround.tex}
\activity{optimization/digInOptimization.tex}




%% %% Applied optimization
\chapterstyle
\activity{appliedOptimization/titlePage.tex}
\sectionstyle
\activity{appliedOptimization/breakGround.tex}
\activity{appliedOptimization/digInAppliedOptimization.tex}




%% %% L'Hosptial's Rule
\chapterstyle
\activity{lhopitalsRule/titlePage.tex}
\sectionstyle
\activity{lhopitalsRule/breakGround.tex}
\activity{lhopitalsRule/digInLhopitalsRule.tex}




%% %% Antiderivatives
\chapterstyle
\activity{antiderivatives/titlePage.tex}
\sectionstyle
\activity{antiderivatives/breakGround.tex}
\activity{antiderivatives/digInBasicAntiderivatives.tex}
\activity{antiderivatives/digInFallingObjects.tex}




%% %%Approximating the area under a curve
\chapterstyle
\activity{approximatingTheAreaUnderACurve/titlePage.tex}
\sectionstyle
\activity{approximatingTheAreaUnderACurve/breakGround.tex}
\activity{approximatingTheAreaUnderACurve/digInSigmaNotation.tex}
\activity{approximatingTheAreaUnderACurve/digInApproximatingAreaWithRectangles.tex}




 %% %%The definite integral
\chapterstyle
\activity{definiteIntegral/titlePage.tex}
\sectionstyle
\activity{definiteIntegral/breakGround.tex}
\activity{definiteIntegral/digInTheDefiniteIntegral.tex}




%% %% Antiderivatives and area
\chapterstyle
\activity{antiderivativesAndArea/titlePage.tex}
\sectionstyle
\activity{antiderivativesAndArea/breakGround.tex}
\activity{antiderivativesAndArea/digInRelatingVelocityAndPositionAntiderivativesAndAreas.tex}












\pdfOnly{\printindex}
\end{document}
