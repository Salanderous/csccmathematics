\documentclass{ximera}


\graphicspath{
  {./}
  {ximeraTutorial/}
  {basicPhilosophy/}
}

\newcommand{\mooculus}{\textsf{\textbf{MOOC}\textnormal{\textsf{ULUS}}}}

\usepackage{tkz-euclide}\usepackage{tikz}
\usepackage{tikz-cd}
\usetikzlibrary{arrows}
\tikzset{>=stealth,commutative diagrams/.cd,
  arrow style=tikz,diagrams={>=stealth}} %% cool arrow head
\tikzset{shorten <>/.style={ shorten >=#1, shorten <=#1 } } %% allows shorter vectors

\usetikzlibrary{backgrounds} %% for boxes around graphs
\usetikzlibrary{shapes,positioning}  %% Clouds and stars
\usetikzlibrary{matrix} %% for matrix
\usepgfplotslibrary{polar} %% for polar plots
\usepgfplotslibrary{fillbetween} %% to shade area between curves in TikZ
\usetkzobj{all}
\usepackage[makeroom]{cancel} %% for strike outs
%\usepackage{mathtools} %% for pretty underbrace % Breaks Ximera
%\usepackage{multicol}
\usepackage{pgffor} %% required for integral for loops



%% http://tex.stackexchange.com/questions/66490/drawing-a-tikz-arc-specifying-the-center
%% Draws beach ball
\tikzset{pics/carc/.style args={#1:#2:#3}{code={\draw[pic actions] (#1:#3) arc(#1:#2:#3);}}}



\usepackage{array}
\setlength{\extrarowheight}{+.1cm}
\newdimen\digitwidth
\settowidth\digitwidth{9}
\def\divrule#1#2{
\noalign{\moveright#1\digitwidth
\vbox{\hrule width#2\digitwidth}}}






\DeclareMathOperator{\arccot}{arccot}
\DeclareMathOperator{\arcsec}{arcsec}
\DeclareMathOperator{\arccsc}{arccsc}

















%%This is to help with formatting on future title pages.
\newenvironment{sectionOutcomes}{}{}


\outcome{Understand what information the derivative gives concerning when a function is increasing or decreasing.}
\outcome{Understand what information the second derivative gives concerning the concavity of a function.}

\author{Nela Lakos \and Kyle Parsons}

\begin{document}
\begin{exercise}

Assume that the function $f$ is continuous on $(-\infty,\infty)$. The graph of $f'(x)$, the derivative of $f$, is shown in the figure below.

\begin{image}
  \begin{tikzpicture}
    \begin{axis}[
        xmin=-6.3,xmax=6.3,ymin=-2.3,ymax=6.3,
        clip=true,
        unit vector ratio*=1 1 1,
        axis lines=center,
        grid = major,
        ytick={-10,-9,...,10},
    xtick={-10,-9,...,10},
        xlabel=$x$, ylabel=$y$,
        every axis y label/.style={at=(current axis.above origin),anchor=south},
        every axis x label/.style={at=(current axis.right of origin),anchor=west},
      ]
      \addplot[very thick,penColor,domain=-6.3:0] plot{-x-1};
      \addplot[very thick,penColor,domain=0:4] plot{x-1};
      \addplot[very thick,penColor,domain=4:6.3] plot{-1/(x-3};
      
      \addplot[color=penColor,fill=white,only marks,mark=*] coordinates{(4,3) (4,-1)};
      
      \node at (axis cs:2,3.5) {$y = f'(x)$};
      \end{axis}`
  \end{tikzpicture}
\end{image}

$f(x)$ is increasing on the intervals (from left to right) $\left(\answer{-\infty},\answer{-1}\right)$ and $\left(\answer{1},\answer{4}\right)$.

Call the critical points of $f$ $a$, $b$, and $c$ with $a<b<c$.  Then
\begin{align*}
a &= \answer{-1}\\
b &= \answer{1}\\
c &= \answer{4}
\end{align*}

The critical point $a$ corresponds to a \wordChoice{\choice[correct]{local max} \choice{local min} \choice{neither a local max nor a local min}}.

The critical point $b$ corresponds to a \wordChoice{\choice{local max} \choice[correct]{local min} \choice{neither a local max nor a local min}}.

The critical point $c$ corresponds to a \wordChoice{\choice[correct]{local max} \choice{local min} \choice{neither a local max nor a local min}}.

$f$ is concave down on the interval $\left(\answer{-\infty},\answer{0}\right)$.

$f$ has an inflection point at $x=\answer{0}$.

\end{exercise}
\end{document}