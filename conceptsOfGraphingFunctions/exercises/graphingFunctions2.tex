\documentclass{ximera}


\graphicspath{
  {./}
  {ximeraTutorial/}
  {basicPhilosophy/}
}

\newcommand{\mooculus}{\textsf{\textbf{MOOC}\textnormal{\textsf{ULUS}}}}

\usepackage{tkz-euclide}\usepackage{tikz}
\usepackage{tikz-cd}
\usetikzlibrary{arrows}
\tikzset{>=stealth,commutative diagrams/.cd,
  arrow style=tikz,diagrams={>=stealth}} %% cool arrow head
\tikzset{shorten <>/.style={ shorten >=#1, shorten <=#1 } } %% allows shorter vectors

\usetikzlibrary{backgrounds} %% for boxes around graphs
\usetikzlibrary{shapes,positioning}  %% Clouds and stars
\usetikzlibrary{matrix} %% for matrix
\usepgfplotslibrary{polar} %% for polar plots
\usepgfplotslibrary{fillbetween} %% to shade area between curves in TikZ
\usetkzobj{all}
\usepackage[makeroom]{cancel} %% for strike outs
%\usepackage{mathtools} %% for pretty underbrace % Breaks Ximera
%\usepackage{multicol}
\usepackage{pgffor} %% required for integral for loops



%% http://tex.stackexchange.com/questions/66490/drawing-a-tikz-arc-specifying-the-center
%% Draws beach ball
\tikzset{pics/carc/.style args={#1:#2:#3}{code={\draw[pic actions] (#1:#3) arc(#1:#2:#3);}}}



\usepackage{array}
\setlength{\extrarowheight}{+.1cm}
\newdimen\digitwidth
\settowidth\digitwidth{9}
\def\divrule#1#2{
\noalign{\moveright#1\digitwidth
\vbox{\hrule width#2\digitwidth}}}






\DeclareMathOperator{\arccot}{arccot}
\DeclareMathOperator{\arcsec}{arcsec}
\DeclareMathOperator{\arccsc}{arccsc}

















%%This is to help with formatting on future title pages.
\newenvironment{sectionOutcomes}{}{}


\outcome{Understand what information the derivative gives concerning when a function is increasing or decreasing.}
\outcome{Understand what information the second derivative gives concerning concavity of a function.}
\outcome{Interpret limits as giving information about functions.}
\outcome{Determine how the graph of a function looks based on an analytic description of the function.}

\author{Nela Lakos \and Kyle Parsons}

\begin{document}
\begin{exercise}

Consider a function $f$ satisfying the following conditions.
\begin{itemize}
\item The domain of $f$ is $(-\infty,-2)$, $(-2,\infty)$.  $f$ is continuous on its domain and differentiable at all points of its domain except at $x=8$.
\item $f(2) = 0$, and $f(8) = 6$.
\item $\lim_{x\to-2}f(x) = -\infty$, and $\lim_{x\to-\infty}f(x) = \lim_{x\to\infty}f(x) = 4$.
\item $f'(x)<0$ on $(-\infty,-2)$ and $(8,\infty)$.
\item $f'(x)>0$ on $(-2,8)$.
\item $f''(x)<0$ on $(-\infty,-2)$ and $(-2,2)$.
\item $f''(x)>0$ on $(2,8)$ and $(8,\infty)$.
\end{itemize}

$f$ is increasing and concave down on the interval $\left(\answer{-2},\answer{2}\right)$.

$f$ is increasing and concave up on the interval $\left(\answer{2},\answer{8}\right)$.

$f$ is decreasing and concave down on the interval $\left(\answer{-\infty},\answer{-2}\right)$.

$f$ is decreasing and concave up on the interval $\left(\answer{8},\answer{\infty}\right)$.

$f$ has $\answer{0}$ local minima.

$f$ has 1 local maximum at $x=\answer{8}$.

$f$ has 1 inflection point at $x=\answer{2}$.

\resizebox{0.45\textwidth}{!}{
  \begin{tikzpicture}
    \begin{axis}[
        xmin=-10.3,xmax=12.3,ymin=-10.3,ymax=10.3,
        clip=true,
        unit vector ratio*=1 1 1,
        axis lines=center,
        grid = major,
        ytick={-10,-8,...,10},
    	xtick={-10,-8,...,12},
        xlabel=$x$, ylabel=$y$,
        y tick label style={anchor=west},
        every axis y label/.style={at=(current axis.above origin),anchor=south},
        every axis x label/.style={at=(current axis.right of origin),anchor=west},
      ]
      %\addplot[thick,dashed,red,domain=-10.3:12.3] plot{4};
      %\draw[thick,dashed,red] (axis cs:-2,0) -- (axis cs:-2,-10.3);
      
      \addplot[very thick,penColor,domain=-10.3:-2.1,samples=50] plot{-1/(x+2)^2+4};
      \addplot[very thick,penColor,domain=-2:2] plot{(x-2)/(x+2)^2};
      \addplot[very thick,penColor,domain=2:12.3] plot{(x-2)^2/6};
      %\addplot[very thick,penColor,domain=8:12.3] plot{2/(x-7)+4};
      
      \addplot[color=penColor,only marks,mark=*] coordinates{(2,0) (8,6)};
      
      \node at (axis cs:3,7) {\huge$A$};
      \end{axis}`
  \end{tikzpicture}}
\hfill
\resizebox{0.45\textwidth}{!}{
  \begin{tikzpicture}
    \begin{axis}[
        xmin=-10.3,xmax=12.3,ymin=-10.3,ymax=10.3,
        clip=true,
        unit vector ratio*=1 1 1,
        axis lines=center,
        grid = major,
        ytick={-10,-8,...,10},
    	xtick={-10,-8,...,12},
        xlabel=$x$, ylabel=$y$,
        y tick label style={anchor=west},
        every axis y label/.style={at=(current axis.above origin),anchor=south},
        every axis x label/.style={at=(current axis.right of origin),anchor=west},
      ]
      %\addplot[thick,dashed,red,domain=-10.3:12.3] plot{4};
      %\draw[thick,dashed,red] (axis cs:-2,0) -- (axis cs:-2,-10.3);
      
      \addplot[very thick,penColor,domain=-10.3:-2.01,samples=50] plot{ln(-(x+2)^3)};
      \addplot[very thick,penColor,domain=-2:8] plot{ln((x+2)/4)*6/ln(5/2)};
      %\addplot[very thick,penColor,domain=-2:2] plot{(x-2)/(x+2)^2};
      %\addplot[very thick,penColor,domain=2:8] plot{(x-2)^2/6};
      \addplot[very thick,penColor,domain=8:12.3] plot{2/(x-7)+4};
      
      \addplot[color=penColor,only marks,mark=*] coordinates{(2,0) (8,6)};
      
      \node at (axis cs:3,7) {\huge$B$};
      \end{axis}`
  \end{tikzpicture}}

\resizebox{0.45\textwidth}{!}{
  \begin{tikzpicture}
    \begin{axis}[
        xmin=-10.3,xmax=12.3,ymin=-10.3,ymax=10.3,
        clip=true,
        unit vector ratio*=1 1 1,
        axis lines=center,
        grid = major,
        ytick={-10,-8,...,10},
    	xtick={-10,-8,...,12},
        xlabel=$x$, ylabel=$y$,
        y tick label style={anchor=west},
        every axis y label/.style={at=(current axis.above origin),anchor=south},
        every axis x label/.style={at=(current axis.right of origin),anchor=west},
      ]
      %\addplot[thick,dashed,red,domain=-10.3:12.3] plot{4};
      %\draw[thick,dashed,red] (axis cs:-2,0) -- (axis cs:-2,-10.3);
      
      \addplot[very thick,penColor,domain=-10.3:-2.1,samples=50] plot{-1/(x+2)^2+4};
      \addplot[very thick,penColor,domain=-2:2] plot{(x-2)/(x+2)^2};
      \addplot[very thick,penColor,domain=2:8] plot{(x-2)^2/6};
      \addplot[very thick,penColor,domain=8:12.3] plot{2/(x-7)+4};
      
      \addplot[color=penColor,only marks,mark=*] coordinates{(2,0) (8,6)};
      
      \node at (axis cs:3,7) {\huge$C$};
      \end{axis}`
  \end{tikzpicture}}
\hfill
\resizebox{0.45\textwidth}{!}{
  \begin{tikzpicture}
    \begin{axis}[
        xmin=-10.3,xmax=12.3,ymin=-10.3,ymax=10.3,
        clip=true,
        unit vector ratio*=1 1 1,
        axis lines=center,
        grid = major,
        ytick={-10,-8,...,10},
    	xtick={-10,-8,...,12},
        xlabel=$x$, ylabel=$y$,
        y tick label style={anchor=west},
        every axis y label/.style={at=(current axis.above origin),anchor=south},
        every axis x label/.style={at=(current axis.right of origin),anchor=west},
      ]
      %\addplot[thick,dashed,red,domain=-10.3:12.3] plot{4};
      %\draw[thick,dashed,red] (axis cs:-2,0) -- (axis cs:-2,-10.3);
      
      \addplot[very thick,penColor,domain=-10.3:-2.01,samples=50] plot{ln(-(x+2)^3)};
      \addplot[very thick,penColor,domain=-2:2] plot{(x-2)/(x+2)^2};
      \addplot[very thick,penColor,domain=2:8] plot{(x-2)^2/6};
      \addplot[very thick,penColor,domain=8:12.3] plot{2/(x-7)+4};
      
      \addplot[color=penColor,only marks,mark=*] coordinates{(2,0) (8,6)};
      
      \node at (axis cs:3,7) {\huge$D$};
      \end{axis}`
  \end{tikzpicture}}
  
Which diagram shows the graph of a function that has all of the above properties?
\begin{multipleChoice}
\choice{$A$}
\choice{$B$}
\choice[correct]{$C$}
\choice{$D$}
\end{multipleChoice}

\end{exercise}
\end{document}