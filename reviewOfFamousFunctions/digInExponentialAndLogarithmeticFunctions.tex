\documentclass{ximera}


\graphicspath{
  {./}
  {ximeraTutorial/}
  {basicPhilosophy/}
}

\newcommand{\mooculus}{\textsf{\textbf{MOOC}\textnormal{\textsf{ULUS}}}}

\usepackage{tkz-euclide}\usepackage{tikz}
\usepackage{tikz-cd}
\usetikzlibrary{arrows}
\tikzset{>=stealth,commutative diagrams/.cd,
  arrow style=tikz,diagrams={>=stealth}} %% cool arrow head
\tikzset{shorten <>/.style={ shorten >=#1, shorten <=#1 } } %% allows shorter vectors

\usetikzlibrary{backgrounds} %% for boxes around graphs
\usetikzlibrary{shapes,positioning}  %% Clouds and stars
\usetikzlibrary{matrix} %% for matrix
\usepgfplotslibrary{polar} %% for polar plots
\usepgfplotslibrary{fillbetween} %% to shade area between curves in TikZ
\usetkzobj{all}
\usepackage[makeroom]{cancel} %% for strike outs
%\usepackage{mathtools} %% for pretty underbrace % Breaks Ximera
%\usepackage{multicol}
\usepackage{pgffor} %% required for integral for loops



%% http://tex.stackexchange.com/questions/66490/drawing-a-tikz-arc-specifying-the-center
%% Draws beach ball
\tikzset{pics/carc/.style args={#1:#2:#3}{code={\draw[pic actions] (#1:#3) arc(#1:#2:#3);}}}



\usepackage{array}
\setlength{\extrarowheight}{+.1cm}
\newdimen\digitwidth
\settowidth\digitwidth{9}
\def\divrule#1#2{
\noalign{\moveright#1\digitwidth
\vbox{\hrule width#2\digitwidth}}}






\DeclareMathOperator{\arccot}{arccot}
\DeclareMathOperator{\arcsec}{arcsec}
\DeclareMathOperator{\arccsc}{arccsc}

















%%This is to help with formatting on future title pages.
\newenvironment{sectionOutcomes}{}{}


\title[Dig-In:]{Exponential and logarithmic functions}


\begin{document}
\begin{abstract}
  Exponential and logarithmic functions illuminated.
\end{abstract}
\maketitle

Exponential and logarithmic functions may seem somewhat esoteric at
first, but they model many phenomena in the real-world.




\section{What are exponential and logarithmic functions?}


\begin{definition}
  An \dfn{exponential function} is a function of the form
  \[
  f(x) = b^x
  \]
  where  $b\ne 1$ is a positive real number. The domain of an
  exponential function is $(-\infty,\infty)$.
\end{definition}

\begin{question}
  Is $b^{-x}$ an exponential function?
  \begin{prompt}
  \begin{multipleChoice}
    \choice[correct]{yes}
    \choice{no}
  \end{multipleChoice}
  \end{prompt}
  \begin{feedback}
    Note that
    \[
    b^{-x} = \left(b^{-1}\right)^x = \left(\frac{1}{b}\right)^x.
    \]
  \end{feedback}
\end{question}



\begin{definition}
  A \dfn{logarithmic function} is a function defined as follows
  \[
  \log_b(x) = y \qquad\text{means that}\qquad b^y = x
  \]
  where  $b\ne 1$ is a positive real number. The domain of a
  logarithmic function is $(0,\infty)$.
\end{definition}

In either definition above $b$ is called the \dfn{base}.

\subsection{Connections between exponential functions and logarithms}

Let $b$ be a positive real number with $b\ne 1$.
\begin{itemize}
\item $b^{\log_b(x)} = x$ for all positive $x$
\item $\log_b(b^x) = x$ for all real $x$
\end{itemize}

\begin{question}
  What exponent makes the following expression true?
  \[
  3^x = e^{\left( x \cdot \answer{\ln 3} \right)}.
  \]
\end{question}


\section{What can the graphs look like?}

\subsection{Graphs of exponential functions}

\begin{example}
  Here we see the the graphs of four exponential functions.
  \begin{image}
    \begin{tikzpicture}
      \begin{axis}[
          domain=-2:2,
          xmin=-2, xmax=2,
          ymin=-.5, ymax=4,
          axis lines =middle, xlabel=$x$, ylabel=$y$,
          every axis y label/.style={at=(current axis.above origin),anchor=south},
          every axis x label/.style={at=(current axis.right of origin),anchor=west},
        ]
	\addplot [very thick, penColor, smooth] {e^x};
        \addplot [very thick, penColor2, smooth] {2^x)};
        \addplot [very thick, penColor3, smooth] {(1/2)^x)};
        \addplot [very thick, penColor4, smooth] {(1/3)^x)};
        
        
        
        \node at (axis cs:-1.5, 2 ) [penColor3,anchor=west] {$A$};
        \node at (axis cs:-.8, 2.6 ) [penColor4,anchor=west] {$B$};
        \node at (axis cs:0.6, 2.6 ) [penColor,anchor=west] {$C$};
        \node at (axis cs:1.2, 2 ) [penColor2,anchor=west] {$D$};
        
      \end{axis}
    \end{tikzpicture}
  \end{image}
  Match the curves $A$, $B$, $C$, and $D$ with the functions
  \[
  e^x, \qquad \left(\frac{1}{2}\right)^{x}, \qquad  \left(\frac{1}{3}\right)^{x}, \qquad 2^{x}.
  \]
  \begin{explanation}
    One way to solve these problems is to compare these functions
    along the vertical line $x=1$,
    \begin{image}
      \begin{tikzpicture}
        \begin{axis}[
            domain=-2:2,
            xmin=-2, xmax=2,
            ymin=-.5, ymax=4,
            axis lines =middle, xlabel=$x$, ylabel=$y$,
            every axis y label/.style={at=(current axis.above origin),anchor=south},
            every axis x label/.style={at=(current axis.right of origin),anchor=west},
          ]
	  \addplot [very thick, penColor, smooth] {e^x}; %C
          \addplot [very thick, penColor2, smooth] {2^x)};%D
          \addplot [very thick, penColor3, smooth] {(1/2)^x)};%A
          \addplot [very thick, penColor4, smooth] {(1/3)^x)};%B
            
          \node at (axis cs:-1.5, 2 ) [penColor3,anchor=west] {$A$};
          \node at (axis cs:-.8, 2.6 ) [penColor4,anchor=west] {$B$};
          \node at (axis cs:0.6, 2.6 ) [penColor,anchor=west] {$C$};
          \node at (axis cs:1.2, 2 ) [penColor2,anchor=west] {$D$};

          \addplot [textColor, dashed] plot coordinates {(1,-.5) (1,4)};

          \addplot[color=penColor,fill=penColor,only marks,mark=*] coordinates{(1,e)}; %C
          \addplot[color=penColor2,fill=penColor2,only marks,mark=*] coordinates{(1,2)}; %D
          \addplot[color=penColor3,fill=penColor3,only marks,mark=*] coordinates{(1,1/2)}; %A
          \addplot[color=penColor4,fill=penColor4,only marks,mark=*] coordinates{(1,1/3)}; %B
        \end{axis}
      \end{tikzpicture}
    \end{image}
    Note
    \[
    \left(\frac{1}{3}\right)^1 < \left(\frac{1}{2}\right)^1  < 2^1 < e^1.
    \]
    Hence we see:
    \begin{itemize}
    \item $\left(\frac{1}{3}\right)^{x}$ corresponds to
      $\answer[given]{B}$.
    \item $\left(\frac{1}{2}\right)^{x}$ corresponds to $\answer[given]{A}$.
    \item $2^x$ corresponds to $\answer[given]{D}$.
    \item $e^x$ corresponds to $\answer[given]{C}$.
    \end{itemize}
  \end{explanation}
\end{example}



\subsection{Graphs of logarithmic functions}


\begin{example}
  Here we see the the graphs of four logarithmic functions.
  \begin{image}
    \begin{tikzpicture}
      \begin{axis}[
          domain=0.05:4,
          xmin=-.5, xmax=4,
          ymin=-2, ymax=2,
          axis lines =middle, xlabel=$x$, ylabel=$y$,
          every axis y label/.style={at=(current axis.above origin),anchor=south},
          every axis x label/.style={at=(current axis.right of origin),anchor=west},
        ]
	\addplot [very thick, penColor, smooth] {ln(x)}; % C
        \addplot [very thick, penColor2, smooth] {ln(x)/ln(2)}; % D
        \addplot [very thick, penColor3, smooth, samples=100] {ln(x)/ln(1/2))}; % A
        \addplot [very thick, penColor4, smooth, samples=100] {ln(x)/ln(1/3))}; %B
        
        
        \node at (axis cs:.5, 1.3 ) [penColor3,anchor=west] {$A$};
        \node at (axis cs:.2, .5 ) [penColor4,anchor=west] {$B$};
        \node at (axis cs:0.2, -.5 ) [penColor,anchor=west] {$C$};
        \node at (axis cs:.5, -1.3 ) [penColor2,anchor=west] {$D$};
        
      \end{axis}
    \end{tikzpicture}
  \end{image}
  Match the curves $A$, $B$, $C$, and $D$ with the functions
  \[
  \ln(x),\qquad \log_{1/2}(x), \qquad \log_{1/3}(x),\qquad \log_2(x).
  \]
  \begin{explanation}
    First remember what $\log_b(x)=y$ means:
    \[
    \log_b(x) = y \qquad\text{means that}\qquad b^y = x.
    \]
    Moreover, $\ln(x) = \log_e(x)$ where $e= 2.71828\dots$.  So now
    examine each of these functions along the horizontal line $y=1$
    \begin{image}
      \begin{tikzpicture}
        \begin{axis}[
            domain=0.05:4,
            xmin=-.5, xmax=4,
            ymin=-2, ymax=2,
            axis lines =middle, xlabel=$x$, ylabel=$y$,
            every axis y label/.style={at=(current axis.above origin),anchor=south},
            every axis x label/.style={at=(current axis.right of origin),anchor=west},
          ]
	  \addplot [very thick, penColor, smooth] {ln(x)}; % C
          \addplot [very thick, penColor2, smooth] {ln(x)/ln(2)}; % D
          \addplot [very thick, penColor3, smooth, samples=100] {ln(x)/ln(1/2))}; % A
          \addplot [very thick, penColor4, smooth, samples=100] {ln(x)/ln(1/3))}; %B
          \addplot [dashed] {1};
        
          
          \node at (axis cs:.5, 1.3 ) [penColor3,anchor=west] {$A$};
          \node at (axis cs:.2, .5 ) [penColor4,anchor=west] {$B$};
          \node at (axis cs:0.2, -.5 ) [penColor,anchor=west] {$C$};
          \node at (axis cs:.5, -1.3 ) [penColor2,anchor=west] {$D$};

          \addplot[color=penColor,fill=penColor,only marks,mark=*] coordinates{(e,1)}; %C
          \addplot[color=penColor2,fill=penColor2,only marks,mark=*] coordinates{(2,1)}; %D
          \addplot[color=penColor3,fill=penColor3,only marks,mark=*] coordinates{(1/2,1)}; %A
          \addplot[color=penColor4,fill=penColor4,only marks,mark=*] coordinates{(1/3,1)}; %B
        \end{axis}
      \end{tikzpicture}
    \end{image}
    Note again (this is from the definition of a logarithm)
    \[
    \left(\frac{1}{3}\right)^1 < \left(\frac{1}{2}\right)^1  < 2^1 < e^1.
    \]
    Hence we see:
    \begin{itemize}
    \item $\log_{1/3}(x)$ corresponds to $\answer[given]{B}$.
    \item $\log_{1/2}(x)$ corresponds to $\answer[given]{A}$.
    \item $\log_2(x)$ corresponds to $\answer[given]{D}$.
    \item $\ln(x)$ corresponds to $\answer[given]{C}$.
    \end{itemize}
  \end{explanation}
\end{example}



\section{Properties of exponential functions and logarithms}

Working with exponential and logarithmic functions is often simplified by  
applying properties of these functions.  These properties will make appearances 
throughout our work.

\subsection{Properties of exponents}
Let $b$ be a positive real number with $b\ne 1$.
\begin{itemize}
  \item $b^m\cdot b^n = b^{m+n}$
  \item $b^{-1} = \frac{1}{b}$
  \item $\left(b^m\right)^n = b^{mn}$
\end{itemize}
\begin{question}
  What exponent makes the following true?
  \[
  2^4 \cdot 2^3 = 2^{\answer{7}}
  \]
  \begin{hint}
    \[
    (2^4) \cdot (2^3) = (2 \cdot 2\cdot 2 \cdot 2) \cdot  (2 \cdot 2\cdot 2)
    \]
  \end{hint}
\end{question}

\subsection{Properties of logarithms}
Let $b$ be a positive real number with $b\ne 1$.
\begin{itemize}
\item $\log_b(m\cdot n) = \log_b(m) + \log_b(n)$
\item $\log_b(m^n) = n\cdot \log_b(m)$
\item $\log_b\left(\frac{1}{m}\right) = \log_b(m^{-1}) = -\log_b(m)$
\item $\log_a(m) = \frac{\log_b(m)}{\log_b(a)}$
\end{itemize}

\begin{question}
  What value makes the following expression true?
  \[
  \log_2\left(\frac{8}{16}\right) = 3-\answer{4}
  \]
\end{question}


\begin{question}
  What makes the following expression true?
  \[
  \log_3(x) = \frac{\ln(x)}{\answer{\ln(3)}}
  \]
\end{question}



\end{document}
