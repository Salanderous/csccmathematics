\documentclass{ximera}


\graphicspath{
  {./}
  {ximeraTutorial/}
  {basicPhilosophy/}
}

\newcommand{\mooculus}{\textsf{\textbf{MOOC}\textnormal{\textsf{ULUS}}}}

\usepackage{tkz-euclide}\usepackage{tikz}
\usepackage{tikz-cd}
\usetikzlibrary{arrows}
\tikzset{>=stealth,commutative diagrams/.cd,
  arrow style=tikz,diagrams={>=stealth}} %% cool arrow head
\tikzset{shorten <>/.style={ shorten >=#1, shorten <=#1 } } %% allows shorter vectors

\usetikzlibrary{backgrounds} %% for boxes around graphs
\usetikzlibrary{shapes,positioning}  %% Clouds and stars
\usetikzlibrary{matrix} %% for matrix
\usepgfplotslibrary{polar} %% for polar plots
\usepgfplotslibrary{fillbetween} %% to shade area between curves in TikZ
\usetkzobj{all}
\usepackage[makeroom]{cancel} %% for strike outs
%\usepackage{mathtools} %% for pretty underbrace % Breaks Ximera
%\usepackage{multicol}
\usepackage{pgffor} %% required for integral for loops



%% http://tex.stackexchange.com/questions/66490/drawing-a-tikz-arc-specifying-the-center
%% Draws beach ball
\tikzset{pics/carc/.style args={#1:#2:#3}{code={\draw[pic actions] (#1:#3) arc(#1:#2:#3);}}}



\usepackage{array}
\setlength{\extrarowheight}{+.1cm}
\newdimen\digitwidth
\settowidth\digitwidth{9}
\def\divrule#1#2{
\noalign{\moveright#1\digitwidth
\vbox{\hrule width#2\digitwidth}}}






\DeclareMathOperator{\arccot}{arccot}
\DeclareMathOperator{\arcsec}{arcsec}
\DeclareMathOperator{\arccsc}{arccsc}

















%%This is to help with formatting on future title pages.
\newenvironment{sectionOutcomes}{}{}


\outcome{Know the graphs and properties of ``famous'' functions.}

\title[Break-Ground:]{How crazy could it be?}

\begin{document}
\begin{abstract}
  Two young mathematicians think about the plots of functions.
\end{abstract}
\maketitle

Check out this dialogue between two calculus students (based on a true
story):

\begin{dialogue}
\item[Devyn] Riley, do you remember when we first starting graphing
  functions? Like with a ``T-chart?''
\item[Riley] I remember everything.
\item[Devyn] I used to get so excited to plot stuff! I would wonder:
  ``What crazy curve would be drawn this time? What crazy picture will
  I see?''
\item[Riley] Then we learned about the slope-intercept form of a
  line. Good-old
  \[
  y = mx +b.
  \]
\item[Devyn] Yeah, but lines are really boring. What about
  polynomials? What could you tell me about
  \[
  y= 5x^6-5x^5-5x^4+5x^3+x^2 -1
  \]
  just by looking at the equation?
\item[Riley] Hmmmm. I'm not sure\dots
\end{dialogue}

\begin{problem}
  When $x$ is a large number (furthest from zero), which term of
  $5x^6-5x^5-5x^4+5x^3+x^2 -1$ is largest (furthest from zero)?
  \begin{multipleChoice}
    \choice{$-1$}
    \choice{$x^2$}
    \choice{$5x^3$}
    \choice{$-5x^4$}
    \choice{$-5x^5$}
    \choice[correct]{$5x^6$}
  \end{multipleChoice}
\end{problem}

\begin{problem}
  When $x$ is a small number (near zero), which term of
  $5x^6-5x^5-5x^4+5x^3+x^2 -1$ is largest (furthest from zero)?
  \begin{multipleChoice}
    \choice[correct]{$-1$}
    \choice{$x^2$}
    \choice{$5x^3$}
    \choice{$-5x^4$}
    \choice{$-5x^5$}
    \choice{$5x^6$}
  \end{multipleChoice}
\end{problem}


\begin{problem}
  Very roughly speaking, what does the graph of
  $y=5x^6-5x^5-5x^4+5x^3+x^2 -1$ look like?
  \begin{multipleChoice}
    \choice{The graph starts in the lower left and ends in the upper
      right of the plane.}
    \choice{The graph starts in the lower right and ends in the upper
      left of the plane.}
    \choice[correct]{The graph looks something like the letter ``U.''}
    \choice{The graph looks something like an upside down letter ``U.''}
  \end{multipleChoice}
\end{problem}



%\input{../leveledQuestions.tex}


\end{document}
