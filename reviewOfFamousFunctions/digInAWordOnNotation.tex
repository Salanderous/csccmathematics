\documentclass{ximera}


\graphicspath{
  {./}
  {ximeraTutorial/}
  {basicPhilosophy/}
}

\newcommand{\mooculus}{\textsf{\textbf{MOOC}\textnormal{\textsf{ULUS}}}}

\usepackage{tkz-euclide}\usepackage{tikz}
\usepackage{tikz-cd}
\usetikzlibrary{arrows}
\tikzset{>=stealth,commutative diagrams/.cd,
  arrow style=tikz,diagrams={>=stealth}} %% cool arrow head
\tikzset{shorten <>/.style={ shorten >=#1, shorten <=#1 } } %% allows shorter vectors

\usetikzlibrary{backgrounds} %% for boxes around graphs
\usetikzlibrary{shapes,positioning}  %% Clouds and stars
\usetikzlibrary{matrix} %% for matrix
\usepgfplotslibrary{polar} %% for polar plots
\usepgfplotslibrary{fillbetween} %% to shade area between curves in TikZ
\usetkzobj{all}
\usepackage[makeroom]{cancel} %% for strike outs
%\usepackage{mathtools} %% for pretty underbrace % Breaks Ximera
%\usepackage{multicol}
\usepackage{pgffor} %% required for integral for loops



%% http://tex.stackexchange.com/questions/66490/drawing-a-tikz-arc-specifying-the-center
%% Draws beach ball
\tikzset{pics/carc/.style args={#1:#2:#3}{code={\draw[pic actions] (#1:#3) arc(#1:#2:#3);}}}



\usepackage{array}
\setlength{\extrarowheight}{+.1cm}
\newdimen\digitwidth
\settowidth\digitwidth{9}
\def\divrule#1#2{
\noalign{\moveright#1\digitwidth
\vbox{\hrule width#2\digitwidth}}}






\DeclareMathOperator{\arccot}{arccot}
\DeclareMathOperator{\arcsec}{arcsec}
\DeclareMathOperator{\arccsc}{arccsc}

















%%This is to help with formatting on future title pages.
\newenvironment{sectionOutcomes}{}{}


\title{A word on notation}

\begin{document}
\begin{abstract}
  We discuss the notation used for functions.
\end{abstract}
\maketitle

Given a function $f$, we have a way of writing an inverse of $f$,
assuming it exists. Given a point $x$, 
\[
f^{-1}(x) = \text{$y$ such that $y = f(x)$, should it exist.}
\]
On the other hand, given $x$
\[
f(x)^{-1} = \frac{1}{f(x)}.
\]
\begin{warning}
It is not usually the case that 
\[
f^{-1}(x) = f(x)^{-1}.
\]
\end{warning}

This confusing notation is often exacerbated by the fact that 
\[
\sin^2(x) = (\sin(x))^2=\sin(x)\cdot \sin(x)\qquad \text{but} \qquad \sin^{-1}(x)
\ne(\sin(x))^{-1}.
\]

\begin{warning}
  Note that 
  \[
  \sin^{-1}(x)=\arcsin(x)\qquad\text{but}\qquad (\sin(x))^{-1} = \frac
  {1}{\sin(x)}.
  \]
  In the case of trigonometric functions, this confusion can be avoided
  by using the notation $\arcsin$ and so on for other trigonometric
  functions.
\end{warning}
\end{document}



