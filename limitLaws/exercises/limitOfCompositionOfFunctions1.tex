\documentclass{ximera}


\graphicspath{
  {./}
  {ximeraTutorial/}
  {basicPhilosophy/}
}

\newcommand{\mooculus}{\textsf{\textbf{MOOC}\textnormal{\textsf{ULUS}}}}

\usepackage{tkz-euclide}\usepackage{tikz}
\usepackage{tikz-cd}
\usetikzlibrary{arrows}
\tikzset{>=stealth,commutative diagrams/.cd,
  arrow style=tikz,diagrams={>=stealth}} %% cool arrow head
\tikzset{shorten <>/.style={ shorten >=#1, shorten <=#1 } } %% allows shorter vectors

\usetikzlibrary{backgrounds} %% for boxes around graphs
\usetikzlibrary{shapes,positioning}  %% Clouds and stars
\usetikzlibrary{matrix} %% for matrix
\usepgfplotslibrary{polar} %% for polar plots
\usepgfplotslibrary{fillbetween} %% to shade area between curves in TikZ
\usetkzobj{all}
\usepackage[makeroom]{cancel} %% for strike outs
%\usepackage{mathtools} %% for pretty underbrace % Breaks Ximera
%\usepackage{multicol}
\usepackage{pgffor} %% required for integral for loops



%% http://tex.stackexchange.com/questions/66490/drawing-a-tikz-arc-specifying-the-center
%% Draws beach ball
\tikzset{pics/carc/.style args={#1:#2:#3}{code={\draw[pic actions] (#1:#3) arc(#1:#2:#3);}}}



\usepackage{array}
\setlength{\extrarowheight}{+.1cm}
\newdimen\digitwidth
\settowidth\digitwidth{9}
\def\divrule#1#2{
\noalign{\moveright#1\digitwidth
\vbox{\hrule width#2\digitwidth}}}






\DeclareMathOperator{\arccot}{arccot}
\DeclareMathOperator{\arcsec}{arcsec}
\DeclareMathOperator{\arccsc}{arccsc}

















%%This is to help with formatting on future title pages.
\newenvironment{sectionOutcomes}{}{}



\outcome{Calculate limits of composition of functions.}

\author{Nela Lakos}

\begin{document}
\begin{exercise}
Evaluate the limit and justify your answer.\\
\begin{enumerate}
\item
 \[
\lim_{x\to \frac{\pi}{2}}e^{\sin{x}} =\answer{e}  
\] 
Justification:\\[1em]
Notice that $e^{\sin{x}} =f(g(x)) $, where
\[
 f(x)=\answer{e^{x}} \hspace{0.2in}and \hspace{0.2in} g(x)=\answer{\sin{x}}
\] 
and both functions, $f$ and $g$,  are \wordChoice{\choice{not continuous}\choice[correct]{continuous}}  on $ (-\infty,\infty)$.\\
So, we can apply
\wordChoice{\choice[correct]{Composition}\choice{Product}}  limit law 
 \[
\lim_{x\to \frac{\pi}{2}}e^{\sin{x}} = e^{\lim_{x\to \frac{\pi}{2}}\answer{\sin{x}}}
\] 
Since $g$ is \wordChoice{\choice{not continuous}\choice[correct]{continuous}}  at $\frac{\pi}{2}$, we have
 \[
\lim_{x\to \frac{\pi}{2}}\sin{x}=\sin{\Bigl(\answer{\frac{\pi}{2}}\Bigr)}=\answer{1}
\] 
and, therefore
\[
\lim_{x\to \frac{\pi}{2}}e^{\sin{x}} =e^{\answer{1}}  =\answer{e}
\] 
\noindent\rule[0.5ex]{\linewidth}{.2pt}

\item
 \[
\lim_{x\to2}\cos{\Bigl(\pi\frac{3x^{2}-4x+1}{4}\Bigr)} = \answer{-\frac{\sqrt{2}}{2}}
\] 
Justification:\\ 
Notice that $\cos{\Bigl(\pi\frac{3x^{2}-4x+1}{4}\Bigr)}=f(g(x)) $, where
\[
 f(x)=\answer{\cos{x}} \hspace{0.2in}and \hspace{0.2in} g(x)=\answer{\pi\frac{3x^{2}-4x+1}{4}}
\] 
and both functions, $f$ and $g$,  are \wordChoice{\choice{not continuous}\choice[correct]{continuous}}  on $ (-\infty,\infty)$.\\
So, we can apply
\wordChoice{\choice[correct]{Composition}\choice{Product}}  limit law 
 \[
\lim_{x\to2}\cos{\Bigl(\pi\frac{3x^{2}-4x+1}{4}\Bigr)}  =\cos{\Bigl(\lim_{x\to2}\answer{\pi\frac{3x^{2}-4x+1}{4}}\Bigr)} 
\] 

Since $g$ is \wordChoice{\choice{not continuous}\choice[correct]{continuous}}  at $2$, we have
 \[
\lim_{x\to2}\pi\frac{3x^{2}-4x+1}{4}=\pi\frac{3\answer{2}^{2}-4\answer{2}+1}{4}=\answer{\frac{5\pi}{4}}
\]
and, therefore
\[
\lim_{x\to2}\cos{\Bigl(\pi\frac{3x^{2}-4x+1}{4}\Bigr)} =\cos{\Bigl(\answer{\frac{5\pi}{4}} \Bigr)}= \answer{-\frac{\sqrt{2}}{2}}
\] 

\noindent\rule[0.5ex]{\linewidth}{.2pt}
\item
 \[
\lim_{x\to e^{3}}(\ln{x}+5)^{2} = \answer{64}
\] 
Justification: Notice that $(\ln{x}+5)^{2} =f(g(x)) $, where
\[
 f(x)=\answer{x^{2}} \hspace{0.2in}and \hspace{0.2in} g(x)=\answer{\ln{x}+5}
\] 
The function $f$ is \wordChoice{\choice{not continuous}\choice[correct]{continuous}}  on $ (-\infty,\infty)$.\\
The function $g$ is a  \wordChoice{\choice{product}\choice[correct]{sum}} of two continuous functions, $\ln{x}$ and $ \answer{5}$.
Therefore, $g$ is  \wordChoice{\choice{not continuous}\choice[correct]{continuous}} on its domain $(\answer{0},\answer{\infty})$.\\
So, we can apply
\wordChoice{\choice[correct]{Composition}\choice{Product}}  limit law 
 \[
\lim_{x\to e^{3}}(\ln{x}+5)^{2} =\Bigl(\lim_{x\to e^{3}}(\ln{x}+5)\Bigr)^{\answer{2}} 
\] 
Since $g$ is \wordChoice{\choice{not continuous}\choice[correct]{continuous}}  at $e^{3}$, we have
 \[
\lim_{x\to e^{3}}(\ln{x}+5)=\ln{(\answer{e^{3}})}+5=\answer{3}+5=\answer{8}
\] 
and, therefore
\[
\lim_{x\to e^{3}}(\ln{x}+5)^{2} = \Bigl(\answer{8}\Bigr)^{2} =\answer{64}
\] 
\noindent\rule[0.5ex]{\linewidth}{.2pt}
\end{enumerate}
\end{exercise}
\end{document}
