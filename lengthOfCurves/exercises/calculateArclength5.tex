\documentclass{ximera}


\graphicspath{
  {./}
  {ximeraTutorial/}
  {basicPhilosophy/}
}

\newcommand{\mooculus}{\textsf{\textbf{MOOC}\textnormal{\textsf{ULUS}}}}

\usepackage{tkz-euclide}\usepackage{tikz}
\usepackage{tikz-cd}
\usetikzlibrary{arrows}
\tikzset{>=stealth,commutative diagrams/.cd,
  arrow style=tikz,diagrams={>=stealth}} %% cool arrow head
\tikzset{shorten <>/.style={ shorten >=#1, shorten <=#1 } } %% allows shorter vectors

\usetikzlibrary{backgrounds} %% for boxes around graphs
\usetikzlibrary{shapes,positioning}  %% Clouds and stars
\usetikzlibrary{matrix} %% for matrix
\usepgfplotslibrary{polar} %% for polar plots
\usepgfplotslibrary{fillbetween} %% to shade area between curves in TikZ
\usetkzobj{all}
\usepackage[makeroom]{cancel} %% for strike outs
%\usepackage{mathtools} %% for pretty underbrace % Breaks Ximera
%\usepackage{multicol}
\usepackage{pgffor} %% required for integral for loops



%% http://tex.stackexchange.com/questions/66490/drawing-a-tikz-arc-specifying-the-center
%% Draws beach ball
\tikzset{pics/carc/.style args={#1:#2:#3}{code={\draw[pic actions] (#1:#3) arc(#1:#2:#3);}}}



\usepackage{array}
\setlength{\extrarowheight}{+.1cm}
\newdimen\digitwidth
\settowidth\digitwidth{9}
\def\divrule#1#2{
\noalign{\moveright#1\digitwidth
\vbox{\hrule width#2\digitwidth}}}






\DeclareMathOperator{\arccot}{arccot}
\DeclareMathOperator{\arcsec}{arcsec}
\DeclareMathOperator{\arccsc}{arccsc}

















%%This is to help with formatting on future title pages.
\newenvironment{sectionOutcomes}{}{}


\author{Jim Talamo}
\license{Creative Commons 3.0 By-NC}


\outcome{Set up an integral that gives the length of a curve segment and evaluate it}

\begin{document}
\begin{exercise}

The graph of $x^2+y^2=r^2$ is a circle of radius $r$ centered at the origin.  From geometry, we know that the circumference of this circle is $\answer{2 \pi r}$ (type your answer in terms of $r$)

\begin{exercise}
We can set up an integral with respect to $x$ or $y$ that verifies this result.  First, we notice that the circumference of the circle is $\answer{4}$ times the circumference of the part of the circle in the first quadrant.

In the first quadrant, we can express the circle as a function of $x$ by $y=\answer{\sqrt{r^2-x^2}}$.

\begin{exercise}
Computing the derivative:

\[
\frac{dy}{dx} = -\frac{x}{\sqrt{r^2-x^2}}
\]

\begin{exercise}

After some algebra, we find:

\[
1+ \left(\frac{dy}{dx} \right)^2 = \frac{r^2}{r^2-x^2}
\]

\begin{hint}
First, find $\left(\frac{dy}{dx} \right)^2$ and simplify it.  Then, find a common denominator and add the fractions.
\end{hint}

\begin{exercise}

The integral with respect to $x$ that gives the length of the curve \emph{in the first quadrant} is:

\[
s= \int_{x=\answer{0}}^{x=\answer{r}} \answer{\frac{r}{\sqrt{r^2-x^2}}} dx
\]

\begin{exercise}
Referring to a table, if necessary, $\int \frac{r}{\sqrt{r^2-x^2}} = \answer{r \arcsin(x/r)}$+C.  Thus,

\[
s= \eval{\answer{r \arcsin(x/r)}}_0^r
\]

Evaluating this gives that the length of the portion of the curve in the first quadrant is $\answer{\frac{\pi}{2}r}$ units. (type your answer in terms of $r$)

\begin{exercise}
Thus, the circumference of the circle is $\answer{2 \pi r}$ units. (type your answer in terms of $r$)

\end{exercise}

\end{exercise}
\end{exercise}
\end{exercise}
\end{exercise}
\end{exercise}
\end{exercise}
\end{document}
