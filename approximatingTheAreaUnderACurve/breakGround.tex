\documentclass{ximera}


\graphicspath{
  {./}
  {ximeraTutorial/}
  {basicPhilosophy/}
}

\newcommand{\mooculus}{\textsf{\textbf{MOOC}\textnormal{\textsf{ULUS}}}}

\usepackage{tkz-euclide}\usepackage{tikz}
\usepackage{tikz-cd}
\usetikzlibrary{arrows}
\tikzset{>=stealth,commutative diagrams/.cd,
  arrow style=tikz,diagrams={>=stealth}} %% cool arrow head
\tikzset{shorten <>/.style={ shorten >=#1, shorten <=#1 } } %% allows shorter vectors

\usetikzlibrary{backgrounds} %% for boxes around graphs
\usetikzlibrary{shapes,positioning}  %% Clouds and stars
\usetikzlibrary{matrix} %% for matrix
\usepgfplotslibrary{polar} %% for polar plots
\usepgfplotslibrary{fillbetween} %% to shade area between curves in TikZ
\usetkzobj{all}
\usepackage[makeroom]{cancel} %% for strike outs
%\usepackage{mathtools} %% for pretty underbrace % Breaks Ximera
%\usepackage{multicol}
\usepackage{pgffor} %% required for integral for loops



%% http://tex.stackexchange.com/questions/66490/drawing-a-tikz-arc-specifying-the-center
%% Draws beach ball
\tikzset{pics/carc/.style args={#1:#2:#3}{code={\draw[pic actions] (#1:#3) arc(#1:#2:#3);}}}



\usepackage{array}
\setlength{\extrarowheight}{+.1cm}
\newdimen\digitwidth
\settowidth\digitwidth{9}
\def\divrule#1#2{
\noalign{\moveright#1\digitwidth
\vbox{\hrule width#2\digitwidth}}}






\DeclareMathOperator{\arccot}{arccot}
\DeclareMathOperator{\arcsec}{arcsec}
\DeclareMathOperator{\arccsc}{arccsc}

















%%This is to help with formatting on future title pages.
\newenvironment{sectionOutcomes}{}{}


\outcome{Define area.}

\title[Break-Ground:]{What is area?}

\begin{document}
\begin{abstract}
Two young mathematicians discuss the idea of area.
\end{abstract}
\maketitle

% Introduce Riemann sums as an approximation of antiderivatives

% Think more broadly than area under the curve

% Think about multiplication: we don't just do an "area" or "array" model for multiplication

Check out this dialogue between two calculus students (based on a true
story):

\begin{dialogue}
\item[Devyn] Riley, I have a troubling question.
\item[Riley] Yes? 
\item[Devyn] What is area?
\item[Riley] Oh, for a rectangle it is just
  \[
  \text{height}
  \times
  \text{width}
  \]
\item[Devyn] Right. But shouldn't it mean more?
\end{dialogue}

Area is a funny thing. What is it really? Well the idea is this:
\begin{quote}
Define something as having a ``unit'' area, and see how many times it
``covers'' something.
\end{quote}
The most obivious thing to use for our ``unit'' area is a unit
square. From this we can quickly move on to find the area of any
rectangle as
  \[
  \text{height}
  \times
  \text{width}
  \]
\begin{problem}
  Sometimes area is described as ``how much paint will cover a surface.''
  Is this accurate? What do you think?
  \begin{freeResponse}
  \end{freeResponse}
\end{problem}
%\input{../leveledQuestions.tex}

\end{document}
