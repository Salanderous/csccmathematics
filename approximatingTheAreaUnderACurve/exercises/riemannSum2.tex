\documentclass{ximera}


\graphicspath{
  {./}
  {ximeraTutorial/}
  {basicPhilosophy/}
}

\newcommand{\mooculus}{\textsf{\textbf{MOOC}\textnormal{\textsf{ULUS}}}}

\usepackage{tkz-euclide}\usepackage{tikz}
\usepackage{tikz-cd}
\usetikzlibrary{arrows}
\tikzset{>=stealth,commutative diagrams/.cd,
  arrow style=tikz,diagrams={>=stealth}} %% cool arrow head
\tikzset{shorten <>/.style={ shorten >=#1, shorten <=#1 } } %% allows shorter vectors

\usetikzlibrary{backgrounds} %% for boxes around graphs
\usetikzlibrary{shapes,positioning}  %% Clouds and stars
\usetikzlibrary{matrix} %% for matrix
\usepgfplotslibrary{polar} %% for polar plots
\usepgfplotslibrary{fillbetween} %% to shade area between curves in TikZ
\usetkzobj{all}
\usepackage[makeroom]{cancel} %% for strike outs
%\usepackage{mathtools} %% for pretty underbrace % Breaks Ximera
%\usepackage{multicol}
\usepackage{pgffor} %% required for integral for loops



%% http://tex.stackexchange.com/questions/66490/drawing-a-tikz-arc-specifying-the-center
%% Draws beach ball
\tikzset{pics/carc/.style args={#1:#2:#3}{code={\draw[pic actions] (#1:#3) arc(#1:#2:#3);}}}



\usepackage{array}
\setlength{\extrarowheight}{+.1cm}
\newdimen\digitwidth
\settowidth\digitwidth{9}
\def\divrule#1#2{
\noalign{\moveright#1\digitwidth
\vbox{\hrule width#2\digitwidth}}}






\DeclareMathOperator{\arccot}{arccot}
\DeclareMathOperator{\arcsec}{arcsec}
\DeclareMathOperator{\arccsc}{arccsc}

















%%This is to help with formatting on future title pages.
\newenvironment{sectionOutcomes}{}{}


%\outcome{Define area.}
\outcome{Understand the relationship between area under a curve and sums of rectangles.}
\outcome{Approximate area under a curve.}
\outcome{Compute left, right, and midpoint Riemann sums with 10 or fewer rectangles.}

\author{Nela Lakos \and Kyle Parsons}

\begin{document}
\begin{exercise}

Consider the function
\[
\sqrt{4-x^2}
\]
on the interval $[-2,2]$.

Consider approximating the area between the graph of $f$ and the $x$-axis using a Midpoint Riemann sum with $n=4$ subintervals.  In this case
\begin{align*}
\Delta x &= \answer{1}\\
x_1^* &= \answer{\frac{-3}{2}}\\
x_2^* &= \answer{\frac{-1}{2}}\\
x_3^* &= \answer{\frac{1}{2}}\text{ and}\\
x_4^* &= \answer{\frac{3}{2}}.
\end{align*}

This gives us that our Midpoint Riemann sum is 
\[
\answer{\sqrt{7}+\sqrt{15}}.
\]
Approximated to two decimals, this is $\answer{6.52}$.

Now consider approximating the same area with a Left Riemann sum with $n=4$ subintervals.  In this case
\begin{align*}
\Delta x &= \answer{1}\\
x_1^* &= \answer{-2}\\
x_2^* &= \answer{-1}\\
x_3^* &= \answer{0}\text{ and}\\
x_4^* &= \answer{1}.
\end{align*}

This gives us that our Left Riemann sum is
\[
\answer{2+2\sqrt{3}}.
\]
Approximated to two decimals, this is $\answer{5.46}$.

Finally, the exact area between the graph of $f$ and the $x$-axis on the interval $[-2,2]$ is
\[
\answer{2\pi}
\]

\begin{hint}
Graph $f$. What shape is its graph?
\end{hint}

\end{exercise}
\end{document}