\documentclass{ximera}


\graphicspath{
  {./}
  {ximeraTutorial/}
  {basicPhilosophy/}
}

\newcommand{\mooculus}{\textsf{\textbf{MOOC}\textnormal{\textsf{ULUS}}}}

\usepackage{tkz-euclide}\usepackage{tikz}
\usepackage{tikz-cd}
\usetikzlibrary{arrows}
\tikzset{>=stealth,commutative diagrams/.cd,
  arrow style=tikz,diagrams={>=stealth}} %% cool arrow head
\tikzset{shorten <>/.style={ shorten >=#1, shorten <=#1 } } %% allows shorter vectors

\usetikzlibrary{backgrounds} %% for boxes around graphs
\usetikzlibrary{shapes,positioning}  %% Clouds and stars
\usetikzlibrary{matrix} %% for matrix
\usepgfplotslibrary{polar} %% for polar plots
\usepgfplotslibrary{fillbetween} %% to shade area between curves in TikZ
\usetkzobj{all}
\usepackage[makeroom]{cancel} %% for strike outs
%\usepackage{mathtools} %% for pretty underbrace % Breaks Ximera
%\usepackage{multicol}
\usepackage{pgffor} %% required for integral for loops



%% http://tex.stackexchange.com/questions/66490/drawing-a-tikz-arc-specifying-the-center
%% Draws beach ball
\tikzset{pics/carc/.style args={#1:#2:#3}{code={\draw[pic actions] (#1:#3) arc(#1:#2:#3);}}}



\usepackage{array}
\setlength{\extrarowheight}{+.1cm}
\newdimen\digitwidth
\settowidth\digitwidth{9}
\def\divrule#1#2{
\noalign{\moveright#1\digitwidth
\vbox{\hrule width#2\digitwidth}}}






\DeclareMathOperator{\arccot}{arccot}
\DeclareMathOperator{\arcsec}{arcsec}
\DeclareMathOperator{\arccsc}{arccsc}

















%%This is to help with formatting on future title pages.
\newenvironment{sectionOutcomes}{}{}


%\outcome{Define area.}
\outcome{Understand the relationship between area under a curve and sums of rectangles.}
\outcome{Approximate area under a curve.}
\outcome{Compute left, right, and midpoint Riemann sums with 10 or fewer rectangles.}

\author{Nela Lakos \and Kyle Parsons}

\begin{document}
\begin{exercise}

Consider the function
\[
f(x) = 36-x^2
\]
on the interval $[0,6]$.  We will approximate the area of the region bounded by the graph of $f$ and the $x$-axis.  See the figure below.

\begin{image}
  \begin{tikzpicture}
    \begin{axis}[
        xmin=-0.3,xmax=6.3,ymin=-0.3,ymax=36.3,
        clip=true,
        unit vector ratio*=6 1 1,
        axis lines=center,
        grid = major,
        ytick={0,4,...,36},
        xtick={0,0.5,...,6},
        xlabel=$x$, ylabel=$y$,
        every axis y label/.style={at=(current axis.above origin),anchor=south},
        every axis x label/.style={at=(current axis.right of origin),anchor=west},
      ]      
      \pgfplotsinvokeforeach{0.5,1,...,6}{\draw[very thick,penColor2,fill,fill opacity=0.3] (axis cs:{#1-0.5},0) rectangle (axis cs:{#1},{36-#1^2});}
      
      \addplot[very thick,penColor,domain=0:6] plot{36-x^2};
      
      \node at (axis cs:4,30) {$y=f(x)$};
      \end{axis}`
  \end{tikzpicture}
\end{image}

The image depicts a \wordChoice{\choice{Left}\choice[correct]{Right}\choice{Midpoint}} Riemann sum with $n=\answer{12}$ rectangles.

\begin{align*}
x_6^* &= \answer{3}\\
f(x_6^*) &= \answer{27}\\
\Delta x &= \answer{\frac{1}{2}}\\
\sum_{k=1}^n f(x_k^*)\Delta x &= \answer{\frac{539}{4}}
\end{align*}

\end{exercise}

\end{document}
