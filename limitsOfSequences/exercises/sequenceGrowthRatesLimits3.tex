\documentclass{ximera}


\graphicspath{
  {./}
  {ximeraTutorial/}
  {basicPhilosophy/}
}

\newcommand{\mooculus}{\textsf{\textbf{MOOC}\textnormal{\textsf{ULUS}}}}

\usepackage{tkz-euclide}\usepackage{tikz}
\usepackage{tikz-cd}
\usetikzlibrary{arrows}
\tikzset{>=stealth,commutative diagrams/.cd,
  arrow style=tikz,diagrams={>=stealth}} %% cool arrow head
\tikzset{shorten <>/.style={ shorten >=#1, shorten <=#1 } } %% allows shorter vectors

\usetikzlibrary{backgrounds} %% for boxes around graphs
\usetikzlibrary{shapes,positioning}  %% Clouds and stars
\usetikzlibrary{matrix} %% for matrix
\usepgfplotslibrary{polar} %% for polar plots
\usepgfplotslibrary{fillbetween} %% to shade area between curves in TikZ
\usetkzobj{all}
\usepackage[makeroom]{cancel} %% for strike outs
%\usepackage{mathtools} %% for pretty underbrace % Breaks Ximera
%\usepackage{multicol}
\usepackage{pgffor} %% required for integral for loops



%% http://tex.stackexchange.com/questions/66490/drawing-a-tikz-arc-specifying-the-center
%% Draws beach ball
\tikzset{pics/carc/.style args={#1:#2:#3}{code={\draw[pic actions] (#1:#3) arc(#1:#2:#3);}}}



\usepackage{array}
\setlength{\extrarowheight}{+.1cm}
\newdimen\digitwidth
\settowidth\digitwidth{9}
\def\divrule#1#2{
\noalign{\moveright#1\digitwidth
\vbox{\hrule width#2\digitwidth}}}






\DeclareMathOperator{\arccot}{arccot}
\DeclareMathOperator{\arcsec}{arcsec}
\DeclareMathOperator{\arccsc}{arccsc}

















%%This is to help with formatting on future title pages.
\newenvironment{sectionOutcomes}{}{}


\author{Jim Talamo}
\license{Creative Commons 3.0 By-bC}


\outcome{}


\begin{document}
\begin{exercise}

Find the limits of the following sequences by inspection.  If a limit is infinite, use either ``$+\infty$" or ``$-\infty$".  If the limit otherwise does not exist, write ``DNE".  To earn credit, you must type these \emph{exactly} as they appear!

\begin{exercise}
Consider$\{a_n \}_{n=1}$ where $a_n = \frac{n^{100}+n!}{ne^n+n^n}$.  Then:
\[
\lim_{n \to \infty} a_n = \answer{0}
\]
\end{exercise}

\begin{exercise}
Consider$\{b_n \}_{n=1}$ where $b_n = \frac{n^2+2^n}{\sqrt{4^n+\ln(n)}}$.  Then:
\[
\lim_{n \to \infty} b_n = \answer{1}
\]
\begin{hint}
Factor a $4^n$ out of the expression under the square root in denominator:
\[
\sqrt{4^n+\ln(n)} = \sqrt{4^n \left(\answer{1+ \frac{\ln(n)}{4^n}}\right)}
\]

\begin{question}
Using growth rates, $\lim_{n \to \infty} \frac{\ln(n)}{4^n} = \answer{0}$.

\begin{question}
So, the denominator should behave like $\sqrt{4^n}$ as $n$ grows large. Using laws of exponents:
\[
\sqrt{4^n} = (\answer{2})^n
\]
\end{question}
\end{question}
\end{hint}

\end{exercise}

\begin{exercise}
Consider$\{c_n \}_{n=1}$ where $c_n = \frac{n^4+5n}{n(2n-1)^3}$.  Then:
\[
\lim_{n \to \infty} c_n = \answer{\frac{1}{8}}
\]
\end{exercise}

\begin{exercise}
Consider$\{d_n \}_{n=1}$ where $d_n = \frac{n^{1.06}}{(\ln(n))^{2000}}$.  Then:
\[
\lim_{n \to \infty} d_n =  \answer{+\infty}
\]
\end{exercise}


\end{exercise}
\end{document}
