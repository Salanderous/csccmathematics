\documentclass{ximera}


\graphicspath{
  {./}
  {ximeraTutorial/}
  {basicPhilosophy/}
}

\newcommand{\mooculus}{\textsf{\textbf{MOOC}\textnormal{\textsf{ULUS}}}}

\usepackage{tkz-euclide}\usepackage{tikz}
\usepackage{tikz-cd}
\usetikzlibrary{arrows}
\tikzset{>=stealth,commutative diagrams/.cd,
  arrow style=tikz,diagrams={>=stealth}} %% cool arrow head
\tikzset{shorten <>/.style={ shorten >=#1, shorten <=#1 } } %% allows shorter vectors

\usetikzlibrary{backgrounds} %% for boxes around graphs
\usetikzlibrary{shapes,positioning}  %% Clouds and stars
\usetikzlibrary{matrix} %% for matrix
\usepgfplotslibrary{polar} %% for polar plots
\usepgfplotslibrary{fillbetween} %% to shade area between curves in TikZ
\usetkzobj{all}
\usepackage[makeroom]{cancel} %% for strike outs
%\usepackage{mathtools} %% for pretty underbrace % Breaks Ximera
%\usepackage{multicol}
\usepackage{pgffor} %% required for integral for loops



%% http://tex.stackexchange.com/questions/66490/drawing-a-tikz-arc-specifying-the-center
%% Draws beach ball
\tikzset{pics/carc/.style args={#1:#2:#3}{code={\draw[pic actions] (#1:#3) arc(#1:#2:#3);}}}



\usepackage{array}
\setlength{\extrarowheight}{+.1cm}
\newdimen\digitwidth
\settowidth\digitwidth{9}
\def\divrule#1#2{
\noalign{\moveright#1\digitwidth
\vbox{\hrule width#2\digitwidth}}}






\DeclareMathOperator{\arccot}{arccot}
\DeclareMathOperator{\arcsec}{arcsec}
\DeclareMathOperator{\arccsc}{arccsc}

















%%This is to help with formatting on future title pages.
\newenvironment{sectionOutcomes}{}{}


\author{Jim Talamo}
\license{Creative Commons 3.0 By-bC}


\outcome{}


\begin{document}
\begin{exercise}


Consider$\{a_n \}_{n=1}$ where $a_n = \left(1-\frac{2}{n}\right)^{2n}$.  Then:
\[
\lim_{n \to \infty} a_n = \answer{e^{-4}}
\]

\begin{hint}
This limit has indeterminate form $\answer{1}^{\answer{\infty}}$
(Use ``$\infty$" or ``$-\infty"$ where appropriate)
Set $L = \lim_{n \to \infty} a_n$.  Taking the logarithm of both sides gives:

\[
\ln L = \ln  \left( \lim_{n \to \infty} \left(1-\frac{2}{n}\right)^{2n} \right) = \lim_{n \to \infty} \ln \left(1-\frac{2}{n}\right)^{2n} 
\]

Using the properties of logarithms:
\[
\ln L = \lim_{n \to \infty} \answer{2n} \ln \left(1-\frac{2}{n}\right) 
\]
\begin{question}
This limit has indeterminate form $\answer{\infty} \cdot \answer{0}$. To convert this, we can move one of the terms into the denominator of the denominator:

\[
\ln L = \lim_{n \to \infty} \frac{2 \ln \left(1-\frac{2}{n}\right)}{\answer{\frac{1}{n}}} 
\]

\begin{question}
Now, the limit has the indeterminate form: $\frac{\answer{0}}{\answer{0}}$ and we can use L'Hopital's Rule to write:
\[
\ln L = \lim_{n \to \infty} \frac{2 \ln \left(1-\frac{2}{n}\right)}{\answer{\frac{1}{n}}} =  \lim_{n \to \infty} \frac{\answer{\frac{2}{1-\frac{2}{n}} \left( \frac{2}{n^2}\right)}}{\answer{-\frac{1}{n^2}}} 
\]

Evaluating this gives: $\ln L = -4$.  Hence, $L = \answer{e^{-4}}$.



\end{question}
\end{question}
\end{hint}
\end{exercise}
\end{document}
