\documentclass{ximera}


\graphicspath{
  {./}
  {ximeraTutorial/}
  {basicPhilosophy/}
}

\newcommand{\mooculus}{\textsf{\textbf{MOOC}\textnormal{\textsf{ULUS}}}}

\usepackage{tkz-euclide}\usepackage{tikz}
\usepackage{tikz-cd}
\usetikzlibrary{arrows}
\tikzset{>=stealth,commutative diagrams/.cd,
  arrow style=tikz,diagrams={>=stealth}} %% cool arrow head
\tikzset{shorten <>/.style={ shorten >=#1, shorten <=#1 } } %% allows shorter vectors

\usetikzlibrary{backgrounds} %% for boxes around graphs
\usetikzlibrary{shapes,positioning}  %% Clouds and stars
\usetikzlibrary{matrix} %% for matrix
\usepgfplotslibrary{polar} %% for polar plots
\usepgfplotslibrary{fillbetween} %% to shade area between curves in TikZ
\usetkzobj{all}
\usepackage[makeroom]{cancel} %% for strike outs
%\usepackage{mathtools} %% for pretty underbrace % Breaks Ximera
%\usepackage{multicol}
\usepackage{pgffor} %% required for integral for loops



%% http://tex.stackexchange.com/questions/66490/drawing-a-tikz-arc-specifying-the-center
%% Draws beach ball
\tikzset{pics/carc/.style args={#1:#2:#3}{code={\draw[pic actions] (#1:#3) arc(#1:#2:#3);}}}



\usepackage{array}
\setlength{\extrarowheight}{+.1cm}
\newdimen\digitwidth
\settowidth\digitwidth{9}
\def\divrule#1#2{
\noalign{\moveright#1\digitwidth
\vbox{\hrule width#2\digitwidth}}}






\DeclareMathOperator{\arccot}{arccot}
\DeclareMathOperator{\arcsec}{arcsec}
\DeclareMathOperator{\arccsc}{arccsc}

















%%This is to help with formatting on future title pages.
\newenvironment{sectionOutcomes}{}{}


\author{Jim Talamo}
\license{Creative Commons 3.0 By-bC}


\outcome{}


\begin{document}
\begin{exercise}
 We study two sequences that are closely related. 
 
Consider$\{a_n \}_{n=1}$ where $a_n = \ln\left(n^{1/n}\right)$.  Then:
\[
\lim_{n \to \infty} a_n = \answer{0}
\]

\begin{hint}
Using the properties of logarithms, we can write:

\[
a_n= \left(\answer{\frac{1}{n}}\right)ln(n)
\]
\end{hint}

Now, Consider$\{b_n \}_{n=1}$ where $b_n = (\ln(n))^{1/n}$.  Then:
\[
\lim_{n \to \infty} a_n = \answer{1}
\]

\begin{hint}
To convert the exponential indeterminate form into one we can handle, we set $L = \lim_{n \to \infty} b_n$ first take the natural logarithm of each side:


\[
\ln L= \ln \left(\lim_{n \to \infty} (\ln(n))^{1/n}\right) = \lim_{n \to \infty}  \ln \left((\ln(n))^{1/n}\right)
\]
\begin{question}

Using the properties of logarithms:

\[
\ln L= \lim_{n \to \infty} \answer{\frac{1}{n}} \ln(\ln(n))
\]
\begin{question}

Manipulate this into a form for which we can use L'Hopital's rule:

\[
\ln L = \lim_{n \to \infty} \frac{\answer{\ln(\ln(n))}}{\answer{n}}
\]

\begin{question}
Evaluating this limit gives:

\[
\ln L = \answer{0}
\]

Hence, $L= \lim_{n \to \infty} b_n = \answer{1}$.

\end{question}
\end{question}
\end{question}
\end{hint}


\end{exercise}
\end{document}
