\documentclass{ximera}


\graphicspath{
  {./}
  {ximeraTutorial/}
  {basicPhilosophy/}
}

\newcommand{\mooculus}{\textsf{\textbf{MOOC}\textnormal{\textsf{ULUS}}}}

\usepackage{tkz-euclide}\usepackage{tikz}
\usepackage{tikz-cd}
\usetikzlibrary{arrows}
\tikzset{>=stealth,commutative diagrams/.cd,
  arrow style=tikz,diagrams={>=stealth}} %% cool arrow head
\tikzset{shorten <>/.style={ shorten >=#1, shorten <=#1 } } %% allows shorter vectors

\usetikzlibrary{backgrounds} %% for boxes around graphs
\usetikzlibrary{shapes,positioning}  %% Clouds and stars
\usetikzlibrary{matrix} %% for matrix
\usepgfplotslibrary{polar} %% for polar plots
\usepgfplotslibrary{fillbetween} %% to shade area between curves in TikZ
\usetkzobj{all}
\usepackage[makeroom]{cancel} %% for strike outs
%\usepackage{mathtools} %% for pretty underbrace % Breaks Ximera
%\usepackage{multicol}
\usepackage{pgffor} %% required for integral for loops



%% http://tex.stackexchange.com/questions/66490/drawing-a-tikz-arc-specifying-the-center
%% Draws beach ball
\tikzset{pics/carc/.style args={#1:#2:#3}{code={\draw[pic actions] (#1:#3) arc(#1:#2:#3);}}}



\usepackage{array}
\setlength{\extrarowheight}{+.1cm}
\newdimen\digitwidth
\settowidth\digitwidth{9}
\def\divrule#1#2{
\noalign{\moveright#1\digitwidth
\vbox{\hrule width#2\digitwidth}}}






\DeclareMathOperator{\arccot}{arccot}
\DeclareMathOperator{\arcsec}{arcsec}
\DeclareMathOperator{\arccsc}{arccsc}

















%%This is to help with formatting on future title pages.
\newenvironment{sectionOutcomes}{}{}


\author{Jim Talamo}
\license{Creative Commons 3.0 By-bC}


\outcome{}


\begin{document}
\begin{exercise}

Given a sequence $\{a_n\}_{n=1}$, there are two limits that can be constructed from it that play an important role later on.  This exercise gives practice constructing and computing them.

Consider the sequence $\{a_n \}_{n=1}$, where $a_n =n \cdot 2^n$.  Then:
\[
\lim_{n \to \infty} \frac{a_{n+1}}{a_n} = \answer{2}
\]

\begin{hint}
The limit that must be computed is:

\[
\lim_{n \to \infty} \frac{a_{n+1}}{a_n} = \lim_{n \to \infty} \frac{\answer{(n+1) 2^{n+1}}}{\answer{n 2^n}}
\]

\end{hint}

\[
\lim_{n \to \infty} \sqrt[n]{a_n} = \answer{2}
\]
\begin{hint}
The limit that must be computed is:

\[
\lim_{n \to \infty} \sqrt[n]{a_n} = \lim_{n \to \infty} \sqrt[n]{n \cdot 2^n} 
\]
Simplifying this gives:
\[
\lim_{n \to \infty} \sqrt[n]{a_n} = \lim_{n \to \infty} \answer{2} \sqrt[n]{\answer{n}} 
\]

To compute the limit, set $L = \lim_{n \to \infty} \sqrt[n]{a_n}$, take the natural logarithm of both sides and use the properties of logarithms to bring the limit into a form where L'Hopital's rule applies.
\end{hint}
\end{exercise}
\end{document}
