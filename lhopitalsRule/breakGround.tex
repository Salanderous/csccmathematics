\documentclass{ximera}


\graphicspath{
  {./}
  {ximeraTutorial/}
  {basicPhilosophy/}
}

\newcommand{\mooculus}{\textsf{\textbf{MOOC}\textnormal{\textsf{ULUS}}}}

\usepackage{tkz-euclide}\usepackage{tikz}
\usepackage{tikz-cd}
\usetikzlibrary{arrows}
\tikzset{>=stealth,commutative diagrams/.cd,
  arrow style=tikz,diagrams={>=stealth}} %% cool arrow head
\tikzset{shorten <>/.style={ shorten >=#1, shorten <=#1 } } %% allows shorter vectors

\usetikzlibrary{backgrounds} %% for boxes around graphs
\usetikzlibrary{shapes,positioning}  %% Clouds and stars
\usetikzlibrary{matrix} %% for matrix
\usepgfplotslibrary{polar} %% for polar plots
\usepgfplotslibrary{fillbetween} %% to shade area between curves in TikZ
\usetkzobj{all}
\usepackage[makeroom]{cancel} %% for strike outs
%\usepackage{mathtools} %% for pretty underbrace % Breaks Ximera
%\usepackage{multicol}
\usepackage{pgffor} %% required for integral for loops



%% http://tex.stackexchange.com/questions/66490/drawing-a-tikz-arc-specifying-the-center
%% Draws beach ball
\tikzset{pics/carc/.style args={#1:#2:#3}{code={\draw[pic actions] (#1:#3) arc(#1:#2:#3);}}}



\usepackage{array}
\setlength{\extrarowheight}{+.1cm}
\newdimen\digitwidth
\settowidth\digitwidth{9}
\def\divrule#1#2{
\noalign{\moveright#1\digitwidth
\vbox{\hrule width#2\digitwidth}}}






\DeclareMathOperator{\arccot}{arccot}
\DeclareMathOperator{\arcsec}{arcsec}
\DeclareMathOperator{\arccsc}{arccsc}

















%%This is to help with formatting on future title pages.
\newenvironment{sectionOutcomes}{}{}


\outcome{Demonstrate that we can only sometimes find a limit value by 
	finding the limit of quotient of the derivative of the numerator and 
	derivative of the denominator.}
\outcome{Explore the hypothesis of L'Hospital's Rule.}
\outcome{Make a guess and test this guess.}


\title[Break-Ground:]{A limitless dialogue}

\begin{document}
\begin{abstract}
%Here we see a dialogue where limits are computed using derivatives.
Two young mathematicians consider a way to compute limits using derivatives.
\end{abstract}
\maketitle

Check out this dialogue between two calculus students (based on a true
story):

\begin{dialogue}
\item[Devyn] Yo Riley, guess what I did last night?
\item[Riley] What?
\item[Devyn] I was doing some calculus.
\item[Riley] That. Is. Awesome.
\item[Devyn] I know! Anyway, I noticed something kinda funny. I
  think you can sometimes take limits by taking the derivative of the
  numerator and the denominator.
\item[Riley] That's crazy.
\item[Devyn] I know! But check it:
  \begin{align*}
    \lim_{x\to 0} \frac{\sin(x)}{x} &= \lim_{x\to 0} \frac{\frac{d}{dx}\sin(x)}{\frac{d}{dx} x}\\
    &= \lim_{x\to 0} \frac{\cos(x)}{1}\\
    &=1.
  \end{align*}
  \item[Riley] Woah. That. Is. Awes\dots weird. Hmmm, but it seems like
    cheating. Wait, it doesn't always work, check this out:
    \[
    \lim_{x\to 0} \frac{x^2+1}{x+1} = 1,
    \]
    but
    \begin{align*}
      \lim_{x\to 0} \frac{\frac{d}{dx}\left(x^2+1\right)}{\frac{d}{dx}\left(x+1\right)} &=
      \lim_{x\to 0} \frac{2x}{1} \\
      &=0.
    \end{align*}
\end{dialogue}

\begin{problem}
  Find \textbf{five} examples where this ``trick'' works, and
  \textbf{five} examples where it doesn't work.
  \begin{hint}
    Start with limits of fractions that you know how to compute. Then
    take the derivative of the numerator and the denominator, and see
    if the new limit equals the old limit or not.
  \end{hint}
  \begin{freeResponse}
  \end{freeResponse}
\end{problem}

\begin{problem}
  What is the pattern for when the ``trick'' works and when it does not work?
  \begin{freeResponse}
\end{freeResponse}
\end{problem}

%\input{../leveledQuestions.tex}


\end{document}
