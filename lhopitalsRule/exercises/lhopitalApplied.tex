\documentclass{ximera}


\graphicspath{
  {./}
  {ximeraTutorial/}
  {basicPhilosophy/}
}

\newcommand{\mooculus}{\textsf{\textbf{MOOC}\textnormal{\textsf{ULUS}}}}

\usepackage{tkz-euclide}\usepackage{tikz}
\usepackage{tikz-cd}
\usetikzlibrary{arrows}
\tikzset{>=stealth,commutative diagrams/.cd,
  arrow style=tikz,diagrams={>=stealth}} %% cool arrow head
\tikzset{shorten <>/.style={ shorten >=#1, shorten <=#1 } } %% allows shorter vectors

\usetikzlibrary{backgrounds} %% for boxes around graphs
\usetikzlibrary{shapes,positioning}  %% Clouds and stars
\usetikzlibrary{matrix} %% for matrix
\usepgfplotslibrary{polar} %% for polar plots
\usepgfplotslibrary{fillbetween} %% to shade area between curves in TikZ
\usetkzobj{all}
\usepackage[makeroom]{cancel} %% for strike outs
%\usepackage{mathtools} %% for pretty underbrace % Breaks Ximera
%\usepackage{multicol}
\usepackage{pgffor} %% required for integral for loops



%% http://tex.stackexchange.com/questions/66490/drawing-a-tikz-arc-specifying-the-center
%% Draws beach ball
\tikzset{pics/carc/.style args={#1:#2:#3}{code={\draw[pic actions] (#1:#3) arc(#1:#2:#3);}}}



\usepackage{array}
\setlength{\extrarowheight}{+.1cm}
\newdimen\digitwidth
\settowidth\digitwidth{9}
\def\divrule#1#2{
\noalign{\moveright#1\digitwidth
\vbox{\hrule width#2\digitwidth}}}






\DeclareMathOperator{\arccot}{arccot}
\DeclareMathOperator{\arcsec}{arcsec}
\DeclareMathOperator{\arccsc}{arccsc}

















%%This is to help with formatting on future title pages.
\newenvironment{sectionOutcomes}{}{}


\outcome{Recall how to find limits for forms that are not indeterminate.}
\outcome{Define an indeterminate form.}
\outcome{Determine if a form is indeterminate.}
\outcome{Convert indeterminate forms to the form zero over zero or infinity over infinity.}
\outcome{Define l’Hôpital’s Rule and identify when it can be used.}
\outcome{Use l’Hôpital’s Rule to find limits.}

\author{Nela Lakos}

\begin{document}
\begin{exercise}
The charge in the LCR circuit, with no resistance ($R=0$), with external voltage $E(t)=\sin{(w t)}$, 

and no charge and no current initially is given by

\[
Q_{w}(t)=\frac{w\sin{(w_{0} t)-w_{0}\sin{(w t)}}}{w_{0}(w^2-w_{0}^2)}
\]

where $t$ is time in seconds and $w_{0}=\frac{1}{2}$ is a "natural frequency" of the circuit.

The charge at resonance, $Q(t)$, is defined by the limit

\[
Q(t)=\lim_{w\to w_{0}} Q_{w}(t)=\lim_{w\to w_{0}} \frac{w\sin{(w_{0} t)-w_{0}\sin{(w t)}}}{w_{0}(w^2-w_{0}^2)}
\]

(a) What is the form of the limit?

\[
FORM:\answer{\frac{0}{0}}
\]

(b)  Evaluate the limit to find the expression for $Q(t)$. 

You may use L’Hôpital’s Rule.


Keep in mind that $w_{0}=\frac{1}{2}$ when writing your answer. 

\[
Q(t)=\lim_{w\to w_{0}} Q_{w}(t)=\lim_{w\to w_{0}} \frac{w\sin{(w_{0} t)-w_{0}\sin{(w t)}}}{w_{0}(w^2-w_{0}^2)}=2\sin{\left(\frac{ t}{2}\right)-t\cdot\answer{\cos{\left(\frac{t}{2}\right)}}}
\]



\end{exercise}
\end{document}