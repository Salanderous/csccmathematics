\documentclass{ximera}

\graphicspath{
  {./}
  {ximeraTutorial/}
  {basicPhilosophy/}
}

\newcommand{\mooculus}{\textsf{\textbf{MOOC}\textnormal{\textsf{ULUS}}}}

\usepackage{tkz-euclide}\usepackage{tikz}
\usepackage{tikz-cd}
\usetikzlibrary{arrows}
\tikzset{>=stealth,commutative diagrams/.cd,
  arrow style=tikz,diagrams={>=stealth}} %% cool arrow head
\tikzset{shorten <>/.style={ shorten >=#1, shorten <=#1 } } %% allows shorter vectors

\usetikzlibrary{backgrounds} %% for boxes around graphs
\usetikzlibrary{shapes,positioning}  %% Clouds and stars
\usetikzlibrary{matrix} %% for matrix
\usepgfplotslibrary{polar} %% for polar plots
\usepgfplotslibrary{fillbetween} %% to shade area between curves in TikZ
\usetkzobj{all}
\usepackage[makeroom]{cancel} %% for strike outs
%\usepackage{mathtools} %% for pretty underbrace % Breaks Ximera
%\usepackage{multicol}
\usepackage{pgffor} %% required for integral for loops



%% http://tex.stackexchange.com/questions/66490/drawing-a-tikz-arc-specifying-the-center
%% Draws beach ball
\tikzset{pics/carc/.style args={#1:#2:#3}{code={\draw[pic actions] (#1:#3) arc(#1:#2:#3);}}}



\usepackage{array}
\setlength{\extrarowheight}{+.1cm}
\newdimen\digitwidth
\settowidth\digitwidth{9}
\def\divrule#1#2{
\noalign{\moveright#1\digitwidth
\vbox{\hrule width#2\digitwidth}}}






\DeclareMathOperator{\arccot}{arccot}
\DeclareMathOperator{\arcsec}{arcsec}
\DeclareMathOperator{\arccsc}{arccsc}

















%%This is to help with formatting on future title pages.
\newenvironment{sectionOutcomes}{}{}

\author{Steven Gubkin\and Nela Lakos}
\license{Creative Commons 3.0 By-NC}

\outcome{Recall how to find limits for forms that are not indeterminate.}
\outcome{Define an indeterminate form.}
\outcome{Determine if a form is indeterminate.}
\outcome{Convert indeterminate forms to the form zero over zero or infinity over infinity.}
\outcome{Define L’hopital’s Rule and identify when it can be used.}
\outcome{Use L’hopital’s Rule to find limits.}
\begin{document}
\begin{exercise}

Decide whether l'H\^{o}pital's rule immediately applies to the following limit.  If it does not, explain why not.  Find the limit by any means necessary, or state that it does not exist. 

\[
\lim_{x \to 0} \frac{\sin^2(x)\sin(\frac{1}{x})}{x\sin(\frac{1}{2x})}
\]

\begin{prompt}
	Call the numerator $f(x)$ and the denominator $g(x)$. 

	\begin{multipleChoice}
	\choice{l'H\^{o}pital's rule applies}
	\choice{l'H\^{o}pital's rule does not apply since the limit is not an indeterminate form }
	\choice[correct]{l'H\^{o}pital's rule does not apply because $g'(x)$, where $x\ne0$, has zeros on every interval $(-a, a)$, where $a$ is a small positive number }
	\choice{l'H\^{o}pital's rule does not apply because $\lim_{x \to \infty} \frac{f'(x)}{g'(x)}$ does not exist.}
\end{multipleChoice}

If the limit does not exist, write $DNE$.


\begin{hint}
Recall the Double-angle formula for sine:
\[
\sin{\left(\frac{1}{x}\right)}=2\sin{\left(\frac{1}{2x}\right)}\cos{\left(\frac{1}{2x}\right)}
\]
\end{hint}


\[
\lim_{x \to 0} \frac{\sin^2(x)\sin\left(\frac{1}{x}\right)}{x\sin\left(\frac{1}{2x}\right)} = \answer{0}
\]

\end{prompt}

\end{exercise}
\end{document}