\documentclass{ximera}


\graphicspath{
  {./}
  {ximeraTutorial/}
  {basicPhilosophy/}
}

\newcommand{\mooculus}{\textsf{\textbf{MOOC}\textnormal{\textsf{ULUS}}}}

\usepackage{tkz-euclide}\usepackage{tikz}
\usepackage{tikz-cd}
\usetikzlibrary{arrows}
\tikzset{>=stealth,commutative diagrams/.cd,
  arrow style=tikz,diagrams={>=stealth}} %% cool arrow head
\tikzset{shorten <>/.style={ shorten >=#1, shorten <=#1 } } %% allows shorter vectors

\usetikzlibrary{backgrounds} %% for boxes around graphs
\usetikzlibrary{shapes,positioning}  %% Clouds and stars
\usetikzlibrary{matrix} %% for matrix
\usepgfplotslibrary{polar} %% for polar plots
\usepgfplotslibrary{fillbetween} %% to shade area between curves in TikZ
\usetkzobj{all}
\usepackage[makeroom]{cancel} %% for strike outs
%\usepackage{mathtools} %% for pretty underbrace % Breaks Ximera
%\usepackage{multicol}
\usepackage{pgffor} %% required for integral for loops



%% http://tex.stackexchange.com/questions/66490/drawing-a-tikz-arc-specifying-the-center
%% Draws beach ball
\tikzset{pics/carc/.style args={#1:#2:#3}{code={\draw[pic actions] (#1:#3) arc(#1:#2:#3);}}}



\usepackage{array}
\setlength{\extrarowheight}{+.1cm}
\newdimen\digitwidth
\settowidth\digitwidth{9}
\def\divrule#1#2{
\noalign{\moveright#1\digitwidth
\vbox{\hrule width#2\digitwidth}}}






\DeclareMathOperator{\arccot}{arccot}
\DeclareMathOperator{\arcsec}{arcsec}
\DeclareMathOperator{\arccsc}{arccsc}

















%%This is to help with formatting on future title pages.
\newenvironment{sectionOutcomes}{}{}


\author{Bart Snapp}
\outcome{Give the equation for a sphere or ball.}
\title[Dig-In:]{Drawing a sphere}

\begin{document}
\begin{abstract}
 Learn how to draw a sphere.
\end{abstract}
\maketitle

A key challenge in mathematics is converting formulas and equations
into ideas. We want to get to the point that when you see something
like
\[
(x-1)^2+(y-2)^2+(z-3)^2=16
\]
you say to yourself, ``Hey, that's a sphere of radius $4$ centered at
the point $(1,2,3)$.'' Let's state this generally.
\begin{theorem}
  The set of points on the surface of a sphere of radius $r$ centered
  at $(a,b,c)$ are given by the set:
  \[
  S = \{(x,y,z):(x-a)^2+(y-b)^2+(z-c)^2=r^2\}
  \]
\end{theorem}

As we work in 3-space it is going to be very helpful to be able to draw
some of the sets we commonly encounter.
For now, let me show you how to draw a sphere yourself. Get
out a sheet of paper, and play along---it will be fun! Start by
drawing a set of axes:
\begin{image}
  \begin{tikzpicture}  
    \begin{axis}[  
        xmin=-1.2,  
        xmax=1.2,  
        ymin=-1.2,  
        ymax=1.2,
        unit vector ratio=1 1 1,
        axis lines=center,
        ticks=none,
        xlabel=$y$,  
        ylabel=$z$,  
        every axis y label/.style={at=(current axis.above origin),anchor=south},  
        every axis x label/.style={at=(current axis.right of origin),anchor=west},  
      ]  
      \addplot [->] coordinates {(.7,.5) (-.7,-.5)};
      %\addplot [ultra thick, penColor,domain=0:360,smooth] ({cos(x)},{sin(x)});
      %\addplot [ultra thick, penColor,domain=180:360,smooth] ({cos(x)},{.4*sin(x)});
      %\addplot [ultra thick, dashed, penColor,domain=0:180,smooth] ({cos(x)},{.4*sin(x)});
      \node at (axis cs: -.75,-.52) {$x$};
    \end{axis}  
  \end{tikzpicture}  
\end{image}
Now draw a circle in the $(y,z)$-plane:
\begin{image}
  \begin{tikzpicture}  
    \begin{axis}[  
        xmin=-1.2,  
        xmax=1.2,  
        ymin=-1.2,  
        ymax=1.2,
        unit vector ratio=1 1 1,
        axis lines=center,
        ticks=none,
        xlabel=$y$,  
        ylabel=$z$,  
        every axis y label/.style={at=(current axis.above origin),anchor=south},  
        every axis x label/.style={at=(current axis.right of origin),anchor=west},  
      ]  
      \addplot [->] coordinates {(.7,.5) (-.7,-.5)};
      \addplot [ultra thick, penColor,domain=0:360,smooth] ({cos(x)},{sin(x)});
      %\addplot [ultra thick, penColor,domain=180:360,smooth] ({cos(x)},{.4*sin(x)});
      %\addplot [ultra thick, dashed, penColor,domain=0:180,smooth] ({cos(x)},{.4*sin(x)});
      \node at (axis cs: -.75,-.52) {$x$};
    \end{axis}  
  \end{tikzpicture}  
\end{image}
Pro-tip: If you have trouble drawing a circle, and most people do, try
drawing circles on graph paper. Practice makes perfect, and if you
practice enough, soon you'll be able to impress your friends and
enemies alike with your circle-drawing skills.  Now draw an ellipse,
dashing the part at the ``back'' of the sphere:
\begin{image}
  \begin{tikzpicture}  
    \begin{axis}[  
        xmin=-1.2,  
        xmax=1.2,  
        ymin=-1.2,  
        ymax=1.2,
        unit vector ratio=1 1 1,
        axis lines=center,
        ticks=none,
        xlabel=$y$,  
        ylabel=$z$,  
        every axis y label/.style={at=(current axis.above origin),anchor=south},  
        every axis x label/.style={at=(current axis.right of origin),anchor=west},  
      ]  
      \addplot [->] coordinates {(.7,.5) (-.7,-.5)};
      \addplot [ultra thick, penColor,domain=0:360,smooth] ({cos(x)},{sin(x)});
      \addplot [ultra thick, penColor,domain=180:360,smooth] ({cos(x)},{.4*sin(x)});
      \addplot [ultra thick, dashed, penColor,domain=0:180,smooth] ({cos(x)},{.4*sin(x)});
      \node at (axis cs: -.75,-.52) {$x$};
    \end{axis}  
  \end{tikzpicture}  
\end{image}
And voli\`a, we have a sphere!

Now, back to some equations!  Above we gave an implicit formula for the surface of the
sphere. Sometimes parametric formulas are easier to work with. We'll be
talking about parametric formulas for surfaces a lot in this
course, so consider these equations your introduction. 
The parameters we are going to use now are $\theta$ and $\phi$ as shown here:

\begin{image}
  \begin{tikzpicture}
    \begin{axis}[tick label style={font=\scriptsize},axis on top,
	axis lines=center,
	view={110}{25},
	name=myplot,
	xtick=\empty,
        ytick=\empty,
        ztick=\empty,
	ymin=-.1,ymax=1.2,
	xmin=-.1,xmax=1.2,
	zmin=-.2, zmax=2.1,
	every axis x label/.style={at={(axis cs:\pgfkeysvalueof{/pgfplots/xmax},0,0)},xshift=-1pt,yshift=-4pt},
	xlabel={\scriptsize $x$},
	every axis y label/.style={at={(axis cs:0,\pgfkeysvalueof{/pgfplots/ymax},0)},xshift=5pt,yshift=-3pt},
	ylabel={\scriptsize $y$},
	every axis z label/.style={at={(axis cs:0,0,\pgfkeysvalueof{/pgfplots/zmax})},xshift=0pt,yshift=4pt},
	zlabel={\scriptsize $z$},
        colormap/cool,
      ]
      \addplot3[gray,->,domain=0:45,samples y=0] ({.3*cos(x)},{.3*sin(x)},0); %% angle for theta
      \addplot3[gray,->,domain=0:27,samples y=0] ({.3*cos(45)*sin(x)},{.3*sin(45)*sin(x)},{.3*cos(x)}); %% angle for phi
      \addplot3[gray,domain=0:2.1,samples y=0,->] ({x*cos(45)*sin(27)},{x*sin(45)*sin(27)},{x*cos(27)}); %% line for rho
      
      \addplot3[gray,domain=0:1,samples y=0,dashed] ({x*cos(45)},{x*sin(45)},0); %% line for theta
      \addplot3[gray,domain=0:2,samples y=0,dashed] ({1*cos(45)},{1*sin(45)},x); %% line for z
      \node at (axis cs:{.5*cos(22.5)},{.5*sin(22.5)},0) {$\theta$};
      \node[above] at (axis cs:{.3*cos(45)*sin(17)},{.3*sin(45)*sin(17)},{.3*cos(17)}) {$\varphi$};
      
      \node[above] at (axis cs:{1*cos(45)*sin(27)},{1*sin(45)*sin(27)},{1*cos(27)}) {$r$};
      
      \filldraw [black] (axis cs:{1*cos(45)},{1*sin(45)},1.95) circle (2.5pt);        
      \node[right] at (axis cs:{1*cos(45)},{1*sin(45)},2) {$(x,y,z)$};
    \end{axis}
  \end{tikzpicture}
\end{image}

It turns out that when we use the parameters $\theta$ and $\phi$, the
formulas below give us a sphere.  (Don't worry about where those
formulas come from.  You'll get to that later in your calculus
journey!)
     
\begin{theorem}
  The parametric formula for a sphere of radius $r$ centered at
  $(a,b,c)$ is give by:
  \begin{align*}
    x(\theta,\phi) &=r\cdot\cos(\theta)\sin(\phi)+a\\
    y(\theta,\phi) &=r\cdot\sin(\theta)\sin(\phi)+b\\
    z(\theta,\phi) &=r\cdot\cos(\phi)+c
  \end{align*}
  for $0\le\theta< 2\pi$ and $0\le \phi\le \pi$.
\end{theorem}

Now let me tell you something: people
who like mathematics really like asking (and answering) questions like
the following.

\begin{example}
  Verify that the two descriptions
  \[
  S = \{(x,y,z):(x-a)^2+(y-b)^2+(z-c)^2=r^2\}
  \]
  and
  \begin{align*}
    x(\theta,\phi) &=r\cdot\cos(\theta)\sin(\phi)+a\\
    y(\theta,\phi) &=r\cdot\sin(\theta)\sin(\phi)+b\\
    z(\theta,\phi) &=r\cdot\cos(\phi)+c
  \end{align*}
  where $0\le \theta<2\pi$ and $0\le \phi\le \pi$, describe the same
  geometric set.
  \begin{explanation}
    What we are being asked to do here is verify that both
    descriptions of the sphere in fact describe the same geometric
    object.  How do we verify that we have the same object? First,
    show that the first description ``contains'' the second.  Second,
    show that the second description ``contains'' the first.


    To show that the first description contains the second, we need to
    explain why every point in $S$ is actually drawn by our parametric
    description. Given a point $(x,y,z)$ on the sphere, we need to
    assure our most stubborn readers that there will be a $\theta$ and
    a $\phi$ that when plugged into our formula, will produce the
    given point $(x,y,z)$. We could do this by solving for $\theta$
    and $\phi$ in terms of $x$, $y$, and $z$---though we will not do
    that here. Instead, we will show you \textit{how} the sphere is
    drawn by $\theta$ and $\phi$.
 
      
    \begin{onlineOnly}
      By dragging the sliders around, you should be able to convince
      yourself that every point on the sphere can be hit by some
      choice of $\theta$ and $\phi$.
      \begin{center}
        \geogebra{XC3FXUdJ}{800}{600}%%https://www.geogebra.org/m/XC3FXUdJ
      \end{center}
    \end{onlineOnly}
    Since every point on the sphere can be obtained by letting
    $\theta$ run from $0$ to $2\pi$, and $\phi$ run from $0$ to $\pi$,
    we have shown that the parametric formula draws the entire sphere.
    
    Now we'll show that every point of the parametric formula
    \begin{align*}
      x(\theta,\phi) &=r\cdot\cos(\theta)\sin(\phi)+a\\
      y(\theta,\phi) &=r\cdot\sin(\theta)\sin(\phi)+b\\
      z(\theta,\phi) &=r\cdot\cos(\phi)+c
    \end{align*}
    is also a point of $S$.  This result will tell us that the second
    description contains the first. We do this by plugging the point
    \[
    \big(x(\theta,\phi),y(\theta,\phi),z(\theta,\phi)\big)
    \]
    into 
    \[
    (x-a)^2+(y-b)^2+(z-c)^2
    \]
    and through a manipulation of symbols, conclude that it does equal
    $r^2$. Write with me.
    \begin{align*}
      (x(\theta,\phi)&-a)^2+(y(\theta,\phi)-b)^2+(z(\theta,\phi)-c)^2 \\
      &= \left(\answer[given]{r\cdot\cos(\theta)\sin(\phi)+a}-a\right)^2\\
      &\quad +\left(\answer[given]{r\cdot\sin(\theta)\sin(\phi)+b}-b\right)^2\\
      &\quad +\left(\answer[given]{r\cdot\cos(\phi)+c}-c\right)^2
    \end{align*}
    Simplifying:
    \begin{align*}
      &= \answer[given]{r^2}\left(\cos^2(\theta)\sin^2(\phi) + \sin^2(\theta)\sin^2(\phi) + \cos^2(\phi)\right)\\
      &= \answer[given]{r^2}\left(\sin^2(\phi) + \cos^2(\phi)\right)\\
      &= r^2.
    \end{align*}
    So we see that every point drawn by our parametric description of
    the sphere is in the set $S$.    
  \end{explanation}
\end{example}


\end{document}
