\documentclass{ximera}


\graphicspath{
  {./}
  {ximeraTutorial/}
  {basicPhilosophy/}
}

\newcommand{\mooculus}{\textsf{\textbf{MOOC}\textnormal{\textsf{ULUS}}}}

\usepackage{tkz-euclide}\usepackage{tikz}
\usepackage{tikz-cd}
\usetikzlibrary{arrows}
\tikzset{>=stealth,commutative diagrams/.cd,
  arrow style=tikz,diagrams={>=stealth}} %% cool arrow head
\tikzset{shorten <>/.style={ shorten >=#1, shorten <=#1 } } %% allows shorter vectors

\usetikzlibrary{backgrounds} %% for boxes around graphs
\usetikzlibrary{shapes,positioning}  %% Clouds and stars
\usetikzlibrary{matrix} %% for matrix
\usepgfplotslibrary{polar} %% for polar plots
\usepgfplotslibrary{fillbetween} %% to shade area between curves in TikZ
\usetkzobj{all}
\usepackage[makeroom]{cancel} %% for strike outs
%\usepackage{mathtools} %% for pretty underbrace % Breaks Ximera
%\usepackage{multicol}
\usepackage{pgffor} %% required for integral for loops



%% http://tex.stackexchange.com/questions/66490/drawing-a-tikz-arc-specifying-the-center
%% Draws beach ball
\tikzset{pics/carc/.style args={#1:#2:#3}{code={\draw[pic actions] (#1:#3) arc(#1:#2:#3);}}}



\usepackage{array}
\setlength{\extrarowheight}{+.1cm}
\newdimen\digitwidth
\settowidth\digitwidth{9}
\def\divrule#1#2{
\noalign{\moveright#1\digitwidth
\vbox{\hrule width#2\digitwidth}}}






\DeclareMathOperator{\arccot}{arccot}
\DeclareMathOperator{\arcsec}{arcsec}
\DeclareMathOperator{\arccsc}{arccsc}

















%%This is to help with formatting on future title pages.
\newenvironment{sectionOutcomes}{}{}


\title{Analyzing}

\begin{document}

\begin{abstract}
describe everything
\end{abstract}
\maketitle







Completely analyze $K(v) = \ln(x^2+2x+3)$

$\blacktriangleright$  The implied domain is all real numbers that make the inside function, $v^2+2v+3$, greater than $0$.  So, we need some information on $v^2+2v+3$.   Let's give this funciton the name $in(v) = v^2+2v+3$

$in(v)$ is a quadratic function.  Its leading coeffident is positive, therefore the graph would be a parabola opening up.  That tells us $in(v)$ has an absoute minimum.  The quadratic formula tells us that the minimum occurs at $v=\frac{-b}{2a} = \frac{-2}{2} = -1$.

The minimum value is $in(-1) = 2$.  Therefore, $in(v) > 0$ for all $v$.

Therefore, the implied domain of $K(v)$ is all real numbers.

Since, $\ln(r)$ is an increasing function, $K(-1) = \ln(2)$ is the minimum value.  The graph has a lowest point at $(-1, \ln(2))$ and the graph must go up in both directions from there.








\begin{image}
\begin{tikzpicture}
  \begin{axis}[
            domain=-10:10, ymax=10, xmax=10, ymin=-10, xmin=-10,
            axis lines =center, xlabel=$v$, ylabel={$y=K(v)$}, grid = major, grid style={dashed},
            ytick={-10,-8,-6,-4,-2,2,4,6,8,10},
            xtick={-10,-8,-6,-4,-2,2,4,6,8,10},
            yticklabels={$-10$,$-8$,$-6$,$-4$,$-2$,$2$,$4$,$6$,$8$,$10$}, 
            xticklabels={$-10$,$-8$,$-6$,$-4$,$-2$,$2$,$4$,$6$,$8$,$10$},
            ticklabel style={font=\scriptsize},
            every axis y label/.style={at=(current axis.above origin),anchor=south},
            every axis x label/.style={at=(current axis.right of origin),anchor=west},
            axis on top
          ]
          
          %\addplot [line width=2, penColor2, smooth,samples=100,domain=(-6:2)] {-2*x-3};
          \addplot [line width=2, penColor, smooth,samples=100,domain=(-8:8),<->] {ln(x^2 + 2*x + 3)};

          %\addplot[color=penColor,fill=penColor2,only marks,mark=*] coordinates{(-6,9)};
          %\addplot[color=penColor,fill=penColor2,only marks,mark=*] coordinates{(2,-7)};

          %\addplot[color=penColor2,fill=white,only marks,mark=*] coordinates{(2,-4.5)};
          %\addplot[color=penColor2,fill=white,only marks,mark=*] coordinates{(8,6)};


           

  \end{axis}
\end{tikzpicture}
\end{image}





$\blacktriangleright$  The range is $[\ln(2), \infty)$.

$\blacktriangleright$  $K(v)$ decreases on $(\infty, -1]$ and increases on $[-1, \infty)$.


$\blacktriangleright$  $K(v)$ has a global minimum of $\ln(2)$, which occurs at $-1$.  This is also the only local minimum.

$\blacktriangleright$  $K(v)$ has no global or local maximum.

$\blacktriangleright$  $-1$ is the only critical number.













\section{with Calculus}

Calculus will give us the derivative $K'(v) = \frac{2v+2}{x^2+2v+3}$.  We could then solve $K'(v) = 0$ and get $v=-1$ as the only critical number, which we already found.





\end{document}
