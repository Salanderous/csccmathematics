\documentclass{ximera}


\graphicspath{
  {./}
  {ximeraTutorial/}
  {basicPhilosophy/}
}

\newcommand{\mooculus}{\textsf{\textbf{MOOC}\textnormal{\textsf{ULUS}}}}

\usepackage{tkz-euclide}\usepackage{tikz}
\usepackage{tikz-cd}
\usetikzlibrary{arrows}
\tikzset{>=stealth,commutative diagrams/.cd,
  arrow style=tikz,diagrams={>=stealth}} %% cool arrow head
\tikzset{shorten <>/.style={ shorten >=#1, shorten <=#1 } } %% allows shorter vectors

\usetikzlibrary{backgrounds} %% for boxes around graphs
\usetikzlibrary{shapes,positioning}  %% Clouds and stars
\usetikzlibrary{matrix} %% for matrix
\usepgfplotslibrary{polar} %% for polar plots
\usepgfplotslibrary{fillbetween} %% to shade area between curves in TikZ
\usetkzobj{all}
\usepackage[makeroom]{cancel} %% for strike outs
%\usepackage{mathtools} %% for pretty underbrace % Breaks Ximera
%\usepackage{multicol}
\usepackage{pgffor} %% required for integral for loops



%% http://tex.stackexchange.com/questions/66490/drawing-a-tikz-arc-specifying-the-center
%% Draws beach ball
\tikzset{pics/carc/.style args={#1:#2:#3}{code={\draw[pic actions] (#1:#3) arc(#1:#2:#3);}}}



\usepackage{array}
\setlength{\extrarowheight}{+.1cm}
\newdimen\digitwidth
\settowidth\digitwidth{9}
\def\divrule#1#2{
\noalign{\moveright#1\digitwidth
\vbox{\hrule width#2\digitwidth}}}






\DeclareMathOperator{\arccot}{arccot}
\DeclareMathOperator{\arcsec}{arcsec}
\DeclareMathOperator{\arccsc}{arccsc}

















%%This is to help with formatting on future title pages.
\newenvironment{sectionOutcomes}{}{}


\title{Analyzing}

\begin{document}

\begin{abstract}
describe everything
\end{abstract}
\maketitle







Completely analyze $g(t) = e^{-t^2} - 4$

$\blacktriangleright$  \textbf{\textcolor{blue!55!black}{Domain}} \\


It might help to apply a little algebra here


\[
g(t) = \frac{1}{e^{t^2}} - 4
\]


The implied domain is all real numbers that don't result in a $0$ denominator. The exponential function cannot equal $0$.  Therefore, the natural or implied domain is all real numbers.







$\blacktriangleright$  \textbf{\textcolor{blue!55!black}{Zeros}} \\

\[
g(t) = 0
\]



\[
\frac{1}{e^{t^2}} - 4 = 0
\]


\[
e^{t^2} = \frac{1}{4}
\]


This would require a negative exponent and $t^2$ cannot be negative.  There are no zeros.






$\blacktriangleright$  \textbf{\textcolor{blue!55!black}{Continuity}} \\


Since, the denominator cannot be $0$, all of the pieces are continuous and $g$ is continuous.





$\blacktriangleright$  \textbf{\textcolor{blue!55!black}{End-Behavior}} \\


\[   \lim\limits_{t \to \infty} g(t) =  \lim\limits_{t \to \infty}  \frac{1}{e^{t^2}} - 4 = -4\]


\[   \lim\limits_{t \to -\infty} g(t) =  \lim\limits_{t \to -\infty}  \frac{1}{e^{t^2}} - 4 = -4\]





$\blacktriangleright$  \textbf{\textcolor{blue!55!black}{Behavior}} \\

The exponential function, $e^x$, is an increasing function.  The minimum value of $t^2$ is $1$, which occurs at $0$, and then $t^2$ gets larger in both directions of $0$.  Thus, $e^{t^2}$ gets large and positive in both directions of $0$.  That means  $\frac{1}{e^{t^2}}$ has a maximum of $1$ at $0$ and then gets smaller and heads to $0$ in both direcrtions of $0$. 

$g(t)$ simply subtracts $4$ from all of this.









Graph of $y = g(t)$.



\begin{image}
\begin{tikzpicture}
  \begin{axis}[
            domain=-10:10, ymax=10, xmax=10, ymin=-10, xmin=-10,
            axis lines =center, xlabel=$t$, ylabel={$y$}, grid = major, grid style={dashed},
            ytick={-10,-8,-6,-4,-2,2,4,6,8,10},
            xtick={-10,-8,-6,-4,-2,2,4,6,8,10},
            yticklabels={$-10$,$-8$,$-6$,$-4$,$-2$,$2$,$4$,$6$,$8$,$10$}, 
            xticklabels={$-10$,$-8$,$-6$,$-4$,$-2$,$2$,$4$,$6$,$8$,$10$},
            ticklabel style={font=\scriptsize},
            every axis y label/.style={at=(current axis.above origin),anchor=south},
            every axis x label/.style={at=(current axis.right of origin),anchor=west},
            axis on top
          ]
          
          \addplot [line width=1, gray, dashed,samples=100,domain=(-10:10),<->] {-4};

          %\addplot [line width=2, penColor2, smooth,samples=100,domain=(-6:2)] {-2*x-3};
          \addplot [line width=2, penColor, smooth,samples=100,domain=(-8:8),<->] {2.718^(-(x^2)) - 4};

          %\addplot[color=penColor,fill=penColor2,only marks,mark=*] coordinates{(-6,9)};
          %\addplot[color=penColor,fill=penColor2,only marks,mark=*] coordinates{(2,-7)};

          %\addplot[color=penColor2,fill=white,only marks,mark=*] coordinates{(2,-4.5)};
          %\addplot[color=penColor2,fill=white,only marks,mark=*] coordinates{(8,6)};


           

  \end{axis}
\end{tikzpicture}
\end{image}


$K$ increases on $(-\infty, 0]$ and decreases on $[0, \infty$.  This gives a global (also a local) maximum of $-3$, which occurs at $0$.   There are no minimums.


$\blacktriangleright$  $-1$ is the only critical number.


$\blacktriangleright$  The range is $(-4, -3]$.

















\section{with Calculus}

Calculus will give us the derivative $g'(t) = e^{-x^2} \cdot (-2x)$.  We could then solve $g'(t) = 0$ and verify that theopnly critical number is $t = 0$ as the only critical number.





\end{document}
