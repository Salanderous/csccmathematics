\documentclass{ximera}


\graphicspath{
  {./}
  {ximeraTutorial/}
  {basicPhilosophy/}
}

\newcommand{\mooculus}{\textsf{\textbf{MOOC}\textnormal{\textsf{ULUS}}}}

\usepackage{tkz-euclide}\usepackage{tikz}
\usepackage{tikz-cd}
\usetikzlibrary{arrows}
\tikzset{>=stealth,commutative diagrams/.cd,
  arrow style=tikz,diagrams={>=stealth}} %% cool arrow head
\tikzset{shorten <>/.style={ shorten >=#1, shorten <=#1 } } %% allows shorter vectors

\usetikzlibrary{backgrounds} %% for boxes around graphs
\usetikzlibrary{shapes,positioning}  %% Clouds and stars
\usetikzlibrary{matrix} %% for matrix
\usepgfplotslibrary{polar} %% for polar plots
\usepgfplotslibrary{fillbetween} %% to shade area between curves in TikZ
\usetkzobj{all}
\usepackage[makeroom]{cancel} %% for strike outs
%\usepackage{mathtools} %% for pretty underbrace % Breaks Ximera
%\usepackage{multicol}
\usepackage{pgffor} %% required for integral for loops



%% http://tex.stackexchange.com/questions/66490/drawing-a-tikz-arc-specifying-the-center
%% Draws beach ball
\tikzset{pics/carc/.style args={#1:#2:#3}{code={\draw[pic actions] (#1:#3) arc(#1:#2:#3);}}}



\usepackage{array}
\setlength{\extrarowheight}{+.1cm}
\newdimen\digitwidth
\settowidth\digitwidth{9}
\def\divrule#1#2{
\noalign{\moveright#1\digitwidth
\vbox{\hrule width#2\digitwidth}}}






\DeclareMathOperator{\arccot}{arccot}
\DeclareMathOperator{\arcsec}{arcsec}
\DeclareMathOperator{\arccsc}{arccsc}

















%%This is to help with formatting on future title pages.
\newenvironment{sectionOutcomes}{}{}


\title{Analyzing}

\begin{document}

\begin{abstract}
describe everything
\end{abstract}
\maketitle







Completely analyze $p(k) = k \sqrt{4-k^2}$






$\blacktriangleright$  \textbf{\textcolor{blue!55!black}{Domain}} \\

We need the inside of the square root to be positive.  The domain is $[-2, 2]$.





$\blacktriangleright$  \textbf{\textcolor{blue!55!black}{Continuity}} \\

Both $k$ and $\sqrt{4 - k^2}$ are continuous.  $p$ is the product of continuous functions, so it is continuous.



$\blacktriangleright$  \textbf{\textcolor{blue!55!black}{Zeros}} \\


\[  p(k) = 0   \]

\[  k \sqrt{4-k^2} = 0  \]

\[
\text{Either } \, k = 0 \, \text{ or } \, \sqrt{4-k^2} = 0
\]


Zeros are $-2$, $0$, and $2$. \\




$\blacktriangleright$ \textbf{Behavior:} 


Since $\sqrt{k^2 - 4} \geq 0$, the sign of $p(k)$ is the same as the sign of $k$.

\begin{itemize}
\item  $p(k) < 0$ on $(-2, 0)$.
\item  $p(k) > 0$ on $(0, 2)$.
\end{itemize}


Since $p$ is continuous, we must have 


\begin{itemize}
\item  $p(-2) = 0$.
\item  $p(k)$ begins decreasing at $-2$.
\item  $p(k)$ eventually increases back to $p(0) = 0$.
\item  $p(k)$ continues increasing beyond $p(0) = 0$.
\item  $p(k)$ eventually decreases back to $p(2) = 0$.
\end{itemize}


There must be at least two critical numbers.  One inside $(-2,0)$ and one inside $(0, 2)$.








Graph of $y = p(k)$.

\begin{image}
\begin{tikzpicture}
  \begin{axis}[
            domain=-2:2, ymax=4, xmax=4, ymin=-4, xmin=-4,
            axis lines =center, xlabel=$k$, ylabel={$y$}, grid = major, grid style={dashed},
            ytick={-4,-2,2,4},
            xtick={-4,-2,2,4},
            yticklabels={$-4$,$-2$,$4$,$6$}, 
            xticklabels={$-4$,$-2$,$2$,$4$},
            ticklabel style={font=\scriptsize},
            every axis y label/.style={at=(current axis.above origin),anchor=south},
            every axis x label/.style={at=(current axis.right of origin),anchor=west},
            axis on top
          ]
          

            %\addplot [line width=2, penColor, smooth,samples=100,domain=(-9:9)] {5/(1 + 3 * e^(-x/2))};
            \addplot [line width=2, penColor, smooth,samples=100,domain=(-2:2)] {x * sqrt(4 - x^2)};


            \addplot[color=penColor,fill=penColor,only marks,mark=*] coordinates{(-2,0) (0,0) (2,0)}; 





           

  \end{axis}
\end{tikzpicture}
\end{image}










$\blacktriangleright$ \textbf{Extrema:}  \\


With a little help from DESMOS, we can approximate the critical numbers as $-1.414$ and $1.414$.



\begin{itemize}
\item  $p(k)$ decreasing on $[-2, 1.414)$.
\item  $p(k)$ increases on $(1.414, 1.414)$.
\item  $p(k)$ decreases on $(1.414, 2]$.
\end{itemize}




$p(-1.414)$ is the approximate global (and local) minimum. \\

$p(1.414)$ is the approximate global (and local) maximum. \\




\section{with Calculus}


Calculus will give us a formula for the derivative.

\[    p'(k) = \sqrt{4 - k^2} + k \cdot \frac{-2k}{\sqrt{4-k^2}}  \]


This derivative has zeros of $-\sqrt{2}$ and $\sqrt{2}$, which are approximately $-1.414$ and $1.4141$. 







\end{document}
