\documentclass{ximera}


\graphicspath{
  {./}
  {ximeraTutorial/}
  {basicPhilosophy/}
}

\newcommand{\mooculus}{\textsf{\textbf{MOOC}\textnormal{\textsf{ULUS}}}}

\usepackage{tkz-euclide}\usepackage{tikz}
\usepackage{tikz-cd}
\usetikzlibrary{arrows}
\tikzset{>=stealth,commutative diagrams/.cd,
  arrow style=tikz,diagrams={>=stealth}} %% cool arrow head
\tikzset{shorten <>/.style={ shorten >=#1, shorten <=#1 } } %% allows shorter vectors

\usetikzlibrary{backgrounds} %% for boxes around graphs
\usetikzlibrary{shapes,positioning}  %% Clouds and stars
\usetikzlibrary{matrix} %% for matrix
\usepgfplotslibrary{polar} %% for polar plots
\usepgfplotslibrary{fillbetween} %% to shade area between curves in TikZ
\usetkzobj{all}
\usepackage[makeroom]{cancel} %% for strike outs
%\usepackage{mathtools} %% for pretty underbrace % Breaks Ximera
%\usepackage{multicol}
\usepackage{pgffor} %% required for integral for loops



%% http://tex.stackexchange.com/questions/66490/drawing-a-tikz-arc-specifying-the-center
%% Draws beach ball
\tikzset{pics/carc/.style args={#1:#2:#3}{code={\draw[pic actions] (#1:#3) arc(#1:#2:#3);}}}



\usepackage{array}
\setlength{\extrarowheight}{+.1cm}
\newdimen\digitwidth
\settowidth\digitwidth{9}
\def\divrule#1#2{
\noalign{\moveright#1\digitwidth
\vbox{\hrule width#2\digitwidth}}}






\DeclareMathOperator{\arccot}{arccot}
\DeclareMathOperator{\arcsec}{arcsec}
\DeclareMathOperator{\arccsc}{arccsc}

















%%This is to help with formatting on future title pages.
\newenvironment{sectionOutcomes}{}{}


\title{Polar Form}

\begin{document}

\begin{abstract}
locating numbers
\end{abstract}
\maketitle



\section{Representing Complex Numbers}

We already have several ways of representing complex numbers. \\



\textbf{\textcolor{red!80!black}{$\blacktriangleright (4, 8)$}}: We represent complex numbers visually with points or dots on the Cartesian plane.  \\





\begin{image}
\begin{tikzpicture}
  \begin{axis}[
            domain=-10:10, ymax=10, xmax=10, ymin=-10, xmin=-10,
            axis lines =center, xlabel=$\mathbb{R}$, ylabel=$\mathbb{R}$, grid = major,
            unit vector ratio*=1 1 1,
            ytick={-10,-8,-6,-4,-2,2,4,6,8,10},
            xtick={-10,-8,-6,-4,-2,2,4,6,8,10},
            ticklabel style={font=\scriptsize},
            every axis y label/.style={at=(current axis.above origin),anchor=south},
            every axis x label/.style={at=(current axis.right of origin),anchor=west},
            axis on top
          ]
          

          \addplot[color=penColor,fill=penColor,only marks,mark=*] coordinates{(4,8)};



  \end{axis}
\end{tikzpicture}
\end{image}








\textbf{\textcolor{red!80!black}{$\blacktriangleright \langle 4, 8 \rangle$}}: We represent complex numbers with vectors.  These are illustrated with arrows on the Cartesian plane.  Algebraically, we write vectors using triangular brackets.






\begin{image}
\begin{tikzpicture}
  \begin{axis}[
            domain=-10:10, ymax=10, xmax=10, ymin=-10, xmin=-10,
            axis lines =center, xlabel=$\mathbb{R}$, ylabel=$\mathbb{R}$, grid = major,
            unit vector ratio*=1 1 1,
            ytick={-10,-8,-6,-4,-2,2,4,6,8,10},
            xtick={-10,-8,-6,-4,-2,2,4,6,8,10},
            ticklabel style={font=\scriptsize},
            every axis y label/.style={at=(current axis.above origin),anchor=south},
            every axis x label/.style={at=(current axis.right of origin),anchor=west},
            axis on top
          ]
          

          \draw[penColor,ultra thick,->] (axis cs:0,0) -- (axis cs:4,8);


  \end{axis}
\end{tikzpicture}
\end{image}

Vectors have a visual arithmetic, where we arrange the vectors tail-to-head or tail-to-tail and then create a resultant vector from the arrangement. These resultant vectors represent the sum or difference of two Complex numbers.

Vectors have a symbolic arithmetic as well, which agrees with the geometric operations:

\[
\langle a, b \rangle + \langle c, d \rangle = \langle a+c, b+d \rangle
\]



\textbf{\textcolor{red!90!darkgray}{$\blacktriangleright 4 + 8 \, i$}}: We represent complex numbers with a 2-dimensional sum.



The left, first, or horizontal part is called the \textbf{\textcolor{purple!85!blue}{real part}} and the right, second, or vertical part is called the \textbf{\textcolor{purple!85!blue}{imaginary part}}.



When viewing the plane in this context, we call it the \textbf{\textcolor{purple!85!blue}{Complex Plane}}.





\begin{image}
\begin{tikzpicture}
  \begin{axis}[
            domain=-10:10, ymax=10, xmax=10, ymin=-10, xmin=-10,
            axis lines =center, xlabel=$Re$, ylabel=$Im$, grid = major,
            unit vector ratio*=1 1 1,
            ytick={-10,-8,-6,-4,-2,2,4,6,8,10},
            xtick={-10,-8,-6,-4,-2,2,4,6,8,10},
            ticklabel style={font=\scriptsize},
            every axis y label/.style={at=(current axis.above origin),anchor=south},
            every axis x label/.style={at=(current axis.right of origin),anchor=west},
            axis on top
          ]
          

          \addplot[color=penColor,fill=penColor,only marks,mark=*] coordinates{(4,8)};


  \end{axis}
\end{tikzpicture}
\end{image}






All of these representations share a common structure.  They all describe complex numbers with rectangular measurements.  All of the descriptions above give horizontal (left/right) and vertical (up/down) information.


There is a different way to describe complex numbers.



















\section{Polar Coordinates}

Instead of rectangular information, we could give circular information.  We could give the angle the vector makes with the positive horizontal axis together with the length or distance from the origin. 

In keeping with the circular idea, the vector resembles a radius.  

Our two pieces of information will be $r$ and $\theta$. These are known as \textbf{polar coordinates}




\begin{definition}   \textbf{\textcolor{green!50!black}{Polar Coordinates}} \\

Each Complex number can be described by $(r, \theta)$, where $r$ is a real number and $\theta$ is an angle measurement.  \\


$r$ can be positive or negative. \\
$\theta$ is measured counterclockwise from the positive real axis. \\

\end{definition}


Of course, that is how we write rectangular coordinates as well.  The context of the situation will tell us how to interpret the coordinates.



\begin{example}



Write $4 + 8i$ in polar form.



\begin{image}
\begin{tikzpicture}
  \begin{axis}[
            domain=-10:10, ymax=10, xmax=10, ymin=-10, xmin=-10,
            axis lines =center, xlabel=$\mathbb{R}$, ylabel=$\mathbb{R}$, grid = major,
            unit vector ratio*=1 1 1,
            ytick={-10,-8,-6,-4,-2,2,4,6,8,10},
            xtick={-10,-8,-6,-4,-2,2,4,6,8,10},
            ticklabel style={font=\scriptsize},
            every axis y label/.style={at=(current axis.above origin),anchor=south},
            every axis x label/.style={at=(current axis.right of origin),anchor=west},
            axis on top
          ]
          
          \addplot[color=penColor,fill=penColor,only marks,mark=*] coordinates{(4,8)};
          \draw[penColor,ultra thick,->] (axis cs:0,0) -- (axis cs:4,8);
          \addplot [textColor,smooth, domain=(0:63),->] ({3*cos(x)},{3*sin(x)});
          \node at (axis cs:2,1) [anchor=east] {$\theta$};
          \node at (axis cs:2.5,5) [anchor=east] {$r$};


           

  \end{axis}
\end{tikzpicture}
\end{image}


$r = \sqrt{4^2 + 8^2} = \sqrt{80} = 16\sqrt{5}$ (the modulus of $4 + 8 \, i$) \\
$\theta = 63.43^{\circ}$ \\



We, again, wrap the coordinates up with parentheses: $(r, \theta) = (16\sqrt{5}, 63.43^{\circ})$.


And, we need to work equally well with radians and degrees: $(r, \theta) = (16\sqrt{5}, 1.107)$


\end{example}












\begin{warning} \textbf{\textcolor{red!80!black}{Multiple Polar Coordinates}}


Unlike with rectangular coordinates, different polar coordinates can describe the same complex number.

\[
 \cdots = (r, \theta - 2\pi) = (r, \theta) = (r, \theta + 2\pi) = (r, \theta + 4\pi) = \cdots
\]

\end{warning}






\begin{warning} \textbf{\textcolor{red!80!black}{Negative ``Distance''}}


The $r$ in $(r, \theta)$ is the distance to move in the $\theta$-direction.  However, $r$ can be negative. In this case, the interpretation is to move in the opposite direction from $\theta$.

\[
 (r, \theta) = (-r, \theta + \pi) = (-r, \theta - \pi)
\]

\end{warning}








  \begin{image}
    \begin{tikzpicture}[scale=5.3,cap=round,>=latex]
        % draw the coordinates
        \draw[->] (-1.5cm,0cm) -- (1.5cm,0cm) node[right,fill=white] {$x$};
        \draw[->] (0cm,-1.5cm) -- (0cm,1.5cm) node[above,fill=white] {$y$};

        % draw the unit circle
        \draw[thick] (0cm,0cm) circle(1cm);

        \foreach \x in {0,30,...,360} {
                % lines from center to point
                \draw[gray] (0cm,0cm) -- (\x:1cm);
                % dots at each point
                \filldraw[black] (\x:1cm) circle(0.4pt);
                % draw each angle in degrees
                \draw (\x:0.6cm) node[fill=white] {$\x^\circ$};
        }

        % draw each angle in radians
        \foreach \x/\xtext in {
            30/\frac{\pi}{6},
            45/\frac{\pi}{4},
            60/\frac{\pi}{3},
            90/\frac{\pi}{2},
            120/\frac{2\pi}{3},
            135/\frac{3\pi}{4},
            150/\frac{5\pi}{6},
            180/\pi,
            210/\frac{7\pi}{6},
            225/\frac{5\pi}{4},
            240/\frac{4\pi}{3},
            270/\frac{3\pi}{2},
            300/\frac{5\pi}{3},
            315/\frac{7\pi}{4},
            330/\frac{11\pi}{6},
            360/2\pi}
                \draw (\x:0.85cm) node[fill=white] {$\xtext$};

        \foreach \x/\xtext/\y in {
            % the coordinates for the first quadrant
            30/\frac{\sqrt{3}}{2}/\frac{1}{2},
            45/\frac{\sqrt{2}}{2}/\frac{\sqrt{2}}{2},
            60/\frac{1}{2}/\frac{\sqrt{3}}{2},
            % the coordinates for the second quadrant
            150/-\frac{\sqrt{3}}{2}/\frac{1}{2},
            135/-\frac{\sqrt{2}}{2}/\frac{\sqrt{2}}{2},
            120/-\frac{1}{2}/\frac{\sqrt{3}}{2},
            % the coordinates for the third quadrant
            210/-\frac{\sqrt{3}}{2}/-\frac{1}{2},
            225/-\frac{\sqrt{2}}{2}/-\frac{\sqrt{2}}{2},
            240/-\frac{1}{2}/-\frac{\sqrt{3}}{2},
            % the coordinates for the fourth quadrant
            330/\frac{\sqrt{3}}{2}/-\frac{1}{2},
            315/\frac{\sqrt{2}}{2}/-\frac{\sqrt{2}}{2},
            300/\frac{1}{2}/-\frac{\sqrt{3}}{2}}
                \draw (\x:1.25cm) node[fill=white] {$\left(\xtext,\y\right)$};

        % draw the horizontal and vertical coordinates
        % the placement is better this way
        \draw (-1.25cm,0cm) node[above=1pt] {$(-1,0)$}
              (1.25cm,0cm)  node[above=1pt] {$(1,0)$}
              (0cm,-1.25cm) node[fill=white] {$(0,-1)$}
              (0cm,1.25cm)  node[fill=white] {$(0,1)$};
    \end{tikzpicture}
  \end{image}








\begin{question}
Which of the following polar coordinates represent the same complex number?
\begin{selectAll}
\choice[correct]{$\left( 5, \frac{\pi}{4} \right)$}
\choice[correct]{$\left( 5, \frac{9\pi}{4} \right)$}
\choice{$\left( 5, \frac{3\pi}{4} \right)$}
\choice[correct]{$\left( -5, \frac{5\pi}{4} \right)$}
\choice{$\left( -5, \frac{7\pi}{4} \right)$}
\end{selectAll}
\end{question}













\begin{center}
\textbf{\textcolor{green!50!black}{ooooo=-=-=-=-=-=-=-=-=-=-=-=-=ooOoo=-=-=-=-=-=-=-=-=-=-=-=-=ooooo}} \\

more examples can be found by following this link\\ \link[More Examples of Polar Form of Complex Numbers]{https://ximera.osu.edu/csccmathematics/precalculus2/precalculus2/polarFormComplexNumbers/examples/exampleList}

\end{center}


\end{document}
