\documentclass{ximera}


\graphicspath{
  {./}
  {ximeraTutorial/}
  {basicPhilosophy/}
}

\newcommand{\mooculus}{\textsf{\textbf{MOOC}\textnormal{\textsf{ULUS}}}}

\usepackage{tkz-euclide}\usepackage{tikz}
\usepackage{tikz-cd}
\usetikzlibrary{arrows}
\tikzset{>=stealth,commutative diagrams/.cd,
  arrow style=tikz,diagrams={>=stealth}} %% cool arrow head
\tikzset{shorten <>/.style={ shorten >=#1, shorten <=#1 } } %% allows shorter vectors

\usetikzlibrary{backgrounds} %% for boxes around graphs
\usetikzlibrary{shapes,positioning}  %% Clouds and stars
\usetikzlibrary{matrix} %% for matrix
\usepgfplotslibrary{polar} %% for polar plots
\usepgfplotslibrary{fillbetween} %% to shade area between curves in TikZ
\usetkzobj{all}
\usepackage[makeroom]{cancel} %% for strike outs
%\usepackage{mathtools} %% for pretty underbrace % Breaks Ximera
%\usepackage{multicol}
\usepackage{pgffor} %% required for integral for loops



%% http://tex.stackexchange.com/questions/66490/drawing-a-tikz-arc-specifying-the-center
%% Draws beach ball
\tikzset{pics/carc/.style args={#1:#2:#3}{code={\draw[pic actions] (#1:#3) arc(#1:#2:#3);}}}



\usepackage{array}
\setlength{\extrarowheight}{+.1cm}
\newdimen\digitwidth
\settowidth\digitwidth{9}
\def\divrule#1#2{
\noalign{\moveright#1\digitwidth
\vbox{\hrule width#2\digitwidth}}}






\DeclareMathOperator{\arccot}{arccot}
\DeclareMathOperator{\arcsec}{arcsec}
\DeclareMathOperator{\arccsc}{arccsc}

















%%This is to help with formatting on future title pages.
\newenvironment{sectionOutcomes}{}{}


\title{Direction Vectors}

\begin{document}

\begin{abstract}
unit vectors
\end{abstract}
\maketitle




On the real number line, we draw ``tick'' marks to mark off distances of $1$ to the left and right of $0$. On the real number line, we have two direction vectors: $-1$ and $1$.  They establish a unit length in each direction. Every real number is a positive scalar multiple of one of these.

Can we do the same thing for the Complex numbers? \\









\begin{definition}  \textbf{\textcolor{green!50!black}{Direction Vectors}}   \\

A \textbf{direction vector}, also called a \textbf{unit vector}, is a vector of length $1$.

\end{definition}









$\blacktriangleright$ \textbf{\textcolor{blue!75!black}{Polar(Circular) Coordinates}}  \\


Of course, this is easy with polar coordinates.  Unit vectors look like $(1, \theta)$. \\

Every complex number can be described as a positive scalar multiple of a polar unit vector:  


\[
r \cdot (1, \theta) = (r, 0) \cdot (1, \theta) = (r, \theta)
\]



\begin{center}
\textbf{\textcolor{red!80!black}{How can we do this with rectangular coordinates?}}
\end{center}



Can we describe each complex number as a scalar multiple of a unit vector in rectangular coordinates? \\






$\blacktriangleright$ \textbf{\textcolor{blue!75!black}{Rectangular Coordinates}}  


Currently, we can describe every complex number as a sum of two scalar multiples of two direction vectors: 

\[ \hat{x} = \langle 1, 0 \rangle  \, \text{ and }  \, \hat{y} = \langle 0, 1 \rangle      \]


The Cartesian plane has two axes, so two sets of tick marks on two lines.  The left/right tick marks are marking off multiples of $\langle 1, 0 \rangle$ on the real axis. The up/down tick marks are marking off multiples of $\langle 0, 1 \rangle$ on the imaginary axis.




We can view our vectors as constructed as a sum of perpendicular pieces, each of size $1$.



\begin{example}


\[ \langle 5, 7 \rangle = 5 \cdot  \langle 1, 0 \rangle + 7 \cdot \langle 0, 1 \rangle  \]









\begin{image}
\begin{tikzpicture}
  \begin{axis}[
            domain=-10:10, ymax=10, xmax=10, ymin=-10, xmin=-10,
            axis lines =center, xlabel=$\mathbb{R}$, ylabel=$\mathbb{R}$, grid = major, grid style={dashed},
            unit vector ratio*=1 1 1,
            ytick={-10,-8,-6,-4,-2,2,4,6,8,10},
            xtick={-10,-8,-6,-4,-2,2,4,6,8,10},
            ticklabel style={font=\scriptsize},
            every axis y label/.style={at=(current axis.above origin),anchor=south},
            every axis x label/.style={at=(current axis.right of origin),anchor=west},
            axis on top
          ]
          

          \draw[black,ultra thick,->] (axis cs:0,0) -- (axis cs:5,7);

          \draw[penColor,ultra thick,->] (axis cs:0,0) -- (axis cs:0,1);
          \draw[penColor,ultra thick,->] (axis cs:0,1) -- (axis cs:0,2);
          \draw[penColor,ultra thick,->] (axis cs:0,2) -- (axis cs:0,3);
          \draw[penColor,ultra thick,->] (axis cs:0,3) -- (axis cs:0,4);
          \draw[penColor,ultra thick,->] (axis cs:0,4) -- (axis cs:0,5);
          \draw[penColor,ultra thick,->] (axis cs:0,5) -- (axis cs:0,6);
          \draw[penColor,ultra thick,->] (axis cs:0,6) -- (axis cs:0,7);



          \draw[penColor2,ultra thick,->] (axis cs:0,0) -- (axis cs:1,0);
          \draw[penColor2,ultra thick,->] (axis cs:1,0) -- (axis cs:2,0);
          \draw[penColor2,ultra thick,->] (axis cs:2,0) -- (axis cs:3,0);
          \draw[penColor2,ultra thick,->] (axis cs:3,0) -- (axis cs:4,0);
          \draw[penColor2,ultra thick,->] (axis cs:4,0) -- (axis cs:5,0);







           

  \end{axis}
\end{tikzpicture}
\end{image}


\end{example}



But, we want just a single vector, not a combination of two vectors. \\

Our idea is to take the vector representing a complex number and factor out its length. The resulting vector is a unit vector. We will have a unit vector pointing in the right direction (a direction vector) and a positive scalar, which is the length of the vector.


Each complex number can be represented by a vector.  This vector can then be written as a scalar (number) times a \textbf{unit vector}.  



\begin{example}

\[  3 + 4 \, i = \langle 3, 4 \rangle = 5 \cdot \left\langle \frac{3}{5}, \frac{4}{5} \right\rangle     \]

\begin{image}
\begin{tikzpicture}
  \begin{axis}[
            domain=-5:5, ymax=5, xmax=5, ymin=-5, xmin=-5,
            axis lines =center, xlabel=$t$, ylabel=$y$, grid = major, grid style={dashed},
            unit vector ratio*=1 1 1,
            ytick={-4,-2,2,4},
            xtick={-4,-2,2,4},
            ticklabel style={font=\scriptsize},
            every axis y label/.style={at=(current axis.above origin),anchor=south},
            every axis x label/.style={at=(current axis.right of origin),anchor=west},
            axis on top
          ]
          
          \addplot[color=penColor,fill=penColor,only marks,mark=*] coordinates{(3,4)};
          \draw[penColor,ultra thick,->] (axis cs:0,0) -- (axis cs:0.6,0.8);
          \draw[penColor,ultra thick,->] (axis cs:0.6,0.8) -- (axis cs:1.2,1.6);
          \draw[penColor,ultra thick,->] (axis cs:1.2,1.6) -- (axis cs:1.8,2.4);
          \draw[penColor,ultra thick,->] (axis cs:1.8,2.4) -- (axis cs:2.4,3.2);
          \draw[penColor,ultra thick,->] (axis cs:2.4,3.2) -- (axis cs:3,4);




           

  \end{axis}
\end{tikzpicture}
\end{image}




\end{example}







Substitute ``modulus'' for ``length'' and we can do the same for complex numbers in standard form. \\ 






\begin{center}

\textbf{\textcolor{purple!50!blue!90!black}{EVERY COMPLEX NUMBER}}

\textbf{\textcolor{purple!50!blue!90!black}{CAN BE DESCRIBED THIS WAY}}

\end{center}




Any complex number can be factored by dividing the real and imaginary parts by the modulus of the complex number.  This modulus is the scalar out front.




\[     a + b \, i    =  \sqrt{a^2 + b^2}  \cdot \left(  \frac{a}{\sqrt{a^2 + b^2}}, \frac{b}{\sqrt{a^2 + b^2}} \, i \right)                      \]


We have factored $a + b \, i $ into a product of a positive real number times a unit ``number''. \\

















\begin{summary} \textbf{\textcolor{red!80!black}{All, Every, Each}}  \\

Every Complex number can be written as a product of a real number times a unit ``number''.


To understand the Complex numbers, we really need to understand the unit complex numbers - the Complex numbers with modulus equal to $1$


The numbers with modulus equal to $1$ make up the unit circle.


$\blacktriangleright$ \textbf{\textcolor{blue!55!black}{We need to understand the unit circle.}}   

\end{summary}
































\begin{center}
\textbf{\textcolor{green!50!black}{ooooo=-=-=-=-=-=-=-=-=-=-=-=-=ooOoo=-=-=-=-=-=-=-=-=-=-=-=-=ooooo}} \\

more examples can be found by following this link\\ \link[More Examples of Polar Form of Complex Numbers]{https://ximera.osu.edu/csccmathematics/precalculus2/precalculus2/polarFormComplexNumbers/examples/exampleList}

\end{center}






\end{document}
