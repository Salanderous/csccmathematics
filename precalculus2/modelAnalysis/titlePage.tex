\documentclass{ximera}


\graphicspath{
  {./}
  {ximeraTutorial/}
  {basicPhilosophy/}
}

\newcommand{\mooculus}{\textsf{\textbf{MOOC}\textnormal{\textsf{ULUS}}}}

\usepackage{tkz-euclide}\usepackage{tikz}
\usepackage{tikz-cd}
\usetikzlibrary{arrows}
\tikzset{>=stealth,commutative diagrams/.cd,
  arrow style=tikz,diagrams={>=stealth}} %% cool arrow head
\tikzset{shorten <>/.style={ shorten >=#1, shorten <=#1 } } %% allows shorter vectors

\usetikzlibrary{backgrounds} %% for boxes around graphs
\usetikzlibrary{shapes,positioning}  %% Clouds and stars
\usetikzlibrary{matrix} %% for matrix
\usepgfplotslibrary{polar} %% for polar plots
\usepgfplotslibrary{fillbetween} %% to shade area between curves in TikZ
\usetkzobj{all}
\usepackage[makeroom]{cancel} %% for strike outs
%\usepackage{mathtools} %% for pretty underbrace % Breaks Ximera
%\usepackage{multicol}
\usepackage{pgffor} %% required for integral for loops



%% http://tex.stackexchange.com/questions/66490/drawing-a-tikz-arc-specifying-the-center
%% Draws beach ball
\tikzset{pics/carc/.style args={#1:#2:#3}{code={\draw[pic actions] (#1:#3) arc(#1:#2:#3);}}}



\usepackage{array}
\setlength{\extrarowheight}{+.1cm}
\newdimen\digitwidth
\settowidth\digitwidth{9}
\def\divrule#1#2{
\noalign{\moveright#1\digitwidth
\vbox{\hrule width#2\digitwidth}}}






\DeclareMathOperator{\arccot}{arccot}
\DeclareMathOperator{\arcsec}{arcsec}
\DeclareMathOperator{\arccsc}{arccsc}

















%%This is to help with formatting on future title pages.
\newenvironment{sectionOutcomes}{}{}


\title{Model Analysis}

\begin{document}

\begin{abstract}
%%%
\end{abstract}
\maketitle






A given situation is built from many attributes, relationships, and measurements. A model is a collection of related mathematical objects that encode this information.

A model might include definitions, functions, equations, diagrams, measurements, etc. The model may hold restrictions on the objects that reflect the constraints of the situation.  Or, it may not. The model may be lacking information available in the real situation.

The model is the model.

With the model in hand, we shelve the situation and enter the model's world. By analyzing the model, we mean to expand the collection of objects to include other objects, which are logical derived from the intial objects.  The collection should always hold objects which are logically consistent with each other.

As the analysis continues, the model expands and conclusions may be drawn.  These become part of the model. Eventually, the model is compared back to the real situation.  



\begin{itemize}
\item Does the expanded model still accurately describe the situation?
\item Does the model need revision?
\item What information about the real situation does the new model suggest that we were not thinking before?
\end{itemize}


Often, there is a question about the real situation and we are using the model to respond to the question.  In this case, the initial model has encoded the question into a question about the model.  When we compare back to the real situation, we must decode the model information back into situational information.


In any case, we must separate the model from the situation.  We allow the model analysis to go wherever the mathematics logically takes it.







\subsection{Learning Outcomes}


\begin{sectionOutcomes}
In this section, students will 

\begin{itemize}
\item encoding and decoding information.
\end{itemize}
\end{sectionOutcomes}









\begin{center}
\textbf{\textcolor{green!50!black}{ooooo=-=-=-=-=-=-=-=-=-=-=-=-=ooOoo=-=-=-=-=-=-=-=-=-=-=-=-=ooooo}} \\

more examples can be found by following this link\\ \link[More Examples of Model Analysis]{https://ximera.osu.edu/csccmathematics/precalculus2/precalculus2/modelAnalysis/examples/exampleList}

\end{center}






\end{document}
