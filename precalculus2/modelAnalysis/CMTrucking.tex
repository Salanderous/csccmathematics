\documentclass{ximera}


\graphicspath{
  {./}
  {ximeraTutorial/}
  {basicPhilosophy/}
}

\newcommand{\mooculus}{\textsf{\textbf{MOOC}\textnormal{\textsf{ULUS}}}}

\usepackage{tkz-euclide}\usepackage{tikz}
\usepackage{tikz-cd}
\usetikzlibrary{arrows}
\tikzset{>=stealth,commutative diagrams/.cd,
  arrow style=tikz,diagrams={>=stealth}} %% cool arrow head
\tikzset{shorten <>/.style={ shorten >=#1, shorten <=#1 } } %% allows shorter vectors

\usetikzlibrary{backgrounds} %% for boxes around graphs
\usetikzlibrary{shapes,positioning}  %% Clouds and stars
\usetikzlibrary{matrix} %% for matrix
\usepgfplotslibrary{polar} %% for polar plots
\usepgfplotslibrary{fillbetween} %% to shade area between curves in TikZ
\usetkzobj{all}
\usepackage[makeroom]{cancel} %% for strike outs
%\usepackage{mathtools} %% for pretty underbrace % Breaks Ximera
%\usepackage{multicol}
\usepackage{pgffor} %% required for integral for loops



%% http://tex.stackexchange.com/questions/66490/drawing-a-tikz-arc-specifying-the-center
%% Draws beach ball
\tikzset{pics/carc/.style args={#1:#2:#3}{code={\draw[pic actions] (#1:#3) arc(#1:#2:#3);}}}



\usepackage{array}
\setlength{\extrarowheight}{+.1cm}
\newdimen\digitwidth
\settowidth\digitwidth{9}
\def\divrule#1#2{
\noalign{\moveright#1\digitwidth
\vbox{\hrule width#2\digitwidth}}}






\DeclareMathOperator{\arccot}{arccot}
\DeclareMathOperator{\arcsec}{arcsec}
\DeclareMathOperator{\arccsc}{arccsc}

















%%This is to help with formatting on future title pages.
\newenvironment{sectionOutcomes}{}{}


\title{C{\&}M Trucking}

\begin{document}

\begin{abstract}
least cost
\end{abstract}
\maketitle


[From Calculus\&\textit{Mathematica}]




You are the chief dispatcher for the C{\&}M Trucking Company, which sends Mack trucks on the straight shot between Chicago and New Orleans on Interstate 57.  You know that:



\begin{itemize}
\item The run between the two cities is $750$ miles.
\item Running at a steady $50$ miles per hour, the Mack gets $4$ miles per gallon.
\item For each mile per hour increase in speed, the big Mack loses $\frac{1}{10}$ of a mile per gallong in its mileage.
\item The driver team gets $27$ dollars per hour.
\item Keeping the truck on the road costs an extra $12$ dollars per hour over and above the cost of the fuel.
\item Diesel fule for the Mack costs $\$1.19$ per gallon. 
\end{itemize}

Approximately, what steady speed should you tell your drivers to hold in order to make the run at least cost? \\





\begin{question} 


According to the story, which of the following are components of the total cost for the Chicago to New Orleans run?

\begin{selectAll}
\choice[correct] {Fuel}
\choice {Food}
\choice[correct] {Driving Team}
\choice[correct] {Truck Maintanence} 
\choice {Tolls}
\end{selectAll}
\end{question}


\begin{question} 


If the truck travels at a steady speed of $v$, then how long with it take to complete the trip? \\

time = $\answer{\frac{750}{v}}$ hours.
\end{question}




\begin{question} 


If the truck travels at a steady speed of $v$, then what will be the cost of the driving team \\

Driving Team Cost = $\answer{\frac{27*750}{v}}$ dollars.
\end{question}




\begin{question} 


If the truck travels at a steady speed of $v$, then what will be the cost of maintaining the truck? \\

Truck Maintanence Cost = $\answer{\frac{12*750}{v}}$ dollars.
\end{question}



The final cost to calculate is for fuel.

Fuel cost $\$3.09$ per gallon, therefore we need to know how many gallons the trip will take. To calculate gallons, we need the gas mileage for the truck.


\begin{explanation}

If the truck travels at a steady speed of $v$, then it is $\answer{v-50}$ miles per hours over $50$. For each mile per hour over $50$, the truck loses $\frac{1}{10}$ of a mile per gallon from $4$ miles per gallon..

Therefore, it will lose $\answer{(v-50)/10}$ miles per gallon when it is traveling at $v$ miles per hour.

That gives a mileage of $\answer{4-(v-50)/10} \frac{miles}{gallon}$.

\end{explanation}


The truck is travelling $750$ miles and gets $\frac{4-(v-50)}{10} \frac{miles}{gallon}$. From this we can calculate the total number of gallons needed, which will then give us the cost of fuel.






\begin{question} 


If the truck travels at a steady speed of $v$, then what will be the cost of fuel? \\

Fuel Cost = $ 3.09 \cdot \answer{\frac{750}{4-(v-50)/10}}$ dollars.
\end{question}



That gives a total cost based on speed

\[
Cost(v) = \frac{27 \cdot 750}{v} + \frac{12 \cdot 750}{v} + \frac{3.09 \cdot 750}{4-(v-50)/10}
\]





\begin{center}
\desmos{byvuwohbqw}{400}{300}
\end{center}



From the graph, we can see that there is a minimum cost of \$1161 when the truck is travelling at a speed of approximately $47.62$ miles per hour.




However, we would like an exact speed. \\






\subsection{with Calculus}

Calculus tells us that the derivative of $Cost(v)$ is


\[
Cost'(v) = \frac{23175}{(90-v)^2} - \frac{29250}{v^2}
\]



We can obtain the critical number by setting $Cost'(v)$ equal to $0$ and solving.



By obtaingin common denominators and combining into a single fraction, we get a numerator of

\[
\answer{23175} \, v^2 - \answer{29250} \, (90-v)^2
\]


We can multiply this out and gather like terms to get a quadratic equation

\[
\answer{6075} \, v^2 - \answer{5265000} \, v + 236925000 = 0
\]



There are two solutions, but only one is near $47.62$.


\[
v = \frac{10}{3} \cdot \left( \answer{130} - \sqrt{13390} \right) \approx 47.62
\]













\begin{center}
\textbf{\textcolor{green!50!black}{ooooo=-=-=-=-=-=-=-=-=-=-=-=-=ooOoo=-=-=-=-=-=-=-=-=-=-=-=-=ooooo}} \\

more examples can be found by following this link\\ \link[More Examples of Model Analysis]{https://ximera.osu.edu/csccmathematics/precalculus2/precalculus2/modelAnalysis/examples/exampleList}

\end{center}





\end{document}
