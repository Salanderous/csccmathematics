\documentclass{ximera}


\graphicspath{
  {./}
  {ximeraTutorial/}
  {basicPhilosophy/}
}

\newcommand{\mooculus}{\textsf{\textbf{MOOC}\textnormal{\textsf{ULUS}}}}

\usepackage{tkz-euclide}\usepackage{tikz}
\usepackage{tikz-cd}
\usetikzlibrary{arrows}
\tikzset{>=stealth,commutative diagrams/.cd,
  arrow style=tikz,diagrams={>=stealth}} %% cool arrow head
\tikzset{shorten <>/.style={ shorten >=#1, shorten <=#1 } } %% allows shorter vectors

\usetikzlibrary{backgrounds} %% for boxes around graphs
\usetikzlibrary{shapes,positioning}  %% Clouds and stars
\usetikzlibrary{matrix} %% for matrix
\usepgfplotslibrary{polar} %% for polar plots
\usepgfplotslibrary{fillbetween} %% to shade area between curves in TikZ
\usetkzobj{all}
\usepackage[makeroom]{cancel} %% for strike outs
%\usepackage{mathtools} %% for pretty underbrace % Breaks Ximera
%\usepackage{multicol}
\usepackage{pgffor} %% required for integral for loops



%% http://tex.stackexchange.com/questions/66490/drawing-a-tikz-arc-specifying-the-center
%% Draws beach ball
\tikzset{pics/carc/.style args={#1:#2:#3}{code={\draw[pic actions] (#1:#3) arc(#1:#2:#3);}}}



\usepackage{array}
\setlength{\extrarowheight}{+.1cm}
\newdimen\digitwidth
\settowidth\digitwidth{9}
\def\divrule#1#2{
\noalign{\moveright#1\digitwidth
\vbox{\hrule width#2\digitwidth}}}






\DeclareMathOperator{\arccot}{arccot}
\DeclareMathOperator{\arcsec}{arcsec}
\DeclareMathOperator{\arccsc}{arccsc}

















%%This is to help with formatting on future title pages.
\newenvironment{sectionOutcomes}{}{}


\title{Theory}

\begin{document}

\begin{abstract}
the logarithmic story
\end{abstract}
\maketitle



$log_A(B)$ is pronounced ``the logarithm base $A$ of $B$'' or ``the log base $A$ of $B$'' or ``the logarithm of $B$ base $A$''. \\




\begin{center}
\textbf{\textcolor{red!80!black}{$log_A(B)$ is the thing you raise $A$ to, to get $B$.}}
\end{center}


Logarithms are exponents.  They are the things we raise bases to, to get targeted outputs. Therefore, to understand logarithms, we should investigate the exponents in exponential functions. \\











An example of a basic exponential function is $E(t) = 2^t$.  

Its graph looks like







\begin{image}
\begin{tikzpicture}
  \begin{axis}[
            domain=-10:10, ymax=10, xmax=10, ymin=-10, xmin=-10,
            axis lines =center, xlabel=$t$, ylabel=$y$, grid = major,
            ytick={-10,-8,-6,-4,-2,2,4,6,8,10},
          	xtick={-10,-8,-6,-4,-2,2,4,6,8,10},
          	ticklabel style={font=\scriptsize},
            every axis y label/.style={at=(current axis.above origin),anchor=south},
            every axis x label/.style={at=(current axis.right of origin),anchor=west},
            axis on top
          ]
          
          	\addplot [line width=1, gray, dashed,samples=200,domain=(-10:10),<->] {0};

      		\addplot [line width=2, penColor, smooth,samples=200,domain=(-10:3.2),<->] {2^x};

      		\addplot[color=penColor,fill=penColor,only marks,mark=*] coordinates{(0,1)};

          


  \end{axis}
\end{tikzpicture}
\end{image}

$E(t)$ is an increasing function, which means when we select a greater exponent in the domain, then we get a greater function value.  



We can evalutae our exponential function:
\begin{itemize}
\item $E(0) = 2^0 = 2$
\item $E\left(\frac{1}{2}\right) = 2^{\tfrac{1}{2}} = \sqrt{2}$
\item $E(-1) = 2^{-1} = \frac{1}{2}$
\end{itemize}


When we evaluate this exponential function, we know the domain number and we seek its range partner. In this case, we know $d$ and we assemble the pair $(d, 2^d)$, where $2^d$ is the value of the function at $d$.



Here we are supplying an exponent from the domain and then obtaining a function value.  We can look at this process in reverse: we know the function value and we are in search of an exponent that will produce the desired output.









\section{Reverse}

We can think about $E(t) = 2^t$ in reverse.
\begin{itemize}
\item $E(t) = 4$
\item $E(t) = {\tfrac{1}{4}} $
\item $E(t) = 16 $
\end{itemize}


Here, we know the value of the function.  We seek the domain numbers paired with it. We seek the exponent. We know $2^d$, we seek the pair $(d, 2^d)$, where $d$ is the solution to the equation $E(t) = d$.




\textbf{\textcolor{red!90!darkgray}{$\bigstar$}}  Since $E(t) = 2^t > 0$ for any $t$, thinking backwards only works if we provide positive function values. $2^t$ cannot equal a negative value. $2^t$ cannot equal $0$.   \\ 



\textbf{\textcolor{red!90!darkgray}{$\bigstar$}} For $E(t) = 2^t$, every positive function value has \textit{exactly one} associated domain number, one exponent, one logarithm. (That is eeriely similar to the function rule for pairs in the exponential function.) \\ 




Thinking of an exponential function in reverse sounds like a new function. We call it a \textbf{\textcolor{purple!85!blue}{logarithmic function}}.   \\


\begin{itemize}
\item In reverse, the range of an exponential function becomes the domain of a logarithmic function.
\item In reverse, the domain of an exponential function becomes the range of a logarithmic function.
\end{itemize}





\[
\begin{array}{lcl}
\text{Exponential Function}  &     &  \text{Logarithmic Function}  \\
domain = (-\infty, \infty)  &  \   &  domain = (0, \infty)  \\
range = (0, \infty)  &    &  range = (-\infty, \infty)  \\
(a, r^a)    &    &   (r^a, a)
\end{array}
\]


The logarithmic function just reverses the pairs in the exponential function.  If $(a, 2^a)$ is a pair in the exponential function, then $(2^a, a)$ is a pair in the logarithmic function.


Therefore, the domains and ranges switch. \\






\subsection{Logarithm}

It seems weird to write $(2^a, a)$, even though it is perfectly correct.  We are supplying values of an exponential function, which look like $2^a$ and then the value of the logarithmic function is the needed exponent.

Instead, we are used to writing the domain number by a letter, like $b$, and then the function value as a formula involving $b$. That's what we are used to.

We would prefer our pairs in the logarithm function to look like 

\[ (b, expression)  \]







Logarithmic functions come from the study of \textbf{logarithms}, which is the study of the exponents, just like we are talking about. This gets shortened to \textbf{log} for notation purposes.  \\



In our example here, we are working with the base $2$ exponential function, $E(t) = 2^t$. So, the reverse is called \textbf{the logarithm base $2$}.  We tack on a subscript $2$ to the log to remind us of the base.



\[   log_2(b)     \]


With this new symbol, we can describe pairs in the logarithmic function and points on its graph like 


\[
(b, log_2(b))
\]


$log_2(b)$ is just the exponent that $2$ needs to equal $b$.





$\blacktriangleright$ \textbf{Remember:} These pairs are the reverse of the exponential pairs : $(a, 2^a)$.  Therefore, $(b, log_2(b))$ is also $(2^c, c)$ for some $c$. \\

$\blacktriangleright$  $log_2(b)$ is the exponent, so that $2^? = b$ \\


$\blacktriangleright$  $log_2(b)$ is the number that you raise $2$ to, to get $b$.  \\

\[   2^{log_2(b)} = b     \]

We have a logarithmic function for every exponential function.  They are designated by their bases.
















\begin{definition} \textbf{\textcolor{green!50!black}{Logarithm Base $A$ : $log_A(B)$}}


$log_A(B)$ is the number you raise $A$ to, to get $B$. \\

$B \in (0, \infty)$. \\

$log_B(A) \in (-\infty, \infty)$.


\end{definition}










\begin{definition} \textbf{\textcolor{green!50!black}{Logarithm Base $A$ Function : $log_A(x)$}}


$log_A(x)$ is the number you raise $A$ to, to get $x$. \\

The domain is $(0, \infty)$. \\

The range is $(-\infty, \infty)$.


\end{definition}




\begin{conclusion}

\begin{itemize}
\item $log_A(1)$ is the thing you raise $A$ to, to get $1$. Therefore, $log_A(1) = 0$, since $A^0 = 1$
\item $log_A(A)$ is the thing you raise $A$ to, to get $A$. Therefore, $log_A(A) = 1$, since $A^1 = A$
\item $log_A(\tfrac{1}{A})$ is the thing you raise $A$ to, to get $\tfrac{1}{A}$. Therefore, $log_A(\tfrac{1}{A}) = -1$, since $A^{-1} = \tfrac{1}{A}$
\item $log_A(A^n)$ is the thing you raise $A$ to, to get $A^n$. Therefore, $log_A(A^n) = n$, since $A^n = A^n$
\end{itemize}

\end{conclusion}



Since all of the pairs of exponential and logarithmic functions are reversed, the graphs switch axes.



The horizontal axis is an asymptote for the basic exponential function. Now, the vertical axis is an asymptote for the basic logarithmic function.


The graph of a basic exponential function has $(0,1)$ as an intercept.  The graph of a basic logarithmic function has $(1,0)$ as an intercept. 


Their graphs are mirror images of each other across the diagonal through quandrants I and III.










\begin{image}
\begin{tikzpicture}
  \begin{axis}[
            domain=-10:10, ymax=10, xmax=10, ymin=-10, xmin=-10,
            axis lines =center, xlabel=$domain$, ylabel=$range$, grid = major,
            ytick={-10,-8,-6,-4,-2,2,4,6,8,10},
          	xtick={-10,-8,-6,-4,-2,2,4,6,8,10},
          	ticklabel style={font=\scriptsize},
            every axis y label/.style={at=(current axis.above origin),anchor=south},
            every axis x label/.style={at=(current axis.right of origin),anchor=west},
            axis on top
          ]
          
          	\addplot [line width=1, gray, dashed,samples=200,domain=(-10:10),<->] {0};
          	\addplot [line width=1, gray, dashed,samples=200,domain=(-10:10),<->] ({0},{x});
          	\addplot [line width=1, gray, dashed,samples=200,domain=(-10:10),<->] ({x},{x});

      		\addplot [line width=2, penColor, smooth,samples=200,domain=(-10:3.2),<->] {2^x};
      		\addplot [line width=2, penColor2, smooth,samples=200,domain=(0.003:8),<->] {ln(x)/ln(2)};

          	


      		\addplot[color=penColor,fill=penColor,only marks,mark=*] coordinates{(0,1)};
      		\addplot[color=penColor2,fill=penColor2,only marks,mark=*] coordinates{(1,0)};

      		\node[penColor] at (axis cs:1.5,8) {$2^d$};
      		\node[penColor2] at (axis cs:7,1) {$log_2(d)$};





           

  \end{axis}
\end{tikzpicture}
\end{image}




From this basic logarithmic function and its graph we can transform and analyze general logarithmic functions.
















\begin{example}  General Logarithmic Function



Analyze   $M(t) = log_2(t+3) - 4$ \\


\begin{explanation}

$\blacktriangleright$ The inside of the logarithm is $t+3$ and this equals $0$ when $t=-3$.  This must be the vertical asymptote.

$\blacktriangleright$ The inside of the logarithm is $t+3$, and this is positive for $t>-3$.  The graph must be on the right side of the vertical asymptote.

$\blacktriangleright$ The leading coefficient is $1$, which is positive.  Therefore, the graph hugs the asymptote down the asymptote. 

$\blacktriangleright$ $t+3=1$ when $t=-2$. Therefore, the anchor point $(1,0)$ has moved to $t = -2$.  $M(-2) = -4$.  This gives us the point $(-2, -4)$.





Graph of $y = M(t)$.

\begin{image}
\begin{tikzpicture}
  \begin{axis}[
            domain=-10:10, ymax=10, xmax=10, ymin=-10, xmin=-10,
            axis lines =center, xlabel=$t$, ylabel=$y$, grid = major,
            ytick={-10,-8,-6,-4,-2,2,4,6,8,10},
            xtick={-10,-8,-6,-4,-2,2,4,6,8,10},
            ticklabel style={font=\scriptsize},
            every axis y label/.style={at=(current axis.above origin),anchor=south},
            every axis x label/.style={at=(current axis.right of origin),anchor=west},
            axis on top
          ]
          
			\addplot [line width=1, gray, dashed,samples=200,domain=(-10:10),<->] ({-3},{x});
			\addplot [line width=2, penColor, smooth,samples=200,domain=(-2.95:8),<->] {ln(x+3)/ln(2)-4};


			\addplot[color=penColor,fill=penColor,only marks,mark=*] coordinates{(-2,-4)};



  \end{axis}
\end{tikzpicture}
\end{image}




Our analysis tells us that:

\begin{itemize}
\item The implied domain of $M$ is $\left( \answer{-3} ,\infty \right)$.
\item The implied range of $M$ is $\mathbb{R}$.
\item $M$ is always \wordChoice{\choice[correct]{increasing} \choice{decreasing}} .
\item $M$ has no maximums or minimums.
\item $\lim\limits_{t \to -3^+} M(t) = -\infty$
\item $\lim\limits_{t \to \infty} M(t) = \infty$
\end{itemize}




We can also see that the graph will have a horizontal intercept, which means the function has a zero. \\


$M(t) = log_2(t+3) - 4 = 0$ \\


\begin{align*}
log_2(t+3) - 4 & = 0 \\
log_2(t+3) & = 4 \\
2^{log_2(t+3)} & = \answer{2^4} \\
\answer{t+3} & = 16 \\
t & = 13
\end{align*}


$\blacktriangleright$ \textbf{Remember:} $log_2(t+3)$ is the thing that you raise $2$ to, to get $t+3$ and $log_2(t+3) = 4$.  Therefore, $4$ is also the thing that you raise $2$ to, to get $t+3$. $t+3$ must be $16$.









\end{explanation}

\end{example}



$M(t)$ is a logarithmic function, so it must have an inverse partner exponential function.  The pairs for $M$ look like $(t, M(t))$ or just $(t,M)$. The pairs for the exponential function would look like $(M, t)$.  The roles of $M$ and $t$ would be switched. $M$ would be the variable in the formula. $t$ would be the function value.\\


$M = log_2(t+3) - 4$ is an equation describing both the logarithmic function, $M(t)$, and the exponential function $t(M)$.  We just are used to having the function name by itself on one side of the equal sign. \\


We can obtain such a formula for this partner exponential function by solving the logarithmic equation for $t$.





\begin{align*}
log_2(t+3) - 4 & = M \\
log_2(t+3) & = M + 4 \\
2^{log_2(t+3)} & = 2^{M+4} \\
t+3 & = 2^{M+4} \\
t & = 2^{M+4} - 3
\end{align*}


$\blacktriangleright$ \textbf{Remember:} $log_2(t+3)$ is the thing that you raise $2$ to, to get $t+3$ and $log_2(t+3) = M+4$.  Therefore, $M+4$ is the thing that you raise $2$ to, to get $t+3$





Here is the graph of both the logarithmic and the associated exponential functions. They are inverses of each other.





\begin{image}
\begin{tikzpicture}
  \begin{axis}[
            domain=-10:10, ymax=10, xmax=10, ymin=-10, xmin=-10,
            axis lines =center, xlabel=$domain$, ylabel=$range$, grid = major,
            ytick={-10,-8,-6,-4,-2,2,4,6,8,10},
            xtick={-10,-8,-6,-4,-2,2,4,6,8,10},
            ticklabel style={font=\scriptsize},
            every axis y label/.style={at=(current axis.above origin),anchor=south},
            every axis x label/.style={at=(current axis.right of origin),anchor=west},
            axis on top
          ]
          
			\addplot [line width=1, gray, dashed,samples=200,domain=(-10:10),<->] ({-3},{x});
            \addplot [line width=1, gray, dashed,samples=200,domain=(-10:10),<->] ({x},{-3});
            \addplot [line width=1, gray, dashed,samples=200,domain=(-10:10),<->] ({x},{x});

			\addplot [line width=2, penColor, smooth,samples=200,domain=(-2.95:8),<->] {ln(x+3)/ln(2)-4};
			\addplot [line width=2, penColor2, smooth,samples=200,domain=(-8:-0.25),<->] {2^(x+4)-3};


			\addplot[color=penColor,fill=penColor,only marks,mark=*] coordinates{(-2,-4)};
			\addplot[color=penColor2,fill=penColor2,only marks,mark=*] coordinates{(-4,-2)};






           

  \end{axis}
\end{tikzpicture}
\end{image}
























\begin{example}  Logarithmic Function


Analyze   $K(x) = -2 \,log_3(4-x)$ \\


\begin{explanation}


$\blacktriangleright$ The inside of the logarithm is $4-x$, and this equals $0$ when $x=4$.  THerefore, $x=4$ has to be the vertical asymptote.

$\blacktriangleright$ The inside of the logarithm is $4-x$, and this is positive for $x<4$.  The graph must be on the left side of the vertical asymptote.

$\blacktriangleright$ The leading coefficient is $-2$, which is negative.  Therefore, the graph is flipped \wordChoice{\choice[correct]{vertically} \choice{horizontally}} from the basic graph. It goes up the asymptote. 

$\blacktriangleright$ $4-x=1$ when $x=3$. Therefore, the anchor point $(1,0)$ has moved to $\left( \answer{3}, 0 \right)$.





Graph of $y = K(x)$.

\begin{image}
\begin{tikzpicture}
  \begin{axis}[
            domain=-10:10, ymax=10, xmax=10, ymin=-10, xmin=-10,
            axis lines =center, xlabel=$x$, ylabel=$y$, grid = major,
            ytick={-10,-8,-6,-4,-2,2,4,6,8,10},
            xtick={-10,-8,-6,-4,-2,2,4,6,8,10},
            ticklabel style={font=\scriptsize},
            every axis y label/.style={at=(current axis.above origin),anchor=south},
            every axis x label/.style={at=(current axis.right of origin),anchor=west},
            axis on top
          ]
          
      \addplot [line width=1, gray, dashed,samples=200,domain=(-10:10),<->] ({4},{x});
			\addplot [line width=2, penColor, smooth,samples=200,domain=(-9:3.995),<->] {-2*ln(4-x)/ln(3)};
      

		
            %\addplot [line width=1, gray, dashed,samples=200,domain=(-10:10),<->] ({x},{x});


			\addplot[color=penColor,fill=penColor,only marks,mark=*] coordinates{(3,0)};






           

  \end{axis}
\end{tikzpicture}
\end{image}









\begin{itemize}
\item The natural or implied domain of $K$ is $(-\infty, 4)$.
\item The implied range of $K$ is $\mathbb{R}$.
\item $K$ is always increasing.
\item $K$ has no maximums or minimums.
\item $\lim\limits_{x \to 4^-} K(x) = \infty$
\item $\lim\limits_{x \to -\infty} K(x) = -\infty$
\end{itemize}




\end{explanation}
\end{example}









Reversing all of the pairs will give the inverse exponential function.  \\



$K(x)$ is a logarithmic function, so it must have a partner exponential function.  The pairs for $K$ look like $(x, K)$. The pairs for the exponential function would look like $(K, x)$.  The roles of $K$ and $x$ would be switched. $K$ would be the variable in the exponential formula. $x$ would be the exponential function value.  $K$ would be the independent variable and $x$ would be the dependent variable.


We can obtain a nice formula for this partner exponential function by solving the logarithmic equation for $x$.





\begin{align*}
-2 \,log_3(4-x) & = M \\
log_3(4-x) & = -\frac{M}{2} \\
3^{log_3(4-x)} & = 3^{-\frac{M}{2}} \\
4 - x & = 3^{-\frac{M}{2}} \\
4 - 3^{-\frac{M}{2}} & = x \\
\end{align*}




$\blacktriangleright$ \textbf{Remember:} $log_3(4-x)$ is the thing that you raise $3$ to, to get $4-x$ and $log_3(4-x) = - \frac{M}{2}$.  Therefore, $- \frac{M}{2}$ is the thing that you raise $3$ to, to get $4-x$





Here is the graph of both the logarithmic and exponential functions.  They are reflected about the diagonal, because the functions are inverses of each other. 





\begin{image}
\begin{tikzpicture}
  \begin{axis}[
            domain=-10:10, ymax=10, xmax=10, ymin=-10, xmin=-10,
            axis lines =center, xlabel=$domain$, ylabel=$range$, grid = major,
            ytick={-10,-8,-6,-4,-2,2,4,6,8,10},
            xtick={-10,-8,-6,-4,-2,2,4,6,8,10},
            ticklabel style={font=\scriptsize},
            every axis y label/.style={at=(current axis.above origin),anchor=south},
            every axis x label/.style={at=(current axis.right of origin),anchor=west},
            axis on top
          ]
          
          	\addplot [line width=1, gray, dashed,samples=200,domain=(-10:10),<->] ({4},{x});
          	\addplot [line width=1, gray, dashed,samples=200,domain=(-10:10),<->] ({x},{4});
          	\addplot [line width=1, gray, dashed,samples=200,domain=(-10:10),<->] ({x},{x});


			\addplot [line width=2, penColor, smooth,samples=200,domain=(-9:3.995),<->] {-2*ln(4-x)/ln(3)};

			\addplot[color=penColor,fill=penColor,only marks,mark=*] coordinates{(3,0)};

			\addplot [line width=2, penColor2, smooth,samples=200,domain=(-4.627:9),<->] {4-3^(-x/2)};

			\addplot[color=penColor2,fill=penColor2,only marks,mark=*] coordinates{(0,3)};

           

  \end{axis}
\end{tikzpicture}
\end{image}








\begin{definition} \textbf{\textcolor{green!50!black}{Natural Logarithm}}   \\


The \textbf{natural logarithm} is the logarithm base $e$.  \\

It has special notation.

\[
log_e(x) = ln(x)
\]

\end{definition}



\begin{question}$\boxdot$

Evaluate the following.

\begin{itemize}
\item $e^{ln(4)} = \answer{4}$ 
\item $ln(e^7) = \answer{7}$
\item $ln\left( \frac{1}{e} \right) = \answer{-1}$
\end{itemize}



\end{question}



\begin{remark} \textbf{\textcolor{purple!85!blue}{e}} \\

\textbf{\textcolor{blue!55!black}{Remember:}} The function $\left( 1 + \frac{1}{x}  \right)^{x}$ has a limiting value.  Its end-behavior is a constant.  Its graph has a horizontal asymptote.  $e$ is this limiting value.

It might seem strange to select that number as the base for a logarithm and give it special notation.  However, it turns out that exponential functions with this base, $e$, are intricately connected to natural growth in the universe.

It is use A LOT!
\end{remark}










\begin{center}
\textbf{\textcolor{green!50!black}{ooooo=-=-=-=-=-=-=-=-=-=-=-=-=ooOoo=-=-=-=-=-=-=-=-=-=-=-=-=ooooo}} \\

more examples can be found by following this link\\ \link[More Examples of Logarithmic Functions]{https://ximera.osu.edu/csccmathematics/precalculus2/precalculus2/logFunctions/examples/exampleList}

\end{center}






\end{document}
