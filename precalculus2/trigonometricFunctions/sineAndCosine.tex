\documentclass{ximera}


\graphicspath{
  {./}
  {ximeraTutorial/}
  {basicPhilosophy/}
}

\newcommand{\mooculus}{\textsf{\textbf{MOOC}\textnormal{\textsf{ULUS}}}}

\usepackage{tkz-euclide}\usepackage{tikz}
\usepackage{tikz-cd}
\usetikzlibrary{arrows}
\tikzset{>=stealth,commutative diagrams/.cd,
  arrow style=tikz,diagrams={>=stealth}} %% cool arrow head
\tikzset{shorten <>/.style={ shorten >=#1, shorten <=#1 } } %% allows shorter vectors

\usetikzlibrary{backgrounds} %% for boxes around graphs
\usetikzlibrary{shapes,positioning}  %% Clouds and stars
\usetikzlibrary{matrix} %% for matrix
\usepgfplotslibrary{polar} %% for polar plots
\usepgfplotslibrary{fillbetween} %% to shade area between curves in TikZ
\usetkzobj{all}
\usepackage[makeroom]{cancel} %% for strike outs
%\usepackage{mathtools} %% for pretty underbrace % Breaks Ximera
%\usepackage{multicol}
\usepackage{pgffor} %% required for integral for loops



%% http://tex.stackexchange.com/questions/66490/drawing-a-tikz-arc-specifying-the-center
%% Draws beach ball
\tikzset{pics/carc/.style args={#1:#2:#3}{code={\draw[pic actions] (#1:#3) arc(#1:#2:#3);}}}



\usepackage{array}
\setlength{\extrarowheight}{+.1cm}
\newdimen\digitwidth
\settowidth\digitwidth{9}
\def\divrule#1#2{
\noalign{\moveright#1\digitwidth
\vbox{\hrule width#2\digitwidth}}}






\DeclareMathOperator{\arccot}{arccot}
\DeclareMathOperator{\arcsec}{arcsec}
\DeclareMathOperator{\arccsc}{arccsc}

















%%This is to help with formatting on future title pages.
\newenvironment{sectionOutcomes}{}{}


\title{Sine \& Cosine}

\begin{document}

\begin{abstract}
around the unit circle
\end{abstract}
\maketitle








For every counterclockwise angle, $\theta$, measured from the positive $x$-axis, we have a point on the unit circle. The coordinates of this point are functions of $\theta$.  We call them \textbf{cosine} and \textbf{sine}.






\begin{image}
\begin{tikzpicture}
  \begin{axis}[
            xmin=-1.1,xmax=1.1,ymin=-1.1,ymax=1.1,
            axis lines=center,
            width=4in,
            xtick={-1,1},
            ytick={-1,1},
            clip=false,
            unit vector ratio*=1 1 1,
            xlabel=$x$, ylabel=$y$,
            ticklabel style={font=\scriptsize},
            every axis y label/.style={at=(current axis.above origin),anchor=south},
            every axis x label/.style={at=(current axis.right of origin),anchor=west},
          ]        
          \addplot [smooth, domain=(0:360)] ({cos(x)},{sin(x)}); %% unit circle

          \addplot [textColor] plot coordinates {(0,0) (.766,.643)}; %% 40 degrees

          \addplot [ultra thick,penColor] plot coordinates {(.766,0) (.766,.643)}; %% 40 degrees
          \addplot [ultra thick,penColor2] plot coordinates {(0,0) (.766,0)}; %% 40 degrees
          
          %\addplot [ultra thick,penColor3] plot coordinates {(1,0) (1,.839)}; %% 40 degrees          

          \addplot [textColor,smooth, domain=(0:40)] ({.15*cos(x)},{.15*sin(x)});
          %\addplot [very thick,penColor] plot coordinates {(0,0) (.766,.643)}; %% sector
          %\addplot [very thick,penColor] plot coordinates {(0,0) (1,0)}; %% sector
          %\addplot [very thick, penColor, smooth, domain=(0:40)] ({cos(x)},{sin(x)}); %% sector
          \node at (axis cs:.15,.07) [anchor=west] {$\theta$};
          \node[penColor, rotate=-90] at (axis cs:.84,.322) {$\sin(\theta)$};
          \node[penColor2] at (axis cs:.383,0) [anchor=north] {$\cos(\theta)$};
          %\node[penColor3, rotate=-90] at (axis cs:1.06,.322) {$\tan(\theta)$};
        \end{axis}
\end{tikzpicture}
\end{image}




As the angle $\theta$ changes, sine and cosine oscillate between $-1$ and $1$.








\begin{image}
\begin{tikzpicture}
  \begin{axis}[
            domain=-10:10, ymax=1.5, xmax=10, ymin=-1.5, xmin=-10,
            axis lines =center, xlabel={$\theta$}, ylabel={$y$}, grid = major, grid style={dashed},
            ytick={-1.5,-1,-0.5,0.5,1,1.5},
            xtick={-7.85, -6.28, -4.71, -3.14, -1.57, 0, 1.57, 3.142, 4.71, 6.28, 7.85},
            xticklabels={$\tfrac{-5\pi}{2}$,$-2\pi$,$\tfrac{-3\pi}{2}$,$-\pi$, $\tfrac{-\pi}{2}$, $0$, $\tfrac{\pi}{2}$, $\pi$, $\tfrac{3\pi}{2}$, $2\pi$, $\tfrac{5\pi}{2}$},
            yticklabels={$1.5$,$-1$,$-0.5$,$0.5$,$1$,$1.5$}, 
            ticklabel style={font=\scriptsize},
            every axis y label/.style={at=(current axis.above origin),anchor=south},
            every axis x label/.style={at=(current axis.right of origin),anchor=west},
            axis on top
          ]
          

            \addplot [line width=2, penColor, smooth,samples=300,domain=(-10:10),<->] {sin(deg(x)};
            \addplot [line width=2, penColor2, smooth,samples=300,domain=(-10:10),<->] {cos(deg(x)};


            \node[penColor] at (axis cs:3.5,1.1) [anchor=north] {$\sin(\theta)$};
            \node[penColor2] at (axis cs:1.4,1.3) [anchor=north] {$\cos(\theta)$};



  \end{axis}
\end{tikzpicture}
\end{image}



Both are periodic functionc with period $2\pi$.  Therefore, we can examine just one wave to understand the whole functions.


Generally, people pick $[0, 2\pi)$ as the interval to investigate.





\begin{image}
\begin{tikzpicture}
  \begin{axis}[
            domain=0:7, ymax=1.5, xmax=7, ymin=-1.5, xmin=0,
            axis lines =center, xlabel={$\theta$}, ylabel={$y$}, grid = major, grid style={dashed},
            ytick={-1.5,-1,-0.5,0.5,1,1.5},
            xtick={1.57, 3.142, 4.71, 6.28, 7.85},
            xticklabels={$\tfrac{\pi}{2}$, $\pi$, $\tfrac{3\pi}{2}$, $2\pi$, $\tfrac{5\pi}{2}$},
            yticklabels={$1.5$,$-1$,$-0.5$,$0.5$,$1$,$1.5$}, 
            ticklabel style={font=\scriptsize},
            every axis y label/.style={at=(current axis.above origin),anchor=south},
            every axis x label/.style={at=(current axis.right of origin),anchor=west},
            axis on top
          ]
          

            \addplot [line width=2, penColor, smooth,samples=300,domain=(0:6.28)] {sin(deg(x)};
            \addplot [line width=2, penColor2, smooth,samples=300,domain=(0:6.28)] {cos(deg(x)};

            \addplot [color=penColor,only marks,mark=*] coordinates{(0,0)};
            \addplot [color=penColor,fill=white,only marks,mark=*] coordinates{(6.28,0)};
            \addplot [color=penColor2,only marks,mark=*] coordinates{(0,1)};
            \addplot [color=penColor2,fill=white,only marks,mark=*] coordinates{(6.28,1)};

            \node[penColor] at (axis cs:2.8,1.1) [anchor=north] {$\sin(\theta)$};
            \node[penColor2] at (axis cs:0.5,1.3) [anchor=north] {$\cos(\theta)$};

  \end{axis}
\end{tikzpicture}
\end{image}





We can see that sine and cosine have similar characteristics, just shifted.


\section{Characteristics}





Picture the radius spinning counterclockwise around the unit circle.  The corresponding point on the unit circle will travel counterclockwise around the circle.  As the point moves around the circle, its coordinates oscillates between $-1$ and $1$.



The sine function collects the vertical coordinates.


\begin{center}
\desmos{m7devoo2w7}{400}{300}
\end{center}

As the point revolves counterclockwise arounf the unit circle, the values of sine 

\begin{itemize}
\item begin at $0$ at an angle of $0$
\item increase up to a maximum value of $1$ at an angle of $\frac{\pi}{2}$
\item decrease to $0$ at an angle of $\pi$
\item continue decreasing to a minimum value of $-1$ at an angle of $\frac{3 \pi}{2}$
\item increase back to $0$ at an angle of $2 \pi$
\end{itemize}










The cosine function collects the horizontal coordinates.

\begin{center}
\desmos{3t8cjtu1ak}{400}{300}
\end{center}

As the point revolves counterclockwise arounf the unit circle, the values of cosine 

\begin{itemize}
\item begin at a maximum value of $1$ at an angle of $0$
\item decrease to a value of $0$ at an angle of $\frac{\pi}{2}$
\item continue decreasing to a minimum value of $-1$ at an angle of $\pi$
\item increasing to a value of $0$ at an angle of $\frac{3 \pi}{2}$
\item continue increasing back to $1$ at an angle of $2 \pi$
\end{itemize}














\begin{example}



Analyze  $T(r) = 3 sin(2r - \pi) - 1$


\begin{explanation}

$\blacktriangleright$ \textbf{Horizontal}


One period of the sine wave occurs from where the inside equals $0$ to where the inside equals $2 \pi$.


\begin{align*}
2r - \pi & =  0 \\
2r       & =  \pi \\
r        & =  \frac{\pi}{2}
\end{align*}



\begin{align*}
2r - \pi & =  2 \pi \\
2r       & =  3 \pi \\
r        & =  \frac{3 \pi}{2}
\end{align*}


$sin(2r - \pi)$ would equal $0$ at $\frac{\pi}{2}$ and $\frac{3 \pi}{2}$.

Halfway is $\pi$ and $sin(2r - \pi)$ would again equal $0$ there.

A quarterway is $\frac{3 \pi}{4}$, where $sin(2r - \pi) = 1$.

Three-quarterway is $\frac{5 \pi}{4}$, where $sin(2r - \pi) = -1$.






$\blacktriangleright$ \textbf{Vertical}


The maximum value of $sin(2r - \pi)$ is $1$.  Therefore, the maximum value of $T(r)$ is $3 \cdot 1 - 1 = 2$, which occurs at $\frac{3 \pi}{4}$. 




The minimum value of $sin(2r - \pi)$ is $-1$.  Therefore, the minimum value of $T(r)$ is $3 \cdot (-1) - 1 = -4$, which occurs at $\frac{5 \pi}{4}$. 










\begin{image}
\begin{tikzpicture}
  \begin{axis}[
            domain=0:7, ymax=5, xmax=7, ymin=-5, xmin=0,
            axis lines =center, xlabel={$r$}, ylabel={$y$}, grid = major, grid style={dashed},
            ytick={-4,-2,0,2,4},
            xtick={1.57, 3.142, 4.71, 6.28, 7.85},
            xticklabels={$\tfrac{\pi}{2}$, $\pi$, $\tfrac{3\pi}{2}$, $2\pi$, $\tfrac{5\pi}{2}$},
            yticklabels={$-4$,$-2$,$0$,$2$,$4$}, 
            ticklabel style={font=\scriptsize},
            every axis y label/.style={at=(current axis.above origin),anchor=south},
            every axis x label/.style={at=(current axis.right of origin),anchor=west},
            axis on top
          ]
          

            \addplot [line width=2, penColor, smooth,samples=300,domain=(1.57:4.71)] {3*sin(deg(2*x-3.14))-1};

            %\addplot [color=penColor,only marks,mark=*] coordinates{(0,0)};
            %\addplot [color=penColor,fill=white,only marks,mark=*] coordinates{(6.28,0)};
            %\addplot [color=penColor2,only marks,mark=*] coordinates{(0,1)};
            %\addplot [color=penColor2,fill=white,only marks,mark=*] coordinates{(6.28,1)};


  \end{axis}
\end{tikzpicture}
\end{image}



Now we can extend this to the whole real line.


The period is $\frac{3 \pi}{2} - \frac{\pi}{2} = \pi$.




\begin{itemize}
\item $T(r)$ has a maximum of $2$, which occurs at $\{  \frac{3 \pi}{4} \pm k \pi \, | \, k \in \mathbb{Z} \}$.

\item $T(r)$ has a minimum of $-4$, which occurs at $\{  \frac{5 \pi}{4} \pm k \pi \, | \, k \in \mathbb{Z} \}$.

\item $T(r)$ decreasing on $\left[  \frac{3 \pi}{4} \pm k \pi,    \frac{5 \pi}{4} \pm k \pi    \right]$, where $k \in \mathbb{Z}$.

\item $T(r)$ increasing on $\left[  \frac{5 \pi}{4} \pm k \pi,    \frac{3 \pi}{4} \pm (k+1) \pi    \right]$, where $k \in \mathbb{Z}$.
\end{itemize}




\begin{image}
\begin{tikzpicture}
  \begin{axis}[
            domain=0:7, ymax=5, xmax=7, ymin=-5, xmin=0,
            axis lines =center, xlabel={$r$}, ylabel={$y$}, grid = major, grid style={dashed},
            ytick={-4,-2,0,2,4},
            xtick={1.57, 3.142, 4.71, 6.28, 7.85},
            xticklabels={$\tfrac{\pi}{2}$, $\pi$, $\tfrac{3\pi}{2}$, $2\pi$, $\tfrac{5\pi}{2}$},
            yticklabels={$-4$,$-2$,$0$,$2$,$4$}, 
            ticklabel style={font=\scriptsize},
            every axis y label/.style={at=(current axis.above origin),anchor=south},
            every axis x label/.style={at=(current axis.right of origin),anchor=west},
            axis on top
          ]
          

            \addplot [line width=2, penColor, smooth,samples=300,domain=(0:6.28),<->] {3*sin(deg(2*x-3.14))-1};

            %\addplot [color=penColor,only marks,mark=*] coordinates{(0,0)};
            %\addplot [color=penColor,fill=white,only marks,mark=*] coordinates{(6.28,0)};
            %\addplot [color=penColor2,only marks,mark=*] coordinates{(0,1)};
            %\addplot [color=penColor2,fill=white,only marks,mark=*] coordinates{(6.28,1)};


  \end{axis}
\end{tikzpicture}
\end{image}




\end{explanation}















\end{example}




























\begin{example}



Analyze  $g(t) = -2 cos(\pi - 3t) + 1$


\begin{explanation}

$\blacktriangleright$ \textbf{Horizontal}


One period of the cosine wave occurs from where the inside equals $0$ to where the inside equals $2 \pi$.


\begin{align*}
\pi - 3t  & =  0 \\
\pi       & =  3t \\
\frac{\pi}{3}       & =  t
\end{align*}



\begin{align*}
\pi - 3t  & =  2 \pi \\
-\pi      & =  3t \\
\frac{-\pi}{3}       & =  t
\end{align*}


$cos(\pi - 3t)$ would equal $1$ at $\frac{\pi}{3}$ and $\frac{-\pi}{3}$. (running backwards)

Halfway is $0$ and $cos(\pi - 3t)$ would equal $-1$ there.

A quarterway is $\frac{\pi}{12}$, where $cos(\pi - 3t) = 0$.  (a zero)

Three-quarterway is $\frac{-\pi}{12}$, where $cos(\pi - 3t) = 0$. (a zero)






$\blacktriangleright$ \textbf{Vertical}


The maximum value of $cos(\pi - 3t)$ is $1$.  The leading coefficient is $-2$, which is negative.  Therefore, the maximum value becomes a minimum value. The minimum value of $g(t)$ is $-2 \cdot 1 + 1 = -1$, which occurs at $\frac{\pi}{3}$ and at $\frac{-\pi}{3}$. 




The minimum value of $cos(\pi - 3t)$ is $-1$.  The leading coefficient is $-2$, which is negative.   Therefore, the minimum value becomes a maximum value.  The maximum value of $g(t)$ is $-2 \cdot (-1) + 1 = 3$, which occurs at $0$. 










\begin{image}
\begin{tikzpicture}
  \begin{axis}[
            domain=-7:7, ymax=5, xmax=7, ymin=-5, xmin=-7,
            axis lines =center, xlabel={$r$}, ylabel={$y$}, grid = major, grid style={dashed},
            ytick={-4,-2,0,2,4},
            xtick={-7.85, -6.28, -4.71, -3.142, -1.57, 0, 1.57, 3.142, 4.71, 6.28, 7.85},
            xticklabels={$\tfrac{-5\pi}{2}$,$-2\pi$,$\tfrac{-3\pi}{2}$,$-\pi$,$\tfrac{-\pi}{2}$,$0$,$\tfrac{\pi}{2}$, $\pi$, $\tfrac{3\pi}{2}$, $2\pi$, $\tfrac{5\pi}{2}$},
            yticklabels={$-4$,$-2$,$0$,$2$,$4$}, 
            ticklabel style={font=\scriptsize},
            every axis y label/.style={at=(current axis.above origin),anchor=south},
            every axis x label/.style={at=(current axis.right of origin),anchor=west},
            axis on top
          ]
          

            \addplot [line width=2, penColor, smooth,samples=300,domain=(-1.047:1.047)] {-2*cos(deg(3.14-(3*x)))+1};

            %\addplot [color=penColor,only marks,mark=*] coordinates{(0,0)};
            %\addplot [color=penColor,fill=white,only marks,mark=*] coordinates{(6.28,0)};
            %\addplot [color=penColor2,only marks,mark=*] coordinates{(0,1)};
            %\addplot [color=penColor2,fill=white,only marks,mark=*] coordinates{(6.28,1)};


  \end{axis}
\end{tikzpicture}
\end{image}



Now we can extend this to the whole real line.


The period is $\frac{\pi}{3} - \frac{-\pi}{3} = \frac{2\pi}{3}$.




\begin{itemize}
\item $T(r)$ has a maximum of $3$, which occurs at $\{  0 \pm k \frac{2\pi}{3} \, | \, k \in \mathbb{Z} \}$.

\item $T(r)$ has a minimum of $-1$, which occurs at $\{  \frac{\pi}{3} \pm k \frac{2\pi}{3} \, | \, k \in \mathbb{Z} \}$.

\item $T(r)$ decreasing on $\left[  0 \pm k \frac{2\pi}{3},    \frac{\pi}{3} \pm k \frac{2\pi}{3}    \right]$, where $k \in \mathbb{Z}$.

\item $T(r)$ increasing on $\left[  \frac{\pi}{3} \pm k \frac{2\pi}{3},    0 \pm (k+1) \frac{2\pi}{3}    \right]$, where $k \in \mathbb{Z}$.
\end{itemize}









\begin{image}
\begin{tikzpicture}
  \begin{axis}[
            domain=-7:7, ymax=5, xmax=7, ymin=-5, xmin=-7,
            axis lines =center, xlabel={$r$}, ylabel={$y$}, grid = major, grid style={dashed},
            ytick={-4,-2,0,2,4},
            xtick={-7.85, -6.28, -4.71, -3.142, -1.57, 0, 1.57, 3.142, 4.71, 6.28, 7.85},
            xticklabels={$\tfrac{-5\pi}{2}$,$-2\pi$,$\tfrac{-3\pi}{2}$,$-\pi$,$\tfrac{-\pi}{2}$,$0$,$\tfrac{\pi}{2}$, $\pi$, $\tfrac{3\pi}{2}$, $2\pi$, $\tfrac{5\pi}{2}$},
            yticklabels={$-4$,$-2$,$0$,$2$,$4$}, 
            ticklabel style={font=\scriptsize},
            every axis y label/.style={at=(current axis.above origin),anchor=south},
            every axis x label/.style={at=(current axis.right of origin),anchor=west},
            axis on top
          ]
          

            \addplot [line width=2, penColor, smooth,samples=300,domain=(-6.5:6.5),<->] {-2*cos(deg(3.14-(3*x)))+1};

            %\addplot [color=penColor,only marks,mark=*] coordinates{(0,0)};
            %\addplot [color=penColor,fill=white,only marks,mark=*] coordinates{(6.28,0)};
            %\addplot [color=penColor2,only marks,mark=*] coordinates{(0,1)};
            %\addplot [color=penColor2,fill=white,only marks,mark=*] coordinates{(6.28,1)};


  \end{axis}
\end{tikzpicture}
\end{image}






\end{explanation}















\end{example}






















\end{document}

