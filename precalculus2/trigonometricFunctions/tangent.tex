\documentclass{ximera}


\graphicspath{
  {./}
  {ximeraTutorial/}
  {basicPhilosophy/}
}

\newcommand{\mooculus}{\textsf{\textbf{MOOC}\textnormal{\textsf{ULUS}}}}

\usepackage{tkz-euclide}\usepackage{tikz}
\usepackage{tikz-cd}
\usetikzlibrary{arrows}
\tikzset{>=stealth,commutative diagrams/.cd,
  arrow style=tikz,diagrams={>=stealth}} %% cool arrow head
\tikzset{shorten <>/.style={ shorten >=#1, shorten <=#1 } } %% allows shorter vectors

\usetikzlibrary{backgrounds} %% for boxes around graphs
\usetikzlibrary{shapes,positioning}  %% Clouds and stars
\usetikzlibrary{matrix} %% for matrix
\usepgfplotslibrary{polar} %% for polar plots
\usepgfplotslibrary{fillbetween} %% to shade area between curves in TikZ
\usetkzobj{all}
\usepackage[makeroom]{cancel} %% for strike outs
%\usepackage{mathtools} %% for pretty underbrace % Breaks Ximera
%\usepackage{multicol}
\usepackage{pgffor} %% required for integral for loops



%% http://tex.stackexchange.com/questions/66490/drawing-a-tikz-arc-specifying-the-center
%% Draws beach ball
\tikzset{pics/carc/.style args={#1:#2:#3}{code={\draw[pic actions] (#1:#3) arc(#1:#2:#3);}}}



\usepackage{array}
\setlength{\extrarowheight}{+.1cm}
\newdimen\digitwidth
\settowidth\digitwidth{9}
\def\divrule#1#2{
\noalign{\moveright#1\digitwidth
\vbox{\hrule width#2\digitwidth}}}






\DeclareMathOperator{\arccot}{arccot}
\DeclareMathOperator{\arcsec}{arcsec}
\DeclareMathOperator{\arccsc}{arccsc}

















%%This is to help with formatting on future title pages.
\newenvironment{sectionOutcomes}{}{}


\title{Tangent}

\begin{document}

\begin{abstract}
quotient
\end{abstract}
\maketitle







Tangent is the ratio of sine and cosine.

\[   \tan(\theta)  =  \frac{\sin(\theta)}{\cos(\theta)}  \]


\begin{itemize}
\item \textbf{zeros:} $\tan(\theta)$ has a zero everywhere that $\sin(\theta)$ has a zero:  all of the half-$\pi$'s.
\item \textbf{singluaries:} $\tan(\theta)$ has a singularity everywhere that $\cos(\theta)$ has a zero:  all of the whole-$\pi$'s.
\end{itemize}

This means intercepts at the half-$\pi$'s and vertical asymptotes at the whole-$\pi$'s.








\begin{image}
\begin{tikzpicture} 
  \begin{axis}[
            domain=-10:10, ymax=10, xmax=10, ymin=-10, xmin=-10,
            xtick={-4.9, -1.7, 1.7, 4.9}, 
            xticklabels={$\frac{-3\pi}{2}$, $\frac{-\pi}{2}$, $\frac{\pi}{2}$, $\frac{3\pi}{2}$},
            axis lines =center,  xlabel={$\theta$}, ylabel=$y$,
            ticklabel style={font=\scriptsize},
            every axis y label/.style={at=(current axis.above origin),anchor=south},
            every axis x label/.style={at=(current axis.right of origin),anchor=west},
            axis on top
          ]
          
            \addplot [line width=2, penColor, smooth,samples=100,domain=(-1.47:1.47), <->] {tan(deg(x))};
            \addplot [line width=2, penColor, smooth,samples=100,domain=(-4.61:-1.67), <->] {tan(deg(x))};
            \addplot [line width=2, penColor, smooth,samples=100,domain=(1.67:4.61), <->] {tan(deg(x))};

      \addplot[color=penColor,fill=penColor,only marks, mark size=1pt, mark=*] coordinates{(-9,5) (-8,5) (-7,5) (7,5) (8,5) (9,5)};


            \addplot [line width=1, gray, dashed,samples=100,domain=(-10:10), <->] ({-4.71},{x});
            \addplot [line width=1, gray, dashed,samples=100,domain=(-10:10), <->] ({-1.57},{x});
            \addplot [line width=1, gray, dashed,samples=100,domain=(-10:10), <->] ({1.57},{x});
            \addplot [line width=1, gray, dashed,samples=100,domain=(-10:10), <->] ({4.71},{x});

           

  \end{axis}
\end{tikzpicture}
\end{image}









Sine and cosine come from measurements of the unit circle.  What about tangent?










\begin{image}
\begin{tikzpicture}
  \begin{axis}[
            xmin=-1.1,xmax=1.1,ymin=-1.1,ymax=1.1,
            axis lines=center,
            width=4in,
            xtick={-1},
            ytick={-1,1},
            clip=false,
            unit vector ratio*=1 1 1,
            xlabel=$ $, ylabel=$y$,
            ticklabel style={font=\scriptsize},
            every axis y label/.style={at=(current axis.above origin),anchor=south},
            every axis x label/.style={at=(current axis.right of origin),anchor=west},
          ]        
          


          \draw [ultra thick] (axis cs:0,0) -- (axis cs:1.305,0);
          \draw [ultra thick] (axis cs:0.766,0.643) -- (axis cs:1.305,0);



          \draw [thin] (axis cs:0.716,0.05) -- (axis cs:0.766,0.05);
          \draw [thin] (axis cs:0.716,0) -- (axis cs:0.716,0.05);

          \draw [thin] (axis cs:0.766,0.05) -- (axis cs:0.812,0.05);
          \draw [thin] (axis cs:0.812,0) -- (axis cs:0.812,0.05);

          \draw [thin] (axis cs:0.7,0.587) -- (axis cs:0.77,0.51);
          \draw [thin] (axis cs:0.77,0.52) -- (axis cs:0.84,0.57);





          \addplot [smooth, domain=(0:360)] ({cos(x)},{sin(x)}); %% unit circle

          \addplot [textColor] plot coordinates {(0,0) (.766,.643)}; %% 40 degrees

          \addplot [ultra thick,penColor] plot coordinates {(.766,0) (.766,.643)}; %% 40 degrees
          \addplot [ultra thick,penColor2] plot coordinates {(0,0) (.766,0)}; %% 40 degrees
          
          %\addplot [ultra thick,penColor3] plot coordinates {(1,0) (1,.839)}; %% 40 degrees          

          \addplot [textColor,smooth, domain=(0:40)] ({.15*cos(x)},{.15*sin(x)});
          %\addplot [very thick,penColor] plot coordinates {(0,0) (.766,.643)}; %% sector
          %\addplot [very thick,penColor] plot coordinates {(0,0) (1,0)}; %% sector
          %\addplot [very thick, penColor, smooth, domain=(0:40)] ({cos(x)},{sin(x)}); %% sector
          \node at (axis cs:.15,.07) [anchor=west] {$\theta$};
          \node[penColor] at (axis cs:0.85,.27) {$b$};
          \node[penColor2] at (axis cs:.383,0) [anchor=north] {$a$};
          %\node[penColor3, rotate=-90] at (axis cs:1.06,.322) {$\tan(\theta)$};
           \node[penColor] at (axis cs:0.37,0.4) {$1$};


          \node at (axis cs:0.815, 0.5) [anchor=north] {$\theta$};
          \addplot [textColor,smooth, domain=(270:310)] ({0.15*cos(x)+0.766},{0.15*sin(x)+0.643});
          \node[textColor] at (axis cs:1.1,0)[anchor=north] {$c$};
          \node[textColor] at (axis cs:1.05,0.5)[anchor=north] {$h$};








        \end{axis}
\end{tikzpicture}
\end{image}



In the diagram above, we know that $a = cos(\theta)$ and $b = sin(\theta)$.

We also know the unit circle right triangles is simialr to the big right triangle, since they both include the angle $\theta$.

Since, they are similar, we know that 


\[    \frac{opp}{adj} = \frac{b}{a}  = \frac{\sin(\theta)}{\cos(\theta)}   = \frac{h}{1}          \]


\[         h = \tan(\theta)\]


















\end{document}

