\documentclass{ximera}


\graphicspath{
  {./}
  {ximeraTutorial/}
  {basicPhilosophy/}
}

\newcommand{\mooculus}{\textsf{\textbf{MOOC}\textnormal{\textsf{ULUS}}}}

\usepackage{tkz-euclide}\usepackage{tikz}
\usepackage{tikz-cd}
\usetikzlibrary{arrows}
\tikzset{>=stealth,commutative diagrams/.cd,
  arrow style=tikz,diagrams={>=stealth}} %% cool arrow head
\tikzset{shorten <>/.style={ shorten >=#1, shorten <=#1 } } %% allows shorter vectors

\usetikzlibrary{backgrounds} %% for boxes around graphs
\usetikzlibrary{shapes,positioning}  %% Clouds and stars
\usetikzlibrary{matrix} %% for matrix
\usepgfplotslibrary{polar} %% for polar plots
\usepgfplotslibrary{fillbetween} %% to shade area between curves in TikZ
\usetkzobj{all}
\usepackage[makeroom]{cancel} %% for strike outs
%\usepackage{mathtools} %% for pretty underbrace % Breaks Ximera
%\usepackage{multicol}
\usepackage{pgffor} %% required for integral for loops



%% http://tex.stackexchange.com/questions/66490/drawing-a-tikz-arc-specifying-the-center
%% Draws beach ball
\tikzset{pics/carc/.style args={#1:#2:#3}{code={\draw[pic actions] (#1:#3) arc(#1:#2:#3);}}}



\usepackage{array}
\setlength{\extrarowheight}{+.1cm}
\newdimen\digitwidth
\settowidth\digitwidth{9}
\def\divrule#1#2{
\noalign{\moveright#1\digitwidth
\vbox{\hrule width#2\digitwidth}}}






\DeclareMathOperator{\arccot}{arccot}
\DeclareMathOperator{\arcsec}{arcsec}
\DeclareMathOperator{\arccsc}{arccsc}

















%%This is to help with formatting on future title pages.
\newenvironment{sectionOutcomes}{}{}


\title{Theory}

\begin{document}

\begin{abstract}
the rational story
\end{abstract}
\maketitle









Rational functions are quotients or fractions of polynomials.  (Polynomials are rational functions, since they can be written in a fraction form with $1$ as the denominator.)  So, it is not surprising that analyzing rational function follows the same plan as for polynomials.



















\begin{definition} Rational Function


A \textbf{rational function} is a function that can be represented by a formula of the form


\[   R(x) = \frac{a_n x^n + a_{n-1} x^{n-1} + \cdots + a_2 x^2 + a_1 x + a_0}{b_m x^m + b_{m-1} x^{m-1} + \cdots + b_2 x^2 + b_1 x + b_0}        \]

where the $a_i$ and $b_i$ are real numbers with $a_n \ne 0$ and $b_m \ne 0$.


\end{definition}



$\blacktriangleright$ \textbf{Domain:} The implied domain of a rational function is all real numbers, except those that make the denominator equal to $0$.  However, a particular rational function may be defined with any subset of the real numbers. IN this case, we might say that it is a restricted rational function.








Following our analysis of polynomials, we would like to factor and collect common factors.


\[  R(x)   \frac{a (x - r_k)^{e_k} (x - r_{k-1})^{e_{k-1}} \cdots (x - r_1)^{e_1} }{b (x - s_h)^{f_h} (x - s_{h-1})^{f_{h-1}} \cdots (x - s_1)^{f_1}}            \]


Here, the roots are distinct:  $r_i \ne r_j$ when $i \ne j$ and $s_i \ne s_j$ when $i \ne j$.



\begin{warning}  

The formula for a rational function could have a root in common between the polynomials in the numerator and denominator.

In this case, we would like to factor out the common factors and simplify the formula.

If this simplification results in the removal of a factor from the denominator, we must remember that the corresponding root is still excluded from the domain as the original formula excluded it.



\end{warning}





\begin{example}

Let $R$ be defined as follows.


\[   R(x) = \frac{(x-3)(x-1)}{x-1}            \]


We would simplify this to $R(x) = x-3$


But the domain is still $(-\infty, 1) \cup (1,\infty)$.

The graph is a line with a hole in it at $(1, -2)$.












Graph of $y = R(x)$.


\begin{image}
\begin{tikzpicture}
  \begin{axis}[
            domain=-10:10, ymax=10, xmax=10, ymin=-10, xmin=-10,
            axis lines =center, xlabel=$x$, ylabel=$y$, grid = major, grid style={dashed},
            ytick={-10,-8,-6,-4,-2,2,4,6,8,10},
            xtick={-10,-8,-6,-4,-2,2,4,6,8,10},
            yticklabels={$-10$,$-8$,$-6$,$-4$,$-2$,$2$,$4$,$6$,$8$,$10$}, 
            xticklabels={$-10$,$-8$,$-6$,$-4$,$-2$,$2$,$4$,$6$,$8$,$10$},
            ticklabel style={font=\scriptsize},
            every axis y label/.style={at=(current axis.above origin),anchor=south},
            every axis x label/.style={at=(current axis.right of origin),anchor=west},
            axis on top
          ]
          

            \addplot [line width=2, penColor, smooth,samples=100,domain=(-7:10),<->] {x-3};

          	\addplot[color=penColor,fill=white,only marks,mark=*] coordinates{(1,-2)};


           

  \end{axis}
\end{tikzpicture}
\end{image}



$1$ does not cause a problem for $x-3$, but that isn't the formula for $R$. $\frac{(3-x)(x-1)}{(x-1)}$ is equal to $x-3$, provided $x \ne 1$.  We cannot reinsert $1$ after simplifying.




\end{example}

We prefer working with a simplified formula.  We just need to keep track of how simplifying affects factors.


\textbf{\textcolor{red!25!blue!75!}{Assuming a simplified formula...}} \\




\textbf{Note:} A simplified formula assures us that the zeros of the numerator polynomial are all different from the zeros of the denominator polynomial. \\





$\blacktriangleright$ \textbf{\textcolor{red!10!blue!90!}{Roots and Zeros:}} \\
   Zeros of a rational function are the zeros of the numerator polynomial.  A polynomial function behaves in one of two ways around a root.

\begin{itemize}
\item If the multipicity is odd then the function changes sign over the root.  The graph crosses over the horizontal axis at corresponding intercepts.
\item If the multipicity is even then the function does not change sign over the root.  The graph does not cross over the horizontal axis at corresponding intercepts. Instead, it bounces back in the direction it came.
\end{itemize}






$\blacktriangleright$ \textbf{\textcolor{red!10!blue!90!}{Singularities:}} \\
Singularities of a rational function are the zeros of the denominator polynomial.  Vertical asymptotes represent singularities graphically.

\begin{itemize}
\item If the multipicity is odd then the function changes sign over the singularity.  The graph crosses over the corresponding vertical asymptote.
\item If the multipicity is even then the function does not change sign over the singularity.  The graph does not cross over the h corresponding vertical asymptote. 
\end{itemize}


\textbf{Note:} If the original formula also had a denominator factor for a root in the simplified form, then there will be a hole in the graph rather than an intercept. 



\begin{example}  Missing Intercept

\[     R(x) = \frac{(x+4)(x-5)^2}{(x-5)(x-7)}     \]

reduces to 

\[     R(x) = \frac{(x+4)(x-5)}{(x-7)}     \]


It appears that $R(x)$ has $5$ as a root and $(5,0)$ as an intercept for the graph.  However, the original formula for $R$ removed $5$ from the domain.  Therefore, the graph has a missing intercept. $(5,0)$ is plotted as a hollow dot.






\end{example}

We must always remember the original formula when it comes to the domain.  The simplification cannot change the domain. \\





$\blacktriangleright$ \textbf{\textcolor{red!10!blue!90!}{Continuity:}} \\
Rational functions are relatively nice functions.  They are continuous everywhere in the domain.  They have no discontinuities.  The singularities correspond to vertical asymptotes or removeable singularities, both of which we understand.



$\blacktriangleright$ \textbf{\textcolor{red!10!blue!90!}{Graph:}} \\
Graphs of polynomials are nice.  They are smooth.  They do not have corners or spikes or jump breaks. They do have vertical asymptotes, which we understand.


Once we have the roots of the numerator and denominmator polynomials, then we can plot the intercepts and asymptotes.  Then we can smoothly connect everything according to their mulitplicities and have a pretty good sketch of the shape of the graph.

With a basic general shape, we can estimate critical numbers and extrema values. \\




$\blacktriangleright$ \textbf{\textcolor{red!10!blue!90!}{Extrema:}} \\
If we have a derivative, then we can attempt to locate exact values of critical numbers.  Without the derivative, we turn to technology for some assistance.




$\blacktriangleright$ \textbf{\textcolor{red!10!blue!90!}{Rate of Change:}} \\
The critical numbers and singularities partition the real line into intervals where the polynomial function increases or decreases.  We can then list these intervals.















\begin{example} Rational Function


Completely analyze $G(t) = \frac{(t+7)(t+5)(t-2)}{(t+4)^2(t-2)}$


The domain is $\left( -\infty, \answer{-4} \right) \cup \left(\answer{-4}, \answer{2} \right) \cup \left( \answer{2}, \infty \right)$.

Let's simplify: $G(t) = \frac{(t+7)(t+5)}{(t+4)^2}$



The $t-2$ factor is gone, but $2$ is still not in the domain.  This will visually show up as a hole in the graph.



\begin{itemize}
\item $-7$ is a root of multiplicity $\answer{1}$.  Since this multiplicity is odd, $G$ will change sign through $\answer{-7}$ and the graph will cross at $(-7,0)$.
\item $-5$ is a root of multiplicity $\answer{1}$.  Since this multiplicity is odd, $G$ will change sign through $-5$ and the graph will cross at $(-5,0)$.
\item $-4$ is a singularity of multiplicity $\answer{2}$.  The function is unbounded near $-4$.  This will show up as a vertical asymptote on the graph. Since this multiplicity is even, $G$ will not change sign across $-4$.  The graph will approach the vertical asymptote similarly on both isdes.
\item Our simplified version does not have the $t-2$ factor.  Therefore, the graph will not have an intercept nor a vertical asymptote.  It will have a hole at $\left( 2, \answer{\frac{63}{36}} \right) = (2, 1.75)$.
\end{itemize}


The end-behavior of $G$ is $\frac{t^2}{t^2}$.  Therefore, $\lim\limits_{t \to -\infty}G(t) = 1$ and $\lim\limits_{t \to \infty}G(t) = 1$.  The graph has a horizontal asymptote.




Thinking left to right on the number line, $G$ starts off near $1$, which is positive.  It changes signs across $-7$ and becomes negative. It changes signs across $-5$ and becomes positive.  It cannot change sign until $-4$.  Therefore, $G$ becomes positively unbounded on the left side of $-4$.  Across $-4$, $G$ does not change sign.  Therefore, it is again unbounded and positive on the right side of the vertical asymptote.  There are no more zeros or singularities.  So, $G$ cannot chnage sign again.  Now, it appraoches $1$ from above.




\begin{image}
\begin{tikzpicture}
  \begin{axis}[
            domain=-10:10, ymax=10, xmax=10, ymin=-10, xmin=-10,
            axis lines =center, xlabel=$t$, ylabel={$y=G(t)$}, grid = major, grid style={dashed},
            ytick={-10,-8,-6,-4,-2,2,4,6,8,10},
            xtick={-10,-8,-6,-4,-2,2,4,6,8,10},
            yticklabels={$-10$,$-8$,$-6$,$-4$,$-2$,$2$,$4$,$6$,$8$,$10$}, 
            xticklabels={$-10$,$-8$,$-6$,$-4$,$-2$,$2$,$4$,$6$,$8$,$10$},
            ticklabel style={font=\scriptsize},
            every axis y label/.style={at=(current axis.above origin),anchor=south},
            every axis x label/.style={at=(current axis.right of origin),anchor=west},
            axis on top
          ]
          
          	\addplot [line width=1, gray, dashed,samples=200,domain=(-9:9),<->] {1};
          	\addplot [line width=1, gray, dashed,samples=200,domain=(-9:9),<->] ({-4},{x});

            \addplot [line width=2, penColor, smooth,samples=100,domain=(-8:-6)] {-(x+7)};
            \addplot [line width=2, penColor, smooth,samples=100,domain=(-5.5:-4.5)] {(x+5)};

            \addplot [line width=2, penColor, smooth,samples=100,domain=(-4.5:-4.3)] {-1/(x+4)};
            \addplot [line width=2, penColor, smooth,samples=100,domain=(-3.9:-3.7)] {1/(x+4)};
            \addplot [line width=2, penColor, smooth,samples=100,domain=(-10:-8)] {1.5)};
            \addplot [line width=2, penColor, smooth,samples=100,domain=(4:6)] {1.5)};


          	\addplot[color=penColor,fill=penColor,only marks,mark=*] coordinates{(-7,0)};
          	\addplot[color=penColor,fill=penColor,only marks,mark=*] coordinates{(-5,0)};


           

  \end{axis}
\end{tikzpicture}
\end{image}









The graph is very suggestive that there is a local minimum somewhere around $-5.5$.  






\begin{image}
\begin{tikzpicture}
  \begin{axis}[
            domain=-10:10, ymax=10, xmax=10, ymin=-10, xmin=-10,
            axis lines =center, xlabel=$t$, ylabel={$y=G(t)$}, grid = major, grid style={dashed},
            ytick={-10,-8,-6,-4,-2,2,4,6,8,10},
            xtick={-10,-8,-6,-4,-2,2,4,6,8,10},
            yticklabels={$-10$,$-8$,$-6$,$-4$,$-2$,$2$,$4$,$6$,$8$,$10$}, 
            xticklabels={$-10$,$-8$,$-6$,$-4$,$-2$,$2$,$4$,$6$,$8$,$10$},
            ticklabel style={font=\scriptsize},
            every axis y label/.style={at=(current axis.above origin),anchor=south},
            every axis x label/.style={at=(current axis.right of origin),anchor=west},
            axis on top
          ]
          
            \addplot [line width=1, gray, dashed,samples=200,domain=(-9:9),<->] {1};
            \addplot [line width=1, gray, dashed,samples=200,domain=(-9:9),<->] ({-4},{x});


            \addplot [line width=2, penColor, smooth,samples=300,domain=(-9:-4.4),<->] {((x+7)*(x+5))/(x+4)^2};
            \addplot [line width=2, penColor, smooth,samples=300,domain=(-3.05:9),<->] {((x+7)*(x+5))/(x+4)^2};


            \addplot[color=penColor,fill=penColor,only marks,mark=*] coordinates{(-7,0)};
            \addplot[color=penColor,fill=penColor,only marks,mark=*] coordinates{(-5,0)};


           

  \end{axis}
\end{tikzpicture}
\end{image}












With some technology, we can approximate the other critical number to be $-5.5$ and the local minimum to be $-0.333$.




\begin{center}
\desmos{st4dfylktg}{400}{300}
\end{center}







\begin{itemize}
\item $G$ decreases on $(-\infty, -5.5]$.
\item $G$ increases on $[-5.5, -4)$.
\item $G$ decreases on $(-4, \infty)$.
\end{itemize}



The local minimum at $-5.5$ is also a global minimum.  There is no global maximum.  There are no local maximums.



\end{example}
























\end{document}
