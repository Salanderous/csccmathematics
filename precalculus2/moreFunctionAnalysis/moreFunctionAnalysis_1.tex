\documentclass{ximera}


\graphicspath{
  {./}
  {ximeraTutorial/}
  {basicPhilosophy/}
}

\newcommand{\mooculus}{\textsf{\textbf{MOOC}\textnormal{\textsf{ULUS}}}}

\usepackage{tkz-euclide}\usepackage{tikz}
\usepackage{tikz-cd}
\usetikzlibrary{arrows}
\tikzset{>=stealth,commutative diagrams/.cd,
  arrow style=tikz,diagrams={>=stealth}} %% cool arrow head
\tikzset{shorten <>/.style={ shorten >=#1, shorten <=#1 } } %% allows shorter vectors

\usetikzlibrary{backgrounds} %% for boxes around graphs
\usetikzlibrary{shapes,positioning}  %% Clouds and stars
\usetikzlibrary{matrix} %% for matrix
\usepgfplotslibrary{polar} %% for polar plots
\usepgfplotslibrary{fillbetween} %% to shade area between curves in TikZ
\usetkzobj{all}
\usepackage[makeroom]{cancel} %% for strike outs
%\usepackage{mathtools} %% for pretty underbrace % Breaks Ximera
%\usepackage{multicol}
\usepackage{pgffor} %% required for integral for loops



%% http://tex.stackexchange.com/questions/66490/drawing-a-tikz-arc-specifying-the-center
%% Draws beach ball
\tikzset{pics/carc/.style args={#1:#2:#3}{code={\draw[pic actions] (#1:#3) arc(#1:#2:#3);}}}



\usepackage{array}
\setlength{\extrarowheight}{+.1cm}
\newdimen\digitwidth
\settowidth\digitwidth{9}
\def\divrule#1#2{
\noalign{\moveright#1\digitwidth
\vbox{\hrule width#2\digitwidth}}}






\DeclareMathOperator{\arccot}{arccot}
\DeclareMathOperator{\arcsec}{arcsec}
\DeclareMathOperator{\arccsc}{arccsc}

















%%This is to help with formatting on future title pages.
\newenvironment{sectionOutcomes}{}{}


\title{Analyzing}

\begin{document}

\begin{abstract}
describe everything
\end{abstract}
\maketitle







Completely analyze $A(\theta) = \sin(\theta) - \sin(2\theta)$

$\blacktriangleright$  The implied domain is all real numbers.

$\blacktriangleright$  There are no discontinuities or singularities.

$\blacktriangleright$ The function is continuous on the whole real line.


$\blacktriangleright$ The function is periodic with a period of $2\pi$.  Therefore, we can just examine one wave.  Let's examine $[0,2\pi)$.






$\blacktriangleright$ To locate zeros, we will need to factor the formula.  




\begin{align*}
A(\theta)   &  = \sin(\theta) - \sin(2\theta)  \\
A(\theta)   &  = \sin(\theta) - 2\sin(\theta)\cos(\theta)   \\
A(\theta)   &  = \sin(\theta) (1 - 2\cos(\theta))   \\
\end{align*}



Either $sin(\theta) = 0$, which happens when $\theta = k\pi$ with $k \in \mathbb{Z}$.
Or, $1 - 2\cos(\theta) = 0$, which happens when $\theta = \frac{\pi}{3} \pm 2k\pi$ or when $\theta = \frac{5\pi}{3} \pm 2k\pi$ with $k \in \mathbb{Z}$.

In our sample period, we have zeros at $0$ and $\pi$. These will get solid dots on the  graph.  There is a zero at $2\pi$, but this is not in our sample period.  However, we should include a solid dot, because that makes the description clearer.  We also have zeros at $\frac{\pi}{3}$and $\frac{5\pi}{3}$.  These will get solid dots on the graph.








\begin{image}
\begin{tikzpicture}
  \begin{axis}[
            domain=-1:8, ymax=10, xmax=8, ymin=-10, xmin=-1,
            axis lines =center, xlabel={$\theta$}, ylabel={$y = A(\theta)$}, grid = major, grid style={dashed},
            ytick={-10,-8,-6,-4,-2,2,4,6,8,10},
            xtick={-7.85, -6.28, -4.71, -3.14, -1.57, 0, 1.57, 3.142, 4.71, 6.28, 7.85},
            xticklabels={$\tfrac{-5\pi}{2}$,$-2\pi$,$\tfrac{-3\pi}{2}$,$-\pi$, $\tfrac{-\pi}{2}$, $0$, $\tfrac{\pi}{2}$, $\pi$, $\tfrac{3\pi}{2}$, $2\pi$, $\tfrac{5\pi}{2}$},
            yticklabels={$-10$,$-8$,$-6$,$-4$,$-2$,$2$,$4$,$6$,$8$,$10$}, 
            ticklabel style={font=\scriptsize},
            every axis y label/.style={at=(current axis.above origin),anchor=south},
            every axis x label/.style={at=(current axis.right of origin),anchor=west},
            axis on top
          ]
          
            \addplot[color=penColor,fill=penColor,only marks,mark=*] coordinates{(0,0)};
            \addplot[color=penColor,fill=penColor,only marks,mark=*] coordinates{(1.047,0)};
            \addplot[color=penColor,fill=penColor,only marks,mark=*] coordinates{(3.141,0)};
            \addplot[color=penColor,fill=penColor,only marks,mark=*] coordinates{(5.236,0)};
            \addplot[color=penColor,fill=penColor,only marks,mark=*] coordinates{(6.282,0)};


  \end{axis}
\end{tikzpicture}
\end{image}




The function is continuous. Therefore, the graph either crosses the axis at the intercepts or the graph bounces back and maintains its sign.


For example: At $\pi$, $\sin(\theta)$ changes sign. Other other hand $1 -2\cos(2\theta)$ stays positive on either side of $\pi$.  Therefore, $A(\theta)$ changes sign and the graph crosses the axis at $(\pi, 0)$.

A similar analysis will show that the graph crosses at all of the intercepts.  We just need a starting direction.

When $\theta$ is a little bit greater than $0$, $\sin(\theta)$ is positive and $1-2\cos(\theta)$ is negative, making $G(t)$ negative.







$\blacktriangleright$  The range is difficult to determine from here.  It depends on how high and low the maximum and minimum are.  But we can say that $\sin(\theta)$ can't get greater than $1$.  



Our graph of $y=A(\theta))$ is piecing together. $1 -2\cos(2\theta)$ can't get greater than $3$.  So, the range must be inside $[-3, 3]$.









\begin{image}
\begin{tikzpicture}
  \begin{axis}[
            domain=-1:8, ymax=3, xmax=8, ymin=-3, xmin=-1,
            axis lines =center, xlabel={$\theta$}, ylabel={$y = A(\theta)$}, grid = major, grid style={dashed},
            ytick={-3,-2,-1,1,2,3},
            xtick={1.57, 3.142, 4.71, 6.28, 7.85},
            xticklabels={$\tfrac{\pi}{2}$, $\pi$, $\tfrac{3\pi}{2}$, $2\pi$, $\tfrac{5\pi}{2}$},
            yticklabels={$-3$,$-2$,$-1$,$1$,$2$,$3$}, 
            ticklabel style={font=\scriptsize},
            every axis y label/.style={at=(current axis.above origin),anchor=south},
            every axis x label/.style={at=(current axis.right of origin),anchor=west},
            axis on top
          ]
          



            \addplot[color=penColor,fill=penColor,only marks,mark=*] coordinates{(0,0)};
            \addplot[color=penColor,fill=penColor,only marks,mark=*] coordinates{(1.047,0)};
             \addplot[color=penColor,fill=penColor,only marks,mark=*] coordinates{(3.14,0)};
            \addplot[color=penColor,fill=penColor,only marks,mark=*] coordinates{(5.236,0)};
            \addplot[color=penColor,fill=penColor,only marks,mark=*] coordinates{(6.28,0)};


            \addplot [line width=2, penColor, smooth,samples=300,domain=(0:6.28)] {sin(deg(x))-sin(deg(2*x))};





  \end{axis}
\end{tikzpicture}
\end{image}






























The graph tells us a lot.







$\blacktriangleright$ maximums and minimums

\begin{itemize}
\item $A$ has a critical number between $\frac{2\pi}{3}$ and $\pi$, which corresponds to a global (and local) maximum.
\item $A$ has a critical number between $\pi$ and $\frac{5\pi}{3}$, which corresponds to a global (and local) minimum.
\item $A$ has a critical number between $\frac{5\pi}{3}$ and $2\pi$, which corresponds to a local maximum.
\item $A$ has a critical number between $0$ and $\frac{2\pi}{3}$, which corresponds to a local minimum.
\end{itemize}


Let's call these critical numbers $c_1$, $c_2$, $c_3$, and $c_4$.













\begin{image}
\begin{tikzpicture}
  \begin{axis}[
            domain=-1:8, ymax=3, xmax=8, ymin=-3, xmin=-1,
            axis lines =center, xlabel={$\theta$}, ylabel={$y = A(\theta)$}, grid = major, grid style={dashed},
            ytick={-3,-2,-1,1,2,3},
            xtick={0.568, 2.206, 4.078, 5.715},
            xticklabels={$c_1$, $c_2$, $c_3$, $c_4$},
            yticklabels={$-3$,$-2$,$-1$,$1$,$2$,$3$}, 
            ticklabel style={font=\scriptsize},
            every axis y label/.style={at=(current axis.above origin),anchor=south},
            every axis x label/.style={at=(current axis.right of origin),anchor=west},
            axis on top
          ]
          



            \addplot[color=penColor,fill=penColor,only marks,mark=*] coordinates{(0,0)};
            \addplot[color=penColor,fill=penColor,only marks,mark=*] coordinates{(1.047,0)};
             \addplot[color=penColor,fill=penColor,only marks,mark=*] coordinates{(3.14,0)};
            \addplot[color=penColor,fill=penColor,only marks,mark=*] coordinates{(5.236,0)};
            \addplot[color=penColor,fill=penColor,only marks,mark=*] coordinates{(6.28,0)};


            \addplot [line width=2, penColor, smooth,samples=300,domain=(0:6.28)] {sin(deg(x))-sin(deg(2*x))};





  \end{axis}
\end{tikzpicture}
\end{image}
















$\blacktriangleright$ Rate-of-Change

\begin{itemize}
\item $A$ is decreasing on $[0, c_1]$.
\item $A$ is increasing on $[c_1, c_2]$.
\item $A$ is decreasing on $[c_2, c_3]$.
\item $A$ is increasing on $[c_3, c_4]$.
\item $A$ is decreasing on $[c_4, 2\pi]$.
\end{itemize}







\section{with Calculus}

Calculus will give us the tools to discover a formula for the derivative of $G$.


\[   A'(\theta) = \cos(\theta)-2 \cos(2\theta)    \]


The critical numbers would be where $A'(\theta) = 0$.



\begin{align*}
0    & = \cos(\theta)-2 \cos(2\theta)   \\
     & = \cos(\theta) - 2 (2 \cos^2(\theta) - 1)    \\
     & = -4 \cos^2(\theta) + \cos(\theta) + 2   \\
     & = 4 \cos^2(\theta) - \cos(\theta) - 2   \\
\end{align*}


We have a quadratic equation.  It doesn't doesn't factor nicely, so we'll go with the quadratic formula.



\[   \cos(\theta) = \frac{1 \pm \sqrt{1 - 4 (4)(-2)}}{2 \cdot 4}  = \frac{1 \pm \sqrt{33}}{8}        \]




Let's take each of these separately.



$\blacktriangleright$  $\cos(\theta) = \frac{1 + \sqrt{33}}{8}   \approx  0.843$


We are looking for an angle whose cosine is $\frac{1 + \sqrt{33}}{8}$.  This is positive, therefore, there are two such angles. One in the first quadrant.  That would be $\arccos\left(\frac{1 + \sqrt{33}}{8}\right)$




\begin{image}
\begin{tikzpicture}
  \begin{axis}[
            xmin=-1.1,xmax=1.1,ymin=-1.1,ymax=1.1,
            axis lines=center,
            width=4in,
            xtick={-1,1},
            ytick={-1,1},
            clip=false,
            unit vector ratio*=1 1 1,
            xlabel=$x$, ylabel=$y$,
            every axis y label/.style={at=(current axis.above origin),anchor=south},
            every axis x label/.style={at=(current axis.right of origin),anchor=west},
          ]        
          \addplot [dashed, smooth, domain=(0:360)] ({cos(x)},{sin(x)}); %% unit circle

          \addplot [textColor] plot coordinates {(0,0) (0.843,0.538)}; %% 40 degrees

          %\addplot [ultra thick,penColor] plot coordinates {(.766,0) (.766,.643)}; %% 40 degrees
          %\addplot [ultra thick,penColor2] plot coordinates {(0,0) (.766,0)}; %% 40 degrees
          
          %\addplot [ultra thick,penColor3] plot coordinates {(1,0) (1,.839)}; %% 40 degrees          

          \addplot [textColor,smooth, domain=(0:32.53)] ({.15*cos(x)},{.15*sin(x)});
          %\addplot [very thick,penColor] plot coordinates {(0,0) (.766,.643)}; %% sector
          %\addplot [very thick,penColor] plot coordinates {(0,0) (1,0)}; %% sector
          %\addplot [very thick, penColor, smooth, domain=(0:40)] ({cos(x)},{sin(x)}); %% sector
          \node at (axis cs:.15,0.07) [anchor=west] {$c_1$};
          %\node[penColor, rotate=-90] at (axis cs:.84,.322) {$\sin(\theta)$};
          %\node[penColor2] at (axis cs:.383,0) [anchor=north] {$\cos(\theta)$};
          %\node[penColor3, rotate=-90] at (axis cs:1.06,.322) {$\tan(\theta)$};
        \end{axis}
\end{tikzpicture}
\end{image}




There is a second angle in the fourth quadrant where cosine is $\frac{1 + \sqrt{33}}{8}$.  This angle could be described as $-\arccos\left(\frac{1 + \sqrt{33}}{8}\right)$.  However, our chosen interval to examine is $[0, 2\pi)$. So, the angle in the fourth quadrant is $2\pi - \arccos\left(\frac{1 + \sqrt{33}}{8}\right)$







\begin{image}
\begin{tikzpicture}
  \begin{axis}[
            xmin=-1.1,xmax=1.1,ymin=-1.1,ymax=1.1,
            axis lines=center,
            width=4in,
            xtick={-1,1},
            ytick={-1,1},
            clip=false,
            unit vector ratio*=1 1 1,
            xlabel=$x$, ylabel=$y$,
            every axis y label/.style={at=(current axis.above origin),anchor=south},
            every axis x label/.style={at=(current axis.right of origin),anchor=west},
          ]        
          \addplot [dashed, smooth, domain=(0:360)] ({cos(x)},{sin(x)}); %% unit circle

          \addplot [textColor] plot coordinates {(0,0) (0.843,0.538)}; %%
          \addplot [textColor] plot coordinates {(0,0) (0.843,-0.538)};

          %\addplot [ultra thick,penColor] plot coordinates {(.766,0) (.766,.643)}; %% 40 degrees
          %\addplot [ultra thick,penColor2] plot coordinates {(0,0) (.766,0)}; %% 40 degrees
          
          %\addplot [ultra thick,penColor3] plot coordinates {(1,0) (1,.839)}; %% 40 degrees          

          \addplot [textColor,smooth, domain=(0:32.53)] ({.15*cos(x)},{.15*sin(x)});
          \addplot [textColor,smooth, domain=(0:327.47)] ({.18*cos(x)},{.18*sin(x)});
          %\addplot [very thick,penColor] plot coordinates {(0,0) (.766,.643)}; %% sector
          %\addplot [very thick,penColor] plot coordinates {(0,0) (1,0)}; %% sector
          %\addplot [very thick, penColor, smooth, domain=(0:40)] ({cos(x)},{sin(x)}); %% sector
          \node at (axis cs:0.15,0.07) [anchor=west] {$c_1$};
          \node at (axis cs:-0.07,-0.20) [anchor=east] {$c_4$};
          %\node[penColor, rotate=-90] at (axis cs:.84,.322) {$\sin(\theta)$};
          %\node[penColor2] at (axis cs:.383,0) [anchor=north] {$\cos(\theta)$};
          %\node[penColor3, rotate=-90] at (axis cs:1.06,.322) {$\tan(\theta)$};
        \end{axis}
\end{tikzpicture}
\end{image}












$\blacktriangleright$  $\cos(\theta) = \frac{1 - \sqrt{33}}{8}   \approx  -0.593$


We are looking for an angle whose cosine is $\frac{1 - \sqrt{33}}{8}$.  This is negative, therefore, there are two such angles. One in the second quadrant.  That would be $\arccos\left(\frac{1 - \sqrt{33}}{8}\right)$




\begin{image}
\begin{tikzpicture}
  \begin{axis}[
            xmin=-1.1,xmax=1.1,ymin=-1.1,ymax=1.1,
            axis lines=center,
            width=4in,
            xtick={-1,1},
            ytick={-1,1},
            clip=false,
            unit vector ratio*=1 1 1,
            xlabel=$x$, ylabel=$y$,
            every axis y label/.style={at=(current axis.above origin),anchor=south},
            every axis x label/.style={at=(current axis.right of origin),anchor=west},
          ]        
          \addplot [dashed, smooth, domain=(0:360)] ({cos(x)},{sin(x)}); %% unit circle

          \addplot [textColor] plot coordinates {(0,0) (-0.593,0.8052)}; %% 40 degrees

          %\addplot [ultra thick,penColor] plot coordinates {(.766,0) (.766,.643)}; %% 40 degrees
          %\addplot [ultra thick,penColor2] plot coordinates {(0,0) (.766,0)}; %% 40 degrees
          
          %\addplot [ultra thick,penColor3] plot coordinates {(1,0) (1,.839)}; %% 40 degrees          

          \addplot [textColor,smooth, domain=(0:126.37)] ({.18*cos(x)},{.18*sin(x)});
          %\addplot [very thick,penColor] plot coordinates {(0,0) (.766,.643)}; %% sector
          %\addplot [very thick,penColor] plot coordinates {(0,0) (1,0)}; %% sector
          %\addplot [very thick, penColor, smooth, domain=(0:40)] ({cos(x)},{sin(x)}); %% sector
          \node at (axis cs:0.14,0.2) [anchor=south] {$c_2$};
          %\node[penColor, rotate=-90] at (axis cs:.84,.322) {$\sin(\theta)$};
          %\node[penColor2] at (axis cs:.383,0) [anchor=north] {$\cos(\theta)$};
          %\node[penColor3, rotate=-90] at (axis cs:1.06,.322) {$\tan(\theta)$};
        \end{axis}
\end{tikzpicture}
\end{image}




There is a fourth angle in the third quadrant where cosine is $\frac{1 - \sqrt{33}}{8}$.  This third quadrant angle has the same reference angle as $c_2$. Therefore, $c_3 = 2\pi - \arccos\left(\frac{1 - \sqrt{33}}{8}\right)$.






\begin{image}
\begin{tikzpicture}
  \begin{axis}[
            xmin=-1.1,xmax=1.1,ymin=-1.1,ymax=1.1,
            axis lines=center,
            width=4in,
            xtick={-1,1},
            ytick={-1,1},
            clip=false,
            unit vector ratio*=1 1 1,
            xlabel=$x$, ylabel=$y$,
            every axis y label/.style={at=(current axis.above origin),anchor=south},
            every axis x label/.style={at=(current axis.right of origin),anchor=west},
          ]        
          \addplot [dashed, smooth, domain=(0:360)] ({cos(x)},{sin(x)}); %% unit circle

          \addplot [textColor] plot coordinates {(0,0) (-0.593,0.8052)}; %% 40 degrees
          \addplot [textColor] plot coordinates {(0,0) (-0.593,-0.8052)};

          %\addplot [ultra thick,penColor] plot coordinates {(.766,0) (.766,.643)}; %% 40 degrees
          %\addplot [ultra thick,penColor2] plot coordinates {(0,0) (.766,0)}; %% 40 degrees
          
          %\addplot [ultra thick,penColor3] plot coordinates {(1,0) (1,.839)}; %% 40 degrees          

          \addplot [textColor,smooth, domain=(0:126.37)] ({.18*cos(x)},{.18*sin(x)});
          \addplot [textColor,smooth, domain=(0:233.63)] ({.14*cos(x)},{.14*sin(x)});
          %\addplot [very thick,penColor] plot coordinates {(0,0) (.766,.643)}; %% sector
          %\addplot [very thick,penColor] plot coordinates {(0,0) (1,0)}; %% sector
          %\addplot [very thick, penColor, smooth, domain=(0:40)] ({cos(x)},{sin(x)}); %% sector
          \node at (axis cs:0.14,0.2) [anchor=south] {$c_2$};
          \node at (axis cs:-0.24,-0.2) [anchor=south] {$c_3$};
          %\node[penColor, rotate=-90] at (axis cs:.84,.322) {$\sin(\theta)$};
          %\node[penColor2] at (axis cs:.383,0) [anchor=north] {$\cos(\theta)$};
          %\node[penColor3, rotate=-90] at (axis cs:1.06,.322) {$\tan(\theta)$};
        \end{axis}
\end{tikzpicture}
\end{image}




We want exact answers, when possible.

In this case, without the derivative and algebra, we could approximate the critical numbers from a graph.



\begin{center}
\desmos{waae40vtar}{400}{300}
\end{center}




\begin{itemize}
  \item $c_1 \approx 0.568$
  \item $c_2 \approx 2.206$
  \item $c_3 \approx 4.078$
  \item $c_4 \approx 5.715$
\end{itemize}







$\blacktriangleright$ \textbf{Extrema}




$\rhd$ We have a global maximum of $A(c_2)$:


\[  A(c_2) = \sin\left(\arccos\left(\frac{1 - \sqrt{33}}{8}\right)\right) - \sin\left(2 \arccos\left(\frac{1 - \sqrt{33}}{8}\right)\right) \approx 1.76   \]




at $c_2 = \arccos\left(\frac{1 - \sqrt{33}}{8}\right) \approx 2.206$. \\









$\rhd$ We have a global minimum of $A(c_3)$:


\[  A(c_3) = \sin\left(   2\pi - \arccos\left(\frac{1 - \sqrt{33}}{8}\right)    \right) - \sin\left(2  \left(     2\pi - \arccos\left(\frac{1 - \sqrt{33}}{8}\right)     \right) \right)  \approx -1.76  \]




at $c_3 = 2\pi - \arccos\left(\frac{1 - \sqrt{33}}{8}\right) \approx 4.078$. \\








$\rhd$ We have a local minimum of $A(c_1)$:


\[  A(c_1) = \sin\left(   \arccos\left(\frac{1 + \sqrt{33}}{8}\right)    \right) - \sin\left(2  \arccos\left(\frac{1 + \sqrt{33}}{8}\right)   \right)  \approx -0.369  \]




at $c_1 = \arccos\left(\frac{1 + \sqrt{33}}{8}\right) \approx 0.568$.   \\












$\rhd$ We have a local maximum of $A(c_4)$:


\[  A(c_4) = \sin\left(   2\pi - \arccos\left(\frac{1 + \sqrt{33}}{8}\right)   \right) - \sin\left(2  (2\pi - \arccos\left(\frac{1 + \sqrt{33}}{8}\right))  \right)  \approx 0.369  \]




at $c_4 = 2\pi - \arccos\left(\frac{1 + \sqrt{33}}{8}\right) \approx 5.715$.   \\









\end{document}
