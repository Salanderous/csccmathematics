\documentclass{ximera}


\graphicspath{
  {./}
  {ximeraTutorial/}
  {basicPhilosophy/}
}

\newcommand{\mooculus}{\textsf{\textbf{MOOC}\textnormal{\textsf{ULUS}}}}

\usepackage{tkz-euclide}\usepackage{tikz}
\usepackage{tikz-cd}
\usetikzlibrary{arrows}
\tikzset{>=stealth,commutative diagrams/.cd,
  arrow style=tikz,diagrams={>=stealth}} %% cool arrow head
\tikzset{shorten <>/.style={ shorten >=#1, shorten <=#1 } } %% allows shorter vectors

\usetikzlibrary{backgrounds} %% for boxes around graphs
\usetikzlibrary{shapes,positioning}  %% Clouds and stars
\usetikzlibrary{matrix} %% for matrix
\usepgfplotslibrary{polar} %% for polar plots
\usepgfplotslibrary{fillbetween} %% to shade area between curves in TikZ
\usetkzobj{all}
\usepackage[makeroom]{cancel} %% for strike outs
%\usepackage{mathtools} %% for pretty underbrace % Breaks Ximera
%\usepackage{multicol}
\usepackage{pgffor} %% required for integral for loops



%% http://tex.stackexchange.com/questions/66490/drawing-a-tikz-arc-specifying-the-center
%% Draws beach ball
\tikzset{pics/carc/.style args={#1:#2:#3}{code={\draw[pic actions] (#1:#3) arc(#1:#2:#3);}}}



\usepackage{array}
\setlength{\extrarowheight}{+.1cm}
\newdimen\digitwidth
\settowidth\digitwidth{9}
\def\divrule#1#2{
\noalign{\moveright#1\digitwidth
\vbox{\hrule width#2\digitwidth}}}






\DeclareMathOperator{\arccot}{arccot}
\DeclareMathOperator{\arcsec}{arcsec}
\DeclareMathOperator{\arccsc}{arccsc}

















%%This is to help with formatting on future title pages.
\newenvironment{sectionOutcomes}{}{}


\title{Around the Circle}

\begin{document}

\begin{abstract}
%Stuff can go here later if we want!
\end{abstract}
\maketitle



As we travel around the Unit Circle, the two coordinates of the points on the circle vary.  The two coordinates are functions of the angle.   The horizontal coordinate makes up the cosine function. The vertical coordinate makes up the sine function.  \\


The values of sine and cosine vary between $-1$ and $1$.  This means that take on every value in the interval $[-1, 1]$.  However, it is rare that the angle and the value of either function are both nice. \\


For us ``nice'' means a fraction involving whole numbers, $\pi$ and square roots. \\


Basically, this happens three times in each quadrant and we over use these occurences extensively to understand what is going on.  We call these the \textbf{easy angles}.  Most of our analysis of trigonometric functions relies on understand what happens at these angles.
















\subsection{Learning Outcomes}

\begin{sectionOutcomes}
In this section, students will 

\begin{itemize}
\item explore the reference angles $\frac{\pi}{6}$, $\frac{\pi}{4}$, and $\frac{\pi}{3}$ in the four quadrants.
\end{itemize}
\end{sectionOutcomes}











\begin{center}
\textbf{\textcolor{green!50!black}{ooooo=-=-=-=-=-=-=-=-=-=-=-=-=ooOoo=-=-=-=-=-=-=-=-=-=-=-=-=ooooo}} \\

more examples can be found by following this link\\ \link[More Examples of Easy Angles]{https://ximera.osu.edu/csccmathematics/precalculus2/precalculus2/easyAngles/examples/exampleList}

\end{center}








\end{document}
