\documentclass{ximera}


\graphicspath{
  {./}
  {ximeraTutorial/}
  {basicPhilosophy/}
}

\newcommand{\mooculus}{\textsf{\textbf{MOOC}\textnormal{\textsf{ULUS}}}}

\usepackage{tkz-euclide}\usepackage{tikz}
\usepackage{tikz-cd}
\usetikzlibrary{arrows}
\tikzset{>=stealth,commutative diagrams/.cd,
  arrow style=tikz,diagrams={>=stealth}} %% cool arrow head
\tikzset{shorten <>/.style={ shorten >=#1, shorten <=#1 } } %% allows shorter vectors

\usetikzlibrary{backgrounds} %% for boxes around graphs
\usetikzlibrary{shapes,positioning}  %% Clouds and stars
\usetikzlibrary{matrix} %% for matrix
\usepgfplotslibrary{polar} %% for polar plots
\usepgfplotslibrary{fillbetween} %% to shade area between curves in TikZ
\usetkzobj{all}
\usepackage[makeroom]{cancel} %% for strike outs
%\usepackage{mathtools} %% for pretty underbrace % Breaks Ximera
%\usepackage{multicol}
\usepackage{pgffor} %% required for integral for loops



%% http://tex.stackexchange.com/questions/66490/drawing-a-tikz-arc-specifying-the-center
%% Draws beach ball
\tikzset{pics/carc/.style args={#1:#2:#3}{code={\draw[pic actions] (#1:#3) arc(#1:#2:#3);}}}



\usepackage{array}
\setlength{\extrarowheight}{+.1cm}
\newdimen\digitwidth
\settowidth\digitwidth{9}
\def\divrule#1#2{
\noalign{\moveright#1\digitwidth
\vbox{\hrule width#2\digitwidth}}}






\DeclareMathOperator{\arccot}{arccot}
\DeclareMathOperator{\arcsec}{arcsec}
\DeclareMathOperator{\arccsc}{arccsc}

















%%This is to help with formatting on future title pages.
\newenvironment{sectionOutcomes}{}{}


\title{Easy Angles}

\begin{document}

\begin{abstract}
$30^{\circ}$, $60^{\circ}$, $90^{\circ}$
\end{abstract}
\maketitle


Sine and cosine are \textbf{transcendental functions}.  They transcend Algebra. They are beyond our algebraic tools.  That makes equations difficult when they involve sine and cosine.


This is true unless you work with angles that just happen to have nice values for sine and cosine.  We have two: $30^{\circ}$ and $60^{\circ}$.




\begin{itemize}
\item $\sin(30^{\circ}) = \frac{1}{2}$
\item $\cos(30^{\circ}) = \frac{\sqrt{3}}{2}$
\end{itemize}


And, since $30^{\circ}$ and $60^{\circ}$ make up a right triangle, we have 


\begin{itemize}
\item $\sin(60^{\circ}) = \frac{\sqrt{3}}{2}$
\item $\cos(60^{\circ}) = \frac{1}{2}$
\end{itemize}



And, then we can walk this around the unit circle.






  \begin{image}
    \begin{tikzpicture}[scale=5.3,cap=round,>=latex]
        % draw the coordinates
        \draw[->] (-1.5cm,0cm) -- (1.5cm,0cm) node[right,fill=white] {$x$};
        \draw[->] (0cm,-1.5cm) -- (0cm,1.5cm) node[above,fill=white] {$y$};

        % draw the unit circle
        \draw[thick] (0cm,0cm) circle(1cm);

        \foreach \x in {0,30,...,360} {
                % lines from center to point
                \draw[gray] (0cm,0cm) -- (\x:1cm);
                % dots at each point
                \filldraw[black] (\x:1cm) circle(0.4pt);
                % draw each angle in degrees
                \draw (\x:0.6cm) node[fill=white] {$\x^\circ$};
        }

        % draw each angle in radians
        \foreach \x/\xtext in {
            30/\frac{\pi}{6},
            45/\frac{\pi}{4},
            60/\frac{\pi}{3},
            90/\frac{\pi}{2},
            120/\frac{2\pi}{3},
            135/\frac{3\pi}{4},
            150/\frac{5\pi}{6},
            180/\pi,
            210/\frac{7\pi}{6},
            225/\frac{5\pi}{4},
            240/\frac{4\pi}{3},
            270/\frac{3\pi}{2},
            300/\frac{5\pi}{3},
            315/\frac{7\pi}{4},
            330/\frac{11\pi}{6},
            360/2\pi}
                \draw (\x:0.85cm) node[fill=white] {$\xtext$};

        \foreach \x/\xtext/\y in {
            % the coordinates for the first quadrant
            30/\frac{\sqrt{3}}{2}/\frac{1}{2},
            45/\frac{\sqrt{2}}{2}/\frac{\sqrt{2}}{2},
            60/\frac{1}{2}/\frac{\sqrt{3}}{2},
            % the coordinates for the second quadrant
            150/-\frac{\sqrt{3}}{2}/\frac{1}{2},
            135/-\frac{\sqrt{2}}{2}/\frac{\sqrt{2}}{2},
            120/-\frac{1}{2}/\frac{\sqrt{3}}{2},
            % the coordinates for the third quadrant
            210/-\frac{\sqrt{3}}{2}/-\frac{1}{2},
            225/-\frac{\sqrt{2}}{2}/-\frac{\sqrt{2}}{2},
            240/-\frac{1}{2}/-\frac{\sqrt{3}}{2},
            % the coordinates for the fourth quadrant
            330/\frac{\sqrt{3}}{2}/-\frac{1}{2},
            315/\frac{\sqrt{2}}{2}/-\frac{\sqrt{2}}{2},
            300/\frac{1}{2}/-\frac{\sqrt{3}}{2}}
                \draw (\x:1.25cm) node[fill=white] {$\left(\xtext,\y\right)$};

        % draw the horizontal and vertical coordinates
        % the placement is better this way
        \draw (-1.25cm,0cm) node[above=1pt] {$(-1,0)$}
              (1.25cm,0cm)  node[above=1pt] {$(1,0)$}
              (0cm,-1.25cm) node[fill=white] {$(0,-1)$}
              (0cm,1.25cm)  node[fill=white] {$(0,1)$};
    \end{tikzpicture}
  \end{image}













\begin{example}



Analyze $f(t) = 3 sin(2t - \pi) - 2$



\begin{explanation}


$\blacktriangleright$ \textbf{One period} \\


Start:  $2t - \pi = 0$, which occurs at $t = \frac{\pi}{2}$where $f = 3 \cdot 0 - 2 = -2$  \\


Finish:  $2t - \pi = 2 \pi$, which occurs at $t = \frac{3 \pi}{2}$, where $f = 3 \cdot 0 - 2 = -2$ \\


Period = $\frac{3 \pi}{2} - \frac{\pi}{2} = \pi$


Halfway is $\pi$.  $sin(2t - \pi)$ equals $0$ at $\pi$, where $f = 3 \cdot 0 - 2 = -2$.


A quarter of the way is $\frac{3 \pi}{4}$, , where $f = 3 \cdot 1 - 2 = 1$


The three-quarter mark is $\frac{5 \pi}{4}$, , where $f = 3 \cdot -1 - 2 = -5$





\begin{image}
\begin{tikzpicture}
  \begin{axis}[
            domain=0:7, ymax=6, xmax=7, ymin=-6, xmin=0,
            axis lines =center, xlabel={$\theta$}, ylabel={$y$}, grid = major, grid style={dashed},
            ytick={-5,-4,-3,-2,-1,0,1,2,3,4,5},
            xtick={0, 1.57, 3.142, 4.71, 6.28},
            xticklabels={$0$, $\tfrac{\pi}{2}$, $\pi$, $\tfrac{3\pi}{2}$, $2\pi$},
            yticklabels={$-5$,$-4$,$-3$,$-2$,$-1$,$0$,$1$,$2$,$3$,$4$,$5$}, 
            ticklabel style={font=\scriptsize},
            every axis y label/.style={at=(current axis.above origin),anchor=south},
            every axis x label/.style={at=(current axis.right of origin),anchor=west},
            axis on top
          ]
          

            \addplot [line width=2, penColor, smooth,samples=300,domain=(1.57:4.71)] {3*sin(deg(2*x-3.14))-2};
            %\addplot [line width=2, penColor2, smooth,samples=300,domain=(-10:10),<->] {cos(deg(x)};


            %\node[penColor] at (axis cs:3.5,1.1) [anchor=north] {$\sin(\theta)$};
            %\node[penColor2] at (axis cs:1.4,1.3) [anchor=north] {$\cos(\theta)$};



  \end{axis}
\end{tikzpicture}
\end{image}



$\blacktriangleright$ \textbf{Then, extend periodically} 








\begin{image}
\begin{tikzpicture}
  \begin{axis}[
            domain=0:7, ymax=6, xmax=7, ymin=-6, xmin=0,
            axis lines =center, xlabel={$\theta$}, ylabel={$y$}, grid = major, grid style={dashed},
            ytick={-5,-4,-3,-2,-1,0,1,2,3,4,5},
            xtick={0, 1.57, 3.142, 4.71, 6.28},
            xticklabels={$0$, $\tfrac{\pi}{2}$, $\pi$, $\tfrac{3\pi}{2}$, $2\pi$},
            yticklabels={$-5$,$-4$,$-3$,$-2$,$-1$,$0$,$1$,$2$,$3$,$4$,$5$}, 
            ticklabel style={font=\scriptsize},
            every axis y label/.style={at=(current axis.above origin),anchor=south},
            every axis x label/.style={at=(current axis.right of origin),anchor=west},
            axis on top
          ]
          

            \addplot [line width=2, penColor, smooth,samples=300,domain=(0.2:6.5),<->] {3*sin(deg(2*x-3.14)) -2};
            %\addplot [line width=2, penColor2, smooth,samples=300,domain=(-10:10),<->] {cos(deg(x)};


            %\node[penColor] at (axis cs:3.5,1.1) [anchor=north] {$\sin(\theta)$};
            %\node[penColor2] at (axis cs:1.4,1.3) [anchor=north] {$\cos(\theta)$};



  \end{axis}
\end{tikzpicture}
\end{image}




We can calculate some additional exact values at the easy angles. \\



Our principle interval was $\left[ \frac{\pi}{2}, \frac{3\pi}{2}   \right]$, which has a length of $\pi$. \\





$\blacktriangleright$ $30^{\circ} = \frac{\pi}{6}$ is $\frac{1}{12}$ of the normal basic interval $[0, 2\pi]$.  $\frac{1}{12}$ of our period here would be $\frac{1}{12} \cdot \pi = \frac{\pi}{12}$.

Therefore, if we move in $\frac{\pi}{12}$ from $\frac{\pi}{2}$, then we should be at the angle where we get $sin(\frac{\pi}{6}) = \frac{1}{2}$.


$\frac{\pi}{12}$ from $\frac{\pi}{2} = \frac{7\pi}{12}$.

Let's evaluate $f\left( \frac{7\pi}{12} \right)$

\[
f\left( \frac{7\pi}{12} \right) = 3 sin\left(2 \left(\frac{7\pi}{12} \right) - \pi \right) - 2 = 3 sin\left( \frac{2\pi}{12} \right) - 2  = 3 sin\left( \frac{\pi}{6} \right) - 2 = 3 \cdot \frac{1}{2} - 2 = -\frac{1}{2}
\]






\begin{image}
\begin{tikzpicture}
  \begin{axis}[
            domain=0:7, ymax=6, xmax=7, ymin=-6, xmin=0,
            axis lines =center, xlabel={$\theta$}, ylabel={$y$}, grid = major, grid style={dashed},
            ytick={-5,-4,-3,-2,-1,0,1,2,3,4,5},
            xtick={0, 1.57, 3.142, 4.71, 6.28},
            xticklabels={$0$, $\tfrac{\pi}{2}$, $\pi$, $\tfrac{3\pi}{2}$, $2\pi$},
            yticklabels={$-5$,$-4$,$-3$,$-2$,$-1$,$0$,$1$,$2$,$3$,$4$,$5$}, 
            ticklabel style={font=\scriptsize},
            every axis y label/.style={at=(current axis.above origin),anchor=south},
            every axis x label/.style={at=(current axis.right of origin),anchor=west},
            axis on top
          ]
          

            \addplot [line width=2, penColor, smooth,samples=300,domain=(1.57:4.71)] {3*sin(deg(2*x-3.14))-2};
            %\addplot [line width=2, penColor2, smooth,samples=300,domain=(-10:10),<->] {cos(deg(x)};


            %\node[penColor] at (axis cs:3.5,1.1) [anchor=north] {$\sin(\theta)$};
            %\node[penColor2] at (axis cs:1.4,1.3) [anchor=north] {$\cos(\theta)$};

            \addplot[color=penColor,fill=penColor,only marks,mark=*] coordinates{(1.833,-0.5)};



  \end{axis}
\end{tikzpicture}
\end{image}









$\blacktriangleright$ $60^{\circ} = \frac{\pi}{3}$ is $\frac{1}{6}$ of the normal basic interval $[0, 2\pi]$.  $\frac{1}{6}$ of our period here would be $\frac{1}{6} \cdot \pi = \frac{\pi}{6}$.

Therefore, if we move in $\frac{\pi}{6}$ from $\frac{\pi}{2}$, then we should be at the angle where we get $sin(\frac{\pi}{3}) = \frac{\sqrt{3}}{2}$.


$\frac{\pi}{6}$ from $\frac{\pi}{2} = \frac{4\pi}{6} =  \frac{2\pi}{3}$.

Let's evaluate $f\left( \frac{2\pi}{3} \right)$

\[
f\left( \frac{2\pi}{3} \right) = 3 sin\left(2 \left(\frac{2\pi}{3} \right) - \pi \right) - 2 = 3 sin\left( \frac{\pi}{3} \right) - 2  =  3 \cdot \frac{\sqrt{3}}{2} - 2 = \frac{3\sqrt{3}-4}{2} \approx 0.5980762114
\]










\begin{image}
\begin{tikzpicture}
  \begin{axis}[
            domain=0:7, ymax=6, xmax=7, ymin=-6, xmin=0,
            axis lines =center, xlabel={$\theta$}, ylabel={$y$}, grid = major, grid style={dashed},
            ytick={-5,-4,-3,-2,-1,0,1,2,3,4,5},
            xtick={0, 1.57, 3.142, 4.71, 6.28},
            xticklabels={$0$, $\tfrac{\pi}{2}$, $\pi$, $\tfrac{3\pi}{2}$, $2\pi$},
            yticklabels={$-5$,$-4$,$-3$,$-2$,$-1$,$0$,$1$,$2$,$3$,$4$,$5$}, 
            ticklabel style={font=\scriptsize},
            every axis y label/.style={at=(current axis.above origin),anchor=south},
            every axis x label/.style={at=(current axis.right of origin),anchor=west},
            axis on top
          ]
          

            \addplot [line width=2, penColor, smooth,samples=300,domain=(1.57:4.71)] {3*sin(deg(2*x-3.14))-2};
            %\addplot [line width=2, penColor2, smooth,samples=300,domain=(-10:10),<->] {cos(deg(x)};


            %\node[penColor] at (axis cs:3.5,1.1) [anchor=north] {$\sin(\theta)$};
            %\node[penColor2] at (axis cs:1.4,1.3) [anchor=north] {$\cos(\theta)$};

            \addplot[color=penColor,fill=penColor,only marks,mark=*] coordinates{(1.833,-0.5)};
            \addplot[color=penColor,fill=penColor,only marks,mark=*] coordinates{(2.094,0.59)};



  \end{axis}
\end{tikzpicture}
\end{image}







\end{explanation}


\end{example}







\end{document}
