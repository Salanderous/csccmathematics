\documentclass{ximera}


\graphicspath{
  {./}
  {ximeraTutorial/}
  {basicPhilosophy/}
}

\newcommand{\mooculus}{\textsf{\textbf{MOOC}\textnormal{\textsf{ULUS}}}}

\usepackage{tkz-euclide}\usepackage{tikz}
\usepackage{tikz-cd}
\usetikzlibrary{arrows}
\tikzset{>=stealth,commutative diagrams/.cd,
  arrow style=tikz,diagrams={>=stealth}} %% cool arrow head
\tikzset{shorten <>/.style={ shorten >=#1, shorten <=#1 } } %% allows shorter vectors

\usetikzlibrary{backgrounds} %% for boxes around graphs
\usetikzlibrary{shapes,positioning}  %% Clouds and stars
\usetikzlibrary{matrix} %% for matrix
\usepgfplotslibrary{polar} %% for polar plots
\usepgfplotslibrary{fillbetween} %% to shade area between curves in TikZ
\usetkzobj{all}
\usepackage[makeroom]{cancel} %% for strike outs
%\usepackage{mathtools} %% for pretty underbrace % Breaks Ximera
%\usepackage{multicol}
\usepackage{pgffor} %% required for integral for loops



%% http://tex.stackexchange.com/questions/66490/drawing-a-tikz-arc-specifying-the-center
%% Draws beach ball
\tikzset{pics/carc/.style args={#1:#2:#3}{code={\draw[pic actions] (#1:#3) arc(#1:#2:#3);}}}



\usepackage{array}
\setlength{\extrarowheight}{+.1cm}
\newdimen\digitwidth
\settowidth\digitwidth{9}
\def\divrule#1#2{
\noalign{\moveright#1\digitwidth
\vbox{\hrule width#2\digitwidth}}}






\DeclareMathOperator{\arccot}{arccot}
\DeclareMathOperator{\arcsec}{arcsec}
\DeclareMathOperator{\arccsc}{arccsc}

















%%This is to help with formatting on future title pages.
\newenvironment{sectionOutcomes}{}{}


\title{Differential Equations}

\begin{document}

\begin{abstract}
%%%
\end{abstract}
\maketitle




We have seen functions described in two algebraic ways: formulas and equations.


Formulas are special equations where the function is isolated by itself on one side of the equation.  The formula provides a procedure whereby you begin with a domain number, perform calculations with it, and produce the function value.

We have already seen instances where a formula for a function leads to a formula for its derivative. 

$\blacktriangleright$ If $f(x) = a \, x^2  b \, x + c$, then $f'(x) = 2a \, x + b$

$\blacktriangleright$ If $g(t) = A \, \sin(t) + B$, , then $g'(t) = A \, \cos(t)$


When the function is described explicitly, then so is the derivative.


We have also seen functions defined implicitly.


$f(x)$ might be described by $x \cos(f) - \sqrt(x + f) = 0$.  There is no way to solve for $f$ in terms of $x$. It cannot be separated.  Same for its derivative. There is no way to separate $f'(x)$ from $f(x)$.

The equation describing $f'(x)$ includes $f(x)$ and ther is no way around it.


We call these equaitons involving the funciton and its derivative \textbf{Differential Equations}.

It turns out that DIffernetial Equations are the way scientists and engineers model the world.










\subsection{Learning Outcomes}


\begin{sectionOutcomes}
In this section, students will 

\begin{itemize}
\item examine differential equations.
\end{itemize}
\end{sectionOutcomes}

\end{document}
