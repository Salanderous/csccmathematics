\documentclass{ximera}


\graphicspath{
  {./}
  {ximeraTutorial/}
  {basicPhilosophy/}
}

\newcommand{\mooculus}{\textsf{\textbf{MOOC}\textnormal{\textsf{ULUS}}}}

\usepackage{tkz-euclide}\usepackage{tikz}
\usepackage{tikz-cd}
\usetikzlibrary{arrows}
\tikzset{>=stealth,commutative diagrams/.cd,
  arrow style=tikz,diagrams={>=stealth}} %% cool arrow head
\tikzset{shorten <>/.style={ shorten >=#1, shorten <=#1 } } %% allows shorter vectors

\usetikzlibrary{backgrounds} %% for boxes around graphs
\usetikzlibrary{shapes,positioning}  %% Clouds and stars
\usetikzlibrary{matrix} %% for matrix
\usepgfplotslibrary{polar} %% for polar plots
\usepgfplotslibrary{fillbetween} %% to shade area between curves in TikZ
\usetkzobj{all}
\usepackage[makeroom]{cancel} %% for strike outs
%\usepackage{mathtools} %% for pretty underbrace % Breaks Ximera
%\usepackage{multicol}
\usepackage{pgffor} %% required for integral for loops



%% http://tex.stackexchange.com/questions/66490/drawing-a-tikz-arc-specifying-the-center
%% Draws beach ball
\tikzset{pics/carc/.style args={#1:#2:#3}{code={\draw[pic actions] (#1:#3) arc(#1:#2:#3);}}}



\usepackage{array}
\setlength{\extrarowheight}{+.1cm}
\newdimen\digitwidth
\settowidth\digitwidth{9}
\def\divrule#1#2{
\noalign{\moveright#1\digitwidth
\vbox{\hrule width#2\digitwidth}}}






\DeclareMathOperator{\arccot}{arccot}
\DeclareMathOperator{\arcsec}{arcsec}
\DeclareMathOperator{\arccsc}{arccsc}

















%%This is to help with formatting on future title pages.
\newenvironment{sectionOutcomes}{}{}




\title{Higer Order}

\begin{document}
\begin{abstract}
  derivatives of derivatives
\end{abstract}
\maketitle





We have investigated formulas for derivative in a couple of situations.


We began with the value of the derivative as the slope of the tangent line.

The constant function $f(x) = c$ has a horizontal line as its graph and every tangent line is also a horizontal tangent line with slope $0$.


\[
\text{If } \, f(x) = c, \, \text{ then } \, f'(x) = 0
\]



The linear function $g(t) = a \, t + b$ has a line as its graph and every tangent line is also that same line with slope $a$.



\[
\text{If } \, g(t) = a \, x + b, \, \text{ then } \, g'(t) = a
\]




The quadratic function $q(r) = a \, r^2 + b \, r + c$ has the linear function $2a \, r + b$ as its derivative.





Looking at this in reverse, we might notice that a quadrtic function has a derivative, which is a linear function, which has a derivative, which is a constant function, which has a derivative, which is the $0$ function.



Capturing this sequence of derivatives is the idea behing \textbf{higher order derivatives}.









\begin{definition} \textbf{\textcolor{green!50!black}{Higher Order Derivatives}} \\


Let $f(x)$ be a function. \\


Its derivative is also known as the \textbf{first derivative of $f(x)$} and is symbolize by \textbf{$f^{\prime}(x)$} or \textbf{$\frac{df}{dx}$}.

This first derivative of $f(x)$ is a function and has a derivative known as the \textbf{second derivative of $f(x)$} and is symbolize by \textbf{$f^{\prime\prime}(x)$} or \textbf{$\frac{d^2f}{dx^2}$} or \textbf{$f^{(2)}(x)$}.


This second derivative of $f(x)$ is a function and has a derivative known as the \textbf{third derivative of $f(x)$} and is symbolize by \textbf{$f^{\prime\prime\prime}(x)$} or \textbf{$\frac{d^3f}{dx^3}$} or \textbf{$f^{(3)}(x)$}.


In general, the \textbf{$n^{th}$ derivative} is s symbolize by \textbf{$\frac{d^{n}f}{dx^n}$} or \textbf{$f^{(n)}(x)$}.


\end{definition}




\begin{example}


Let $G(t) = 4 \sin(t)$. Then,\\

$G^{\prime}(t) = 4 \cos(t)$ \\

$\frac{d^{2}G}{dt^{2}} = -4 \sin(t)$ \\

$\frac{d^{3}G}{dt^{3}} = -4 \cos(t)$ \\

$\frac{d^{4}G}{dt^{4}} = 4 \sin(t)$ \\

Back to $G(t)$.

\end{example}







\end{document}
