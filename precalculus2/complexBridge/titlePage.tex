\documentclass{ximera}


\graphicspath{
  {./}
  {ximeraTutorial/}
  {basicPhilosophy/}
}

\newcommand{\mooculus}{\textsf{\textbf{MOOC}\textnormal{\textsf{ULUS}}}}

\usepackage{tkz-euclide}\usepackage{tikz}
\usepackage{tikz-cd}
\usetikzlibrary{arrows}
\tikzset{>=stealth,commutative diagrams/.cd,
  arrow style=tikz,diagrams={>=stealth}} %% cool arrow head
\tikzset{shorten <>/.style={ shorten >=#1, shorten <=#1 } } %% allows shorter vectors

\usetikzlibrary{backgrounds} %% for boxes around graphs
\usetikzlibrary{shapes,positioning}  %% Clouds and stars
\usetikzlibrary{matrix} %% for matrix
\usepgfplotslibrary{polar} %% for polar plots
\usepgfplotslibrary{fillbetween} %% to shade area between curves in TikZ
\usetkzobj{all}
\usepackage[makeroom]{cancel} %% for strike outs
%\usepackage{mathtools} %% for pretty underbrace % Breaks Ximera
%\usepackage{multicol}
\usepackage{pgffor} %% required for integral for loops



%% http://tex.stackexchange.com/questions/66490/drawing-a-tikz-arc-specifying-the-center
%% Draws beach ball
\tikzset{pics/carc/.style args={#1:#2:#3}{code={\draw[pic actions] (#1:#3) arc(#1:#2:#3);}}}



\usepackage{array}
\setlength{\extrarowheight}{+.1cm}
\newdimen\digitwidth
\settowidth\digitwidth{9}
\def\divrule#1#2{
\noalign{\moveright#1\digitwidth
\vbox{\hrule width#2\digitwidth}}}






\DeclareMathOperator{\arccot}{arccot}
\DeclareMathOperator{\arcsec}{arcsec}
\DeclareMathOperator{\arccsc}{arccsc}

















%%This is to help with formatting on future title pages.
\newenvironment{sectionOutcomes}{}{}


\title{Complex Bridge}

\begin{document}

\begin{abstract}
%Stuff can go here later if we want!
\end{abstract}
\maketitle




We have many descriptions of our 2D Complex Number System:

\begin{itemize}
\item Points with Rectangular Coordinates, $(a,b)$
\item Vectors with Rectangular Coordinates, $\langle a, b \rangle$
\item Points with Circular Coordinates, $(r,\theta)$
\item Numbers with Rectangular Dimensions, $a + b \, i$
\item Numbers with Angles and Radii, $r \, (\cos(\theta) + \, \sin(\theta))$
\item Numbers via Exponentials, $e^{r + i \, \theta}$
\end{itemize}

These are all describing the same structure, which means all of these descriptions must be mathematically connected. \\


The bridges connecting all of these are built from Trigonometry, Hyperbolic-Trigonometry, and Exponential Algebra.





\subsection{Learning Outcomes}

\begin{sectionOutcomes}
In this section, students will 

\begin{itemize}
\item explore the arithmetic of Complex Numbers.
\end{itemize}
\end{sectionOutcomes}













\begin{center}
\textbf{\textcolor{green!50!black}{ooooo=-=-=-=-=-=-=-=-=-=-=-=-=ooOoo=-=-=-=-=-=-=-=-=-=-=-=-=ooooo}} \\

more examples can be found by following this link\\ \link[More Examples of the Complex Bridge]{https://ximera.osu.edu/csccmathematics/precalculus2/precalculus2/complexBridge/examples/exampleList}

\end{center}






\end{document}
