\documentclass{ximera}


\graphicspath{
  {./}
  {ximeraTutorial/}
  {basicPhilosophy/}
}

\newcommand{\mooculus}{\textsf{\textbf{MOOC}\textnormal{\textsf{ULUS}}}}

\usepackage{tkz-euclide}\usepackage{tikz}
\usepackage{tikz-cd}
\usetikzlibrary{arrows}
\tikzset{>=stealth,commutative diagrams/.cd,
  arrow style=tikz,diagrams={>=stealth}} %% cool arrow head
\tikzset{shorten <>/.style={ shorten >=#1, shorten <=#1 } } %% allows shorter vectors

\usetikzlibrary{backgrounds} %% for boxes around graphs
\usetikzlibrary{shapes,positioning}  %% Clouds and stars
\usetikzlibrary{matrix} %% for matrix
\usepgfplotslibrary{polar} %% for polar plots
\usepgfplotslibrary{fillbetween} %% to shade area between curves in TikZ
\usetkzobj{all}
\usepackage[makeroom]{cancel} %% for strike outs
%\usepackage{mathtools} %% for pretty underbrace % Breaks Ximera
%\usepackage{multicol}
\usepackage{pgffor} %% required for integral for loops



%% http://tex.stackexchange.com/questions/66490/drawing-a-tikz-arc-specifying-the-center
%% Draws beach ball
\tikzset{pics/carc/.style args={#1:#2:#3}{code={\draw[pic actions] (#1:#3) arc(#1:#2:#3);}}}



\usepackage{array}
\setlength{\extrarowheight}{+.1cm}
\newdimen\digitwidth
\settowidth\digitwidth{9}
\def\divrule#1#2{
\noalign{\moveright#1\digitwidth
\vbox{\hrule width#2\digitwidth}}}






\DeclareMathOperator{\arccot}{arccot}
\DeclareMathOperator{\arcsec}{arcsec}
\DeclareMathOperator{\arccsc}{arccsc}

















%%This is to help with formatting on future title pages.
\newenvironment{sectionOutcomes}{}{}


\title{Trig Identities}

\begin{document}

\begin{abstract}
double and half
\end{abstract}
\maketitle






\section{Double Angle Formulas}


\[   \sin(2t) = \frac{e^{i 2t} - e^{-i 2t}}{2 i}      \]


This is a difference of two squares.  It factors.



\[   \sin(2t) = \frac{(e^{i t} - e^{-i t})(e^{i t} + e^{-i t})}{2 i}      \]




\[   \sin(2t) = 2 \cdot \frac{(e^{i t} - e^{-i t})}{2i}    \cdot    \frac{(e^{i t} + e^{-i t})}{2}      \]



\[   \sin(2t) = 2 \sin(t)   \cos(t)      \]

This is the double angle formla for sine.  How about cosine?
Let's begin with $\cos^2(t) - \sin^2(t)$ and rearrange it.











\[      \cos^2(t) - \sin^2(t)      \]



\[      \cos^2(t) - \sin^2(t)   = \left( \frac{e^{i t} + e^{-i t}}{2} \right)^2 - \left( \frac{e^{i t} - e^{-i t}}{2i} \right)^2  \]


\[      \cos^2(t) - \sin^2(t)   = \left( \frac{e^{2i t} + 2 + e^{-2i t}}{4} \right) - \left( \frac{e^{2 i t} - 2 + e^{-2i t}}{-4} \right)  \]



\[      \cos^2(t) - \sin^2(t)   = \frac{2 e^{2i t}  + 2 e^{-2i t}}{4}   = \frac{e^{2i t}  + e^{-2i t}}{2} = \cos(2t)   \]




\[      \cos^2(t) - \sin^2(t)  = \cos(2t)    \]


This is the double angle formula for cosine. \\

Additionally, we have 




\[   \cos(2t) =    \cos^2(t) - \sin^2(t)  =    \cos^2(t) - (1 - \cos^2(t)) =  2 \cos^2(t) - 1 \]


and

\[   \cos(2t) =    \cos^2(t) - \sin^2(t)  =    (1 - \sin^2(t)) - \sin^2(t)  = 1 - 2 \sin^2(t)     \]




































\section{Half Angle Formulas}



\[   \cos(2t)  =  2 \cos^2(t) - 1 \]


\[   \frac{1 + \cos(2t)}{2}  =   \cos^2(t)  \]



and we know that $\cos^2(t) = 1- \sin^2(t)$. Applying this to the formula above gives us

\[  \frac{1 -  \cos(2t)}{2}  =  \sin^2(t)     \]



These are half-angle formulas.  We can see the half better by replacing $t$ with $\frac{x}{2}$



\[   \frac{1 + \cos(x)}{2}  =   \cos^2\left( \tfrac{x}{2} \right)  \]



\[   \frac{1 - \cos(x)}{2}  =   \sin^2\left( \tfrac{x}{2} \right)  \]







\begin{example}

Obtain an expression for $\cos\left(  \frac{\pi}{8}  \right)$ that only uses square roots.
Let $t = \frac{\pi}{8}$ in the equation above.



\begin{explanation}

$\frac{\pi}{8}$ is half of $\frac{\pi}{4}$




\[   \frac{1 + \cos( \frac{\pi}{4} )}{2}  =   \cos^2\left( \frac{\pi}{8} \right)  \]


\[   \frac{1 + \frac{1}{\sqrt{2}}}{2}  =   \cos^2\left( \frac{\pi}{8} \right)  \]



$\frac{\pi}{8}$ is in quadrant I, therefore cosine is positive.




\[   \sqrt{ \frac{1 + \frac{1}{\sqrt{2}}}{2} }  =   \cos\left( \frac{\pi}{8} \right)  \]





\end{explanation}

\end{example}


The double angle formulas help you reduce the argument inside the trigonometric function. \\


You replace $\sin(2 \theta)$ with $2 \sin(\theta) \cos(\theta)$.  You end up with more functions, but they are easier to work with. \\


The half angle formulas get rid of squaring.


$\sin^2\left( \tfrac{x}{2} \right)$ is replaced with $\frac{1 - \cos(x)}{2}$.  Much easier to work with.















\begin{center}
\textbf{\textcolor{green!50!black}{ooooo=-=-=-=-=-=-=-=-=-=-=-=-=ooOoo=-=-=-=-=-=-=-=-=-=-=-=-=ooooo}} \\

more examples can be found by following this link\\ \link[More Examples of the Complex Bridge]{https://ximera.osu.edu/csccmathematics/precalculus2/precalculus2/complexBridge/examples/exampleList}

\end{center}



\end{document}
