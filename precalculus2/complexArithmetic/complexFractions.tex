\documentclass{ximera}


\graphicspath{
  {./}
  {ximeraTutorial/}
  {basicPhilosophy/}
}

\newcommand{\mooculus}{\textsf{\textbf{MOOC}\textnormal{\textsf{ULUS}}}}

\usepackage{tkz-euclide}\usepackage{tikz}
\usepackage{tikz-cd}
\usetikzlibrary{arrows}
\tikzset{>=stealth,commutative diagrams/.cd,
  arrow style=tikz,diagrams={>=stealth}} %% cool arrow head
\tikzset{shorten <>/.style={ shorten >=#1, shorten <=#1 } } %% allows shorter vectors

\usetikzlibrary{backgrounds} %% for boxes around graphs
\usetikzlibrary{shapes,positioning}  %% Clouds and stars
\usetikzlibrary{matrix} %% for matrix
\usepgfplotslibrary{polar} %% for polar plots
\usepgfplotslibrary{fillbetween} %% to shade area between curves in TikZ
\usetkzobj{all}
\usepackage[makeroom]{cancel} %% for strike outs
%\usepackage{mathtools} %% for pretty underbrace % Breaks Ximera
%\usepackage{multicol}
\usepackage{pgffor} %% required for integral for loops



%% http://tex.stackexchange.com/questions/66490/drawing-a-tikz-arc-specifying-the-center
%% Draws beach ball
\tikzset{pics/carc/.style args={#1:#2:#3}{code={\draw[pic actions] (#1:#3) arc(#1:#2:#3);}}}



\usepackage{array}
\setlength{\extrarowheight}{+.1cm}
\newdimen\digitwidth
\settowidth\digitwidth{9}
\def\divrule#1#2{
\noalign{\moveright#1\digitwidth
\vbox{\hrule width#2\digitwidth}}}






\DeclareMathOperator{\arccot}{arccot}
\DeclareMathOperator{\arcsec}{arcsec}
\DeclareMathOperator{\arccsc}{arccsc}

















%%This is to help with formatting on future title pages.
\newenvironment{sectionOutcomes}{}{}


\title{Fractions}

\begin{document}

\begin{abstract}
division
\end{abstract}
\maketitle




Complex numbers look like $a + b /, i$ where $a, b \in \mathbb{R}$.


$a$ and $b$ can be any real numbers:



\begin{itemize}
	\item $2 + 3 \, i$
	\item $\frac{2}{3} - 5 \, i$
	\item $\pi + \sqrt{7} \, i$
	\item $\frac{\sqrt{5}}{7} - \frac{\pi}{\sqrt{3}} \, i$
\end{itemize}



This makes division by a real number straight forward.




\[
\frac{a + b \, i}{c} = \frac{a}{c} + \frac{b}{c} \, i
\]





We will use this fact along with complex conjugates to give us a rule for division.



In Calculus, we don't really divide very much.  Instead this is all rephrased in terms of multiplication by the reciprocal.  We encounter division in terms of quotients.


The quotient of two Complex numbers should be a Complex number.

\[
\frac{a + b \, i}{c + d \, i}  = A + B \, i  \text{ for some real numbers } A \text{ and } B
\]


Our plan is to convert the denominator to a real number and then separate the pieces.



\[
\frac{a + b \, i}{c + d \, i} = \frac{a + b \, i}{c + d \, i}  \cdot 1 
\]

\[
= \frac{a + b \, i}{c + d \, i} \cdot \frac{c - d \, i}{c - d \, i} 
\]

\[
= \frac{(a + b \, i)(c + d \, i)}{c^2 + d^2} 
\]

\[
= \frac{(ac - bd) + (ad + bc) \, i}{c^2 + d^2} 
\]

\[
= \frac{(ac - bd)}{c^2 + d^2}  + \frac{(ad + bc)}{c^2 + d^2} \, i
\]







\end{document}
