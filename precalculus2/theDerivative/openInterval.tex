\documentclass{ximera}


\graphicspath{
  {./}
  {ximeraTutorial/}
  {basicPhilosophy/}
}

\newcommand{\mooculus}{\textsf{\textbf{MOOC}\textnormal{\textsf{ULUS}}}}

\usepackage{tkz-euclide}\usepackage{tikz}
\usepackage{tikz-cd}
\usetikzlibrary{arrows}
\tikzset{>=stealth,commutative diagrams/.cd,
  arrow style=tikz,diagrams={>=stealth}} %% cool arrow head
\tikzset{shorten <>/.style={ shorten >=#1, shorten <=#1 } } %% allows shorter vectors

\usetikzlibrary{backgrounds} %% for boxes around graphs
\usetikzlibrary{shapes,positioning}  %% Clouds and stars
\usetikzlibrary{matrix} %% for matrix
\usepgfplotslibrary{polar} %% for polar plots
\usepgfplotslibrary{fillbetween} %% to shade area between curves in TikZ
\usetkzobj{all}
\usepackage[makeroom]{cancel} %% for strike outs
%\usepackage{mathtools} %% for pretty underbrace % Breaks Ximera
%\usepackage{multicol}
\usepackage{pgffor} %% required for integral for loops



%% http://tex.stackexchange.com/questions/66490/drawing-a-tikz-arc-specifying-the-center
%% Draws beach ball
\tikzset{pics/carc/.style args={#1:#2:#3}{code={\draw[pic actions] (#1:#3) arc(#1:#2:#3);}}}



\usepackage{array}
\setlength{\extrarowheight}{+.1cm}
\newdimen\digitwidth
\settowidth\digitwidth{9}
\def\divrule#1#2{
\noalign{\moveright#1\digitwidth
\vbox{\hrule width#2\digitwidth}}}






\DeclareMathOperator{\arccot}{arccot}
\DeclareMathOperator{\arcsec}{arcsec}
\DeclareMathOperator{\arccsc}{arccsc}

















%%This is to help with formatting on future title pages.
\newenvironment{sectionOutcomes}{}{}


\title{Refining}

\begin{document}

\begin{abstract}
open interval
\end{abstract}
\maketitle



Everything gets more complicated the deeper you look. \\


Mathematics is no different. \\


We are just getting started with the derivative and the more we examine it, the more complicated it will become. \\

We are ready for the next layer to this story. \\




\subsection{So far...}


Our definition of the derivative has been a graphical definition.

\begin{idea}

Let $f(x)$ be a function with domain $D$. \\
Let $a \in D$ be a domain number. \\

Then $f(x)$ has a graph. \\

There are two possibilities:  \\


\textbf{\textcolor{blue!55!black}{1)}} \textbf{the graph has a tangent line at $(a, f(a))$}

\textbf{\textcolor{blue!55!black}{1)}} \textbf{the graph does not have a tangent line at $(a, f(a))$}




If the graph has a tangent line at $(a, f(a))$ and this tangent line has a slope, then


\begin{center}

$f'(a)$ = the slope of the tangnet line at $(a, f(a))$,

\end{center}



Otherwise, we say that $f'(a)$ does not exist (DNE).






\end{idea}

There are several ways in which $f'(a)$ might not exist. \\



There may be a tangent line, but the tangent line is vertical and thus has no slope. \\
$f(x) = 4 \sqrt[3]{x}$ is an example.


\begin{image}
\begin{tikzpicture}
  \begin{axis}[
            domain=-10:10, ymax=10, xmax=10, ymin=-10, xmin=-10,
            axis lines =center, xlabel=$t$, ylabel=$y$, grid = major,
            ytick={-10,-8,-6,-4,-2,2,4,6,8,10},
            xtick={-10,-8,-6,-4,-2,2,4,6,8,10},
            ticklabel style={font=\scriptsize},
            every axis y label/.style={at=(current axis.above origin),anchor=south},
            every axis x label/.style={at=(current axis.right of origin),anchor=west},
            axis on top
          ]
          

          addplot [line width=2, penColor, smooth,samples=200,domain=(-8:0),<-] {4*((-x)^0.333)};
          addplot [line width=2, penColor, smooth,samples=200,domain=(0:8),->] {4*((x)^0.333)};

          addplot [line width=1, gray, smooth,samples=200,domain=(-10:10),<->] ({0},{x});
            %\addplot [line width=1, gray, dashed,samples=200,domain=(-10:10),<->] ({x},{x});


          %\addplot[color=penColor,fill=penColor,only marks,mark=*] coordinates{(-2,-4)};


           

  \end{axis}
\end{tikzpicture}
\end{image}


This graph has a tangentline at $(0,0)$.  However, the tangent line is vertical, which means it doesn't have slope.  \\

Thefore, $f(0)$ does not exist.

























\begin{template}

Let $f$ be a function. \\

Then the derivative is denoted as $f'$. \\


$f$ is increasing on intervals where $f'$ is positive. \\


$f$ is decreasing on intervals where $f'$ is negative. \\


\end{template}

This allows us to bring our algebraic tools to the question of function behavior. \\











\begin{example} \textbf{\textcolor{blue!55!black}{Quadratics}} \\


Any quadratic can be written in the form $a \, x^2 + b \, x + c$. \\

The derivative is given by $2a \, x + b$.


The derivative switches signs at $\frac{-b}{2a}$, which is the first coordinate of the vertex. \\



\textbf{\textcolor{red!90!darkgray}{$\blacktriangleright$}} If $a < 0$, then 

\begin{itemize}
\item $f' > 0$ on $\left( -\infty, \frac{-b}{2a} \right)$ and $f$ is increasing on $\left( -\infty, \frac{-b}{2a} \right)$. \\
\item $f' < 0$ on $\left( \frac{-b}{2a}, -\infty \right)$ and $f$ is decreasing on $\left( \frac{-b}{2a}, -\infty \right)$. 
\end{itemize}







\textbf{\textcolor{red!90!darkgray}{$\blacktriangleright$}} If $a > 0$, then 

\begin{itemize}
\item $f' < 0$ on $\left( -\infty, \frac{-b}{2a} \right)$ and $f$ is decreasing on $\left( -\infty, \frac{-b}{2a} \right)$. \\
\item $f' > 0$ on $\left( \frac{-b}{2a}, -\infty \right)$ and $f$ is increasing on $\left( \frac{-b}{2a}, -\infty \right)$. 
\end{itemize}






\end{example}


























\begin{center}
\textbf{\textcolor{green!50!black}{ooooo=-=-=-=-=-=-=-=-=-=-=-=-=ooOoo=-=-=-=-=-=-=-=-=-=-=-=-=ooooo}} \\

more examples can be found by following this link\\ \link[More Examples of the Elementary Library]{https://ximera.osu.edu/csccmathematics/precalculus2/precalculus2/theDerivative/examples/exampleList}

\end{center}







\end{document}
