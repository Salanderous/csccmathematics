\documentclass{ximera}


\graphicspath{
  {./}
  {ximeraTutorial/}
  {basicPhilosophy/}
}

\newcommand{\mooculus}{\textsf{\textbf{MOOC}\textnormal{\textsf{ULUS}}}}

\usepackage{tkz-euclide}\usepackage{tikz}
\usepackage{tikz-cd}
\usetikzlibrary{arrows}
\tikzset{>=stealth,commutative diagrams/.cd,
  arrow style=tikz,diagrams={>=stealth}} %% cool arrow head
\tikzset{shorten <>/.style={ shorten >=#1, shorten <=#1 } } %% allows shorter vectors

\usetikzlibrary{backgrounds} %% for boxes around graphs
\usetikzlibrary{shapes,positioning}  %% Clouds and stars
\usetikzlibrary{matrix} %% for matrix
\usepgfplotslibrary{polar} %% for polar plots
\usepgfplotslibrary{fillbetween} %% to shade area between curves in TikZ
\usetkzobj{all}
\usepackage[makeroom]{cancel} %% for strike outs
%\usepackage{mathtools} %% for pretty underbrace % Breaks Ximera
%\usepackage{multicol}
\usepackage{pgffor} %% required for integral for loops



%% http://tex.stackexchange.com/questions/66490/drawing-a-tikz-arc-specifying-the-center
%% Draws beach ball
\tikzset{pics/carc/.style args={#1:#2:#3}{code={\draw[pic actions] (#1:#3) arc(#1:#2:#3);}}}



\usepackage{array}
\setlength{\extrarowheight}{+.1cm}
\newdimen\digitwidth
\settowidth\digitwidth{9}
\def\divrule#1#2{
\noalign{\moveright#1\digitwidth
\vbox{\hrule width#2\digitwidth}}}






\DeclareMathOperator{\arccot}{arccot}
\DeclareMathOperator{\arcsec}{arcsec}
\DeclareMathOperator{\arccsc}{arccsc}

















%%This is to help with formatting on future title pages.
\newenvironment{sectionOutcomes}{}{}


\title{Critical Numbers+}

\begin{document}

\begin{abstract}
candidates for change
\end{abstract}
\maketitle








The sign of the derivative tells us where the original function is increasing or decreasing.  Therefore, we are very interested in where the derivative changes sign.\\

This can happen at three types of numbers. \\


\textbf{\textcolor{red!90!darkgray}{$\blacktriangleright$}} The derivative can change signs across a zero. \\


\textbf{\textcolor{red!90!darkgray}{$\blacktriangleright$}} The derivative can change signs across a discontiuity. \\


\textbf{\textcolor{red!90!darkgray}{$\blacktriangleright$}}  The derivative can change signs across a singularity. \\



\begin{center}
The operative word here is \textbf{\textcolor{red!80!black}{CAN}}.
\end{center}




\begin{template}  \textbf{\textcolor{blue!55!black}{Critical Numbers}}  \\

Let $f$ be a function with derivative, $f'$. \\


A \textbf{\textcolor{green!50!black}{critical number}} of $f$ is a domain number of $f$ where $f' = 0$ or $f$ does not exist.


\end{template}

Functions \textbf{\textcolor{red!80!black}{CAN}} switch behavior at critical numbers. \\








\begin{template}  \textbf{\textcolor{blue!55!black}{Singularities}}  \\

Let $f$ be a function. \\

A \textbf{\textcolor{green!50!black}{singularity}} of $f$ is a non-domain number where $f$ has weird behavior.



\begin{explanation}

We have expressed somewhat stable descriptions for discontiuities, but not for singularities.  Calculus will help us with this.
\end{explanation}


\end{template}

Functions \textbf{\textcolor{red!80!black}{CAN}} switch behavior at singularities. \\




Hunting down zeros, discontinuities, and singularities have risen to the top of our to-do list.  They are the crux to function behavior. \\













\begin{center}
\textbf{\textcolor{green!50!black}{ooooo=-=-=-=-=-=-=-=-=-=-=-=-=ooOoo=-=-=-=-=-=-=-=-=-=-=-=-=ooooo}} \\

more examples can be found by following this link\\ \link[More Examples of the Elementary Library]{https://ximera.osu.edu/csccmathematics/precalculus2/precalculus2/theDerivative/examples/exampleList}

\end{center}







\end{document}
