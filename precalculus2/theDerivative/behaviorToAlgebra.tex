\documentclass{ximera}


\graphicspath{
  {./}
  {ximeraTutorial/}
  {basicPhilosophy/}
}

\newcommand{\mooculus}{\textsf{\textbf{MOOC}\textnormal{\textsf{ULUS}}}}

\usepackage{tkz-euclide}\usepackage{tikz}
\usepackage{tikz-cd}
\usetikzlibrary{arrows}
\tikzset{>=stealth,commutative diagrams/.cd,
  arrow style=tikz,diagrams={>=stealth}} %% cool arrow head
\tikzset{shorten <>/.style={ shorten >=#1, shorten <=#1 } } %% allows shorter vectors

\usetikzlibrary{backgrounds} %% for boxes around graphs
\usetikzlibrary{shapes,positioning}  %% Clouds and stars
\usetikzlibrary{matrix} %% for matrix
\usepgfplotslibrary{polar} %% for polar plots
\usepgfplotslibrary{fillbetween} %% to shade area between curves in TikZ
\usetkzobj{all}
\usepackage[makeroom]{cancel} %% for strike outs
%\usepackage{mathtools} %% for pretty underbrace % Breaks Ximera
%\usepackage{multicol}
\usepackage{pgffor} %% required for integral for loops



%% http://tex.stackexchange.com/questions/66490/drawing-a-tikz-arc-specifying-the-center
%% Draws beach ball
\tikzset{pics/carc/.style args={#1:#2:#3}{code={\draw[pic actions] (#1:#3) arc(#1:#2:#3);}}}



\usepackage{array}
\setlength{\extrarowheight}{+.1cm}
\newdimen\digitwidth
\settowidth\digitwidth{9}
\def\divrule#1#2{
\noalign{\moveright#1\digitwidth
\vbox{\hrule width#2\digitwidth}}}






\DeclareMathOperator{\arccot}{arccot}
\DeclareMathOperator{\arcsec}{arcsec}
\DeclareMathOperator{\arccsc}{arccsc}

















%%This is to help with formatting on future title pages.
\newenvironment{sectionOutcomes}{}{}


\title{Algebra}

\begin{document}

\begin{abstract}
positive and negative
\end{abstract}
\maketitle



Algebra's strength is in identifying zeros. \\


This is largely due to the Zero Property Property, which says that 



\[
a = 0 \, \text{ and } \, b = 0 \, \text{ are the ONLY solutions to } \, a \cdot b = 0
\]


Unless you know of some handy property of a funciton, our procedure often begin with ``get eveything on one side and $0$ on the other''.




On the other hand, increasing and decreasing are not equations to solve.  They are comparisons of movement or change between the range and domain. \\

Our algebra is not the good at such comparisons. \\


The derivative rephrases this comparison of change back in algebra, where we have methods. \\





\begin{template}

Let $f$ be a function. \\

Then the derivative is denoted as $f'$. \\


$f$ is increasing on intervals where $f'$ is positive. \\


$f$ is decreasing on intervals where $f'$ is negative. \\


\end{template}

This allows us to bring our algebraic tools to the question of function behavior. \\











\begin{example} \textbf{\textcolor{blue!55!black}{Quadratics}} \\


Any quadratic can be written in the form $a \, x^2 + b \, x + c$. \\

The derivative is given by $2a \, x + b$.


The derivative switches signs at $\frac{-b}{2a}$, which is the first coordinate of the vertex. \\



\textbf{\textcolor{red!90!darkgray}{$\blacktriangleright$}} If $a < 0$, then 

\begin{itemize}
\item $f' > 0$ on $\left( -\infty, \frac{-b}{2a} \right)$ and $f$ is increasing on $\left( -\infty, \frac{-b}{2a} \right)$. \\
\item $f' < 0$ on $\left( \frac{-b}{2a}, -\infty \right)$ and $f$ is decreasing on $\left( \frac{-b}{2a}, -\infty \right)$. 
\end{itemize}







\textbf{\textcolor{red!90!darkgray}{$\blacktriangleright$}} If $a > 0$, then 

\begin{itemize}
\item $f' < 0$ on $\left( -\infty, \frac{-b}{2a} \right)$ and $f$ is decreasing on $\left( -\infty, \frac{-b}{2a} \right)$. \\
\item $f' > 0$ on $\left( \frac{-b}{2a}, -\infty \right)$ and $f$ is increasing on $\left( \frac{-b}{2a}, -\infty \right)$. 
\end{itemize}






\end{example}


























\begin{center}
\textbf{\textcolor{green!50!black}{ooooo=-=-=-=-=-=-=-=-=-=-=-=-=ooOoo=-=-=-=-=-=-=-=-=-=-=-=-=ooooo}} \\

more examples can be found by following this link\\ \link[More Examples of the Elementary Library]{https://ximera.osu.edu/csccmathematics/precalculus2/precalculus2/theDerivative/examples/exampleList}

\end{center}







\end{document}
