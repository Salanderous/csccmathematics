\documentclass{ximera}


\graphicspath{
  {./}
  {ximeraTutorial/}
  {basicPhilosophy/}
}

\newcommand{\mooculus}{\textsf{\textbf{MOOC}\textnormal{\textsf{ULUS}}}}

\usepackage{tkz-euclide}\usepackage{tikz}
\usepackage{tikz-cd}
\usetikzlibrary{arrows}
\tikzset{>=stealth,commutative diagrams/.cd,
  arrow style=tikz,diagrams={>=stealth}} %% cool arrow head
\tikzset{shorten <>/.style={ shorten >=#1, shorten <=#1 } } %% allows shorter vectors

\usetikzlibrary{backgrounds} %% for boxes around graphs
\usetikzlibrary{shapes,positioning}  %% Clouds and stars
\usetikzlibrary{matrix} %% for matrix
\usepgfplotslibrary{polar} %% for polar plots
\usepgfplotslibrary{fillbetween} %% to shade area between curves in TikZ
\usetkzobj{all}
\usepackage[makeroom]{cancel} %% for strike outs
%\usepackage{mathtools} %% for pretty underbrace % Breaks Ximera
%\usepackage{multicol}
\usepackage{pgffor} %% required for integral for loops



%% http://tex.stackexchange.com/questions/66490/drawing-a-tikz-arc-specifying-the-center
%% Draws beach ball
\tikzset{pics/carc/.style args={#1:#2:#3}{code={\draw[pic actions] (#1:#3) arc(#1:#2:#3);}}}



\usepackage{array}
\setlength{\extrarowheight}{+.1cm}
\newdimen\digitwidth
\settowidth\digitwidth{9}
\def\divrule#1#2{
\noalign{\moveright#1\digitwidth
\vbox{\hrule width#2\digitwidth}}}






\DeclareMathOperator{\arccot}{arccot}
\DeclareMathOperator{\arcsec}{arcsec}
\DeclareMathOperator{\arccsc}{arccsc}

















%%This is to help with formatting on future title pages.
\newenvironment{sectionOutcomes}{}{}


\title{Complex Powers}

\begin{document}

\begin{abstract}
$i^i$
\end{abstract}
\maketitle






We know that every nonzero Complex number can be written as the product of a positive real number (a scalar) and a Complex number on the unit circle.  Euler's Formula tells us how each Complex number on the unit circle can be written as a Complex exponential.


$\blacktriangleright$ \textbf{\textcolor{purple!85!blue}{Euler's Formula}}   


Every complex number on the unit circle can be written in the form


\[   e^{i \theta} = \cos(\theta) + i \sin(\theta)         \]


And, we already know how to write any real number in exponential form: $r = e^{ln(r)}$.  





$\blacktriangleright$ \textbf{\textcolor{purple!85!blue}{Complex Numbers}}   




Combining these together, we get that every nonzero complex number can be written in the form  
\[
r \cdot (\cos(\theta) + i \sin(\theta))  = e^{ln(r)} \cdot e^{i \, \theta} = e^{\ln(r) + i \, \theta}
\]



If $z = a + b \, i$, then $r = \sqrt{a^2 + b^2}$ and $\theta$ is the counterclockwise angle from the positive real-axis to $z$.




\begin{center}

\textbf{\textcolor{red!80!black}{EVERY complex number can be written in the form $e^{\ln(r) + i \, \theta}$}}

\end{center}









\begin{procedure} Complex Powers


When raising a complex number to a complex power, we first rewrite in exponential form, then take advantage of some logarithm and exponential rules.


$\blacktriangleright$ \textbf{Example:}  
\[  (2+2i)^{1+\sqrt{3} \, i}    \]


\textbf{\textcolor{blue!75!black}{Step 1)}}  Rewrite in exponential form 




\[  (2+2i)^{1+\sqrt{3} \, i}   =        e^{\ln\left( (2+2i)^{1+\sqrt{3} \, i} \right)} \]




\textbf{\textcolor{blue!75!black}{Step 2)}}  Log Rule : $\ln(a^b) = b \ln(a)$ 


\[   =        e^{(1+\sqrt{3} \, i) \ln(2+2i)} \]




\textbf{\textcolor{blue!75!black}{Step 3)}}  Complex Logarithm 


\[   \ln(2+2i) = \ln(2\sqrt{2}) + \frac{\pi}{4} \, i \]



\[   (2+2i)^{1+\sqrt{3} \, i}   =       e^{(1+\sqrt{3} \, i) \cdot (\ln(2\sqrt{2}) + \frac{\pi}{4} \, i)} \]






\textbf{\textcolor{blue!75!black}{Step 4)}}  Multiply 



\[
=    e^{   \ln(2\sqrt{2}) +   \frac{\pi}{4} \, i     +  \sqrt{3} \ln(2\sqrt{2}) \, i - \sqrt{3}  \frac{\pi}{4} (-1) }
\]



\textbf{\textcolor{blue!75!black}{Step 5)}}  Collect Real and Imaginary parts 



\[
=   e^{   \ln(2\sqrt{2}) + \frac{\sqrt{3}\pi}{4} +  (\frac{\pi}{4} + \sqrt{3} \ln(2\sqrt{2}) )  \, i   }
\]




\textbf{\textcolor{blue!75!black}{Step 6)}}  Separate 


\[
=   e^{   \ln(2\sqrt{2}) + \frac{\sqrt{3}\pi}{4}} \cdot  e^{(\frac{\pi}{4} + \sqrt{3} \ln(2\sqrt{2}) )  \, i}
\]





\[
=   (2\sqrt{2})  e^{\frac{\sqrt{3}\pi}{4}} \cdot  e^{(\frac{\pi}{4} + \sqrt{3} \ln(2\sqrt{2}) )  \, i}
\]




\textbf{\textcolor{blue!75!black}{Step 6)}}  $r$ and $\theta$ 



\[ r =  2\sqrt{2}  e^{\frac{\sqrt{3}\pi}{4}} \]

\[ \theta =  \frac{\pi}{4} + \sqrt{3} \ln(2\sqrt{2}) \]


\textbf{\textcolor{blue!75!black}{Step 6)}}  $a + b \, i$ form


\[
 (2+2i)^{1+\sqrt{3} \, i}   = 2\sqrt{2}  e^{\frac{\sqrt{3}\pi}{4}} \cdot \left( \cos\left( \frac{\pi}{4} + \sqrt{3} \ln(2\sqrt{2}) \right) + i \, \sin\left( \frac{\pi}{4} + \sqrt{3} \ln(2\sqrt{2}) \right) \right)
\]




\[
   = 2\sqrt{2}  e^{\frac{\sqrt{3}\pi}{4}} \cdot  \cos\left( \frac{\pi}{4} + \sqrt{3} \ln(2\sqrt{2}) \right) + 2\sqrt{2}  e^{\frac{\sqrt{3}\pi}{4}} \sin\left( \frac{\pi}{4} + \sqrt{3} \ln(2\sqrt{2}) \right) \, i
\]

\end{procedure}



We put everything into an exponent of $e$, so that we can use a logarithm rule and bring the power down and outside the logarithm.  Then we use the definition of the complex logarithm to change the complex logarithm in the $a + b \, i$ form. Then we multiply everything out and gather back into the $a + b \, i$ form. 

Then we separate into two exponentials, because the first will be real and the second will be imaginary. The second exponential is of the form $e^{i \, \theta}$.  Then we can reaad off $\theta$ and replace the exponential with $cos(\theta) + i \, sin(\theta)$.  Now, we are back to $a + b \, i$ form.




\subsection{Calculus}

That may seem like a lot of steps at first, but the over all idea is to target the exponent and make it not an exponent. By pushing the logarithm into an exponent, we can change it to a product.  Once it is a product, then we have lots of algebra to help us.

This mimics exactly how we will differentiate functions with formulas involving exponents:  $\frac{d}{dx} f(x)^{g(x)}$

















\end{document}
