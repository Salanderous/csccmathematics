\documentclass{ximera}


\graphicspath{
  {./}
  {ximeraTutorial/}
  {basicPhilosophy/}
}

\newcommand{\mooculus}{\textsf{\textbf{MOOC}\textnormal{\textsf{ULUS}}}}

\usepackage{tkz-euclide}\usepackage{tikz}
\usepackage{tikz-cd}
\usetikzlibrary{arrows}
\tikzset{>=stealth,commutative diagrams/.cd,
  arrow style=tikz,diagrams={>=stealth}} %% cool arrow head
\tikzset{shorten <>/.style={ shorten >=#1, shorten <=#1 } } %% allows shorter vectors

\usetikzlibrary{backgrounds} %% for boxes around graphs
\usetikzlibrary{shapes,positioning}  %% Clouds and stars
\usetikzlibrary{matrix} %% for matrix
\usepgfplotslibrary{polar} %% for polar plots
\usepgfplotslibrary{fillbetween} %% to shade area between curves in TikZ
\usetkzobj{all}
\usepackage[makeroom]{cancel} %% for strike outs
%\usepackage{mathtools} %% for pretty underbrace % Breaks Ximera
%\usepackage{multicol}
\usepackage{pgffor} %% required for integral for loops



%% http://tex.stackexchange.com/questions/66490/drawing-a-tikz-arc-specifying-the-center
%% Draws beach ball
\tikzset{pics/carc/.style args={#1:#2:#3}{code={\draw[pic actions] (#1:#3) arc(#1:#2:#3);}}}



\usepackage{array}
\setlength{\extrarowheight}{+.1cm}
\newdimen\digitwidth
\settowidth\digitwidth{9}
\def\divrule#1#2{
\noalign{\moveright#1\digitwidth
\vbox{\hrule width#2\digitwidth}}}






\DeclareMathOperator{\arccot}{arccot}
\DeclareMathOperator{\arcsec}{arcsec}
\DeclareMathOperator{\arccsc}{arccsc}

















%%This is to help with formatting on future title pages.
\newenvironment{sectionOutcomes}{}{}


\title{Complex Powers}

\begin{document}

\begin{abstract}
$i^i$
\end{abstract}
\maketitle





We know that every nonzero Complex number can be wirtten as the product of a positive real number (a scalar) and a COmplex number on the unit circle.  Euler's Formula tells us how each Complex number on the unit circle can be written as a Complex exponential.


$\blacktriangleright$ \textbf{Euler's Formula}


\[   e^{i \theta} = \cos(\theta) + i \sin(\theta)         \]


And, we already know how to write and real number in exponential form: $r = e^{ln(r)}$.  

Combining these together, we get that every nonzero Complex number can be written in the form  $e^{ln(r)} \cdot e^{i t} = e^{\ln(r) + i \theta}$




If $z = a + b \, i$, then $r = \sqrt{a^2 + b^2}$ and $\theta = $ angle from positive $x$-axis to $z$.







\begin{example} Complex Logarithm


Let $z = -3 - 3 \, i$.


$|z| = \sqrt{(-3)^2 + (-3)^2} =  \sqrt{18} = 3 \sqrt{2}$


$ArcTan\left(\frac{-3}{-3}\right) = \frac{\pi}{4}$.  This reference angle gives us $\theta = \frac{5 \pi}{4}$




$z = -3 - 3 \, i =  e^{\ln(3 \sqrt{2}) + (\tfrac{5 \pi}{4} \pm 2 k \pi) \, i}     \, \text{ where } \, k \in \mathbb{N}  $

\end{example}




There are an infinite number of Complex logarithms for each Complex number.


Following our range restrictions for ArcSine and ArcCosine, we pick a \textbf{principal value} for the Complex logarithm. Since we would like to keep the positive real axis as clean as possible, we chooose the interval $(-\pi, \pi]$ for the principal value of the Complex logarithm.

This interval is known as the \textbf{principal branch} of the Complex logarithm.  The negative real axis is called a \textbf{branch cut} for the Complex logarithm.





\begin{definition}

If $z = a + b \, i$, then 

\[   \ln(z) = \ln(\sqrt{a^2 + b^2}) + \theta \, i       \]



where $\theta \in (-\pi, \pi]$ and is the counterclockwise angle from the positive real axis to $z$.



This $\theta$ is often referred to as $Arg(z)$.



\end{definition}




Since $Arg(r) = 0$ for any positive real number, our Complex logarithm will agree with our real logarithm.








\begin{example} Some Logarithms


\begin{itemize}
\item $\ln(5) = \ln(5) + 0 \, i = \ln(5)$

\item $\ln(2i) = \ln(2) + \frac{\pi}{2} \, i$

\item $\ln(-5 + 5 \, i) = \ln(5 \sqrt{2}) + \frac{3\pi}{4}\, i$

\item $ln(-1) = \pi \, i$



\end{itemize}




\end{example}




This gives us a famous equation.



\begin{theorem} Euler's Identity

\[  e^{\pi \, i} + 1 = 0       \]

\end{theorem}









\begin{example}  $i^i$


\[    i^i =  e^{ln(i^i)} = e^{i \cdot ln(i)} = e^{ i \cdot \tfrac{\pi}{2} \, i}   = e^{-\tfrac{\pi}{2}} \approx  0.2078795764\]






\end{example}





























\end{document}
