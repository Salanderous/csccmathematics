\documentclass{ximera}


\graphicspath{
  {./}
  {ximeraTutorial/}
  {basicPhilosophy/}
}

\newcommand{\mooculus}{\textsf{\textbf{MOOC}\textnormal{\textsf{ULUS}}}}

\usepackage{tkz-euclide}\usepackage{tikz}
\usepackage{tikz-cd}
\usetikzlibrary{arrows}
\tikzset{>=stealth,commutative diagrams/.cd,
  arrow style=tikz,diagrams={>=stealth}} %% cool arrow head
\tikzset{shorten <>/.style={ shorten >=#1, shorten <=#1 } } %% allows shorter vectors

\usetikzlibrary{backgrounds} %% for boxes around graphs
\usetikzlibrary{shapes,positioning}  %% Clouds and stars
\usetikzlibrary{matrix} %% for matrix
\usepgfplotslibrary{polar} %% for polar plots
\usepgfplotslibrary{fillbetween} %% to shade area between curves in TikZ
\usetkzobj{all}
\usepackage[makeroom]{cancel} %% for strike outs
%\usepackage{mathtools} %% for pretty underbrace % Breaks Ximera
%\usepackage{multicol}
\usepackage{pgffor} %% required for integral for loops



%% http://tex.stackexchange.com/questions/66490/drawing-a-tikz-arc-specifying-the-center
%% Draws beach ball
\tikzset{pics/carc/.style args={#1:#2:#3}{code={\draw[pic actions] (#1:#3) arc(#1:#2:#3);}}}



\usepackage{array}
\setlength{\extrarowheight}{+.1cm}
\newdimen\digitwidth
\settowidth\digitwidth{9}
\def\divrule#1#2{
\noalign{\moveright#1\digitwidth
\vbox{\hrule width#2\digitwidth}}}






\DeclareMathOperator{\arccot}{arccot}
\DeclareMathOperator{\arcsec}{arcsec}
\DeclareMathOperator{\arccsc}{arccsc}

















%%This is to help with formatting on future title pages.
\newenvironment{sectionOutcomes}{}{}


\title{Complex Logarithm}

\begin{document}

\begin{abstract}
exponential inverse
\end{abstract}
\maketitle





We know that every nonzero Complex number can be wirtten as the product of a positive real number (a scalar) and a COmplex number on the unit circle.  Euler's Formula tells us how each Complex number on the unit circle can be written as a Complex exponential.


$\blacktriangleright$ \textbf{Euler's Formula}


\[   e^{i \theta} = \cos(\theta) + i \sin(\theta)         \]


And, we already know how to write and real number in exponential form: $r = e^{ln(r)}$.  

Combining these together, we get that every nonzero Complex number can be written in the form  $e^{ln(r)} \cdot e^{i t} = e^{ln(r) + i \theta}$




If $z = a + b \, i$, then $r = \sqrt{a^2 + b^2}$ and $\theta = \ArcTan\left(\frac{b}{a}\right) \pm 2 k \pi$, where $k \in \mathbb{N}$.



































\end{document}
