\documentclass{ximera}


\graphicspath{
  {./}
  {ximeraTutorial/}
  {basicPhilosophy/}
}

\newcommand{\mooculus}{\textsf{\textbf{MOOC}\textnormal{\textsf{ULUS}}}}

\usepackage{tkz-euclide}\usepackage{tikz}
\usepackage{tikz-cd}
\usetikzlibrary{arrows}
\tikzset{>=stealth,commutative diagrams/.cd,
  arrow style=tikz,diagrams={>=stealth}} %% cool arrow head
\tikzset{shorten <>/.style={ shorten >=#1, shorten <=#1 } } %% allows shorter vectors

\usetikzlibrary{backgrounds} %% for boxes around graphs
\usetikzlibrary{shapes,positioning}  %% Clouds and stars
\usetikzlibrary{matrix} %% for matrix
\usepgfplotslibrary{polar} %% for polar plots
\usepgfplotslibrary{fillbetween} %% to shade area between curves in TikZ
\usetkzobj{all}
\usepackage[makeroom]{cancel} %% for strike outs
%\usepackage{mathtools} %% for pretty underbrace % Breaks Ximera
%\usepackage{multicol}
\usepackage{pgffor} %% required for integral for loops



%% http://tex.stackexchange.com/questions/66490/drawing-a-tikz-arc-specifying-the-center
%% Draws beach ball
\tikzset{pics/carc/.style args={#1:#2:#3}{code={\draw[pic actions] (#1:#3) arc(#1:#2:#3);}}}



\usepackage{array}
\setlength{\extrarowheight}{+.1cm}
\newdimen\digitwidth
\settowidth\digitwidth{9}
\def\divrule#1#2{
\noalign{\moveright#1\digitwidth
\vbox{\hrule width#2\digitwidth}}}






\DeclareMathOperator{\arccot}{arccot}
\DeclareMathOperator{\arcsec}{arcsec}
\DeclareMathOperator{\arccsc}{arccsc}

















%%This is to help with formatting on future title pages.
\newenvironment{sectionOutcomes}{}{}


\title{Powers}

\begin{document}

\begin{abstract}
whole powers
\end{abstract}
\maketitle





Powers and roots of complex numbers have a strong geometric flavor to them.


$\blacktriangleright$ How do we calculate $(3 + 3\sqrt{3} \, i)^{\tfrac{5}{2}}$  ? \\





\subsection{Trigonometric Thinking}

\textbf{\textcolor{blue!55!black}{Step 1:}} Convert the base to trigonometric form.


\[
3 + 3\sqrt{3} \, i
\]

\[
= 6 \left( \frac{1}{2} + \frac{\sqrt{3}}{2} \, i \right) 
\]

\[
= 6 \left(\cos\left( \frac{\pi}{3} \right)+i \, \sin\left( \frac{\pi}{3} \right) \right)
\]

\[
= 6 \, e^{\tfrac{\pi}{3} \, i}
\]




\textbf{\textcolor{blue!55!black}{Step 2:}} The Angle

Find $\frac{\pi}{3}$ on the unit circle 




\begin{image}
\begin{tikzpicture}
  \begin{axis}[
            xmin=-1.1,xmax=1.1,ymin=-1.1,ymax=1.1,
            axis lines=center,
            width=4in,
            xtick={-1,1},
            ytick={-1,1},
            clip=false,
            unit vector ratio*=1 1 1,
            xlabel=$x$, ylabel=$y$,
            ticklabel style={font=\scriptsize},
            every axis y label/.style={at=(current axis.above origin),anchor=south},
            every axis x label/.style={at=(current axis.right of origin),anchor=west},
          ]        
          \addplot [smooth, domain=(0:360)] ({cos(x)},{sin(x)}); %% unit circle

          \addplot [textColor] plot coordinates {(0,0) (0.5,0.866)}; %% 60 degrees


          \addplot [textColor,smooth, domain=(0:60)] ({.15*cos(x)},{.15*sin(x)});



        \end{axis}
\end{tikzpicture}
\end{image}





We need to move the angle around. 

First, the $\tfrac{1}{2}$ part of the exponent means we need $\tfrac{1}{2}$ of the angle: $\frac{\pi}{6}$.





\begin{image}
\begin{tikzpicture}
  \begin{axis}[
            xmin=-1.1,xmax=1.1,ymin=-1.1,ymax=1.1,
            axis lines=center,
            width=4in,
            xtick={-1,1},
            ytick={-1,1},
            clip=false,
            unit vector ratio*=1 1 1,
            xlabel=$x$, ylabel=$y$,
            ticklabel style={font=\scriptsize},
            every axis y label/.style={at=(current axis.above origin),anchor=south},
            every axis x label/.style={at=(current axis.right of origin),anchor=west},
          ]        
          \addplot [smooth, domain=(0:360)] ({cos(x)},{sin(x)}); %% unit circle

          \addplot [textColor] plot coordinates {(0,0) (0.5,0.866)}; %% 60 degrees
          \addplot [penColor] plot coordinates {(0,0) (0.866,0.5)}; %% 30 degrees


          \addplot [textColor,smooth, domain=(0:60)] ({.15*cos(x)},{.15*sin(x)});



        \end{axis}
\end{tikzpicture}
\end{image}




Second, the $5$ in the exponent means we need to multiply the angle by $5$: $\frac{5\pi}{6}$.






\begin{image}
\begin{tikzpicture}
  \begin{axis}[
            xmin=-1.1,xmax=1.1,ymin=-1.1,ymax=1.1,
            axis lines=center,
            width=4in,
            xtick={-1,1},
            ytick={-1,1},
            clip=false,
            unit vector ratio*=1 1 1,
            xlabel=$x$, ylabel=$y$,
            ticklabel style={font=\scriptsize},
            every axis y label/.style={at=(current axis.above origin),anchor=south},
            every axis x label/.style={at=(current axis.right of origin),anchor=west},
          ]        
          \addplot [smooth, domain=(0:360)] ({cos(x)},{sin(x)}); %% unit circle

          \addplot [textColor] plot coordinates {(0,0) (0.5,0.866)}; %% 60 degrees
          \addplot [penColor] plot coordinates {(0,0) (0.866,0.5)}; %% 30 degrees
          \addplot [penColor2] plot coordinates {(0,0) (-0.866,0.5)}; %% 30 degrees

          \addplot [textColor,smooth, domain=(0:60)] ({.15*cos(x)},{.15*sin(x)});



        \end{axis}
\end{tikzpicture}
\end{image}






\textbf{\textcolor{blue!55!black}{Step 3:}} The Modulus


The $6$ also has the exponent.  Here we think just in terms of real numbers:  $6^{\tfrac{5}{2}}$.





\textbf{\textcolor{blue!55!black}{Step 4:}} Reassemble


\[
6^{\tfrac{5}{2}} \left(\cos\left( \frac{5\pi}{6} \right)+i \, \sin\left( \frac{5\pi}{6} \right) \right)
\]


\[
= 6^{\tfrac{5}{2}} \left( -\frac{\sqrt{3}}{2} + \frac{1}{2} \, i \right)
\]








\subsection{Exponential Thinking}



\textbf{\textcolor{blue!55!black}{Step 1:}} Convert the base to exponential form.


\[
3 + 3\sqrt{3} \, i
\]


$\sqrt{3^2 + (3\sqrt{3})^2} = \sqrt{9 + 9 \cdot 3} = \sqrt{36} = 6$ \\

$\arctan\left( \frac{3\sqrt{3}}{3}  \right) = \arctan\left( \sqrt{3}  \right) = \frac{\pi}{3}$



\[
= 6 \cdot e^{\tfrac{\pi}{3} \, i} = e^{ln(6) + \tfrac{\pi}{3} \, i}
\]



\textbf{\textcolor{blue!55!black}{Step 2:}} The Exponent



\[
= \left( e^{ln(6) + \tfrac{\pi}{3} \, i} \right)^{\tfrac{5}{2}} 
\]


\[
= e^{\tfrac{5}{2} (ln(6) + \tfrac{\pi}{3} \, i)}
\]



\[
= e^{\tfrac{5}{2} ln(6) + \tfrac{5\pi}{6} \, i}
\]


\[
= 6^{\tfrac{5}{2}} \cdot e^{\tfrac{5\pi}{6} \, i}
\]





\end{document}
