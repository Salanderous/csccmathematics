\documentclass{ximera}


\graphicspath{
  {./}
  {ximeraTutorial/}
  {basicPhilosophy/}
}

\newcommand{\mooculus}{\textsf{\textbf{MOOC}\textnormal{\textsf{ULUS}}}}

\usepackage{tkz-euclide}\usepackage{tikz}
\usepackage{tikz-cd}
\usetikzlibrary{arrows}
\tikzset{>=stealth,commutative diagrams/.cd,
  arrow style=tikz,diagrams={>=stealth}} %% cool arrow head
\tikzset{shorten <>/.style={ shorten >=#1, shorten <=#1 } } %% allows shorter vectors

\usetikzlibrary{backgrounds} %% for boxes around graphs
\usetikzlibrary{shapes,positioning}  %% Clouds and stars
\usetikzlibrary{matrix} %% for matrix
\usepgfplotslibrary{polar} %% for polar plots
\usepgfplotslibrary{fillbetween} %% to shade area between curves in TikZ
\usetkzobj{all}
\usepackage[makeroom]{cancel} %% for strike outs
%\usepackage{mathtools} %% for pretty underbrace % Breaks Ximera
%\usepackage{multicol}
\usepackage{pgffor} %% required for integral for loops



%% http://tex.stackexchange.com/questions/66490/drawing-a-tikz-arc-specifying-the-center
%% Draws beach ball
\tikzset{pics/carc/.style args={#1:#2:#3}{code={\draw[pic actions] (#1:#3) arc(#1:#2:#3);}}}



\usepackage{array}
\setlength{\extrarowheight}{+.1cm}
\newdimen\digitwidth
\settowidth\digitwidth{9}
\def\divrule#1#2{
\noalign{\moveright#1\digitwidth
\vbox{\hrule width#2\digitwidth}}}






\DeclareMathOperator{\arccot}{arccot}
\DeclareMathOperator{\arcsec}{arcsec}
\DeclareMathOperator{\arccsc}{arccsc}

















%%This is to help with formatting on future title pages.
\newenvironment{sectionOutcomes}{}{}


\title{Euler's Formula}

\begin{document}

\begin{abstract}
complex exponentials
\end{abstract}
\maketitle





\section{Rates of Change}

We have already seen that the derivative of $\sin(x)$ is $\cos(x)$ and derivative of $\cos(x)$ is $-\sin(x)$.

We need one more derivative.


Consider the function $f(t) = e^{r t}$

The average rate of change of $f(t)$ over the interval $[t, t+h]$ is given by 

\[   \frac{f(t+h)-f(t)}{h} = \frac{e^{r(t+h)}-e^{r t}}{h}         \]


We get closer to the derivative as $h \to 0$.


\begin{center}
\desmos{igck8p6w15}{400}{300}
\end{center}

Play with the value for $h$.

The applet makes it believable that the derivative of $f(t) = e^{r t}$  is $f'(t) = r e^{r t}$.





\section{Mean Value Theorem}



In Calculus you will study the \textbf{Mean Value Theorem}. One version follows this race track story.


Suppose two horses are racing.  They both line up at the starting line.  During the race, they both run at exactly the same speed at every second of the race.  Then the two horses are at exactly the same place on the track at every second of the race and end the race in a tie.




In function language, the story goes like this:

Suppose 

\begin{itemize}
\item $f$ and $g$ are two functions,
\item $f(0) = g(0)$, and
\item $f'(t) = g'(t)$  for all $t$.
\end{itemize}

Then $f(t)=g(t)$ for all $t$.   $f$ and $g$ are the same function.



If two functions start with the same value and then they have the exact same derivative everywhere, then they are the same function. \\









$\blacktriangleright$  When you study differential equations, this idea will lead to the fact that a first-order differential equation with an initial condition has only one solution. \\

Example: $y'(t) = k \cdot y(t)$ with $y(0)=1$


There is only one function that could be $y(t)$.

Well, as we shall see, we have two.  And, we can't.  That means our two functions must be the same function. \\








\section{Euler's Formula}

All of that was for real numbers, but we are going to take a slight leap of faith and extend it to complex numbers.




Consider two functions.


\begin{itemize}
	\item $f(t) = e^{i t}$
	\item $g(t) = \cos(t) + i \sin(t)$
\end{itemize}




$\blacktriangleright$   \textbf{Step 1:} Two functions have the same initial value.

\begin{itemize}
	\item $f(0) = e^0 = 1$
	\item $g(0) = \cos(0) + i \sin(0) = 1$
\end{itemize}





$\blacktriangleright$   \textbf{Step 2:} Two functions satisfy our differential equation: $y'(t) = i \cdot y(t)$



\begin{itemize}
	\item $f'(t) = i e^{i t} = i f(t)$
	\item $g'(t) = -\sin(t) + i \cos(t) = i g(t)$
\end{itemize}



We have two functions satisfying the same differential equation withthe same initial value.  They must, in fact, be the same function. \\




\begin{theorem} Euler's Formula


\[   e^{i t} = \cos(t) + i \sin(t)         \]


\end{theorem}




Euler's Formula is a bridge between exponential functions and trigonometric functions.





\end{document}
