\documentclass{ximera}


\graphicspath{
  {./}
  {ximeraTutorial/}
  {basicPhilosophy/}
}

\newcommand{\mooculus}{\textsf{\textbf{MOOC}\textnormal{\textsf{ULUS}}}}

\usepackage{tkz-euclide}\usepackage{tikz}
\usepackage{tikz-cd}
\usetikzlibrary{arrows}
\tikzset{>=stealth,commutative diagrams/.cd,
  arrow style=tikz,diagrams={>=stealth}} %% cool arrow head
\tikzset{shorten <>/.style={ shorten >=#1, shorten <=#1 } } %% allows shorter vectors

\usetikzlibrary{backgrounds} %% for boxes around graphs
\usetikzlibrary{shapes,positioning}  %% Clouds and stars
\usetikzlibrary{matrix} %% for matrix
\usepgfplotslibrary{polar} %% for polar plots
\usepgfplotslibrary{fillbetween} %% to shade area between curves in TikZ
\usetkzobj{all}
\usepackage[makeroom]{cancel} %% for strike outs
%\usepackage{mathtools} %% for pretty underbrace % Breaks Ximera
%\usepackage{multicol}
\usepackage{pgffor} %% required for integral for loops



%% http://tex.stackexchange.com/questions/66490/drawing-a-tikz-arc-specifying-the-center
%% Draws beach ball
\tikzset{pics/carc/.style args={#1:#2:#3}{code={\draw[pic actions] (#1:#3) arc(#1:#2:#3);}}}



\usepackage{array}
\setlength{\extrarowheight}{+.1cm}
\newdimen\digitwidth
\settowidth\digitwidth{9}
\def\divrule#1#2{
\noalign{\moveright#1\digitwidth
\vbox{\hrule width#2\digitwidth}}}






\DeclareMathOperator{\arccot}{arccot}
\DeclareMathOperator{\arcsec}{arcsec}
\DeclareMathOperator{\arccsc}{arccsc}

















%%This is to help with formatting on future title pages.
\newenvironment{sectionOutcomes}{}{}


\title{Bridge}

\begin{document}

\begin{abstract}
exp, trig, hyp
\end{abstract}
\maketitle





What can we do with Euler's Formula?



$e^{i t} = \cos(t) + i \, \sin(t)$

Substituting $-t$ for $t$ gives $e^{i (-t)} = \cos(-t) + i \, \sin(-t) = \cos(t) - i \, \sin(t)$, because $\cos(t)$ is an even funciton and $\sin(t)$ is an odd function.



That gives us 

\[   e^{i t}  + e^{-i t} = 2 \cos(t)    \]


\[   \frac{e^{i t}  + e^{-i t}}{2} = \cos(t)    \]


It also gives us 



\[   e^{i t}  - e^{-i t} = 2 i \sin(t)    \]


\[   \frac{e^{i t}  + e^{-i t}}{2 i} = \sin(t)    \]





We have also seen that 


\[  \cosh(t)  = \frac{e^t + e^{-t}}{2}      \,   \text{ and } \,  \sinh(t)  = \frac{e^t - e^{-t}}{2}     \]



This tells us that 

\[  \cos(t) =  \frac{e^{i t} + e^{-i t}}{2} = \cosh(i t)      \]




\[  \sin(t) =  \frac{e^{i t} - e^{-i t}}{2 i} = -i \sinh(i t)      \]






If we work with Complex numbers, then trigonometric and hyperbolic funcitons are expressible in terms of exponential functions.


























\end{document}
