\documentclass{ximera}


\graphicspath{
  {./}
  {ximeraTutorial/}
  {basicPhilosophy/}
}

\newcommand{\mooculus}{\textsf{\textbf{MOOC}\textnormal{\textsf{ULUS}}}}

\usepackage{tkz-euclide}\usepackage{tikz}
\usepackage{tikz-cd}
\usetikzlibrary{arrows}
\tikzset{>=stealth,commutative diagrams/.cd,
  arrow style=tikz,diagrams={>=stealth}} %% cool arrow head
\tikzset{shorten <>/.style={ shorten >=#1, shorten <=#1 } } %% allows shorter vectors

\usetikzlibrary{backgrounds} %% for boxes around graphs
\usetikzlibrary{shapes,positioning}  %% Clouds and stars
\usetikzlibrary{matrix} %% for matrix
\usepgfplotslibrary{polar} %% for polar plots
\usepgfplotslibrary{fillbetween} %% to shade area between curves in TikZ
\usetkzobj{all}
\usepackage[makeroom]{cancel} %% for strike outs
%\usepackage{mathtools} %% for pretty underbrace % Breaks Ximera
%\usepackage{multicol}
\usepackage{pgffor} %% required for integral for loops



%% http://tex.stackexchange.com/questions/66490/drawing-a-tikz-arc-specifying-the-center
%% Draws beach ball
\tikzset{pics/carc/.style args={#1:#2:#3}{code={\draw[pic actions] (#1:#3) arc(#1:#2:#3);}}}



\usepackage{array}
\setlength{\extrarowheight}{+.1cm}
\newdimen\digitwidth
\settowidth\digitwidth{9}
\def\divrule#1#2{
\noalign{\moveright#1\digitwidth
\vbox{\hrule width#2\digitwidth}}}






\DeclareMathOperator{\arccot}{arccot}
\DeclareMathOperator{\arcsec}{arcsec}
\DeclareMathOperator{\arccsc}{arccsc}

















%%This is to help with formatting on future title pages.
\newenvironment{sectionOutcomes}{}{}


\title{Roots}

\begin{document}

\begin{abstract}
roots of complex
\end{abstract}
\maketitle





Roots of any complex number.  Or real








Euler's Formula tells us that $e^{i \theta} = \cos(\theta) + i \, \sin(i \theta)$, which describes all of the points on the unit circle.

We also know that every nonzero Complex number is a scalar (positive real number) times a number on the unit circle:  $r e^{i \theta}$.

We also know that every positive real number can be written as an exponential: $r = e^{ln(r)}$.

Therefore, every Complex number can be written as a complex exponential: $ e^{ln(r) + i \theta}$


\[   a + b \, i =    e^{ln(r) + i \theta}   \]

\[   \text{ where } \,   r = |a + b \, i| = \sqrt{a^2 + b^2}   \, \text{ and } \,   \theta = ArcTan\left (\frac{b}{a} \right) \pm 2k\pi\]





\section{$n^{th}$ roots of Unity}


Focusing on the Complex numbers on the unit circle, they can be written in the form $e^{i \theta}$.


This would include all of the solutions to $z^n = 1$, with $n$ a Natural number - the roots of unity.




If we are looking for solutions to $(e^{i \theta})^n = 1$, then we need $n \cdot \theta = 2k\pi$, for some $k \in \mathbb{N}$.


Therefore, the $n^{th}$ roots of $1$ are equally spaced on the unit circle.  Begin with $1$, then go $\frac{1}{n^{th}}$ around the circle, then another, then another, until you get back to the beginning.




\begin{example}  Cube roots of Unity

There are $3$ third roots of unity.

$1$ is the first.

Then move $\frac{1}{3}$ of a circle ($120^{\circ}$) to $e^{\tfrac{2 \pi}{3} i} = \cos\left(\frac{2 \pi}{3}\right) + i \, \sin\left(\frac{2 \pi}{3}\right) = \frac{-1}{2} + i \, \frac{\sqrt{3}}{2}$

Then move another $\frac{1}{3}$ of a circle ($240^{\circ}$) to $e^{\tfrac{4 \pi}{3} i} = \cos\left(\frac{4 \pi}{3}\right) + i \, \sin\left(\frac{4 \pi}{3}\right) = \frac{-1}{2} - i \, \frac{\sqrt{3}}{2}$



Then move another $\frac{1}{3}$ of a circle, which returns you to $1$.




\end{example}











\begin{example}  Fourth roots of Unity

There are $4$ fourth roots of unity.

$1$ is the first.

A quarter circle gets you to $i$. 

Another quarter circle gets you to $-1$. 

Another quarter circle gets you to $-i$. 

Another quarter circle gets you back to $1$.


\end{example}


















\end{document}
