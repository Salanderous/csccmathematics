\documentclass{ximera}


\graphicspath{
  {./}
  {ximeraTutorial/}
  {basicPhilosophy/}
}

\newcommand{\mooculus}{\textsf{\textbf{MOOC}\textnormal{\textsf{ULUS}}}}

\usepackage{tkz-euclide}\usepackage{tikz}
\usepackage{tikz-cd}
\usetikzlibrary{arrows}
\tikzset{>=stealth,commutative diagrams/.cd,
  arrow style=tikz,diagrams={>=stealth}} %% cool arrow head
\tikzset{shorten <>/.style={ shorten >=#1, shorten <=#1 } } %% allows shorter vectors

\usetikzlibrary{backgrounds} %% for boxes around graphs
\usetikzlibrary{shapes,positioning}  %% Clouds and stars
\usetikzlibrary{matrix} %% for matrix
\usepgfplotslibrary{polar} %% for polar plots
\usepgfplotslibrary{fillbetween} %% to shade area between curves in TikZ
\usetkzobj{all}
\usepackage[makeroom]{cancel} %% for strike outs
%\usepackage{mathtools} %% for pretty underbrace % Breaks Ximera
%\usepackage{multicol}
\usepackage{pgffor} %% required for integral for loops



%% http://tex.stackexchange.com/questions/66490/drawing-a-tikz-arc-specifying-the-center
%% Draws beach ball
\tikzset{pics/carc/.style args={#1:#2:#3}{code={\draw[pic actions] (#1:#3) arc(#1:#2:#3);}}}



\usepackage{array}
\setlength{\extrarowheight}{+.1cm}
\newdimen\digitwidth
\settowidth\digitwidth{9}
\def\divrule#1#2{
\noalign{\moveright#1\digitwidth
\vbox{\hrule width#2\digitwidth}}}






\DeclareMathOperator{\arccot}{arccot}
\DeclareMathOperator{\arcsec}{arcsec}
\DeclareMathOperator{\arccsc}{arccsc}

















%%This is to help with formatting on future title pages.
\newenvironment{sectionOutcomes}{}{}


\title{Roots}

\begin{document}

\begin{abstract}
roots of complex
\end{abstract}
\maketitle




$\blacktriangleright$ \textbf{Different Stories}

In the real numbers, we have been following an off-balanced story for roots.

Let $r$ be a real number.

Sometimes, there are two square roots of $r$ and sometimes there are none.  It depends on the sign of $r$. The Complex Numbers have repaired this story.  There are always two solutions to the equation $z^2 = r$.  Even $0$ works when we bring in multiplicities.

The Complex Numbers have settled the issue for all roots. 

Let $r$ be a real number.
Let $n$ be a natural number.

There are always $n$ $n^{th}$ roots of $r$ in the Complex Numbers. \\


In the real numbers, we can express every positive number in terms of $e$: $r = e^{ln(r)}$ \\


And, we can express the whole story of roots in terms of $e$.

\[
\sqrt[n]{r} = r^{\tfrac{1}{n}}= e^{\tfrac{1}{n} ln(r)}
\]






We want to do the same for the Complex story. \\


Euler's Formula tells us that $e^{i \theta} = \cos(\theta) + i \, \sin(\theta)$, which describes all of the points on the unit circle.

We also know that every nonzero complex number is a scalar (positive real number) times a number on the unit circle:  $r e^{i \theta}$.

We also know that every positive real number can be written as an exponential: $r = e^{ln(r)}$.

Therefore, every complex number can be written as a complex exponential: $ e^{ln(r) + i \theta}$


\[   a + b \, i =    e^{ln(r) + i \theta}   \]

\[   \text{ where } \,   r = |a + b \, i| = \sqrt{a^2 + b^2}   \, \text{ and } \,   \theta = \arctan\left (\frac{b}{a} \right) \pm 2k\pi\]





\section{$n^{th}$ roots of Unity}


Focusing on the Complex Numbers on the unit circle, they can be written in the form $e^{i \theta}$.


This would include all of the solutions to $z^n = 1$, with $n$ a natural number (the roots of unity).




If we are looking for solutions to $(e^{i \theta})^n = 1$, then we need $n \cdot \theta = 2k\pi$, for some $k \in \mathbb{N}$.


Therefore, the $n^{th}$ roots of $1$ are equally spaced on the unit circle.  



\[
1=e^{\tfrac{0\cdot2\pi}{n}\,i}, e^{\tfrac{1\cdot2\pi}{n}\,i}, e^{\tfrac{2\cdot2\pi}{n}\,i}, e^{\tfrac{3\cdot2\pi}{n}\,i}, \cdots, e^{\tfrac{(n-1)\cdot2\pi}{n}\,i}, e^{\tfrac{n\cdot2\pi}{n}\,i}=1
\]



Begin with $1$, then go $\frac{1}{n^{th}}$ around the circle, then another, then another, until you get back to the beginning.




\begin{example}  Cube roots of Unity

There are $3$ third roots of unity.

$1$ is the first.

Then move $\frac{1}{3}$ of a circle ($120^{\circ}$) to $e^{\tfrac{2 \pi}{3} i} = \cos\left(\frac{2 \pi}{3}\right) + i \, \sin\left(\frac{2 \pi}{3}\right) = \frac{-1}{2} + i \, \frac{\sqrt{3}}{2}$

Then move another $\frac{1}{3}$ of a circle ($240^{\circ}$) to $e^{\tfrac{4 \pi}{3} i} = \cos\left(\frac{4 \pi}{3}\right) + i \, \sin\left(\frac{4 \pi}{3}\right) = \frac{-1}{2} - i \, \frac{\sqrt{3}}{2}$



Then move another $\frac{1}{3}$ of a circle, which returns you to $1$.




\end{example}











\begin{example}  Fourth roots of Unity

There are $4$ fourth roots of unity.

$1$ is the first.

A quarter circle gets you to $e^{\tfrac{\pi}{2}} = i$. 

Another quarter circle gets you to $e^{\pi} = -1$. 

Another quarter circle gets you to $e^{\tfrac{3\pi}{2}} = -i$. 

Another quarter circle gets you back to $1$.


\end{example}






Our ideas of numbers have grown as we have used them. \\

When we were counting, we used the whole numbers.

The whole numbers were not enough when we started spliting things apart.  We filled in the gaps with rational numbers, fractions.

The rational numbers were not enough when we started measuring lengths. We filled in the gaps with irrational numbers.

Now that we are thinking of solutions to equations, we are discovering that the real numbers are missing numbers.  We have filled in the gaps with the Complex Numbers.

The difference here is that filling in the gaps before maintained the 1-dimensional number line.  Filling in the gaps to equations has revealed that our numbers are really 2-dimensional.  It is a plane of numbers rather than a line of numbers.

This is added A LOT of geometry to our number descriptions.

We are now blurrying the distinction between geometry and arithmetic.

\begin{itemize}
\item We think of complex numbers as dots in the Cartesian plane:  $(a, b)$
\item We think of complex numbers as vectors in the Cartesian plane:  $\langle a, b \rangle$
\item We think of complex numbers as sums of rectangular directions:  $a + b \, i$
\item We think of complex numbers as a circular (polar) directions:  $(r, \theta)$
\item We think of complex numbers as a extensions of numbers on the unit circle:  $r(\cos(\theta)+i \, \sin(\theta))$
\item We think of complex numbers as a complex exponentials:  $r\cdot e^{i\, \theta} = e^{ln(r)+i\, \theta}$
\end{itemize}


It turns out that the Complex Numbers have all of the solutions to all of the equations they can make.  The Complex Numbers are ``complete''.  That means we can work confidently now knowing that we are not longer missing any numbers that we need.






We can now write a famous identity that ties all of our important constants together ($e$, $i$, $\pi$, $1$, and $0$).

\begin{formula} \textbf{\textcolor{purple!85!blue}{Euler's Identity}}



\[  e^{i \, \pi} + 1 = 0  \]




\end{formula}






\begin{center}
\textbf{\textcolor{green!50!black}{ooooo=-=-=-=-=-=-=-=-=-=-=-=-=ooOoo=-=-=-=-=-=-=-=-=-=-=-=-=ooooo}} \\

more examples can be found by following this link\\ \link[More Examples of Complex Exponentials]{https://ximera.osu.edu/csccmathematics/precalculus2/precalculus2/complexExponentials/examples/exampleList}

\end{center}



\end{document}
