\documentclass{ximera}


\graphicspath{
  {./}
  {ximeraTutorial/}
  {basicPhilosophy/}
}

\newcommand{\mooculus}{\textsf{\textbf{MOOC}\textnormal{\textsf{ULUS}}}}

\usepackage{tkz-euclide}\usepackage{tikz}
\usepackage{tikz-cd}
\usetikzlibrary{arrows}
\tikzset{>=stealth,commutative diagrams/.cd,
  arrow style=tikz,diagrams={>=stealth}} %% cool arrow head
\tikzset{shorten <>/.style={ shorten >=#1, shorten <=#1 } } %% allows shorter vectors

\usetikzlibrary{backgrounds} %% for boxes around graphs
\usetikzlibrary{shapes,positioning}  %% Clouds and stars
\usetikzlibrary{matrix} %% for matrix
\usepgfplotslibrary{polar} %% for polar plots
\usepgfplotslibrary{fillbetween} %% to shade area between curves in TikZ
\usetkzobj{all}
\usepackage[makeroom]{cancel} %% for strike outs
%\usepackage{mathtools} %% for pretty underbrace % Breaks Ximera
%\usepackage{multicol}
\usepackage{pgffor} %% required for integral for loops



%% http://tex.stackexchange.com/questions/66490/drawing-a-tikz-arc-specifying-the-center
%% Draws beach ball
\tikzset{pics/carc/.style args={#1:#2:#3}{code={\draw[pic actions] (#1:#3) arc(#1:#2:#3);}}}



\usepackage{array}
\setlength{\extrarowheight}{+.1cm}
\newdimen\digitwidth
\settowidth\digitwidth{9}
\def\divrule#1#2{
\noalign{\moveright#1\digitwidth
\vbox{\hrule width#2\digitwidth}}}






\DeclareMathOperator{\arccot}{arccot}
\DeclareMathOperator{\arcsec}{arcsec}
\DeclareMathOperator{\arccsc}{arccsc}

















%%This is to help with formatting on future title pages.
\newenvironment{sectionOutcomes}{}{}


\title{Roots of Unity}

\begin{document}

\begin{abstract}
unit circle
\end{abstract}
\maketitle




We have seen that every Complex number can be written as $r \cdot (\cos(\theta) + i \, \sin(\theta))$.  This is a scalar times a Complex number on the unit circle.

We have also seen that if $z = r \cdot (\cos(\theta) + i \, \sin(\theta))$, then $z^n = r^n \cdot (\cos(n\theta) + i \, \sin(n\theta))$. Raising complex numbers to powers is accomplished by raising the modulus to the power and then multiplying the angle.


$\blacktriangleright$  \textbf{Roots of Unity}

Roots or unity refer to solutions to equations of the form $z^n = 1$.




For instance, the square roots of unity are the solutions to $z^2 = 1$.  The two solutions are $1$ and $-1$.


For instance, the $4^{th}$ roots of unity are the solutions to $z^4 = 1$.  The four solutions are $1$, $-1$, $i$, and $-i$.




$1$ is always a root of unity for any power. Then, there are $n-1$ other $n^{th}$ roots of unity.



If $z = r \cdot (\cos(\theta) + i \, \sin(\theta))$ is going to be a root of unity, then $z^n = r^n \cdot (\cos(n\theta) + i \, \sin(n\theta)) = 1$, which means $r=1$.

$n^{th}$ roots of unity all look like $\cos(\theta) + i \, \sin(\theta)$.  They all lie on the unit circle.



In addition, if $\cos(n\theta) + i \, \sin(n\theta) = 1$, then $n \theta = 2 k \pi$ (a mulitple of $2 \pi$), because $1$ is on the positive real axis.


Therefore, $\theta = \frac{2 k \pi}{n}$  












\begin{example}  $4^{th}$ roots of $1$

We need multiples of $\frac{2 \pi}{4} = \frac{\pi}{2}$.

\begin{itemize}
\item $\theta = \frac{\pi}{2}$
\item $\theta = \frac{2 \pi}{2} = \pi$
\item $\theta = \frac{3 \pi}{2}$
\item $\theta = \frac{4 \pi}{2} = 2 \pi$
\end{itemize}



The fourth roots of unity are:
\begin{itemize}
\item $\cos\left(\frac{\pi}{2}\right) + i \, \sin\left(\frac{\pi}{2}\right) = i$
\item $\cos(\pi) + i \, \sin(\pi) = -1$
\item $\cos\left(\frac{3 \pi}{2}\right) + i \, \sin\left(\frac{3 \pi}{2}\right) = -i$
\item $\cos(2 \pi) + i \, \sin(2 \pi) = 1$
\end{itemize}





The roots of unity are spread out equidistant along the unit circle.






\end{example}










\begin{example}  Cube roots of $1$

We need multiples of $\frac{2 \pi}{3}$.

\begin{itemize}
\item $\theta = \frac{2 \pi}{3}$
\item $\theta = \frac{4 \pi}{3}$
\item $\theta = \frac{6 \pi}{3} = 2 \pi$
\end{itemize}



The fourth roots of unity are:
\begin{itemize}
\item $\cos\left(\frac{2 \pi}{3}\right) + i \, \sin\left(\frac{2 \pi}{3}\right) = \frac{1}{2} + \frac{\sqrt{3}}{2} \, i$
\item $\cos\left(\frac{4 \pi}{3}\right) + i \, \sin\left(\frac{4 \pi}{3}\right) = \frac{1}{2} - \frac{\sqrt{3}}{2} \, i$
\item $\cos(2 \pi) + i \, \sin(2 \pi) = 1$
\end{itemize}



\end{example}








\begin{question} 


$\cos\left( \frac{\pi}{6} \right) + i \, \sin\left( \frac{\pi}{6} \right)$ is which root of unity?

\begin{multipleChoice}
\choice {sixth}
\choice {eighth}
\choice {tenth}
\choice[correct] {twelfeth}
\end{multipleChoice}

\end{question}







\begin{example}


Using the quadratic formula, we know that $\frac{-3 + \sqrt{11} \, i}{2}$ is a root of $x^2 + 3x + 5$. \\

We know that $\frac{-3 + \sqrt{11} \, i}{2}$ can be written in the form $r (\cos(\theta) + i \, \sin(\theta))$.

$r$ is the modulus of $\frac{-3 + \sqrt{11} \, i}{2}$.

$r = \answer{\sqrt{5}}$ 

The angle for $\frac{-3 + \sqrt{11} \, i}{2}$ is in which quadrant?


\begin{multipleChoice}
\choice {I}
\choice[correct] {II}
\choice {III}
\choice {IV}
\end{multipleChoice}


$\arctan\left( -\frac{\sqrt{11}}{3} \right)$ gives the angle, but in quadrant IV.    What must be added to $\arctan\left( -\frac{\sqrt{11}}{3} \right)$ to get $\theta$?

Add $\answer{\pi}$


\[   \frac{-3 + \sqrt{11} \, i}{2} = \sqrt{5} \left( \cos\left( \arctan\left( -\frac{\sqrt{11}}{3} \right) + \pi \right) + i \, \sin\left( \arctan\left( -\frac{\sqrt{11}}{3} \right) +\pi  \right) \right)     \]


\end{example}


As long as we are here...\\


\begin{observation}

In the example above, $x^2 + 3x + 5$ is a polynomial with real coefficients.  Therefore, if $\frac{-3 + \sqrt{11} \, i}{2}$ is a root, then so is $\frac{-3 - \sqrt{11} \, i}{2}$.



Therefore, $x^2 + 3x + 5$ factors as



\[ \left( x - \frac{-3 - \sqrt{11} \, i}{2} \right)  \left( x - \frac{-3 + \sqrt{11} \, i}{2} \right)    \]


The constant terms have to be equal, which tells us that 


\[    \left( \frac{-3 - \sqrt{11} \, i}{2} \right)  \left( \frac{-3 + \sqrt{11} \, i}{2} \right)  = 5    \]




The constant term of the polynomial is the product of the roots, which are conjugates.  The constant term is the square of the modulus of either root.





\end{observation}













\begin{center}
\textbf{\textcolor{green!50!black}{ooooo=-=-=-=-=-=-=-=-=-=-=-=-=ooOoo=-=-=-=-=-=-=-=-=-=-=-=-=ooooo}} \\

more examples can be found by following this link\\ \link[More Examples of Complex Exponentials]{https://ximera.osu.edu/csccmathematics/precalculus2/precalculus2/complexExponentials/examples/exampleList}

\end{center}



\end{document}
