\documentclass{ximera}


\graphicspath{
  {./}
  {ximeraTutorial/}
  {basicPhilosophy/}
}

\newcommand{\mooculus}{\textsf{\textbf{MOOC}\textnormal{\textsf{ULUS}}}}

\usepackage{tkz-euclide}\usepackage{tikz}
\usepackage{tikz-cd}
\usetikzlibrary{arrows}
\tikzset{>=stealth,commutative diagrams/.cd,
  arrow style=tikz,diagrams={>=stealth}} %% cool arrow head
\tikzset{shorten <>/.style={ shorten >=#1, shorten <=#1 } } %% allows shorter vectors

\usetikzlibrary{backgrounds} %% for boxes around graphs
\usetikzlibrary{shapes,positioning}  %% Clouds and stars
\usetikzlibrary{matrix} %% for matrix
\usepgfplotslibrary{polar} %% for polar plots
\usepgfplotslibrary{fillbetween} %% to shade area between curves in TikZ
\usetkzobj{all}
\usepackage[makeroom]{cancel} %% for strike outs
%\usepackage{mathtools} %% for pretty underbrace % Breaks Ximera
%\usepackage{multicol}
\usepackage{pgffor} %% required for integral for loops



%% http://tex.stackexchange.com/questions/66490/drawing-a-tikz-arc-specifying-the-center
%% Draws beach ball
\tikzset{pics/carc/.style args={#1:#2:#3}{code={\draw[pic actions] (#1:#3) arc(#1:#2:#3);}}}



\usepackage{array}
\setlength{\extrarowheight}{+.1cm}
\newdimen\digitwidth
\settowidth\digitwidth{9}
\def\divrule#1#2{
\noalign{\moveright#1\digitwidth
\vbox{\hrule width#2\digitwidth}}}






\DeclareMathOperator{\arccot}{arccot}
\DeclareMathOperator{\arcsec}{arcsec}
\DeclareMathOperator{\arccsc}{arccsc}

















%%This is to help with formatting on future title pages.
\newenvironment{sectionOutcomes}{}{}


\title{Side Lengths}

\begin{document}

\begin{abstract}
pythagorean theorem
\end{abstract}
\maketitle




The \textbf{Unit Circle} is the circle on the Cartesian plane, centered at the origin of radius $1$.


It is the set of points whose distance from the origin is $1$.


We could describe these points with the Complex Number equation $| z | = 1$.



\begin{image}
\begin{tikzpicture}
  \begin{axis}[
            xmin=-1.1,xmax=1.1,ymin=-1.1,ymax=1.1,
            axis lines=center,
            width=4in,
            xtick={-1,1},
            ytick={-1,1},
            clip=false,
            unit vector ratio*=1 1 1,
            xlabel=$x$, ylabel=$y$,
            ticklabel style={font=\scriptsize},
            every axis y label/.style={at=(current axis.above origin),anchor=south},
            every axis x label/.style={at=(current axis.right of origin),anchor=west},
          ]        
          \addplot [smooth, domain=(0:360)] ({cos(x)},{sin(x)}); %% unit circle

          \addplot [textColor] plot coordinates {(0,0) (.766,.643)}; %% 40 degrees

          \addplot [ultra thick,penColor] plot coordinates {(.766,0) (.766,.643)}; %% 40 degrees
          \addplot [ultra thick,penColor2] plot coordinates {(0,0) (.766,0)}; %% 40 degrees
          
          %\addplot [ultra thick,penColor3] plot coordinates {(1,0) (1,.839)}; %% 40 degrees          

          \addplot [textColor,smooth, domain=(0:40)] ({.15*cos(x)},{.15*sin(x)});
          %\addplot [very thick,penColor] plot coordinates {(0,0) (.766,.643)}; %% sector
          %\addplot [very thick,penColor] plot coordinates {(0,0) (1,0)}; %% sector
          %\addplot [very thick, penColor, smooth, domain=(0:40)] ({cos(x)},{sin(x)}); %% sector
          \node at (axis cs:.15,.07) [anchor=west] {$\theta$};
          \node[penColor, rotate=-90] at (axis cs:.84,.322) {$\sin(\theta)$};
          \node[penColor2] at (axis cs:.383,0) [anchor=north] {$\cos(\theta)$};
          %\node[penColor3, rotate=-90] at (axis cs:1.06,.322) {$\tan(\theta)$};

          \addplot[color=black,fill=black,only marks,mark=*] coordinates{(0.766,0.643)}; 

        \end{axis}
\end{tikzpicture}
\end{image}


We can describe these points with the Cartesian equation $x^2 + y^2 = 1$, the equation for a circle of radius $1$. Also, known as the \textbf{Pythagorean Theorem}.\\



\begin{theorem}   \textbf{\textcolor{green!50!black}{Pythagorean Theorem}} \\

Let $a$ and $b$ be the lengths of the legs of a right triangle. Let $c$ be the length of the hypotenuse. These three numbers satisfy the following equation

\[
a^2 + b^2 = c^2
\]

\end{theorem}




Each point on the Unit Circle sits at the end of a radius and that radius makes an angle with the positive $x$-axis, which is labelled $\theta$ in the diagram.




We have defined the functions sine and cosine as the coordinates of the points on the Unit Circle and the values of these functions depend on the angle $\theta$.

\[    ( \sin(\theta), \cos(\theta) ) \]




And, we know that any right triangle is similar to one of these right triangles defined on the Unit Circle.


This allows us to describe the sides of every right triangle.











\begin{example} 




Determine the length of the side $a$. 






\begin{image}[3in]
    \begin{tikzpicture}[line cap=round]



  \draw [ultra thick] (0,0) -- (3,0);
  \draw [ultra thick] (3,0) -- (3,2);
  \draw [ultra thick] (0,0) -- (3,2);

  \draw [thin] (2.9,0) -- (2.9,0.1);
  \draw [thin] (2.9,0.1) -- (3,0.1);


  %\draw [rotate=0] (0.4,0.13) node {\scriptsize{$\theta$}};


  \draw (3.05,1) node  [anchor=west]{\scriptsize{$4$}};
  \draw (1.2,1) node {\scriptsize{$7$}};
  \draw (2,-0.2) node {\scriptsize{$a$}};





    \end{tikzpicture}
  \end{image}



\begin{explanation}

The sides of a right triangle satisfy the Pythagorean Theorem: $4^2 + a^2 = 7^2$.


\[  16 + a^2 = 49       \]

\[  a^2 = 49 - 16 = 33     \]

\[  a = \sqrt{33} \, \text{ or } \, a = -\sqrt{33}    \]



Since sides of triangles have a positive length, $a = \sqrt{33}$.



\end{explanation}




\end{example}

















\begin{example}


Determine the length of the side $c$. 






\begin{image}[3in]
    \begin{tikzpicture}[line cap=round]



  \draw [ultra thick] (0,0) -- (3,0);
  \draw [ultra thick] (3,0) -- (3,2);
  \draw [ultra thick] (0,0) -- (3,2);

  \draw [thin] (2.9,0) -- (2.9,0.1);
  \draw [thin] (2.9,0.1) -- (3,0.1);


  %\draw [rotate=0] (0.4,0.13) node {\scriptsize{$\theta$}};


  \draw (3.05,1) node  [anchor=west]{\scriptsize{$4$}};
  \draw (1.2,1) node {\scriptsize{$c$}};
  \draw (2,-0.2) node {\scriptsize{$5$}};





    \end{tikzpicture}
  \end{image}



\begin{explanation}

The sides of a right triangle satisfy the Pythagorean Theorem: $4^2 + 5^2 = c^2$.


\[  16 + 25 = c^2       \]

\[  41 = c^2     \]

\[  c = \sqrt{41} \, \text{ or } \, c = -\sqrt{41}    \]



Since sides of triangles have a positive length, $c = \sqrt{41}$.



\end{explanation}




\end{example}






















\section{Algebra}

We have defined the functions sine and cosine as the coordinates of the points on the Unit Circle and the values of these functions depend on the angle $\theta$.

\[    ( \sin(\theta), \cos(\theta) ) \]


These coordinates must satisfy the equation for the Unit Circle. 



\[    ( \sin(\theta) )^2 + ( \cos(\theta) )^2 = 1 \]



From this, we get a list of identities:



\begin{observation}


\begin{itemize}
\item $( \sin(\theta) )^2 + ( \cos(\theta) )^2 = 1$
\item $( \sin(\theta) )^2 =  1 - ( \cos(\theta) )^2$
\item $( \cos(\theta) )^2 =  1 - ( \sin(\theta) )^2$
\item $( \sin(\theta) )^2 =  1 - ( \cos(\theta) )^2 = (1 + \cos(\theta))(1 - \cos(\theta))$
\item $( \cos(\theta) )^2 =  1 - ( \sin(\theta) )^2 = (1 + \sin(\theta))(1 - \sin(\theta))$
\end{itemize}


\end{observation}













\begin{example} Alternate Forms


Transform $\frac{\cos(t)}{1 - \sin(t)}$ into $\frac{1 + \sin(t)}{\cos(t)}$



\begin{explanation}

\begin{align*}
\frac{\cos(t)}{1 - \sin(t)}  &  = &  \frac{\cos(t)}{1 - \sin(t)}  \cdot \frac{1 + \sin(t)}{1 + \sin(t)}  \\
                           &  = &  \frac{\cos(t)(1 + \sin(t))}{1 - \sin(t)(1 + \sin(t))}  \\
                           &  = &  \frac{\cos(t)(1 + \sin(t))}{1 - (\sin(t))^2}  \\
                           &  = &  \frac{\cos(t)(1 + \sin(t))}{(\cos(t))^2}  \\
                           &  = &  \frac{(1 + \sin(t))}{\cos(t)}  
\end{align*}



\end{explanation}




\end{example}













\end{document}
