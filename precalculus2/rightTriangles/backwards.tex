\documentclass{ximera}


\graphicspath{
  {./}
  {ximeraTutorial/}
  {basicPhilosophy/}
}

\newcommand{\mooculus}{\textsf{\textbf{MOOC}\textnormal{\textsf{ULUS}}}}

\usepackage{tkz-euclide}\usepackage{tikz}
\usepackage{tikz-cd}
\usetikzlibrary{arrows}
\tikzset{>=stealth,commutative diagrams/.cd,
  arrow style=tikz,diagrams={>=stealth}} %% cool arrow head
\tikzset{shorten <>/.style={ shorten >=#1, shorten <=#1 } } %% allows shorter vectors

\usetikzlibrary{backgrounds} %% for boxes around graphs
\usetikzlibrary{shapes,positioning}  %% Clouds and stars
\usetikzlibrary{matrix} %% for matrix
\usepgfplotslibrary{polar} %% for polar plots
\usepgfplotslibrary{fillbetween} %% to shade area between curves in TikZ
\usetkzobj{all}
\usepackage[makeroom]{cancel} %% for strike outs
%\usepackage{mathtools} %% for pretty underbrace % Breaks Ximera
%\usepackage{multicol}
\usepackage{pgffor} %% required for integral for loops



%% http://tex.stackexchange.com/questions/66490/drawing-a-tikz-arc-specifying-the-center
%% Draws beach ball
\tikzset{pics/carc/.style args={#1:#2:#3}{code={\draw[pic actions] (#1:#3) arc(#1:#2:#3);}}}



\usepackage{array}
\setlength{\extrarowheight}{+.1cm}
\newdimen\digitwidth
\settowidth\digitwidth{9}
\def\divrule#1#2{
\noalign{\moveright#1\digitwidth
\vbox{\hrule width#2\digitwidth}}}






\DeclareMathOperator{\arccot}{arccot}
\DeclareMathOperator{\arcsec}{arcsec}
\DeclareMathOperator{\arccsc}{arccsc}

















%%This is to help with formatting on future title pages.
\newenvironment{sectionOutcomes}{}{}


\title{Angles}

\begin{document}

\begin{abstract}
inverses
\end{abstract}
\maketitle




We have two types of trigonometric functions.


We have functions that map angles to ratios:


\[
\sin(\theta), \cos(\theta) :  \text{ Angles } \to \text{ Ratios }
\]


And, we have their inverses:



\[
\sin^{-1}(t), \cos^{-1}(t) :  \text{ Ratios } \to \text{ Angles }
\]








Your calculator has buttons for all of these.


\textbf{Note:}  Your calculator has two mode for angle measurements: degrees and radians.  Remember to switch between these modes when calculating.



\begin{question} Calculator \\

Approximate the following.

\begin{itemize}
\item $\cos(20^{\circ}) \approx \answer[tolerance=0.001]{0.9396926208}$
\item $\sin(65^{\circ}) \approx \answer[tolerance=0.001]{0.906307787}$
\item $\cos\left( \frac{\pi}{3} \right) \approx \answer[tolerance=0.001]{0.5}$
\item $\sin\left( \frac{\pi}{4} \right) \approx \answer[tolerance=0.001]{0.7071067812}$
\end{itemize}

\end{question}





Your calculator also has buttons for the inverse functions. \\




You can also supply the calculator with the ratio and the calculator returns the angle.



\begin{question} Calculator \\

Approximate the following.

\begin{itemize}
\item $\cos^{-1}(0.9396926208) \approx \answer[tolerance=0.001]{20}^{\circ}$
\item $\sin^{-1}(0.906307787) \approx \answer[tolerance=0.001]{65}^{\circ}$
\item $\cos^{-1}(0.5) \approx \answer[tolerance=0.001]{1.047197551}$
\item $\sin^{-1}(0.7071067812) \approx \answer[tolerance=0.001]{0.7853981634}$
\end{itemize}

\end{question}




\subsection{Other Quadrants}

Remember that the signs of sine and cosine changes as we move through the four quadrants.  We introduced a reference angle and a refrence triangle to help with lengths and then we also had to think of the sign. \\

The calculator also knows that the signs change. \\

The calculator will give the correct values of sine and cosine along with their proper sign.  However, going backwards is a different story. \\

If you give the calculator a value of sine and ask it for the angle, then there are an infinite number of angles that have that same value of sine.  Just keep spinning around the circle. Therefore, the calculator just focuses on two quadrants, one for each sine, and then let's you figure out what quadrant you were really in.











\begin{question} Backwards \\

$150^{\circ}$ is an angle in the second quadrant. $\sin(150^{\circ}) = \frac{1}{2} = 0.5$.


Now, ask your calculator to give you an angle whose sine is $0.5$.


$sin^{-1}(0.5) = \answer{30}^{\circ}$


It doesn't give $150^{\circ}$.  It gives $30^{\circ}$.

\end{question}




We'll study this more later in the course.  The short story is




\begin{itemize}
\item $sin^{-1}(x)$ will give angles in the fourth and first quadrants.
\item $cos^{-1}(x)$ will give angles in the first and second quadrants.
\item $tan^{-1}(x)$ will give angles in the fourth and first quadrants.
\end{itemize}






\begin{center}
\textbf{\textcolor{green!50!black}{ooooo=-=-=-=-=-=-=-=-=-=-=-=-=ooOoo=-=-=-=-=-=-=-=-=-=-=-=-=ooooo}} \\

more examples can be found by following this link\\ \link[More Examples of Right Triangles]{https://ximera.osu.edu/csccmathematics/precalculus2/precalculus2/rightTriangles/examples/exampleList}

\end{center}






\end{document}
