\documentclass{ximera}


\graphicspath{
  {./}
  {ximeraTutorial/}
  {basicPhilosophy/}
}

\newcommand{\mooculus}{\textsf{\textbf{MOOC}\textnormal{\textsf{ULUS}}}}

\usepackage{tkz-euclide}\usepackage{tikz}
\usepackage{tikz-cd}
\usetikzlibrary{arrows}
\tikzset{>=stealth,commutative diagrams/.cd,
  arrow style=tikz,diagrams={>=stealth}} %% cool arrow head
\tikzset{shorten <>/.style={ shorten >=#1, shorten <=#1 } } %% allows shorter vectors

\usetikzlibrary{backgrounds} %% for boxes around graphs
\usetikzlibrary{shapes,positioning}  %% Clouds and stars
\usetikzlibrary{matrix} %% for matrix
\usepgfplotslibrary{polar} %% for polar plots
\usepgfplotslibrary{fillbetween} %% to shade area between curves in TikZ
\usetkzobj{all}
\usepackage[makeroom]{cancel} %% for strike outs
%\usepackage{mathtools} %% for pretty underbrace % Breaks Ximera
%\usepackage{multicol}
\usepackage{pgffor} %% required for integral for loops



%% http://tex.stackexchange.com/questions/66490/drawing-a-tikz-arc-specifying-the-center
%% Draws beach ball
\tikzset{pics/carc/.style args={#1:#2:#3}{code={\draw[pic actions] (#1:#3) arc(#1:#2:#3);}}}



\usepackage{array}
\setlength{\extrarowheight}{+.1cm}
\newdimen\digitwidth
\settowidth\digitwidth{9}
\def\divrule#1#2{
\noalign{\moveright#1\digitwidth
\vbox{\hrule width#2\digitwidth}}}






\DeclareMathOperator{\arccot}{arccot}
\DeclareMathOperator{\arcsec}{arcsec}
\DeclareMathOperator{\arccsc}{arccsc}

















%%This is to help with formatting on future title pages.
\newenvironment{sectionOutcomes}{}{}


\title{Geometric Series}

\begin{document}

\begin{abstract}
%Stuff can go here later if we want!
\end{abstract}
\maketitle







\begin{theorem}
The \textbf{\textcolor{purple!85!blue}{geometric series}} $\sum\limits_{k= k_0}^\infty a \cdot r^k$ 
  
  \begin{itemize} 
  \item converges to $\frac{a \, r^{k_0}}{1-r}$ when $|r| < 1$.
  \item diverges if $|r| \geq 1$.  
  \end{itemize}
\end{theorem}




In other words, 


\textbf{\textcolor{blue!55!black}{$\blacktriangleright$}}   $\sum\limits_{k= k_0}^\infty a \cdot r^k$ represents a number when $|r| < 1$. \\

\textbf{\textcolor{blue!55!black}{$\blacktriangleright$}}  Otherwise, it does not represent a number. \\





That sounds like a function.



\[
\sum\limits_{k= k_0}^\infty a \cdot r^k  \text{ and } \frac{a \, r^{k_0}}{1-r} \text{ represent the same function when } |r| < 1.
\]





\subsection{Learning Outcomes}

\begin{sectionOutcomes}
In this this section, students will  

\begin{itemize}
\item treat the geometric series as a function.
\item obtain closed forms for geometric functions.
\end{itemize}
\end{sectionOutcomes}











\begin{center}
\textbf{\textcolor{green!50!black}{ooooo=-=-=-=-=-=-=-=-=-=-=-=-=ooOoo=-=-=-=-=-=-=-=-=-=-=-=-=ooooo}} \\

more examples can be found by following this link\\ \link[More Examples of Geometric Series]{https://ximera.osu.edu/csccmathematics/precalculus1/precalculus1/geometricSeries/examples/exampleList}

\end{center}







\end{document}
