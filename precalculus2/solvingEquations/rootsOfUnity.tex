\documentclass{ximera}


\graphicspath{
  {./}
  {ximeraTutorial/}
  {basicPhilosophy/}
}

\newcommand{\mooculus}{\textsf{\textbf{MOOC}\textnormal{\textsf{ULUS}}}}

\usepackage{tkz-euclide}\usepackage{tikz}
\usepackage{tikz-cd}
\usetikzlibrary{arrows}
\tikzset{>=stealth,commutative diagrams/.cd,
  arrow style=tikz,diagrams={>=stealth}} %% cool arrow head
\tikzset{shorten <>/.style={ shorten >=#1, shorten <=#1 } } %% allows shorter vectors

\usetikzlibrary{backgrounds} %% for boxes around graphs
\usetikzlibrary{shapes,positioning}  %% Clouds and stars
\usetikzlibrary{matrix} %% for matrix
\usepgfplotslibrary{polar} %% for polar plots
\usepgfplotslibrary{fillbetween} %% to shade area between curves in TikZ
\usetkzobj{all}
\usepackage[makeroom]{cancel} %% for strike outs
%\usepackage{mathtools} %% for pretty underbrace % Breaks Ximera
%\usepackage{multicol}
\usepackage{pgffor} %% required for integral for loops



%% http://tex.stackexchange.com/questions/66490/drawing-a-tikz-arc-specifying-the-center
%% Draws beach ball
\tikzset{pics/carc/.style args={#1:#2:#3}{code={\draw[pic actions] (#1:#3) arc(#1:#2:#3);}}}



\usepackage{array}
\setlength{\extrarowheight}{+.1cm}
\newdimen\digitwidth
\settowidth\digitwidth{9}
\def\divrule#1#2{
\noalign{\moveright#1\digitwidth
\vbox{\hrule width#2\digitwidth}}}






\DeclareMathOperator{\arccot}{arccot}
\DeclareMathOperator{\arcsec}{arcsec}
\DeclareMathOperator{\arccsc}{arccsc}

















%%This is to help with formatting on future title pages.
\newenvironment{sectionOutcomes}{}{}


\title{Roots of Unity}

\begin{document}

\begin{abstract}
unit circle
\end{abstract}
\maketitle




We have seen that every Complex number can be written as $r \cdot (\cos(\theta) + i \, \sin(\theta))$ - a scalar times a Complex number on the unit circle.

We have also seen that if $z = r \cdot (\cos(\theta) + i \, \sin(\theta))$, then $z^n = r^n \cdot (\cos(n\theta) + i \, \sin(n\theta))$. Raising complex numbers to powers is accomplished by raising the modulus to the power and then multiplying the angle.


$\blacktriangleright$  \textbf{Roots of Unity}

Roots or unity refer to solutions to equations of the form $z^n = 1$.




For instance, the square roots of unity are the solutions to $z^2 = 1$.  The two solutions are $1$ and $-1$.


For instance, the $4{th}$ roots of unity are the solutions to $z^4 = 1$.  The four solutions are $1$, $-1$, $i$, and $-i$.




$1$ is always a root of unity for any power. Then, there are $n-1$ other $n^{th}$ roots of unity.



If $z = r \cdot (\cos(\theta) + i \, \sin(\theta))$ is going to be a root of unity, then $z^n = r^n \cdot (\cos(n\theta) + i \, \sin(n\theta)) = 1$, which means $r=1$.

$n^{th}$ roots of unity all look like $\cos(\theta) + i \, \sin(\theta)$.  They all lie on the unit circle.



In addition, if $\cos(n\theta) + i \, \sin(n\theta) = 1$, then $n \theta = 2 k \pi$ - a mulitple of $2 \pi$, because $1$ is on the positive $x$-axis.


Therefore, $\theta = \frac{2 k \pi}{n}$  












\begin{example}  $4^{th}$ roots of $1$

We need mulitples of $\frac{2 \pi}{4} = \frac{\pi}{2}$.

\begin{itemize}
\item $\theta = \frac{\pi}{2}$
\item $\theta = \frac{2 \pi}{2} = \pi$
\item $\theta = \frac{3 \pi}{2}$
\item $\theta = \frac{4 \pi}{2} = 2 \pi$
\end{itemize}



The fourth roots of unity are:
\begin{itemize}
\item $\cos\left(\frac{\pi}{2}\right) + i \, \sin\left(\frac{\pi}{2}\right) = i$
\item $\cos(\pi) + i \, \sin(\pi) = -1$
\item $\cos\left(\frac{3 \pi}{2}\right) + i \, \sin\left(\frac{3 \pi}{2}\right) = -i$
\item $\cos(2 \pi) + i \, \sin(2 \pi) = 1$
\end{itemize}






\end{example}















\begin{example}  Cube roots of $1$

We need mulitples of $\frac{2 \pi}{3}$.

\begin{itemize}
\item $\theta = \frac{2 \pi}{3}$
\item $\theta = \frac{4 \pi}{3}$
\item $\theta = \frac{6 \pi}{3} = 2 \pi$
\end{itemize}



The fourth roots of unity are:
\begin{itemize}
\item $\cos\left(\frac{2 \pi}{3}\right) + i \, \sin\left(\frac{2 \pi}{3}\right) = \frac{1}{2} + \frac{\sqrt{3}}{2} \, i$
\item $\cos\left(\frac{4 \pi}{3}\right) + i \, \sin\left(\frac{4 \pi}{3}\right) = \frac{1}{2} - \frac{\sqrt{3}}{2} \, i$
\item $\cos(2 \pi) + i \, \sin(2 \pi) = 1$
\end{itemize}



\end{example}












The roots of unity are spread out equidistant along the unit circle.
























\end{document}
