\documentclass{ximera}


\graphicspath{
  {./}
  {ximeraTutorial/}
  {basicPhilosophy/}
}

\newcommand{\mooculus}{\textsf{\textbf{MOOC}\textnormal{\textsf{ULUS}}}}

\usepackage{tkz-euclide}\usepackage{tikz}
\usepackage{tikz-cd}
\usetikzlibrary{arrows}
\tikzset{>=stealth,commutative diagrams/.cd,
  arrow style=tikz,diagrams={>=stealth}} %% cool arrow head
\tikzset{shorten <>/.style={ shorten >=#1, shorten <=#1 } } %% allows shorter vectors

\usetikzlibrary{backgrounds} %% for boxes around graphs
\usetikzlibrary{shapes,positioning}  %% Clouds and stars
\usetikzlibrary{matrix} %% for matrix
\usepgfplotslibrary{polar} %% for polar plots
\usepgfplotslibrary{fillbetween} %% to shade area between curves in TikZ
\usetkzobj{all}
\usepackage[makeroom]{cancel} %% for strike outs
%\usepackage{mathtools} %% for pretty underbrace % Breaks Ximera
%\usepackage{multicol}
\usepackage{pgffor} %% required for integral for loops



%% http://tex.stackexchange.com/questions/66490/drawing-a-tikz-arc-specifying-the-center
%% Draws beach ball
\tikzset{pics/carc/.style args={#1:#2:#3}{code={\draw[pic actions] (#1:#3) arc(#1:#2:#3);}}}



\usepackage{array}
\setlength{\extrarowheight}{+.1cm}
\newdimen\digitwidth
\settowidth\digitwidth{9}
\def\divrule#1#2{
\noalign{\moveright#1\digitwidth
\vbox{\hrule width#2\digitwidth}}}






\DeclareMathOperator{\arccot}{arccot}
\DeclareMathOperator{\arcsec}{arcsec}
\DeclareMathOperator{\arccsc}{arccsc}

















%%This is to help with formatting on future title pages.
\newenvironment{sectionOutcomes}{}{}


\title{Quadratic}

\begin{document}

\begin{abstract}
complex roots
\end{abstract}
\maketitle



\textbf{\textcolor{blue!55!black}{Zeros of Quadratic Functions}}


To obtain the zeros of a quadratic function, we solve quadratic equations and we have already seen three methods of solving quadratic equations.

\begin{itemize}
	\item Factoring
	\item Completing the Square
	\item Quadratic Formula
	\end{itemize}


All three methods rely on first rewriting the equation, so that one side equals $0$.

When we first investigated quadratic equations, we were only interested in real roots or real solutions.  We are now in a position to fill-in the rest of the picture, because we can now work with square roots of negative numbers.


\begin{example} Complex Solutions

Solve $3t^2 + 2t + 5 = 0$

Using the quadratic formula, we get two solutions

\[  t = \frac{-2 \pm \sqrt{2^2 - 4 \cdot 3 \cdot 5}}{2 \cdot 3}   =  \frac{-2 \pm \sqrt{-56}}{6}  =  \frac{-2 \pm \sqrt{56}\sqrt{-1}}{6}   =  \frac{-2 \pm \sqrt{56} \, i}{6}  =  \frac{-2 \pm i \,\sqrt{56}}{6}   \]



We can simplify this, if you would like.


\[  t = \frac{-2 \pm \sqrt{4 \cdot 14} \, i}{6}  = \\frac{-2 \pm i \, \sqrt{4 \cdot 14}}{6}  =  frac{-2 \pm i \,\sqrt{4} \sqrt{14}}{6} = \frac{-2 \pm 2 i \, \sqrt{14}}{6} = \frac{2(-1 \pm i \,\sqrt{14} \, i)}{6}   = \frac{-1 \pm i \,\sqrt{14}}{3}   \]



\end{example}













\begin{example} Complex Roots

Factor $3 t^2 + 2 t + 5$


In the previous example we found the roots of this polynomial to be


\[     \frac{-1 + \sqrt{14} \, i}{3}   \,  \text{ and } \, \frac{-1 - \sqrt{14} \, i}{3}         \]

Therefore, the factored form of this polynomial looks like


\[     A \left(t - \frac{-1 + \sqrt{14} \, i}{3} \right) \left(t -  \frac{-1 - \sqrt{14} \, i}{3} \right)        \]


If we were to mulitply out the two factors, we would end up with $1$ as a leading coefficient.  Therefore, $A = 3$.

\[     3 \left(t - \frac{-1 + \sqrt{14} \, i}{3} \right) \left(t -  \frac{-1 - \sqrt{14} \, i}{3} \right)        \]

\end{example}





Let's multiply out the factorization in the last example and see if it works.



\[     3 \left(t - \frac{-1 + \sqrt{14} \, i}{3} \right) \left(t -  \frac{-1 - \sqrt{14} \, i}{3} \right)        \]


\[     3 \left(  t^2 - \left(\frac{-1 + \sqrt{14} \, i}{3} + \frac{-1 - \sqrt{14} \, i}{3}\right) t + \left(\frac{-1 + \sqrt{14} \, i}{3}\right) \left(\frac{-1 - \sqrt{14} \, i}{3}\right)      \right)       \]



\[     3 \left(  t^2 - \left(\frac{-2}{3} \right) t + \frac{(-1 + \sqrt{14} \, i)(-1 - \sqrt{14} \, i)}{9}\right)     \]



\[     3 \left(  t^2 - \left(\frac{-2}{3} \right) t + \frac{1 - 14 \, i^2}{9}\right)     \]


\[     3 \left(  t^2 - \left(\frac{-2}{3} \right) t + \frac{15}{9} \right)     \]


\[     3 \left(  t^2 - \left(\frac{-2}{3} \right) t + \frac{5}{3}\right)     \]



\[      t^2 + 2t + 5   \]








\begin{example}  Analyze   $g(k) = \frac{1}{10} (k+6)(k+2)(k-4)$


\textbf{Domain: } This is a polynomial funciton, so its domain is all real numbers. \\



\textbf{Zeros: } $g$ is given in factored form and the zeros of $g$ are the zeros of these factors: $-$, $-2$, and $4$.\\



\textbf{Continuity: } This is a polynomial function, so it is continuous on its domain. There are not discontinuities or singularities.\\



\textbf{End-Behavior: }  This is a polynomial function of odd degree with a positive leading coefficient.


\[  \limlimits_{x \to -\infty} g(k) = -\infty  \,   \text{ and }     \,   \limlimits_{x \to \infty} g(k) = \infty    \]



\textbf{Range: } This also tells us that the range is $(-\infty, \infty)$.

\textbf{Behavior: }


Algebra really has no way of finding the critical numbers of $g$ and obtaining exact intervals where $g$ increases and decreases.  Caluclus will provide a method of obtaining the derivative, which will help with this problem. 



A DESMOS graph suggests that there are two critical numbers.  DESMOS approximates these to be around $-4.239$ and $1.573$.

\textbf{Note: } These are not the middle points between the zeros.


These critical numbers are the locations of a local maximum and minimum for our polynomial function.  And, a graph is about the best we can do for reasoning on local extrema.





\begin{center}
\desmos{zqgjgiermr}{400}{300}
\end{center}








Calculus will show us how to get the derivative:

\[  g'(k) = \frac{3}{10} k^2 + \frac{4}{5} x - 2  \]



The critical numbers of $g(k)$ are the zeros of the derivative, which is a quadratic function. We can use the quadratic formula.




To make things a little eaier to write and read, let's factor out $\frac{1}{10}$.


\[    g'(k) = \frac{3}{10} k^2 + \frac{4}{5} x - 2  =  \frac{1}{10} ( 3 k^2 + 8 k - 20 )   \]


The zeros of $g'(k)$ are the zeros of $3 k^2 + 8 k - 20$.

We need to solve $3 k^2 + 8 k - 20 = 0$


\[  k = \frac{-8 \pm \sqrt{64 + 4 \cdot 3 \cdot 20}}{6}  = \frac{-8 \pm \sqrt{304}}{6} \]


\[   = \frac{-8 \pm 4 \sqrt{19}}{6}   = \frac{-4 \pm 2 \sqrt{19}}{3}      \]



\[   \frac{-4 - 2 \sqrt{19}}{3}  \approx   1.573    \]


\[   \frac{-4 + 2 \sqrt{19}}{3}  \approx   -4.239   \]


And, that agrees with our graph.




\begin{itemize}
\item  $g$ increases on $\left( -\infty, \frac{-4 - 2 \sqrt{19}}{3} \right)$ 
\item  $g$ decreases on $\left( \frac{-4 - 2 \sqrt{19}}{3}, \frac{-4 + 2 \sqrt{19}}{3}  \right)$ 
\item  $g$ increases on $\left( \frac{-4 + 2 \sqrt{19}}{3}, \infty \right)$ 
\end{itemize}




\textbf{Extrema: (Maximums and minimums) }


$\blacktriangleright$ $g\left( \frac{-4 - 2 \sqrt{19}}{3} \right)$ is a local maximum of $g$ occuring at $\frac{-4 - 2 \sqrt{19}}{3}$

$\blacktriangleright$ $g\left( \frac{-4 + 2 \sqrt{19}}{3} \right)$ is a local minimum of $g$ occuring at $\frac{-4 + 2 \sqrt{19}}{3}$





The end-behavior of $g$ tells us that there is no global maximum or minimum.



\end{example}

















\begin{center}
\textbf{\textcolor{green!50!black}{ooooo=-=-=-=-=-=-=-=-=-=-=-=-=ooOoo=-=-=-=-=-=-=-=-=-=-=-=-=ooooo}} \\

more examples can be found by following this link\\ \link[More Examples of Zeros]{https://ximera.osu.edu/csccmathematics/precalculus2/precalculus2/solvingEquations/examples/exampleList}

\end{center}




\end{document}
