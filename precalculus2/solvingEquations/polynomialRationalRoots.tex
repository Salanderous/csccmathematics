\documentclass{ximera}


\graphicspath{
  {./}
  {ximeraTutorial/}
  {basicPhilosophy/}
}

\newcommand{\mooculus}{\textsf{\textbf{MOOC}\textnormal{\textsf{ULUS}}}}

\usepackage{tkz-euclide}\usepackage{tikz}
\usepackage{tikz-cd}
\usetikzlibrary{arrows}
\tikzset{>=stealth,commutative diagrams/.cd,
  arrow style=tikz,diagrams={>=stealth}} %% cool arrow head
\tikzset{shorten <>/.style={ shorten >=#1, shorten <=#1 } } %% allows shorter vectors

\usetikzlibrary{backgrounds} %% for boxes around graphs
\usetikzlibrary{shapes,positioning}  %% Clouds and stars
\usetikzlibrary{matrix} %% for matrix
\usepgfplotslibrary{polar} %% for polar plots
\usepgfplotslibrary{fillbetween} %% to shade area between curves in TikZ
\usetkzobj{all}
\usepackage[makeroom]{cancel} %% for strike outs
%\usepackage{mathtools} %% for pretty underbrace % Breaks Ximera
%\usepackage{multicol}
\usepackage{pgffor} %% required for integral for loops



%% http://tex.stackexchange.com/questions/66490/drawing-a-tikz-arc-specifying-the-center
%% Draws beach ball
\tikzset{pics/carc/.style args={#1:#2:#3}{code={\draw[pic actions] (#1:#3) arc(#1:#2:#3);}}}



\usepackage{array}
\setlength{\extrarowheight}{+.1cm}
\newdimen\digitwidth
\settowidth\digitwidth{9}
\def\divrule#1#2{
\noalign{\moveright#1\digitwidth
\vbox{\hrule width#2\digitwidth}}}






\DeclareMathOperator{\arccot}{arccot}
\DeclareMathOperator{\arcsec}{arcsec}
\DeclareMathOperator{\arccsc}{arccsc}

















%%This is to help with formatting on future title pages.
\newenvironment{sectionOutcomes}{}{}


\title{Rationals}

\begin{document}

\begin{abstract}
RRT
\end{abstract}
\maketitle




Let $f(x)$ be a polynomial with rational coefficients.

\[   f(x) = b_n x^n + b_{n-1} x^{n-1} + \cdots + b_2 x^2 + b_1 x + b_0     \]


where $b_k \in \mathbb{Q}$:  $b_k = \frac{n_k}{d_k}$


Then we could get a common denominator of all of the coefficients and factor that out front



\[   f(x) = \frac{1}{d}(c_n x^n + c_{n-1} x^{n-1} + \cdots + c_2 x^2 + c_1 x + c_0)    \]


These are the same polynomial function, so they have the same roots and factorization.


So, when we are investigating zeros and roots and factors, we can always assume that a polynomial with rational coefficients just has integer coefficents.




\section{Rational Roots}

Let $f(x)$ be a polynomial with integer coefficients.

\[   f(x) = a_n x^n + a_{n-1} x^{n-1} + \cdots + a_2 x^2 + a_1 x + a_0     \]


Suppose $f(x)$ has a rational root:  $\frac{N}{D}$ in reduced form.  That is $N$ and $D$ do not share any prime factors.



That means


\[    f \left( \frac{N}{D} \right) = a_n \left( \frac{N}{D} \right)^n + a_{n-1} \left( \frac{N}{D} \right)^{n-1} + \cdots + a_2 \left( \frac{N}{D} \right)^2 + a_1 \left( \frac{N}{D} \right) + a_0  = 0       \]


Let's factor out $\frac{1}{D^n}$





\[   \frac{1}{D^n} (a_n N^n + a_{n-1} N^{n-1} D + \cdots + a_2 N^2 D^{n-2}+ a_1 N D^{n-1} + a_0 D^n)  = 0       \]


That tells us

\[   a_n N^n + a_{n-1} N^{n-1} D + \cdots + a_2 N^2 D^{n-2}+ a_1 N D^{n-1} + a_0 D^n  = 0       \]



\[   a_n N^n  =   -(a_{n-1} N^{n-1} D + \cdots + a_2 N^2 D^{n-2}+ a_1 N D^{n-1} + a_0 D^n)      \]


$D$ is a factor of the right side, so $D$ must be a factor of the left side,  But $D$ is not a factor of $N$, because the original rational root was in reduced form.  Therefore $D$ must be a factor of $a_n$.




Similarly, 

\[   -(a_n N^n + a_{n-1} N^{n-1} D + \cdots + a_2 N^2 D^{n-2}+ a_1 N D^{n-1}) =  a_0 D^n        \]


$N$ is a factor of the left side, so $N$ must be a factor of the right side,  But $N$ is not a factor of $D$, because the original rational root was in reduced form.  Therefore $N$ must be a factor of $a_0$.




This is very helpful when factoring polynomials.





\begin{theorem} \textbf{\textcolor{blue!55!black}{Rational Root Theorem}} \\


Let $ f(x) = a_n x^n + a_{n-1} x^{n-1} + \cdots + a_2 x^2 + a_1 x + a_0 $ be a polynomial with integer coefficients.

Let $\frac{N}{D}$ be a rational root of $f(x)$.

Then $D$ is a factor of $a_n$, the leading coefficient and $N$ is a factor of $a_0$, the constant term.

\end{theorem}











\begin{example}  Roots and Factoring


Let $T(h) = 2 h^2 - 9h - 5$



If $T(h)$ has a rational root, $\frac{N}{D}$, then


\begin{itemize}
\item $N$ is a factor of $5$: $-5$, $-1$, $1$, or $5$
\item $D$ is a factor of $2$: $-2$, $-1$, $1$, or $2$
\end{itemize}



Possibilities are 

\[  \frac{5}{2}, \frac{-5}{2}, \frac{5}{1}, \frac{-5}{1}, \frac{-1}{-1}, \frac{-1}{1}, \frac{1}{2}, \frac{-1}{2}   \]


$T$ is a quadratic, so there are only two roots. If they are rational roots, then they are among these.


We can try each one.



\begin{itemize}
\item $T\left( \frac{5}{2} \right) = -15$
\item $T\left( \frac{-5}{2} \right) = 30$
\item $T(5) = 0$   : a root
\item $T(-5) = 90$
\item $T(1) = -12$   
\item $T(-1) = 6$
\item $T\left( \frac{1}{2} \right) = -9$
\item $T\left( \frac{-1}{2} \right) = 0$ : a root
\end{itemize}

We have two roots, which gives us two factors.


$T(h) = 2\left(h-\frac{1}{2} \right)(h-5)$

We could also multiply the $2$ inside the first factor.


$T(h) = (2h - 1)(h-5)$



\end{example}










\begin{center}
\textbf{\textcolor{green!50!black}{ooooo=-=-=-=-=-=-=-=-=-=-=-=-=ooOoo=-=-=-=-=-=-=-=-=-=-=-=-=ooooo}} \\

more examples can be found by following this link\\ \link[More Examples of Zeros]{https://ximera.osu.edu/csccmathematics/precalculus2/precalculus2/solvingEquations/examples/exampleList}

\end{center}

\end{document}
