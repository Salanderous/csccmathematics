\documentclass{ximera}


\graphicspath{
  {./}
  {ximeraTutorial/}
  {basicPhilosophy/}
}

\newcommand{\mooculus}{\textsf{\textbf{MOOC}\textnormal{\textsf{ULUS}}}}

\usepackage{tkz-euclide}\usepackage{tikz}
\usepackage{tikz-cd}
\usetikzlibrary{arrows}
\tikzset{>=stealth,commutative diagrams/.cd,
  arrow style=tikz,diagrams={>=stealth}} %% cool arrow head
\tikzset{shorten <>/.style={ shorten >=#1, shorten <=#1 } } %% allows shorter vectors

\usetikzlibrary{backgrounds} %% for boxes around graphs
\usetikzlibrary{shapes,positioning}  %% Clouds and stars
\usetikzlibrary{matrix} %% for matrix
\usepgfplotslibrary{polar} %% for polar plots
\usepgfplotslibrary{fillbetween} %% to shade area between curves in TikZ
\usetkzobj{all}
\usepackage[makeroom]{cancel} %% for strike outs
%\usepackage{mathtools} %% for pretty underbrace % Breaks Ximera
%\usepackage{multicol}
\usepackage{pgffor} %% required for integral for loops



%% http://tex.stackexchange.com/questions/66490/drawing-a-tikz-arc-specifying-the-center
%% Draws beach ball
\tikzset{pics/carc/.style args={#1:#2:#3}{code={\draw[pic actions] (#1:#3) arc(#1:#2:#3);}}}



\usepackage{array}
\setlength{\extrarowheight}{+.1cm}
\newdimen\digitwidth
\settowidth\digitwidth{9}
\def\divrule#1#2{
\noalign{\moveright#1\digitwidth
\vbox{\hrule width#2\digitwidth}}}






\DeclareMathOperator{\arccot}{arccot}
\DeclareMathOperator{\arcsec}{arcsec}
\DeclareMathOperator{\arccsc}{arccsc}

















%%This is to help with formatting on future title pages.
\newenvironment{sectionOutcomes}{}{}


\title{FTA}

\begin{document}

\begin{abstract}
Fundamental Theory of Algebra
\end{abstract}
\maketitle









The previous examples are very suggestive that polynomials factor into a product of linear factors.  And, there are as many linear factors as the degree of the polynomial, if you count multiplicities. \\



This turns out to be the underlying foundation of all of Algebra. \\




\begin{theorem} \textbf{\textcolor{green!50!black}{The Fundamental Theorem of Algebra (Complex)}} 



Let $p(x) = a_n x^n + a_{n-1} x^{n-1} + \cdots + a_1 x + a_0$ be a polynomial of degree $n$ with complex coefficients. \\


Then $p(x)$ can be written as a product of exactly $n$ linear factors:

\[
 a (x - r_n) (x - r_{n-1}) \cdots (x - r_2)  (x - r_1) 
\]


where $r_i \in \mathbb{C}$.


\end{theorem}








\begin{theorem} \textbf{\textcolor{green!50!black}{The Fundamental Theorem of Algebra (Real)}} 



Let $p(x) = a_n x^n + a_{n-1} x^{n-1} + \cdots + a_1 x + a_0$ be a polynomial of degree $n$ with real coefficients. \\


Then $p(x)$ can be written as a product of linear and irreducible quadratic factors with real coefficients.


\end{theorem}


Our intuition is correct if we allow complex numbers to be used.  If we wish to stay inside the real numbers for our coefficients, then we can factor into linear and irreducible quadratics.  Irreducible quadratics have complex roots, so cannot be factored over the reals.



\begin{idea}


The way we prove the Fundamental Theorem of Algebra involves two steps. \\


First, a polynomial cannot have MORE linear factors than its degree. Otherwise, when you multiplied out the factors, you would end up with a polynomial with a greater degree than you started with.\\


Second, how do we know the Complex Numbers actually do have all of the roots and there isn't some bigger number system needed? \\

The way we prove this second part is to simply show that each polynomial MUST have one complex root. If this is true, then we can keep factoring out one factor until we get all of them.

The second part is not so easy to show.  We'll leave it for another class.


\end{idea}






















\end{document}
