\documentclass{ximera}


\graphicspath{
  {./}
  {ximeraTutorial/}
  {basicPhilosophy/}
}

\newcommand{\mooculus}{\textsf{\textbf{MOOC}\textnormal{\textsf{ULUS}}}}

\usepackage{tkz-euclide}\usepackage{tikz}
\usepackage{tikz-cd}
\usetikzlibrary{arrows}
\tikzset{>=stealth,commutative diagrams/.cd,
  arrow style=tikz,diagrams={>=stealth}} %% cool arrow head
\tikzset{shorten <>/.style={ shorten >=#1, shorten <=#1 } } %% allows shorter vectors

\usetikzlibrary{backgrounds} %% for boxes around graphs
\usetikzlibrary{shapes,positioning}  %% Clouds and stars
\usetikzlibrary{matrix} %% for matrix
\usepgfplotslibrary{polar} %% for polar plots
\usepgfplotslibrary{fillbetween} %% to shade area between curves in TikZ
\usetkzobj{all}
\usepackage[makeroom]{cancel} %% for strike outs
%\usepackage{mathtools} %% for pretty underbrace % Breaks Ximera
%\usepackage{multicol}
\usepackage{pgffor} %% required for integral for loops



%% http://tex.stackexchange.com/questions/66490/drawing-a-tikz-arc-specifying-the-center
%% Draws beach ball
\tikzset{pics/carc/.style args={#1:#2:#3}{code={\draw[pic actions] (#1:#3) arc(#1:#2:#3);}}}



\usepackage{array}
\setlength{\extrarowheight}{+.1cm}
\newdimen\digitwidth
\settowidth\digitwidth{9}
\def\divrule#1#2{
\noalign{\moveright#1\digitwidth
\vbox{\hrule width#2\digitwidth}}}






\DeclareMathOperator{\arccot}{arccot}
\DeclareMathOperator{\arcsec}{arcsec}
\DeclareMathOperator{\arccsc}{arccsc}

















%%This is to help with formatting on future title pages.
\newenvironment{sectionOutcomes}{}{}


\title{Arctangent}

\begin{document}

\begin{abstract}
useful
\end{abstract}
\maketitle



top

The tangent function is not a one-to-one function.

middle


submiddle


bottom





\begin{image}
\begin{tikzpicture} 
  \begin{axis}[
            domain=-10:10, ymax=10, xmax=10, ymin=-10, xmin=-10,
            xtick={-4.71, -1.57, 1.57, 4.71}, 
            xticklabels={$-\frac{3\pi}{2}$, $-\frac{\pi}{2}$, $\frac{\pi}{2}$, $\frac{3\pi}{2}$},
            axis lines =center,  xlabel={$\theta$}, ylabel=$y$,
            ticklabel style={font=\scriptsize},
            every axis y label/.style={at=(current axis.above origin),anchor=south},
            every axis x label/.style={at=(current axis.right of origin),anchor=west},
            axis on top
          ]
          
            \addplot [line width=1, gray, dashed,samples=100,domain=(-10:10), <->] ({-4.71},{x});
            \addplot [line width=1, gray, dashed,samples=100,domain=(-10:10), <->] ({-1.57},{x});
            \addplot [line width=1, gray, dashed,samples=100,domain=(-10:10), <->] ({1.57},{x});
            \addplot [line width=1, gray, dashed,samples=100,domain=(-10:10), <->] ({4.71},{x});

            \addplot [line width=2, penColor, smooth,samples=100,domain=(-1.47:1.47), <->] {tan(deg(x))};
            \addplot [line width=2, penColor, smooth,samples=100,domain=(-4.61:-1.67), <->] {tan(deg(x))};
            \addplot [line width=2, penColor, smooth,samples=100,domain=(1.67:4.61), <->] {tan(deg(x))};

      \addplot[color=penColor,fill=penColor,only marks, mark size=1pt, mark=*] coordinates{(-9,5) (-8,5) (-7,5) (7,5) (8,5) (9,5)};

           

  \end{axis}
\end{tikzpicture}
\end{image}




The usual choice for restricting the domain is $\left(-\frac{\pi}{2},\frac{\pi}{2}\right)$.  











\begin{image}
\begin{tikzpicture} 
  \begin{axis}[
            domain=-4.75:4.75, ymax=10, xmax=4.75, ymin=-10, xmin=-4.75,
            xtick={-4.71, -1.57, 1.57, 4.71}, 
            xticklabels={$-\frac{3\pi}{2}$, $-\frac{\pi}{2}$, $\frac{\pi}{2}$, $\frac{3\pi}{2}$},
            axis lines =center,  xlabel={$\theta$}, ylabel=$y$,
            ticklabel style={font=\scriptsize},
            every axis y label/.style={at=(current axis.above origin),anchor=south},
            every axis x label/.style={at=(current axis.right of origin),anchor=west},
            axis on top
          ]
          
            \addplot [line width=1, gray, dashed,samples=100,domain=(-10:10), <->] ({-1.57},{x});
            \addplot [line width=1, gray, dashed,samples=100,domain=(-10:10), <->] ({1.57},{x});

            \addplot [line width=2, penColor, smooth,samples=100,domain=(-1.47:1.47), <->] {tan(deg(x))};
           

  \end{axis}
\end{tikzpicture}
\end{image}








We then revolve the graph around the $y=\theta$ diagonal.












\begin{image}
\begin{tikzpicture}
  \begin{axis}[
            domain=-20:20, ymax=1.7, xmax=20, ymin=-1.7, xmin=-20,
            axis lines =center, xlabel={$x$}, ylabel={$y = \arctan(x)$}, grid = major, grid style={dashed},
            ytick={-1.57,-0.785,0.785,1.57},
            xtick={-20,-15,-10,-5,5,10,15,20},
            xticklabels={$-20$,$-15$,$-10$,$-5$,$5$,$10$,$15$,$20$},
            yticklabels={$-\tfrac{\pi}{2}$, $-\tfrac{\pi}{4}$,$\tfrac{\pi}{4}$, $\tfrac{\pi}{2}$}, 
            ticklabel style={font=\scriptsize},
            every axis y label/.style={at=(current axis.above origin),anchor=south},
            every axis x label/.style={at=(current axis.right of origin),anchor=west},
            axis on top
          ]
          

            \addplot [line width=2, penColor, smooth,samples=300,domain=(-1.52:1.52),<->] ({tan(deg(x))},{x});
            \addplot [line width=1, gray, dashed,samples=300,domain=(-20:20),<->] {-1.57};
            \addplot [line width=1, gray, dashed,samples=300,domain=(-20:20),<->] {1.57};



  \end{axis}
\end{tikzpicture}
\end{image}

The vertical asymptotes become horizontal asymptote and represent end-behavior. \\









\subsection{\textbf{\textcolor{purple!85!blue}{Characteristics}}} 

We can deduce many characteristics about arctangent from tangent.


$\blacktriangleright$ The domain is all real numbers, $(-\infty, \infty)$.


$\blacktriangleright$ The range is $\left( -\frac{\pi}{2}, \frac{\pi}{2} \right)$


$\blacktriangleright$ Arctan has one zero and that is $0$.


$\blacktriangleright$ Arctan is an increasing function.

$\blacktriangleright$ Arctan is a continuous function.

$\blacktriangleright$ Arctan has no maximums or minimums.


$\blacktriangleright$ End-Behavior





\begin{itemize}
  \item $\lim\limits_{x \to -\infty} \arctan(x) = -\frac{\pi}{2}$
  \item $\lim\limits_{x \to \infty} \arctan(x) = \frac{\pi}{2}$
\end{itemize}










\subsection{\textbf{\textcolor{purple!85!blue}{Coordinates}}} 

Given a point described with Cartesian (rectangular) coordinates, $(x, y)$, we can obtain polar coordinates via the equations


\begin{itemize}
\item $r^2 = x^2 + y^2$
\item $\theta = \arctan\left( \frac{y}{x} \right)$
\end{itemize}









\end{document}

