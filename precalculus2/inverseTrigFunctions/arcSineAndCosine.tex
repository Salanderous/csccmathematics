\documentclass{ximera}


\graphicspath{
  {./}
  {ximeraTutorial/}
  {basicPhilosophy/}
}

\newcommand{\mooculus}{\textsf{\textbf{MOOC}\textnormal{\textsf{ULUS}}}}

\usepackage{tkz-euclide}\usepackage{tikz}
\usepackage{tikz-cd}
\usetikzlibrary{arrows}
\tikzset{>=stealth,commutative diagrams/.cd,
  arrow style=tikz,diagrams={>=stealth}} %% cool arrow head
\tikzset{shorten <>/.style={ shorten >=#1, shorten <=#1 } } %% allows shorter vectors

\usetikzlibrary{backgrounds} %% for boxes around graphs
\usetikzlibrary{shapes,positioning}  %% Clouds and stars
\usetikzlibrary{matrix} %% for matrix
\usepgfplotslibrary{polar} %% for polar plots
\usepgfplotslibrary{fillbetween} %% to shade area between curves in TikZ
\usetkzobj{all}
\usepackage[makeroom]{cancel} %% for strike outs
%\usepackage{mathtools} %% for pretty underbrace % Breaks Ximera
%\usepackage{multicol}
\usepackage{pgffor} %% required for integral for loops



%% http://tex.stackexchange.com/questions/66490/drawing-a-tikz-arc-specifying-the-center
%% Draws beach ball
\tikzset{pics/carc/.style args={#1:#2:#3}{code={\draw[pic actions] (#1:#3) arc(#1:#2:#3);}}}



\usepackage{array}
\setlength{\extrarowheight}{+.1cm}
\newdimen\digitwidth
\settowidth\digitwidth{9}
\def\divrule#1#2{
\noalign{\moveright#1\digitwidth
\vbox{\hrule width#2\digitwidth}}}






\DeclareMathOperator{\arccot}{arccot}
\DeclareMathOperator{\arcsec}{arcsec}
\DeclareMathOperator{\arccsc}{arccsc}

















%%This is to help with formatting on future title pages.
\newenvironment{sectionOutcomes}{}{}


\title{ArcSine \& ArcCosine}

\begin{document}

\begin{abstract}
restricting the domain
\end{abstract}
\maketitle





We have a new family of Trigonometric functions. And, to complete the picture, we need some inverse functions.  Since none of our Trigonometric functions are one-to-one functions, we will restrict their domains for the purposes of obtaining inverse functions.





\section{Arcsine}


The sine function is not one-to-one.









\begin{image}
\begin{tikzpicture}
  \begin{axis}[
            domain=-10:10, ymax=1.5, xmax=10, ymin=-1.5, xmin=-10,
            axis lines =center, xlabel={$\theta$}, ylabel={$y = \sin(\theta)$}, grid = major, grid style={dashed},
            ytick={-1.5,-1,-0.5,0.5,1,1.5},
            xtick={-7.85, -6.28, -4.71, -3.14, -1.57, 0, 1.57, 3.142, 4.71, 6.28, 7.85},
            xticklabels={$-\tfrac{5\pi}{2}$,$-2\pi$,$-\tfrac{3\pi}{2}$,$-\pi$, $-\tfrac{\pi}{2}$, $0$, $\tfrac{\pi}{2}$, $\pi$, $\tfrac{3\pi}{2}$, $2\pi$, $\tfrac{5\pi}{2}$},
            yticklabels={$1.5$,$-1$,$-0.5$,$0.5$,$1$,$1.5$}, 
            ticklabel style={font=\scriptsize},
            every axis y label/.style={at=(current axis.above origin),anchor=south},
            every axis x label/.style={at=(current axis.right of origin),anchor=west},
            axis on top
          ]
          

            \addplot [line width=2, penColor, smooth,samples=300,domain=(-10:10),<->] {sin(deg(x)};
            %\addplot [line width=2, penColor2, smooth,samples=300,domain=(-10:10),<->] {cos(deg(x)};



  \end{axis}
\end{tikzpicture}
\end{image}





This function doesn't have an inverse.

We can't even restrict our domain to one period, because of the hills and valleys in the graph.

Therefore, our plan is to restrict the domain to half of one period of sine.  We can select any interval of length $\pi$.  The usual choice is $\left[-\frac{\pi}{2},\frac{\pi}{2}\right]$.  














\begin{image}
\begin{tikzpicture}
  \begin{axis}[
            domain=-2:2, ymax=1.5, xmax=2, ymin=-1.5, xmin=-2,
            axis lines =center, xlabel={$\theta$}, ylabel={$y = \sin(\theta)$}, grid = major, grid style={dashed},
            ytick={-1.5,-1,-0.5,0.5,1,1.5},
            xtick={-1.57, 1.57},
            xticklabels={$-\tfrac{\pi}{2}$, $\tfrac{\pi}{2}$},
            yticklabels={$1.5$,$-1$,$-0.5$,$0.5$,$1$,$1.5$}, 
            ticklabel style={font=\scriptsize},
            every axis y label/.style={at=(current axis.above origin),anchor=south},
            every axis x label/.style={at=(current axis.right of origin),anchor=west},
            axis on top
          ]
          

            \addplot [line width=2, penColor, smooth,samples=300,domain=(-1.57:1.57)] {sin(deg(x)};
            \addplot[color=penColor,fill=penColor,only marks,mark=*] coordinates{(-1.57,-1)};
            \addplot[color=penColor,fill=penColor,only marks,mark=*] coordinates{(1.57,1)};
            %\addplot [line width=2, penColor2, smooth,samples=300,domain=(-10:10),<->] {cos(deg(x)};



  \end{axis}
\end{tikzpicture}
\end{image}



It is only half a period, but it does cover the whole range of sine.

We can now reverse all of the pairs and obtain the inverse function known as \textbf{arcsine}.









\begin{image}
\begin{tikzpicture}
  \begin{axis}[
            domain=-2:2, ymax=1.6, xmax=2, ymin=-1.6, xmin=-2,
            axis lines =center, xlabel={$x$}, ylabel={$y = \arcsin(x)$}, grid = major, grid style={dashed},
            xtick={-1.5,-1,-0.5,0.5,1,1.5},
            ytick={-1.57,-0.785,0.785,1.57},
            yticklabels={$-\tfrac{\pi}{2}$, $-\tfrac{\pi}{4}$, $\tfrac{\pi}{4}$, $\tfrac{\pi}{2}$},
            xticklabels={$1.5$,$-1$,$-0.5$,$0.5$,$1$,$1.5$}, 
            ticklabel style={font=\scriptsize},
            every axis y label/.style={at=(current axis.above origin),anchor=south},
            every axis x label/.style={at=(current axis.right of origin),anchor=west},
            axis on top
          ]
          

            \addplot [line width=2, penColor, smooth,samples=300,domain=(-1.57:1.57)] ({sin(deg(x)},{x});
            \addplot[color=penColor,fill=penColor,only marks,mark=*] coordinates{(-1,-1.57)};
            \addplot[color=penColor,fill=penColor,only marks,mark=*] coordinates{(1,1.57)};
            %\addplot [line width=2, penColor2, smooth,samples=300,domain=(-10:10),<->] {cos(deg(x)};



  \end{axis}
\end{tikzpicture}
\end{image}




$Arcsine$ only returns angles between $-\frac{\pi}{2}$ and $\frac{\pi}{2}$. These are the fourth and first quadrants. Arcsine is an inverse function for sine for these angles.



\begin{example}



$\sin\left(\frac{\pi}{6}\right) = \frac{1}{2}$

$\arcsin\left(\frac{1}{2}\right) = \frac{\pi}{6}$


$\arcsin\left(\sin\left(\frac{\pi}{6}\right)\right) = \frac{\pi}{6}$


\end{example}




When angles are outside $\left[-\frac{\pi}{2},\frac{\pi}{2}\right]$, then the Arcsine brings them back into this interval.




\begin{example}



$\sin\left(\frac{5\pi}{6}\right) = \frac{1}{2}$

$\arcsin\left(\frac{1}{2}\right) = \frac{\pi}{6}$


$\arcsin\left(\sin\left(\frac{5\pi}{6}\right)\right) = \frac{\pi}{6}$




We begin with the angle $\frac{5\pi}{6}$ whose sine is $\frac{1}{2}$.  Then Arcsine looks for an angle whose sine is $\frac{1}{2}$.  But Arcsine only looks in quadrants I and IV.  So, it finds $\frac{\pi}{6}$.


\end{example}









\subsection{\textbf{\textcolor{purple!85!blue}{Characteristics}}} 

We can deduce many characteristics about arcsine from sine.


$\blacktriangleright$ The domain is  $[-1, 1]$.


$\blacktriangleright$ The range is $\left[ -\frac{\pi}{2}, \frac{\pi}{2} \right]$


$\blacktriangleright$ Arcsin has one zero and that is $0$.


$\blacktriangleright$ Arcsin is an increasing function.

$\blacktriangleright$ Arcsin is a continuous function.

$\blacktriangleright$ Arcsin has a maximum of $\frac{\pi}{2}$, which occurs at $1$.

$\blacktriangleright$ Arcsin has a minimum of $-\frac{\pi}{2}$, which occurs at $-$.



\begin{question}


If $0 < \theta < \frac{\pi}{2}$, then $\arcsin(\sin(\theta))$ is in quadrant

\begin{multipleChoice}
\choice[correct] {I}
\choice {II}
\choice {III}
\choice {IV}
\end{multipleChoice}

\end{question}






\begin{question}


If $\frac{\pi}{2} < \theta < \pi$, then $\arcsin(\sin(\theta))$ is in quadrant

\begin{multipleChoice}
\choice[correct] {I}
\choice {II}
\choice {III}
\choice {IV}
\end{multipleChoice}

\end{question}









\begin{question}


If $\pi < \theta < \frac{3\pi}{2}$, then $\arcsin(\sin(\theta))$ is in quadrant

\begin{multipleChoice}
\choice {I}
\choice {II}
\choice {III}
\choice[correct] {IV}
\end{multipleChoice}

\end{question}







\begin{question}


If $\frac{3\pi}{2} < \theta < 2\pi$, then $\arcsin(\sin(\theta))$ is in quadrant

\begin{multipleChoice}
\choice {I}
\choice {II}
\choice {III}
\choice[correct] {IV}
\end{multipleChoice}

\end{question}


















\begin{question}


If $0 < \theta < \frac{\pi}{2}$, then $\arcsin(\sin(\theta))$ equals

\begin{multipleChoice}
\choice[correct] {$\theta$}
\choice {$\pi - \theta$}
\choice {$\pi + \theta$}
\choice {$\theta - 2\pi$}
\end{multipleChoice}

\end{question}







\begin{question}


If $\frac{\pi}{2} < \theta < \pi$, then $\arcsin(\sin(\theta))$ equals

\begin{multipleChoice}
\choice {$\theta$}
\choice[correct] {$\pi - \theta$}
\choice {$\pi + \theta$}
\choice {$\theta - 2\pi$}
\end{multipleChoice}

\end{question}






\begin{question}


If $\pi < \theta < \frac{3\pi}{2}$, then $\arcsin(\sin(\theta))$ equals

\begin{multipleChoice}
\choice {$\theta$}
\choice[correct] {$\pi - \theta$}
\choice {$\pi + \theta$}
\choice {$\theta - 2\pi$}
\end{multipleChoice}

\end{question}






\begin{question}


If $\frac{3\pi}{2} < \theta < 2\pi$, then $\arcsin(\sin(\theta))$ equals

\begin{multipleChoice}
\choice {$\theta$}
\choice {$\pi - \theta$}
\choice {$\pi + \theta$}
\choice[correct] {$\theta - 2\pi$}
\end{multipleChoice}

\end{question}


























\section{Arccosine}











The cosine function is not one-to-one.









\begin{image}
\begin{tikzpicture}
  \begin{axis}[
            domain=-10:10, ymax=1.5, xmax=10, ymin=-1.5, xmin=-10,
            axis lines =center, xlabel={$\theta$}, ylabel={$y = \cos(\theta)$}, grid = major, grid style={dashed},
            ytick={-1.5,-1,-0.5,0.5,1,1.5},
            xtick={-7.85, -6.28, -4.71, -3.14, -1.57, 0, 1.57, 3.142, 4.71, 6.28, 7.85},
            xticklabels={$-\tfrac{5\pi}{2}$,$-2\pi$,$-\tfrac{3\pi}{2}$,$-\pi$, $-\tfrac{\pi}{2}$, $0$, $\tfrac{\pi}{2}$, $\pi$, $\tfrac{3\pi}{2}$, $2\pi$, $\tfrac{5\pi}{2}$},
            yticklabels={$1.5$,$-1$,$-0.5$,$0.5$,$1$,$1.5$}, 
            ticklabel style={font=\scriptsize},
            every axis y label/.style={at=(current axis.above origin),anchor=south},
            every axis x label/.style={at=(current axis.right of origin),anchor=west},
            axis on top
          ]
          

            \addplot [line width=2, penColor, smooth,samples=300,domain=(-10:10),<->] {cos(deg(x)};
            %\addplot [line width=2, penColor2, smooth,samples=300,domain=(-10:10),<->] {cos(deg(x)};



  \end{axis}
\end{tikzpicture}
\end{image}





This function doesn't have an inverse.

We can't even restrict our domain to one period, because of the hills and valleys in the graph.

Therefore, our plan is to restrict the domain to half of one period of cosine.  We can select any interval of length $\pi$.  The usual choice is $[0,\pi]$.  














\begin{image}
\begin{tikzpicture}
  \begin{axis}[
            domain=-1:4, ymax=1.5, xmax=4, ymin=-1.5, xmin=-1,
            axis lines =center, xlabel={$\theta$}, ylabel={$y = \cos(\theta)$}, grid = major, grid style={dashed},
            ytick={-1.5,-1,-0.5,0.5,1,1.5},
            xtick={1.57,3.14},
            xticklabels={$\tfrac{\pi}{2}$, $\pi$},
            yticklabels={$1.5$,$-1$,$-0.5$,$0.5$,$1$,$1.5$}, 
            ticklabel style={font=\scriptsize},
            every axis y label/.style={at=(current axis.above origin),anchor=south},
            every axis x label/.style={at=(current axis.right of origin),anchor=west},
            axis on top
          ]
          

            \addplot [line width=2, penColor, smooth,samples=300,domain=(0:3.14)] {cos(deg(x)};
            \addplot[color=penColor,fill=penColor,only marks,mark=*] coordinates{(0,1)};
            \addplot[color=penColor,fill=penColor,only marks,mark=*] coordinates{(3.14,-1)};
            %\addplot [line width=2, penColor2, smooth,samples=300,domain=(-10:10),<->] {cos(deg(x)};



  \end{axis}
\end{tikzpicture}
\end{image}



It is only half a period, but it does cover the whole range of cosine.

We can now reverse all of the pairs and obtain the inverse function known as \textbf{arccosine}.










\begin{image}
\begin{tikzpicture}
  \begin{axis}[
            domain=-2:2, ymax=3.2, xmax=2, ymin=0, xmin=-2,
            axis lines =center, xlabel={$x$}, ylabel={$y = \arccos(x)$}, grid = major, grid style={dashed},
            xtick={-1.5,-1,-0.5,0.5,1,1.5},
            ytick={0.785,1.57,2.36,3.14},
            yticklabels={$\tfrac{\pi}{4}$, $\tfrac{\pi}{2}$, $\tfrac{3\pi}{4}$, $\pi$},
            xticklabels={$1.5$,$-1$,$-0.5$,$0.5$,$1$,$1.5$}, 
            ticklabel style={font=\scriptsize},
            every axis y label/.style={at=(current axis.above origin),anchor=south},
            every axis x label/.style={at=(current axis.right of origin),anchor=west},
            axis on top
          ]
          

            \addplot [line width=2, penColor, smooth,samples=300,domain=(0:3.14)] ({cos(deg(x)},{x});
            \addplot[color=penColor,fill=penColor,only marks,mark=*] coordinates{(-1,3.14)};
            \addplot[color=penColor,fill=penColor,only marks,mark=*] coordinates{(1,0)};
            %\addplot [line width=2, penColor2, smooth,samples=300,domain=(-10:10),<->] {cos(deg(x)};



  \end{axis}
\end{tikzpicture}
\end{image}









$Arccosine$ only returns angles between $0$ and $\pi$.  It is an inverse function for these angles.





\begin{example}



$\cos\left(\frac{2\pi}{3}\right) = -\frac{1}{2}$

$\arccos\left(-\frac{1}{2}\right) = \frac{2\pi}{3}$


$\arccos\left( \cos\left( \frac{2\pi}{3} \right) \right) = \frac{2\pi}{3}$


\end{example}




When angles are outside $\left[ 0,\pi \right]$, then the Arcsine brings them back into this interval.




\begin{example}



$\cos\left(\frac{7\pi}{6}\right) = -\frac{\sqrt{3}}{2}$

$\arccos\left(\frac{-\sqrt{3}}{2}\right) = \frac{5\pi}{6}$


$\arccos\left(\cos\left(\frac{7\pi}{6}\right)\right) = \frac{5\pi}{6}$




We begin with the angle $\frac{5\pi}{6}$ whose cosine is $-\frac{\sqrt{3}}{2}$.  Then Arccosine looks for an angle whose cosine is $-\frac{\sqrt{3}}{2}$.  But Arccosine only looks in quadrants I and II.  So, it finds $\frac{5\pi}{6}$.


\end{example}











\subsection{\textbf{\textcolor{purple!85!blue}{Characteristics}}} 

We can deduce many characteristics about arccosine from cosine.


$\blacktriangleright$ The domain is  $[-1, 1]$.


$\blacktriangleright$ The range is $[0, \pi]$


$\blacktriangleright$ Arcsin has one zero and that is $1$.


$\blacktriangleright$ Arcsin is an decreasing function.

$\blacktriangleright$ Arcsin is a continuous function.

$\blacktriangleright$ Arcsin has a maximum of $\pi$, which occurs at $-1$.

$\blacktriangleright$ Arcsin has a minimum of $0$, which occurs at $1$.








\begin{question}


If $0 < \theta < \frac{\pi}{2}$, then $\arccos(\cos(\theta))$ is in quadrant

\begin{multipleChoice}
\choice[correct] {I}
\choice {II}
\choice {III}
\choice {IV}
\end{multipleChoice}

\end{question}






\begin{question}


If $\frac{\pi}{2} < \theta < \pi$, then $\arccos(\cos(\theta))$ is in quadrant

\begin{multipleChoice}
\choice {I}
\choice[correct] {II}
\choice {III}
\choice {IV}
\end{multipleChoice}

\end{question}









\begin{question}


If $\pi < \theta < \frac{3\pi}{2}$, then $\arccos(\cos(\theta))$ is in quadrant

\begin{multipleChoice}
\choice {I}
\choice[correct] {II}
\choice {III}
\choice {IV}
\end{multipleChoice}

\end{question}







\begin{question}


If $\frac{3\pi}{2} < \theta < 2\pi$, then $\arccos(\cos(\theta))$ is in quadrant

\begin{multipleChoice}
\choice[correct] {I}
\choice {II}
\choice {III}
\choice {IV}
\end{multipleChoice}

\end{question}


















\begin{question}


If $0 < \theta < \frac{\pi}{2}$, then $\arccos(\cos(\theta))$ equals

\begin{multipleChoice}
\choice[correct] {$\theta$}
\choice {$\pi - \theta$}
\choice {$\pi + \theta$}
\choice {$2\pi - \theta$}
\end{multipleChoice}

\end{question}







\begin{question}


If $\frac{\pi}{2} < \theta < \pi$, then $\arccos(\cos(\theta))$ equals

\begin{multipleChoice}
\choice[correct] {$\theta$}
\choice {$\pi - \theta$}
\choice {$\pi + \theta$}
\choice {$2\pi - \theta$}
\end{multipleChoice}

\end{question}






\begin{question}


If $\pi < \theta < \frac{3\pi}{2}$, then $\arccos(\cos(\theta))$ equals

\begin{multipleChoice}
\choice {$\theta$}
\choice {$\pi - \theta$}
\choice {$\pi + \theta$}
\choice[correct] {$2\pi - \theta$}
\end{multipleChoice}

\end{question}






\begin{question}


If $\frac{3\pi}{2} < \theta < 2\pi$, then $\arccos(\cos(\theta))$ equals

\begin{multipleChoice}
\choice {$\theta$}
\choice {$\pi - \theta$}
\choice {$\pi + \theta$}
\choice[correct] {$2\pi - \theta$}
\end{multipleChoice}

\end{question}









\end{document}
