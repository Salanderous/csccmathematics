\documentclass{ximera}


\graphicspath{
  {./}
  {ximeraTutorial/}
  {basicPhilosophy/}
}

\newcommand{\mooculus}{\textsf{\textbf{MOOC}\textnormal{\textsf{ULUS}}}}

\usepackage{tkz-euclide}\usepackage{tikz}
\usepackage{tikz-cd}
\usetikzlibrary{arrows}
\tikzset{>=stealth,commutative diagrams/.cd,
  arrow style=tikz,diagrams={>=stealth}} %% cool arrow head
\tikzset{shorten <>/.style={ shorten >=#1, shorten <=#1 } } %% allows shorter vectors

\usetikzlibrary{backgrounds} %% for boxes around graphs
\usetikzlibrary{shapes,positioning}  %% Clouds and stars
\usetikzlibrary{matrix} %% for matrix
\usepgfplotslibrary{polar} %% for polar plots
\usepgfplotslibrary{fillbetween} %% to shade area between curves in TikZ
\usetkzobj{all}
\usepackage[makeroom]{cancel} %% for strike outs
%\usepackage{mathtools} %% for pretty underbrace % Breaks Ximera
%\usepackage{multicol}
\usepackage{pgffor} %% required for integral for loops



%% http://tex.stackexchange.com/questions/66490/drawing-a-tikz-arc-specifying-the-center
%% Draws beach ball
\tikzset{pics/carc/.style args={#1:#2:#3}{code={\draw[pic actions] (#1:#3) arc(#1:#2:#3);}}}



\usepackage{array}
\setlength{\extrarowheight}{+.1cm}
\newdimen\digitwidth
\settowidth\digitwidth{9}
\def\divrule#1#2{
\noalign{\moveright#1\digitwidth
\vbox{\hrule width#2\digitwidth}}}






\DeclareMathOperator{\arccot}{arccot}
\DeclareMathOperator{\arcsec}{arcsec}
\DeclareMathOperator{\arccsc}{arccsc}

















%%This is to help with formatting on future title pages.
\newenvironment{sectionOutcomes}{}{}


\title{FTA}

\begin{document}

\begin{abstract}
Fundamental Theory of Algebra
\end{abstract}
\maketitle






So far, our examples are very suggestive that polynomials factor into a product of linear factors.  And, there are as many linear factors as the degree of the polynomial, if you count multiplicities. \\



Our intuition is correct if we allow complex numbers into the story. 


\begin{theorem} \textbf{\textcolor{green!50!black}{The Fundamental Theorem of Algebra (Complex)}} 



Let $p(x) = a_n x^n + a_{n-1} x^{n-1} + \cdots + a_1 x + a_0$ be a polynomial of degree $n$ with complex coefficients. \\


Then $p(x)$ can be written as a product of exactly $n$ linear factors:

\[
 a (x - r_n) (x - r_{n-1}) \cdots (x - r_2)  (x - r_1) 
\]


where $r_i \in \mathbb{C}$.


\end{theorem}




If we wish to stay inside the real numbers for our coefficients, then we can factor into linear and irreducible quadratics.  Irreducible quadratics have complex roots, so cannot be factored over the reals.



\begin{theorem} \textbf{\textcolor{green!50!black}{The Fundamental Theorem of Algebra (Real)}} 



Let $p(x) = a_n x^n + a_{n-1} x^{n-1} + \cdots + a_1 x + a_0$ be a polynomial of degree $n$ with real coefficients. \\


Then $p(x)$ can be written as a product of linear and irreducible quadratic factors with real coefficients.


\end{theorem}






We'll need some information about Complex Numbers to see why the Fundamental Theorem of Algebra is true.

For now, we can get halfway there. We can say that there is no more linear factors than the degree of the polynomial.



\begin{idea} \textbf{\textcolor{blue!55!black}{Number of Factors = Degree}}  \\


\begin{itemize}
\item The degree of any polynomial written in factored form is the sum of the degress of each factor.

\[
 p(x) = p_1(x) \cdot p_2(x) \cdots p_k(x)
\]

\[
deg(p(x)) = deg(p_1(x)) + deg(p_2(x)) + \cdots + deg(p_k(x))
\]

If all of these factors were linear, then 

\[
 p(x) = a (x - r_n) (x - r_{n-1}) \cdots (x - r_2)  (x - r_1) 
\]

\item The degree of each linear factor is $1$. \\

\item The degree of 

\[
p(x) = a (x - r_n) (x - r_{n-1}) \cdots (x - r_2)  (x - r_1) 
\]

would be $1 + 1 + 1 + \cdots + 1 = n$.
\end{itemize}



Therefore, if a polynomial is written as a factorization with only linear terms, then there can't be more factors than the degree of the polynomial. \\



To get ``equals'' we need the other half.  We need that there are indeed $n$ or more linear factors.  This will need some thinking to prove.

\end{idea}

This also tells us that there cannot be more roots to a polynomial than the degree of the polynomial. \\


This was all supposing that we can factor into linear factors.  We now know a maximum number of factors. \\


What about a minimum number of factors? \\

The Fundamental Theorem of Algebra says that the minimum numbers is also $n$, when you take multiplicities into consideration. \\



We can get one step closer with some past information. \\








$\blacktriangleright$ \textbf{\textcolor{blue!55!black}{Graphs helping our algebra}} \\


Previously, we established a connection between roots, zeros, and intercepts for polynomials. \\


This is also a helpful idea for the Fundamental Theorem of Algebra. \\




\begin{idea} \textbf{\textcolor{blue!55!black}{Intercepts Correspond to Factors}}   \\



We have seen that if the graph has an intercept, then this corresponds to a zero of a function.  This would be a root of a polynomial function. \\

Let $p$ be a polynomila and $(r, 0)$ be an intercept on the graph of $y = p(x)$, then 

\[
p(x) = (x-r) \cdot q(x), \, \text{ where } \, q(x) \, \text{ is a polynomial. }
\]

Now, think of graphs of polynomials with odd degrees. \\

One side goes up, unbounded.  The other side goes down, unbounded.  The range is $\mathbb{R}$.  Which means it \textbf{MUST} have an intercept.


All polynomials of odd degree must have a linear factor.




\end{idea}


This means we really only need to worry that polynomials of even degree can be factored.  We'll need complex numbers to think further. \\


Stay tuned.


























\end{document}
