\documentclass{ximera}


\graphicspath{
  {./}
  {ximeraTutorial/}
  {basicPhilosophy/}
}

\newcommand{\mooculus}{\textsf{\textbf{MOOC}\textnormal{\textsf{ULUS}}}}

\usepackage{tkz-euclide}\usepackage{tikz}
\usepackage{tikz-cd}
\usetikzlibrary{arrows}
\tikzset{>=stealth,commutative diagrams/.cd,
  arrow style=tikz,diagrams={>=stealth}} %% cool arrow head
\tikzset{shorten <>/.style={ shorten >=#1, shorten <=#1 } } %% allows shorter vectors

\usetikzlibrary{backgrounds} %% for boxes around graphs
\usetikzlibrary{shapes,positioning}  %% Clouds and stars
\usetikzlibrary{matrix} %% for matrix
\usepgfplotslibrary{polar} %% for polar plots
\usepgfplotslibrary{fillbetween} %% to shade area between curves in TikZ
\usetkzobj{all}
\usepackage[makeroom]{cancel} %% for strike outs
%\usepackage{mathtools} %% for pretty underbrace % Breaks Ximera
%\usepackage{multicol}
\usepackage{pgffor} %% required for integral for loops



%% http://tex.stackexchange.com/questions/66490/drawing-a-tikz-arc-specifying-the-center
%% Draws beach ball
\tikzset{pics/carc/.style args={#1:#2:#3}{code={\draw[pic actions] (#1:#3) arc(#1:#2:#3);}}}



\usepackage{array}
\setlength{\extrarowheight}{+.1cm}
\newdimen\digitwidth
\settowidth\digitwidth{9}
\def\divrule#1#2{
\noalign{\moveright#1\digitwidth
\vbox{\hrule width#2\digitwidth}}}






\DeclareMathOperator{\arccot}{arccot}
\DeclareMathOperator{\arcsec}{arcsec}
\DeclareMathOperator{\arccsc}{arccsc}

















%%This is to help with formatting on future title pages.
\newenvironment{sectionOutcomes}{}{}


\title{Theory}

\begin{document}

\begin{abstract}
the polynomial story
\end{abstract}
\maketitle







In the first course, we investigated polynomial functions from several viewpoints.  Time to collect all of our thoughts and characterize all polynomial functions.





\begin{definition} \textbf{\textcolor{green!50!black}{Polynomial Functions}} 


A \textbf{polynomial function} is a function that can be represented by a formula of the form


\[   P(x) = a_n x^n + a_{n-1} x^{n-1} + \cdots + a_2 x^2 + a_1 x + a_0        \]

where the $a_i$ are real numbers and $a_n \ne 0$.

$n$ is called the \textbf{degree} of the polynomial


\end{definition}



$\blacktriangleright$ \textbf{\textcolor{red!10!blue!90!}{Domain:}} \\ 
The natural or implied domain of a polynomial function is all real numbers.  Of course, a particular polynomial function may be defined with a restricted domain..




\begin{formula} \textbf{\textcolor{purple!85!blue}{Forms}}

Formulas for polynomial functions have two two forms:



\begin{itemize}
\item \textbf{Standard Form:}  The standard form for a polynomial function is a sum. It looks like 
\[ a_n x^n + a_{n-1} x^{n-1} + \cdots + a_2 x^2 + a_1 x + a_0 \]
\item \textbf{Factored Form:}  The factored form for a polynomial function is a product. It looks like 
\[ a(x - r_n)(x - r_{n-1}) \cdots (x - r_1) \]
\end{itemize}


\end{formula}


A polynomial may have repeated zeros.  For this reason, the factored form usually collects like factors:


\[   a (x - r_k)^{e_k} (x - r_{k-1})^{e_{k-1}} \cdots (x - r_1)^{e_1}             \]


Here, the zeros are distinct:  $r_i \ne r_j$ when $i \ne j$.

The exponents are called \textbf{\textcolor{purple!85!blue}{multiplicities}} of the zeros or the factors. 






$\blacktriangleright$   \textbf{\textcolor{red!10!blue!90!}{Roots and Zeros:}} \\ 
\textbf{Zeros} of polynomials are also called \textbf{roots} of the polynomials.  A polynomial function behaves in one of two ways around a root.

\begin{itemize}
\item If the multiplicity is odd then the function changes sign over the root.  The graph crosses over the horizontal axis at the corresponding intercept.
\item If the multiplicity is even then the function does not change sign over the root.  The graph does not cross over the horizontal axis at the corresponding intercept. Instead, it bounces back in the direction from which it came.
\end{itemize}


\textbf{Constant functions} are polynomial functions.  They don't have zeros, unless it is the zero constant function.  \\


\textbf{Linear functions} are polynomial functions. They have one real root, unless it is a constant function.  You can solve for this root. \\


\textbf{Quadratic functions} are polynomial functions. They have $0$, $1$, or $2$ real roots.  You can solve for these roots. \\


\textbf{Higher Order Polynomials} are polynomial functions with degree $3$ or greater.  We do not have quick formulas to obtain their roots.  Our usual strategy is to factor, if we can.  





$\blacktriangleright$ \textbf{\textcolor{red!10!blue!90!}{Continuity:}} \\ 
Polynomials are nice functions.  They are continuous everywhere.  They have no discontinuities or singularities.



$\blacktriangleright$ \textbf{\textcolor{red!10!blue!90!}{Graphs:}} \\ 
Graphs of polynomials are nice.  They are smooth.  They do not have corners or spikes or breaks or asymptotes. Once we have the roots, we can plot the intercepts.  Then we can smoothly connect them according to their multiplicities and have a pretty good sketch of the shape of the graph.

With a basic general shape, we can estimate critical numbers and types of extrema values.




$\blacktriangleright$  \textbf{\textcolor{red!10!blue!90!}{Extrema:}} \\ 
Polynomial functions can have global and local maximums and/or minimums. If we have a derivative, then we can attempt to locate exact values of critical numbers.  Without the derivative, we turn to technology for some assistance.




$\blacktriangleright$ \textbf{\textcolor{red!10!blue!90!}{Rate-of-Change:}} \\ 
The critical numbers partition the real line into intervals where the polynomial function increases or decreases.  With the help of critical numbers, we can list these intervals.






\begin{example} Polynomial


Completely analyze $p(w) = -\frac{1}{5}(w+4)(w-3)(w-3)$




\begin{explanation}



First, let's collect like factors: $p(w) = -\frac{1}{5}(w+4)(w-3)^2$

We have a polynomial of degree $3$.  It has two roots.

\begin{itemize}
\item $-4$ is a root of multiplicity $\answer{1}$.  Since this multiplicity is odd, $p$ will \wordChoice{\choice[correct]{change sign} \choice{not change sign}} through $-4$ and the graph will cross at $(-4,0)$.
\item $3$ is a root of multiplicity $\answer{2}$.  Since this multiplicity is even, $p$ will \wordChoice{\choice{change sign} \choice[correct]{not change sign}} sign through $3$ and the graph will not cross at $(2,0)$.  The graph will touch and then bounce back.
\end{itemize}


The end-behavior of $p$ is dictated by the leading term, $-\frac{1}{5} w^3$.  Therefore, $\lim\limits_{w \to -\infty}p(w) = \infty$ and $\lim\limits_{w \to \infty}p(w) = -\infty$.



Let's collect our ideas graphically.


\begin{image}
\begin{tikzpicture}
  \begin{axis}[
            domain=-10:10, ymax=10, xmax=10, ymin=-10, xmin=-10,
            axis lines =center, xlabel=$w$, ylabel={$y=p(w)$}, grid = major, grid style={dashed},
            ytick={-10,-8,-6,-4,-2,2,4,6,8,10},
            xtick={-10,-8,-6,-4,-2,2,4,6,8,10},
            yticklabels={$-10$,$-8$,$-6$,$-4$,$-2$,$2$,$4$,$6$,$8$,$10$}, 
            xticklabels={$-10$,$-8$,$-6$,$-4$,$-2$,$2$,$4$,$6$,$8$,$10$},
            ticklabel style={font=\scriptsize},
            every axis y label/.style={at=(current axis.above origin),anchor=south},
            every axis x label/.style={at=(current axis.right of origin),anchor=west},
            axis on top
          ]
          
          %\addplot [line width=2, penColor2, smooth,samples=100,domain=(-6:2)] {-2*x-3};
            \addplot [line width=2, penColor, smooth,samples=100,domain=(-9:-7),<-] {-(x+4)};
            \addplot [line width=2, penColor, smooth,samples=100,domain=(-5:-3)] {-(x+4)};
            \addplot [line width=2, penColor, smooth,samples=100,domain=(2:4)] {-(x-3)^2)};
            \addplot [line width=2, penColor, smooth,samples=100,domain=(7:9),->] {-(x-4)};

          %\addplot[color=penColor,fill=penColor2,only marks,mark=*] coordinates{(-6,9)};
          %\addplot[color=penColor,fill=penColor2,only marks,mark=*] coordinates{(2,-7)};

          \addplot[color=penColor,fill=penColor,only marks,mark=*] coordinates{(-4,0)};
          \addplot[color=penColor,fill=penColor,only marks,mark=*] coordinates{(3,0)};


           

  \end{axis}
\end{tikzpicture}
\end{image}




The graph is very suggestive that there is a local minimum somewhere around $-1$ and a local maximum at $3$.  \\


$3$ is a zero, $p(3) = 0$. In addition, $p$ is negative around $3$.  That makes $p(3) = 0$ a local maximum.  That makes $3$ a critical number as well.






\begin{image}
\begin{tikzpicture}
  \begin{axis}[
            domain=-10:10, ymax=10, xmax=10, ymin=-10, xmin=-10,
            axis lines =center, xlabel=$w$, ylabel={$y=p(w)$}, grid = major, grid style={dashed},
            ytick={-10,-8,-6,-4,-2,2,4,6,8,10},
            xtick={-10,-8,-6,-4,-2,2,4,6,8,10},
            yticklabels={$-10$,$-8$,$-6$,$-4$,$-2$,$2$,$4$,$6$,$8$,$10$}, 
            xticklabels={$-10$,$-8$,$-6$,$-4$,$-2$,$2$,$4$,$6$,$8$,$10$},
            ticklabel style={font=\scriptsize},
            every axis y label/.style={at=(current axis.above origin),anchor=south},
            every axis x label/.style={at=(current axis.right of origin),anchor=west},
            axis on top
          ]
          
          %\addplot [line width=2, penColor2, smooth,samples=100,domain=(-6:2)] {-2*x-3};
            \addplot [line width=2, penColor, smooth,samples=300,domain=(-4.75:5.25),<->] {-0.2*(x+4)*(x-3)^2};


          %\addplot[color=penColor,fill=penColor2,only marks,mark=*] coordinates{(-6,9)};
          %\addplot[color=penColor,fill=penColor2,only marks,mark=*] coordinates{(2,-7)};

          \addplot[color=penColor,fill=penColor,only marks,mark=*] coordinates{(-4,0)};
          \addplot[color=penColor,fill=penColor,only marks,mark=*] coordinates{(3,0)};


           

  \end{axis}
\end{tikzpicture}
\end{image}

With some technology, we can approximate the other critical number to be $-1.67$ and the local minimum to be $-10.163$.


\begin{itemize}
\item $p$ \wordChoice{\choice{increasing} \choice[correct]{decreasing}}  on $(-\infty, -1.67]$.
\item $p$ \wordChoice{\choice[correct]{increasing} \choice{decreasing}}  on $[-1.67, 3]$.
\item $p$ \wordChoice{\choice{increasing} \choice[correct]{decreasing}}  on $[3, \infty)$.
\end{itemize}



There is no global maximum or minimum, because  $\lim\limits_{w \to -\infty}p(w) = \infty$ and $\lim\limits_{w \to \infty}p(w) = -\infty$. 


\end{explanation}

\end{example}


$\blacktriangleright$  \textbf{\textcolor{blue!55!black}{with some Calculus...}} \\


With some Calculus, we could get the exact values of the critical numbers. \\



If we were also given that $p'(w) = -\frac{1}{5}(3w^2 - 4w - 15)$, then its zeros would be critical numbers.  This is a quadratic.  We can obtain its zeros via the quadratic formula.


\[  \frac{4 \pm \sqrt{(-4)^2 - 4 \cdot 3 \cdot (-15)}}{2 \cdot 3} =    \frac{4 \pm \sqrt{196}}{6}  = \frac{4 \pm 14}{6}       \]

We get two real roots: $\frac{4 + 14}{6} = \frac{18}{6} = 3$  and $\frac{4 - 14}{6} = \frac{-10}{6} = \frac{-5}{3} \approx -1.67$




This allows us to factor the derivative

\[
p'(w) = -\frac{3}{5}(w - 3) \left( w + \frac{5}{3} \right)
\]

Each of these roots or factors has a multiplicy if $1$, which means $p'(w)$ changes signs over $-\frac{5}{3}$ and $3$. \\


$p'(w)$ is certainly negative for very large negative values of $w$.  Then it changes sign over $-\frac{5}{3}$.  So, $p'(w)$ is positive between $-\frac{5}{3}$ and $3$.  Then, it changes sign to negative at $3$.





\begin{itemize}
\item $p$ decreases on $\left(-\infty, \frac{-5}{3}\right]$.
\item $p$ increases on $\left[\frac{-5}{3}, 3\right]$.
\item $p$ decreases on $[3, \infty)$.
\end{itemize}



Another approach to obtaining the roots of $p'(w) = \frac{1}{5}(3w^2 - 4w - 15)$ would be to notice that we already know that $3$ is a root. $w+3$ must be a factor.



\begin{align*}
3w^2 - 4w - 15     & = 0        \\
3w^2 - 4w - 15     & = (w+3) (???)        \\
3w^2 - 4w - 15     & = (w+3) (3w + ?)        \\
3w^2 - 4w - 15     & = (w+3) \left( \answer{3w + 5} \right)        
\end{align*}














\end{document}
