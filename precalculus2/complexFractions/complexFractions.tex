\documentclass{ximera}


\graphicspath{
  {./}
  {ximeraTutorial/}
  {basicPhilosophy/}
}

\newcommand{\mooculus}{\textsf{\textbf{MOOC}\textnormal{\textsf{ULUS}}}}

\usepackage{tkz-euclide}\usepackage{tikz}
\usepackage{tikz-cd}
\usetikzlibrary{arrows}
\tikzset{>=stealth,commutative diagrams/.cd,
  arrow style=tikz,diagrams={>=stealth}} %% cool arrow head
\tikzset{shorten <>/.style={ shorten >=#1, shorten <=#1 } } %% allows shorter vectors

\usetikzlibrary{backgrounds} %% for boxes around graphs
\usetikzlibrary{shapes,positioning}  %% Clouds and stars
\usetikzlibrary{matrix} %% for matrix
\usepgfplotslibrary{polar} %% for polar plots
\usepgfplotslibrary{fillbetween} %% to shade area between curves in TikZ
\usetkzobj{all}
\usepackage[makeroom]{cancel} %% for strike outs
%\usepackage{mathtools} %% for pretty underbrace % Breaks Ximera
%\usepackage{multicol}
\usepackage{pgffor} %% required for integral for loops



%% http://tex.stackexchange.com/questions/66490/drawing-a-tikz-arc-specifying-the-center
%% Draws beach ball
\tikzset{pics/carc/.style args={#1:#2:#3}{code={\draw[pic actions] (#1:#3) arc(#1:#2:#3);}}}



\usepackage{array}
\setlength{\extrarowheight}{+.1cm}
\newdimen\digitwidth
\settowidth\digitwidth{9}
\def\divrule#1#2{
\noalign{\moveright#1\digitwidth
\vbox{\hrule width#2\digitwidth}}}






\DeclareMathOperator{\arccot}{arccot}
\DeclareMathOperator{\arcsec}{arcsec}
\DeclareMathOperator{\arccsc}{arccsc}

















%%This is to help with formatting on future title pages.
\newenvironment{sectionOutcomes}{}{}


\title{Fractions}

\begin{document}

\begin{abstract}
complex quotients
\end{abstract}
\maketitle




Complex numbers look like $a + b \, i$ where $a, b \in \mathbb{R}$.


$a$ and $b$ can be any real numbers:



\begin{itemize}
	\item $2 + 3 \, i$
	\item $\frac{2}{3} - 5 \, i$
	\item $\pi + \sqrt{7} \, i$
	\item $\frac{\sqrt{5}}{7} - \frac{\pi}{\sqrt{3}} \, i$
\end{itemize}



This makes division by a real number straight forward.




\[
\frac{a + b \, i}{c} = \frac{a}{c} + \frac{b}{c} \, i
\]





We will use this fact along with complex conjugates to give us a rule for division.



In Calculus, we don't really divide very much.  Instead this is all rephrased in terms of multiplication by the reciprocal.  We encounter division in terms of quotients.


The quotient of two Complex numbers should be a Complex number.

\[
\frac{a + b \, i}{c + d \, i}  = A + B \, i  \text{ for some real numbers } A \text{ and } B
\]


Our plan is to convert the denominator to a real number and then separate the pieces.  This is accomplished by using the complex conjugate.

\begin{definition}  \textbf{\textcolor{green!50!black}{Complex Conjugate}} \\


The complex conjugate of $a + b \, i$ is $a - b \, i$. \\


The notation for the complex conjugate is an overhead bar.

\[
\overline{a + b \, i} = a - b \, i
\]


\end{definition}


The utility of the complex conjugate is that the product of a complex number with its complex conjugate produces a positive real number.



\begin{observation}

\[
(a + b \, i) \cdot (\overline{a + b \, i}) = (a + b \, i) \cdot (a - b \, i) = a^2 + b^2
\]

\end{observation}






\[
\frac{a + b \, i}{c + d \, i} = \frac{a + b \, i}{c + d \, i}  \cdot 1 
\]

\[
= \frac{a + b \, i}{c + d \, i} \cdot \frac{c - d \, i}{c - d \, i} 
\]

\[
= \frac{(a + b \, i)(c + d \, i)}{c^2 + d^2} 
\]

\[
= \frac{(ac - bd) + (ad + bc) \, i}{c^2 + d^2} 
\]

\[
= \frac{(ac - bd)}{c^2 + d^2}  + \frac{(ad + bc)}{c^2 + d^2} \, i
\]







\begin{procedure}


How do we convert a complex fraction, $\frac{a + b \, i}{c + d \, i}$ into standard form, $A + B \, i$?



$\blacktriangleright$ \textbf{Step 1:}   Get the complex conjugate of the denominator, $c - d \, i$.


$\blacktriangleright$ \textbf{Step 2:} Mulitply by $1$ in the form $\frac{c - d \, i}{c - d \, i}$.

\[
\frac{a + b \, i}{c + d \, i} \cdot \frac{c - d \, i}{c - d \, i}
\]

$\blacktriangleright$ \textbf{Step 3:}  The denominator will become a real number:  

\[ (c + d \, i) \cdot (c - d \, i) = c^2 + d^2 \]



$\blacktriangleright$ \textbf{Step 4:}  Expand the numerator and collect real and imaginary parts: 

\[ (a + b \, i) \cdot (c - d \, i) = (ac-bd) + (ad + bc) \, i \]


$\blacktriangleright$ \textbf{Step 5:} Separate the real and imaginary parts: 

\[
\frac{(ac-bd) + (ad + bc) \, i}{c^2 + d^2} = \frac{ac - bd}{c^2 + d^2} + \frac{ad + bc}{c^2 + d^2} \, i
\] 



\end{procedure}





\begin{example}




Write $\frac{2 + 2 \, i}{5 - i}$ as $A + B \, i$


\begin{explanation}

\[
\frac{2 + 2 \, i}{5 - i} = \frac{2 + 2 \, i}{5 - i} \cdot \frac{5 + i}{5 + i} = \frac{(2 + 2 \, i)(5 + i)}{5^2 + 1^2} = \frac{8 + 12 \, i}{5^2 + 1^2} = \frac{8}{26} + \frac{12}{26} \, i
\]


\end{explanation}


\end{example}










\begin{question}


Write $\frac{1 + 2 \, i}{3 - 2i}$ as $A + B \, i$



\[
\frac{1 + 2 \, i}{3 - 2i} = \frac{1 + 2 \, i}{3 - 2i} \cdot  1
\]


\[
\frac{1 + 2 \, i}{3 - 2i} = \frac{1 + 2 \, i}{3 - 2i} \cdot  \frac{\answer{3 + 2i}}{\answer{3 + 2i}}
\]


\[
\frac{1 + 2 \, i}{3 - 2i} = \frac{(1 + 2 \, i)(3 + 2i)}{\answer{13}} 
\]


\[
\frac{1 + 2 \, i}{3 - 2i} = \frac{\answer{-1} + \answer{8} \, i}{13} 
\]


\[
\frac{1 + 2 \, i}{3 - 2i} = \answer{\frac{-1}{13}} + \answer{\frac{8}{13}} \, i
\]

\end{question}







We now have a full 2-dimensional number system.  We have extended all of our arithmetic operations from the real numbers to the complex numbers.  We can perform all of our arithmetic with Complex Numbers.



We have a 2-dimensional number line. We can plot all of our Complex Numbers and compare them.

The real number line had two directions and multiplication by $-1$ would flip between them.  Now we have an infinite number of directions and we see that multiplication by $i$ makes quarter-circle turns.

From the quadratic formula, we can see that the complex numbers iron out all of our quadratic questions.

\begin{center}
\textbf{\textcolor{red!80!black}{When you include multiplicity, all quadratic functions have exactly $2$ zeros or roots.}}
\end{center}

























\end{document}
