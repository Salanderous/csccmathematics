\documentclass{ximera}


\graphicspath{
  {./}
  {ximeraTutorial/}
  {basicPhilosophy/}
}

\newcommand{\mooculus}{\textsf{\textbf{MOOC}\textnormal{\textsf{ULUS}}}}

\usepackage{tkz-euclide}\usepackage{tikz}
\usepackage{tikz-cd}
\usetikzlibrary{arrows}
\tikzset{>=stealth,commutative diagrams/.cd,
  arrow style=tikz,diagrams={>=stealth}} %% cool arrow head
\tikzset{shorten <>/.style={ shorten >=#1, shorten <=#1 } } %% allows shorter vectors

\usetikzlibrary{backgrounds} %% for boxes around graphs
\usetikzlibrary{shapes,positioning}  %% Clouds and stars
\usetikzlibrary{matrix} %% for matrix
\usepgfplotslibrary{polar} %% for polar plots
\usepgfplotslibrary{fillbetween} %% to shade area between curves in TikZ
\usetkzobj{all}
\usepackage[makeroom]{cancel} %% for strike outs
%\usepackage{mathtools} %% for pretty underbrace % Breaks Ximera
%\usepackage{multicol}
\usepackage{pgffor} %% required for integral for loops



%% http://tex.stackexchange.com/questions/66490/drawing-a-tikz-arc-specifying-the-center
%% Draws beach ball
\tikzset{pics/carc/.style args={#1:#2:#3}{code={\draw[pic actions] (#1:#3) arc(#1:#2:#3);}}}



\usepackage{array}
\setlength{\extrarowheight}{+.1cm}
\newdimen\digitwidth
\settowidth\digitwidth{9}
\def\divrule#1#2{
\noalign{\moveright#1\digitwidth
\vbox{\hrule width#2\digitwidth}}}






\DeclareMathOperator{\arccot}{arccot}
\DeclareMathOperator{\arcsec}{arcsec}
\DeclareMathOperator{\arccsc}{arccsc}

















%%This is to help with formatting on future title pages.
\newenvironment{sectionOutcomes}{}{}


\title{Rate of Change}

\begin{document}

\begin{abstract}
no formula
\end{abstract}
\maketitle




Let $y(x)$ be described by $x^3 + y^3 - 3 x y = 0$. \textbf{NOte: } It isn't a function.



The graph of $x^3 + y^3 - 3 x y = 0$ looks like




\begin{center}
\desmos{3lydfchapn}{400}{300}
\end{center}


As you can see, no matter how you restrict the domain alone around the loop, it still isn't a function. The range must also be restricted in conjunction with the domain.

However, with suitable domain and range restrictions, $y(x)$ can be a function. \\


Once we have refocused our attention to just a piece of the curve and we have a working function, we can talk about its instantaneous rate of change.  The idea is the same as before.  The value of the derivative at a domain number is the slope of the associated tangent line.


At this point we are approximating with our graph via technology, because our algebra is unable to break this cubic apart.



Fortunately, Calculus can lend a hand.




\section{with Calculus}

Let's consider $y(1)$.  There are three scenarios for our function $y$.


\begin{center}
\desmos{83u6jqgicu}{400}{300}
\end{center}

Each scenario has its own restricted range, which then surrounds one of the three points on the curve. What are the coordinates of these points?



With $x=1$, our equation becomes $y^3 - 3 y + 1 = 0$ and we would like to solve this equation. From the graph, it appears that the solutions are approximately $-1.9$, $0.35$, and $1.5$.

How might we obtain better approximations to these solutions?





Let's switch how we are viewing $y^3 - 3 y + 1 = 0$.  Instead of thinking of solving the equation, which is an algebraic viewpoint, let's think in terms of functions.

Let's make a new function called, $f$, with $f(y) = y^3 - 3 y + 1$.  Now we are thinking of function zeros. \\








\subsection{a Function Viewpoint}


Well, how well does $1.5$ do as a zero of the function $f(y) = y^3 - 3 y + 1$ ?


\[  f(1.5) = 1.5^3 - 3 \cdot 1.5 + 1 = -0.125 \]


How should we change $1.5$ to obtain a better approximation to the actual zero of $f$? \\


We notice that $f(1.5)$ is negative - too low.  We need to increase $f$ a little bit.  How do we do this?  Well, it depends on the rate of change of $f$ around $1.5$.  If $f'(1.5) > 0$, then $f$ is increasing at $1.5$, so move to the right a little bit. If $f'(1.5) < 0$, then $f$ is decreasing at $1.5$, so move to the left a little bit. 



We need a derivative.

And, now we are in Calculus.





The good news is that $f(y) = y^3 - 3 y + 1$ is a polynomial and Calculus will present a nice procedure to get derivatives of polynomials.


\[
f'(y) = 3 y^2 - 3
\]




\textbf{Recap: } Remember we are examing $y(x)$, where $x^3 + y^3 - 3xy = 0$.  We have narrowed our attention to $x=1$, which presented three choices. We have selected the corresponding value near $1.5$.  We have $y(1) \approx 1.5$. We would like a better approximation.  To do this, we have changed our viewpoint, from algebraic to functional.  We have invented a new function $f(y) = y^3 - 3 y + 1$ and we would like its zero near $1.5$.

Whoa!  We were looking for function values of $y$ and now $y$ is a variable.  What is going on?

This is the whole point of implicitly defined functions.  They are described via equations, where there is really no obvious choice for independent and dependent variable.  It just depends on how you want to look at it.




While we do not have the algebra tools to solve for these values exactly, Calculus allows us to inch closer.



\begin{explanation} Approximating


Remember, we are examining $y$ near $x = 1$.   More specifically, we are restrict our range to a small interval around $y = 1.5$.  

We set $x = 1$ and obtained a function $f(y)$ and we are looking for its zero, which should be around $1.5$.


Remember, our zero was off by $-0.125$.  We need the value of $f$ to change by $0.125$.


Now, we have a derivative, $f'(1.5) = 3 (1.5)^2 - 3 = 3.75$


The derivative tells us that $\frac{\Delta f}{\Delta y} \approx 3.75$. \\

Therefore, if we wish $\Delta f = 0.125$, then we need 



\[
\frac{0.125}{\Delta y} \approx 3.75
\]


\[
\Delta y \approx \frac{0.125}{3.75} = \frac{1}{30}
\]


Try $y = 1.5 + \frac{1}{30} = \frac{23}{15}$.


\[  f\left( \frac{23}{15} \right) \approx 0.005037037 \]


Pretty good!


Our point should be close to $\left(1, \frac{23}{15} \right) \approx (1, 1.53333)$


The derivative allows us to inch our way closer and closer and closer to values our algebra cannot reach.

\end{explanation}



Whew!  That was quite the journey to get a good approximation to our point.

\[ y(1) \approx 1.53333   \]


We made a new function, $f$, and used that to help with $y$.  $f$ and $f'$ were really algebraic tools to help us analyze $y$. Why not skip $f$ and just jump to $y'(x)$  Then, we can just get $y'(1)$?

The reason is that we cannot get a formula for $y(x)$, so that we can get a formula for $y'(x)$. Or, is there a way around this problem?



\textbf{\textcolor{red!80!black}{$\blacktriangleright$  [Weird Stuff]}}  Even though we cannot solve for $y$ and get a formula for $y(x)$, Calculus does give us a way to obtain a FORMULA for $y'(x)$.



\[
y'(x) = \frac{y - x^2}{y^2 - x}
\]


It is a weird formula, because it involves both $x$ and $y$.  The analysis of implicit functions is not quite as clean as with explicitly defined functions.


\[
y'(1) \approx \frac{1.53333 - 1^2}{1.5333^2 - 1} \approx 0.394737
\]



From this we can get an approximate tangent line.

\[
y - 1.53333 = 0.394737(x-1)
\]


\[
y = 0.394737(x-1) + 1.53333
\]


Now we can move a little bit around $x=1$ and watch the values of $y$.



\begin{center}
\desmos{miclho6mgx}{400}{300}
\end{center}






\section{without Calculus}



Calculus gives us the derivative, which tells us how to change domain values to affect change in function values.  This allows us to move around proportionaly.  

This is called \textbf{linearization} and is studied later in the course.

Without Calculus, it is difficult to talk algebraically about changing values.  However, technology will crunch the numbers much quicker and allow us to think about the larger picture, approximately.



\begin{example}

Define $y(x)$ by $x^3 + y^3 - 3 x y = 0$ with domain around $1$ and range around $1.5$.


We can work graphically with DESMOS to obtain numeric information, which we can then use to create equations of tangent lines.







\begin{center}
\desmos{83u6jqgicu}{400}{300}
\end{center}



What about a derivative?

We can trace the curve in DESMOS and obtain nearby approximations.


We can zoom in on the point and use the gridlines to approximate coordinates.






\begin{center}
\desmos{wkzdfrdom3}{400}{300}
\end{center}

It appears that $y(1) \approx 1.532$.



We can zoom in until the graph itself looks like a tangent line.



\begin{center}
\desmos{agpg5z5wwj}{400}{300}
\end{center}

A couple of points might be $(0.995, 1.5302)$ and $(1.005, 1.534)$.

That gives us an approximate slope of 

\[
\frac{1.534 - 1.5302}{1.005 - 0.995} = \frac{0.0038}{0.01} = 0.38
\]


Pretty good!


\end{example}



We can think analytically with technology, but we are approximating.  Calculus will give us algebraic tools to think analytically in an exact way.






\begin{center}
\textbf{\textcolor{green!50!black}{ooooo=-=-=-=-=-=-=-=-=-=-=-=-=ooOoo=-=-=-=-=-=-=-=-=-=-=-=-=ooooo}} \\

more examples can be found by following this link\\ \link[More Examples of Implicit Functions]{https://ximera.osu.edu/csccmathematics/precalculus2/precalculus2/implicitFunctions/examples/exampleList}

\end{center}



\end{document}
