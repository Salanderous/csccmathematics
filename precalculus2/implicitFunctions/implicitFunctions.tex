\documentclass{ximera}


\graphicspath{
  {./}
  {ximeraTutorial/}
  {basicPhilosophy/}
}

\newcommand{\mooculus}{\textsf{\textbf{MOOC}\textnormal{\textsf{ULUS}}}}

\usepackage{tkz-euclide}\usepackage{tikz}
\usepackage{tikz-cd}
\usetikzlibrary{arrows}
\tikzset{>=stealth,commutative diagrams/.cd,
  arrow style=tikz,diagrams={>=stealth}} %% cool arrow head
\tikzset{shorten <>/.style={ shorten >=#1, shorten <=#1 } } %% allows shorter vectors

\usetikzlibrary{backgrounds} %% for boxes around graphs
\usetikzlibrary{shapes,positioning}  %% Clouds and stars
\usetikzlibrary{matrix} %% for matrix
\usepgfplotslibrary{polar} %% for polar plots
\usepgfplotslibrary{fillbetween} %% to shade area between curves in TikZ
\usetkzobj{all}
\usepackage[makeroom]{cancel} %% for strike outs
%\usepackage{mathtools} %% for pretty underbrace % Breaks Ximera
%\usepackage{multicol}
\usepackage{pgffor} %% required for integral for loops



%% http://tex.stackexchange.com/questions/66490/drawing-a-tikz-arc-specifying-the-center
%% Draws beach ball
\tikzset{pics/carc/.style args={#1:#2:#3}{code={\draw[pic actions] (#1:#3) arc(#1:#2:#3);}}}



\usepackage{array}
\setlength{\extrarowheight}{+.1cm}
\newdimen\digitwidth
\settowidth\digitwidth{9}
\def\divrule#1#2{
\noalign{\moveright#1\digitwidth
\vbox{\hrule width#2\digitwidth}}}






\DeclareMathOperator{\arccot}{arccot}
\DeclareMathOperator{\arcsec}{arcsec}
\DeclareMathOperator{\arccsc}{arccsc}

















%%This is to help with formatting on future title pages.
\newenvironment{sectionOutcomes}{}{}


\title{Implicit Functions}

\begin{document}

\begin{abstract}
no formula
\end{abstract}
\maketitle






Let $y(x)$ be a function whose values are given by the equaiton $x^2 + y^2 = 1$.

What is the value of $y(0)$? \\

The value of $y(0)$ is the value of $y$ that makes the equation $0^2 + y^2 = 1$ true. \\

There are two of them: $y(0) = -1$ and $y(0) = -1$. \\


Whoops!  $y(x)$ is not a function, which we can easily see from the graph.






\begin{image}
\begin{tikzpicture}
  \begin{axis}[
            xmin=-1.1,xmax=1.1,ymin=-1.1,ymax=1.1,
            axis lines=center,
            width=4in,
            xtick={-1,1},
            ytick={-1,1},
            clip=false,
            unit vector ratio*=1 1 1,
            xlabel=$x$, ylabel=$y$,
            every axis y label/.style={at=(current axis.above origin),anchor=south},
            every axis x label/.style={at=(current axis.right of origin),anchor=west},
          ]        

          \addplot [thick,penColor,smooth,domain=(0:360)] ({cos(x)},{sin(x)}); %% unit circle

          %\addplot[color=black,fill=black,only marks,mark=*] coordinates{(0.906,0.423)};


        \end{axis}
\end{tikzpicture}
\end{image}


Most values for $x$ have two corresponding values for $y$. \\







\subsection{Restrictions}


Suppose we are really interested in negative values of $y(x)$ for domain values around $x = 0.6$.  


We are interested in this part of the graph.





\begin{image}
\begin{tikzpicture}
  \begin{axis}[
            xmin=-1.1,xmax=1.1,ymin=-1.1,ymax=1.1,
            axis lines=center,
            width=4in,
            xtick={-1,1},
            ytick={-1,1},
            clip=false,
            unit vector ratio*=1 1 1,
            xlabel=$x$, ylabel=$y$,
            every axis y label/.style={at=(current axis.above origin),anchor=south},
            every axis x label/.style={at=(current axis.right of origin),anchor=west},
          ]        

          \addplot [thick,penColor,smooth,domain=(0:360)] ({cos(x)},{sin(x)}); %% unit circle
          \addplot [ultra thick,penColor2,smooth,domain=(290:320)] ({cos(x)},{sin(x)}); %% unit circle


          \addplot[color=black,fill=black,only marks,mark=*] coordinates{(0.6,-0.8)};


        \end{axis}
\end{tikzpicture}
\end{image}






Then we could create a restricted version of $y$ by selecting an appropriate domain and range.










\begin{image}
\begin{tikzpicture}
  \begin{axis}[
            xmin=-1.1,xmax=1.1,ymin=-1.1,ymax=1.1,
            axis lines=center,
            width=4in,
            xtick={-1,1},
            ytick={-1,1},
            clip=false,
            unit vector ratio*=1 1 1,
            xlabel=$x$, ylabel=$y$,
            every axis y label/.style={at=(current axis.above origin),anchor=south},
            every axis x label/.style={at=(current axis.right of origin),anchor=west},
          ]        

          \addplot [thick,penColor,smooth,domain=(0:360)] ({cos(x)},{sin(x)}); %% unit circle
          \addplot [ultra thick,penColor2,smooth,domain=(290:320)] ({cos(x)},{sin(x)}); %% unit circle


          \addplot[color=black,fill=black,only marks,mark=*] coordinates{(0.6,-0.8)};

          \addplot [smooth, domain=(-0.939:-0.642)] ({0.342},{x});
          \addplot [smooth, domain=(-0.939:-0.642)] ({0.766},{x});
          \addplot [smooth, domain=(0.342:0.766)] ({x},{-0.642});
          \addplot [smooth, domain=(0.342:0.766)] ({x},{-0.939});


        \end{axis}
\end{tikzpicture}
\end{image}




We have selected a domain and range the isolates the part of the curve around our point in such a way that each $x$-value has a single (unique) corresponding $y$-value. \\



$\blacktriangleright$ New domain is $(0.342, 0.766)$


$\blacktriangleright$ New range is $(-0.939, -0.642)$


With this new domain and range and our equation, $y(x)$ is a function whose pairs are described by $x^2 + y^2 = 1$.



\section{Solving}


We cannot solve $x^2 + y^2 = 1$, but we can solve it for two versions of $y$.


$\blacktriangleright$ \textbf{Version 1:} $y = \sqrt{1-x^2}$

$\blacktriangleright$ \textbf{Version 2:} $y = -\sqrt{1-x^2}$

For our situation above, we could have used $y = -\sqrt{1-x^2}$ with domain $[-1, 1]$.  Our graph would look like 





\begin{image}
\begin{tikzpicture}
  \begin{axis}[
            xmin=-1.1,xmax=1.1,ymin=-1.1,ymax=1.1,
            axis lines=center,
            width=4in,
            xtick={-1,1},
            ytick={-1,1},
            clip=false,
            unit vector ratio*=1 1 1,
            xlabel=$x$, ylabel=$y$,
            every axis y label/.style={at=(current axis.above origin),anchor=south},
            every axis x label/.style={at=(current axis.right of origin),anchor=west},
          ]        

          \addplot [thick,penColor,smooth,domain=(180:360)] ({cos(x)},{sin(x)}); %% unit circle



        \end{axis}
\end{tikzpicture}
\end{image}




However, this is not always possible. \\


The graph of $x^3 + y^3 - 3 x y = 0$ looks like




\begin{center}
\desmos{bz5gq9mpkn}{400}{300}
\end{center}


There is no way to get $y$ by itself on one side of the equation.  We are stuck with the equation $x^3 + y^3 - 3xy = 0$.



\begin{example}


The curve described by $(x^2 + y^2)^2 = 20 x y^2$ looks like



\begin{center}
\desmos{9ujcpluhpr}{400}{300}
\end{center}


We cannot isolate $y$.  However, the equation involves squares and squares or squares.  It might be possible to apply the quadratic formula.


\[ (x^2 + y^2)^2 = 20 x y^2  \]

\[ x^4 + 2 x^2 y^2 + y^4 = 20 x y^2  \]

\[ y^4 + (2 x^2 - 20 x) y^2 + x^4 = 0 \]




This is a quadratic in $y^2$.  We could split it into two pieces.  



\[ y^2 = \frac{-(2 x^2 - 20 x) + \sqrt{(2x^2-20x)^2 - 4 \cdot 1 \cdot x^4}}{2}   \]


\[ y^2 = \frac{-(2 x^2 - 20 x) - \sqrt{(2x^2-20x)^2 - 4 \cdot 1 \cdot x^4}}{2}   \]



And then, each of these can be broken down.




\[ y = \sqrt{\frac{-(2 x^2 - 20 x) + \sqrt{(2x^2-20x)^2 - 4 \cdot 1 \cdot x^4}}{2}}   \]


\[ y = -\sqrt{\frac{-(2 x^2 - 20 x) + \sqrt{(2x^2-20x)^2 - 4 \cdot 1 \cdot x^4}}{2}}   \]



\[ y^2 = \sqrt{\frac{-(2 x^2 - 20 x) - \sqrt{(2x^2-20x)^2 - 4 \cdot 1 \cdot x^4}}{2}}   \]


\[ y^2 = -\sqrt{\frac{-(2 x^2 - 20 x) - \sqrt{(2x^2-20x)^2 - 4 \cdot 1 \cdot x^4}}{2}}   \]





\begin{center}
\desmos{x68yk97cik}{400}{300}
\end{center}


If our equation only involves squares, then we have a chance to break it into pieces using the quadratic formula.  However, we don't have any other ``formulas'' for other types of expressions.  Generally, other types of equations stay as they are.


\end{example}






















\end{document}
