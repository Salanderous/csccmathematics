\documentclass{ximera}


\graphicspath{
  {./}
  {ximeraTutorial/}
  {basicPhilosophy/}
}

\newcommand{\mooculus}{\textsf{\textbf{MOOC}\textnormal{\textsf{ULUS}}}}

\usepackage{tkz-euclide}\usepackage{tikz}
\usepackage{tikz-cd}
\usetikzlibrary{arrows}
\tikzset{>=stealth,commutative diagrams/.cd,
  arrow style=tikz,diagrams={>=stealth}} %% cool arrow head
\tikzset{shorten <>/.style={ shorten >=#1, shorten <=#1 } } %% allows shorter vectors

\usetikzlibrary{backgrounds} %% for boxes around graphs
\usetikzlibrary{shapes,positioning}  %% Clouds and stars
\usetikzlibrary{matrix} %% for matrix
\usepgfplotslibrary{polar} %% for polar plots
\usepgfplotslibrary{fillbetween} %% to shade area between curves in TikZ
\usetkzobj{all}
\usepackage[makeroom]{cancel} %% for strike outs
%\usepackage{mathtools} %% for pretty underbrace % Breaks Ximera
%\usepackage{multicol}
\usepackage{pgffor} %% required for integral for loops



%% http://tex.stackexchange.com/questions/66490/drawing-a-tikz-arc-specifying-the-center
%% Draws beach ball
\tikzset{pics/carc/.style args={#1:#2:#3}{code={\draw[pic actions] (#1:#3) arc(#1:#2:#3);}}}



\usepackage{array}
\setlength{\extrarowheight}{+.1cm}
\newdimen\digitwidth
\settowidth\digitwidth{9}
\def\divrule#1#2{
\noalign{\moveright#1\digitwidth
\vbox{\hrule width#2\digitwidth}}}






\DeclareMathOperator{\arccot}{arccot}
\DeclareMathOperator{\arcsec}{arcsec}
\DeclareMathOperator{\arccsc}{arccsc}

















%%This is to help with formatting on future title pages.
\newenvironment{sectionOutcomes}{}{}


\title{Holes}

\begin{document}

\begin{abstract}
need more numbers
\end{abstract}
\maketitle




We have investigated quadratic functions from several viewpoints.  Time to collect all of our thoughts and charcterize quadratic functions.



Complex Numbers




\begin{example}  Quadratic Analysis



Analyze $M(t) = -\frac{1}{2} (t-3)^2 + 5$ \\


\begin{explanation}

$M$ is a quadratic function, which means its graph is a parabola.  The leading coefficient is negative, which means the parabola is opening down.  The highest point on the parabola is $(3, 5)$. From this, we know there are two intercepts, which means two roots and two factors.






\begin{image}
\begin{tikzpicture}
  \begin{axis}[
            domain=-10:10, ymax=10, xmax=10, ymin=-10, xmin=-10,
            axis lines =center, xlabel=$t$, ylabel={$y=M(t)$}, grid = major, grid style={dashed},
            ytick={-10,-8,-6,-4,-2,2,4,6,8,10},
            xtick={-10,-8,-6,-4,-2,2,4,6,8,10},
            yticklabels={$-10$,$-8$,$-6$,$-4$,$-2$,$2$,$4$,$6$,$8$,$10$}, 
            xticklabels={$-10$,$-8$,$-6$,$-4$,$-2$,$2$,$4$,$6$,$8$,$10$},
            ticklabel style={font=\scriptsize},
            every axis y label/.style={at=(current axis.above origin),anchor=south},
            every axis x label/.style={at=(current axis.right of origin),anchor=west},
            axis on top
          ]
          
          %\addplot [line width=2, penColor2, smooth,samples=100,domain=(-6:2)] {-2*x-3};
          \addplot [line width=2, penColor, smooth,samples=200,domain=(-2:8),<->] {-0.5*(x-3)^2 + 5};
          %\addplot [line width=2, penColor, smooth,samples=200,domain=(-7:-4),<-] {-e^(-x-5)};

          %\addplot[color=penColor,fill=penColor2,only marks,mark=*] coordinates{(-6,9)};
          %\addplot[color=penColor,fill=penColor2,only marks,mark=*] coordinates{(2,-7)};

          \addplot[color=penColor,fill=penColor,only marks,mark=*] coordinates{(3,5)};
          \addplot[color=penColor,fill=penColor,only marks,mark=*] coordinates{(-0.162,0)};
          \addplot[color=penColor,fill=penColor,only marks,mark=*] coordinates{(6.162,0)};



           

  \end{axis}
\end{tikzpicture}
\end{image}



The domain is $(-\infty, \infty)$ and the range is $(-\infty, 5]$.

The global maximum is $5$, which ocurrs at $3$.  There is no global minimum.  $5$ is also a local maximum and there are no local minimums.




$\blacktriangleright$ Behavior: \\


$M$ increases on $(-\infty, 3]$. $M$ decreases on $[3, \infty)$.



$\blacktriangleright$ Zeros: \\





\begin{align*}
-\frac{1}{2} (t-3)^2 + 5 & = 0  \\
(t-3)^2     &= \answer{10)} 
\end{align*}

Either $t - 3 = -\sqrt{10}$ or $t - 3 = \answer{\sqrt{10}}$

EIther $t = 3 - \sqrt{10}$ or $t = \answer{3 - \sqrt{10}}$






We have two roots or zeros: $t = 3 - \sqrt{10}$ and $t = \answer{3 + \sqrt{10}}$. \\


We have two intercepts: $(3 - \sqrt{10}, 0)$ and $\left( \answer{3 + \sqrt{10}}, 0 \right)$. \\


We have two factors: $t - (3 - \sqrt{10})$ and $\answer{t - (3 - \sqrt{10})}$. \\


$M(t) = -\frac{1}{2} (t - (3 - \sqrt{10})) (t - (3 + \sqrt{10}))$ \\

$M(t) =  -\frac{1}{2} t^2 + 3x + \frac{1}{2}$


\end{explanation}

\end{example}


















\end{document}
