\documentclass{ximera}


\graphicspath{
  {./}
  {ximeraTutorial/}
  {basicPhilosophy/}
}

\newcommand{\mooculus}{\textsf{\textbf{MOOC}\textnormal{\textsf{ULUS}}}}

\usepackage{tkz-euclide}\usepackage{tikz}
\usepackage{tikz-cd}
\usetikzlibrary{arrows}
\tikzset{>=stealth,commutative diagrams/.cd,
  arrow style=tikz,diagrams={>=stealth}} %% cool arrow head
\tikzset{shorten <>/.style={ shorten >=#1, shorten <=#1 } } %% allows shorter vectors

\usetikzlibrary{backgrounds} %% for boxes around graphs
\usetikzlibrary{shapes,positioning}  %% Clouds and stars
\usetikzlibrary{matrix} %% for matrix
\usepgfplotslibrary{polar} %% for polar plots
\usepgfplotslibrary{fillbetween} %% to shade area between curves in TikZ
\usetkzobj{all}
\usepackage[makeroom]{cancel} %% for strike outs
%\usepackage{mathtools} %% for pretty underbrace % Breaks Ximera
%\usepackage{multicol}
\usepackage{pgffor} %% required for integral for loops



%% http://tex.stackexchange.com/questions/66490/drawing-a-tikz-arc-specifying-the-center
%% Draws beach ball
\tikzset{pics/carc/.style args={#1:#2:#3}{code={\draw[pic actions] (#1:#3) arc(#1:#2:#3);}}}



\usepackage{array}
\setlength{\extrarowheight}{+.1cm}
\newdimen\digitwidth
\settowidth\digitwidth{9}
\def\divrule#1#2{
\noalign{\moveright#1\digitwidth
\vbox{\hrule width#2\digitwidth}}}






\DeclareMathOperator{\arccot}{arccot}
\DeclareMathOperator{\arcsec}{arcsec}
\DeclareMathOperator{\arccsc}{arccsc}

















%%This is to help with formatting on future title pages.
\newenvironment{sectionOutcomes}{}{}


\title{Pieces}

\begin{document}

\begin{abstract}
piecewise defined
\end{abstract}
\maketitle



We combine numbers together with the operations addition, subtraction, multiplication, and division.  We can combine functions by applying these same operations on the function values.  \\

In this way, we can build new functions from old funcitons. \\

We can also build new functions by gluing together pieces of old functions.



\begin{template} \textbf{\textcolor{blue!55!black}{Piecewise Defined Functions}} \\


Suppose we have several functions with their domains.

\begin{itemize}
\item Function $F$ with domain $D_F$. 
\item Function $G$ with domain $D_G$. 
\item Function $H$ with domain $D_H$. 
\end{itemize}


We can create a new function, $N$, by gluing together pieces of $F$, $G$, and $H$.


First, select disjoint subsets of the three given domains.


\begin{itemize}
\item $S_F \subseteq D_F$. 
\item $S_G \subseteq D_G$. 
\item $S_H \subseteq D_H$. 
\end{itemize}


The domain of $N$ will be the union of these three subsets: $S_F \cup S_G \cup S_H$. \\

When evaluating $N(a)$, use the formula from $F$, $G$, or $H$ that goes with the domain subset holding $a$.


\end{template}

$N$ is called a \textbf{\textcolor{blue!55!black}{piecewise defined function}}. \\




\begin{notation}


The notation for a piecewise defined function must list the formulas to use along with the somain subset corresponding to formula.



\[
N(x) = 
\begin{cases}
  F & S_F     \\
  G & S_G \\
  H & S_H
\end{cases}
\]



The right column lists the subsets of the original domains.  \\
The left column lists the orignal formulas corresponding to the subdomains.

\end{notation}






\begin{example}

Define $N$ as follows.


\[
N(x) = 
\begin{cases}
  2x - 1   &    [-8, 0)     \\
  3 - x^2  &    (0, 2) \\
  \ln(x)    &    (5, 9]
\end{cases}
\]


\begin{explanation}

The domain of $N$ is the union of the given intervals: $[-8, 0) \cup (0,2) \cup (5, 9]$.


To evaluate $N(d)$, first identify to which interval $d$ belongs.  Then, use the corresponding formula to evaluate.







\end{explanation}




To evaluate $N(-4)$, first identify that $-4 \in [-8, 0)$.  That tells us to use the formula $2x - 1$. \\

\[  N(-4) = 2(-4) - 1 = -9  \]





To evaluate $N(1)$, first identify that $1 \in (0, 2)$.  That tells us to use the formula $3 - x^2$. \\

\[  N(1) = 3 - (1)^1 = 2  \]





To evaluate $N(7)$, first identify that $7 \in (5, 9]$.  That tells us to use the formula $\ln(x)$. \\

\[  N(-7) = \ln(7)  \]



\end{example}














\begin{example}

Define $N$ as follows.


\[
N(x) = 
\begin{cases}
  2x - 1   &    [-8, 0)     \\
  3 - x^2  &    (0, 2) \\
  \ln(x)    &    (5, 9]
\end{cases}
\]







\begin{explanation}


To evaluate $\lim\limits_{x \to 0^- N(x)}$, we first notice that we are only considering numbers in the interval $[8, 0)$.


\[
\lim\limits_{x \to 0^-} N(x) = \lim\limits_{x \to 0^-} 2x - 1 = -1
\]



\end{explanation}









\begin{explanation}


To evaluate $\lim\limits_{x \to 0^+ N(x)}$, we first notice that we are only considering numbers in the interval $(0,2)$.


\[
\lim\limits_{x \to 0^+} N(x) = \lim\limits_{x \to 0^+} 3 - x^2 = 3
\]



\end{explanation}





\end{example}

























\begin{example}

Define $F$ as follows.


\[
F(t) = 
\begin{cases}
  e^x - x   &    [-10, -3)     \\
  6x + 1  &    (-1, 5) \\
  4 \ln(x+2)    &    (5, 9]
\end{cases}
\]




\begin{question}

\[
F(-5) = \answer{e^(-5) + 5}
\]


\end{question}




\begin{question}

\[
F(0) = \answer{1}
\]


\end{question}






\begin{question}

\[
F(7) = \answer{4 \ln(9)}
\]


\end{question}




\end{example}


















\begin{center}
\textbf{\textcolor{green!50!black}{ooooo=-=-=-=-=-=-=-=-=-=-=-=-=ooOoo=-=-=-=-=-=-=-=-=-=-=-=-=ooooo}} \\

more examples can be found by following this link\\ \link[More Examples of the Functionm Forms]{https://ximera.osu.edu/csccmathematics/precalculus2/precalculus2/functionForm/examples/exampleList}

\end{center}







\end{document}
