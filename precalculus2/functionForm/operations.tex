\documentclass{ximera}


\graphicspath{
  {./}
  {ximeraTutorial/}
  {basicPhilosophy/}
}

\newcommand{\mooculus}{\textsf{\textbf{MOOC}\textnormal{\textsf{ULUS}}}}

\usepackage{tkz-euclide}\usepackage{tikz}
\usepackage{tikz-cd}
\usetikzlibrary{arrows}
\tikzset{>=stealth,commutative diagrams/.cd,
  arrow style=tikz,diagrams={>=stealth}} %% cool arrow head
\tikzset{shorten <>/.style={ shorten >=#1, shorten <=#1 } } %% allows shorter vectors

\usetikzlibrary{backgrounds} %% for boxes around graphs
\usetikzlibrary{shapes,positioning}  %% Clouds and stars
\usetikzlibrary{matrix} %% for matrix
\usepgfplotslibrary{polar} %% for polar plots
\usepgfplotslibrary{fillbetween} %% to shade area between curves in TikZ
\usetkzobj{all}
\usepackage[makeroom]{cancel} %% for strike outs
%\usepackage{mathtools} %% for pretty underbrace % Breaks Ximera
%\usepackage{multicol}
\usepackage{pgffor} %% required for integral for loops



%% http://tex.stackexchange.com/questions/66490/drawing-a-tikz-arc-specifying-the-center
%% Draws beach ball
\tikzset{pics/carc/.style args={#1:#2:#3}{code={\draw[pic actions] (#1:#3) arc(#1:#2:#3);}}}



\usepackage{array}
\setlength{\extrarowheight}{+.1cm}
\newdimen\digitwidth
\settowidth\digitwidth{9}
\def\divrule#1#2{
\noalign{\moveright#1\digitwidth
\vbox{\hrule width#2\digitwidth}}}






\DeclareMathOperator{\arccot}{arccot}
\DeclareMathOperator{\arcsec}{arcsec}
\DeclareMathOperator{\arccsc}{arccsc}

















%%This is to help with formatting on future title pages.
\newenvironment{sectionOutcomes}{}{}


\title{Operations}

\begin{document}

\begin{abstract}
new functions from old
\end{abstract}
\maketitle



Values of functions are numbers.  So, it is no surprise that we can make new functions by combining old functions with addition, subtraction, multiplication, and division. \\





\subsection{Sums}


\begin{template}  \textbf{\textcolor{blue!55!black}{Sum}} \\


If  $f$ and $g$ are functions, then $f + g$ is called the \textbf{\textcolor{green!50!black}{sum}} of $f$ and $g$. \\

The sum of two functions is defined on the intersection of their two domains. \\


\[ (f + g)(a) = f(a) + g(a)  \]



\end{template}



\begin{warning}

If $f$ or $g$ belongs to a nice category of functions, their sum usually does not.

\end{warning}




\begin{example}

$f(x) = e^x$ is an exponential function. \\
$g(t) = 3 t + 4$ is a linear function. \\

However, $(f + g)(k) = f(k) + g(k) = e^k + 3 k + 4$ is neither exponential nor linear.

\end{example}








\subsection{Differences}



\begin{template}  \textbf{\textcolor{blue!55!black}{Difference}} \\


If  $f$ and $g$ are functions, then $f - g$ is called the \textbf{\textcolor{green!50!black}{difference}} of $f$ and $g$. \\

The difference of two functions is defined on the intersection of their two domains. \\


\[ (f - g)(a) = f(a) - g(a)  \]



\end{template}



\begin{warning}

If $f$ or $g$ belongs to a nice category of functions, their difference usually does not.

\end{warning}





\begin{example}

$f(x) = e^x$ is an exponential function. \\
$g(t) = 3 t + 4$ is a linear function. \\

However, $(f - g)(k) = f(k) - g(k) = e^k - 3 k - 4$ is neither exponential nor linear.

\end{example}
















\subsection{Products}



\begin{template}  \textbf{\textcolor{blue!55!black}{Product}} \\


If  $f$ and $g$ are functions, then $f \cdot g$, or just $f g$, is called the \textbf{\textcolor{green!50!black}{product}} of $f$ and $g$. \\

The product of two functions is defined on the intersection of their two domains. \\


\[ (f \cdot g)(a) = f(a) \cdot g(a)  \]



\end{template}



\begin{warning}

If $f$ or $g$ belongs to a nice category of functions, their product usually does not.

\end{warning}





\begin{example}

$f(x) = e^x$ is an exponential function. \\
$g(t) = 3 t + 4$ is a linear function. \\

However, $(f \cdot g)(k) = f(k) \cdot g(k) = e^k (3 k + 4)$ is neither exponential nor linear.

\end{example}























\subsection{Quotients}



It has been very common to see numbers involved in division: $15 \div 7$. However, functions are never written as division. They are always written as quotients (fractions).




\begin{template}  \textbf{\textcolor{blue!55!black}{Quotients}} \\


If  $f$ and $g$ are functions, then $\frac{f}{g}$ is called the \textbf{\textcolor{green!50!black}{quotient}} of $f$ and $g$. \\

The quotient of two functions is defined on the intersection of their two domains except at the zeros of the function in the denominator. \\


\[ \left(\frac{f}{g}\right)(a) = \frac{f(a)}{g(a)}  \]



\end{template}



\begin{warning}

If $f$ or $g$ belongs to a nice category of functions, their quotient usually does not.

\end{warning}





\begin{example}

$f(x) = e^x$ is an exponential function. \\
$g(t) = 3 t + 4$ is a linear function. \\

However, $\left(\frac{f}{g}\right)(k) = \frac{f(k)}{g(k)} = \frac{e^k}{(3 k + 4)}$ is neither exponential nor linear.

\end{example}











\begin{warning}


\begin{itemize}
\item Just because you see an exponential formula does not mean you have an exponential function. \\ 
\item Just because you see a logarithmic formula does not mean you have a logarithmic function. \\ 
\item Just because you see a quadratic formula does not mean you have a quadratic function. \\ 
\item Just because you see a linear formula does not mean you have a linear function. \\ 
\end{itemize}

\end{warning}












\begin{center}
\textbf{\textcolor{green!50!black}{ooooo=-=-=-=-=-=-=-=-=-=-=-=-=ooOoo=-=-=-=-=-=-=-=-=-=-=-=-=ooooo}} \\

more examples can be found by following this link\\ \link[More Examples of the Function Forms]{https://ximera.osu.edu/csccmathematics/precalculus2/precalculus2/functionForm/examples/exampleList}

\end{center}







\end{document}
