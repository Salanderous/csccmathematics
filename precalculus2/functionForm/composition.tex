\documentclass{ximera}


\graphicspath{
  {./}
  {ximeraTutorial/}
  {basicPhilosophy/}
}

\newcommand{\mooculus}{\textsf{\textbf{MOOC}\textnormal{\textsf{ULUS}}}}

\usepackage{tkz-euclide}\usepackage{tikz}
\usepackage{tikz-cd}
\usetikzlibrary{arrows}
\tikzset{>=stealth,commutative diagrams/.cd,
  arrow style=tikz,diagrams={>=stealth}} %% cool arrow head
\tikzset{shorten <>/.style={ shorten >=#1, shorten <=#1 } } %% allows shorter vectors

\usetikzlibrary{backgrounds} %% for boxes around graphs
\usetikzlibrary{shapes,positioning}  %% Clouds and stars
\usetikzlibrary{matrix} %% for matrix
\usepgfplotslibrary{polar} %% for polar plots
\usepgfplotslibrary{fillbetween} %% to shade area between curves in TikZ
\usetkzobj{all}
\usepackage[makeroom]{cancel} %% for strike outs
%\usepackage{mathtools} %% for pretty underbrace % Breaks Ximera
%\usepackage{multicol}
\usepackage{pgffor} %% required for integral for loops



%% http://tex.stackexchange.com/questions/66490/drawing-a-tikz-arc-specifying-the-center
%% Draws beach ball
\tikzset{pics/carc/.style args={#1:#2:#3}{code={\draw[pic actions] (#1:#3) arc(#1:#2:#3);}}}



\usepackage{array}
\setlength{\extrarowheight}{+.1cm}
\newdimen\digitwidth
\settowidth\digitwidth{9}
\def\divrule#1#2{
\noalign{\moveright#1\digitwidth
\vbox{\hrule width#2\digitwidth}}}






\DeclareMathOperator{\arccot}{arccot}
\DeclareMathOperator{\arcsec}{arcsec}
\DeclareMathOperator{\arccsc}{arccsc}

















%%This is to help with formatting on future title pages.
\newenvironment{sectionOutcomes}{}{}


\title{Composition}

\begin{document}

\begin{abstract}
step by step
\end{abstract}
\maketitle



The elementary functions are too nice.

They don't do anything interesting.

However, the world of funcitons is crazy.  Functions behave in very unexpected ways, which we'll never see if we stick to the elementary functions.

For a meaningful investigation of functions, we need to explore other types of functions.

We can create new functions from the elementary functions by creating sums, differences, products, and quotients.


We can create new functions from the elementary functions by creating piecewise defined functions.


We can create new functions from the elementary functions by composing them. 




\begin{template} \textbf{\textcolor{blue!55!black}{Composition}}  \\


Let $f$ be a function with domain $D_f$.


Let $g$ be a function with domain $D_g$.




Then the \textbf{\textcolor{green!50!black}{composition of f and g, $f \circ g$,}} is defined as

\[
(f \circ g)(x) = f(g(x))
\] 


The value of $g$ becomes a domain number for $f$.


The composition is defined on a subset of the domain of $g$.  The composition is defined at those numbers in the domain of $g$ where the value of $g$ is in the domain of $f$.




\end{template}







\begin{example}


Let $f(x) = 3 - x$ with its natural domain.


Let $g(y) = \ln(y)$ with its natural domain.


Defines $H = f \circ g$ with its natural domain.



\begin{explanation}


$H(t) = (f \circ g)(t) = f(g(t))$


The domain of $H = f \circ g$ is a subset of the domain of $g$, which is $(0, \infty)$.  

We can only use those numbers where $ln(t)$ is in the domain of $f$.  However, the domain of $f$ is all real numbers.  So, we can use everything from the domain of $g$.

The induced domain of $H = f \circ g$ is the domain of $g$, which is $(0, \infty)$. 




\end{explanation}







\begin{explanation}


$K(r) = (g \circ f)(r) = g(f(r))$


The domain of $K = g \circ f$ is a subset of the domain of $f$, which is $(-\infty, \infty)$.  

We can only use those numbers where $f(r)$ is in the domain of $g$.  The domain of $g$ is $(0, \infty)$.  

Therefore, we need $f(x) = 3 - x > 0$.  Or, $x < 3$.

The induced domain of $K = g \circ f$ is $(-\infty, 3)$. 

These are the numbers in the domain of $f$ where the value of $f$ is insdie the domain of $g$.


\end{explanation}




\end{example}


An \textbf{induced} domain is a domain that is force to happen due to given restrictions. \\







\begin{example}



Let $T(x) = (3-x)(x-5)$ with its natural domain.


Let $p(t) = \frac{1}{t}$ with its natural domain.


Defines $Y = T \circ p$ with its natural domain.



\begin{question}

Evaluate


\[
(T \circ p)\left( \frac{1}{2} \right) = \answer{-3}
\]



\end{question}






\begin{question}

Evaluate


\[
(p \circ T)\left( \frac{1}{2} \right) = \answer{-\frac{45}{4}}
\]



\end{question}





\begin{question}

What is the induced domain of $(T \circ p)$?


\[
\left( \answer{-\infty}, \answer{0}  \right) \cup \left( \answer{0}, \answer{\infty}  \right)
\]



\end{question}








\begin{question}

What is the induced domain of $(p \circ T)$?


\[
\left( \answer{-\infty}, \answer{3}  \right) \cup \left( \answer{3}, \answer{5}  \right)  \cup \left( \answer{5}, \answer{\infty}  \right)
\]



\end{question}







\end{example}














\begin{center}
\textbf{\textcolor{green!50!black}{ooooo=-=-=-=-=-=-=-=-=-=-=-=-=ooOoo=-=-=-=-=-=-=-=-=-=-=-=-=ooooo}} \\

more examples can be found by following this link\\ \link[More Examples of the Elementary Library]{https://ximera.osu.edu/csccmathematics/precalculus2/precalculus2/functionForm/examples/exampleList}

\end{center}







\end{document}
