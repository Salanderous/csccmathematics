\documentclass{ximera}


\graphicspath{
  {./}
  {ximeraTutorial/}
  {basicPhilosophy/}
}

\newcommand{\mooculus}{\textsf{\textbf{MOOC}\textnormal{\textsf{ULUS}}}}

\usepackage{tkz-euclide}\usepackage{tikz}
\usepackage{tikz-cd}
\usetikzlibrary{arrows}
\tikzset{>=stealth,commutative diagrams/.cd,
  arrow style=tikz,diagrams={>=stealth}} %% cool arrow head
\tikzset{shorten <>/.style={ shorten >=#1, shorten <=#1 } } %% allows shorter vectors

\usetikzlibrary{backgrounds} %% for boxes around graphs
\usetikzlibrary{shapes,positioning}  %% Clouds and stars
\usetikzlibrary{matrix} %% for matrix
\usepgfplotslibrary{polar} %% for polar plots
\usepgfplotslibrary{fillbetween} %% to shade area between curves in TikZ
\usetkzobj{all}
\usepackage[makeroom]{cancel} %% for strike outs
%\usepackage{mathtools} %% for pretty underbrace % Breaks Ximera
%\usepackage{multicol}
\usepackage{pgffor} %% required for integral for loops



%% http://tex.stackexchange.com/questions/66490/drawing-a-tikz-arc-specifying-the-center
%% Draws beach ball
\tikzset{pics/carc/.style args={#1:#2:#3}{code={\draw[pic actions] (#1:#3) arc(#1:#2:#3);}}}



\usepackage{array}
\setlength{\extrarowheight}{+.1cm}
\newdimen\digitwidth
\settowidth\digitwidth{9}
\def\divrule#1#2{
\noalign{\moveright#1\digitwidth
\vbox{\hrule width#2\digitwidth}}}






\DeclareMathOperator{\arccot}{arccot}
\DeclareMathOperator{\arcsec}{arcsec}
\DeclareMathOperator{\arccsc}{arccsc}

















%%This is to help with formatting on future title pages.
\newenvironment{sectionOutcomes}{}{}


\title{Categories}

\begin{document}

\begin{abstract}
elementary library
\end{abstract}
\maketitle




Success at analyzing functions rests entirely on our ability to recognize what type of function we have.  Categorizing functions is the quickest and best way of gaining information about our function.

Our library of elementary functions gives us a wonerful filter to begin idenitfying functions.  \\



\begin{template}  \textbf{\textcolor{blue!55!black}{Constant Functions}} \\



Constant funtions are functions which give the same value for all domain numbers. \\


They can be represented by formulas that look like


\[ f(x) = C  \,  \text{for all } x  \, in the domain.   \]



\end{template}


\begin{warning}   \textbf{\textcolor{red!80!black}{can}}  \\

All of our definitions for the elementary functions use the word \textbf{\textcolor{red!80!black}{can}}.


\begin{example}

The function representated by 

\[ f(x) = \frac{x^2 - 1}{x + 1} - x  \,  \text{ on } \,   [0, \infty)   \]

is a constant function.   $-1$ is the only function value.


\end{example}





\begin{example}

The function representated by 

\[ C(\theta) = (\cos(\theta))^2 + (\sin(\theta))^2   \,  \text{ on } \,   (-\infty, \infty)   \]

is a constant function.   $1$ is the only function value.


\end{example}




Constant functions are disguised with many formulas.  They \textbf{\textcolor{red!80!black}{can}} be written in the form $f(x) = C$.



\end{warning}














\begin{template}  \textbf{\textcolor{blue!55!black}{LInear Funcitons}} \\
\end{template}







\begin{center}
\textbf{\textcolor{green!50!black}{ooooo=-=-=-=-=-=-=-=-=-=-=-=-=ooOoo=-=-=-=-=-=-=-=-=-=-=-=-=ooooo}} \\

more examples can be found by following this link\\ \link[More Examples of the Elementary Library]{https://ximera.osu.edu/csccmathematics/precalculus2/precalculus2/functionForm/examples/exampleList}

\end{center}







\end{document}
