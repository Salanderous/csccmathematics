\documentclass{ximera}


\graphicspath{
  {./}
  {ximeraTutorial/}
  {basicPhilosophy/}
}

\newcommand{\mooculus}{\textsf{\textbf{MOOC}\textnormal{\textsf{ULUS}}}}

\usepackage{tkz-euclide}\usepackage{tikz}
\usepackage{tikz-cd}
\usetikzlibrary{arrows}
\tikzset{>=stealth,commutative diagrams/.cd,
  arrow style=tikz,diagrams={>=stealth}} %% cool arrow head
\tikzset{shorten <>/.style={ shorten >=#1, shorten <=#1 } } %% allows shorter vectors

\usetikzlibrary{backgrounds} %% for boxes around graphs
\usetikzlibrary{shapes,positioning}  %% Clouds and stars
\usetikzlibrary{matrix} %% for matrix
\usepgfplotslibrary{polar} %% for polar plots
\usepgfplotslibrary{fillbetween} %% to shade area between curves in TikZ
\usetkzobj{all}
\usepackage[makeroom]{cancel} %% for strike outs
%\usepackage{mathtools} %% for pretty underbrace % Breaks Ximera
%\usepackage{multicol}
\usepackage{pgffor} %% required for integral for loops



%% http://tex.stackexchange.com/questions/66490/drawing-a-tikz-arc-specifying-the-center
%% Draws beach ball
\tikzset{pics/carc/.style args={#1:#2:#3}{code={\draw[pic actions] (#1:#3) arc(#1:#2:#3);}}}



\usepackage{array}
\setlength{\extrarowheight}{+.1cm}
\newdimen\digitwidth
\settowidth\digitwidth{9}
\def\divrule#1#2{
\noalign{\moveright#1\digitwidth
\vbox{\hrule width#2\digitwidth}}}






\DeclareMathOperator{\arccot}{arccot}
\DeclareMathOperator{\arcsec}{arcsec}
\DeclareMathOperator{\arccsc}{arccsc}

















%%This is to help with formatting on future title pages.
\newenvironment{sectionOutcomes}{}{}


\title{Categories}

\begin{document}

\begin{abstract}
elementary library
\end{abstract}
\maketitle




Success at analyzing functions rests entirely on our ability to recognize what type of function we have.  Categorizing functions is the quickest and best way of gaining information about our function.

Our library of elementary functions gives us a wonerful filter to begin idenitfying functions.  \\



\begin{template}  \textbf{\textcolor{blue!55!black}{Constant Functions}} \\



Constant funtions are functions which give the same value for all domain numbers. \\


They can be represented by formulas that look like


\[ f(x) = C  \,  \text{ for all } x  \, \text{ in the domain. }   \]



\end{template}



The natural domain of a constant function is all real numbers.  However, we can always state a restricted domain.




\begin{warning}   \textbf{\textcolor{red!80!black}{can}}  \\

All of our definitions for the elementary functions use the word \textbf{\textcolor{red!80!black}{can}}.


\begin{example}

The function representated by 

\[ f(x) = \frac{x^2 - 1}{x + 1} - x  \,  \text{ on } \,   [0, \infty)   \]

is a constant function.   $-1$ is the only function value.


\end{example}





\begin{example}

The function representated by 

\[ C(\theta) = (\cos(\theta))^2 + (\sin(\theta))^2   \,  \text{ on } \,   (-\infty, \infty)   \]

is a constant function.   $1$ is the only function value.


\end{example}




Constant functions are disguised with many formulas.  They \textbf{\textcolor{red!80!black}{can}} be written in the form $f(x) = C$.



\end{warning}




Graphs of constant functions are horizontal lines.
\[ f(x) = 3\]





\begin{image}
\begin{tikzpicture}
  \begin{axis}[
            domain=-10:10, ymax=10, xmax=10, ymin=-10, xmin=-10,
            axis lines =center, xlabel=$x$, ylabel={$y=f(x)$}, grid = major, grid style={dashed},
            ytick={-10,-8,-6,-4,-2,2,4,6,8,10},
            xtick={-10,-8,-6,-4,-2,2,4,6,8,10},
            yticklabels={$-10$,$-8$,$-6$,$-4$,$-2$,$2$,$4$,$6$,$8$,$10$}, 
            xticklabels={$-10$,$-8$,$-6$,$-4$,$-2$,$2$,$4$,$6$,$8$,$10$},
            ticklabel style={font=\scriptsize},
            every axis y label/.style={at=(current axis.above origin),anchor=south},
            every axis x label/.style={at=(current axis.right of origin),anchor=west},
            axis on top
          ]
          
            
      \addplot [line width=2, penColor, smooth,samples=200,domain=(-9:9),<->] {3};




  \end{axis}
\end{tikzpicture}
\end{image}

















\begin{template}  \textbf{\textcolor{blue!55!black}{Linear Functions}} \\



Linear funtions are functions which exhibit a constant rate of change. \\


They can be represented by formulas that look like


\[ f(x) = m x + b  \,  \text{ for all } x  \, \text{ in the domain. }   \]



\end{template}



The natural domain of a linear function is all real numbers.  However, we can always state a restricted domain.







Graphs of linear functions are lines. 
(This make constant functions special linear functions)
\[ f(x) = 2x - 6\]





\begin{image}
\begin{tikzpicture}
  \begin{axis}[
            domain=-10:10, ymax=10, xmax=10, ymin=-10, xmin=-10,
            axis lines =center, xlabel=$x$, ylabel={$y=f(x)$}, grid = major, grid style={dashed},
            ytick={-10,-8,-6,-4,-2,2,4,6,8,10},
            xtick={-10,-8,-6,-4,-2,2,4,6,8,10},
            yticklabels={$-10$,$-8$,$-6$,$-4$,$-2$,$2$,$4$,$6$,$8$,$10$}, 
            xticklabels={$-10$,$-8$,$-6$,$-4$,$-2$,$2$,$4$,$6$,$8$,$10$},
            ticklabel style={font=\scriptsize},
            every axis y label/.style={at=(current axis.above origin),anchor=south},
            every axis x label/.style={at=(current axis.right of origin),anchor=west},
            axis on top
          ]
          
            
      \addplot [line width=2, penColor, smooth,samples=200,domain=(-1.5:7),<->] {2*x-6};




  \end{axis}
\end{tikzpicture}
\end{image}






Although the definition quotes the slope-intercept form ($m x + b$), this is really not very useful.  In most situations, we do not have the intercept.  Instead, it is more common to have a slope and a random point or just two random points.



\begin{itemize}

\item \textbf{Point-Slope} : $(y - y_0) = m (x - x_0)$

\item \textbf{Two Points} : $(y - y_0) = \frac{y_1 - y_0}{x_1 - x_0} \, (x - x_0)$

\end{intemize}


All three of these forms are equivalent.  They can all be rearranged into the form of the others.  They are all reflecting the geometry of lines.  


On the other hand, we have seen that the analysis of functions often begins with zeros.  Products of factors is a preferred form.  WIth that in mind, linear functions can be written in factored form.


\[  f(x) = a (x - z_0)  \]

Where $a$ is the leading coefficient and $z_0$ is the zero of the function.





\begin{itemize}

\item $a = m$, the rate of change of the function is the slope of the line.

\item $z_0$ is the zero of the linear function, which is visually encoded as the intercept $(z_0, 0)$.

\end{itemize}



\textbf{Note:} Constant functions are linear functions, because they can be written as $L(x) = 0 x + C$. \\












\begin{template}  \textbf{\textcolor{blue!55!black}{Quadratic Functions}} \\



Quadratic funtions are the next level of polynomial. \\


They can be represented by formulas that look like


\[ f(x) = a x^2 + b x + c  \,  \text{ for all } x  \, \text{ in the domain. }   \]



\end{template}



The natural domain of a quadratic function is all real numbers.  However, we can always state a restricted domain.







Graphs of quadratic functions are parabolas. 
\[ f(x) = x^2 + x - 6\]





\begin{image}
\begin{tikzpicture}
  \begin{axis}[
            domain=-10:10, ymax=10, xmax=10, ymin=-10, xmin=-10,
            axis lines =center, xlabel=$x$, ylabel={$y=f(x)$}, grid = major, grid style={dashed},
            ytick={-10,-8,-6,-4,-2,2,4,6,8,10},
            xtick={-10,-8,-6,-4,-2,2,4,6,8,10},
            yticklabels={$-10$,$-8$,$-6$,$-4$,$-2$,$2$,$4$,$6$,$8$,$10$}, 
            xticklabels={$-10$,$-8$,$-6$,$-4$,$-2$,$2$,$4$,$6$,$8$,$10$},
            ticklabel style={font=\scriptsize},
            every axis y label/.style={at=(current axis.above origin),anchor=south},
            every axis x label/.style={at=(current axis.right of origin),anchor=west},
            axis on top
          ]
          
            
      \addplot [line width=2, penColor, smooth,samples=200,domain=(-1.5:7),<->] {x^2 + x -6};




  \end{axis}
\end{tikzpicture}
\end{image}






Although the definition quotes the standard form ($a x^2 + b x + c$), there are more useful forms. 











\begin{itemize}

\item \textbf{\textcolor{blue!55!black}{Vertex Form}}  $a (x - h)^2 + k$ 

\item \textbf{\textcolor{blue!55!black}{Factored Form}}  $a (x - r_0)(x - r_1)$ 

\end{itemize}

























\begin{center}
\textbf{\textcolor{green!50!black}{ooooo=-=-=-=-=-=-=-=-=-=-=-=-=ooOoo=-=-=-=-=-=-=-=-=-=-=-=-=ooooo}} \\

more examples can be found by following this link\\ \link[More Examples of Function Forms]{https://ximera.osu.edu/csccmathematics/precalculus2/precalculus2/functionForm/examples/exampleList}

\end{center}







\end{document}
