\documentclass{ximera}


\graphicspath{
  {./}
  {ximeraTutorial/}
  {basicPhilosophy/}
}

\newcommand{\mooculus}{\textsf{\textbf{MOOC}\textnormal{\textsf{ULUS}}}}

\usepackage{tkz-euclide}\usepackage{tikz}
\usepackage{tikz-cd}
\usetikzlibrary{arrows}
\tikzset{>=stealth,commutative diagrams/.cd,
  arrow style=tikz,diagrams={>=stealth}} %% cool arrow head
\tikzset{shorten <>/.style={ shorten >=#1, shorten <=#1 } } %% allows shorter vectors

\usetikzlibrary{backgrounds} %% for boxes around graphs
\usetikzlibrary{shapes,positioning}  %% Clouds and stars
\usetikzlibrary{matrix} %% for matrix
\usepgfplotslibrary{polar} %% for polar plots
\usepgfplotslibrary{fillbetween} %% to shade area between curves in TikZ
\usetkzobj{all}
\usepackage[makeroom]{cancel} %% for strike outs
%\usepackage{mathtools} %% for pretty underbrace % Breaks Ximera
%\usepackage{multicol}
\usepackage{pgffor} %% required for integral for loops



%% http://tex.stackexchange.com/questions/66490/drawing-a-tikz-arc-specifying-the-center
%% Draws beach ball
\tikzset{pics/carc/.style args={#1:#2:#3}{code={\draw[pic actions] (#1:#3) arc(#1:#2:#3);}}}



\usepackage{array}
\setlength{\extrarowheight}{+.1cm}
\newdimen\digitwidth
\settowidth\digitwidth{9}
\def\divrule#1#2{
\noalign{\moveright#1\digitwidth
\vbox{\hrule width#2\digitwidth}}}






\DeclareMathOperator{\arccot}{arccot}
\DeclareMathOperator{\arcsec}{arcsec}
\DeclareMathOperator{\arccsc}{arccsc}

















%%This is to help with formatting on future title pages.
\newenvironment{sectionOutcomes}{}{}


\title{Exponential Functions}

\begin{document}

\begin{abstract}
%Stuff can go here later if we want!
\end{abstract}
\maketitle




Precalculus is largely about the structure of the Elementary Functions and how we reason about them.  This includes power; radical and root functions; linear, quadratic, and other polynomial functions; rational functions; exponential functions; logarithmic functions; and a library of triginometric functions (which are introduced in this course).

As a segue from the first half of Precalculus, let's quickly review our growing library of Elementary Functions. We'll begin with the exponential functions.   \\





$\blacktriangleright$ \textbf{\textcolor{blue!55!black}{Basic Exponential Functions}} 


The template for the basic exponential function looks like \textbf{\textcolor{blue!55!black}{$exp(x) = a \cdot r^x$}}. 

\begin{itemize}
\item $a$ is the \textbf{\textcolor{purple!85!blue}{leading coefficient}}.  It primarily determines the sign of the function values.
\item $r$ is the \textbf{\textcolor{purple!85!blue}{base}}. It determines how the function values grow, depending on whether $0 < r < 1$ or $1 < r$.
\end{itemize}





$\blacktriangleright$ \textbf{\textcolor{blue!55!black}{General Exponential Functions}}   

The template for the general exponential function looks like \textbf{\textcolor{blue!55!black}{$exp(x) = a \cdot r^{b \, x + c} + d$}}  or just \textbf{\textcolor{blue!55!black}{$a \cdot r^x + d$}}. 

\begin{itemize}
\item $c$ determines the \textbf{\textcolor{purple!85!blue}{shift}} in the domain from the basic exponential function.  
\item $d$ determines the \textbf{\textcolor{purple!85!blue}{shift}} in the function value from the basic exponential function. 
\item Both $b$ and $c$ can be absorbed into the base.
\end{itemize}









\subsection{Learning Outcomes}


\begin{sectionOutcomes}
In this section, students will 

\begin{itemize}
\item describe the full picture of general exponential functions and their graphs.
\end{itemize}
\end{sectionOutcomes}

\end{document}
