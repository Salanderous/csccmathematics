\documentclass{ximera}


\graphicspath{
  {./}
  {ximeraTutorial/}
  {basicPhilosophy/}
}

\newcommand{\mooculus}{\textsf{\textbf{MOOC}\textnormal{\textsf{ULUS}}}}

\usepackage{tkz-euclide}\usepackage{tikz}
\usepackage{tikz-cd}
\usetikzlibrary{arrows}
\tikzset{>=stealth,commutative diagrams/.cd,
  arrow style=tikz,diagrams={>=stealth}} %% cool arrow head
\tikzset{shorten <>/.style={ shorten >=#1, shorten <=#1 } } %% allows shorter vectors

\usetikzlibrary{backgrounds} %% for boxes around graphs
\usetikzlibrary{shapes,positioning}  %% Clouds and stars
\usetikzlibrary{matrix} %% for matrix
\usepgfplotslibrary{polar} %% for polar plots
\usepgfplotslibrary{fillbetween} %% to shade area between curves in TikZ
\usetkzobj{all}
\usepackage[makeroom]{cancel} %% for strike outs
%\usepackage{mathtools} %% for pretty underbrace % Breaks Ximera
%\usepackage{multicol}
\usepackage{pgffor} %% required for integral for loops



%% http://tex.stackexchange.com/questions/66490/drawing-a-tikz-arc-specifying-the-center
%% Draws beach ball
\tikzset{pics/carc/.style args={#1:#2:#3}{code={\draw[pic actions] (#1:#3) arc(#1:#2:#3);}}}



\usepackage{array}
\setlength{\extrarowheight}{+.1cm}
\newdimen\digitwidth
\settowidth\digitwidth{9}
\def\divrule#1#2{
\noalign{\moveright#1\digitwidth
\vbox{\hrule width#2\digitwidth}}}






\DeclareMathOperator{\arccot}{arccot}
\DeclareMathOperator{\arcsec}{arcsec}
\DeclareMathOperator{\arccsc}{arccsc}

















%%This is to help with formatting on future title pages.
\newenvironment{sectionOutcomes}{}{}


\title{Theory}

\begin{document}

\begin{abstract}
the exponential story
\end{abstract}
\maketitle





We have two levels of exponential formulas: the basic template and then the general template, which includes shifts and stretches.  All of these are based on our original idea of \textbf{\textcolor{purple!85!blue}{constant percentage growth}}.


Constant percentage growth means that the value  of our function grows by an equal percentage over equal size intervals in the domain.



For example, let $f$ be a function that grows by $30\%$ over every domain interval of length $1$. \\


\begin{itemize}
\item If $f(0) = 1$, then $f(1) = 1.3 \cdot 1 = 1.3$
\item If $f(5) = 21$, then $f(6) = 1.3 \cdot 21 = 27.3$
\item If $f(\pi) = \sqrt{3}$, then $f(\pi + 1) = 1.3 \cdot \sqrt{3} = 1.3 \sqrt{3}$
\end{itemize}

Every time the domain interval changes by $1$, then the function value is multiplied by $1.3$ (an increase of $30\%$).




There are many basic exponential functions that possess this characteristic. They all can be written in the form \textbf{\textcolor{blue!55!black}{$f(x) = a \cdot r^x$}}. We call \textbf{\textcolor{purple!85!blue}{$a=f(0)$}} the \textbf{\textcolor{purple!85!blue}{initial value}}.







\begin{image}
\begin{tikzpicture}
   \begin{axis}[name = leftgraph, 
            domain=-10:10, ymax=10, xmax=10, ymin=-10, xmin=-10,
            axis lines =center, xlabel=$x$, ylabel={$y$},
            ticklabel style={font=\scriptsize},
            every axis y label/.style={at=(current axis.above origin),anchor=south},
            every axis x label/.style={at=(current axis.right of origin),anchor=west},
            axis on top
          ]
          
          \addplot [line width=1, gray, dashed,samples=200,domain=(-10:10),<->] {0};


          \addplot [line width=2, penColor, smooth,samples=100,domain=(-9:8)] {2 * (1.3^x)};
          \addplot [line width=2, penColor, smooth,samples=100,domain=(-9:8)] {3 * (1.3^x)};
          \addplot [line width=2, penColor, smooth,samples=100,domain=(-9:8)] {5 * (1.3^x)};
          \addplot [line width=2, penColor, smooth,samples=100,domain=(-9:8)] {0.3 * (1.3^x)};

          \addplot [line width=2, penColor2, smooth,samples=100,domain=(-9:8), <->] {1.3^x};
          \addplot [color=penColor2,only marks,mark=*] coordinates{(0,1)};

  \end{axis}
\end{tikzpicture}
\end{image}



All of these are described by a similar formula we call the basic exponential formula.




\subsection{The Basic Exponential Formula}


The template for the formula of the basic exponential function looks like




\[  a \, r^x   \, \text{ with } \,  a, r \in \mathbb{R} \, | \,  a \ne 0, \, r > 0   \]


As we have seen before, graphically, \textbf{\textcolor{purple!85!blue}{the coefficient}}, $a$, controls vertical stretching or compression. The sign of $a$ dictates the sign of our function values, which is illustrated graphically as a vertical reflection. \textbf{\textcolor{purple!85!blue}{The base}}, $r$, dictates a growing or decaying function.




We have four combinations of our coefficient and base parameters.

\begin{itemize}
\item $a>0$ and $r>1$
\item $a>0$ and $r<1$
\item $a<0$ and $r>1$
\item $a<0$ and $r<1$
\end{itemize}


Graphs of the basic exponential functions have the horizontal axis as a horizontal asymptote in one direction or the other.

\begin{explanation}


When $r$ \wordChoice{\choice{$<$} \choice[correct]{$>$}} $1$ we have a growing function. It could be growing positively or negatively, depending on the sign of $a$.

\[  \lim_{x \to -\infty} a \, r^x = 0 \, \text{ and } \, \lim_{x \to \infty} a \, r^x = \pm\infty \]


Since $r^x$ is always positive, it is the sign of $a$ that dictates if the unbounded growth is positive or negative.




\begin{image}
\begin{tikzpicture}
   \begin{axis}[name = leftgraph, 
            domain=-10:10, ymax=10, xmax=10, ymin=-10, xmin=-10,
            axis lines =center, xlabel=$x$, ylabel={$a>0$},
            every axis y label/.style={at=(current axis.above origin),anchor=south},
            every axis x label/.style={at=(current axis.right of origin),anchor=west},
            axis on top
          ]
          
          \addplot [line width=1, gray, dashed,samples=200,domain=(-10:10),<->] {0};

          \addplot [line width=2, penColor, smooth,samples=100,domain=(-9:8), <->] {1.3^x};
          \addplot [color=penColor,only marks,mark=*] coordinates{(0,1)};

           

  \end{axis}
  \begin{axis}[at={(leftgraph.outer east)},anchor=outer west, 
            domain=-10:10, ymax=10, xmax=10, ymin=-10, xmin=-10,
            axis lines =center, xlabel=$x$, ylabel={$a<0$},
            every axis y label/.style={at=(current axis.above origin),anchor=south},
            every axis x label/.style={at=(current axis.right of origin),anchor=west},
            axis on top
          ]
          
          \addplot [line width=1, gray, dashed,samples=200,domain=(-10:10),<->] {0};

          \addplot [line width=2, penColor, smooth,samples=100,domain=(-9:8),<->] {-(1.3^x)};
          \addplot [color=penColor,only marks,mark=*] coordinates{(0,-1)};

           

  \end{axis}
\end{tikzpicture}
\end{image}


Graphs of basic exponential functions have one strategic point, which occurs when the exponent equals zero. For the basic exponential function, this point is $\left( \answer{0}, a \right)$.

\end{explanation}





\begin{explanation}



When $r$ \wordChoice{\choice[correct]{$<$} \choice{$>$}} $1$, the horizontal axis is still a horizontal asymptote, just in the direction where $x$ is large and positive. The function now decays. It could decay positively or negatively, depending on the sign of $a$.


\[ \lim_{x \to -\infty} a \, r^x = \pm\infty \, \text{ and } \, \lim_{x \to \infty} a \, r^x = 0 \]


 
The sign of $a$ dictates if the function decays through positive or negative values, because $r^x > 0$.



\begin{image}
\begin{tikzpicture}
   \begin{axis}[name = leftgraph, 
            domain=-10:10, ymax=10, xmax=10, ymin=-10, xmin=-10,
            axis lines =center, xlabel=$x$, ylabel={$a>0$},
            every axis y label/.style={at=(current axis.above origin),anchor=south},
            every axis x label/.style={at=(current axis.right of origin),anchor=west},
            axis on top
          ]
          
          \addplot [line width=1, gray, dashed,samples=200,domain=(-10:10),<->] {0};

          \addplot [line width=2, penColor, smooth,samples=200,domain=(-8:9), <->] {1.3^(-x)};
          \addplot [color=penColor,only marks,mark=*] coordinates{(0,1)};
           

  \end{axis}
  \begin{axis}[at={(leftgraph.outer east)},anchor=outer west, 
            domain=-10:10, ymax=10, xmax=10, ymin=-10, xmin=-10,
            axis lines =center, xlabel=$x$, ylabel={$a<0$},
            every axis y label/.style={at=(current axis.above origin),anchor=south},
            every axis x label/.style={at=(current axis.right of origin),anchor=west},
            axis on top
          ]
          
          \addplot [line width=1, gray, dashed,samples=200,domain=(-10:10),<->] {0};

          \addplot [line width=2, penColor, smooth,samples=200,domain=(-8:9),<->] {-(1.3^(-x)};
          \addplot [color=penColor,only marks,mark=*] coordinates{(0,-1)};
           

  \end{axis}
\end{tikzpicture}
\end{image}



Graphs of basic exponential functions have one strategic point, which occurs when the exponent equals zero. For the basic exponential function, this point is $(0, a)$, which is now below the horizontal asymptote, because $a < 0$.



\end{explanation}






All four basic exponential graphs share a common structure.


\begin{itemize}
\item All have the horizontal axis as an asymptote in one direction, and
\item In the other direction, they grow unbounded.
\item All are a distance of $a$ from the asymptote, when the exponent equals $0$.
\end{itemize}

These are the important aspects or characteristics that we watch when shifting and stretching graphs of exponential functions.












\begin{summary} \textbf{\textcolor{blue!75!black}{Exponential Behavior}}


Basic exponential functions, $a \cdot r^x$, are either increasing functions or decreasing functions.


$\blacktriangleright$  \textbf{\textcolor{purple!85!blue}{Base Greater than $1$: $r > 1$}} 


\begin{itemize}
\item greater positive exponents mean multiplying by the base more, which results in larger values.  
\item greater negative exponents mean multiplying by the reciprocal of the base more, which results in smaller values.  
\end{itemize}


The coefficient in front, $a$, tells us if this larger/smaller value is positive or negative.


\begin{itemize}
\item $a > 0$ and $r > 1$ : \wordChoice{\choice[correct]{increasing} \choice{decreasing}}  (positive) function
\item $a < 0$ and $r > 1$ : \wordChoice{\choice{increasing} \choice[correct]{decreasing}}  (negative) function  
\end{itemize}






$\blacktriangleright$  \textbf{\textcolor{purple!85!blue}{Base Less than $1$: $r < 1$}}  


\begin{itemize} 
\item greater positive exponents mean multiplying by the base more, which results in smaller values.  
\item greater negative exponents mean multiplying by the reciprocal of the base more, which results in larger values.  
\end{itemize}


The coefficient in front, $a$, tells us if this larger/smaller value is positive or negative.


\begin{itemize}
\item $a > 0$ and $r < 1$ : \wordChoice{\choice{increasing} \choice[correct]{decreasing}}  (positive) function
\item $a < 0$ and $r < 1$ : \wordChoice{\choice[correct]{increasing} \choice{decreasing}}  (negative) function  
\end{itemize}



\end{summary}




As we see, basic exponential functions can be

\begin{itemize}
\item positive and increasing
\item positive and decreasing
\item negative and increasing
\item negative and decreasing
\end{itemize}




















\subsection{General Exponential Functions}


General exponential functions are shifts and stretches of basic exponential functions.  


\[
f(x) = a \, r^{b \, x + c} + d
\]



They maintain the same general characteristics and features as basic exponential functions, just in different locations.





\begin{example}  General Exponential Function \\



Analyze   $f(x) = \frac{1}{3} \, 2^{x+5} - 7$ \\


\begin{explanation}

Compared to a basic exponential function graph, where the horizontal axis is the horizontal asymptote, here, the horizontal asymptote has been moved down $\answer{7}$ units.



The ``inside'' of $f(x)$, representing the domain, is $x+5$.  This equals $0$, when $x=-5$.  The exponent is positive for $x>-5$, and because the base is $2 > 1$, this is the direction (right) of unbounded growth.  Therefore, the other direction (left) is where the horizontal asymptote is in effect.  Since the leading coefficient is $\frac{1}{3} > 0$, the unbounded growth is positive.

At $x=-5$, we have our one anchor point for the graph.  The point is $\left(-5, \frac{1}{3} - 7 \right)$, which is $\frac{1}{3}$ above the horizontal asymptote, $y = -7$.


Graph of $y = f(x)$.

\begin{image}
\begin{tikzpicture}
  \begin{axis}[
            domain=-10:10, ymax=10, xmax=10, ymin=-10, xmin=-10,
            axis lines =center, xlabel=$x$, ylabel=$y$, grid = major,
            ytick={-10,-8,-6,-4,-2,2,4,6,8,10},
          	xtick={-10,-8,-6,-4,-2,2,4,6,8,10},
          	ticklabel style={font=\scriptsize},
            every axis y label/.style={at=(current axis.above origin),anchor=south},
            every axis x label/.style={at=(current axis.right of origin),anchor=west},
            axis on top
          ]
          
          \addplot [line width=1, gray, dashed,samples=200,domain=(-10:10),<->] {-7};

      	  \addplot [line width=2, penColor, smooth,samples=200,domain=(-10:0.6),<->] {0.33 * 2^(x+5)-7};

      	  \addplot[color=penColor,fill=penColor,only marks,mark=*] coordinates{(-5,-6.66)};

          


 

  \end{axis}
\end{tikzpicture}
\end{image}




Our graph agrees with our analysis.

\begin{itemize}
\item The natural or implied domain of $f$ is $\mathbb{R}$.
\item $f$ is always increasing.
\item $f$ has no maximums or minimums.
\item $\lim\limits_{x \to -\infty} f(x) = \answer{-7}$
\item $\lim\limits_{x \to \infty} f(x) = \answer{\infty}$
\end{itemize}



For this function, when the domain number changes by $1$, the height above the horizontal asymptote changes by $100\%$, because it is multiplied by another $2$.




\end{explanation}

\end{example}


The shape of the graph looks like that of the basic exponential graph. \\


\begin{itemize}
\item The horizontal asymptote moved.
\item The strategic point moved.
\end{itemize}















$\blacktriangleright$ \textbf{\textcolor{blue!55!black}{What happened to stretching?}} \\



Basic exponential functions swallow up stretching and compressing in the leading coefficient and base.


\begin{itemize}
\item If we multiply on the outside, by $b$, for a vertical stretch, then we have

\[
b \cdot (a \cdot r^x) = (a \cdot b) \cdot r^x
\]

We obtain a new leading coefficient, which tells us about the sign of the function values. \\






\item If we multiply on the inside, by $b$, for a horizontal stretch, then our exponential rules give us

\[
a \cdot r^{b \cdot x} = a \cdot (r^b)^x = a \cdot R^x
\]

We obtain a new base, which tells us about the growth of the function.


\end{itemize}





Similarly, horizontal stretching can also be absorbed by the leading coefficent.

 \[
 a \cdot r^{x+c} = a \cdot r^x \cdot r^c = (a \cdot r^c) \cdot r^x
\]




We have a choice.  For a general exponential functions, we can think about stretching or we can apply a little algebra and return to a basic exponential function.




















\begin{example} Another View



$f(x) = \frac{1}{3} \, 2^{x+5} - 7$ is the function in the last example.  We could take advantage of exponential algebra to simplify the formula.   \\

\[ 
f(x) = \frac{1}{3} \, 2^{x+5} - 7 = \frac{1}{3} \, 2^x \cdot 2^5 - 7 = \frac{32}{3} \, 2^x  - 7
\]


We now have a basic exponential function shifted down $7$.  The new base is still $2$ and the new leading coefficient is $\frac{32}{3}$.





By doing this, we change our strategic point to $\left( 0, \frac{32}{3} - 7 \right) = \left( 0, \frac{11}{3} \right)$.




\begin{image}
\begin{tikzpicture}
  \begin{axis}[
            domain=-10:10, ymax=10, xmax=10, ymin=-10, xmin=-10,
            axis lines =center, xlabel=$x$, ylabel=$y$, grid = major,
            ytick={-10,-8,-6,-4,-2,2,4,6,8,10},
            xtick={-10,-8,-6,-4,-2,2,4,6,8,10},
            ticklabel style={font=\scriptsize},
            every axis y label/.style={at=(current axis.above origin),anchor=south},
            every axis x label/.style={at=(current axis.right of origin),anchor=west},
            axis on top
          ]
          
          \addplot [line width=1, gray, dashed,samples=200,domain=(-10:10),<->] {-7};

          \addplot [line width=2, penColor, smooth,samples=200,domain=(-10:0.6),<->] {0.33 * 2^(x+5)-7};

          \addplot[color=penColor,fill=penColor,only marks,mark=*] coordinates{(0,3.66)};

          


 

  \end{axis}
\end{tikzpicture}
\end{image}




\end{example}






















\begin{example} Altering Algebraicaaly



$G(t) = 4 \, 3^{2t-1} - 5$.  

We could take advantage of exponential algebra to simplify the formula.   \\

\[ 
G(t) = 4 \, 3^{2t-1} - 5 = 4 \, 3^{2t} \cdot \frac{1}{3} - 5 = \frac{4}{3} \, (3^2)^t - 5  = \frac{4}{3} \, 9^t - 5
\]


We now have a basic exponential function shifted down $5$.  The new base is $9$ and the new leading coefficient is $\frac{4}{3}$.




The leading coefficient is $\frac{4}{3} > 0$ and the base is $9 > 1$. Therefore, $G(t)$ is an increasing function.  The function values will be positive, once they overcome the drop of $5$ to the horizontal asymptote.


Since the new base $9$ is greater than the old base of $3$, the funciton grows faster.  The graph goes up to the right faster and steeper.  This means the graph is compressed horizontally, which is the result of multiplying by $2$ in the exponent.





End-Behavior:
\[
\lim\limits^{t \to -\infty}G(t) = -5 \, \text{ and } \, \lim\limits^{t \to \infty}G(t) = \infty
\]




\end{example}










\begin{observation} \textbf{\textcolor{blue!55!black}{For $r > 1$}}     \\



When multiplying by $b$ on the inside for a horizontal stretch or compression, 


if $1 < b$, then we are compressing horizontally.  This works out in the algebra. When $1 < b$, then $r < r^b$. The base becomes larger, which means the function grows faster, which means the graph goes up faster, which means the graph has been compressed horizontally. \\



if $0 < b < 1$, then we are stretching horizontally.  This works out in the algebra. When $0 < b < 1$, then $r^b < r$. The base becomes smaller, which means the function grows slower, which means the graph goes up slower, which means the graph has been stretched horizontally.


\end{observation}









\begin{example}  General Exponential Function



Analyze   $B(t) = -2 \, \left( \frac{2}{3} \right)^{3-t} + 4$ \\


\begin{explanation}


\textbf{\textcolor{purple!85!blue}{Formula dissection:}}  \\


$\blacktriangleright$  the base: $\frac{2}{3} < 1$\\
$\blacktriangleright$  the exponent: $3-t$ is positive for large negative $t$. \\
$\blacktriangleright$  the exponent: $3-t$ is negative for large positive $t$. \\
$\blacktriangleright$  the leading coefficient is negative. \\


Together, these tell us that $B(t)$ settles down for large negative $t$ and that $B(t)$ becomes unbounded when $t$ is large and positive.



This is a transformed version of the basic exponential function template $\left( \frac{2}{3} \right)^{-t} = \left( \frac{3}{2} \right)^t$.  



When $t < 0$, then $-t > 0$ and we get  $\left( \frac{2}{3} \right)^{positive}$ and the basic exponential portion is becoming smaller, approaching $0$.  





\[ \lim\limits_{t \to -\infty} \left( \frac{2}{3} \right)^{-t}  = 0 \]



When $t > 0$, then $-t < 0$ and we get  $\left( \frac{2}{3} \right)^{negative} = \left( \frac{3}{2} \right)^{positive}$ and $B(t)$ is becoming unbounded.  



\[ \lim\limits_{t \to \infty} \left( \frac{2}{3} \right)^{-t}  = \lim\limits_{t \to \infty} \left( \frac{3}{2} \right)^t  = \infty \]




Exponential growth to the right and decay to the left.






Since $a = -2 < 0$ these smaller/larger values of $-2 \, \left( \frac{2}{3} \right)^{-t}$ are smaller/larger negative values.



We also have two shifts from the $3$ in the exponent and the constant term, $4$:




$\blacktriangleright$ \textbf{Vertical Shift}


Adding $4$ to the outside shifts the graph vertically up $4$.  The asymptote is $y = 4$ and 

\[ \lim\limits_{t \to -\infty} B(t) = 4 \]



$\blacktriangleright$ \textbf{Horizontal Shift}

Our exponent here is $3 - t$.  Our function's exponent is zero, $3-t=0$ when $t=3$. Our one anchor point is shifted over to $3$.  Multipying by $-2$, means the dot is $2$ away(below) from the horizontal asymptote, which is now $y=4$.  Our anchor point is $(3, 2)$.








Graph of $y = B(t)$.

\begin{image}
\begin{tikzpicture}
  \begin{axis}[
            domain=-10:10, ymax=10, xmax=10, ymin=-10, xmin=-10,
            axis lines =center, xlabel=$t$, ylabel=$y$, grid = major,
            ytick={-10,-8,-6,-4,-2,2,4,6,8,10},
          	xtick={-10,-8,-6,-4,-2,2,4,6,8,10},
          	ticklabel style={font=\scriptsize},
            every axis y label/.style={at=(current axis.above origin),anchor=south},
            every axis x label/.style={at=(current axis.right of origin),anchor=west},
            axis on top
          ]
          
          \addplot [line width=1, gray, dashed,samples=200,domain=(-10:10),<->] {4};

      	  \addplot [line width=2, penColor, smooth,samples=200,domain=(-10:7.5),<->] {-2 * (0.666^(3-x)) + 4};

      	  \addplot[color=penColor,fill=penColor,only marks,mark=*] coordinates{(3,2)};

          


  \end{axis}
\end{tikzpicture}
\end{image}




Our analysis tells us that 

\begin{itemize}
\item The natural or implied domain of $B$ is $\mathbb{R}$.
\item $B$ is always decreasing.
\item $B$ has no maximums or minimums.
\item $\lim\limits_{t \to -\infty} B(t) = 4$
\item $\lim\limits_{t \to \infty} B(t) = -\infty$
\end{itemize}


\end{explanation}

\end{example}





In the example above, we could have algebraically moved the horizontal shift to the leading coefficient.



\[
-2 \, \left( \frac{2}{3} \right)^{3-t} + 4
\]


\[
-2 \, \left( \frac{2}{3} \right)^{3} \cdot \left( \frac{2}{3} \right)^{-t}+ 4
\]


\[
-2 \, \left( \frac{8}{27} \right)  \cdot \left( \frac{2}{3} \right)^{-t}+ 4
\]


\[
\left( \frac{-16}{27} \right)  \cdot \left( \frac{2}{3} \right)^{-t}+ 4
\]


We could have further transfered the negative sign in the exponent to the base by reciprocating the base.

\[
\left( \frac{-16}{27} \right)  \cdot \left( \frac{3}{2} \right)^t+ 4
\]



Algebra provides many tools for modifying the representing formula. \\








\begin{example}  General Exponential Function



Analyze   $K(f) = 3^{5-f} - 5$ \\

\begin{question}. 

The base is \wordChoice{\choice{less than}\choice[correct]{greater than}} $1$.
\end{question}
\begin{question}. 

The exponent gets big and positive when $f$ gets big and \wordChoice{\choice{positive}\choice[correct]{negative}}.
\end{question}
\begin{question}. 

The graph will become unbounded to the \wordChoice{\choice{right}\choice[correct]{left}}.\\
\end{question}

\begin{question}. 

The horizontal asymptote is $y = \answer{-5}$.
\end{question}

\begin{question}. 

The graph will approach the asymptote to the \wordChoice{\choice[correct]{right}\choice{left}}.\\
\end{question}
\begin{question}. 

Our one strategic anchor point moves to $\left(\answer{5}, \answer{-4}\right)$.
\end{question}
\begin{question}. 

The graph will become unbounded \wordChoice{\choice[correct]{up}\choice{down}}.\\
\end{question}




Graph of $y = K(f)$.

\begin{image}
\begin{tikzpicture}
  \begin{axis}[
            domain=-10:10, ymax=10, xmax=10, ymin=-10, xmin=-10,
            axis lines =center, xlabel=$f$, ylabel=$y$, grid = major,
            ytick={-10,-8,-6,-4,-2,2,4,6,8,10},
          	xtick={-10,-8,-6,-4,-2,2,4,6,8,10},
          	ticklabel style={font=\scriptsize},
            every axis y label/.style={at=(current axis.above origin),anchor=south},
            every axis x label/.style={at=(current axis.right of origin),anchor=west},
            axis on top
          ]
          
          \addplot [line width=1, gray, dashed,samples=200,domain=(-10:10),<->] {-5};

      	  \addplot [line width=2, penColor, smooth,samples=200,domain=(2.75:10),<->] {3^(5-x) - 5};

      	  \addplot[color=penColor,fill=penColor,only marks,mark=*] coordinates{(5,-4)};

 

  \end{axis}
\end{tikzpicture}
\end{image}





Our analysis tells us that

\begin{itemize}
\item The natural or implied domain of $K$ is $\mathbb{R}$.
\item $K$ is always decreasing.
\item $K$ has no maximums or minimums.
\item $\lim\limits_{f \to -\infty} K(f) = \infty$
\item $\lim\limits_{f \to \infty} K(f) = -5$
\end{itemize}


\end{example}
























\end{document}
