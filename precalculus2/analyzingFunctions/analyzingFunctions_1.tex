\documentclass{ximera}


\graphicspath{
  {./}
  {ximeraTutorial/}
  {basicPhilosophy/}
}

\newcommand{\mooculus}{\textsf{\textbf{MOOC}\textnormal{\textsf{ULUS}}}}

\usepackage{tkz-euclide}\usepackage{tikz}
\usepackage{tikz-cd}
\usetikzlibrary{arrows}
\tikzset{>=stealth,commutative diagrams/.cd,
  arrow style=tikz,diagrams={>=stealth}} %% cool arrow head
\tikzset{shorten <>/.style={ shorten >=#1, shorten <=#1 } } %% allows shorter vectors

\usetikzlibrary{backgrounds} %% for boxes around graphs
\usetikzlibrary{shapes,positioning}  %% Clouds and stars
\usetikzlibrary{matrix} %% for matrix
\usepgfplotslibrary{polar} %% for polar plots
\usepgfplotslibrary{fillbetween} %% to shade area between curves in TikZ
\usetkzobj{all}
\usepackage[makeroom]{cancel} %% for strike outs
%\usepackage{mathtools} %% for pretty underbrace % Breaks Ximera
%\usepackage{multicol}
\usepackage{pgffor} %% required for integral for loops



%% http://tex.stackexchange.com/questions/66490/drawing-a-tikz-arc-specifying-the-center
%% Draws beach ball
\tikzset{pics/carc/.style args={#1:#2:#3}{code={\draw[pic actions] (#1:#3) arc(#1:#2:#3);}}}



\usepackage{array}
\setlength{\extrarowheight}{+.1cm}
\newdimen\digitwidth
\settowidth\digitwidth{9}
\def\divrule#1#2{
\noalign{\moveright#1\digitwidth
\vbox{\hrule width#2\digitwidth}}}






\DeclareMathOperator{\arccot}{arccot}
\DeclareMathOperator{\arcsec}{arcsec}
\DeclareMathOperator{\arccsc}{arccsc}

















%%This is to help with formatting on future title pages.
\newenvironment{sectionOutcomes}{}{}


\title{Analyzing}

\begin{document}

\begin{abstract}
describe everything
\end{abstract}
\maketitle



$\blacktriangleright$ \textbf{\textcolor{red!80!black}{Reasoning:}} Reasoning is a logical explanation that describes our conclusions, how we arrived at those conclusions, and why we think those conclusions are correct. \\

Analysis is not a list of conclusions. We are not looking for such a list. \\

We are looking for the thought process that arrived at the list of conclusions. \\








\begin{example}

\textbf{\textcolor{purple!85!blue}{Completely analyze $K(x) = \ln(x^2+2x+3)$}} \\



$\blacktriangleright$  $K$ is a composition of a logarithm function and a quadratic function. Therefore, the natural or implied domain is all real numbers that make the inside function, $inside(v)=v^2+2v+3$, greater than $0$.  So, we need some information about $insdie(v)=v^2+2v+3$.   





Graph of $y = inside(v)$




\begin{image}
\begin{tikzpicture}
  \begin{axis}[
            domain=-10:10, ymax=10, xmax=10, ymin=-10, xmin=-10,
            axis lines =center, xlabel=$v$, ylabel={$y=inside(v)$}, grid = major, grid style={dashed},
            ytick={-10,-8,-6,-4,-2,2,4,6,8,10},
            xtick={-10,-8,-6,-4,-2,2,4,6,8,10},
            yticklabels={$-10$,$-8$,$-6$,$-4$,$-2$,$2$,$4$,$6$,$8$,$10$}, 
            xticklabels={$-10$,$-8$,$-6$,$-4$,$-2$,$2$,$4$,$6$,$8$,$10$},
            ticklabel style={font=\scriptsize},
            every axis y label/.style={at=(current axis.above origin),anchor=south},
            every axis x label/.style={at=(current axis.right of origin),anchor=west},
            axis on top
          ]
          
          %\addplot [line width=2, penColor2, smooth,samples=100,domain=(-6:2)] {-2*x-3};
          \addplot [line width=2, penColor, smooth,samples=100,domain=(-3.8:1.8),<->] {x^2 + 2*x + 3};



           

  \end{axis}
\end{tikzpicture}
\end{image}




$inside(v)$ is a \wordChoice{\choice{linear} \choice{radical} \choice[correct]{quadratic} \choice{rational}}  function.  Its leading coefficient is \wordChoice{\choice[correct]{positive} \choice{negative}}, therefore the graph is a parabola opening \wordChoice{\choice[correct]{up} \choice{down}} .  


That tells us that $inside(v)$ has an absolute \wordChoice{\choice{maximum} \choice[correct]{minimum}}.  The quadratic formula tells us that the minimum occurs at $v=\frac{-b}{2a} = \frac{\answer{-2}}{\answer{2}} = -1$.

The minimum value of $inside(v)$ is $inside(-1) = \answer{2}$.  Therefore, $inside(v) > 0$ for all $v$. Therefore, all values of $inside(v)$ are in the domain of the natural logarithm.

Therefore, the natural or implied domain of $K(x)$ is all real numbers.



As we move away from $-1$, in both directions, in the domain, $inside(v)$ gets larger positively. Since the natural logarithm function is an increasing function, the values of $K(x)$ must also get larger as we move in both directions away from $-1$.




$K(-1) = \ln(2)$ is the minimum value of $K$.  The graph has a lowest point at $\left( \answer{-1}, \answer{\ln(2)} \right)$ and the graph must go up in both directions from there.








\begin{image}
\begin{tikzpicture}
  \begin{axis}[
            domain=-10:10, ymax=10, xmax=10, ymin=-10, xmin=-10,
            axis lines =center, xlabel=$x$, ylabel={$y=K(x)$}, grid = major, grid style={dashed},
            ytick={-10,-8,-6,-4,-2,2,4,6,8,10},
            xtick={-10,-8,-6,-4,-2,2,4,6,8,10},
            yticklabels={$-10$,$-8$,$-6$,$-4$,$-2$,$2$,$4$,$6$,$8$,$10$}, 
            xticklabels={$-10$,$-8$,$-6$,$-4$,$-2$,$2$,$4$,$6$,$8$,$10$},
            ticklabel style={font=\scriptsize},
            every axis y label/.style={at=(current axis.above origin),anchor=south},
            every axis x label/.style={at=(current axis.right of origin),anchor=west},
            axis on top
          ]
          
          %\addplot [line width=2, penColor2, smooth,samples=100,domain=(-6:2)] {-2*x-3};
          \addplot [line width=2, penColor, smooth,samples=100,domain=(-8:8),<->] {ln(x^2 + 2*x + 3)};

          %\addplot[color=penColor,fill=penColor2,only marks,mark=*] coordinates{(-6,9)};
          %\addplot[color=penColor,fill=penColor2,only marks,mark=*] coordinates{(2,-7)};

          %\addplot[color=penColor2,fill=white,only marks,mark=*] coordinates{(2,-4.5)};
          %\addplot[color=penColor2,fill=white,only marks,mark=*] coordinates{(8,6)};


           

  \end{axis}
\end{tikzpicture}
\end{image}








$\blacktriangleright$ Since $inside(v) = v^2 + 2v + 3$ is a quadratic with postive leading coefficient, we know that 
\[  \lim\limits_{v \to \infty}inside(v) = \answer{\infty}  \, \text{ and } \,  \lim\limits_{v \to -\infty}inside(v) = \answer{\infty}  \]


$\blacktriangleright$  The range is $[\ln(2), \infty)$.

$\blacktriangleright$  $K(x)$ decreases on $\left( -\infty, \answer{-1} \right]$ and increases on $\left[ \answer{-1}, \infty \right)$.


$\blacktriangleright$  $K(x)$ has a global minimum of $\answer{\ln(2)}$, which occurs at $-1$.  This is also the only local minimum, because $K$ is decreasing and then increasing.

$\blacktriangleright$  $K(x)$ has no global or local maximum, because $v^2 + 2v + 3$ is unbounded and $ln(t)$ is unbounded.

$\blacktriangleright$  $-1$ is the only critical number. It is the only location of an extreme value.


\end{example}










\section{with Calculus}

Calculus will give us the derivative: $K'(x) = \frac{2x+2}{x^2+2x+3}$.  We could then solve $K'(x) = 0$ and get $x=-1$ as the only critical number, which agress with what we found.











\begin{center}
\textbf{\textcolor{green!50!black}{ooooo=-=-=-=-=-=-=-=-=-=-=-=-=ooOoo=-=-=-=-=-=-=-=-=-=-=-=-=ooooo}} \\

more examples can be found by following this link\\ \link[More Examples of Analyzing Functions]{https://ximera.osu.edu/csccmathematics/precalculus2/precalculus2/analyzingFunctions/examples/exampleList}

\end{center}




\end{document}
