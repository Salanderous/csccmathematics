\documentclass{ximera}


\graphicspath{
  {./}
  {ximeraTutorial/}
  {basicPhilosophy/}
}

\newcommand{\mooculus}{\textsf{\textbf{MOOC}\textnormal{\textsf{ULUS}}}}

\usepackage{tkz-euclide}\usepackage{tikz}
\usepackage{tikz-cd}
\usetikzlibrary{arrows}
\tikzset{>=stealth,commutative diagrams/.cd,
  arrow style=tikz,diagrams={>=stealth}} %% cool arrow head
\tikzset{shorten <>/.style={ shorten >=#1, shorten <=#1 } } %% allows shorter vectors

\usetikzlibrary{backgrounds} %% for boxes around graphs
\usetikzlibrary{shapes,positioning}  %% Clouds and stars
\usetikzlibrary{matrix} %% for matrix
\usepgfplotslibrary{polar} %% for polar plots
\usepgfplotslibrary{fillbetween} %% to shade area between curves in TikZ
\usetkzobj{all}
\usepackage[makeroom]{cancel} %% for strike outs
%\usepackage{mathtools} %% for pretty underbrace % Breaks Ximera
%\usepackage{multicol}
\usepackage{pgffor} %% required for integral for loops



%% http://tex.stackexchange.com/questions/66490/drawing-a-tikz-arc-specifying-the-center
%% Draws beach ball
\tikzset{pics/carc/.style args={#1:#2:#3}{code={\draw[pic actions] (#1:#3) arc(#1:#2:#3);}}}



\usepackage{array}
\setlength{\extrarowheight}{+.1cm}
\newdimen\digitwidth
\settowidth\digitwidth{9}
\def\divrule#1#2{
\noalign{\moveright#1\digitwidth
\vbox{\hrule width#2\digitwidth}}}






\DeclareMathOperator{\arccot}{arccot}
\DeclareMathOperator{\arcsec}{arcsec}
\DeclareMathOperator{\arccsc}{arccsc}

















%%This is to help with formatting on future title pages.
\newenvironment{sectionOutcomes}{}{}


\title{Analyzing Functions}

\begin{document}

\begin{abstract}
%
\end{abstract}
\maketitle






What do we want when analyzing a function?


We want \textbf{\textcolor{red!80!black}{reasoning}} on how we figure out

\begin{itemize}
\item domain
\item zeros 
\item discontinuities and singularities
\item intervals of continuity
\item critical numbers
\item intervals where increasing and decreasing
\item global maximum and minimum
\item local maximums and minimums
\item end-behavior
\item limiting behavior
\item corresponding important points
\item a nice graph, including auxillary graphing items
\begin{itemize}
	\item intercepts
	\item endpoints
	\item vertical asymptotes
	\item horizontal asymptotes
\end{itemize}
\item range
\end{itemize}



$\blacktriangleright$ \textbf{\textcolor{red!80!black}{Reasoning:}} Reasoning is a logical explanation that describes our conclusions, how we arrived at those conclusions, and why we think those conclusions are correct. \\

Analysis is not a list of conclusions. We are not looking for such a list. \\

We are looking for the thought process that arrived at the list of conclusions. \\





\subsection{Learning Outcomes}


\begin{sectionOutcomes}
In this section, students will 

\begin{itemize}
\item analyze functions algebraically from their formula.
\end{itemize}
\end{sectionOutcomes}


















\begin{center}
\textbf{\textcolor{green!50!black}{ooooo=-=-=-=-=-=-=-=-=-=-=-=-=ooOoo=-=-=-=-=-=-=-=-=-=-=-=-=ooooo}} \\

more examples can be found by following this link\\ \link[More Examples of Analyzing Functions]{https://ximera.osu.edu/csccmathematics/precalculus2/precalculus2/analyzingFunctions/examples/exampleList}

\end{center}







\end{document}
