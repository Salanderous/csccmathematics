\documentclass{ximera}


\graphicspath{
  {./}
  {ximeraTutorial/}
  {basicPhilosophy/}
}

\newcommand{\mooculus}{\textsf{\textbf{MOOC}\textnormal{\textsf{ULUS}}}}

\usepackage{tkz-euclide}\usepackage{tikz}
\usepackage{tikz-cd}
\usetikzlibrary{arrows}
\tikzset{>=stealth,commutative diagrams/.cd,
  arrow style=tikz,diagrams={>=stealth}} %% cool arrow head
\tikzset{shorten <>/.style={ shorten >=#1, shorten <=#1 } } %% allows shorter vectors

\usetikzlibrary{backgrounds} %% for boxes around graphs
\usetikzlibrary{shapes,positioning}  %% Clouds and stars
\usetikzlibrary{matrix} %% for matrix
\usepgfplotslibrary{polar} %% for polar plots
\usepgfplotslibrary{fillbetween} %% to shade area between curves in TikZ
\usetkzobj{all}
\usepackage[makeroom]{cancel} %% for strike outs
%\usepackage{mathtools} %% for pretty underbrace % Breaks Ximera
%\usepackage{multicol}
\usepackage{pgffor} %% required for integral for loops



%% http://tex.stackexchange.com/questions/66490/drawing-a-tikz-arc-specifying-the-center
%% Draws beach ball
\tikzset{pics/carc/.style args={#1:#2:#3}{code={\draw[pic actions] (#1:#3) arc(#1:#2:#3);}}}



\usepackage{array}
\setlength{\extrarowheight}{+.1cm}
\newdimen\digitwidth
\settowidth\digitwidth{9}
\def\divrule#1#2{
\noalign{\moveright#1\digitwidth
\vbox{\hrule width#2\digitwidth}}}






\DeclareMathOperator{\arccot}{arccot}
\DeclareMathOperator{\arcsec}{arcsec}
\DeclareMathOperator{\arccsc}{arccsc}

















%%This is to help with formatting on future title pages.
\newenvironment{sectionOutcomes}{}{}


\title{Analyzing}

\begin{document}

\begin{abstract}
describe everything
\end{abstract}
\maketitle







Completely analyze $L(x) = \frac{5}{1+3 e^{-\tfrac{x}{2}}}$






$\blacktriangleright$  \textbf{Domain:} $x$ is a real number and the range of the exponential function does not include negative numbers. Therefore, the denominator of $L(x)$ cannot be $0$.  That makes the implied domain all real numbers.






$\blacktriangleright$ \textbf{Range:} For the range, we need to determine the endbehavior.



\[   \lim_{x \to \infty} \frac{5}{1+3 e^{-\tfrac{x}{2}}} =   \lim_{x \to \infty} \frac{5}{1 + 0}   = 5 \]



\[   \lim_{x \to -\infty} \frac{5}{1+3 e^{-\tfrac{x}{2}}} =   \lim_{x \to -\infty} \frac{5}{3 e^{-\tfrac{x}{2}}}   = 0 \]



Our range is $(0, 5)$.


The graph has horizontal asymptotes.










\begin{image}
\begin{tikzpicture}
  \begin{axis}[
            domain=-10:10, ymax=6, xmax=10, ymin=-1, xmin=-10,
            axis lines =center, xlabel=$x$, ylabel={$y=L(x)$}, grid = major, grid style={dashed},
            ytick={2,4,6},
            xtick={-10,-8,-6,-4,-2,2,4,6,8,10},
            yticklabels={$2$,$4$,$6$}, 
            xticklabels={$-10$,$-8$,$-6$,$-4$,$-2$,$2$,$4$,$6$,$8$,$10$},
            ticklabel style={font=\scriptsize},
            every axis y label/.style={at=(current axis.above origin),anchor=south},
            every axis x label/.style={at=(current axis.right of origin),anchor=west},
            axis on top
          ]
          

            %\addplot [line width=2, penColor, smooth,samples=100,domain=(-9:9)] {5/(1 + 3 * e^(-x/2))};
            \addplot [line width=2, penColor, smooth,samples=100,domain=(-9:9),<->] {5/(1 + 3 * 2.7182^(-x/2))};

            \addplot [line width=1, gray, dashed,samples=100,domain=(-10:10),<->] {5};
            \addplot [line width=1, gray, dashed,samples=100,domain=(-10:10),<->] {0};




           

  \end{axis}
\end{tikzpicture}
\end{image}










$\blacktriangleright$ \textbf{Rate-of-Change:}    Since $e^{-\tfrac{x}{2}}$ is a strictly decreasing function, and it is in the denminator of $L(x)$, we know that $L(x)$ will be a strictly increasing function.





$\blacktriangleright$ \textbf{Extrema:} $L(x)$ has no global or local maximums or minimums and so no crtical numbers



\section{with Calculus}


Calculus will give us a formula for the derivative.

\[    \frac{-15}{2} \cdot \frac{e^{-\tfrac{x}{2}}}{\left(1+3 e^{-\tfrac{x}{2}}\right)^2}    \]


This function has no zeros, which tells us that $L(x)$ has no critical numbers.







\end{document}
