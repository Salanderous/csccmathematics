\documentclass{ximera}


\graphicspath{
  {./}
  {ximeraTutorial/}
  {basicPhilosophy/}
}

\newcommand{\mooculus}{\textsf{\textbf{MOOC}\textnormal{\textsf{ULUS}}}}

\usepackage{tkz-euclide}\usepackage{tikz}
\usepackage{tikz-cd}
\usetikzlibrary{arrows}
\tikzset{>=stealth,commutative diagrams/.cd,
  arrow style=tikz,diagrams={>=stealth}} %% cool arrow head
\tikzset{shorten <>/.style={ shorten >=#1, shorten <=#1 } } %% allows shorter vectors

\usetikzlibrary{backgrounds} %% for boxes around graphs
\usetikzlibrary{shapes,positioning}  %% Clouds and stars
\usetikzlibrary{matrix} %% for matrix
\usepgfplotslibrary{polar} %% for polar plots
\usepgfplotslibrary{fillbetween} %% to shade area between curves in TikZ
\usetkzobj{all}
\usepackage[makeroom]{cancel} %% for strike outs
%\usepackage{mathtools} %% for pretty underbrace % Breaks Ximera
%\usepackage{multicol}
\usepackage{pgffor} %% required for integral for loops



%% http://tex.stackexchange.com/questions/66490/drawing-a-tikz-arc-specifying-the-center
%% Draws beach ball
\tikzset{pics/carc/.style args={#1:#2:#3}{code={\draw[pic actions] (#1:#3) arc(#1:#2:#3);}}}



\usepackage{array}
\setlength{\extrarowheight}{+.1cm}
\newdimen\digitwidth
\settowidth\digitwidth{9}
\def\divrule#1#2{
\noalign{\moveright#1\digitwidth
\vbox{\hrule width#2\digitwidth}}}






\DeclareMathOperator{\arccot}{arccot}
\DeclareMathOperator{\arcsec}{arcsec}
\DeclareMathOperator{\arccsc}{arccsc}

















%%This is to help with formatting on future title pages.
\newenvironment{sectionOutcomes}{}{}


\title{Hyperbolic Functions}

\begin{document}

\begin{abstract}
sinh and cosh
\end{abstract}
\maketitle



We say that $\sin(\theta)$ and $\cos(\theta)$ are parameterized by angle $\theta$. Given an angle $\theta$, draw a line from the origin at that angle.  The line intersects the unit circle.  $\sin(\theta)$ and $\cos(\theta)$ are defined to be the coordinates of that point.





\begin{image}
\begin{tikzpicture}
  \begin{axis}[
            xmin=-1.1,xmax=1.1,ymin=-1.1,ymax=1.1,
            axis lines=center,
            width=4in,
            xtick={-1,1},
            ytick={-1,1},
            clip=false,
            unit vector ratio*=1 1 1,
            xlabel=$x$, ylabel=$y$,
            ticklabel style={font=\scriptsize},
            every axis y label/.style={at=(current axis.above origin),anchor=south},
            every axis x label/.style={at=(current axis.right of origin),anchor=west},
          ]        
          \addplot [smooth, domain=(0:360)] ({cos(x)},{sin(x)}); %% unit circle

          \addplot [textColor] plot coordinates {(0,0) (0.766,0.643)}; %% 40 degrees
          \addplot[color=penColor,fill=penColor,only marks,mark=*] coordinates{(0.766,0.643)};

          %\addplot [ultra thick,penColor] plot coordinates {(.766,0) (.766,.643)}; %% 40 degrees
          %\addplot [ultra thick,penColor2] plot coordinates {(0,0) (.766,0)}; %% 40 degrees
          
          %\addplot [ultra thick,penColor3] plot coordinates {(1,0) (1,.839)}; %% 40 degrees          

          \addplot [textColor,smooth, domain=(0:40)] ({.15*cos(x)},{.15*sin(x)});
          %\addplot [very thick,penColor] plot coordinates {(0,0) (.766,.643)}; %% sector
          %\addplot [very thick,penColor] plot coordinates {(0,0) (1,0)}; %% sector
          %\addplot [very thick, penColor, smooth, domain=(0:40)] ({cos(x)},{sin(x)}); %% sector
          \node at (axis cs:.15,.07) [anchor=west] {$\theta$};
          %\node[penColor, rotate=-90] at (axis cs:.84,.322) {$\sin(\theta)$};
          \node[textColor] at (axis cs:0.766,0.7) [anchor=west] {$(\cos(\theta), \sin(\theta))$};
          %\node[penColor3, rotate=-90] at (axis cs:1.06,.322) {$\tan(\theta)$};
        \end{axis}
\end{tikzpicture}
\end{image}






We could parameterize $\sin(\theta)$ and $\cos(\theta)$ with something besides the angle.



\section{Arc Length}


If you begin at $(1,0)$ and move counterclockwise on the unit circle, then the arc length you have travelled identifies points just as well as an angle does.


If you travel an arc distance of $\ell$ on the unit circle, then you arrive at a point and its coordinates depend on $\ell$ insead of $\theta$: $\sin(\ell)$ and $\cos(\ell)$.









\begin{image}
\begin{tikzpicture}
  \begin{axis}[
            xmin=-1.1,xmax=1.1,ymin=-1.1,ymax=1.1,
            axis lines=center,
            width=4in,
            xtick={-1,1},
            ytick={-1,1},
            clip=false,
            unit vector ratio*=1 1 1,
            xlabel=$x$, ylabel=$y$,
            ticklabel style={font=\scriptsize},
            every axis y label/.style={at=(current axis.above origin),anchor=south},
            every axis x label/.style={at=(current axis.right of origin),anchor=west},
          ]        
          \addplot [smooth, domain=(0:360)] ({cos(x)},{sin(x)}); %% unit circle

          \addplot [textColor] plot coordinates {(0,0) (.766,.643)}; %% 40 degrees

          %\addplot [ultra thick,penColor] plot coordinates {(.766,0) (.766,.643)}; %% 40 degrees
          %\addplot [ultra thick,penColor2] plot coordinates {(0,0) (.766,0)}; %% 40 degrees
          
          %\addplot [ultra thick,penColor3] plot coordinates {(1,0) (1,.839)}; %% 40 degrees          

          %\addplot [textColor,smooth, domain=(0:40)] ({.15*cos(x)},{.15*sin(x)});
          %\addplot [very thick,penColor] plot coordinates {(0,0) (.766,.643)}; %% sector
          %\addplot [very thick,penColor] plot coordinates {(0,0) (1,0)}; %% sector
          %\addplot [very thick, penColor, smooth, domain=(0:40)] ({cos(x)},{sin(x)}); %% sector
          %\node at (axis cs:.15,.07) [anchor=west] {$\theta$};
          %\node[penColor, rotate=-90] at (axis cs:.84,.322) {$\sin(\ell)$};
          %\node[penColor2] at (axis cs:.383,0) [anchor=north] {$\cos(\ell)$};
          %\node[penColor3, rotate=-90] at (axis cs:1.06,.322) {$\tan(\theta)$};


			\addplot[color=penColor,fill=penColor,only marks,mark=*] coordinates{(0.766,0.643)};
          \addplot [textColor,smooth, domain=(0:40),<->] ({1.05*cos(x)},{1.05*sin(x)});
          \node at (axis cs:1,0.4) [anchor=west] {$\ell$};
           \node[textColor] at (axis cs:0.766,0.7) [anchor=west] {$(\cos(\ell), \sin(\ell))$};


        \end{axis}
\end{tikzpicture}
\end{image}





Of course, on the unit circle $\theta$ and $\ell$ have the exact same measurement.  The unit circle is $2\pi radians$ around.  The circumference of hte unit circle is $2\pi$.




























\section{Sector Area}


There are other choices for parameter.

If you begin at $(1,0)$ and move counterclockwise on the unit circle, then the arc length you have travelled identifies points just as well as an angle does.


If you travel an arc distance of $\ell$ on the unit circle, then you arrive at a point and its coordinates depend on $\ell$ insead of $\theta$: $\sin(\ell)$ and $\cos(\ell)$.









\begin{image}
\begin{tikzpicture}
  \begin{axis}[
            xmin=-1.1,xmax=1.1,ymin=-1.1,ymax=1.1,
            axis lines=center,
            width=4in,
            xtick={-1,1},
            ytick={-1,1},
            clip=false,
            unit vector ratio*=1 1 1,
            xlabel=$x$, ylabel=$y$,
            ticklabel style={font=\scriptsize},
            every axis y label/.style={at=(current axis.above origin),anchor=south},
            every axis x label/.style={at=(current axis.right of origin),anchor=west},
          ]  





			\addplot [draw=none, fill=lightgray,samples=200,domain=0.764:1] {sqrt(1-x^2)} \closedcycle;
            \addplot [draw=none, fill=lightgray,samples=200,domain=0:0.767] {0.8394*x} \closedcycle;

            %\node[textColor] at (axis cs:0.766,0.7) [anchor=west] {$(\cos(A), \sin(A))$};

			\addplot [smooth, domain=(0:360)] ({cos(x)},{sin(x)}); %% unit circle

			\addplot [textColor] plot coordinates {(0,0) (.766,.643)}; %% 40 degrees

			\node[black] at (axis cs:0.5,0.25) [anchor=west] {\Large \textbf{$A$}};
			\addplot[color=penColor,fill=penColor,only marks,mark=*] coordinates{(0.766,0.643)};

          %\addplot [ultra thick,penColor] plot coordinates {(.766,0) (.766,.643)}; %% 40 degrees
          %\addplot [ultra thick,penColor2] plot coordinates {(0,0) (.766,0)}; %% 40 degrees
          
          %\addplot [ultra thick,penColor3] plot coordinates {(1,0) (1,.839)}; %% 40 degrees          

          %\addplot [textColor,smooth, domain=(0:40)] ({.15*cos(x)},{.15*sin(x)});
          %\addplot [very thick,penColor] plot coordinates {(0,0) (.766,.643)}; %% sector
          %\addplot [very thick,penColor] plot coordinates {(0,0) (1,0)}; %% sector
          %\addplot [very thick, penColor, smooth, domain=(0:40)] ({cos(x)},{sin(x)}); %% sector
          %\node at (axis cs:.15,.07) [anchor=west] {$\theta$};
          %\node[penColor, rotate=-90] at (axis cs:.84,.322) {$\sin(\ell)$};
          %\node[penColor2] at (axis cs:.383,0) [anchor=north] {$\cos(\ell)$};
          %\node[penColor3, rotate=-90] at (axis cs:1.06,.322) {$\tan(\theta)$};



          %\addplot [textColor,smooth, domain=(0:40),<->] ({1.05*cos(x)},{1.05*sin(x)});
          %\node at (axis cs:1,0.4) [anchor=west] {$\ell$};





          %\draw [thick,fill opacity=0.25,fill=gray] (axis cs:0,0) -- (axis cs:0,1) arc(90:180:3.1415cm) -- cycle;






        \end{axis}
\end{tikzpicture}
\end{image}




Actually, this isn't much different that the other two parameterizations.

If the sector seewps out an angle $\theta$, then the area of the sector is $A = \frac{\theta}{2\pi} \pi 1^2 = \frac{\theta}{2}$

The area is just half the angle.  

Therefore, we can connect this back up to the two other parameterizations, by $\theta = 2 A$.










\begin{image}
\begin{tikzpicture}
  \begin{axis}[
            xmin=-1.1,xmax=1.1,ymin=-1.1,ymax=1.1,
            axis lines=center,
            width=4in,
            xtick={-1,1},
            ytick={-1,1},
            clip=false,
            unit vector ratio*=1 1 1,
            xlabel=$x$, ylabel=$y$,
            ticklabel style={font=\scriptsize},
            every axis y label/.style={at=(current axis.above origin),anchor=south},
            every axis x label/.style={at=(current axis.right of origin),anchor=west},
          ]  





			\addplot [draw=none, fill=lightgray,samples=200,domain=0.764:1] {sqrt(1-x^2)} \closedcycle;
            \addplot [draw=none, fill=lightgray,samples=200,domain=0:0.767] {0.8394*x} \closedcycle;

            \node[textColor] at (axis cs:0.766,0.7) [anchor=west] {$(\cos(2A), \sin(2A))$};

			\addplot [smooth, domain=(0:360)] ({cos(x)},{sin(x)}); %% unit circle

			\addplot [textColor] plot coordinates {(0,0) (.766,.643)}; %% 40 degrees

			\node[black] at (axis cs:0.5,0.25) [anchor=west] {\Large \textbf{$A$}};
			\addplot[color=penColor,fill=penColor,only marks,mark=*] coordinates{(0.766,0.643)};

          %\addplot [ultra thick,penColor] plot coordinates {(.766,0) (.766,.643)}; %% 40 degrees
          %\addplot [ultra thick,penColor2] plot coordinates {(0,0) (.766,0)}; %% 40 degrees
          
          %\addplot [ultra thick,penColor3] plot coordinates {(1,0) (1,.839)}; %% 40 degrees          

          %\addplot [textColor,smooth, domain=(0:40)] ({.15*cos(x)},{.15*sin(x)});
          %\addplot [very thick,penColor] plot coordinates {(0,0) (.766,.643)}; %% sector
          %\addplot [very thick,penColor] plot coordinates {(0,0) (1,0)}; %% sector
          %\addplot [very thick, penColor, smooth, domain=(0:40)] ({cos(x)},{sin(x)}); %% sector
          %\node at (axis cs:.15,.07) [anchor=west] {$\theta$};
          %\node[penColor, rotate=-90] at (axis cs:.84,.322) {$\sin(\ell)$};
          %\node[penColor2] at (axis cs:.383,0) [anchor=north] {$\cos(\ell)$};
          %\node[penColor3, rotate=-90] at (axis cs:1.06,.322) {$\tan(\theta)$};



          %\addplot [textColor,smooth, domain=(0:40),<->] ({1.05*cos(x)},{1.05*sin(x)});
          %\node at (axis cs:1,0.4) [anchor=west] {$\ell$};





          %\draw [thick,fill opacity=0.25,fill=gray] (axis cs:0,0) -- (axis cs:0,1) arc(90:180:3.1415cm) -- cycle;






        \end{axis}
\end{tikzpicture}
\end{image}








All of out trigonometric functions are based off of these coordinates of the unit circle.




















\begin{image}
\begin{tikzpicture}
  \begin{axis}[
            xmin=-1.1,xmax=1.1,ymin=-1.1,ymax=1.1,
            axis lines=center,
            width=4in,
            xtick={-1},
            ytick={-1,1},
            clip=false,
            unit vector ratio*=1 1 1,
            xlabel=$ $, ylabel=$ $,
            ticklabel style={font=\scriptsize},
            every axis y label/.style={at=(current axis.above origin),anchor=south},
            every axis x label/.style={at=(current axis.right of origin),anchor=west},
          ]        
          


          \draw [ultra thick] (axis cs:0,0) -- (axis cs:1.305,0);
          %\draw [ultra thick] (axis cs:0.766,0.643) -- (axis cs:1.305,0);
          \draw [ultra thick] (axis cs:0,1.557) -- (axis cs:1.305,0);
          \draw [ultra thick] (axis cs:0,1.557) -- (axis cs:0,0);



          \draw [thin] (axis cs:0.716,0.05) -- (axis cs:0.766,0.05);
          \draw [thin] (axis cs:0.716,0) -- (axis cs:0.716,0.05);

          \draw [thin] (axis cs:0.766,0.05) -- (axis cs:0.812,0.05);
          \draw [thin] (axis cs:0.812,0) -- (axis cs:0.812,0.05);

          \draw [thin] (axis cs:0.7,0.587) -- (axis cs:0.77,0.51);
          \draw [thin] (axis cs:0.77,0.52) -- (axis cs:0.84,0.57);





          \addplot [smooth, domain=(0:360)] ({cos(x)},{sin(x)}); %% unit circle

          \addplot [textColor] plot coordinates {(0,0) (.766,.643)}; %% 40 degrees

          \addplot [ultra thick,penColor] plot coordinates {(.766,0) (.766,.643)}; %% 40 degrees
          \addplot [ultra thick,penColor2] plot coordinates {(0,0) (.766,0)}; %% 40 degrees
          
          %\addplot [ultra thick,penColor3] plot coordinates {(1,0) (1,.839)}; %% 40 degrees          

          \addplot [textColor,smooth, domain=(0:40)] ({.15*cos(x)},{.15*sin(x)});
          %\addplot [very thick,penColor] plot coordinates {(0,0) (.766,.643)}; %% sector
          %\addplot [very thick,penColor] plot coordinates {(0,0) (1,0)}; %% sector
          %\addplot [very thick, penColor, smooth, domain=(0:40)] ({cos(x)},{sin(x)}); %% sector
          \node at (axis cs:.15,.07) [anchor=west] {$\theta$};
          %\node[penColor] at (axis cs:0.85,.27) {$b$};
          %\node[penColor2] at (axis cs:.383,0) [anchor=north] {$a$};
          %\node[penColor3, rotate=-90] at (axis cs:1.06,.322) {$\tan(\theta)$};
           \node[penColor] at (axis cs:0.37,0.4) {$1$};


          \node at (axis cs:0.84, 0.5) [anchor=north] {$\theta$};
          %\addplot [textColor,smooth, domain=(270:310)] ({0.15*cos(x)+0.766},{0.15*sin(x)+0.643});
          %\node[textColor] at (axis cs:1.1,0)[anchor=north] {$c$};
          %\node[textColor] at (axis cs:1.05,0.5)[anchor=north] {$h$};

          %\node[textColor] at (axis cs:0.4,1.3)[anchor=north] {$k$};
          \node at (axis cs:0.07,1.4) [anchor=north] {$\theta$};


          \node[textColor, rotate=-50] at (axis cs:1.05,0.5) {$\tan(\theta)$};
          \node[textColor, rotate=-50] at (axis cs:0.4,1.3) {$\cot(\theta)$};
          %\node[textColor] at (axis cs:0,0.75)[anchor=east] {$m$};
          \node[penColor, rotate=-90] at (axis cs:-0.09,0.6) {$\csc(\theta)$};


			\node[penColor, rotate=-90] at (axis cs:0.7,0.15) [anchor=east] {$\sin(\theta)$};
          \node[penColor2] at (axis cs:0.35,0.07) [anchor=west] {$\cos(\theta)$};
          \node[textColor] at (axis cs:0.6,-0.07) [anchor=west] {$\sec(\theta)$};


        \end{axis}
\end{tikzpicture}
\end{image}






All of out trigonometric functions are based off of these curve defined by $x^2 + y^2 = 1$.


But, a circle is just one type of \textbf{conic sections}.


We could do the same thing with the curve defined by $x^2 - y^2 = 1$ - a \textbf{hyperbola}.









\section{Hyperbolic Functions}


Let's repeat the whole process with $x^2 - y^2 = 1$.




The graph is a unit hyperbola, rather than a circle.


\begin{image}
\begin{tikzpicture}
  \begin{axis}[
            xmin=-1.1,xmax=1.1,ymin=-1.1,ymax=1.1,
            axis lines=center,
            width=4in,
            xtick={-1,1},
            ytick={-1,1},
            clip=false,
            unit vector ratio*=1 1 1,
            xlabel=$x$, ylabel=$y$,
            ticklabel style={font=\scriptsize},
            every axis y label/.style={at=(current axis.above origin),anchor=south},
            every axis x label/.style={at=(current axis.right of origin),anchor=west},
          ]        
          \addplot [thick,smooth, domain=(-1:1),<->] ({cosh(x)},{sinh(x)}); %% unit circle
          \addplot [thick,smooth, domain=(-1:1),<->] ({-cosh(x)},{sinh(x)});
          \addplot [thin, gray,dashed, domain=(-1.1:1.1),<->] {x};
          \addplot [thin, gray,dashed, domain=(-1.1:1.1),<->] {-x};


        \end{axis}
\end{tikzpicture}
\end{image}




The points on the parabola have two coordinates and we will call them \textbf{hyperbolic-sine} and \textbf{hyperbolic-cosine}.



We'll keep $\theta$ for circluar trig functions and adopt $\alpha$ as the hyperbolic angle.







\begin{image}
\begin{tikzpicture}
  \begin{axis}[
            xmin=0,xmax=4,ymin=-4,ymax=4,
            axis lines=center,
            width=4in,
            xtick={2,3,4},
            ytick={-4,-3,-2,-1,1,2,3,4},
            clip=false,
            %unit vector ratio*=1 1 1,
            xlabel=$x$, ylabel=$y$,
            ticklabel style={font=\scriptsize},
            every axis y label/.style={at=(current axis.above origin),anchor=south},
            every axis x label/.style={at=(current axis.right of origin),anchor=west},
          ]        


          	\addplot [thick,smooth, domain=(-2:2),<->] ({cosh(x)},{sinh(x)}); %% unit circle
          	 \addplot [thin, gray,dashed, domain=(0:4),<->] {x};
          	\addplot [thin, gray,dashed, domain=(0:4),<->] {-x};


			\addplot[color=penColor,fill=penColor,only marks,mark=*] coordinates{(2.352,2.129)};

			%\addplot [textColor] plot coordinates {(0,0) (2.352,2.129)}; %% 40 degrees

			\addplot [ultra thick,penColor] plot coordinates {(2.352,0) (2.352,2.129)}; %% 40 degrees
			\addplot [ultra thick,penColor2] plot coordinates {(0,0) (2.352,0)}; %% 40 degrees
          
			%\addplot [ultra thick,penColor3] plot coordinates {(1,0) (1,.839)}; %% 40 degrees          

			%\addplot [textColor,smooth, domain=(0:40)] ({.15*cos(x)},{.15*sin(x)});
			%\addplot [very thick,penColor] plot coordinates {(0,0) (.766,.643)}; %% sector
			%\addplot [very thick,penColor] plot coordinates {(0,0) (1,0)}; %% sector
			%\addplot [very thick, penColor, smooth, domain=(0:40)] ({cos(x)},{sin(x)}); %% sector
			%\node at (axis cs:0.4,0.25) [anchor=west] {$A$};
			\node[penColor, rotate=-90] at (axis cs:2.5,1.75) [anchor=west] {$\sinh(\alpha)$};
			\node[penColor2] at (axis cs:1,0) [anchor=north] {$\cosh(\alpha)$};
			%\node[penColor3, rotate=-90] at (axis cs:1.06,.322) {$\tan(\theta)$};
        \end{axis}
\end{tikzpicture}
\end{image}

We need a way to walk along the hyperbola, like we moved around the unit circle.







The hyperbolic trig functions are parameterized by area.



We can still think that there is an angle $\alpha$ and $area = A = \frac{\alpha}{2}$.  However, due to the hyperbola configuration, the angle is not the geometric focus.  Instead, the diagrams focus on area.





















\begin{image}
\begin{tikzpicture}
  \begin{axis}[
            xmin=0,xmax=4,ymin=-4,ymax=4,
            axis lines=center,
            width=4in,
            xtick={2,3,4},
            ytick={-4,-3,-2,-1,1,2,3,4},
            clip=false,
            %unit vector ratio*=1 1 1,
            xlabel=$x$, ylabel=$y$,
            ticklabel style={font=\scriptsize},
            every axis y label/.style={at=(current axis.above origin),anchor=south},
            every axis x label/.style={at=(current axis.right of origin),anchor=west},
          ]        

			\addplot [draw=none, fill=gray,samples=300,domain=0:2.352] {0.90519*x} \closedcycle;
			\addplot [draw=none, fill=white,samples=300,domain=0:1.5] ({cosh(x)},{sinh(x)}) \closedcycle;

          	\addplot [thick,smooth, domain=(-2:2),<->] ({cosh(x)},{sinh(x)}); %% unit circle
          	 \addplot [thin, gray,dashed, domain=(0:4),<->] {x};
          	\addplot [thin, gray,dashed, domain=(0:4),<->] {-x};


			\addplot[color=penColor,fill=penColor,only marks,mark=*] coordinates{(2.352,2.129)};

			\addplot [textColor] plot coordinates {(0,0) (2.352,2.129)}; %% 40 degrees

			\addplot [ultra thick,penColor] plot coordinates {(2.352,0) (2.352,2.129)}; %% 40 degrees
			\addplot [ultra thick,penColor2] plot coordinates {(0,0) (2.352,0)}; %% 40 degrees
          
			%\addplot [ultra thick,penColor3] plot coordinates {(1,0) (1,.839)}; %% 40 degrees          

			%\addplot [textColor,smooth, domain=(0:40)] ({.15*cos(x)},{.15*sin(x)});
			%\addplot [very thick,penColor] plot coordinates {(0,0) (.766,.643)}; %% sector
			%\addplot [very thick,penColor] plot coordinates {(0,0) (1,0)}; %% sector
			%\addplot [very thick, penColor, smooth, domain=(0:40)] ({cos(x)},{sin(x)}); %% sector
			\node at (axis cs:0.5,0.25) [anchor=west] {$A$};
			\node[penColor, rotate=-90] at (axis cs:2.5,1.75) [anchor=west] {$\sinh(\alpha)$};
			\node[penColor2] at (axis cs:1,0) [anchor=north] {$\cosh(\alpha)$};
			%\node[penColor3, rotate=-90] at (axis cs:1.06,.322) {$\tan(\theta)$};
        \end{axis}
\end{tikzpicture}
\end{image}



From this definition we get 

\[     (\cosh(\alpha))^2 - (\sinh(\alpha))^2 = 1       \]

\[     \cosh^2(\alpha) - \sinh^2(\alpha) = 1       \]

Compare this to the relationship for the circular trig functions:

\[     (\cos(\theta))^2 + (\sin(\theta))^2 = 1       \]

\[     \cos^2(\theta) + \sin^2(\theta) = 1       \]

















\end{document}
