\documentclass{ximera}


\graphicspath{
  {./}
  {ximeraTutorial/}
  {basicPhilosophy/}
}

\newcommand{\mooculus}{\textsf{\textbf{MOOC}\textnormal{\textsf{ULUS}}}}

\usepackage{tkz-euclide}\usepackage{tikz}
\usepackage{tikz-cd}
\usetikzlibrary{arrows}
\tikzset{>=stealth,commutative diagrams/.cd,
  arrow style=tikz,diagrams={>=stealth}} %% cool arrow head
\tikzset{shorten <>/.style={ shorten >=#1, shorten <=#1 } } %% allows shorter vectors

\usetikzlibrary{backgrounds} %% for boxes around graphs
\usetikzlibrary{shapes,positioning}  %% Clouds and stars
\usetikzlibrary{matrix} %% for matrix
\usepgfplotslibrary{polar} %% for polar plots
\usepgfplotslibrary{fillbetween} %% to shade area between curves in TikZ
\usetkzobj{all}
\usepackage[makeroom]{cancel} %% for strike outs
%\usepackage{mathtools} %% for pretty underbrace % Breaks Ximera
%\usepackage{multicol}
\usepackage{pgffor} %% required for integral for loops



%% http://tex.stackexchange.com/questions/66490/drawing-a-tikz-arc-specifying-the-center
%% Draws beach ball
\tikzset{pics/carc/.style args={#1:#2:#3}{code={\draw[pic actions] (#1:#3) arc(#1:#2:#3);}}}



\usepackage{array}
\setlength{\extrarowheight}{+.1cm}
\newdimen\digitwidth
\settowidth\digitwidth{9}
\def\divrule#1#2{
\noalign{\moveright#1\digitwidth
\vbox{\hrule width#2\digitwidth}}}






\DeclareMathOperator{\arccot}{arccot}
\DeclareMathOperator{\arcsec}{arcsec}
\DeclareMathOperator{\arccsc}{arccsc}

















%%This is to help with formatting on future title pages.
\newenvironment{sectionOutcomes}{}{}


\title{Sinh}

\begin{document}

\begin{abstract}
Analysis
\end{abstract}
\maketitle



Following a story similar to the unit circle, $\sinh(\alpha)$ and $\cosh(\alpha)$ are defined as the coordinates of the points on $x^2 - y^2 = 1$, where $A = \frac{\alpha}{2}$.










\begin{image}
\begin{tikzpicture}
  \begin{axis}[
            xmin=0,xmax=4,ymin=-4,ymax=4,
            axis lines=center,
            width=4in,
            xtick={2,3,4},
            ytick={-4,-3,-2,-1,1,2,3,4},
            clip=false,
            %unit vector ratio*=1 1 1,
            xlabel=$x$, ylabel=$y$,
            ticklabel style={font=\scriptsize},
            every axis y label/.style={at=(current axis.above origin),anchor=south},
            every axis x label/.style={at=(current axis.right of origin),anchor=west},
          ]        

			\addplot [draw=none, fill=gray,samples=300,domain=0:2.352] {0.90519*x} \closedcycle;
			\addplot [draw=none, fill=white,samples=300,domain=0:1.5] ({cosh(x)},{sinh(x)}) \closedcycle;

          	\addplot [thick,smooth, domain=(-2:2),<->] ({cosh(x)},{sinh(x)}); %% unit circle
          	 \addplot [thin, gray,dashed, domain=(0:4),<->] {x};
          	\addplot [thin, gray,dashed, domain=(0:4),<->] {-x};


			\addplot[color=penColor,fill=penColor,only marks,mark=*] coordinates{(2.352,2.129)};

			\addplot [textColor] plot coordinates {(0,0) (2.352,2.129)}; %% 40 degrees

			\addplot [ultra thick,penColor] plot coordinates {(2.352,0) (2.352,2.129)}; %% 40 degrees
			\addplot [ultra thick,penColor2] plot coordinates {(0,0) (2.352,0)}; %% 40 degrees
          
			%\addplot [ultra thick,penColor3] plot coordinates {(1,0) (1,.839)}; %% 40 degrees          

			%\addplot [textColor,smooth, domain=(0:40)] ({.15*cos(x)},{.15*sin(x)});
			%\addplot [very thick,penColor] plot coordinates {(0,0) (.766,.643)}; %% sector
			%\addplot [very thick,penColor] plot coordinates {(0,0) (1,0)}; %% sector
			%\addplot [very thick, penColor, smooth, domain=(0:40)] ({cos(x)},{sin(x)}); %% sector
			\node at (axis cs:0.5,0.25) [anchor=west] {$A$};
			\node[penColor, rotate=-90] at (axis cs:2.5,1.75) [anchor=west] {$\sinh(\alpha)$};
			\node[penColor2] at (axis cs:1,0) [anchor=north] {$\cosh(\alpha)$};
			%\node[penColor3, rotate=-90] at (axis cs:1.06,.322) {$\tan(\theta)$};
        \end{axis}
\end{tikzpicture}
\end{image}


From the geometry, we can get formulas for $\sinh(\alpha)$ and $\cosh(\alpha)$ that relate back to our Elementary Functions.



Let's zoom in on the diagram above. For formula purposes, let's use $b = \cosh(\alpha)$.










\begin{image}
\begin{tikzpicture}
  \begin{axis}[
            xmin=0,xmax=3,ymin=0,ymax=3,
            axis lines=center,
            width=4in,
            xtick={1,2,3},
            ytick={1,2,3},
            clip=false,
            %unit vector ratio*=1 1 1,
            xlabel=$x$, ylabel=$y$,
            ticklabel style={font=\scriptsize},
            every axis y label/.style={at=(current axis.above origin),anchor=south},
            every axis x label/.style={at=(current axis.right of origin),anchor=west},
          ]        

      \addplot [draw=none, fill=pink,samples=300,domain=0:2.352] {0.90519*x} \closedcycle;
      \addplot [draw=none, fill=teal,samples=300,domain=0:1.5] ({cosh(x)},{sinh(x)}) \closedcycle;

            \addplot [thick,smooth, domain=(0:1.7),->] ({cosh(x)},{sinh(x)}); %% unit circle
             \addplot [thin, gray,dashed, domain=(0:2.5),<->] {x};
            %\addplot [thin, gray,dashed, domain=(0:4),<->] {-x};


      \addplot[color=penColor,fill=penColor,only marks,mark=*] coordinates{(2.352,2.129)};

      \addplot [textColor] plot coordinates {(0,0) (2.352,2.129)}; %% 40 degrees

      \addplot [ultra thick,penColor] plot coordinates {(2.352,0) (2.352,2.129)}; %% 40 degrees
      \addplot [ultra thick,penColor2] plot coordinates {(0,0) (2.352,0)}; %% 40 degrees
          
      %\addplot [ultra thick,penColor3] plot coordinates {(1,0) (1,.839)}; %% 40 degrees          

      %\addplot [textColor,smooth, domain=(0:40)] ({.15*cos(x)},{.15*sin(x)});
      %\addplot [very thick,penColor] plot coordinates {(0,0) (.766,.643)}; %% sector
      %\addplot [very thick,penColor] plot coordinates {(0,0) (1,0)}; %% sector
      %\addplot [very thick, penColor, smooth, domain=(0:40)] ({cos(x)},{sin(x)}); %% sector
      \node at (axis cs:0.4,0.25) [anchor=west] {$A = \tfrac{\alpha}{2}$};
      %\node[penColor, rotate=-90] at (axis cs:2.5,1.75) [anchor=west] {$\sinh(\alpha)$};
      \node[penColor2] at (axis cs:2.4,0) [anchor=north] {$(b, 0)$};
      %\node[penColor3, rotate=-90] at (axis cs:1.06,.322) {$\tan(\theta)$};
        \end{axis}
\end{tikzpicture}
\end{image}





Calculus tells us that green area equals


\[      \int_1^b \sqrt{x^2-1} \, dx = \frac{b \sqrt{b^2 - 1} - \ln(b + \sqrt{b^2 - 1})}{2}           \]


The pink area is the area of the right triangle minues the green area.



\[   A = \frac{b \sqrt{b^2 - 1}}{2} -     \frac{b \sqrt{b^2 - 1} - \ln(b + \sqrt{b^2 - 1})}{2}         \]



\[    A =    \frac{\ln(b + \sqrt{b^2 - 1})}{2}     \]


This gives us 



 \[    \frac{\alpha}{2} =    \frac{\ln(b + \sqrt{b^2 - 1})}{2}     \]




 \[    \alpha  =   \ln(b + \sqrt{b^2 - 1})  \]

Now, solve for $b$.




 \[    e^{\alpha}  =   b + \sqrt{b^2 - 1}  \]


 \[    e^{\alpha} - b =   \sqrt{b^2 - 1}  \]


 \[    e^{2\alpha} - 2 b e^{\alpha}  + b^2   =   b^2 - 1  \]


 \[    e^{2\alpha} - 2 b e^{\alpha}   =    - 1  \]


 \[    e^{2\alpha} +1  =   2 b e^{\alpha}     \]


 \[    \frac{ 1 + e^{2\alpha}}{2 e^{\alpha}}  =   b    \]


 \[    \frac{ e^{-\alpha} + e^{\alpha}}{2}  =   b    \]


 \[    b = \frac{ e^{\alpha} + e^{-\alpha}}{2}    \]





\section{Graph}





From  $\cosh(\alpha) = \frac{ e^{\alpha} + e^{-\alpha}}{2}$, we can see the function is an even function.  That is, $\cosh(\alpha) = \cosh(-\alpha)$, making the graph symmetric.



We can also see that $\lim_{\alpha \to \infty} \cosh(\alpha) = \infty$ $\lim_{\alpha \to -\infty} \cosh(\alpha) = \infty$.

Finally, $\cosh(0) = 1$





Graph of $y = \cosh(\alpha)$.



\begin{image}
\begin{tikzpicture}
  \begin{axis}[
            domain=-10:10, ymax=10, xmax=10, ymin=-10, xmin=-10,
            axis lines =center, xlabel={$\alpha$}, ylabel=$y$, grid = major, grid style={dashed},
            ytick={-10,-8,-6,-4,-2,2,4,6,8,10},
            xtick={-10,-8,-6,-4,-2,2,4,6,8,10},
            yticklabels={$-10$,$-8$,$-6$,$-4$,$-2$,$2$,$4$,$6$,$8$,$10$}, 
            xticklabels={$-10$,$-8$,$-6$,$-4$,$-2$,$2$,$4$,$6$,$8$,$10$},
            ticklabel style={font=\scriptsize},
            every axis y label/.style={at=(current axis.above origin),anchor=south},
            every axis x label/.style={at=(current axis.right of origin),anchor=west},
            axis on top
          ]


          \addplot [line width=2, penColor, smooth,samples=200,domain=(-2.8:2.8),<->] {cosh(x)};



  \end{axis}
\end{tikzpicture}
\end{image}









The graph isn't a parabola.  Compare the graphs of $\cosh(x)$ and $x^2$.





\begin{image}
\begin{tikzpicture}
  \begin{axis}[
            domain=-4:4, ymax=10, xmax=4, ymin=0, xmin=-4,
            axis lines =center, xlabel={$\alpha$}, ylabel=$y$, grid = major, grid style={dashed},
            ytick={2,4,6,8,10},
            xtick={-4,-2,2,4},
            yticklabels={$2$,$4$,$6$,$8$,$10$}, 
            xticklabels={$-4$,$-2$,$2$,$4$},
            ticklabel style={font=\scriptsize},
            every axis y label/.style={at=(current axis.above origin),anchor=south},
            every axis x label/.style={at=(current axis.right of origin),anchor=west},
            axis on top
          ]


          \addplot [line width=2, penColor, smooth,samples=200,domain=(-2.8:2.8),<->] {cosh(x)};
          \addplot [line width=2, penColor2, smooth,samples=200,domain=(-2.8:2.8),<->] {x^2+1};



  \end{axis}
\end{tikzpicture}
\end{image}









Instead, the graph is a \textbf{catenary}.

A catenary is what you get when you have a telephone line from one pole to the next or a hanging chain.  The resulting curve has the unique property that the tension forces are all parallel to the curve. The telphone line of chain doesn't move side to side.

Now flip this curve upside down.  It is an arch with no sheer stress.  All of the forces are along the arch.  This makes it very stable. \\






$\blacktriangleright$ \textbf{End-Behavior}

Since $\cosh(x)$ is exponential, it is going to dominate over $x^2$.

\[     \lim_{x \to \infty}\frac{x^2}{\cosh(x)} = 0     \]









\end{document}
