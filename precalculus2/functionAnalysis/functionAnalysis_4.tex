\documentclass{ximera}


\graphicspath{
  {./}
  {ximeraTutorial/}
  {basicPhilosophy/}
}

\newcommand{\mooculus}{\textsf{\textbf{MOOC}\textnormal{\textsf{ULUS}}}}

\usepackage{tkz-euclide}\usepackage{tikz}
\usepackage{tikz-cd}
\usetikzlibrary{arrows}
\tikzset{>=stealth,commutative diagrams/.cd,
  arrow style=tikz,diagrams={>=stealth}} %% cool arrow head
\tikzset{shorten <>/.style={ shorten >=#1, shorten <=#1 } } %% allows shorter vectors

\usetikzlibrary{backgrounds} %% for boxes around graphs
\usetikzlibrary{shapes,positioning}  %% Clouds and stars
\usetikzlibrary{matrix} %% for matrix
\usepgfplotslibrary{polar} %% for polar plots
\usepgfplotslibrary{fillbetween} %% to shade area between curves in TikZ
\usetkzobj{all}
\usepackage[makeroom]{cancel} %% for strike outs
%\usepackage{mathtools} %% for pretty underbrace % Breaks Ximera
%\usepackage{multicol}
\usepackage{pgffor} %% required for integral for loops



%% http://tex.stackexchange.com/questions/66490/drawing-a-tikz-arc-specifying-the-center
%% Draws beach ball
\tikzset{pics/carc/.style args={#1:#2:#3}{code={\draw[pic actions] (#1:#3) arc(#1:#2:#3);}}}



\usepackage{array}
\setlength{\extrarowheight}{+.1cm}
\newdimen\digitwidth
\settowidth\digitwidth{9}
\def\divrule#1#2{
\noalign{\moveright#1\digitwidth
\vbox{\hrule width#2\digitwidth}}}






\DeclareMathOperator{\arccot}{arccot}
\DeclareMathOperator{\arcsec}{arcsec}
\DeclareMathOperator{\arccsc}{arccsc}

















%%This is to help with formatting on future title pages.
\newenvironment{sectionOutcomes}{}{}


\title{Analyzing}

\begin{document}

\begin{abstract}
describe everything
\end{abstract}
\maketitle







Completely analyze $T(r) = \tan(\sin(\pi r))$ \\




$\blacktriangleright$  \textbf{\textcolor{blue!55!black}{Domain: }}


We know that $-1 \leq \sin(\pi r) \leq 1$ and $[-1,1] \subset \left( -\frac{\pi}{2}, \frac{\pi}{2} \right)$ and $\tan(x)$ is continuous on $\left( -\frac{\pi}{2}, \frac{\pi}{2} \right)$.  Therefore $T(r)$ is defined for all $r$.  




$\blacktriangleright$  \textbf{\textcolor{blue!55!black}{Continuity: }}




Since both $\tan(x)$ and $\sin(x)$ are continuous, $T$ is the composition of two continuous functions. This also tells us that $T(r)$ is a continuous function.















$T(r)$ will be using $[-1,1]$ out of tangent's $\left( -\frac{\pi}{2}, \frac{\pi}{2} \right)$ waveform.



Graph of $y = T(r)$.

\begin{image}
\begin{tikzpicture} 
  \begin{axis}[
            domain=-5:5, ymax=10, xmax=5, ymin=-10, xmin=-5,
            xtick={-4.9, -1.7, 1.7, 4.9}, 
            xticklabels={$-\frac{3\pi}{2}$, $-\frac{\pi}{2}$, $\frac{\pi}{2}$, $\frac{3\pi}{2}$},
            axis lines =center,  xlabel={$r$}, ylabel=$y$,
            ticklabel style={font=\scriptsize},
            every axis y label/.style={at=(current axis.above origin),anchor=south},
            every axis x label/.style={at=(current axis.right of origin),anchor=west},
            axis on top
          ]
          
            \addplot [line width=2, penColor, smooth,samples=100,domain=(-1.47:1.47), <->] {tan(deg(x))};
            \addplot [line width=2, penColor2, smooth,samples=100,domain=(-1:1)] {tan(deg(x))};
            \addplot[color=penColor2,only marks,mark=*] coordinates{(-1,-1.557)}; 
            \addplot[color=penColor2,only marks,mark=*] coordinates{(1,1.557)}; 


      %\addplot[color=penColor,fill=penColor,only marks, mark size=1pt, mark=*] coordinates{(-9,5) (-8,5) (-7,5) (7,5) (8,5) (9,5)};



            \addplot [line width=1, gray, dashed,samples=100,domain=(-10:10), <->] ({-1.57},{x});
            \addplot [line width=1, gray, dashed,samples=100,domain=(-10:10), <->] ({1.57},{x});


           

  \end{axis}
\end{tikzpicture}
\end{image}



$[-1,1]$ is the output of $\sin(\pi r)$, which then becomes the input to tangent.  




For this to happen, the input to sine must be $\left[-\frac{\pi}{2}, \frac{\pi}{2}\right]$, which means that $r \in \left[-\frac{1}{2}, \frac{1}{2}\right]$.


But that is only half the story. \\


As $r$ moves from $-\frac{1}{2}$ to $\frac{1}{2}$, $\sin(\pi r)$ moves from $-1$ to $1$.  As $r$ continues, it moves from $\frac{1}{2}$ to $\frac{3}{2}$ and $\sin(\pi r)$ moves from $1$ back down to $-1$.  Then the whole thing repeats.

Now we have a full period. \\



That gives a full wave over the interval $\left[-\frac{1}{2}, \frac{3}{2}\right]$ and a period of $2$.



As $r$ increases, the sine wave will oscillate between $-1$ and $1$ and this piece of the tangent graph will also oscillate.  






\begin{image}
\begin{tikzpicture} 
  \begin{axis}[
            domain=-5:5, ymax=2, xmax=5, ymin=-2, xmin=-5,
            xtick={-4.9, -1.7, 1.7, 4.9}, 
            xticklabels={$-\frac{3\pi}{2}$, $-\frac{\pi}{2}$, $\frac{\pi}{2}$, $\frac{3\pi}{2}$},
            axis lines =center,  xlabel={$r$}, ylabel=$y$,
            ticklabel style={font=\scriptsize},
            every axis y label/.style={at=(current axis.above origin),anchor=south},
            every axis x label/.style={at=(current axis.right of origin),anchor=west},
            axis on top
          ]
          
            \addplot [line width=2, penColor, smooth,samples=100,domain=(-4.9:4.9), <->] {tan(deg(sin(deg(3.1415*x)))};


  \end{axis}
\end{tikzpicture}
\end{image}




$\blacktriangleright$ \textbf{\textcolor{blue!55!black}{Maximums and Minimums: }}



The maximum of sine is $1$ and it occurs at $\frac{\pi}{2}$.\\

Tangent is an increasing function, so the maximum of $T(r)$ is also going to occur at $\frac{\pi}{2}$. However, in $T(r)$, the inside of sine is $\pi r$.\\


The maximum of the waveform occurs at $r = \frac{1}{2}$.


\[   \tan\left(\sin\left(\pi \cdot \frac{1}{2}\right)\right)  =   \tan\left(\sin\left(\frac{\pi}{2}\right)\right)  = \tan(1)  \approx 1.557  \]


The minimum of $-\tan(1) \approx -1.557$ occurs at $r = -\frac{1}{2}$.


That is just for one period. These maximums and minimums are periodic with a period of $2$.


We have also discovered that the critical numbers for $T$ are $\left\{ \frac{1}{2} + 2k \, | \, k \in \mathbb{Z} \right\}$ and $\left\{ -\frac{1}{2} + 2k \, | \, k \in \mathbb{Z} \right\}$



$\blacktriangleright$ \textbf{\textcolor{blue!55!black}{Rate-of-Change: }} 


$\tan(t)$ is an increasing function.  Therefore, $T(r) = \tan(\sin(\pi r))$ will increase and decreases with $\sin(\pi r)$, which we know everything about.




$T(r)$ increases on $\left[ -\frac{1}{2}, \frac{1}{2} \right]$ and decreases on $\left[\frac{1}{2}, \frac{3}{2} \right]$.



And, this agrees with our graph.








\begin{center}
\textbf{\textcolor{green!50!black}{ooooo=-=-=-=-=-=-=-=-=-=-=-=-=ooOoo=-=-=-=-=-=-=-=-=-=-=-=-=ooooo}} \\

more examples can be found by following this link\\ \link[More Examples of Function Analysis]{https://ximera.osu.edu/csccmathematics/precalculus2/precalculus2/functionAnalysis/examples/exampleList}

\end{center}





\end{document}
