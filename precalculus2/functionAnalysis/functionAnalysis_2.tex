\documentclass{ximera}


\graphicspath{
  {./}
  {ximeraTutorial/}
  {basicPhilosophy/}
}

\newcommand{\mooculus}{\textsf{\textbf{MOOC}\textnormal{\textsf{ULUS}}}}

\usepackage{tkz-euclide}\usepackage{tikz}
\usepackage{tikz-cd}
\usetikzlibrary{arrows}
\tikzset{>=stealth,commutative diagrams/.cd,
  arrow style=tikz,diagrams={>=stealth}} %% cool arrow head
\tikzset{shorten <>/.style={ shorten >=#1, shorten <=#1 } } %% allows shorter vectors

\usetikzlibrary{backgrounds} %% for boxes around graphs
\usetikzlibrary{shapes,positioning}  %% Clouds and stars
\usetikzlibrary{matrix} %% for matrix
\usepgfplotslibrary{polar} %% for polar plots
\usepgfplotslibrary{fillbetween} %% to shade area between curves in TikZ
\usetkzobj{all}
\usepackage[makeroom]{cancel} %% for strike outs
%\usepackage{mathtools} %% for pretty underbrace % Breaks Ximera
%\usepackage{multicol}
\usepackage{pgffor} %% required for integral for loops



%% http://tex.stackexchange.com/questions/66490/drawing-a-tikz-arc-specifying-the-center
%% Draws beach ball
\tikzset{pics/carc/.style args={#1:#2:#3}{code={\draw[pic actions] (#1:#3) arc(#1:#2:#3);}}}



\usepackage{array}
\setlength{\extrarowheight}{+.1cm}
\newdimen\digitwidth
\settowidth\digitwidth{9}
\def\divrule#1#2{
\noalign{\moveright#1\digitwidth
\vbox{\hrule width#2\digitwidth}}}






\DeclareMathOperator{\arccot}{arccot}
\DeclareMathOperator{\arcsec}{arcsec}
\DeclareMathOperator{\arccsc}{arccsc}

















%%This is to help with formatting on future title pages.
\newenvironment{sectionOutcomes}{}{}


\title{Analyzing}

\begin{document}

\begin{abstract}
describe everything
\end{abstract}
\maketitle







Completely analyze $A(\theta) = \sin(\theta) - \sin(2\theta)$

$\blacktriangleright$  The implied domain is all real numbers.

$\blacktriangleright$  There are no discontinuities or singularities.

$\blacktriangleright$ The function is continuous on the whole real line.


$\blacktriangleright$ The function is periodic with a period of $2\pi$.  Therefore, we can just examine on wave.  Let's examine $[0,2\pi)$.






$\blacktriangleright$ To locate zeros, we will need to factor the formula.  




\begin{align*}
A(\theta)   &  = \sin(\theta) - \sin(2\theta)  \\
A(\theta)   &  = \sin(\theta) - 2\sin(\theta)\cos(\theta)   \\
A(\theta)   &  = \sin(\theta) (1 - 2\cos(\theta))   \\
\end{align*}



Either $sin(\theta) = 0$, which happens when $\theta = k\pi$ with $k \in \mathbb{N}$.
Or, $1 - 2\cos(\theta) = 0$, which happens when $\theta = \frac{\pi}{3} \pm 2\pi$ or when $\theta = \frac{5\pi}{3} \pm 2\pi$ with $k \in \mathbb{N}$.

In our sample period, we have zeros at $0$ and $\pi$. These will get solid dots on the  graph.  There is a zero at $2\pi$, but this is not in our sample period.  However, we should include a solid dot, because that makes the description clearer.  We also have zeros at $\frac{\pi}{3}$and $\frac{5\pi}{3}$.  These will get solid dots on the graph.








Our graph of $y=A(\theta))$ is piecing together.









\begin{image}
\begin{tikzpicture}
  \begin{axis}[
            domain=-1:8, ymax=10, xmax=8, ymin=-10, xmin=-1,
            axis lines =center, xlabel={$t$}, ylabel={$y = G(t)$}, grid = major, grid style={dashed},
            ytick={-10,-8,-6,-4,-2,2,4,6,8,10},
            xtick={-7.85, -6.28, -4.71, -3.14, -1.57, 0, 1.57, 3.142, 4.71, 6.28, 7.85},
            xticklabels={$\tfrac{-5\pi}{2}$,$-2\pi$,$\tfrac{-3\pi}{2}$,$-\pi$, $\tfrac{-\pi}{2}$, $0$, $\tfrac{\pi}{2}$, $\pi$, $\tfrac{3\pi}{2}$, $2\pi$, $\tfrac{5\pi}{2}$},
            yticklabels={$-10$,$-8$,$-6$,$-4$,$-2$,$2$,$4$,$6$,$8$,$10$}, 
            ticklabel style={font=\scriptsize},
            every axis y label/.style={at=(current axis.above origin),anchor=south},
            every axis x label/.style={at=(current axis.right of origin),anchor=west},
            axis on top
          ]
          



            \addplot[color=penColor,fill=penColor,only marks,mark=*] coordinates{(0,0)};
            \addplot[color=penColor,fill=penColor,only marks,mark=*] coordinates{(1.047,0)};
            \addplot[color=penColor,fill=penColor,only marks,mark=*] coordinates{(5.236,0)};
            \addplot[color=penColor,fill=penColor,only marks,mark=*] coordinates{6.28,0)};


            \addplot [line width=2, penColor, smooth,samples=300,domain=(0:2.9),->] {sin(x)-sin(2*x)};





  \end{axis}
\end{tikzpicture}
\end{image}







Hmmm, maybe we need a different period to examine.  Let's take a look at $(\pi, \pi)$.






\begin{image}
\begin{tikzpicture}
  \begin{axis}[
            domain=-4:4, ymax=10, xmax=4, ymin=-10, xmin=-4,
            axis lines =center, xlabel={$t$}, ylabel={$y = G(t)$}, grid = major, grid style={dashed},
            ytick={-10,-8,-6,-4,-2,2,4,6,8,10},
            xtick={-7.85, -6.28, -4.71, -3.14, -1.57, 0, 1.57, 3.142, 4.71, 6.28, 7.85},
            xticklabels={$\tfrac{-5\pi}{2}$,$-2\pi$,$\tfrac{-3\pi}{2}$,$-\pi$, $\tfrac{-\pi}{2}$, $0$, $\tfrac{\pi}{2}$, $\pi$, $\tfrac{3\pi}{2}$, $2\pi$, $\tfrac{5\pi}{2}$},
            yticklabels={$-10$,$-8$,$-6$,$-4$,$-2$,$2$,$4$,$6$,$8$,$10$}, 
            ticklabel style={font=\scriptsize},
            every axis y label/.style={at=(current axis.above origin),anchor=south},
            every axis x label/.style={at=(current axis.right of origin),anchor=west},
            axis on top
          ]
          

			\addplot [line width=1, gray, dashed,samples=300,domain=(-10:10),<->] ({3.14},{x});
			\addplot [line width=1, gray, dashed,samples=300,domain=(-10:10),<->] ({-3.14},{x});

			\addplot[color=penColor,fill=penColor,only marks,mark=*] coordinates{(0,0)};


            \addplot [line width=2, penColor, smooth,samples=300,domain=(-2.9:2.9),<->] {sin(deg(x))/(1 + cos(deg(x)))};





  \end{axis}
\end{tikzpicture}
\end{image}

That gives a better idea of what is going on.  Remember, this is just one period.



$\blacktriangleright$ $G$ has no discontinuities.  It is a continuous function.


$\blacktriangleright$ $G$ is increasing in between the singularities, $((2k-1)\pi, (2k+1)\pi)$, where $k$ is any integer.

$\blacktriangleright$ $G$ has no global or local maximums or minimums.






























\end{document}
