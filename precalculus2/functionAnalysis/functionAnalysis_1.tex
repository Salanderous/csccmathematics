\documentclass{ximera}


\graphicspath{
  {./}
  {ximeraTutorial/}
  {basicPhilosophy/}
}

\newcommand{\mooculus}{\textsf{\textbf{MOOC}\textnormal{\textsf{ULUS}}}}

\usepackage{tkz-euclide}\usepackage{tikz}
\usepackage{tikz-cd}
\usetikzlibrary{arrows}
\tikzset{>=stealth,commutative diagrams/.cd,
  arrow style=tikz,diagrams={>=stealth}} %% cool arrow head
\tikzset{shorten <>/.style={ shorten >=#1, shorten <=#1 } } %% allows shorter vectors

\usetikzlibrary{backgrounds} %% for boxes around graphs
\usetikzlibrary{shapes,positioning}  %% Clouds and stars
\usetikzlibrary{matrix} %% for matrix
\usepgfplotslibrary{polar} %% for polar plots
\usepgfplotslibrary{fillbetween} %% to shade area between curves in TikZ
\usetkzobj{all}
\usepackage[makeroom]{cancel} %% for strike outs
%\usepackage{mathtools} %% for pretty underbrace % Breaks Ximera
%\usepackage{multicol}
\usepackage{pgffor} %% required for integral for loops



%% http://tex.stackexchange.com/questions/66490/drawing-a-tikz-arc-specifying-the-center
%% Draws beach ball
\tikzset{pics/carc/.style args={#1:#2:#3}{code={\draw[pic actions] (#1:#3) arc(#1:#2:#3);}}}



\usepackage{array}
\setlength{\extrarowheight}{+.1cm}
\newdimen\digitwidth
\settowidth\digitwidth{9}
\def\divrule#1#2{
\noalign{\moveright#1\digitwidth
\vbox{\hrule width#2\digitwidth}}}






\DeclareMathOperator{\arccot}{arccot}
\DeclareMathOperator{\arcsec}{arcsec}
\DeclareMathOperator{\arccsc}{arccsc}

















%%This is to help with formatting on future title pages.
\newenvironment{sectionOutcomes}{}{}


\title{Analyzing}

\begin{document}

\begin{abstract}
describe everything
\end{abstract}
\maketitle







Completely analyze $G(t) = \frac{\sin(t)}{1 + \cos(t)}$

$\blacktriangleright$  The implied domain is all real numbers except those that make the denominator equal to $0$.



\begin{align*}
1  + \cos(t) & = 0 \\
\cos(t) & = -1
\end{align*}


We need to exclude all odd $\pi$.  The domain is $\{  r \in \mathbb{R} \, | \, r \ne (2k+1)\pi \text{ where } k \in \mathbb{Z}   \}$



\textbf{Note:} $G(t+2\pi) = G(t)$.  $G$ is a periodic function with a period of $2\pi$.  Therefore, we only need to analyze one period.  Let's investigate $[0, 2\pi)$.

\begin{itemize}
\item $G(0) = 0$   Closed point on the graph
\item $G(2\pi) = 0$  Open point on the graph.
\end{itemize}




What happens near $\pi$?  Numbers near $\pi$ make the denominator near $0$, but they also make the numerator near $0$, since $\sin(\pi)=0$.

It might be heplful to view the function with an equivalent formula.


\[   G(t) = \frac{\sin(t)}{1 + \cos(t)}  = \frac{\sin(t)}{1 + \cos(t)}  \cdot 1 = \frac{\sin(t)}{1 + \cos(t)}  \cdot \frac{1-\cos(t)}{1-\cos(t)} = \frac{\sin(t) (1-\cos(t))}{1-\cos^2(t)}    \]



\[    G(t) = \frac{1-\cos(t)}{\sin(t)}  \]


Near $\pi$, $1-\cos(t)$ is near $2$ and $\sin(t)$ is near $0$.  Therefore, $G(t)$ is unbounded near $\pi$.  The graph will have a vertical asymptote at $\pi$.


\begin{itemize}
\item $1-\cos(t)$ is near $2$, therefore, the numerator is positive when $t$ is near $\pi$.
\item the denominator is $\sin(t)$. It is near $0$, when $t$ is near $\pi$. But, it is positive on one side and negative on the other side.
\item therefore, the whole fraction is unbounded as $t$ approaches $\pi$.  It is positive on the left and negative on the right.
\end{itemize}




On the left side of $\pi$, $\sin(t) > 0$, which makes $G(t) > 0$.   $\lim\limits_{t \to \pi^{-}}G(t) = \infty$ 

On the right side of $\pi$, $\sin(t) < 0$, which makes $G(t) < 0$.   $\lim\limits_{t \to \pi^{+}}G(t) = -\infty$ 

Therefore, there is no global maximum or global minimum.

The graph contains the vertical asymptote $t=\pi$.



$\blacktriangleright$ $G$ has a zero at $0$.  

Remember, we are only examining one period.  $G$ has an infinite number of zeros at all even-$\pi$.





$\blacktriangleright$ Every odd-$\pi$ is a singularity of $G$.




Our graph of $y=G(t)$ is piecing together.













\begin{image}
\begin{tikzpicture}
  \begin{axis}[
            domain=-1:8, ymax=10, xmax=8, ymin=-10, xmin=-1,
            axis lines =center, xlabel={$t$}, ylabel={$y = G(t)$}, grid = major, grid style={dashed},
            ytick={-10,-8,-6,-4,-2,2,4,6,8,10},
            xtick={-7.85, -6.28, -4.71, -3.14, -1.57, 0, 1.57, 3.142, 4.71, 6.28, 7.85},
            xticklabels={$\tfrac{-5\pi}{2}$,$-2\pi$,$\tfrac{-3\pi}{2}$,$-\pi$, $\tfrac{-\pi}{2}$, $0$, $\tfrac{\pi}{2}$, $\pi$, $\tfrac{3\pi}{2}$, $2\pi$, $\tfrac{5\pi}{2}$},
            yticklabels={$-10$,$-8$,$-6$,$-4$,$-2$,$2$,$4$,$6$,$8$,$10$}, 
            ticklabel style={font=\scriptsize},
            every axis y label/.style={at=(current axis.above origin),anchor=south},
            every axis x label/.style={at=(current axis.right of origin),anchor=west},
            axis on top
          ]
          

      \addplot [line width=1, gray, dashed,samples=300,domain=(-10:10),<->] ({3.14},{x});

      \addplot[color=penColor,fill=penColor,only marks,mark=*] coordinates{(0,0)};
      \addplot[color=penColor,fill=white,only marks,mark=*] coordinates{(6.28,0)};


            \addplot [line width=2, penColor, smooth,samples=300,domain=(0:0.7),->] {sin(deg(x))/(1 + cos(deg(x)))};
            \addplot [line width=2, penColor, smooth,samples=300,domain=(2.8:2.9),<-] {sin(deg(x))/(1 + cos(deg(x)))};
            \addplot [line width=2, penColor, smooth,samples=300,domain=(3.4:3.5),->] {sin(deg(x))/(1 + cos(deg(x)))};
            \addplot [line width=2, penColor, smooth,samples=300,domain=(4.5:6.28),<-] {sin(deg(x))/(1 + cos(deg(x)))};




  \end{axis}
\end{tikzpicture}
\end{image}






From this we can sketch in a graph.








\begin{image}
\begin{tikzpicture}
  \begin{axis}[
            domain=-1:8, ymax=10, xmax=8, ymin=-10, xmin=-1,
            axis lines =center, xlabel={$t$}, ylabel={$y = G(t)$}, grid = major, grid style={dashed},
            ytick={-10,-8,-6,-4,-2,2,4,6,8,10},
            xtick={-7.85, -6.28, -4.71, -3.14, -1.57, 0, 1.57, 3.142, 4.71, 6.28, 7.85},
            xticklabels={$\tfrac{-5\pi}{2}$,$-2\pi$,$\tfrac{-3\pi}{2}$,$-\pi$, $\tfrac{-\pi}{2}$, $0$, $\tfrac{\pi}{2}$, $\pi$, $\tfrac{3\pi}{2}$, $2\pi$, $\tfrac{5\pi}{2}$},
            yticklabels={$-10$,$-8$,$-6$,$-4$,$-2$,$2$,$4$,$6$,$8$,$10$}, 
            ticklabel style={font=\scriptsize},
            every axis y label/.style={at=(current axis.above origin),anchor=south},
            every axis x label/.style={at=(current axis.right of origin),anchor=west},
            axis on top
          ]
          

			\addplot [line width=1, gray, dashed,samples=300,domain=(-10:10),<->] ({3.14},{x});

			\addplot[color=penColor,fill=penColor,only marks,mark=*] coordinates{(0,0)};
			\addplot[color=penColor,fill=white,only marks,mark=*] coordinates{(6.28,0)};


            \addplot [line width=2, penColor, smooth,samples=300,domain=(0:2.9),->] {sin(deg(x))/(1 + cos(deg(x)))};
            \addplot [line width=2, penColor, smooth,samples=300,domain=(3.4:6.28),<-] {sin(deg(x))/(1 + cos(deg(x)))};




  \end{axis}
\end{tikzpicture}
\end{image}







Hmmm, maybe we need a different period to examine.  Let's take a look at $(-\pi, \pi)$.






\begin{image}
\begin{tikzpicture}
  \begin{axis}[
            domain=-4:4, ymax=10, xmax=4, ymin=-10, xmin=-4,
            axis lines =center, xlabel={$t$}, ylabel={$y = G(t)$}, grid = major, grid style={dashed},
            ytick={-10,-8,-6,-4,-2,2,4,6,8,10},
            xtick={-7.85, -6.28, -4.71, -3.14, -1.57, 0, 1.57, 3.142, 4.71, 6.28, 7.85},
            xticklabels={$\tfrac{-5\pi}{2}$,$-2\pi$,$\tfrac{-3\pi}{2}$,$-\pi$, $\tfrac{-\pi}{2}$, $0$, $\tfrac{\pi}{2}$, $\pi$, $\tfrac{3\pi}{2}$, $2\pi$, $\tfrac{5\pi}{2}$},
            yticklabels={$-10$,$-8$,$-6$,$-4$,$-2$,$2$,$4$,$6$,$8$,$10$}, 
            ticklabel style={font=\scriptsize},
            every axis y label/.style={at=(current axis.above origin),anchor=south},
            every axis x label/.style={at=(current axis.right of origin),anchor=west},
            axis on top
          ]
          

			\addplot [line width=1, gray, dashed,samples=300,domain=(-10:10),<->] ({3.14},{x});
			\addplot [line width=1, gray, dashed,samples=300,domain=(-10:10),<->] ({-3.14},{x});

			\addplot[color=penColor,fill=penColor,only marks,mark=*] coordinates{(0,0)};


            \addplot [line width=2, penColor, smooth,samples=300,domain=(-2.9:2.9),<->] {sin(deg(x))/(1 + cos(deg(x)))};





  \end{axis}
\end{tikzpicture}
\end{image}

That gives a better idea of what is going on.  Remember, this is just one period.



$\blacktriangleright$ $G$ has no discontinuities.  It is a continuous function.


$\blacktriangleright$ $G$ is increasing in between the singularities, $((2k-1)\pi, (2k+1)\pi)$, where $k$ is any integer.

$\blacktriangleright$ $G$ has no global or local maximums or minimums.






























\end{document}
