\documentclass{ximera}


\graphicspath{
  {./}
  {ximeraTutorial/}
  {basicPhilosophy/}
}

\newcommand{\mooculus}{\textsf{\textbf{MOOC}\textnormal{\textsf{ULUS}}}}

\usepackage{tkz-euclide}\usepackage{tikz}
\usepackage{tikz-cd}
\usetikzlibrary{arrows}
\tikzset{>=stealth,commutative diagrams/.cd,
  arrow style=tikz,diagrams={>=stealth}} %% cool arrow head
\tikzset{shorten <>/.style={ shorten >=#1, shorten <=#1 } } %% allows shorter vectors

\usetikzlibrary{backgrounds} %% for boxes around graphs
\usetikzlibrary{shapes,positioning}  %% Clouds and stars
\usetikzlibrary{matrix} %% for matrix
\usepgfplotslibrary{polar} %% for polar plots
\usepgfplotslibrary{fillbetween} %% to shade area between curves in TikZ
\usetkzobj{all}
\usepackage[makeroom]{cancel} %% for strike outs
%\usepackage{mathtools} %% for pretty underbrace % Breaks Ximera
%\usepackage{multicol}
\usepackage{pgffor} %% required for integral for loops



%% http://tex.stackexchange.com/questions/66490/drawing-a-tikz-arc-specifying-the-center
%% Draws beach ball
\tikzset{pics/carc/.style args={#1:#2:#3}{code={\draw[pic actions] (#1:#3) arc(#1:#2:#3);}}}



\usepackage{array}
\setlength{\extrarowheight}{+.1cm}
\newdimen\digitwidth
\settowidth\digitwidth{9}
\def\divrule#1#2{
\noalign{\moveright#1\digitwidth
\vbox{\hrule width#2\digitwidth}}}






\DeclareMathOperator{\arccot}{arccot}
\DeclareMathOperator{\arcsec}{arcsec}
\DeclareMathOperator{\arccsc}{arccsc}

















%%This is to help with formatting on future title pages.
\newenvironment{sectionOutcomes}{}{}


\title{Sine and Cosine}

\begin{document}

\begin{abstract}
derivatives
\end{abstract}
\maketitle








In the graph below, 

\begin{itemize}
\item the measurement for the angle $\theta$ is plotted horizontally
\item $\sin(\theta)$ is the blue curve
\item the line is a tangent line to $\sin(\theta)$
\item the slope of the tangent line is a number and is plotted as the red dot
\end{itemize}

As you move long the $\sin(\theta)$ curve, the slope of the tangent line follows the dotted $\cos(\theta)$ curve.



\begin{center}
\desmos{ysuoxzmvfq}{400}{300}
\end{center}



The values of $\cos(\theta)$ measure the instantaneous rate of change of $\sin(\theta)$.



\begin{theorem}  \textbf{\textcolor{green!50!black}{Derivative of $\sin(\theta)$}}

\begin{center}
$\cos(\theta)$ is the derivative of $\sin(\theta)$.  
\end{center}


\[
\frac{d}{d\theta} \sin(\theta) = \cos(\theta)
\]

\end{theorem}















In the graph below, 

\begin{itemize}
\item the measurement for the angle $\theta$ is plotted horizontally
\item $\cos(\theta)$ is the blue curve
\item the line is a tangent line to $\cos(\theta)$
\item the slope of the tangent line is a number and is plotted as the red dot
\end{itemize}

As you move long the $\cos(\theta)$ curve, the slope of the tangent line follows the dotted $-\sin(\theta)$ curve.



\begin{center}
\desmos{mushid6bj8}{400}{300}
\end{center}



The values of $-\sin(\theta)$ measure the instantaneous rate of change of $\cos(\theta)$.



\begin{theorem}  \textbf{\textcolor{green!50!black}{Derivative of $\cos(\theta)$}}

\begin{center}
$-\sin(\theta)$ is the derivative of $\cos(\theta)$. 
\end{center}


\[
\frac{d}{d\theta} \cos(\theta) = -\sin(\theta)
\]


\end{theorem}







We use these two facts as our algebraic reasoning for the behavior of $\sin(\theta)$ and $\cos(\theta)$.




\begin{example}



$f(t) = \sin(t)$ is decreasing at $t=\frac{2\pi}{3}$, because $\cos\left( \frac{2\pi}{3} \right) < 0$ \\


$g(t) = \cos(t)$ is decreasing at $t=\frac{\pi}{4}$, because $-\sin\left( \frac{\pi}{4} \right) < 0$

\end{example}

















\subsection{Shifts}




If the graph $y = \sin(t)$ is vertically shifted to $y = \sin(t)+C$, then the shape of the graph does not change. The graph just rigidly moves up. Therefore, the slopes of the tangent lines at the shifted points are the same as for $y = \sin(t)$ .



\begin{observation} \textbf{\textcolor{purple!85!blue}{Vertical Shifts}}  \\

Shifting a graph vertically doesn't change the shape.  It just repositions the graph.  Therefore, all of the slopes of the tangents lines stay the same. \\



$\blacktriangleright$ The derivative of $\sin(t)+C$ is $\cos(t)$. \\

$\blacktriangleright$ The derivative of $\cos(t)+C$ is $-\sin(t)$. \\

\end{observation}


\begin{notation}

\[
\frac{d}{d\theta} \sin(\theta) + C = \cos(\theta)
\]

\[
\frac{d}{d\theta} \cos(\theta) + C = -\sin(\theta)
\]

\end{notation}


If the graph $y = \sin(t)$ is horizontally shifted to $y = \sin(t - B)$, then the shape of the graph does not change. The whole graph is just picked up and set down at a new location. Therefore, the slopes of the tangent lines at the shifted points don't change.



\begin{observation} \textbf{\textcolor{purple!85!blue}{Horizontal Shifts}} \\

$\blacktriangleright$ The derivative of $\sin(t - B)$ is $\cos(t - B)$. \\

$\blacktriangleright$ The derivative of $\cos(t - B)$ is $-\sin(t - B)$. \\

\end{observation}


\begin{notation}

\[
\frac{d}{d\theta} \sin(\theta - B) = \cos(\theta - B)
\]

\[
\frac{d}{d\theta} \cos(\theta - B) = -\sin(\theta - B)
\]

\end{notation}













\subsection{Stretching}


Suppose you have a line $L$ with slope $m$. \\

Let $(x_1, y_1)$ and $(x_2, y_2)$ be any two points on $L_1$. \\

Then, we have


\[
m = \frac{y_2 - y_1}{x_2 - x_1}
\]






Now, suppose we stretch the line vertically by a factor of $k$.  That means that all of the points change from $(a, b)$ to $(a, k \cdot b)$

Then, the new slope, $M$, is 



\[
M = \frac{k \cdot y_2 - k \cdot y_1}{x_2 - x_1} = \frac{k \cdot (y_2 - y_1)}{x_2 - x_1} = k \cdot \frac{y_2 - y_1}{x_2 - x_1} = k \cdot m
\]


If you stretch the graph of a funciton vertically by a factor of $k$, then you also stretch all of the tangent lines by the same factor.


\begin{observation} \textbf{\textcolor{purple!85!blue}{Vertical Stretching}}   \\

$\blacktriangleright$ The derivative of $k \cdot \sin(t)$ is $k \cdot \cos(t)$. \\

$\blacktriangleright$ The derivative of $k \cdot \cos(t)$ is $-k \cdot \sin(t)$. \\

\end{observation}




\begin{notation}

\[
\frac{d}{d\theta} k \, \sin(\theta) = k \, \cos(\theta)
\]

\[
\frac{d}{d\theta} k \, \cos(\theta) = -k \, \sin(\theta)
\]

\end{notation}








Let's put all of this together.







\begin{theorem} \textbf{\textcolor{green!50!black}{Derivatives}}  \\

$\blacktriangleright$ The derivative of \textbf{\textcolor{purple!85!blue}{$A \, \sin(t - B) + C$}} is \textbf{\textcolor{blue!55!black}{$A \, \cos(t - B)$}}. \\

$\blacktriangleright$ The derivative of \textbf{\textcolor{purple!85!blue}{$A \, \cos(t - B) + C$}} is \textbf{\textcolor{blue!55!black}{$-A \, \sin(t - B)$}}. \\

\end{theorem}






\begin{notation}

\[
\frac{d}{d\theta} A \, \sin(\theta - B) + C = A \, \cos(\theta - B)
\]

\[
\frac{d}{d\theta} A \, \cos(\theta - B) + C = -A \, \sin(\theta - B)
\]

\end{notation}










\begin{example}


What is the slope of the tangent line to the graph of $H(t) = -2 \sin\left( t - \frac{\pi}{6} \right) + 3$ at $\left( \frac{5 \pi}{6}, 3-\sqrt{3} \right)$?




\begin{image}
\begin{tikzpicture}
  \begin{axis}[
            domain=-10:10, ymax=6.5, xmax=10, ymin=-1.5, xmin=-10,
            axis lines =center, xlabel={$t$}, ylabel={$y = H(t)$}, grid = major, grid style={dashed},
            ytick={-1,1,2,3,4,5,6},
            xtick={-7.85, -6.28, -4.71, -3.14, -1.57, 0, 1.57, 3.142, 4.71, 6.28, 7.85},
            xticklabels={$-\tfrac{5\pi}{2}$,$-2\pi$,$-\tfrac{3\pi}{2}$,$-\pi$, $-\tfrac{\pi}{2}$, $0$, $\tfrac{\pi}{2}$, $\pi$, $\tfrac{3\pi}{2}$, $2\pi$, $\tfrac{5\pi}{2}$},
            yticklabels={$-1$,$1$,$2$,$3$,$4$,$5$,$6$}, 
            ticklabel style={font=\scriptsize},
            every axis y label/.style={at=(current axis.above origin),anchor=south},
            every axis x label/.style={at=(current axis.right of origin),anchor=west},
            axis on top
          ]
          

            \addplot [line width=2, penColor, smooth,samples=300,domain=(-10:10),<->] {-2 * sin(deg(x - 0.523)) + 3};
            %\addplot [line width=2, penColor2, smooth,samples=300,domain=(-10:10),<->] {cos(deg(x)};

			\addplot [line width=2, penColor2, smooth,samples=300,domain=(1.5:5),<->] { (x-2.618) + 1.268};

            \addplot [color=penColor,only marks,mark=*] coordinates{(2.618,1.2679)};



  \end{axis}
\end{tikzpicture}
\end{image}



\begin{explanation}



The derivative of $f(t) = -2 \sin\left( t - \frac{\pi}{6} \right) + 3$ is $f'(t) = -2 \cos\left( t - \frac{\pi}{6} \right)$.



\[ f'\left( \frac{5 \pi}{6} \right) = -2 \cos\left( \frac{5 \pi}{6} - \frac{\pi}{6} \right) = -2 \cos\left( \frac{2 \pi}{3} \right)  = -2 \cdot -\frac{1}{2} = 1\]


The slope of the tangent line is $1$.

\end{explanation}

\end{example}












\begin{example}


Let $G(x) = 3 \cos\left( x +\frac{\pi}{4} \right) - 1$.

Where does the graph of $y = G(x)$ have horizontal tangent lines?




\begin{image}
\begin{tikzpicture}
  \begin{axis}[
            domain=-10:10, ymax=4.5, xmax=10, ymin=-5.5, xmin=-10,
            axis lines =center, xlabel={$x$}, ylabel={$y = G(x)$}, grid = major, grid style={dashed},
            ytick={-3,-2,-1,1,2,3,4,5},
            xtick={-7.85, -6.28, -4.71, -3.14, -1.57, 0, 1.57, 3.142, 4.71, 6.28, 7.85},
            xticklabels={$-\tfrac{5\pi}{2}$,$-2\pi$,$-\tfrac{3\pi}{2}$,$-\pi$, $-\tfrac{\pi}{2}$, $0$, $\tfrac{\pi}{2}$, $\pi$, $\tfrac{3\pi}{2}$, $2\pi$, $\tfrac{5\pi}{2}$},
            yticklabels={$-3$,$-2$,$-1$,$1$,$2$,$3$,$4$,$5$,$6$}, 
            ticklabel style={font=\scriptsize},
            every axis y label/.style={at=(current axis.above origin),anchor=south},
            every axis x label/.style={at=(current axis.right of origin),anchor=west},
            axis on top
          ]
          

            \addplot [line width=2, penColor, smooth,samples=300,domain=(-10:10),<->] {3 * cos(deg(x + 0.785)) - 1};
            %\addplot [line width=2, penColor2, smooth,samples=300,domain=(-10:10),<->] {cos(deg(x)};

			%\addplot [line width=2, penColor2, smooth,samples=300,domain=(1.5:5),<->] { (x-2.618) + 1.268};

            %\addplot [color=penColor,only marks,mark=*] coordinates{(2.618,1.2679)};



  \end{axis}
\end{tikzpicture}
\end{image}



We have two approaches:

\begin{explanation} \textbf{\textcolor{blue!55!black}{Maximums and Minimums}}  \\

The horizontal tangent lines occur at points which correspond to maximums and minimums of cosine.  


$\blacktriangleright$ Maximums of cosine functions occur when the inside of the function equals $2 k \pi$ where $k \in \mathbb{Z}$.


\[
x + \frac{\pi}{4} = 2 k \pi
\]

\[
x = 2 k \pi - \frac{\pi}{4} = \frac{8 k \pi}{4} - \frac{\pi}{4} = \frac{(8k - 1)\pi}{4}
\]




$\blacktriangleright$ Minimums of cosine functions occur when the inside of the function equals $(2 k + 1) \pi$ where $k \in \mathbb{Z}$.


\[
x + \frac{\pi}{4} = (2 k + 1) \pi
\]

\[
x = (2 k + 1) \pi - \frac{\pi}{4} = \frac{4(2 k + 1) \pi}{4} - \frac{\pi}{4} = \frac{(8k + 3)\pi}{4}
\]


\end{explanation}








\begin{explanation} \textbf{\textcolor{blue!55!black}{Critical Numbers}} \\


The horizontal tangent lines occur at points which correspond to where the derivative equal $0$.  






The derivative of $G(x) = 3 \cos\left( x + \frac{\pi}{4} \right) - 1$ is $G'(x) = -3 \sin\left( x + \frac{\pi}{4} \right)$.


The sine function equals $0$ and $k \pi$ where $k \in \mathbb{Z}$.



\[
x + \frac{\pi}{4} =  k \pi
\]


\[
x  =    k \pi - \frac{\pi}{4} =  \frac{4 k \pi}{4} - \frac{\pi}{4} = \frac{(4k - 1)\pi}{4}
\]









\end{explanation}



Did these two methods give the same set of domain numbers? \\



\[
\frac{(8k - 1)\pi}{4} :  \cdots \frac{-17\pi}{4}, \frac{-9\pi}{4}, \frac{-\pi}{4}, \frac{7\pi}{4} \cdots
\]

\[
\frac{(8k + 3)\pi}{4} :  \cdots \frac{-13\pi}{4}, \frac{-5\pi}{4}, \frac{3\pi}{4}, \frac{11\pi}{4} \cdots
\]


\[
\frac{(4k - 1)\pi}{4} :  \cdots \frac{-9\pi}{4}, \frac{-5\pi}{4}, \frac{-\pi}{4}, \frac{3\pi}{4} \cdots
\]



Yep.  They are generating the same list.


\end{example}










\end{document}
