\documentclass{ximera}


\graphicspath{
  {./}
  {ximeraTutorial/}
  {basicPhilosophy/}
}

\newcommand{\mooculus}{\textsf{\textbf{MOOC}\textnormal{\textsf{ULUS}}}}

\usepackage{tkz-euclide}\usepackage{tikz}
\usepackage{tikz-cd}
\usetikzlibrary{arrows}
\tikzset{>=stealth,commutative diagrams/.cd,
  arrow style=tikz,diagrams={>=stealth}} %% cool arrow head
\tikzset{shorten <>/.style={ shorten >=#1, shorten <=#1 } } %% allows shorter vectors

\usetikzlibrary{backgrounds} %% for boxes around graphs
\usetikzlibrary{shapes,positioning}  %% Clouds and stars
\usetikzlibrary{matrix} %% for matrix
\usepgfplotslibrary{polar} %% for polar plots
\usepgfplotslibrary{fillbetween} %% to shade area between curves in TikZ
\usetkzobj{all}
\usepackage[makeroom]{cancel} %% for strike outs
%\usepackage{mathtools} %% for pretty underbrace % Breaks Ximera
%\usepackage{multicol}
\usepackage{pgffor} %% required for integral for loops



%% http://tex.stackexchange.com/questions/66490/drawing-a-tikz-arc-specifying-the-center
%% Draws beach ball
\tikzset{pics/carc/.style args={#1:#2:#3}{code={\draw[pic actions] (#1:#3) arc(#1:#2:#3);}}}



\usepackage{array}
\setlength{\extrarowheight}{+.1cm}
\newdimen\digitwidth
\settowidth\digitwidth{9}
\def\divrule#1#2{
\noalign{\moveright#1\digitwidth
\vbox{\hrule width#2\digitwidth}}}






\DeclareMathOperator{\arccot}{arccot}
\DeclareMathOperator{\arcsec}{arcsec}
\DeclareMathOperator{\arccsc}{arccsc}

















%%This is to help with formatting on future title pages.
\newenvironment{sectionOutcomes}{}{}


\title{Sine and Cosine}

\begin{document}

\begin{abstract}
derivatives
\end{abstract}
\maketitle








In the graph below, 

\begin{itemize}
\item the measurement for the angle $\theta$ is plotted horizontally
\item $\sin(\theta)$ is the blue curve
\item the line is a tangent line to $\sin(\theta)$
\item the slope of the tangent line is a number and is plotted as the red dot
\end{itemize}

As you move long the $\sin(\theta)$ curve, the slope of the tangent line follows the dotted $\cos(\theta)$ curve.



\begin{center}
\desmos{ysuoxzmvfq}{400}{300}
\end{center}



$\cos(\theta)$ is the derivative of $\sin(\theta)$.  The values of $\cos(\theta)$ measure the instantaneous rate of change of $\sin(\theta)$.


















In the graph below, 

\begin{itemize}
\item the measurement for the angle $\theta$ is plotted horizontally
\item $\cos(\theta)$ is the blue curve
\item the line is a tangent line to $\cos(\theta)$
\item the slope of the tangent line is a number and is plotted as the red dot
\end{itemize}

As you move long the $\cos(\theta)$ curve, the slope of the tangent line follows the dotted $-\sin(\theta)$ curve.



\begin{center}
\desmos{mushid6bj8}{400}{300}
\end{center}



$-\sin(\theta)$ is the derivative of $\cos(\theta)$.  Its values measure the instantaneous rate of change of $\cos(\theta)$.









This is our algebraic reasoning for the behavior of $\sin(\theta)$ and $\cos(\theta)$.






$f(t) = \sin(t)$ is decreasing at $t=\frac{2\pi}{3}$, because $\cos\left( \frac{2\pi}{3} \right) < 0$ \\


$g(t) = \cos(t)$ is decreasing at $t=\frac{\pi}{4}$, because $-\sin\left( \frac{\pi}{4} \right) < 0$



\subsection{Shifts}




If the graph $y = \sin(t)$ is vertically shifted to $y = \sin(t)+C$, then the shape of the graph does not change. Therefore, the slopes of the tangent lines at the shifted points don't change.



\begin{observation} Vertical Shifts \\

$\blacktriangleright$ The derivative of $\sin(t)+C$ is $\cos(t)$. \\

$\blacktriangleright$ The derivative of $\cos(t)+C$ is $-\sin(t)$. \\

\end{observation}




If the graph $y = \sin(t)$ is horizontally shifted to $y = \sin(t - B)$, then the shape of the graph does not change. Therefore, the slopes of the tangent lines at the shifted points don't change.



\begin{observation} Horizontal Shifts \\

$\blacktriangleright$ The derivative of $\sin(t - B)$ is $\cos(t - B)$. \\

$\blacktriangleright$ The derivative of $\cos(t - B)$ is $-\sin(t - B)$. \\

\end{observation}















\subsection{Stretching}


Suppose you have a line $L$ with slope $m$. \\

Let $(x_1, y_1)$ and $(x_2, y_2)$ be any two points on $L_1$. \\

Then, we have


\[
m = \frac{y_2 - y_1}{x_2 - x_1}
\]






Now, suppose we stretch the line vertically by a factor of $k$.  That means that all of the points change from $(a, b)$ to $(a, k \cdot b)$

Then, the new slope, $M$, is 



\[
M = \frac{k \cdot y_2 - k \cdot y_1}{x_2 - x_1} = \frac{k \cdot (y_2 - y_1)}{x_2 - x_1} = k \cdot \frac{y_2 - y_1}{x_2 - x_1} = k \cdot m
\]





\begin{observation} Vertical Stretch \\

$\blacktriangleright$ The derivative of $k \cdot \sin(t)$ is $k \cdot \cos(t)$. \\

$\blacktriangleright$ The derivative of $k \cdot \cos(t)$ is $-k \cdot \sin(t)$. \\

\end{observation}











\begin{theorem} \textbf{\textcolor{green!50!black}{Derivatives}}  \\

$\blacktriangleright$ The derivative of $A \, \sin(t - h) + k$ is $A \, \cos(t - h) + k$. \\

$\blacktriangleright$ The derivative of $A \, \cos(t - h) + k$ is $-A \, \sin(t - h) + k$. \\

\end{theorem}














\end{document}
