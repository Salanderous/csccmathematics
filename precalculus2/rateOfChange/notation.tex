\documentclass{ximera}


\graphicspath{
  {./}
  {ximeraTutorial/}
  {basicPhilosophy/}
}

\newcommand{\mooculus}{\textsf{\textbf{MOOC}\textnormal{\textsf{ULUS}}}}

\usepackage{tkz-euclide}\usepackage{tikz}
\usepackage{tikz-cd}
\usetikzlibrary{arrows}
\tikzset{>=stealth,commutative diagrams/.cd,
  arrow style=tikz,diagrams={>=stealth}} %% cool arrow head
\tikzset{shorten <>/.style={ shorten >=#1, shorten <=#1 } } %% allows shorter vectors

\usetikzlibrary{backgrounds} %% for boxes around graphs
\usetikzlibrary{shapes,positioning}  %% Clouds and stars
\usetikzlibrary{matrix} %% for matrix
\usepgfplotslibrary{polar} %% for polar plots
\usepgfplotslibrary{fillbetween} %% to shade area between curves in TikZ
\usetkzobj{all}
\usepackage[makeroom]{cancel} %% for strike outs
%\usepackage{mathtools} %% for pretty underbrace % Breaks Ximera
%\usepackage{multicol}
\usepackage{pgffor} %% required for integral for loops



%% http://tex.stackexchange.com/questions/66490/drawing-a-tikz-arc-specifying-the-center
%% Draws beach ball
\tikzset{pics/carc/.style args={#1:#2:#3}{code={\draw[pic actions] (#1:#3) arc(#1:#2:#3);}}}



\usepackage{array}
\setlength{\extrarowheight}{+.1cm}
\newdimen\digitwidth
\settowidth\digitwidth{9}
\def\divrule#1#2{
\noalign{\moveright#1\digitwidth
\vbox{\hrule width#2\digitwidth}}}






\DeclareMathOperator{\arccot}{arccot}
\DeclareMathOperator{\arcsec}{arcsec}
\DeclareMathOperator{\arccsc}{arccsc}

















%%This is to help with formatting on future title pages.
\newenvironment{sectionOutcomes}{}{}


\title{Infinitesimal}

\begin{document}

\begin{abstract}
notation
\end{abstract}
\maketitle




Calculus is the story about the very small.  We talk about quantities that are smaller than any specific number.  ``Infinitesimal'' is the mathematics word for these quantities. Since infinitesimals are smaller than any of our numbers, we can't describe them with our numbers.  \\

The language of \textbf{limits} is our way of discussing the concepts of infinitesimals.  


For rates of change, the infinitesimal story involves instantaneous rate of change.  Our tool for measuring this type of change is called the \textbf{derivative}.

We have a couple of ways of representing the derivative of a function. \\


$\blacktriangleright$  \textbf{\textcolor{purple!85!blue}{Prime Notation}}


Suppose $f$ is the name of a function.

The derivative of $f$ is a new function and it is denoted as \textbf{\textcolor{purple!85!blue}{$f'$}}.

The little ``tick mark'' in the exponent position is called the \textbf{\textcolor{purple!85!blue}{prime sign}}.


$f'$ is pronounced as ``f prime''.


The derivative measures the change in the function values compared to changes in the domain values - at the infinitesimal level. This seems straightforward and for most situations it is.  However, we often encounter situations where our function representation involves parameters, like


\[
f = A \sin(t - B) + C
\]

Which symbol is representing the domain?


\[
f(A)   \, \text{ or } \, f(t)   \, \text{ or } \, f(B)   \, \text{ or } \, f(C)   
\]


Or, more to the point, funcitons of more than one variable:


\[
f(x, y) = \sin(2x - 3y)
\]

To which variable is $f'$ referring? \\






It can be confusing at times.  Therefore, we have an alternative notation for derivatives to clear up the confusion



$\blacktriangleright$  \textbf{\textcolor{purple!85!blue}{Leibniz Notation}}


\textbf{Gottfried Leibniz} was a German mathematician who was investigating the concepts of Calculus along with Newton.  While Newton developed his ``dot'' notation for derivative, Leibniz invented a more useful notation.


The derivative of $f$ with respect to $x$ is represented as $\frac{df}{dx}$. \\

$\frac{df}{dx}$ is pronounced as ``dfdx''.  It looks like a fraction, but it isn't.


``with respect to $x$'' means $x$ is representing the domain for measurment purposes. 



\begin{notation}


Just like with other mathematical notation, we allow useful modifications to this notation.



As the symbol $\frac{df}{dx}$ shows, the function should be on top next to the $d$.    We should write $\frac{d sin(x)}{dx}$.  However, if the formula for the function gets long, then it becomes difficult to read the notation:

\[
\frac{d (sin(3x^2+1)ln(cos(5x+7))e^{tan(4x)})}{dx}
\]


Therefore, it is sometimes clearer if the function is place to the right of the differentiation notation:




\[
\frac{d}{dx} \, (sin(3x^2+1)ln(cos(5x+7))e^{tan(4x)})
\]






\end{notation}














\end{document}
