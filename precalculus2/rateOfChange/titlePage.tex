\documentclass{ximera}


\graphicspath{
  {./}
  {ximeraTutorial/}
  {basicPhilosophy/}
}

\newcommand{\mooculus}{\textsf{\textbf{MOOC}\textnormal{\textsf{ULUS}}}}

\usepackage{tkz-euclide}\usepackage{tikz}
\usepackage{tikz-cd}
\usetikzlibrary{arrows}
\tikzset{>=stealth,commutative diagrams/.cd,
  arrow style=tikz,diagrams={>=stealth}} %% cool arrow head
\tikzset{shorten <>/.style={ shorten >=#1, shorten <=#1 } } %% allows shorter vectors

\usetikzlibrary{backgrounds} %% for boxes around graphs
\usetikzlibrary{shapes,positioning}  %% Clouds and stars
\usetikzlibrary{matrix} %% for matrix
\usepgfplotslibrary{polar} %% for polar plots
\usepgfplotslibrary{fillbetween} %% to shade area between curves in TikZ
\usetkzobj{all}
\usepackage[makeroom]{cancel} %% for strike outs
%\usepackage{mathtools} %% for pretty underbrace % Breaks Ximera
%\usepackage{multicol}
\usepackage{pgffor} %% required for integral for loops



%% http://tex.stackexchange.com/questions/66490/drawing-a-tikz-arc-specifying-the-center
%% Draws beach ball
\tikzset{pics/carc/.style args={#1:#2:#3}{code={\draw[pic actions] (#1:#3) arc(#1:#2:#3);}}}



\usepackage{array}
\setlength{\extrarowheight}{+.1cm}
\newdimen\digitwidth
\settowidth\digitwidth{9}
\def\divrule#1#2{
\noalign{\moveright#1\digitwidth
\vbox{\hrule width#2\digitwidth}}}






\DeclareMathOperator{\arccot}{arccot}
\DeclareMathOperator{\arcsec}{arcsec}
\DeclareMathOperator{\arccsc}{arccsc}

















%%This is to help with formatting on future title pages.
\newenvironment{sectionOutcomes}{}{}


\title{Rate Of Change}

\begin{document}

\begin{abstract}
%
\end{abstract}
\maketitle




The sine and cosine functions are functions of the central angle of the unit circle.  As the measurement of the angle changes, the value of sine changes and the value of cosine changes.  


As the angle rotates counterclockwise around the origin, the value of sine sometimes increases and sometimes decreases.  We can compare the changes in $\theta$ with the changes in the value of $\sin(\theta)$ in the form of a rate. 

We can visualize this rate of change as the slope of the tangent lines to the graph of $\sin(\theta)$.

If we collect these slopes of the tangent lines and plot them on the Cartesian plane, they form a pattern.


The slopes of the tangent line create a function called the instantaneous rate of change (i.e. the derivative) of $\sin(\theta)$.

It turns out this derivative of $\sin(\theta)$ is $\cos(\theta)$.












\subsection{Learning Outcomes}

\begin{sectionOutcomes}
In this section, students will 

\begin{itemize}
\item compare rates of change of sine and cosine.
\end{itemize}
\end{sectionOutcomes}








\begin{center}
\textbf{\textcolor{green!50!black}{ooooo=-=-=-=-=-=-=-=-=-=-=-=-=ooOoo=-=-=-=-=-=-=-=-=-=-=-=-=ooooo}} \\

more examples can be found by following this link\\ \link[More Examples of Rates of Change]{https://ximera.osu.edu/csccmathematics/precalculus2/precalculus2/rateOfChange/examples/exampleList}

\end{center}



\end{document}
