\documentclass{ximera}


\graphicspath{
  {./}
  {ximeraTutorial/}
  {basicPhilosophy/}
}

\newcommand{\mooculus}{\textsf{\textbf{MOOC}\textnormal{\textsf{ULUS}}}}

\usepackage{tkz-euclide}\usepackage{tikz}
\usepackage{tikz-cd}
\usetikzlibrary{arrows}
\tikzset{>=stealth,commutative diagrams/.cd,
  arrow style=tikz,diagrams={>=stealth}} %% cool arrow head
\tikzset{shorten <>/.style={ shorten >=#1, shorten <=#1 } } %% allows shorter vectors

\usetikzlibrary{backgrounds} %% for boxes around graphs
\usetikzlibrary{shapes,positioning}  %% Clouds and stars
\usetikzlibrary{matrix} %% for matrix
\usepgfplotslibrary{polar} %% for polar plots
\usepgfplotslibrary{fillbetween} %% to shade area between curves in TikZ
\usetkzobj{all}
\usepackage[makeroom]{cancel} %% for strike outs
%\usepackage{mathtools} %% for pretty underbrace % Breaks Ximera
%\usepackage{multicol}
\usepackage{pgffor} %% required for integral for loops



%% http://tex.stackexchange.com/questions/66490/drawing-a-tikz-arc-specifying-the-center
%% Draws beach ball
\tikzset{pics/carc/.style args={#1:#2:#3}{code={\draw[pic actions] (#1:#3) arc(#1:#2:#3);}}}



\usepackage{array}
\setlength{\extrarowheight}{+.1cm}
\newdimen\digitwidth
\settowidth\digitwidth{9}
\def\divrule#1#2{
\noalign{\moveright#1\digitwidth
\vbox{\hrule width#2\digitwidth}}}






\DeclareMathOperator{\arccot}{arccot}
\DeclareMathOperator{\arcsec}{arcsec}
\DeclareMathOperator{\arccsc}{arccsc}

















%%This is to help with formatting on future title pages.
\newenvironment{sectionOutcomes}{}{}


\title{Transform}

\begin{document}

\begin{abstract}
graphs
\end{abstract}
\maketitle



When two sine functions have the same period, then their graphs (or waveforms) have the same distance between the hills and valleys.   The graphs of the two sine functions could be in synch, which means the tops of the hills and bottoms of valleys occur at the exact same angle.  In this case, we say the waveforms are \textbf{in phase}.


If one of the waveforms experiences a horizontal shift, then we say that the waveforms are \textbf{out of phase}.

If the shift makes the hills of one waveform align perfectly with the valleys of the other waveform, then we say that the two waveforms are \textbf{completely out of phase}.




\begin{example}  Sine



Analyze $s(\theta) = 2 \sin\left(\theta - \frac{\pi}{4}\right) + 2$ \\





\begin{image}
\begin{tikzpicture}
  \begin{axis}[
            domain=-10:10, ymax=4.5, xmax=10, ymin=-4, xmin=-10,
            axis lines =center, xlabel={$\theta$}, ylabel=$y$, grid = major, grid style={dashed},
            ytick={-3,-2,-1,1,2,3,4},
            xtick={-7.85, -6.28, -4.71, -3.14, -1.57, 0, 1.57, 3.142, 4.71, 6.28, 7.85},
            xticklabels={$\tfrac{-5\pi}{2}$,$-2\pi$,$\tfrac{-3\pi}{2}$,$-\pi$, $\tfrac{-\pi}{2}$, $0$, $\tfrac{\pi}{2}$, $\pi$, $\tfrac{3\pi}{2}$, $2\pi$, $\tfrac{5\pi}{2}$},
            yticklabels={$-3$,$-2$,$-1$,$1$,$2$,$3$,$4$}, 
            ticklabel style={font=\scriptsize},
            every axis y label/.style={at=(current axis.above origin),anchor=south},
            every axis x label/.style={at=(current axis.right of origin),anchor=west},
            axis on top
          ]
          

            \addplot [line width=2, penColor, dashed,samples=300,domain=(-10:10),<->] {sin(deg(x)};
            \addplot [line width=2, penColor2, smooth,samples=300,domain=(-10:10),<->] {2*sin(deg(x-0.785398))+2};



  \end{axis}
\end{tikzpicture}
\end{image}





\end{example}





Since $\theta$ is multiplied by $1$, the period is still $2\pi$.


\begin{itemize}
\item period - wave length
\item frequency
\end{itemize}





$\blacktriangleright$ \textbf{phase}

If you thnk of an oscillating waveform as a signal, like a sound wave, then the magnitude goes up and down as time passes. The magnitude is called the \textbf{phase} of the waveform.

Two waveforms of the same frequency are in phase, if their maximum and minimum magnitudes align perfectly in time.

If one waveform is shifted by half a period, then the waveforms are said to be \textbf{out of phase}.




\begin{center}
\desmos{fwuxnrac77}{400}{300}
\end{center}







































\end{document}

