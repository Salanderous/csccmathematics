\documentclass{ximera}


\graphicspath{
  {./}
  {ximeraTutorial/}
  {basicPhilosophy/}
}

\newcommand{\mooculus}{\textsf{\textbf{MOOC}\textnormal{\textsf{ULUS}}}}

\usepackage{tkz-euclide}\usepackage{tikz}
\usepackage{tikz-cd}
\usetikzlibrary{arrows}
\tikzset{>=stealth,commutative diagrams/.cd,
  arrow style=tikz,diagrams={>=stealth}} %% cool arrow head
\tikzset{shorten <>/.style={ shorten >=#1, shorten <=#1 } } %% allows shorter vectors

\usetikzlibrary{backgrounds} %% for boxes around graphs
\usetikzlibrary{shapes,positioning}  %% Clouds and stars
\usetikzlibrary{matrix} %% for matrix
\usepgfplotslibrary{polar} %% for polar plots
\usepgfplotslibrary{fillbetween} %% to shade area between curves in TikZ
\usetkzobj{all}
\usepackage[makeroom]{cancel} %% for strike outs
%\usepackage{mathtools} %% for pretty underbrace % Breaks Ximera
%\usepackage{multicol}
\usepackage{pgffor} %% required for integral for loops



%% http://tex.stackexchange.com/questions/66490/drawing-a-tikz-arc-specifying-the-center
%% Draws beach ball
\tikzset{pics/carc/.style args={#1:#2:#3}{code={\draw[pic actions] (#1:#3) arc(#1:#2:#3);}}}



\usepackage{array}
\setlength{\extrarowheight}{+.1cm}
\newdimen\digitwidth
\settowidth\digitwidth{9}
\def\divrule#1#2{
\noalign{\moveright#1\digitwidth
\vbox{\hrule width#2\digitwidth}}}






\DeclareMathOperator{\arccot}{arccot}
\DeclareMathOperator{\arcsec}{arcsec}
\DeclareMathOperator{\arccsc}{arccsc}

















%%This is to help with formatting on future title pages.
\newenvironment{sectionOutcomes}{}{}


\title{2D Numbers}

\begin{document}

\begin{abstract}
%Stuff can go here later if we want!
\end{abstract}
\maketitle





The real numbers are not enough. \\


We are missing some numbers. \\


What is a number? \\


When you start off in school, the numbers are the whole numbers.  At some point, you realize that the whole numbers can't account for what you see and you bring in the integers.  At some point, you would like to describe situations that the integers just cannot describe.  You bring in the rational numbers. 


This story continues.

We study circles and discover $\pi$.

We study squares and discover a diagonal of length $\sqrt{2}$.


Eventually, we need irrational numbers to describe the structure we discover.


We continually discover that we need more numbers than we have. \\


\begin{center}
\textbf{\textcolor{red!90!darkgray}{We are there again!}}
\end{center}


Our story is yet evolving once again.


\begin{explanation}

We have a situation that the real numbers are unable to describe.


We have quadratic functions whose roots are not in the real numbers.  


We need to expand our idea of number.

\end{explanation}



Our plan is to insert a new number whose square is $-1$.  We are inserting $\sqrt{-1}$.  


This will fix our quadratic problem. \\


Will it fix other problems?






\subsection{Learning Outcomes}



\begin{sectionOutcomes}
In this section, students will 

\begin{itemize}
\item extend the real numbers.
\end{itemize}
\end{sectionOutcomes}










\begin{center}
\textbf{\textcolor{green!50!black}{ooooo=-=-=-=-=-=-=-=-=-=-=-=-=ooOoo=-=-=-=-=-=-=-=-=-=-=-=-=ooooo}} \\

more examples can be found by following this link\\ \link[More Examples of Representing Numbers]{https://ximera.osu.edu/csccmathematics/precalculus2/precalculus2/representingNumbers/examples/exampleList}

\end{center}







\end{document}
