\documentclass{ximera}


\graphicspath{
  {./}
  {ximeraTutorial/}
  {basicPhilosophy/}
}

\newcommand{\mooculus}{\textsf{\textbf{MOOC}\textnormal{\textsf{ULUS}}}}

\usepackage{tkz-euclide}\usepackage{tikz}
\usepackage{tikz-cd}
\usetikzlibrary{arrows}
\tikzset{>=stealth,commutative diagrams/.cd,
  arrow style=tikz,diagrams={>=stealth}} %% cool arrow head
\tikzset{shorten <>/.style={ shorten >=#1, shorten <=#1 } } %% allows shorter vectors

\usetikzlibrary{backgrounds} %% for boxes around graphs
\usetikzlibrary{shapes,positioning}  %% Clouds and stars
\usetikzlibrary{matrix} %% for matrix
\usepgfplotslibrary{polar} %% for polar plots
\usepgfplotslibrary{fillbetween} %% to shade area between curves in TikZ
\usetkzobj{all}
\usepackage[makeroom]{cancel} %% for strike outs
%\usepackage{mathtools} %% for pretty underbrace % Breaks Ximera
%\usepackage{multicol}
\usepackage{pgffor} %% required for integral for loops



%% http://tex.stackexchange.com/questions/66490/drawing-a-tikz-arc-specifying-the-center
%% Draws beach ball
\tikzset{pics/carc/.style args={#1:#2:#3}{code={\draw[pic actions] (#1:#3) arc(#1:#2:#3);}}}



\usepackage{array}
\setlength{\extrarowheight}{+.1cm}
\newdimen\digitwidth
\settowidth\digitwidth{9}
\def\divrule#1#2{
\noalign{\moveright#1\digitwidth
\vbox{\hrule width#2\digitwidth}}}






\DeclareMathOperator{\arccot}{arccot}
\DeclareMathOperator{\arcsec}{arcsec}
\DeclareMathOperator{\arccsc}{arccsc}

















%%This is to help with formatting on future title pages.
\newenvironment{sectionOutcomes}{}{}


\title{Real Numbers}

\begin{document}

\begin{abstract}
dots and arrows
\end{abstract}
\maketitle



\textbf{$\mathbb{R}$} is not enough.  Our investigation of the real numbers quickly expanded in many directions. We invented functions to help us see relationships within the real numbers.  However, this investigation also revealed shortcomings of the real numbers. The easiest example of which is that many quadratic functions cannot be factored using just the real numbers. We need more numbers.


\begin{center}
An immediate idea is to just use two copies of the real numbers.
\end{center}


In fact, we have been using two copies of the real numbers all along. Our graphs and curves on the Cartesian plane use two copies of the real numbers.  We have been using the Cartesian plane to visually encode numbers as distances in an effort to illustrate the pairs in a function. Maybe we can repurpose the Cartesian plane to also have an algebraic structure - hopefully in a manner similar to how we use the real number. \\








We have a couple of tools we use to describe real numbers and our operations on them.


$\blacktriangleright$  We use written symbols to represent numbers and operations.  A string of valid symbols is called an expression and two expressions representing the same numbers are called equivalent. Accompanying these symbols are a list of manipulation rules and operations dictating how these symbols can be rearranged and replaced and yet still represent the same number.  


$\blacktriangleright$  We also have diagrams that use space, direction, and size to visually represent numbers and operations. The basic diagram has a line representing an ordering of the real numbers. Dots on the number line are one option for representing numbers, while arrows are often used to describe operations.


We would like to extend this system from one copy of \textbf{$\mathbb{R}$} to two copies of \textbf{$\mathbb{R}$}.








\section{The Real Line:  $\mathbb{R}$}





Generally speaking we draw the real line with the negative numbers growing to the left and positive numbers growing to the right. Somewhere in the middle we draw a tick mark for $0$. Our first use it to plot dots (points) on the real line at distances from $0$ that encode the value of the number.












  \begin{image}
  \begin{tikzpicture}
    \begin{axis}[
            xmin=-10,xmax=10,ymin=-1,ymax=1,
            %width=3in,
            clip=false,
            axis lines=center,
            %ticks=none,
            unit vector ratio*=1 1 1,
            ymajorticks=false,
            xtick={-8,-6,-4,-2,2,4,6,8},
            %xlabel=$x$, ylabel=$y$,
            ticklabel style={font=\scriptsize},
            %every axis y label/.style={at=(current axis.above origin),anchor=south},
            every axis x label/.style={at=(current axis.right of origin),anchor=west},
          ]     

            \addplot [color=penColor2,only marks,mark=*] coordinates{(-4,0)};
            \addplot [color=penColor2,only marks,mark=*] coordinates{(3,0)};
            \addplot [color=penColor2,only marks,mark=*] coordinates{(5,0)};




        \end{axis}
  \end{tikzpicture}
  \end{image}


Dots or points are a way of visually comparing numbers.  BIgger numbers are further from $0$.  Greater numbers are to the right.  Lesser numbers are to the left.




Operations require something more than just dots. We often use arrows to indicate addition and subtraction.

Unlike dots, an arrow's position doesn't matter. An arrow's length and direction represent the value of the number.

A positive number is represented by an arrow facing to the right.

These arrows all represent $3$.








  \begin{image}
  \begin{tikzpicture}
    \begin{axis}[
            xmin=-10,xmax=10,ymin=-1,ymax=1,
            %width=3in,
            clip=false,
            axis lines=center,
            %ticks=none,
            unit vector ratio*=1 1 1,
            ymajorticks=false,
            xtick={-8,-6,-4,-2,2,4,6,8},
            %xlabel=$x$, ylabel=$y$,
            ticklabel style={font=\scriptsize},
            %every axis y label/.style={at=(current axis.above origin),anchor=south},
            every axis x label/.style={at=(current axis.right of origin),anchor=west},
          ]     


            \addplot [line width=2, penColor, smooth,samples=200,domain=(0:3),->] ({x},{0.5});

        \end{axis}
  \end{tikzpicture}
  \end{image}











  \begin{image}
  \begin{tikzpicture}
    \begin{axis}[
            xmin=-10,xmax=10,ymin=-1,ymax=1,
            %width=3in,
            clip=false,
            axis lines=center,
            %ticks=none,
            unit vector ratio*=1 1 1,
            ymajorticks=false,
            xtick={-8,-6,-4,-2,2,4,6,8},
            %xlabel=$x$, ylabel=$y$,
            ticklabel style={font=\scriptsize},
            %every axis y label/.style={at=(current axis.above origin),anchor=south},
            every axis x label/.style={at=(current axis.right of origin),anchor=west},
          ]     


            \addplot [line width=2, penColor, smooth,samples=200,domain=(-7:-4),->] ({x},{0.5});

        \end{axis}
  \end{tikzpicture}
  \end{image}






  \begin{image}
  \begin{tikzpicture}
    \begin{axis}[
            xmin=-10,xmax=10,ymin=-1,ymax=1,
            %width=3in,
            clip=false,
            axis lines=center,
            %ticks=none,
            unit vector ratio*=1 1 1,
            ymajorticks=false,
            xtick={-8,-6,-4,-2,2,4,6,8},
            %xlabel=$x$, ylabel=$y$,
            ticklabel style={font=\scriptsize},
            %every axis y label/.style={at=(current axis.above origin),anchor=south},
            every axis x label/.style={at=(current axis.right of origin),anchor=west},
          ]     


            \addplot [line width=2, penColor, smooth,samples=200,domain=(5:8),->] ({x},{0.5});

        \end{axis}
  \end{tikzpicture}
  \end{image}







A negative number is represented by an arrow facing to the left.

These arrows all represent $-3$.








  \begin{image}
  \begin{tikzpicture}
    \begin{axis}[
            xmin=-10,xmax=10,ymin=-1,ymax=1,
            %width=3in,
            clip=false,
            axis lines=center,
            %ticks=none,
            unit vector ratio*=1 1 1,
            ymajorticks=false,
            xtick={-8,-6,-4,-2,2,4,6,8},
            %xlabel=$x$, ylabel=$y$,
            ticklabel style={font=\scriptsize},
            %every axis y label/.style={at=(current axis.above origin),anchor=south},
            every axis x label/.style={at=(current axis.right of origin),anchor=west},
          ]     


            \addplot [line width=2, penColor2, smooth,samples=200,domain=(0:3),<-] ({x},{0.5});

        \end{axis}
  \end{tikzpicture}
  \end{image}











  \begin{image}
  \begin{tikzpicture}
    \begin{axis}[
            xmin=-10,xmax=10,ymin=-1,ymax=1,
            %width=3in,
            clip=false,
            axis lines=center,
            %ticks=none,
            unit vector ratio*=1 1 1,
            ymajorticks=false,
            xtick={-8,-6,-4,-2,2,4,6,8},
            %xlabel=$x$, ylabel=$y$,
            ticklabel style={font=\scriptsize},
            %every axis y label/.style={at=(current axis.above origin),anchor=south},
            every axis x label/.style={at=(current axis.right of origin),anchor=west},
          ]     


            \addplot [line width=2, penColor2, smooth,samples=200,domain=(-7:-4),<-] ({x},{0.5});

        \end{axis}
  \end{tikzpicture}
  \end{image}






  \begin{image}
  \begin{tikzpicture}
    \begin{axis}[
            xmin=-10,xmax=10,ymin=-1,ymax=1,
            %width=3in,
            clip=false,
            axis lines=center,
            %ticks=none,
            unit vector ratio*=1 1 1,
            ymajorticks=false,
            xtick={-8,-6,-4,-2,2,4,6,8},
            %xlabel=$x$, ylabel=$y$,
            ticklabel style={font=\scriptsize},
            %every axis y label/.style={at=(current axis.above origin),anchor=south},
            every axis x label/.style={at=(current axis.right of origin),anchor=west},
          ]     


            \addplot [line width=2, penColor2, smooth,samples=200,domain=(5:8),<-] ({x},{0.5});

        \end{axis}
  \end{tikzpicture}
  \end{image}







The arrow end of an arrow is called the \textbf{head} and the other end is called the \textbf{tail}.  We envision the arrow encoding the direction going from tail to head.  Or, the arrow points in its direction.



\textbf{\textcolor{red!25!blue!75!}{Operations}} \\


Arrows help us tell the story of addition and subtraction.






The story of $2+3=5$ as told through vectors is a picture story.  First, an arrow representing $2$ is placed with its tail at $0$.  Secondly, an arrow representing $3$ is placed.  Addition is illustrated by a head-to-tail placement.  Finally, the result is represented by a arrow running form the tail of the first vector to the head of the second vector. This vector is pointing to the right and has a length of $5$.  The result is $5$.




  \begin{image}
  \begin{tikzpicture}
    \begin{axis}[
            xmin=-10,xmax=10,ymin=-1,ymax=1,
            %width=3in,
            clip=false,
            axis lines=center,
            %ticks=none,
            unit vector ratio*=1 1 1,
            ymajorticks=false,
            xtick={-8,-6,-4,-2,2,4,6,8},
            %xlabel=$x$, ylabel=$y$,
            ticklabel style={font=\scriptsize},
            %every axis y label/.style={at=(current axis.above origin),anchor=south},
            every axis x label/.style={at=(current axis.right of origin),anchor=west},
          ]     


            \addplot [line width=2, penColor, smooth,samples=200,domain=(0:2),->] ({x},{0.5});
            \addplot [line width=2, penColor, smooth,samples=200,domain=(2:5),->] ({x},{0.5});

            \addplot [line width=2, penColor, smooth,samples=200,domain=(0:5),->] ({x},{-0.5});

        \end{axis}
  \end{tikzpicture}
  \end{image}















The story of $2+(-5)=-3$ as told through vectors is a picture story.  First, an arrow representing $2$ is placed with its tail at $0$.  Secondly, an arrow representing $-5$ is placed.  Addition is illustrated by a head-to-tail placement.  Finally, the result is represented by a arrow running form the tail of the first vector to the head of the second vector. This vector is pointing to the left and has a length of $3$.  The result is $-3$.




  \begin{image}
  \begin{tikzpicture}
    \begin{axis}[
            xmin=-10,xmax=10,ymin=-1,ymax=1,
            %width=3in,
            clip=false,
            axis lines=center,
            %ticks=none,
            unit vector ratio*=1 1 1,
            ymajorticks=false,
            xtick={-8,-6,-4,-2,2,4,6,8},
            %xlabel=$x$, ylabel=$y$,
            ticklabel style={font=\scriptsize},
            %every axis y label/.style={at=(current axis.above origin),anchor=south},
            every axis x label/.style={at=(current axis.right of origin),anchor=west},
          ]     


            \addplot [line width=2, penColor, smooth,samples=200,domain=(0:2),->] ({x},{0.5});
            \addplot [line width=2, penColor2, smooth,samples=200,domain=(-3:2),<-] ({x},{0.75});

            \addplot [line width=2, penColor2, smooth,samples=200,domain=(-3:0),<-] ({x},{-0.5});

        \end{axis}
  \end{tikzpicture}
  \end{image}















\textbf{Subtraction}





The story of $7 - 4 = 3$ as told through vectors is a picture story.  First, an arrow representing $7$ is placed with its tail at $0$.  Secondly, an arrow representing $4$ is also placed with its tail at $0$.  Subtraction is illustrated by a tail-to-tail placement.  Finally, the result is represented by a arrow running form the head of the second vector to the head of the first vector. This vector is pointing to the right and has a length of $3$.  The result is $3$.




  \begin{image}
  \begin{tikzpicture}
    \begin{axis}[
            xmin=-10,xmax=10,ymin=-1,ymax=1,
            %width=3in,
            clip=false,
            axis lines=center,
            %ticks=none,
            unit vector ratio*=1 1 1,
            ymajorticks=false,
            xtick={-8,-6,-4,-2,2,4,6,8},
            %xlabel=$x$, ylabel=$y$,
            ticklabel style={font=\scriptsize},
            %every axis y label/.style={at=(current axis.above origin),anchor=south},
            every axis x label/.style={at=(current axis.right of origin),anchor=west},
          ]     


            \addplot [line width=2, penColor, smooth,samples=200,domain=(0:7),->] ({x},{0.5});
            \addplot [line width=2, penColor, smooth,samples=200,domain=(0:4),->] ({x},{0.75});

            \addplot [line width=2, penColor, smooth,samples=200,domain=(4:7),->] ({x},{-0.5});

        \end{axis}
  \end{tikzpicture}
  \end{image}







The story of $2-(-3)=5$ as told through vectors is a picture story.  First, an arrow representing $2$ is placed with its tail at $0$.  Secondly, an arrow representing $-3$ is also placed with its tail at $0$.  Subtraction is illustrated by a tail-to-tail placement.  Finally, the result is represented by a arrow running form the head of the second vector to the head of the first vector. This vector is pointing to the right and has a length of $5$.  The result is $5$.




  \begin{image}
  \begin{tikzpicture}
    \begin{axis}[
            xmin=-10,xmax=10,ymin=-1,ymax=1,
            %width=3in,
            clip=false,
            axis lines=center,
            %ticks=none,
            unit vector ratio*=1 1 1,
            ymajorticks=false,
            xtick={-8,-6,-4,-2,2,4,6,8},
            %xlabel=$x$, ylabel=$y$,
            ticklabel style={font=\scriptsize},
            %every axis y label/.style={at=(current axis.above origin),anchor=south},
            every axis x label/.style={at=(current axis.right of origin),anchor=west},
          ]     


            \addplot [line width=2, penColor, smooth,samples=200,domain=(0:2),->] ({x},{0.5});
            \addplot [line width=2, penColor2, smooth,samples=200,domain=(-3:0),<-] ({x},{0.5});

            \addplot [line width=2, penColor, smooth,samples=200,domain=(-3:2),->] ({x},{-0.5});

        \end{axis}
  \end{tikzpicture}
  \end{image}













The story of $-2-(-5)=3$ as told through vectors is a picture story.  First, an arrow representing $-2$ is placed with its tail at $0$.  Secondly, an arrow representing $-5$ is also placed with its tail at $0$.  Subtraction is illustrated by a tail-to-tail placement.  Finally, the result is represented by a arrow running form the head of the second vector to the head of the first vector. This vector is pointing to the right and has a length of $3$.  The result is $3$.




  \begin{image}
  \begin{tikzpicture}
    \begin{axis}[
            xmin=-10,xmax=10,ymin=-1,ymax=1,
            %width=3in,
            clip=false,
            axis lines=center,
            %ticks=none,
            unit vector ratio*=1 1 1,
            ymajorticks=false,
            xtick={-8,-6,-4,-2,2,4,6,8},
            %xlabel=$x$, ylabel=$y$,
            ticklabel style={font=\scriptsize},
            %every axis y label/.style={at=(current axis.above origin),anchor=south},
            every axis x label/.style={at=(current axis.right of origin),anchor=west},
          ]     


            \addplot [line width=2, penColor, smooth,samples=200,domain=(-2:0),<-] ({x},{0.5});
            \addplot [line width=2, penColor2, smooth,samples=200,domain=(-5:0),<-] ({x},{0.75});

            \addplot [line width=2, penColor, smooth,samples=200,domain=(-5:-2),->] ({x},{-0.5});

        \end{axis}
  \end{tikzpicture}
  \end{image}




From these diagrams we can see that 

\begin{itemize}
\item Addition is commutative.  The head-to-head alignment results in the same arrow regardless of which arrow comes first.
\item Subtraction is not commutattive. The tail-to-tail alignmenbt will reverse the resulting arrow if the order is reversed.
\end{itemize}






We want to carry this arrow idea for operations over from 1-dimensional numbers to 2-dimensional numbers.





































\section{The Plane:  $\mathbb{R}^2 = \mathbb{R} \times \mathbb{R}$}




We would like to extend this system from one copy of \textbf{$\mathbb{R}$} to two copies of \textbf{$\mathbb{R}$}. \textbf{$\mathbb{R} \times \mathbb{R}$} is shorthand notation for two copies of the real line forming the Cartesian plane.  \textbf{$\mathbb{R}^2$} is even shorter shorthand notation.

$\mathbb{R}^2$ is just the collection of ordered pairs of real numbers.

\[  \mathbb{R}^2 = \{  (r_1, r_2)    \, | \,  r_1, r_2 \in \mathbb{R}  \}      \]


The Cartesian plane is our geometric, visual representation of these ordered pairs. We plot points or dots to represent a 2-dimensional number.



Plotting dots for the numbers $(-4,7)$, $(3,5)$, $(5,-6)$, and $(-7,-3)$.

\begin{image}
\begin{tikzpicture}
  \begin{axis}[
            domain=-10:10, ymax=10, xmax=10, ymin=-10, xmin=-10,
            axis lines =center, xlabel=$x$, ylabel=$y$, grid = major, grid style={dashed},
            ytick={-10,-8,-6,-4,-2,2,4,6,8,10},
            xtick={-10,-8,-6,-4,-2,2,4,6,8,10},
            yticklabels={$-10$,$-8$,$-6$,$-4$,$-2$,$2$,$4$,$6$,$8$,$10$}, 
            xticklabels={$-10$,$-8$,$-6$,$-4$,$-2$,$2$,$4$,$6$,$8$,$10$},
            ticklabel style={font=\scriptsize},
            every axis y label/.style={at=(current axis.above origin),anchor=south},
            every axis x label/.style={at=(current axis.right of origin),anchor=west},
            axis on top
          ]
          
            \addplot [color=penColor2,only marks,mark=*] coordinates{(-4,7)};
            \addplot [color=penColor2,only marks,mark=*] coordinates{(3,5)};
            \addplot [color=penColor2,only marks,mark=*] coordinates{(5,-6)};
            \addplot [color=penColor2,only marks,mark=*] coordinates{(-7,-3)};

  \end{axis}
\end{tikzpicture}
\end{image}




We now have a 2-dimensional number line.








\textbf{\textcolor{red!25!blue!75!}{Operations}} \\



We now enhance the plane with an arithmetic.  We are just going to adopt the arrow rules.  

\textbf{Step 1)} Convert each dot to an arrow.


$\blacktriangleright$ \textbf{Addition:}  Draw the first arrow with its tail at the origin.  Draw the second arrow head-to-tail. The  resulting arrow is drawn from the tail of the first arrow to the head of the second arrow.  This resulting arrow represents the sum of the two.



$\blacktriangleright$ \textbf{Subtraction:}  Draw the first arrow with its tail at the origin.  Draw the second arrow with its tail at the origin. The resulting arrow is drawn from the head of the second arrow to the head of the first arrow.  This resulting arrow represents the difference of the two.



\begin{example}


$(4,6) + (2, -10)$


\begin{image}
\begin{tikzpicture}
  \begin{axis}[
            domain=-10:10, ymax=10, xmax=10, ymin=-10, xmin=-10,
            axis lines =center, xlabel=$x$, ylabel=$y$, grid = major, grid style={dashed},
            ytick={-10,-8,-6,-4,-2,2,4,6,8,10},
            xtick={-10,-8,-6,-4,-2,2,4,6,8,10},
            yticklabels={$-10$,$-8$,$-6$,$-4$,$-2$,$2$,$4$,$6$,$8$,$10$}, 
            xticklabels={$-10$,$-8$,$-6$,$-4$,$-2$,$2$,$4$,$6$,$8$,$10$},
            ticklabel style={font=\scriptsize},
            every axis y label/.style={at=(current axis.above origin),anchor=south},
            every axis x label/.style={at=(current axis.right of origin),anchor=west},
            axis on top
          ]


            \draw[penColor,ultra thick,->] (axis cs:0,0) -- (axis cs:4,6);
            \draw[penColor,ultra thick,->] (axis cs:4,6) -- (axis cs:6,-4);
            \draw[penColor2,ultra thick,->] (axis cs:0,0) -- (axis cs:6,-4);



  \end{axis}
\end{tikzpicture}
\end{image}


$(4,6) + (2, -10) = (6, -4)$


\end{example}





This discussion is very suggestive of a symbolic version of 2-dimensional addition and subtraction.







\section{Vectors}


The arrow combination suggest an algebraic description.

$(4,6) + (2, -10) = (6, -4)$



Except, we are already using the parentheses to represent points or dots.




\begin{image}
\begin{tikzpicture}
  \begin{axis}[
            domain=-10:10, ymax=10, xmax=10, ymin=-10, xmin=-10,
            axis lines =center, xlabel=$x$, ylabel=$y$, grid = major, grid style={dashed},
            ytick={-10,-8,-6,-4,-2,2,4,6,8,10},
            xtick={-10,-8,-6,-4,-2,2,4,6,8,10},
            yticklabels={$-10$,$-8$,$-6$,$-4$,$-2$,$2$,$4$,$6$,$8$,$10$}, 
            xticklabels={$-10$,$-8$,$-6$,$-4$,$-2$,$2$,$4$,$6$,$8$,$10$},
            ticklabel style={font=\scriptsize},
            every axis y label/.style={at=(current axis.above origin),anchor=south},
            every axis x label/.style={at=(current axis.right of origin),anchor=west},
            axis on top
          ]

            \addplot [color=penColor2,only marks,mark=*] coordinates{(4,6)};
            \addplot [color=penColor2,only marks,mark=*] coordinates{(2,-10)};
            \addplot [color=penColor2,only marks,mark=*] coordinates{(6,-4)};



  \end{axis}
\end{tikzpicture}
\end{image}




We need new words and symbols.




A \textbf{vector} is going to play the part of a number.  We'll represent vectors with arrows. Just like on the 1-dimensional number line, vectors don't have a position.  They have a length and a direction.  Many arrows drawn at different positions on the plane could represent the same vector.

We'll use \textbf{triangular brackets} as algebraic notation for vectors. 

Example:  $\langle 4, 5 \rangle$










\end{document}
