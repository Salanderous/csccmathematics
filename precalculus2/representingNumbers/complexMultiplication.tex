\documentclass{ximera}


\graphicspath{
  {./}
  {ximeraTutorial/}
  {basicPhilosophy/}
}

\newcommand{\mooculus}{\textsf{\textbf{MOOC}\textnormal{\textsf{ULUS}}}}

\usepackage{tkz-euclide}\usepackage{tikz}
\usepackage{tikz-cd}
\usetikzlibrary{arrows}
\tikzset{>=stealth,commutative diagrams/.cd,
  arrow style=tikz,diagrams={>=stealth}} %% cool arrow head
\tikzset{shorten <>/.style={ shorten >=#1, shorten <=#1 } } %% allows shorter vectors

\usetikzlibrary{backgrounds} %% for boxes around graphs
\usetikzlibrary{shapes,positioning}  %% Clouds and stars
\usetikzlibrary{matrix} %% for matrix
\usepgfplotslibrary{polar} %% for polar plots
\usepgfplotslibrary{fillbetween} %% to shade area between curves in TikZ
\usetkzobj{all}
\usepackage[makeroom]{cancel} %% for strike outs
%\usepackage{mathtools} %% for pretty underbrace % Breaks Ximera
%\usepackage{multicol}
\usepackage{pgffor} %% required for integral for loops



%% http://tex.stackexchange.com/questions/66490/drawing-a-tikz-arc-specifying-the-center
%% Draws beach ball
\tikzset{pics/carc/.style args={#1:#2:#3}{code={\draw[pic actions] (#1:#3) arc(#1:#2:#3);}}}



\usepackage{array}
\setlength{\extrarowheight}{+.1cm}
\newdimen\digitwidth
\settowidth\digitwidth{9}
\def\divrule#1#2{
\noalign{\moveright#1\digitwidth
\vbox{\hrule width#2\digitwidth}}}






\DeclareMathOperator{\arccot}{arccot}
\DeclareMathOperator{\arcsec}{arcsec}
\DeclareMathOperator{\arccsc}{arccsc}

















%%This is to help with formatting on future title pages.
\newenvironment{sectionOutcomes}{}{}


\title{Complex Multiplication}

\begin{document}

\begin{abstract}
intervals
\end{abstract}
\maketitle


Multiplication of complex numbers is weird.





\begin{definition}


\[    (a + b \, i) \dot (c + d \, i) = (ac-bd) + (ad+bc) \, i           \]

\end{definition}


It is defined this way to naintain some relationships that we thoght should be valid.



Division requires a little discussion.



With real numbers, we have stories the give meaning to division.


$10 \div 5$ has two meanings.  This expresion might represent how many groups are needed to split a group of $10$ marbles into groups of size $5$.  That would be $2$ groups.  It could also represent how many marbles would be in each group, if $10$ marbles were split up into $5$ equal groups.  That would be $2$ marbles per group.


But, even real numbers stretch this story.  $10 \div \sqrt{2}$ ?  What does it mean to split a pile of marbles into $\sqrt{2}$ piles.

This marble story is going to break with ocmplex numbers.  Fortunately, the real numbers have a different view of division, which we can bring with us over to complex numbers.





Just from a strictly algebaic viewpoint, $10 \div 5$ represents the number that you multipliy $5$ by, to get $10$.  In this way, we can think about division in terms of multiplication.  This thought leads to reciprocals and fractions.  Luckily, every real number, except $0$ has an inverse.  What about complex numbers?  Do they have reciprocals?  DOes each complex number have a partner complex number such that their product equals $1$?



\begin{example} Multiplicative Inverse


Can we solve

\[   1 = (2 + 3 \, i) \cdot (a + b \, i)       \]



\begin{align*}
1     & = 1 + 0 \, i  \\
      & =    (2 + 3 \, i) \cdot (a + b \, i)   \\
      & =    (2a - 3b) + (2b+3a) \, i   
\end{align*}


We need $1 = 2a - 3b$ and $0 = 2b+3a$.


The second  equation tells us that $\frac{-3a}{2} = b$.  Substituting this into the first equation tells us that $1 = 2a - 3 \cdot \frac{-3a}{2}$.  Now we can solve for $a$.



\begin{align*}
1     & = 2a - 3\frac{-3a}{2}  \\
      & =   2a + \frac{9a}{2}   \\
      & =    \frac{13a}{2}   \\
  \frac{2}{13}    & =  a
\end{align*}




With that, we can get a value for $b$.   $ b = \frac{-3a}{2} = \frac{2}{13}  \cdot \frac{-3}{2} = -\frac{3}{13}$.

Let's check:



\[    (2 + 3 \, i) \cdot (\frac{2}{13} - \frac{3}{13} \, i)     =   \left( 2 \cdot  \frac{2}{13} - 3 \cdot \left(- \frac{3}{13}\right)\right)  + \left( 3 \cdot \frac{2}{13} - 2 \cdot \frac{3}{13} \right) \, i = \frac{4+9}{13} + \frac{6-6}{13} \, i = 1  \]







\end{example}


Does this process work for all nonzero complex numbers? \\


$\blacktriangleright$  We know it works for $r + 0 \, i$ with $r \ne 0$, because these are real numbers.


\[   (r + 0 \, i) \cdot \left(\frac{1}{r} + 0 \, i \right) = 1        \]







$\blacktriangleright$  Similarly, it works for $0 + r \, i$ with $r \ne 0$.


\[   (0 + r \, i) \cdot \left(0 - \frac{1}{r} \, i \right) = 1        \]





$\blacktriangleright$ Now for complex numbers where neither component is $0$


Given real numbers $A \ne 0$ and $B \ne 0$, solve the following for $x$ and $y$.


\[       1 = (A + B \, i) \cdot (x + y \, i)            \]



\begin{align*}
1          & = (A + B \, i) \cdot (x + y \, i)      \\
           & = (Ax-By) + (Bx+Ay) \, i
\end{align*}


For this to work, we need $A x - B y = 1$ and $B x + A y = 0$.



The second equaiton tells us that $x = -\frac{A y}{B}$, which is ok, since $B \ne 0$ in this case.

Substituting that into the first equation gives us


\[   A \cdot \left(-\frac{A y}{B}\right) - B y = 1     \]


\[   -\frac{A^2 y}{B} - B y = 1     \]

\[   -\frac{A^2 y}{B} - \frac{B^2}{B} y = 1     \]


\[   -\frac{A^2 + B^2}{B} y = 1     \]


\[  y = \frac{-B}{A^2 + B^2}     \]

This is ok, since $A^2 + B^2 \ne 0$, because neither $A$ nor $B$ equals $0$.




$\blacktriangleright$  Every nonzero complex number has a reciprocal.  We will use this to define division as multiplication by the reciprocal.





\section{Complex Conjugate}


The \textbf{modulus} of a complex number is its distance from the origin or $0$.  The symbol for the conjuagte of a complex number, $z$, is $|z|$.  if $z = a + b \, i$, then

\[    |z| = |a + b \, i| = \sqrt{a^b + b^2}          \]



Just like with real numbers, complex arithmetic doesn't work very well with absolute value. In complex arithmetic, we have an alternative.



\[   (a + b \, i) \cdot    (a - b \, i)  = a^2 + b^2   \]



$\blacktriangleright$  $a - b \, i$ is called the \textbf{complex conjugate} of $a + b \, i$ and it will help us with division.




\[   \frac{1}{a + b \, i} =   \frac{1}{a + b \, i} \cdot 1 = \frac{1}{a + b \, i} \cdot \frac{a - b \, i}{a - b \, i}  =  \frac{a - b \, i}{(a + b \, i)(a - b \, i)} =   \frac{a - b \, i}{a^2 + b^2}  \]


\[  \frac{1}{a + b \, i}   =  \frac{a}{a^2 + b^2} - \frac{b}{a^2 + b^2} \, i      \]


We can use the complex conjugate of the denominator to get us back to multipication.













\end{document}
