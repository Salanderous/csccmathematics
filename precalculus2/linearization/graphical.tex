\documentclass{ximera}



\graphicspath{
  {./}
  {ximeraTutorial/}
  {basicPhilosophy/}
}

\newcommand{\mooculus}{\textsf{\textbf{MOOC}\textnormal{\textsf{ULUS}}}}

\usepackage{tkz-euclide}\usepackage{tikz}
\usepackage{tikz-cd}
\usetikzlibrary{arrows}
\tikzset{>=stealth,commutative diagrams/.cd,
  arrow style=tikz,diagrams={>=stealth}} %% cool arrow head
\tikzset{shorten <>/.style={ shorten >=#1, shorten <=#1 } } %% allows shorter vectors

\usetikzlibrary{backgrounds} %% for boxes around graphs
\usetikzlibrary{shapes,positioning}  %% Clouds and stars
\usetikzlibrary{matrix} %% for matrix
\usepgfplotslibrary{polar} %% for polar plots
\usepgfplotslibrary{fillbetween} %% to shade area between curves in TikZ
\usetkzobj{all}
\usepackage[makeroom]{cancel} %% for strike outs
%\usepackage{mathtools} %% for pretty underbrace % Breaks Ximera
%\usepackage{multicol}
\usepackage{pgffor} %% required for integral for loops



%% http://tex.stackexchange.com/questions/66490/drawing-a-tikz-arc-specifying-the-center
%% Draws beach ball
\tikzset{pics/carc/.style args={#1:#2:#3}{code={\draw[pic actions] (#1:#3) arc(#1:#2:#3);}}}



\usepackage{array}
\setlength{\extrarowheight}{+.1cm}
\newdimen\digitwidth
\settowidth\digitwidth{9}
\def\divrule#1#2{
\noalign{\moveright#1\digitwidth
\vbox{\hrule width#2\digitwidth}}}






\DeclareMathOperator{\arccot}{arccot}
\DeclareMathOperator{\arcsec}{arcsec}
\DeclareMathOperator{\arccsc}{arccsc}

















%%This is to help with formatting on future title pages.
\newenvironment{sectionOutcomes}{}{}





\title[{Graphs}

\begin{document}
\begin{abstract}
tangent lines
\end{abstract}
\maketitle

%\section{Linear approximation}

Given a function, a \textit{linear approximation} is a fancy phrase
for something you already know:
\begin{quote}
  \textbf{The line tangent to the graph of a function at a point is very close to the graph of the function near that point.}
\end{quote}
This tangent line is the graph of a linear function, called  the \textbf{linear approximation}.
\begin{example}
Let $f$ be a function that is differentiable on some interval I that contains the point $a$. The graph of a function $f$  and the line tangent to the curve

 $y=f(x)$ at the point where $x=a$ are given in the figure below.
Find the equation of the tangent line.
 \begin{image}
%\begin{marginfigure}
\begin{tikzpicture}
	\begin{axis}[
            xmin=0,xmax=2,ymin=0,ymax=2,
            axis lines=center,
            ticks=none,
            %width=3in,
            %height=2in,
            unit vector ratio*=1 1 1,
            xlabel=$x$, ylabel=$y$,
            every axis y label/.style={at=(current axis.above origin),anchor=south},
            every axis x label/.style={at=(current axis.right of origin),anchor=west},
          ]        
          \addplot [ thick, penColor, smooth, domain=(0:3)] {3*sqrt(x)-2};
          \addplot [thick, penColor2,smooth] {(3/2)*x+3/2-2};
          \node at (axis cs:1.7,1.5) [penColor] {$y=f(x)$};
          \node at (axis cs:1,1.5) [penColor2] {$y=L(x)$}; 
          \node at (axis cs:1.3,1) [penColor2] {$(a,f(a))$}; 
            \addplot[color=penColor3,fill=penColor3,only marks,mark=*] coordinates{(1,1)};  %% closed hole         
        \end{axis}
\end{tikzpicture}
%\caption{A linear approximation of $f(x) = \sin(x)$ at $x=0$.}
%\label{figure:la sin}
%\end{marginfigure}
\end{image}
First, find the expression for $m$, the slope of the tangent line to the curve $y=f(x)$ at the point $(a,f(a))$.
 Select the correct choice.
 \begin{multipleChoice}
  \choice{$m= f(a)$}
  \choice[correct] {$m=f'(a)$}
  \choice{$m= \frac{f(a)-f(0)}{a-0}$}
  \choice{$m= \frac{f(a+h)-f(a)}{h}$}
  \choice{ We don't have enough information to determine the slope.}
  \end{multipleChoice}
Since we know that the point $(a,f(a))$  lies on the tangent line,  we can write an equation of the tangent line. 

\[
y= f'(a)(x-a) +f(a).
\]
Now, we define a function, $L$,  by $L(x)= f'(a)(x-a) +f(a)$. This function is linear and its graph is the line tangent to the curve $y=f(x)$ at the point where $x=a$.
This function deserves a special name.
\end{example}
\begin{definition}\index{linear approximation}
If $f$ is a function differentiable at $x=a$, then a \textbf{linear
  approximation} to the function $f$ at $x=a$ is given by
\[
L(x) = f'(a)(x-a) +f(a).
\]
\end{definition}

















\begin{example}
Let $f$ be a function defined by 
\[
f(x) =\sqrt[3]{x}.
\]
Approximate $\sqrt[3]{50}$, using $L$,  a linear approximation to the function $f$ at $a=64$.

\begin{explanation}
To start, write
\[
\frac{d}{dx} f(x) = \frac{d}{dx} x^{1/3} = \frac{1}{3x^{\answer[given]{2/3}}}.
\]
\begin{align*}
L(x) &= \answer[given]{4}+ \frac{1}{3\cdot 64^{2/3}} (x-64)  \\
&=4+ \frac{1}{\answer[given]{48}} (x-64) \\
&= \frac{x}{48} +\frac{8}{3}.
\end{align*}
\begin{image}
%\begin{marginfigure}
\begin{tikzpicture}
  \begin{axis}[
            xmin=0,xmax=100,ymin=0,ymax=5,
            axis lines=center,
             xtick={0, 20, 40, 50, 64, 80, 90, 100},
        xticklabels={0, 20, 40, 50, 64, 80, 90, 100},
            xlabel=$x$, ylabel=$y$,
            every axis y label/.style={at=(current axis.above origin),anchor=south},
            every axis x label/.style={at=(current axis.right of origin),anchor=west},
          ]        
          \addplot [very thick, penColor, samples=150,smooth,domain=(0:100)] {x^(1/3))};
          \addplot [very thick, penColor2, domain=(0:100)] {x/48+8/3};
          \addplot [textColor,dashed] plot coordinates {(64,0) (64,4)};
          \addplot [textColor,dashed] plot coordinates {(0,4) (64,4)};
          \addplot [textColor,dashed] plot coordinates {(50,0) (50,3.68)};
          \node at (axis cs:20,2.3) [penColor] {$f$};
          \node at (axis cs:20,3.3) [penColor2] {$L$};
          \addplot[color=penColor3,fill=penColor3,only marks,mark=*] coordinates{(64,4)};  %% closed hole     
           \addplot[color=penColor,fill=penColor,only marks,mark=*] coordinates{(50,0)};  %% closed hole  
                  
        \end{axis}
\end{tikzpicture}
%\caption{A linear approximation of $f(x) = \sqrt[3]{x}$ at $x=64$.}
%\label{figure:la sqrt3x}
%\end{marginfigure}
\end{image}
Now we evaluate $L(50) \approx 3.71$ and compare it to
$\sqrt[3]{50}\approx 3.68$.  From this we see that the linear
approximation, while perhaps inexact, is computationally \textbf{easier}
than computing the cube root.
\end{explanation}

What would happen if we chose $a=27$ instead?

Then we would use $L_{27}$, the linear approximation to the function $f$ at $a=27$.
In that case, $L_{27}=f(27)+f'(27)(x-27)=3+\frac{1}{27}(x-27)$.
The graph of  $L_{27}$, together with the graphs of $f$ and $L=L_{64}$ is given in the figure below.
\begin{image}
%\begin{marginfigure}
\begin{tikzpicture}
  \begin{axis}[
            xmin=1,xmax=100,ymin=0,ymax=5,
            axis lines=center,
              xtick={0, 27, 40, 50, 64, 80, 90, 100},
        xticklabels={0, 27, 40, 50, 64, 80, 90, 100},
            xlabel=$x$, ylabel=$y$,
            every axis y label/.style={at=(current axis.above origin),anchor=south},
            every axis x label/.style={at=(current axis.right of origin),anchor=west},
          ]        
          \addplot [very thick, penColor, samples=150,smooth,domain=(0:100)] {x^(1/3))};
          \addplot [very thick, penColor2, domain=(0:100)] {x/48+8/3};
            \addplot [very thick, penColor4, domain=(0:100)] {(x-27)/27+3};
              \addplot [textColor,dashed] plot coordinates {(50,0) (50,3.68)};
          \node at (axis cs:6,1.3) [penColor] {$f$};
          \node at (axis cs:20,3.3) [penColor2] {$L_{64}$};
           \node at (axis cs:6,2.5) [penColor4] {$L_{27}$};
          \addplot[color=penColor3,fill=penColor3,only marks,mark=*] coordinates{(64,4)};  %% closed hole         
            \addplot[color=penColor3,fill=penColor4,only marks,mark=*] coordinates{(27,3)};  %% closed hole    
             \addplot[color=penColor3,fill=penColor,only marks,mark=*] coordinates{(50,0)};  %% closed hole     
        \end{axis}
\end{tikzpicture}
%\caption{A linear approximation of $f(x) = \sqrt[3]{x}$ at $x=64$.}
%\label{figure:la sqrt3x}
%\end{marginfigure}
\end{image}
From the picture we can see that 

 $L_{27}(50)>L_{64}(50)>f(50)$.
 
 So, our choice, $a=64$, was better!
\end{example}
With modern calculators and computing software, it may not appear
necessary to use linear approximations. In fact they are quite
useful. In cases requiring an explicit numerical approximation, they
allow us to get a quick rough estimate which can be used as a
``reality check'' on a more complex calculation. In some complex
calculations involving functions, the linear approximation makes an
otherwise intractable calculation possible, without serious loss of
accuracy.














\begin{example}%\label{exam:linear approximation of sine}
Use a linear approximation of $f(x) =\sin(x)$ at $a=0$ to approximate
$\sin(0.3)$.
\begin{explanation}
To start, write
\[
\frac{d}{dx} f(x) = \answer[given]{\cos(x)},
\]
so our linear approximation is
\begin{align*}
L(x) &= 0+\answer[given]{\cos(0)}\cdot(x-0)\\
&= x.
\end{align*}
\begin{image}
%\begin{marginfigure}
\begin{tikzpicture}
  \begin{axis}[
            xmin=-1.6,xmax=1.6,ymin=-1.5,ymax=1.5,
            axis lines=center,
            xtick={-1.57, 0,0.3, 1.57},
            xticklabels={$-\pi/2$, $0$, 0.3, $\pi/2$},
            ytick={-1,1},
            %ticks=none,
            %width=3in,
            %height=2in,
            unit vector ratio*=1 1 1,
            xlabel=$x$, ylabel=$y$,
            every axis y label/.style={at=(current axis.above origin),anchor=south},
            every axis x label/.style={at=(current axis.right of origin),anchor=west},
          ]        
          \addplot [very thick, penColor, samples=100,smooth, domain=(-1.6:1.6)] {sin(deg(x))};
          \addplot [very thick, penColor2, samples=100,smooth] {x};
          \addplot [textColor,dashed] plot coordinates {(0.3,0) (0.3,0.295)};
          \node at (axis cs:1,.7) [penColor] {$f$};
          \node at (axis cs:1,1.18) [penColor2] {$L$};
          \addplot[color=penColor3,fill=penColor3,only marks,mark=*] coordinates{(0,0)};  %% closed hole   
            \addplot[color=penColor3,fill=penColor,only marks,mark=*] coordinates{(0.3,0)};  %% closed hole           
        \end{axis}
\end{tikzpicture}
%\caption{A linear approximation of $f(x) = \sin(x)$ at $x=0$.}
%\label{figure:la sin}
%\end{marginfigure}
\end{image}
Hence, a linear approximation for $\sin(x)$ at $a=0$ is $L(x) = x$,
and so $L(0.3) = 0.3$.  Comparing this to $\sin(.3) \approx 0.295$,
we see that the approximation is quite good. For this reason, it is common
to approximate $\sin(x)$ with its linear approximation $L(x) = x$
when $x$ is near zero.  
%see Figure~\ref{figure:la sin}.
\end{explanation}
\end{example}










\end{document}
