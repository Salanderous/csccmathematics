\documentclass{ximera}



\graphicspath{
  {./}
  {ximeraTutorial/}
  {basicPhilosophy/}
}

\newcommand{\mooculus}{\textsf{\textbf{MOOC}\textnormal{\textsf{ULUS}}}}

\usepackage{tkz-euclide}\usepackage{tikz}
\usepackage{tikz-cd}
\usetikzlibrary{arrows}
\tikzset{>=stealth,commutative diagrams/.cd,
  arrow style=tikz,diagrams={>=stealth}} %% cool arrow head
\tikzset{shorten <>/.style={ shorten >=#1, shorten <=#1 } } %% allows shorter vectors

\usetikzlibrary{backgrounds} %% for boxes around graphs
\usetikzlibrary{shapes,positioning}  %% Clouds and stars
\usetikzlibrary{matrix} %% for matrix
\usepgfplotslibrary{polar} %% for polar plots
\usepgfplotslibrary{fillbetween} %% to shade area between curves in TikZ
\usetkzobj{all}
\usepackage[makeroom]{cancel} %% for strike outs
%\usepackage{mathtools} %% for pretty underbrace % Breaks Ximera
%\usepackage{multicol}
\usepackage{pgffor} %% required for integral for loops



%% http://tex.stackexchange.com/questions/66490/drawing-a-tikz-arc-specifying-the-center
%% Draws beach ball
\tikzset{pics/carc/.style args={#1:#2:#3}{code={\draw[pic actions] (#1:#3) arc(#1:#2:#3);}}}



\usepackage{array}
\setlength{\extrarowheight}{+.1cm}
\newdimen\digitwidth
\settowidth\digitwidth{9}
\def\divrule#1#2{
\noalign{\moveright#1\digitwidth
\vbox{\hrule width#2\digitwidth}}}






\DeclareMathOperator{\arccot}{arccot}
\DeclareMathOperator{\arcsec}{arcsec}
\DeclareMathOperator{\arccsc}{arccsc}

















%%This is to help with formatting on future title pages.
\newenvironment{sectionOutcomes}{}{}





\title{$\Delta$ vs. $d$}

\begin{document}
\begin{abstract}
Absolute vs. approximate
\end{abstract}
\maketitle





\section{New and old friends}


Let $f(x)$ be a function and $L(x)$ a linearization of $f(x)$ at some domain number.

You might be wondering, given a plot of $y=f(x)$, and a plot of the tangent line $y=L(x)$



\begin{quote}
  What's the difference between $\Delta x$ and $dx$? 
\end{quote}


\begin{quote}
  What about $\Delta y$ and $dy$?
\end{quote}




Regardless, it is now a pressing question. Here's the deal: 
\[
\frac{\Delta y}{\Delta x}
\]
is the \textbf{average rate of change} of $y=f(x)$ with respect to $x$.
On the other hand:
\[
\frac{dy}{dx}
\]
is the \textbf{instantaneous rate of change} of $y=f(x)$ with respect to
$x$. Essentially, $\Delta x$ and $dx$ are the same type of thing,
they are (usually small) changes in $x$. However, $\Delta y$ and $dy$ are very different things.




\begin{itemize}
\item $\Delta y=f(x+\Delta x)-f(x)$; it  is the change in $y=f(x)$ associated to $\Delta x$.
\item $dy=L(x+dx)-L(x)$, it is the change in $y=L(x)$ associated to $\Delta x=dx$.
  \[
  dy =f'(x)dx
  \]
  Note: $ L(x+dx)= f(x)+f'(x)dx$.
   
  So, the change
  \begin{align*}
    dy &= L(x+dx)-L(x)\\
    &= f(x)+f'(x)dx-L(x)\\
    &= f(x)+f'(x)dx-f(x)\\
    &=f'(x)dx
  \end{align*}
\end{itemize}











\begin{image}
%\begin{marginfigure}[0in]
\begin{tikzpicture}
  \begin{axis}[
            xmin=1, xmax=2, range=0:6,ymax=6,ymin=0,
            axis lines =left, xlabel=$x$, ylabel=$y$,
            every axis y label/.style={at=(current axis.above origin),anchor=south},
            every axis x label/.style={at=(current axis.right of origin),anchor=west},
            ticks=none,
            axis on top,
          ]         
          \addplot [draw=black,dashed,->,>=stealth'] plot coordinates {(1.4,10/6) (1.7,10/6)};
          \addplot [draw=black,dashed,->,>=stealth'] plot coordinates {(1.7,10/6) (1.7,10/6 +.3/.36)};
          \addplot [very thick,penColor, smooth,samples=100,domain=(0:1.833)] {-1/(x-2)};
          \addplot [very thick,penColor2, smooth,samples=100,domain=(0:1.833)] {1/0.6+(1/0.36)*(x-1.4)};
          \addplot[color=penColor,fill=penColor,only marks,mark=*] coordinates{(1.7,1/0.3)};  %% closed hole  
          \addplot[color=penColor3,fill=penColor3,only marks,mark=*] coordinates{(1.7,10/6 +.3/.36)};  %% closed hole  
          \addplot[color=penColor3,fill=penColor3,only marks,mark=*] coordinates{(1.4,10/6)};  %% closed hole   

          \addplot [draw=black,dashed,->,>=stealth'] plot coordinates {(1.8,10/6) (1.8,3.6)};


          \node at (axis cs:1.55,1.67) [below] {$dx = \Delta x$};
          \node at (axis cs:1.8,3) [right] {$\Delta y$};
          \node at (axis cs:1.7,2.08) [right] {$df$};
          \node at (axis cs:1.15,1.85) [below] {$f$};
          \node at (axis cs:1.05,0.55) [right] {$L$};
          \node at (axis cs:1.24,2.0) [right] {$(x,f(x))$};
          \node at (axis cs:1.26,3.6) [right] {$(x+dx,f(x+dx))$};
          
        \end{axis}
\end{tikzpicture}
%\caption{While $dy$ and $\d x$ are both variables, $dy$ depends on $\d x$,
  %and approximates how much a function grows after a change of size
  %$\d x$ from a given point.}
%\label{figure:differentials}
%\end{marginfigure}
\end{image}












\begin{question}
  Suppose $f(x) = x^2$. If we are at the point $x=1$ and $\Delta x =dx
  = 0.1$, what is $\Delta y$? What is $dy$?
  \begin{hint}
 $ \Delta y=f(1+\Delta x)-f(1)=f(1.1)-f(1)$
    \end{hint}
      \begin{hint}
 $dy=f'(1)\cdot dx=f'(1)\cdot0.1$
    \end{hint}
  \begin{prompt}
    \[
    \Delta y = \answer{0.21}\qquad dy = \answer{0.2}
    \]
  \end{prompt}
\end{question}
Differentials can be confusing at first. However, when you master
them, you will have a powerful tool at your disposal.










\begin{center}
\textbf{\textcolor{green!50!black}{ooooo=-=-=-=-=-=-=-=-=-=-=-=-=ooOoo=-=-=-=-=-=-=-=-=-=-=-=-=ooooo}} \\

more examples can be found by following this link\\ \link[More Examples of Linearization]{https://ximera.osu.edu/csccmathematics/precalculus2/precalculus2/linearization/examples/exampleList}

\end{center}

\end{document}
