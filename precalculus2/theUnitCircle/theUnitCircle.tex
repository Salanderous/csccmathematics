\documentclass{ximera}


\graphicspath{
  {./}
  {ximeraTutorial/}
  {basicPhilosophy/}
}

\newcommand{\mooculus}{\textsf{\textbf{MOOC}\textnormal{\textsf{ULUS}}}}

\usepackage{tkz-euclide}\usepackage{tikz}
\usepackage{tikz-cd}
\usetikzlibrary{arrows}
\tikzset{>=stealth,commutative diagrams/.cd,
  arrow style=tikz,diagrams={>=stealth}} %% cool arrow head
\tikzset{shorten <>/.style={ shorten >=#1, shorten <=#1 } } %% allows shorter vectors

\usetikzlibrary{backgrounds} %% for boxes around graphs
\usetikzlibrary{shapes,positioning}  %% Clouds and stars
\usetikzlibrary{matrix} %% for matrix
\usepgfplotslibrary{polar} %% for polar plots
\usepgfplotslibrary{fillbetween} %% to shade area between curves in TikZ
\usetkzobj{all}
\usepackage[makeroom]{cancel} %% for strike outs
%\usepackage{mathtools} %% for pretty underbrace % Breaks Ximera
%\usepackage{multicol}
\usepackage{pgffor} %% required for integral for loops



%% http://tex.stackexchange.com/questions/66490/drawing-a-tikz-arc-specifying-the-center
%% Draws beach ball
\tikzset{pics/carc/.style args={#1:#2:#3}{code={\draw[pic actions] (#1:#3) arc(#1:#2:#3);}}}



\usepackage{array}
\setlength{\extrarowheight}{+.1cm}
\newdimen\digitwidth
\settowidth\digitwidth{9}
\def\divrule#1#2{
\noalign{\moveright#1\digitwidth
\vbox{\hrule width#2\digitwidth}}}






\DeclareMathOperator{\arccot}{arccot}
\DeclareMathOperator{\arcsec}{arcsec}
\DeclareMathOperator{\arccsc}{arccsc}

















%%This is to help with formatting on future title pages.
\newenvironment{sectionOutcomes}{}{}


\title{The Key}

\begin{document}

\begin{abstract}
circles and triangles
\end{abstract}
\maketitle





\textbf{\textcolor{red!80!black}{All}}, \textbf{\textcolor{red!80!black}{Every}}, \textbf{\textcolor{red!80!black}{None}} - These are the words of mathematics. 



\begin{itemize}
\item The position of \textbf{\textcolor{red!80!black}{every}} point in the Cartesian plane can be described as a scalar times a direction vector.  

\item \textbf{\textcolor{red!80!black}{Every}} Complex number can be represented as the product of a real number and a complex number with modulus $1$.


\item \textbf{\textcolor{red!80!black}{Every}} Complex number can be represented as the product of a real number and a complex number that sits on the unit circle.
\end{itemize}








\begin{image}
\begin{tikzpicture}
  \begin{axis}[
            xmin=-1.1,xmax=1.1,ymin=-1.1,ymax=1.1,
            axis lines=center,
            width=4in,
            xtick={-1,1},
            ytick={-1,1},
            clip=false,
            unit vector ratio*=1 1 1,
            xlabel=$x$, ylabel=$y$,
            every axis y label/.style={at=(current axis.above origin),anchor=south},
            every axis x label/.style={at=(current axis.right of origin),anchor=west},
          ]        
          \addplot [smooth, domain=(0:360)] ({cos(x)},{sin(x)}); %% unit circle

          \addplot [textColor] plot coordinates {(0,0) (0.766,0.643)}; %% 40 degrees

          \addplot [ultra thick,penColor] plot coordinates {(0.766,0) (0.766,0.643)}; %% 40 degrees
          \addplot [ultra thick,penColor2] plot coordinates {(0,0) (0.766,0)}; %% 40 degrees
          
          %\addplot [ultra thick,penColor3] plot coordinates {(1,0) (1,.839)}; %% 40 degrees          

          \addplot [textColor,smooth, domain=(0:40)] ({0.15*cos(x)},{0.15*sin(x)});
          %\addplot [very thick,penColor] plot coordinates {(0,0) (.766,.643)}; %% sector
          %\addplot [very thick,penColor] plot coordinates {(0,0) (1,0)}; %% sector
          %\addplot [very thick, penColor, smooth, domain=(0:40)] ({cos(x)},{sin(x)}); %% sector
          \node at (axis cs:0.15,0.07) [anchor=west] {$\theta$};
          \node[penColor, rotate=-90] at (axis cs:0.84,0.322) {$\sin(\theta)$};
          \node[penColor2] at (axis cs:0.383,0) [anchor=north] {$\cos(\theta)$};
          %\node[penColor3, rotate=-90] at (axis cs:1.06,.322) {$\tan(\theta)$};

          \addplot[color=black,fill=black,only marks,mark=*] coordinates{(0.766,0.643)};


        \end{axis}
\end{tikzpicture}
\end{image}



If we can understand the unit circle in terms of Complex numbers, then we will have a good understanding of all Complex numbers.




Pretend we have two Complex numbers on the unit circle.

\begin{itemize}
\item $C_1 = a + b \, i$ with $|C_1| = 1$
\item $C_2 = c + d \, i$ with $|C_2| = 1$
\end{itemize}



We have $a^2 + b^1 = 1$  and $c^2 + d^2 = 1$.



Now, let's examine their product.






\begin{align*}
(a + b \, i) \cdot (c + d \, i)      & = ac + ad \, i + bc \, i - bd   \\
                & = (ac - bd) + (ad + bd) \, i
\end{align*}



What is the modulus of the product?





\begin{align*}
|(a + b \, i) \cdot (c + d \, i)|      & = (ac - bd)^2 + (ad + bd)^2   \\
                & = a^2c^2 - 2abcd + b^2d^2 + a^2d^2 + 2abcd + b^2c^2  \\
                & = a^2c^2 + b^2d^2 + a^2d^2 + b^2c^2  \\
                & = (a^2 + b^2) \cdot (c^2 + d^2)   \\
                & = 1 \cdot 1 = 1
\end{align*}



The product is again on the unit circle.


\textbf{Question:} Where is the product on the unit circle?




\section{Polar Perspective}



The Complex number $a + b \, i$, is positioned at an angle $\theta$ from teh positive $x$-axis and a distance $r_1$ from the origin. Therefore, it has the polar form $(r_1, \theta)$.  If  $a + b \, i$ is on the unit circle, then $r_1 = 1$ and the polar form is $(1, \theta)$.




This point defines a right triangle with base $a$ and height $b$ and angle $\theta$ .  









\begin{image}
\begin{tikzpicture}
  \begin{axis}[
            xmin=-1.1,xmax=1.1,ymin=-1.1,ymax=1.1,
            axis lines=center,
            width=4in,
            xtick={-1,1},
            ytick={-1,1},
            clip=false,
            unit vector ratio*=1 1 1,
            xlabel=$x$, ylabel=$y$,
            every axis y label/.style={at=(current axis.above origin),anchor=south},
            every axis x label/.style={at=(current axis.right of origin),anchor=west},
          ]        
          \addplot [smooth, domain=(0:360)] ({cos(x)},{sin(x)}); %% unit circle

          \addplot [textColor] plot coordinates {(0,0) (.766,.643)}; %% 40 degrees

          \addplot [ultra thick,penColor] plot coordinates {(.766,0) (.766,.643)}; %% 40 degrees
          \addplot [ultra thick,penColor2] plot coordinates {(0,0) (.766,0)}; %% 40 degrees
          
          %\addplot [ultra thick,penColor3] plot coordinates {(1,0) (1,.839)}; %% 40 degrees          

          \addplot [textColor,smooth, domain=(0:40)] ({.15*cos(x)},{.15*sin(x)});
          %\addplot [very thick,penColor] plot coordinates {(0,0) (.766,.643)}; %% sector
          %\addplot [very thick,penColor] plot coordinates {(0,0) (1,0)}; %% sector
          %\addplot [very thick, penColor, smooth, domain=(0:40)] ({cos(x)},{sin(x)}); %% sector
          \node at (axis cs:.15,.07) [anchor=west] {$\theta$};
          \node[penColor] at (axis cs:.84,.322) {$b$};
          \node[penColor2] at (axis cs:.383,0) [anchor=north] {$a$};
          %\node[penColor3, rotate=-90] at (axis cs:1.06,.322) {$\tan(\theta)$};
        \end{axis}
\end{tikzpicture}
\end{image}





Let's focus on this triangle.




The Complex number $c + d \, i$, has a polar form $(r_2, \phi)$, which similarly is $(1, \phi)$.  Normally, we would plot a point for this complex number by rotating counterclockwise and angle from the positive $x$-axis.  Instead, we are going to tack on this angle to $\theta$.

So, we are sort of pretending that the radius for the first complex number, $a + b \, i$ is acting like an $x$-axis and we are going to roatate counterclockwise an angle $\phi$ from that radius.





And, we begin making a right triangle for   $c + d \, i$.





\begin{image}[3in]
    \begin{tikzpicture}

  %\draw [ultra thick] (0,0) -- (0,3);
  \draw [ultra thick] (0,0) -- (3,0);
  \draw [ultra thick,->] (3,0) -- (3,3);
  %\draw [ultra thick] (0,3) -- (3,3);

  \draw [ultra thick] (0,0) -- (3,1);
  \draw [ultra thick,->] (3,1) -- (2.5,2);
  \draw [ultra thick,->] (0,0) -- (1.6,2.4);

  \draw [thin] (2.9,0) -- (2.9,0.1);
  \draw [thin] (2.9,0.1) -- (3,0.1);

  \draw [thin] (2.9,0.97) -- (2.85,1.06);
  \draw [thin] (2.85,1.06) -- (2.95,1.10);




  \draw [->] (.75,0) arc(0:18.4:.75); 
  \draw [rotate=9] (1,0) node {\scriptsize{$\theta$}};
  \draw [->] (.75,0) arc(0:55:.75); 
  \draw [rotate=36] (1,0) node {\scriptsize{$\phi$}};


  \draw (1.5,-0.25) node {\scriptsize{$a$}};
  \draw (3.25,0.5) node {\scriptsize{$b$}};
  \draw (1.5,0.75) node {\scriptsize{$1$}};


  %\draw (2.9,1.5) node {\scriptsize{$\theta$}};






    \end{tikzpicture}
  \end{image}




Eventually, the sides meet to make a right triangle.  Let's enclose the whole diagram with a rectangle at that height.




The problem is that this second right triangle is too big. It is supposed to be a right triangle for $c + d \, i$.  That means the sides were suppose to be $c$ and $d$.  But, in our diagram, the bottom side has length $1$.  However, it is a right triangle with angle $\phi$.  That means it is similar to the right triangle for $c + d \, i$.  The sides are porportional to the sides of the $c-d-1$ right triangle for $c + d \, i$.  We can use this proportionality.





\begin{image}[3in]
    \begin{tikzpicture}

  \draw [ultra thick] (0,0) -- (0,3);
  \draw [ultra thick] (0,0) -- (3,0);
  \draw [ultra thick] (3,0) -- (3,3);
  \draw [ultra thick] (0,3) -- (3,3);

  \draw [ultra thick] (0,0) -- (3,1);
  \draw [ultra thick] (3,1) -- (2,3);
  \draw [ultra thick] (0,0) -- (2,3);

  \draw [thin] (2.9,0) -- (2.9,0.1);
  \draw [thin] (2.9,0.1) -- (3,0.1);

  \draw [thin] (2.9,0.97) -- (2.85,1.06);
  \draw [thin] (2.85,1.06) -- (2.95,1.10);




  \draw [->] (.75,0) arc(0:18.4:.75); 
  \draw [rotate=9] (1,0) node {\scriptsize{$\theta$}};
  \draw [->] (.75,0) arc(0:55:.75); 
  \draw [rotate=36] (1,0) node {\scriptsize{$\phi$}};


  \draw (1.5,-0.25) node {\scriptsize{$a$}};
  \draw (3.25,0.5) node {\scriptsize{$b$}};
  \draw (1.5,0.75) node {\scriptsize{$1$}};


  %\draw (2.9,1.5) node {\scriptsize{$\theta$}};






    \end{tikzpicture}
  \end{image}



Looking at this second right triangle for $c + d \, i$, it's sides are not $c$ and $d$.  It is a similar triangle with $c$ resized to $1$.  That makes the maginification factor $\frac{1}{c}$.  The other side is scaled by that factor as well.  The other leg has length $\frac{d}{c}$








\begin{image}[3in]
    \begin{tikzpicture}

  \draw [ultra thick] (0,0) -- (0,3);
  \draw [ultra thick] (0,0) -- (3,0);
  \draw [ultra thick] (3,0) -- (3,3);
  \draw [ultra thick] (0,3) -- (3,3);

  \draw [ultra thick] (0,0) -- (3,1);
  \draw [ultra thick] (3,1) -- (2,3);
  \draw [ultra thick] (0,0) -- (2,3);

  \draw [thin] (2.9,0) -- (2.9,0.1);
  \draw [thin] (2.9,0.1) -- (3,0.1);

  \draw [thin] (2.9,0.97) -- (2.85,1.06);
  \draw [thin] (2.85,1.06) -- (2.95,1.10);




  \draw [->] (.75,0) arc(0:18.4:.75); 
  \draw [rotate=9] (1,0) node {\scriptsize{$\theta$}};
  \draw [->] (.75,0) arc(0:55:.75); 
  \draw [rotate=36] (1,0) node {\scriptsize{$\phi$}};


  \draw (1.5,-0.25) node {\scriptsize{$a$}};
  \draw (3.25,0.5) node {\scriptsize{$b$}};
  \draw (1.5,0.75) node {\scriptsize{$1$}};


  %\draw (2.9,1.5) node {\scriptsize{$\theta$}};
  \draw (2.6,2.2) node {\tiny{$\tfrac{d}{c}$}};






    \end{tikzpicture}
  \end{image}








We can also tell something about another angle in this diagram. We can get the measurement for the bottom angle in the right triangle for $\tfrac{d}{c}$. We'll call this angle $A$ in the digram below.

\begin{itemize}
  \item The bottom right triangle has a right angle, the angle $\theta$, and angle $B$.
  \item The right side of the rectangle forms a straight angle and it is cut up into three angles: a right angle, $B$ on one side, and $A$ on the other side.
\end{itemize}






\begin{image}[3in]
    \begin{tikzpicture}

  \draw [ultra thick] (0,0) -- (0,3);
  \draw [ultra thick] (0,0) -- (3,0);
  \draw [ultra thick] (3,0) -- (3,3);
  \draw [ultra thick] (0,3) -- (3,3);

  \draw [ultra thick] (0,0) -- (3,1);
  \draw [ultra thick] (3,1) -- (2,3);
  \draw [ultra thick] (0,0) -- (2,3);

  \draw [thin] (2.9,0) -- (2.9,0.1);
  \draw [thin] (2.9,0.1) -- (3,0.1);

  \draw [thin] (2.9,0.97) -- (2.85,1.06);
  \draw [thin] (2.85,1.06) -- (2.95,1.10);




  \draw [->] (.75,0) arc(0:18.4:.75); 
  \draw [rotate=9] (1,0) node {\scriptsize{$\theta$}};
  \draw [->] (.75,0) arc(0:55:.75); 
  \draw [rotate=36] (1,0) node {\scriptsize{$\phi$}};


  \draw (1.5,-0.25) node {\scriptsize{$a$}};
  \draw (3.25,0.5) node {\scriptsize{$b$}};
  \draw (1.5,0.75) node {\scriptsize{$1$}};


  \draw (2.9,1.6) node {\scriptsize{$A$}};
  \draw (2.85,0.75) node {\scriptsize{$B$}};
  \draw (2.6,2.2) node {\tiny{$\tfrac{d}{c}$}};






    \end{tikzpicture}
  \end{image}




The sum of the angles of the right triangle is $\theta + B + 90^{\circ} = 180^{\circ}$ and the angles forming the straight angle sum to $\theta + A + 90^{\circ} = 180^{\circ}$.  




\[
\theta + B + 90^{\circ}   =   B + A + 90^{\circ}
\]



\[  \theta = A  \]





















We now have a pair of similar right triangles.  The bottom right triangle representing $a+b \, i$ and the triangle formed in the upper-right corner of the rectangle.

Comparing the hypotnueses, we that the scaling factor is $\frac{d}{c}$. Therefore, the other sides must also be resized proportionally.










\begin{image}[3in]
    \begin{tikzpicture}

  \draw [ultra thick] (0,0) -- (0,3);
  \draw [ultra thick] (0,0) -- (3,0);
  \draw [ultra thick] (3,0) -- (3,3);
  \draw [ultra thick] (0,3) -- (3,3);

  \draw [ultra thick] (0,0) -- (3,1);
  \draw [ultra thick] (3,1) -- (2,3);
  \draw [ultra thick] (0,0) -- (2,3);

  \draw [thin] (2.9,0) -- (2.9,0.1);
  \draw [thin] (2.9,0.1) -- (3,0.1);

  \draw [thin] (2.9,0.97) -- (2.85,1.06);
  \draw [thin] (2.85,1.06) -- (2.95,1.10);




  \draw [->] (.75,0) arc(0:18.4:.75); 
  \draw [rotate=9] (1,0) node {\scriptsize{$\theta$}};
  \draw [->] (.75,0) arc(0:55:.75); 
  \draw [rotate=36] (1,0) node {\scriptsize{$\phi$}};


  \draw (1.5,-0.25) node {\scriptsize{$a$}};
  \draw (3.25,0.5) node {\scriptsize{$b$}};
  \draw (1.5,0.75) node {\scriptsize{$1$}};


  \draw (2.9,1.5) node {\scriptsize{$\theta$}};
  \draw (2.6,2.2) node {\tiny{$\tfrac{d}{c}$}};

  \draw (3.25,2.2) node {\tiny{$\tfrac{da}{c}$}};
  \draw (2.6,3.2) node {\tiny{$\tfrac{db}{c}$}};






    \end{tikzpicture}
  \end{image}






We can now deduce the lengths of the other two pieces of the rectangle in terms of $a$, $b$, $c$, and $d$.







\begin{image}[3in]
    \begin{tikzpicture}

  \draw [ultra thick] (0,0) -- (0,3);
  \draw [ultra thick] (0,0) -- (3,0);
  \draw [ultra thick] (3,0) -- (3,3);
  \draw [ultra thick] (0,3) -- (3,3);

  \draw [ultra thick] (0,0) -- (3,1);
  \draw [ultra thick] (3,1) -- (2,3);
  \draw [ultra thick] (0,0) -- (2,3);

  \draw [thin] (2.9,0) -- (2.9,0.1);
  \draw [thin] (2.9,0.1) -- (3,0.1);

  \draw [thin] (2.9,0.97) -- (2.85,1.06);
  \draw [thin] (2.85,1.06) -- (2.95,1.10);




  \draw [->] (.75,0) arc(0:18.4:.75); 
  \draw [rotate=9] (1,0) node {\scriptsize{$\theta$}};
  \draw [->] (.75,0) arc(0:55:.75); 
  \draw [rotate=36] (1,0) node {\scriptsize{$\phi$}};


  \draw (1.5,-0.25) node {\scriptsize{$a$}};
  \draw (3.25,0.5) node {\scriptsize{$b$}};
  \draw (1.5,0.75) node {\scriptsize{$1$}};


  \draw (2.9,1.5) node {\scriptsize{$\theta$}};
  \draw (2.6,2.2) node {\tiny{$\tfrac{d}{c}$}};

  \draw (3.25,2.2) node {\tiny{$\tfrac{da}{c}$}};
  \draw (2.6,3.2) node {\tiny{$\tfrac{db}{c}$}};

  \draw (-0.5,1.5) node {\tiny{b+$\tfrac{da}{c}$}};
  \draw (1,3.2) node {\tiny{$a-\tfrac{db}{c}$}};






    \end{tikzpicture}
  \end{image}







Let's focus our attention on the right triangle formed in the upper-left corner of the rectangle.

The bottom angle is the sum of $\theta$ and $\phi$.

The upper angle must measure $\theta + \phi$ also since it is an alternate interior angle.




\begin{image}[3in]
    \begin{tikzpicture}

  \draw [ultra thick] (0,0) -- (0,3);
  \draw [ultra thick] (0,0) -- (3,0);
  \draw [ultra thick] (0,3) -- (3,3);



  
  \draw [ultra thick] (0,0) -- (2,3);

 





  \draw [->] (.75,0) arc(0:55:.75); 
  \draw [rotate=18] (1.1,0) node {\tiny{$\theta + \phi$}};
  \draw (1.25,2.75) node {\tiny{$\theta + \phi$}};




  \draw (-0.5,1.5) node {\tiny{b+$\tfrac{da}{c}$}};
  \draw (1,3.2) node {\tiny{$a-\tfrac{db}{c}$}};






    \end{tikzpicture}
  \end{image}




Finally, let's resize this triangle by a factor of $c$.









\begin{image}[3in]
    \begin{tikzpicture}

  \draw [ultra thick] (0,0) -- (0,3);
  \draw [ultra thick] (0,0) -- (3,0);
  \draw [ultra thick] (0,3) -- (3,3);



  
  \draw [ultra thick] (0,0) -- (2,3);

 





  \draw [->] (.75,0) arc(0:55:.75); 
  \draw [rotate=18] (1.1,0) node {\tiny{$\theta + \phi$}};
  \draw (1.25,2.75) node {\tiny{$\theta + \phi$}};




  \draw (-0.5,1.5) node {\tiny{$ac-bd$}};
  \draw (1,3.2) node {\tiny{$ad+bc$}};






    \end{tikzpicture}
  \end{image}





It is upsidedown, but this is the right trinagle defined on the unit circle by the product

\[  (a + b \, i) \cdot (c + d \, i) = (ac-bd) + (ad+bc) \, i    \]




 

\textcolor{red!80!black}{$\blacktriangleright$} When multiplying Complex numbers on the unit circle, just add their angles.

\textcolor{red!80!black}{$\blacktriangleright$} When multiplying any Complex numbers, multiply their moduli and add their angles.



\begin{theorem}  \textbf{\textcolor{green!50!black}{Complex Multiplication (Polar)}}

\[  (r_1, \theta_1) \cdot (r_2, \theta_2) = (r_1 \cdot r_2, \theta_1 + \theta_2)              \]


\end{theorem}























\end{document}
