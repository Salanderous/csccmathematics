\documentclass{ximera}


\graphicspath{
  {./}
  {ximeraTutorial/}
  {basicPhilosophy/}
}

\newcommand{\mooculus}{\textsf{\textbf{MOOC}\textnormal{\textsf{ULUS}}}}

\usepackage{tkz-euclide}\usepackage{tikz}
\usepackage{tikz-cd}
\usetikzlibrary{arrows}
\tikzset{>=stealth,commutative diagrams/.cd,
  arrow style=tikz,diagrams={>=stealth}} %% cool arrow head
\tikzset{shorten <>/.style={ shorten >=#1, shorten <=#1 } } %% allows shorter vectors

\usetikzlibrary{backgrounds} %% for boxes around graphs
\usetikzlibrary{shapes,positioning}  %% Clouds and stars
\usetikzlibrary{matrix} %% for matrix
\usepgfplotslibrary{polar} %% for polar plots
\usepgfplotslibrary{fillbetween} %% to shade area between curves in TikZ
\usetkzobj{all}
\usepackage[makeroom]{cancel} %% for strike outs
%\usepackage{mathtools} %% for pretty underbrace % Breaks Ximera
%\usepackage{multicol}
\usepackage{pgffor} %% required for integral for loops



%% http://tex.stackexchange.com/questions/66490/drawing-a-tikz-arc-specifying-the-center
%% Draws beach ball
\tikzset{pics/carc/.style args={#1:#2:#3}{code={\draw[pic actions] (#1:#3) arc(#1:#2:#3);}}}



\usepackage{array}
\setlength{\extrarowheight}{+.1cm}
\newdimen\digitwidth
\settowidth\digitwidth{9}
\def\divrule#1#2{
\noalign{\moveright#1\digitwidth
\vbox{\hrule width#2\digitwidth}}}






\DeclareMathOperator{\arccot}{arccot}
\DeclareMathOperator{\arcsec}{arcsec}
\DeclareMathOperator{\arccsc}{arccsc}

















%%This is to help with formatting on future title pages.
\newenvironment{sectionOutcomes}{}{}




\title{Cooling}

\begin{document}
\begin{abstract}
  revisiting Newton's Law
\end{abstract}
\maketitle




In modern notation, Newton's Law of Cooling states


\[
\frac{dT}{dt} = r \, (T_{env} - T(t))
\]

where $r$ is the coefficient of heat transfer.\\





The derivative (rate of change) of temperature is proportional to the difference betweent the current temperature and the currounding temperature.




If we have an initial temperature, then we can calculate a value for the derivative, which means we can approximate close values of the temperature function.  In this way, we can piece together an approximation of the temperature function.


\begin{procedure}  \textbf{\textcolor{blue!55!black}{Euler's Method}} \\

Given a differential equation of the form
\begin{center}
$f^{\prime}(x) =$ formula invovling $f$ and $x$.
\end{center}

and an intial value of $f$, $f(0) = c$. \\


Decide on an increment for $x$, $\Delta x$.

\begin{enumerate}[label=(\arabic*)]
\item $f(0) = c$
\item Using $f(0) = c$, get a value for $f^{\prime}(0)$
\item Using $f(0) = c$ and $f^{\prime}(0)$ approximate a value for $f(\Delta x)$
\item Using your value for $f(\Delta x)$, get a value for $f^{\prime}(\Delta x)$
\item Using $f(\Delta x)$ and $f^{\prime}(\Delta x)$ approximate a value for $f(2 \Delta x)$
\item Similarly get an approximation for $f(3 \Delta x)$, then approximate $f(4 \Delta x)$, etc
\end{enumerate}

This is slowly develop a table of approximate values for $f$.

\end{procedure}

We did one step of this procedure in the last section. \\


However, we can do better in our example from the last section, by using smaller increments and reusing the differential equation to give us an updated rate.






\begin{example} Law of Cooling 



We can do better. \\



A freshly brewed pot of tea is removed from the stove at $200^{\circ}$.  It is set on the table where the room temperature is $72^{\circ}$ and the heat transfer coefficient is $0.025$. \\


Newton's Law of Cooling says

\[
\frac{dT}{dt} = 0.025 \, (72 - T(t))
\]


where $t$ is measured in minutes.


$\blacktriangleright$ \textbf{First Minute} \\

The tea has an initial temperature of $200^{\circ}$, which means the temperature is changing approximately at a rate of 


\[
\frac{dT}{dt} = 0.025 \, (72 - 200) = 3.2 \, \text{degrees per minute}
\]


After $30$ seconds or $0.5$ minute the tea is at approximately $200^{\circ} - 3.2 \frac{\text{degrees}}{\text{min}} \cdot 0.5 \text{min} = $.


The new rate of cooling at the $30$ second mark is


\[
\frac{dT}{dt} = 0.025 \, (72 - 198.4) = 3.16 \, \, \text{degrees per minute}
\]



After another $30$ seconds or now $1$ minute from teh beginning, the tea is at approximately $198.4^{\circ} - 3.16 \frac{\text{degrees}}{\text{min}} \cdot 0.5 \text{min} = 196.82^{\circ}$.


In the previous section we arrived at $196.8^{\circ}$.

\end{example}





\begin{observation}

Our revised calculation is higher than the previous.  That is because the rate decreases over time.  In the example in the previous section, we used a higher rate over a longer time interval.

This time we revised our rate and it was lower during the second $30$-second period.  Thus, the tea did not cool as much as before.
\end{observation}











\end{document}
