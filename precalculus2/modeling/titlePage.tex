\documentclass{ximera}


\graphicspath{
  {./}
  {ximeraTutorial/}
  {basicPhilosophy/}
}

\newcommand{\mooculus}{\textsf{\textbf{MOOC}\textnormal{\textsf{ULUS}}}}

\usepackage{tkz-euclide}\usepackage{tikz}
\usepackage{tikz-cd}
\usetikzlibrary{arrows}
\tikzset{>=stealth,commutative diagrams/.cd,
  arrow style=tikz,diagrams={>=stealth}} %% cool arrow head
\tikzset{shorten <>/.style={ shorten >=#1, shorten <=#1 } } %% allows shorter vectors

\usetikzlibrary{backgrounds} %% for boxes around graphs
\usetikzlibrary{shapes,positioning}  %% Clouds and stars
\usetikzlibrary{matrix} %% for matrix
\usepgfplotslibrary{polar} %% for polar plots
\usepgfplotslibrary{fillbetween} %% to shade area between curves in TikZ
\usetkzobj{all}
\usepackage[makeroom]{cancel} %% for strike outs
%\usepackage{mathtools} %% for pretty underbrace % Breaks Ximera
%\usepackage{multicol}
\usepackage{pgffor} %% required for integral for loops



%% http://tex.stackexchange.com/questions/66490/drawing-a-tikz-arc-specifying-the-center
%% Draws beach ball
\tikzset{pics/carc/.style args={#1:#2:#3}{code={\draw[pic actions] (#1:#3) arc(#1:#2:#3);}}}



\usepackage{array}
\setlength{\extrarowheight}{+.1cm}
\newdimen\digitwidth
\settowidth\digitwidth{9}
\def\divrule#1#2{
\noalign{\moveright#1\digitwidth
\vbox{\hrule width#2\digitwidth}}}






\DeclareMathOperator{\arccot}{arccot}
\DeclareMathOperator{\arcsec}{arcsec}
\DeclareMathOperator{\arccsc}{arccsc}

















%%This is to help with formatting on future title pages.
\newenvironment{sectionOutcomes}{}{}


\title{Modeling}

\begin{document}

\begin{abstract}
%%%
\end{abstract}
\maketitle





We encounter situations everyday in which we extract measurements, either exactly or approximately. We then compare these measurements and draw conclusions.  With enough conclusions, we develop plans and proceed to discover more details.


Comparing measurements is at the heart of any investigation.

Comparing measurements requires us to notice relationships within the situation.  There is no way to learn steps that systematically lead to the proper relationship in any situation.  Instead, you have a roldex of known relationships in your head and you attempt to ``see'' these in the situation.



Our tools for describing and comparing measurements are functions.  And, when used in this fashion, we refer to them as \textbf{models}.

























\subsection{Learning Outcomes}


\begin{sectionOutcomes}
In this section, students will 

\begin{itemize}
\item find relationships in situations.
\end{itemize}
\end{sectionOutcomes}














\begin{center}
\textbf{\textcolor{green!50!black}{ooooo=-=-=-=-=-=-=-=-=-=-=-=-=ooOoo=-=-=-=-=-=-=-=-=-=-=-=-=ooooo}} \\

more examples can be found by following this link\\ \link[More Examples of Modeling]{https://ximera.osu.edu/csccmathematics/precalculus2/precalculus2/modeling/examples/exampleList}

\end{center}









\end{document}
