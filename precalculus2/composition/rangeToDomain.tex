\documentclass{ximera}


\graphicspath{
  {./}
  {ximeraTutorial/}
  {basicPhilosophy/}
}

\newcommand{\mooculus}{\textsf{\textbf{MOOC}\textnormal{\textsf{ULUS}}}}

\usepackage{tkz-euclide}\usepackage{tikz}
\usepackage{tikz-cd}
\usetikzlibrary{arrows}
\tikzset{>=stealth,commutative diagrams/.cd,
  arrow style=tikz,diagrams={>=stealth}} %% cool arrow head
\tikzset{shorten <>/.style={ shorten >=#1, shorten <=#1 } } %% allows shorter vectors

\usetikzlibrary{backgrounds} %% for boxes around graphs
\usetikzlibrary{shapes,positioning}  %% Clouds and stars
\usetikzlibrary{matrix} %% for matrix
\usepgfplotslibrary{polar} %% for polar plots
\usepgfplotslibrary{fillbetween} %% to shade area between curves in TikZ
\usetkzobj{all}
\usepackage[makeroom]{cancel} %% for strike outs
%\usepackage{mathtools} %% for pretty underbrace % Breaks Ximera
%\usepackage{multicol}
\usepackage{pgffor} %% required for integral for loops



%% http://tex.stackexchange.com/questions/66490/drawing-a-tikz-arc-specifying-the-center
%% Draws beach ball
\tikzset{pics/carc/.style args={#1:#2:#3}{code={\draw[pic actions] (#1:#3) arc(#1:#2:#3);}}}



\usepackage{array}
\setlength{\extrarowheight}{+.1cm}
\newdimen\digitwidth
\settowidth\digitwidth{9}
\def\divrule#1#2{
\noalign{\moveright#1\digitwidth
\vbox{\hrule width#2\digitwidth}}}






\DeclareMathOperator{\arccot}{arccot}
\DeclareMathOperator{\arcsec}{arcsec}
\DeclareMathOperator{\arccsc}{arccsc}

















%%This is to help with formatting on future title pages.
\newenvironment{sectionOutcomes}{}{}


\title{Steps}

\begin{document}

\begin{abstract}
range to domain
\end{abstract}
\maketitle







\begin{template} \textbf{\textcolor{blue!55!black}{Composition}}  \\


Let $f$ be a function with domain $D_f$.


Let $g$ be a function with domain $D_g$.




Then the \textbf{\textcolor{green!50!black}{composition of f and g, $f \circ g$,}} is defined as

\[
(f \circ g)(x) = f(g(x))
\] 


The value of $g$ becomes a domain number for $f$.


The composition is defined on a subset of the domain of $g$.  The composition is defined at those numbers in the domain of $g$ where the value of $g$ is in the domain of $f$.




\end{template}

The values of $g$ are becoming domain numbers for $f$. \\



\begin{warning}

\begin{itemize}
\item The domain of $(f \circ g)(x) = f(g(x))$ is not the domain of $g$.
\item The domain of $(f \circ g)(x) = f(g(x))$ is a subset of the domain of $g$.
\item The domain of $(f \circ g)(x) = f(g(x))$ is consists of the domain numbers of $g$, for which the value of $g$ is in the domain of $f$.
\end{itemize}

\end{warning}










\begin{example}


Let $X(a) = -\sqrt{3 - a}$ with its natural domain.


Let $Y(b) = \ln(1 + b)$ with its natural domain.


Let $Z(c) = \frac{3}{5 - c}$ with its natural domain.



\begin{question}

Is $2$ in the domain of $X$?

\begin{multipleChoice}
\choice [correct] {Yes}
\choice {No}
\end{multipleChoice}

\end{question}





\begin{question}

Is $2$ in the domain of $Y \circ X$?

\begin{multipleChoice}
\choice {Yes}
\choice [correct] {No}
\end{multipleChoice}

\end{question}



\begin{question}

Is $2$ in the domain of $Z \circ X$?

\begin{multipleChoice}
\choice [correct] {Yes}
\choice {No}
\end{multipleChoice}

\end{question}




\end{example}























\begin{example}


Let $X(a) = -\sqrt{3 - a}$ with its natural domain.


Let $Y(b) = \ln(1 + b)$ with its natural domain.


Let $Z(c) = \frac{3}{5 - c}$ with its natural domain.



\begin{question}

Is $27$ in the domain of $Y$?

\begin{multipleChoice}
\choice [correct] {Yes}
\choice {No}
\end{multipleChoice}

\end{question}





\begin{question}

Is $27$ in the domain of $X \circ Y$?

\begin{multipleChoice}
\choice {Yes}
\choice [correct] {No}
\end{multipleChoice}

\end{question}



\begin{question}

Is $27$ in the domain of $Z \circ Y$?

\begin{multipleChoice}
\choice [correct] {Yes}
\choice {No}
\end{multipleChoice}

\end{question}




\end{example}
















\begin{example}


Let $X(a) = -\sqrt{3 - a}$ with its natural domain.


Let $Y(b) = \ln(1 + b)$ with its natural domain.


Let $Z(c) = \frac{3}{5 - c}$ with its natural domain.



\begin{question}

Is $5$ in the domain of $Z$?

\begin{multipleChoice}
\choice [correct] {Yes}
\choice {No}
\end{multipleChoice}

\end{question}





\begin{question}

Is $5$ in the domain of $X \circ Z$?

\begin{multipleChoice}
\choice {Yes}
\choice [correct] {No}
\end{multipleChoice}

\end{question}



\begin{question}

Is $5$ in the domain of $Y \circ Z$?

\begin{multipleChoice}
\choice {Yes}
\choice [correct] {No}
\end{multipleChoice}

\end{question}




\end{example}








Define $f$ and $g$ graphically as follows:



\begin{image}
\begin{tikzpicture}
    \begin{axis}[name = sinax, domain=-10:10, ymax=10, xmax=10, ymin=-10, xmin=-10,
            axis lines =center, xlabel=$x$, ylabel={$y=f(x)$}, grid = major, grid style={dashed},
            ytick={-10,-8,-6,-4,-2,2,4,6,8,10},
            xtick={-10,-8,-6,-4,-2,2,4,6,8,10},
            yticklabels={$-10$,$-8$,$-6$,$-4$,$-2$,$2$,$4$,$6$,$8$,$10$}, 
            xticklabels={$-10$,$-8$,$-6$,$-4$,$-2$,$2$,$4$,$6$,$8$,$10$},
            ticklabel style={font=\scriptsize},
            every axis y label/.style={at=(current axis.above origin),anchor=south},
            every axis x label/.style={at=(current axis.right of origin),anchor=west},
            axis on top
          ]
          
          \addplot [line width=2, penColor, smooth,samples=100,domain=(-10:-6)] ({x},{-0.5*x+1});
          \addplot [color=penColor,only marks,mark=*] coordinates{(-10,6)};
          \addplot [color=penColor,fill=background,only marks,mark=*] coordinates{(-6,4)}; 


          \addplot [line width=2, penColor, smooth,samples=100,domain=(0:5)] ({x},{0.5*x+1});
          \addplot [color=penColor,only marks,mark=*] coordinates{(0,1)};
          \addplot [color=penColor,only marks,mark=*] coordinates{(5,3.5)}; 

           

    \end{axis}
    \begin{axis}[at={(sinax.outer east)},anchor=outer west, domain=-10:10, ymax=10, xmax=10, ymin=-10, xmin=-10,
            axis lines =center, xlabel=$t$, ylabel={$y=g(t)$}, grid = major, grid style={dashed},
            ytick={-10,-8,-6,-4,-2,2,4,6,8,10},
            xtick={-10,-8,-6,-4,-2,2,4,6,8,10},
            yticklabels={$-10$,$-8$,$-6$,$-4$,$-2$,$2$,$4$,$6$,$8$,$10$}, 
            xticklabels={$-10$,$-8$,$-6$,$-4$,$-2$,$2$,$4$,$6$,$8$,$10$},
            ticklabel style={font=\scriptsize},
            every axis y label/.style={at=(current axis.above origin),anchor=south},
            every axis x label/.style={at=(current axis.right of origin),anchor=west},
            axis on top
          ]
          
          \addplot [line width=2, penColor, smooth,samples=100,domain=(-10:-6)] ({x},{-0.5*x+1});
          \addplot [color=penColor,only marks,mark=*] coordinates{(-10,6)};
          \addplot [color=penColor,fill=background,only marks,mark=*] coordinates{(-6,4)}; 


          \addplot [line width=2, penColor, smooth,samples=100,domain=(0:5)] ({x},{0.5*x+1});
          \addplot [color=penColor,only marks,mark=*] coordinates{(0,1)};
          \addplot [color=penColor,fill=background,only marks,mark=*] coordinates{(5,3.5)}; 

  \end{axis}

\end{tikzpicture}
\end{image}













\begin{center}
\textbf{\textcolor{green!50!black}{ooooo=-=-=-=-=-=-=-=-=-=-=-=-=ooOoo=-=-=-=-=-=-=-=-=-=-=-=-=ooooo}} \\

more examples can be found by following this link\\ \link[More Examples of the Elementary Library]{https://ximera.osu.edu/csccmathematics/precalculus2/precalculus2/composition/examples/exampleList}

\end{center}







\end{document}
