\documentclass{ximera}


\graphicspath{
  {./}
  {ximeraTutorial/}
  {basicPhilosophy/}
}

\newcommand{\mooculus}{\textsf{\textbf{MOOC}\textnormal{\textsf{ULUS}}}}

\usepackage{tkz-euclide}\usepackage{tikz}
\usepackage{tikz-cd}
\usetikzlibrary{arrows}
\tikzset{>=stealth,commutative diagrams/.cd,
  arrow style=tikz,diagrams={>=stealth}} %% cool arrow head
\tikzset{shorten <>/.style={ shorten >=#1, shorten <=#1 } } %% allows shorter vectors

\usetikzlibrary{backgrounds} %% for boxes around graphs
\usetikzlibrary{shapes,positioning}  %% Clouds and stars
\usetikzlibrary{matrix} %% for matrix
\usepgfplotslibrary{polar} %% for polar plots
\usepgfplotslibrary{fillbetween} %% to shade area between curves in TikZ
\usetkzobj{all}
\usepackage[makeroom]{cancel} %% for strike outs
%\usepackage{mathtools} %% for pretty underbrace % Breaks Ximera
%\usepackage{multicol}
\usepackage{pgffor} %% required for integral for loops



%% http://tex.stackexchange.com/questions/66490/drawing-a-tikz-arc-specifying-the-center
%% Draws beach ball
\tikzset{pics/carc/.style args={#1:#2:#3}{code={\draw[pic actions] (#1:#3) arc(#1:#2:#3);}}}



\usepackage{array}
\setlength{\extrarowheight}{+.1cm}
\newdimen\digitwidth
\settowidth\digitwidth{9}
\def\divrule#1#2{
\noalign{\moveright#1\digitwidth
\vbox{\hrule width#2\digitwidth}}}






\DeclareMathOperator{\arccot}{arccot}
\DeclareMathOperator{\arcsec}{arcsec}
\DeclareMathOperator{\arccsc}{arccsc}

















%%This is to help with formatting on future title pages.
\newenvironment{sectionOutcomes}{}{}


\title{Formulas}

\begin{document}

\begin{abstract}
functions
\end{abstract}
\maketitle









\begin{template} \textbf{\textcolor{blue!55!black}{Composition}}  \\


Let $f$ be a function with domain $D_f$.


Let $g$ be a function with domain $D_g$.




Then the \textbf{\textcolor{green!50!black}{composition of f and g, $f \circ g$,}} is defined as

\[
(f \circ g)(x) = f(g(x))
\] 


The value of $g$ becomes a domain number for $f$.


The composition is defined on a subset of the domain of $g$.  The composition is defined at those numbers in the domain of $g$ where the value of $g$ is in the domain of $f$.




\end{template}





Algebraically, $f(g(x))$ is our map. \\


$f(g(x))$ tells us to replace all occurrences of the variable in the formula for $f$ with the whole formula for $g$. \\




\begin{example}


Let $f(k) = (k + 5)(k - 3)$ with its natural domain.


Let $g(t) = \sqrt{3t}$ with its natural domain.



\begin{explanation}

We would like a formula for $f \circ g$. \\

$(f \circ g)(x) = f(g(x))$ \\

This tells us to replace all occurrences of the variable in $f$ with the whole formula for $g$.  There are two instances of the variable $k$ in the formula for $f$.  These should be replaced with $\sqrt{3t}$. We just have to remember to use the correct variable for $(f \circ g)(x)$. \\



\[
(f \circ g)(x) = f(g(x)) = (\sqrt{3x} + 5)(\sqrt{3x} - 3)
\]




\end{explanation}


\begin{observation}


We must always pay attention to domains.  Here the domain of $g$ is $[0, \infty)$ and the range of $g$ is $[0, \infty)$.  The range fits totally inside the domain of $f$, which is $(-\infty, \infty)$.  Therefore, we can use the whole domain of $g$ as the domnain of $(f \circ g)$.

\end{observation}


\end{example}




















\begin{example}


Let $f(k) = \ln(k - 4)$ with its natural domain.


Let $g(t) = \frac{1}{t}$ with its natural domain.



\begin{explanation}

We would like a formula for $f \circ g$. \\

$(f \circ g)(x) = f(g(x))$ \\

This tells us to replace all occurrences of the variable in $f$ with the whole formula for $g$.  There is only one instance of the variable $k$ in the formula for $f$.  This should be replaced with $\frac{1}{t}$. We just have to remember to use the correct variable for $(f \circ g)(x)$. \\



\[
(f \circ g)(x) = f(g(x)) = \ln\left( \frac{1}{t} - 4 \right)
\]




\end{explanation}


\begin{observation}


We must always pay attention to domains.  Here the domain of $g$ is $(-\infty, 0) \cup (0, \infty)$ and the range of $g$ is $(-\infty, 0) \cup (0, \infty)$.  This range does not fit totally inside the domain of $f$, which is $(4, \infty)$.  We cannot have negative values of $g$ less than $4$.  \\


We need $\frac{1}{t} > 4$

Therefore, we need to restrict the domain of $g$ to $\left(0, \frac{1}{4} \right)$.



Therefore, the domnain of $(f \circ g)$ is $\left(0, \frac{1}{4} \right)$.

\end{observation}


\end{example}






















\begin{example}


Let $H(x) = 5 \ln(\sqrt{x^2-3})$ with its natural domain. \\


We would like to view this as a composition. \\



\begin{explanation}


We are looking for functions $f$ and $g$ such that $(f \circ g)(x) = f(g(x)) = 5 \ln(\sqrt{x^2-3})$ \\


First, identify ``insides'' of functions.  In this case, $x^2 - 3$ is the inside of $5 \ln(\sqrt{x})$. \\




Let $f(t) = 5 \ln(\sqrt{t})$. \\

Let $g(y) = y^2 - 3$. \\


Algebraically, this produces the formula want, $(f \circ g)(x) = f(g(x)) = 5 \ln(\sqrt{x^2-3})$.








\end{explanation}


\begin{observation}

Our formula presents two problems.\\

First, the inside of the square root must be $0$ of positive.\\

Second, the inside of the logarithm must be positive. \\



So, we need $x^2 - 3 \geq 0 $, which means $x \in (-\infty, -\sqrt{3}] \cup [\sqrt{3}, \infty)$. \\


However, this domain for $g$ will produce $0$ as a function value for $g$, which cannot go into $f$.  SO, we need to remove $-\sqrt{3}$ and $\sqrt{3}$.




The domain of $f \circ g$ is $(-\infty, -\sqrt{3}) \cup (\sqrt{3}, \infty)$.




\end{observation}


\end{example}




















\begin{center}
\textbf{\textcolor{green!50!black}{ooooo=-=-=-=-=-=-=-=-=-=-=-=-=ooOoo=-=-=-=-=-=-=-=-=-=-=-=-=ooooo}} \\

more examples can be found by following this link\\ \link[More Examples of the Composition]{https://ximera.osu.edu/csccmathematics/precalculus2/precalculus2/composition/examples/exampleList}

\end{center}







\end{document}
