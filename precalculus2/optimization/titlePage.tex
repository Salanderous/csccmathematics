\documentclass{ximera}


\graphicspath{
  {./}
  {ximeraTutorial/}
  {basicPhilosophy/}
}

\newcommand{\mooculus}{\textsf{\textbf{MOOC}\textnormal{\textsf{ULUS}}}}

\usepackage{tkz-euclide}\usepackage{tikz}
\usepackage{tikz-cd}
\usetikzlibrary{arrows}
\tikzset{>=stealth,commutative diagrams/.cd,
  arrow style=tikz,diagrams={>=stealth}} %% cool arrow head
\tikzset{shorten <>/.style={ shorten >=#1, shorten <=#1 } } %% allows shorter vectors

\usetikzlibrary{backgrounds} %% for boxes around graphs
\usetikzlibrary{shapes,positioning}  %% Clouds and stars
\usetikzlibrary{matrix} %% for matrix
\usepgfplotslibrary{polar} %% for polar plots
\usepgfplotslibrary{fillbetween} %% to shade area between curves in TikZ
\usetkzobj{all}
\usepackage[makeroom]{cancel} %% for strike outs
%\usepackage{mathtools} %% for pretty underbrace % Breaks Ximera
%\usepackage{multicol}
\usepackage{pgffor} %% required for integral for loops



%% http://tex.stackexchange.com/questions/66490/drawing-a-tikz-arc-specifying-the-center
%% Draws beach ball
\tikzset{pics/carc/.style args={#1:#2:#3}{code={\draw[pic actions] (#1:#3) arc(#1:#2:#3);}}}



\usepackage{array}
\setlength{\extrarowheight}{+.1cm}
\newdimen\digitwidth
\settowidth\digitwidth{9}
\def\divrule#1#2{
\noalign{\moveright#1\digitwidth
\vbox{\hrule width#2\digitwidth}}}






\DeclareMathOperator{\arccot}{arccot}
\DeclareMathOperator{\arcsec}{arcsec}
\DeclareMathOperator{\arccsc}{arccsc}

















%%This is to help with formatting on future title pages.
\newenvironment{sectionOutcomes}{}{}


\title{Optimization}

\begin{document}

\begin{abstract}
%%%
\end{abstract}
\maketitle








Optimizing means to get the most out of something.  This could be a maximum output or if you are trying to reduce output, it could be a minimum. Together, these are called \textbf{extema} of the function.

These maximums and minimums could be global(absolute) or local (relative).


We call the places in the domain where these maximums and minimums occur \textbf{critical numbers}.  And, we have seen that critical numbers mark where derivatives are $0$ or undefined.


We also have been connecting our algebraic analysis and function reasoninbg to characteristics of the graph.  The pair consisting of a critical number and an extrema is visually encoded as a highest or lowest point on the graph.

The graph makes it easy to see that 

\begin{itemize}
\item the function increases then decreases around a maximum
\item the function decreases then increases around a minimum
\end{itemize}

\textbf{Note:} unless the crtical point corresponds to an endpoint.


















\subsection{Learning Outcomes}


\begin{sectionOutcomes}
In this section, students will 

\begin{itemize}
\item Define a critical number.
\item Find critical numbers.
\item Define absolute maximum and absolute minimum.
\item Find the absolute max or min of a continuous function on a closed interval.
\item Define local maximum and local minimum.
\item Compare and contrast local and absolute maxima and minima.
\item Identify situations in which an absolute maximum or minimum is guaranteed.
\item Classify critical numbers.
\item State the First Derivative Test.
\item Apply the First Derivative Test.
\end{itemize}
\end{sectionOutcomes}

\end{document}

