\documentclass{ximera}


\graphicspath{
  {./}
  {ximeraTutorial/}
  {basicPhilosophy/}
}

\newcommand{\mooculus}{\textsf{\textbf{MOOC}\textnormal{\textsf{ULUS}}}}

\usepackage{tkz-euclide}\usepackage{tikz}
\usepackage{tikz-cd}
\usetikzlibrary{arrows}
\tikzset{>=stealth,commutative diagrams/.cd,
  arrow style=tikz,diagrams={>=stealth}} %% cool arrow head
\tikzset{shorten <>/.style={ shorten >=#1, shorten <=#1 } } %% allows shorter vectors

\usetikzlibrary{backgrounds} %% for boxes around graphs
\usetikzlibrary{shapes,positioning}  %% Clouds and stars
\usetikzlibrary{matrix} %% for matrix
\usepgfplotslibrary{polar} %% for polar plots
\usepgfplotslibrary{fillbetween} %% to shade area between curves in TikZ
\usetkzobj{all}
\usepackage[makeroom]{cancel} %% for strike outs
%\usepackage{mathtools} %% for pretty underbrace % Breaks Ximera
%\usepackage{multicol}
\usepackage{pgffor} %% required for integral for loops



%% http://tex.stackexchange.com/questions/66490/drawing-a-tikz-arc-specifying-the-center
%% Draws beach ball
\tikzset{pics/carc/.style args={#1:#2:#3}{code={\draw[pic actions] (#1:#3) arc(#1:#2:#3);}}}



\usepackage{array}
\setlength{\extrarowheight}{+.1cm}
\newdimen\digitwidth
\settowidth\digitwidth{9}
\def\divrule#1#2{
\noalign{\moveright#1\digitwidth
\vbox{\hrule width#2\digitwidth}}}






\DeclareMathOperator{\arccot}{arccot}
\DeclareMathOperator{\arcsec}{arcsec}
\DeclareMathOperator{\arccsc}{arccsc}

















%%This is to help with formatting on future title pages.
\newenvironment{sectionOutcomes}{}{}


\author{Jim Talamo}
\license{Creative Commons 3.0 By-bC}


\outcome{}


\begin{document}
\begin{exercise}
Suppose that $\{a_n\}_{n=1}$ is a sequence and define its sequence of partial sums $\{s_n\}_{n=1}$ by the usual rule $s_n = \sum_{k=1}^n a_k$.  Suppose it is known that:

\[
s_n = \frac{3^n+n^3+18}{3^n+3n}
\]

Then, what is $a_1+a_2+a_3$?

\[
a_1+a_2+a_3 = \answer{2}
\]
\begin{hint}
By definition, $a_1+a_2+a_3 = s_3$, so plug $n=3$ into the given formula for $s_n$ to find $a_1+a_2+a_3$.
\end{hint}
\begin{exercise}
\[
\lim_{n \to \infty} s_n = \answer{1}
\]
\begin{hint}
Use the Growth Rates Result!
\end{hint}
\begin{exercise}
Since $\lim_{n \to \infty} s_n$ exists, then $\sum_{k=1}^{\infty} a_k$:


\begin{multipleChoice}
\choice[correct]{converges}
\choice{diverges}
\end{multipleChoice}

Since the limit $\lim_{n \to \infty} s_n$ is $3$, then $\sum_{k=1}^{\infty} a_k$:
\begin{multipleChoice}
\choice[correct]{converges to 3.}
\choice{converges, but more information is needed to determine its value.}
\end{multipleChoice}

\begin{exercise}
Determine $\sum_{k=4}^{\infty} a_k$.

\[
\sum_{k=4}^{\infty} a_k = \answer{-1}
\]

\begin{hint}
Note that the lower index is different from before!  We have an earlier result that allows us to compute an infinite sum, so in some sense, the heavy lifting has been done!  Indeed, write out the sum we found earlier:
\begin{image}
  \begin{tikzpicture}
        \node at (0,0) {
          $\sum_{k=1}^{\infty} a_k= \underbrace{a_1+a_2+a_3} + \underbrace{a_4+a_5+\ldots}$
        };
        \node at (-.5,-.5) {\tiny{this is $s_3$}};
        \node at (-2.5,-.6) {\tiny{This is the sum }};
         \node at (-2.5,-.8) {\tiny{we found earlier}};
        \node at (1.6,-.8) {\tiny{This is $\sum_{k=4}^{\infty} a_k$}};
      \end{tikzpicture}
  \end{image}
\end{hint}

Thus:

\[
\answer{1} = \answer{2} + \sum_{k=4}^{\infty} a_k
\]
\begin{exercise}
Find a formula for $a_n$:

\[
a_n = \answer{\frac{3^n+n^3+18}{3^n+3n}-\frac{3^{n-1}+(n-1)^3+18}{3^{n-1}+3(n-1)}}
\]

\begin{hint}
We know: 
\begin{image}
  \begin{tikzpicture}
        \node at (0,0) {
          $s_n = \underbrace{a_1+\ldots+a_{n-1}} + a_n = s_{n-1} + a_n$
        };
        \node at (-.8,-.5) {\small{this is $s_{n-1}$}};
      \end{tikzpicture}
  \end{image}
  
  Thus, $a_n = s_n-s_{n-1}$
\end{hint}
\end{exercise}
\end{exercise}
\end{exercise}
\end{exercise}
\end{exercise}


\end{document}
