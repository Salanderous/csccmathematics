\documentclass{ximera}


\graphicspath{
  {./}
  {ximeraTutorial/}
  {basicPhilosophy/}
}

\newcommand{\mooculus}{\textsf{\textbf{MOOC}\textnormal{\textsf{ULUS}}}}

\usepackage{tkz-euclide}\usepackage{tikz}
\usepackage{tikz-cd}
\usetikzlibrary{arrows}
\tikzset{>=stealth,commutative diagrams/.cd,
  arrow style=tikz,diagrams={>=stealth}} %% cool arrow head
\tikzset{shorten <>/.style={ shorten >=#1, shorten <=#1 } } %% allows shorter vectors

\usetikzlibrary{backgrounds} %% for boxes around graphs
\usetikzlibrary{shapes,positioning}  %% Clouds and stars
\usetikzlibrary{matrix} %% for matrix
\usepgfplotslibrary{polar} %% for polar plots
\usepgfplotslibrary{fillbetween} %% to shade area between curves in TikZ
\usetkzobj{all}
\usepackage[makeroom]{cancel} %% for strike outs
%\usepackage{mathtools} %% for pretty underbrace % Breaks Ximera
%\usepackage{multicol}
\usepackage{pgffor} %% required for integral for loops



%% http://tex.stackexchange.com/questions/66490/drawing-a-tikz-arc-specifying-the-center
%% Draws beach ball
\tikzset{pics/carc/.style args={#1:#2:#3}{code={\draw[pic actions] (#1:#3) arc(#1:#2:#3);}}}



\usepackage{array}
\setlength{\extrarowheight}{+.1cm}
\newdimen\digitwidth
\settowidth\digitwidth{9}
\def\divrule#1#2{
\noalign{\moveright#1\digitwidth
\vbox{\hrule width#2\digitwidth}}}






\DeclareMathOperator{\arccot}{arccot}
\DeclareMathOperator{\arcsec}{arcsec}
\DeclareMathOperator{\arccsc}{arccsc}

















%%This is to help with formatting on future title pages.
\newenvironment{sectionOutcomes}{}{}


\author{Jim Talamo}
\license{Creative Commons 3.0 By-bC}


\outcome{}


\begin{document}
\begin{exercise}
Determine whether the series below converges or diverges.

\[
\sum_{k=1}^{\infty} \ln \left(\frac{k}{k+1}\right)
\]

Let $s_n = \sum_{k=1}^n a_k$.  An explicit formula for $s_n$ for $n \geq 1$ is:

\[
s_n = \answer{-\ln(n+1)}
\]

\begin{exercise}
Hence, $\lim_{n \to \infty} s_n = \answer{-\infty}$ and:

\begin{multipleChoice}
\choice{The series converges.}
\choice[correct]{The series diverges.}
\end{multipleChoice}

\begin{hint}
To determine whether the series converges or diverges, complete the following argument.

The first step to try to determine if a series $\sum_{k=1}^{\infty}a_k$ converges is to consider the sequence of partial sums, defined by $s_n = \sum_{k=1}^{n} a_k = a_1+\ldots + a_n$ for $n \geq 1$, and to try to find an \emph{explicit} formula for $s_n$.


Write out the first several terms in the sequence $\{a_n\}_{n=1}$ fact that  $\ln\left(\frac{k}{k+1}\right) = \ln(k)-\ln(k+1)$.

\begin{align*}
a_1 &= \answer{\ln(1)} - \answer{\ln(2)} & a_2 &= \answer{\ln(2)}-\answer{\ln(3)} & a_3 &= \answer{\ln(3)}-\answer{\ln(4)} \\
 a_4 &= \answer{\ln(4)}-\answer{\ln(5)}& a_5 &= \answer{\ln(5)}-\answer{\ln(6)}
\end{align*}

\begin{question}
Find $s_1$, $s_2$, $s_3$, $s_4$, and $s_5$.  Pay attention for convenient cancellation!

\begin{align*}
s_1 &= \answer{-\ln(2)} & s_2 &=  \answer{-\ln(3)} & s_3 &=  \answer{-\ln(4)} & s_4 &= \answer{-\ln(5)}& s_5 &= \answer{-\ln(6)} 
\end{align*}

\begin{question}
We always have a \emph{recursive} formula for $s_n$:

\[
s_n = s_{n-1} +a_n
\]
 which in this case is:
 
 \[
 s_n= s_{n-1} + \answer{\ln(n)-\ln(n+1)}
 \]
 
Note that this does not allow us to determine whether the sequence $s_n$ has a limit; if it does and we call the limit $L$, by taking the limit of both sides of the above equation, we obtain:

\[
L = L +\answer{0}
\]

We can do better though!  By examining the terms that you wrote out for $s_n$, you should notice a pattern!  In fact, we can conjecture that an \emph{explicit} formula for $s_n$:

\[
s_n = \answer{-\ln(n+1)}
\]
\begin{question}
By taking the limit as $n \to \infty$, we find:

\[
\lim_{n \to \infty} s_n = \answer{-\infty}
\]
Hence:
\begin{multipleChoice}
\choice{The series $\sum_{k=1}^{\infty} a_k$ converges but more information is needed to determine its value.}
\choice{The series $\sum_{k=1}^{\infty} a_k$ converges to $2$.}
\choice[correct]{The series $\sum_{k=1}^{\infty} a_k$ diverges.}
\end{multipleChoice}

\end{question}
\end{question}
\end{question}
\end{hint}

\end{exercise}
\end{exercise}


\end{document}
