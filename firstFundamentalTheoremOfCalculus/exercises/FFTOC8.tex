\documentclass{ximera}


\graphicspath{
  {./}
  {ximeraTutorial/}
  {basicPhilosophy/}
}

\newcommand{\mooculus}{\textsf{\textbf{MOOC}\textnormal{\textsf{ULUS}}}}

\usepackage{tkz-euclide}\usepackage{tikz}
\usepackage{tikz-cd}
\usetikzlibrary{arrows}
\tikzset{>=stealth,commutative diagrams/.cd,
  arrow style=tikz,diagrams={>=stealth}} %% cool arrow head
\tikzset{shorten <>/.style={ shorten >=#1, shorten <=#1 } } %% allows shorter vectors

\usetikzlibrary{backgrounds} %% for boxes around graphs
\usetikzlibrary{shapes,positioning}  %% Clouds and stars
\usetikzlibrary{matrix} %% for matrix
\usepgfplotslibrary{polar} %% for polar plots
\usepgfplotslibrary{fillbetween} %% to shade area between curves in TikZ
\usetkzobj{all}
\usepackage[makeroom]{cancel} %% for strike outs
%\usepackage{mathtools} %% for pretty underbrace % Breaks Ximera
%\usepackage{multicol}
\usepackage{pgffor} %% required for integral for loops



%% http://tex.stackexchange.com/questions/66490/drawing-a-tikz-arc-specifying-the-center
%% Draws beach ball
\tikzset{pics/carc/.style args={#1:#2:#3}{code={\draw[pic actions] (#1:#3) arc(#1:#2:#3);}}}



\usepackage{array}
\setlength{\extrarowheight}{+.1cm}
\newdimen\digitwidth
\settowidth\digitwidth{9}
\def\divrule#1#2{
\noalign{\moveright#1\digitwidth
\vbox{\hrule width#2\digitwidth}}}






\DeclareMathOperator{\arccot}{arccot}
\DeclareMathOperator{\arcsec}{arcsec}
\DeclareMathOperator{\arccsc}{arccsc}

















%%This is to help with formatting on future title pages.
\newenvironment{sectionOutcomes}{}{}


\outcome{Define accumulation functions.}
\outcome{Calculate and evaluate accumulation functions.}
\outcome{State the First Fundamental Theorem of Calculus.}
\outcome{Take derivatives of accumulation functions using the First Fundamental Theorem of Calculus.}
\outcome{Use accumulation functions to find information about the original function.}
\outcome{Understand the relationship between the function and the derivative of its accumulation function.}

\author{Nela Lakos \and Kyle Parsons}

\begin{document}
\begin{exercise}

The graph of $f$ is given below.

\begin{image}
  \begin{tikzpicture}
    \begin{axis}[
        xmin=-0.3,xmax=6.3,ymin=-2.3,ymax=2.3,
        clip=true,
        unit vector ratio*=1 1 1,
        axis lines=center,
        grid = major,
        ytick={-2,-1,...,36},
        xtick={0,1,...,10},
        xlabel=$t$, ylabel=$y$,
        every axis y label/.style={at=(current axis.above origin),anchor=south},
        every axis x label/.style={at=(current axis.right of origin),anchor=west},
      ]
      \draw[ultra thick,penColor] (axis cs:0,-2) arc[radius=200,start angle=-90,end angle=0] (axis cs:2,0);
      \addplot[ultra thick,penColor,domain=2:4] {x-2};
      \addplot[ultra thick,penColor,domain=4:6] {2};    
        
      \node at (axis cs:1.5,1.5) {$y=f(t)$};
      \end{axis}`
  \end{tikzpicture}
\end{image}

Let $A(x) = \int_0^x f(t) dt$.

Evaluate the following.
\begin{align*}
A(0) &= \answer{0}\\
A(2) &= \answer{-\pi}\\
A(4) &= \answer{2-\pi}\\
A(6) &= \answer{6-\pi}\\
\end{align*}

The graph of $A(x)$ is given below.

\begin{image}
  \begin{tikzpicture}
    \begin{axis}[
        xmin=-0.3,xmax=6.3,ymin=-4.3,ymax=4.3,
        clip=true,
        unit vector ratio*=1 1 1,
        axis lines=center,
        grid = major,
        ytick={-4,-3,...,36},
        xtick={0,1,...,10},
        xlabel=$x$, ylabel=$y$,
        every axis y label/.style={at=(current axis.above origin),anchor=south},
        every axis x label/.style={at=(current axis.right of origin),anchor=west},
      ]
      \addplot[ultra thick,penColor,domain=0:2] {-x*sqrt(4-x^2)/2 - 2*asin(x/2)*pi/180};
      \addplot[ultra thick,penColor,domain=2:4] {(x-2)^2/2-pi};
      \addplot[ultra thick,penColor,domain=4:6] {2*(x-4)-pi+2};    
        
      \node at (axis cs:1.5,1.5) {$y=A(x)$};
      \end{axis}`
  \end{tikzpicture}
\end{image}

Evaluate the following.
\begin{align*}
A'(2) &= \answer{0}\\
A'(5) &= \answer{2}
\end{align*}

Solve the initial value problem $y'(x) = f(x)$ and $y(0) = 6$ in terms of $A$.
\begin{hint}
Recall: The function $A$ is an antiderivative of $f$.
\end{hint}
\begin{hint}
Therefore, any antiderivative of $f$ can be written as $F(x)=A(x)+C$, for some constant $C$.\\

Any such function satisfies the differential equation
 $y'(x) = f(x)$: \\
 $F'(x) = A'(x)=f(x)$!\\
Find the constant C that also satisfies the initial condition:
$F(0)=A(0)+C=C=6$
\end{hint}
\[
y(x) = \answer{A(x)+6}
\]

\end{exercise}
\end{document}