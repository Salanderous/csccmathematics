\documentclass{ximera}


\graphicspath{
  {./}
  {ximeraTutorial/}
  {basicPhilosophy/}
}

\newcommand{\mooculus}{\textsf{\textbf{MOOC}\textnormal{\textsf{ULUS}}}}

\usepackage{tkz-euclide}\usepackage{tikz}
\usepackage{tikz-cd}
\usetikzlibrary{arrows}
\tikzset{>=stealth,commutative diagrams/.cd,
  arrow style=tikz,diagrams={>=stealth}} %% cool arrow head
\tikzset{shorten <>/.style={ shorten >=#1, shorten <=#1 } } %% allows shorter vectors

\usetikzlibrary{backgrounds} %% for boxes around graphs
\usetikzlibrary{shapes,positioning}  %% Clouds and stars
\usetikzlibrary{matrix} %% for matrix
\usepgfplotslibrary{polar} %% for polar plots
\usepgfplotslibrary{fillbetween} %% to shade area between curves in TikZ
\usetkzobj{all}
\usepackage[makeroom]{cancel} %% for strike outs
%\usepackage{mathtools} %% for pretty underbrace % Breaks Ximera
%\usepackage{multicol}
\usepackage{pgffor} %% required for integral for loops



%% http://tex.stackexchange.com/questions/66490/drawing-a-tikz-arc-specifying-the-center
%% Draws beach ball
\tikzset{pics/carc/.style args={#1:#2:#3}{code={\draw[pic actions] (#1:#3) arc(#1:#2:#3);}}}



\usepackage{array}
\setlength{\extrarowheight}{+.1cm}
\newdimen\digitwidth
\settowidth\digitwidth{9}
\def\divrule#1#2{
\noalign{\moveright#1\digitwidth
\vbox{\hrule width#2\digitwidth}}}






\DeclareMathOperator{\arccot}{arccot}
\DeclareMathOperator{\arcsec}{arcsec}
\DeclareMathOperator{\arccsc}{arccsc}

















%%This is to help with formatting on future title pages.
\newenvironment{sectionOutcomes}{}{}


\author{Jim Talamo}
\license{Creative Commons 3.0 By-bC}


\outcome{}


\begin{document}
\begin{exercise}
Consider$\{a_n \}_{n=1}$ where $a_n = \frac{1}{n^2+n}$.  Then, the sequence is:

\begin{selectAll}
\choice{increasing}
\choice[correct]{decreasing}
\choice[correct]{monotonic}
\choice[correct]{bounded above}
\choice[correct]{bounded below}
\choice[correct]{bounded}
\end{selectAll}
(Select all that apply)

Now, let $s_n = \sum_{k=1}^{n} a_k$.  Then, the sequence $\{s_n \}_{n=1}$ is:
\begin{selectAll}
\choice[correct]{increasing}
\choice{decreasing}
\choice[correct]{monotonic}
\choice[correct]{bounded above}
\choice[correct]{bounded below}
\choice[correct]{bounded}
\end{selectAll}
(Select all that apply)

\begin{hint}
Since all of the $a_n$ terms are positive, $s_n$ should be:

\begin{multipleChoice}
\choice[correct]{increasing}
\choice{decreasing}
\end{multipleChoice}

Hence, $s_n$ must be:
\begin{multipleChoice}
\choice[correct]{bounded below}
\choice{bounded above}
\end{multipleChoice}

To determine if $\{s_n\}$ is bounded above, we must determine if $s_n$ has a limit; indeed if $\{s_n\}$ is increasing, the only way it will not be bounded above is if $\lim_{n \to \infty} s_n = +\infty$.  Writing out this limit gives.

\[
\lim_{n \to \infty} s_n = \sum_{k=1}^{\infty}  \frac{1}{k^2+k}.
\]

Using partial fraction decomposition shows that

\[
\frac{1}{k^2+k} = \frac{\answer{1}}{k}+\frac{\answer{-1}}{k+1}.
\]

Thus, we can write an explicit formula for $s_n$.

\[
s_n=\answer{1 -\frac{1}{n+1}}.
\]

Hence, $\lim_{n \to \infty} s_n = \answer{1}$.
\end{hint}
\end{exercise}

\end{document}
