\documentclass{ximera}


\graphicspath{
  {./}
  {ximeraTutorial/}
  {basicPhilosophy/}
}

\newcommand{\mooculus}{\textsf{\textbf{MOOC}\textnormal{\textsf{ULUS}}}}

\usepackage{tkz-euclide}\usepackage{tikz}
\usepackage{tikz-cd}
\usetikzlibrary{arrows}
\tikzset{>=stealth,commutative diagrams/.cd,
  arrow style=tikz,diagrams={>=stealth}} %% cool arrow head
\tikzset{shorten <>/.style={ shorten >=#1, shorten <=#1 } } %% allows shorter vectors

\usetikzlibrary{backgrounds} %% for boxes around graphs
\usetikzlibrary{shapes,positioning}  %% Clouds and stars
\usetikzlibrary{matrix} %% for matrix
\usepgfplotslibrary{polar} %% for polar plots
\usepgfplotslibrary{fillbetween} %% to shade area between curves in TikZ
\usetkzobj{all}
\usepackage[makeroom]{cancel} %% for strike outs
%\usepackage{mathtools} %% for pretty underbrace % Breaks Ximera
%\usepackage{multicol}
\usepackage{pgffor} %% required for integral for loops



%% http://tex.stackexchange.com/questions/66490/drawing-a-tikz-arc-specifying-the-center
%% Draws beach ball
\tikzset{pics/carc/.style args={#1:#2:#3}{code={\draw[pic actions] (#1:#3) arc(#1:#2:#3);}}}



\usepackage{array}
\setlength{\extrarowheight}{+.1cm}
\newdimen\digitwidth
\settowidth\digitwidth{9}
\def\divrule#1#2{
\noalign{\moveright#1\digitwidth
\vbox{\hrule width#2\digitwidth}}}






\DeclareMathOperator{\arccot}{arccot}
\DeclareMathOperator{\arcsec}{arcsec}
\DeclareMathOperator{\arccsc}{arccsc}

















%%This is to help with formatting on future title pages.
\newenvironment{sectionOutcomes}{}{}


\author{Jim Talamo}
\license{Creative Commons 3.0 By-bC}


\outcome{}


\begin{document}
\begin{exercise}

Suppose that $\{a_n\}$ and $\{b_n\}$ are sequences and it is known that:

\[
\lim_{n \to \infty} a_n = 3 \qquad \textrm{ and } \qquad \sum_{k=1}^{n} b_k = \frac{5n+2}{2n-1}
\]
Answer the following:

The series $\sum_{k=1}^{\infty} a_k$:
\begin{multipleChoice}
\choice{converges to $0$.}
\choice{converges to $3$.}
\choice{converges, but more information is needed to determine its value.}
\choice{could converge or diverge; more information is needed.}
\choice[correct]{diverges by divergence test.}
\end{multipleChoice}

\begin{exercise}
The series $\sum_{k=1}^{\infty} b_k$:
\begin{multipleChoice}
\choice{converges to $0$.}
\choice[correct]{converges to $\frac{5}{2}$.}
\choice{converges, but more information is needed to determine its value.}
\choice{could converge or diverge; more information is needed.}
\choice{diverges by divergence test.}
\end{multipleChoice}

\begin{hint}
Note that $\lim_{n \to \infty} \frac{5n+2}{2n-1} = \answer{\frac{5}{2}}$.  Now, look at the given information.  We are told:

\begin{multipleChoice}
\choice{The $n$-th term in the sequence $\{b_n\}$ is $b_n =  \frac{5n+2}{2n-1}$.}
\choice[correct]{The $n$-th term in the sequence of partial sums for $\{b_n\}$ is $\frac{5n+2}{2n-1}$.}
\end{multipleChoice}

We're given the result of adding the first $n$ terms for the sequence $\{b_n\}$!  That is, we're told:

\[
b_1+b_2 + \ldots + b_n =  \frac{5n+2}{2n-1}
\]
Hence, $\lim_{n \to \infty} \frac{5n+2}{2n-1}$ is actually analogous to $\sum_{k=1}^{\infty} b_k$!
\end{hint}

\begin{exercise}
For the sequence $\{b_n\}$:
\begin{multipleChoice}
\choice[correct]{$\lim_{n \to \infty} b_n = 0.$}
\choice{$\lim_{n \to \infty} b_n = \frac{5}{2}$.}
\choice{$\lim_{n \to \infty} b_n = \infty$}
\choice{$\lim_{n \to \infty} b_n$ diverges by divergence test.}
\end{multipleChoice}

\begin{hint}
A \emph{series} is the sum of terms in a \emph{sequence}.  If the series converges, what must the limit of the \emph{sequence} (whose terms we are summing) be?
\end{hint}
\end{exercise}
\end{exercise}

\begin{exercise}
The series $\sum_{k=1}^{\infty} (a_k -b_k)$:
\begin{multipleChoice}
\choice{converges to $3$.}
\choice{converges to $\frac{1}{2}$.}
\choice{converges, but more information is needed to determine its value.}
\choice{could converge or diverge; more information is needed.}
\choice[correct]{diverges by divergence test.}
\end{multipleChoice}

\begin{hint}
Since $\sum_{k=1}^{\infty} b_k$ converges, we know $\lim_{n \to \infty} b_n = \answer{0}$.  Hence, $\lim_{n \to \infty} (a_n-b_n) = \answer{3}$.  Thus:

\begin{multipleChoice}
\choice{$\sum_{k=1}^{\infty} (a_k -b_k)$ converges to $3$.}
\choice[correct]{$\sum_{k=1}^{\infty} (a_k -b_k)$ diverges by the divergence test.}
\end{multipleChoice}
\end{hint}
\end{exercise}

\end{exercise}
\end{document}
