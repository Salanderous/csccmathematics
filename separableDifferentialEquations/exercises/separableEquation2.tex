\documentclass{ximera}


\graphicspath{
  {./}
  {ximeraTutorial/}
  {basicPhilosophy/}
}

\newcommand{\mooculus}{\textsf{\textbf{MOOC}\textnormal{\textsf{ULUS}}}}

\usepackage{tkz-euclide}\usepackage{tikz}
\usepackage{tikz-cd}
\usetikzlibrary{arrows}
\tikzset{>=stealth,commutative diagrams/.cd,
  arrow style=tikz,diagrams={>=stealth}} %% cool arrow head
\tikzset{shorten <>/.style={ shorten >=#1, shorten <=#1 } } %% allows shorter vectors

\usetikzlibrary{backgrounds} %% for boxes around graphs
\usetikzlibrary{shapes,positioning}  %% Clouds and stars
\usetikzlibrary{matrix} %% for matrix
\usepgfplotslibrary{polar} %% for polar plots
\usepgfplotslibrary{fillbetween} %% to shade area between curves in TikZ
\usetkzobj{all}
\usepackage[makeroom]{cancel} %% for strike outs
%\usepackage{mathtools} %% for pretty underbrace % Breaks Ximera
%\usepackage{multicol}
\usepackage{pgffor} %% required for integral for loops



%% http://tex.stackexchange.com/questions/66490/drawing-a-tikz-arc-specifying-the-center
%% Draws beach ball
\tikzset{pics/carc/.style args={#1:#2:#3}{code={\draw[pic actions] (#1:#3) arc(#1:#2:#3);}}}



\usepackage{array}
\setlength{\extrarowheight}{+.1cm}
\newdimen\digitwidth
\settowidth\digitwidth{9}
\def\divrule#1#2{
\noalign{\moveright#1\digitwidth
\vbox{\hrule width#2\digitwidth}}}






\DeclareMathOperator{\arccot}{arccot}
\DeclareMathOperator{\arcsec}{arcsec}
\DeclareMathOperator{\arccsc}{arccsc}

















%%This is to help with formatting on future title pages.
\newenvironment{sectionOutcomes}{}{}


\author{Jim Talamo}
\license{Creative Commons 3.0 By-NC}


\outcome{Review integration techniques}


\begin{document}
\begin{exercise}
Solve the initial value problem below and explicitly solve for $y$ in your final answer:

\[
3y y' +2x = 5\sin(2x)  , \qquad y(0)=-1
\]

The solution is $y(x) = \answer{ - \sqrt{ -\frac{2}{3}x^2 -\frac{5}{3} \cos(2x)+\frac{8}{3}} }$.

\begin{hint}
The differentials $\d y$ and $\d x$ are related by:

\[
\left(\answer{3y} \right) \d y = \left(\answer{ -2x+5\sin(2x) } \right) \d x
\]

\begin{question}
We can now integrate both sides to obtain:

\begin{align*}
\frac{3}{2}y^2 &= \answer{-x^2-\frac{5}{2}\cos(2x)} +C\\
y^2 &= \answer{-\frac{2}{3}x^2 -\frac{5}{3}\cos(2x)}+C
\end{align*}
(Note that we are allowing $C$ to represent an arbitrary constant, so we simply leave $C$ as $C$ when we multiply both sides by $\frac{2}{3}$ in the last step).

Using the initial condition, we find $C=\answer{\frac{8}{3}}$.

\begin{question}

\end{question}
We thus have:

\[
y^2 = -\frac{2}{3}x^2 -\frac{5}{3}\cos(2x)+\frac{8}{3}
\]

Now, remember that the solution to the differential equation is supposed to be a \emph{function}.  Does the above equation describe $y$ as a function of $x$?

\begin{multipleChoice}
\choice{Yes; for each $x$-value, there is only one associated $y$-value.}
\choice{Yes; for each $y$-value, there is only one associated $x$-value.}
\choice[correct]{No; for each $x$-value, there is more than one associated $y$-value.}
\choice{No; for each $y$-value, there is more than one associated $x$-value.}
\end{multipleChoice}

This is manifested algebraically by remember we need both a positive and negative square root!  To determine which to use:

\begin{multipleChoice}
\choice{we always use the positive square root, so choose it.}
\choice{we always use the negative square root, so choose it.}
\choice{The initial condition can be used to show we need to use the positive square root.}
\choice[correct]{The initial condition can be used to show we need to use the negative square root.}
\end{multipleChoice}
\end{question}
 \end{hint}
 
\end{exercise}
\end{document}
