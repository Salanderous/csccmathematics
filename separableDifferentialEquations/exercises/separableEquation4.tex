\documentclass{ximera}


\graphicspath{
  {./}
  {ximeraTutorial/}
  {basicPhilosophy/}
}

\newcommand{\mooculus}{\textsf{\textbf{MOOC}\textnormal{\textsf{ULUS}}}}

\usepackage{tkz-euclide}\usepackage{tikz}
\usepackage{tikz-cd}
\usetikzlibrary{arrows}
\tikzset{>=stealth,commutative diagrams/.cd,
  arrow style=tikz,diagrams={>=stealth}} %% cool arrow head
\tikzset{shorten <>/.style={ shorten >=#1, shorten <=#1 } } %% allows shorter vectors

\usetikzlibrary{backgrounds} %% for boxes around graphs
\usetikzlibrary{shapes,positioning}  %% Clouds and stars
\usetikzlibrary{matrix} %% for matrix
\usepgfplotslibrary{polar} %% for polar plots
\usepgfplotslibrary{fillbetween} %% to shade area between curves in TikZ
\usetkzobj{all}
\usepackage[makeroom]{cancel} %% for strike outs
%\usepackage{mathtools} %% for pretty underbrace % Breaks Ximera
%\usepackage{multicol}
\usepackage{pgffor} %% required for integral for loops



%% http://tex.stackexchange.com/questions/66490/drawing-a-tikz-arc-specifying-the-center
%% Draws beach ball
\tikzset{pics/carc/.style args={#1:#2:#3}{code={\draw[pic actions] (#1:#3) arc(#1:#2:#3);}}}



\usepackage{array}
\setlength{\extrarowheight}{+.1cm}
\newdimen\digitwidth
\settowidth\digitwidth{9}
\def\divrule#1#2{
\noalign{\moveright#1\digitwidth
\vbox{\hrule width#2\digitwidth}}}






\DeclareMathOperator{\arccot}{arccot}
\DeclareMathOperator{\arcsec}{arcsec}
\DeclareMathOperator{\arccsc}{arccsc}

















%%This is to help with formatting on future title pages.
\newenvironment{sectionOutcomes}{}{}


\author{Jim Talamo}
\license{Creative Commons 3.0 By-NC}


\outcome{Review integration techniques}


\begin{document}
\begin{exercise}
Consider the initial value problem below:

\[
\frac{dy}{dx} = \frac{6y}{x^2-2x-8}  , \qquad y(0)=b \textrm{ where } b>0.
\]

The direction field for this differential equation is:

 \begin{image}
  \includegraphics[width=.5 \textwidth]{separableEquation4Image.png}
\end{image}

\begin{exercise}
The solution to the IVP is likely to depend on the choice of $b>0$.  However:

For $b=\frac{1}{2}$, it looks like $\lim_{x\to -2^+}y(x) = \answer{\infty} $ from the direction field.

For $b=2$, it looks like $\lim_{x\to -2^+} y(x) = \answer{\infty}$ from the direction field.

For any $b>0$, does is appear that $\lim_{x\to -2^+} y(x)$ depends on $b$?
\begin{multipleChoice}
\choice{Yes}
\choice[correct]{No}
\end{multipleChoice}
\end{exercise}

\begin{exercise}
To verify the conjecture above, we need to solve the initial value problem.

The solution of the initial value problem is $y(x) = \answer{ - \frac{b}{2} \cdot \frac{x-4}{x+2} }$.

\begin{hint}
The differentials $dy$ and $dx$ are related by:

\[
\left(\answer{\frac{1}{y}} \right) \d y = \left(\frac{6}{\answer{x^2-2x-8} } \right) dx
\]

\begin{question}
We can now integrate both sides.

To integrate the righthand side:

\begin{multipleChoice}
\choice{Use the integration formula involving an inverse tangent.}
\choice{Use a $u$-substitution}
\choice{Use integration by parts}
\choice[correct]{Use partial fraction decomposition}
\end{multipleChoice}

To perform the partial fraction decomposition:

\[
\frac{6}{x^2-2x-8} = \frac{6}{\left(x-\answer{4}\right)\left(x+\answer{2}\right)} = \frac{A}{x-\answer{4}}+\frac{B}{x+\answer{2}}
\]

We should multiply both sides by $\answer{(x-4)(x+2)}$.  After doing so, we obtain:

\[
6=A\left(\answer{x+2}\right)+B\left(\answer{x-4}\right)
\]

From this, we can find: $A=\answer{1}$ and $B=\answer{-1}$, so:

\[
\frac{6}{x^2-2x-8} = \frac{\answer{1}}{x-4}+\frac{\answer{-1}}{x+2}
\]


\begin{question}
Since the differentials $dy$ and $dx$ are related by:

\[
\frac{1}{y}  dy = \left(\frac{1}{x-4}-\frac{1}{x+2} \right) dx
\]

we can integrate both sides to obtain:

\[
\answer{\ln|y|} = \answer{\ln|x-4| - \ln|x+2|} +C\\
\]

Using the rules of logarithms, $ \ln|x-4| - \ln|x+2| =  \ln\left| \answer{\frac{x-4}{x+2}} \right|$, so we can solve the above for $y$ to find:

\[
y= C \cdot \answer{\frac{x-4}{x+2}} \\
\]
  
\begin{question}  
Using the initial condition $y(0)=b$, we find $C=\answer{-\frac{b}{2}}$.

(type your answer in terms of $b$)

We thus have $y = \answer{-\frac{b}{2}} \cdot \frac{x-4}{x+2}$.


\end{question}
\end{question}
\end{question}
 \end{hint}
 
\begin{exercise}
Note that we specified $b>0$ at the start of the problem.  Does $\lim_{b \to -2^+} y(x)$ depend on the choice of $b$?
\begin{multipleChoice}
\choice{Yes}
\choice[correct]{No}
\end{multipleChoice}
In fact, we find:

\[
\lim_{b \to -2^+} y(x) = \answer{\infty}
\]
(Use $\infty$ or $-\infty$ where appropriate, or write ``DNE'' when the limit does not exist otherwise)
\end{exercise}
 \end{exercise}
\end{exercise}
\end{document}
