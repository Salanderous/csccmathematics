\documentclass{ximera}


\graphicspath{
  {./}
  {ximeraTutorial/}
  {basicPhilosophy/}
}

\newcommand{\mooculus}{\textsf{\textbf{MOOC}\textnormal{\textsf{ULUS}}}}

\usepackage{tkz-euclide}\usepackage{tikz}
\usepackage{tikz-cd}
\usetikzlibrary{arrows}
\tikzset{>=stealth,commutative diagrams/.cd,
  arrow style=tikz,diagrams={>=stealth}} %% cool arrow head
\tikzset{shorten <>/.style={ shorten >=#1, shorten <=#1 } } %% allows shorter vectors

\usetikzlibrary{backgrounds} %% for boxes around graphs
\usetikzlibrary{shapes,positioning}  %% Clouds and stars
\usetikzlibrary{matrix} %% for matrix
\usepgfplotslibrary{polar} %% for polar plots
\usepgfplotslibrary{fillbetween} %% to shade area between curves in TikZ
\usetkzobj{all}
\usepackage[makeroom]{cancel} %% for strike outs
%\usepackage{mathtools} %% for pretty underbrace % Breaks Ximera
%\usepackage{multicol}
\usepackage{pgffor} %% required for integral for loops



%% http://tex.stackexchange.com/questions/66490/drawing-a-tikz-arc-specifying-the-center
%% Draws beach ball
\tikzset{pics/carc/.style args={#1:#2:#3}{code={\draw[pic actions] (#1:#3) arc(#1:#2:#3);}}}



\usepackage{array}
\setlength{\extrarowheight}{+.1cm}
\newdimen\digitwidth
\settowidth\digitwidth{9}
\def\divrule#1#2{
\noalign{\moveright#1\digitwidth
\vbox{\hrule width#2\digitwidth}}}






\DeclareMathOperator{\arccot}{arccot}
\DeclareMathOperator{\arcsec}{arcsec}
\DeclareMathOperator{\arccsc}{arccsc}

















%%This is to help with formatting on future title pages.
\newenvironment{sectionOutcomes}{}{}


\outcome{Take derivatives of functions raised to functions.}

\author{Nela Lakos \and Kyle Parsons}

\begin{document}
\begin{exercise}

The total number of people who have contracted a common cold by a time $t$ days after its outbreak on an island is given by
\[
N(t) = \frac{200000}{1+100e^{-0.1t}},\quad t\geq0.
\]

The graph of $N$ on the intervall $[0,100]$ is given below.

\begin{image}
  \begin{tikzpicture}
    \begin{axis}[
        xmin=0,xmax=100,ymin=0,ymax=200,
        clip=true,
        unit vector ratio*=4 1 1,
        axis lines=center,
        grid = major,
        ytick={0,20,...,200},
		xtick={0,10,...,100},
    	yticklabels={0,20000,40000,60000,80000,100000,120000,140000,160000,180000,200000},
        xlabel=$t$, ylabel=$N$,
        every axis y label/.style={at=(current axis.above origin),anchor=south},
        every axis x label/.style={at=(current axis.right of origin),anchor=west},
      ]
      \addplot[very thick,penColor,domain=0:100,samples=50] plot{200/(1+100*e^(-0.1*x))};
      \end{axis}`
  \end{tikzpicture}
\end{image}

$N'(t)$ is increasing on the interval
\[
\left(\answer{0},\answer{10 \ln(100)}\right).
\]

The time at which the number of people who have contracted the cold is growing the fastest is
\[
t = \answer{10 \ln(100)}.
\]

The rate at which the number of people who have contracted the cold is changing is
\[
N'(t) = \answer{\frac{2000000e^{-0.1t}}{\left(1+100e^{-0.1t}\right)^2}}.
\]

The limit of $N'(t)$ as $t$ goes to $\infty$ is
\[
\lim_{t\to\infty}N'(t) = \answer{0}.
\]

Choose the best interpretation of the limit above.
\begin{multipleChoice}
\choice{The number of people who have contracted the cold will keep growing in the long run.}
\choice[correct]{The number of people who have contracted the cold will stop growing in the long run.}
\choice{The number of people who have contracted the cold will decrease in the long run.}
\choice{The number of people who have contracted the cold will oscillate in the long run.}
\end{multipleChoice}

The average rate that $N$ changed on the time interval $[5,6]$ rounded to one decimal is
\[
N_{\text{avg}} = \answer{335.1}.
\]

The instantaneous growth rate of $N$ at $t=5$ rounded to one decimal place is
\[
N_{\text{inst}} = \answer{319.1}.
\]

It seems that 
\[
N_{\text{avg}}\approx N_{\text{inst}}.
\]
This can be explained since
\[
N_{\text{inst}} = N'\left(\answer{5}\right) = \lim_{t\to\answer{5}}\frac{N(t)-N(5)}{t-5} \approx \frac{N(6) - N(\answer{5})}{6-\answer{5}} = N_{\text{avg}}.
\]

\end{exercise}
\end{document}