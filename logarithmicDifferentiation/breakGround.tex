\documentclass{ximera}


\graphicspath{
  {./}
  {ximeraTutorial/}
  {basicPhilosophy/}
}

\newcommand{\mooculus}{\textsf{\textbf{MOOC}\textnormal{\textsf{ULUS}}}}

\usepackage{tkz-euclide}\usepackage{tikz}
\usepackage{tikz-cd}
\usetikzlibrary{arrows}
\tikzset{>=stealth,commutative diagrams/.cd,
  arrow style=tikz,diagrams={>=stealth}} %% cool arrow head
\tikzset{shorten <>/.style={ shorten >=#1, shorten <=#1 } } %% allows shorter vectors

\usetikzlibrary{backgrounds} %% for boxes around graphs
\usetikzlibrary{shapes,positioning}  %% Clouds and stars
\usetikzlibrary{matrix} %% for matrix
\usepgfplotslibrary{polar} %% for polar plots
\usepgfplotslibrary{fillbetween} %% to shade area between curves in TikZ
\usetkzobj{all}
\usepackage[makeroom]{cancel} %% for strike outs
%\usepackage{mathtools} %% for pretty underbrace % Breaks Ximera
%\usepackage{multicol}
\usepackage{pgffor} %% required for integral for loops



%% http://tex.stackexchange.com/questions/66490/drawing-a-tikz-arc-specifying-the-center
%% Draws beach ball
\tikzset{pics/carc/.style args={#1:#2:#3}{code={\draw[pic actions] (#1:#3) arc(#1:#2:#3);}}}



\usepackage{array}
\setlength{\extrarowheight}{+.1cm}
\newdimen\digitwidth
\settowidth\digitwidth{9}
\def\divrule#1#2{
\noalign{\moveright#1\digitwidth
\vbox{\hrule width#2\digitwidth}}}






\DeclareMathOperator{\arccot}{arccot}
\DeclareMathOperator{\arcsec}{arcsec}
\DeclareMathOperator{\arccsc}{arccsc}

















%%This is to help with formatting on future title pages.
\newenvironment{sectionOutcomes}{}{}


\outcome{}

\title[Break-Ground:]{Multiplication to addition}

\begin{document}
\begin{abstract}
Two young mathematicians think about derivatives and logarithms.
\end{abstract}
\maketitle


Check out this dialogue between two calculus students (based on a true
story):

\begin{dialogue}
\item[Devyn] Riley, why is the product rule so much harder than the sum rule?
\item[Riley] Ever since 2nd grade, I've known that multiplication is
  \textit{harder} than addition.
\item[Devyn] I know! I was reading somewhere that a slide-rule somehow
  turns ``multiplication into addition.''
\item[Riley] Wow! I wonder how that works?
\item[Devyn] I \textit{think} it has something to do with logs?
\item[Riley] What? How does this work?
\end{dialogue}

Devyn is right, logarithms are used (and were invented) to convert
difficult multiplication problems into simpler addition problems.

\begin{problem}
  Let $f(x) =  \sqrt{x} \cos(x) \cdot e^x$. Compute
  \[
  \ddx f(x)\begin{prompt} = \answer{\frac{1}{2\sqrt{x}}\cos(x)e^x -  \sqrt{x}\sin(x)e^x +  \sqrt{x}\cos(x) e^x}\end{prompt}
  \]
\end{problem}

Now, let's see what happens if we do the same problem but we take the
natural log of both sides first:
\begin{align*}
  f(x) &= \sqrt{x}\cdot\cos(x)\cdot e^x\\
  \ln(f(x)) &= \ln( \sqrt{x}\cdot\cos(x)\cdot e^x)\\
  \ln(f(x)) &=\frac{1}{2}\ln{(x)} + \ln(\cos(x)) + \ln(e^x)
\end{align*}

Now we'll take the derivative of both sides of the equation.
By the chain rule
\[
\frac{d}{dx} \ln(f(x))= \frac{f'(x)}{f(x)}
\]


\begin{problem}
  Compute %(using the chain rule)%
  \[
  \frac{d}{dx} \ln(x)  \begin{prompt}=\answer{\frac{1}{x}}\end{prompt}
  \]
\end{problem}

\begin{problem}
  Compute %(using the chain rule)%
  \[
  \frac{d}{dx} \ln(\cos(x))  \begin{prompt}=\answer{\frac{-\sin(x)}{\cos(x)}}\end{prompt}
  \]
\end{problem}

\begin{problem}
  Compute %(using the chain rule)%
  \[
  \frac{d}{dx} \ln(e^x)  \begin{prompt}=\answer{1}\end{prompt}
  \]
\end{problem}

So we have
\begin{align*}
  \frac{f'(x)}{f(x)} &= \frac{1}{2x} - \frac{\sin(x)}{\cos(x)} + 1\\
  f'(x) &= f(x) \left(\frac{1}{2x} - \frac{\sin(x)}{\cos(x)} + 1\right)\\
  &= \sqrt{x}\cos(x)e^x\left(\frac{1}{2x} - \frac{\sin(x)}{\cos(x)} + 1\right)
\end{align*}



%\input{../leveledQuestions.tex}


\end{document}
