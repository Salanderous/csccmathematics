\documentclass{ximera}


\graphicspath{
  {./}
  {ximeraTutorial/}
  {basicPhilosophy/}
}

\newcommand{\mooculus}{\textsf{\textbf{MOOC}\textnormal{\textsf{ULUS}}}}

\usepackage{tkz-euclide}\usepackage{tikz}
\usepackage{tikz-cd}
\usetikzlibrary{arrows}
\tikzset{>=stealth,commutative diagrams/.cd,
  arrow style=tikz,diagrams={>=stealth}} %% cool arrow head
\tikzset{shorten <>/.style={ shorten >=#1, shorten <=#1 } } %% allows shorter vectors

\usetikzlibrary{backgrounds} %% for boxes around graphs
\usetikzlibrary{shapes,positioning}  %% Clouds and stars
\usetikzlibrary{matrix} %% for matrix
\usepgfplotslibrary{polar} %% for polar plots
\usepgfplotslibrary{fillbetween} %% to shade area between curves in TikZ
\usetkzobj{all}
\usepackage[makeroom]{cancel} %% for strike outs
%\usepackage{mathtools} %% for pretty underbrace % Breaks Ximera
%\usepackage{multicol}
\usepackage{pgffor} %% required for integral for loops



%% http://tex.stackexchange.com/questions/66490/drawing-a-tikz-arc-specifying-the-center
%% Draws beach ball
\tikzset{pics/carc/.style args={#1:#2:#3}{code={\draw[pic actions] (#1:#3) arc(#1:#2:#3);}}}



\usepackage{array}
\setlength{\extrarowheight}{+.1cm}
\newdimen\digitwidth
\settowidth\digitwidth{9}
\def\divrule#1#2{
\noalign{\moveright#1\digitwidth
\vbox{\hrule width#2\digitwidth}}}






\DeclareMathOperator{\arccot}{arccot}
\DeclareMathOperator{\arcsec}{arcsec}
\DeclareMathOperator{\arccsc}{arccsc}

















%%This is to help with formatting on future title pages.
\newenvironment{sectionOutcomes}{}{}


\outcome{To be able to use the method of substitution to solve some ``simple'' integrals, with an emphasis on being able to correctly identify what to substitute for.}
\outcome{Undo the Chain Rule.}
\outcome{Calculate indefinite integrals (antiderivatives) using basic substitution.}
\outcome{Calculate definite integrals using basic substitution.}
\title[Dig-In:]{The idea of substitution}
\begin{document}
\begin{abstract}
  We learn a new technique, called substitution, to help us solve
  problems involving integration.
\end{abstract}
\maketitle


Computing antiderivatives is not as easy as computing derivatives.
One issue is that the chain rule can be difficult to ``undo.''  We
have a general method called ``integration by substitution'' that will
somewhat help with this difficulty. 

If the functions are differentiable, we can apply the chain rule and obtain
\[
\frac{d}{dx} f(g(x)) = f'(g(x))g'(x)
\]
If the derivatives are continuous, we can use this equality to evaluate a definite integral. Namely,
\begin{align*}
  \int_a^b f'(g(x))g'(x) dx &= \bigg[ f(g(x)) \bigg]_a^b \\
  &= f(g(b)) - f(g(a)). \\
 \end{align*}
On the other hand, it is also true that
\begin{align*}
  \int_{g(a)}^{g(b)} f'(u) du &= \bigg[ f(u) \bigg]_{g(a)}^{g(b)}\\
  &= f(g(b)) - f(g(a)). \\
 \end{align*}
 Since the right hand sides of these equalities are equal,  the left hand sides must be equal, too. So, it follows that
 
\[
  \int_a^b f'(g(x))g'(x) dx =  \int_{g(a)}^{g(b)} f'(u) du .
\]

 






















\end{document}
