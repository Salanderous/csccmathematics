\documentclass{ximera}


\graphicspath{
  {./}
  {ximeraTutorial/}
  {basicPhilosophy/}
}

\newcommand{\mooculus}{\textsf{\textbf{MOOC}\textnormal{\textsf{ULUS}}}}

\usepackage{tkz-euclide}\usepackage{tikz}
\usepackage{tikz-cd}
\usetikzlibrary{arrows}
\tikzset{>=stealth,commutative diagrams/.cd,
  arrow style=tikz,diagrams={>=stealth}} %% cool arrow head
\tikzset{shorten <>/.style={ shorten >=#1, shorten <=#1 } } %% allows shorter vectors

\usetikzlibrary{backgrounds} %% for boxes around graphs
\usetikzlibrary{shapes,positioning}  %% Clouds and stars
\usetikzlibrary{matrix} %% for matrix
\usepgfplotslibrary{polar} %% for polar plots
\usepgfplotslibrary{fillbetween} %% to shade area between curves in TikZ
\usetkzobj{all}
\usepackage[makeroom]{cancel} %% for strike outs
%\usepackage{mathtools} %% for pretty underbrace % Breaks Ximera
%\usepackage{multicol}
\usepackage{pgffor} %% required for integral for loops



%% http://tex.stackexchange.com/questions/66490/drawing-a-tikz-arc-specifying-the-center
%% Draws beach ball
\tikzset{pics/carc/.style args={#1:#2:#3}{code={\draw[pic actions] (#1:#3) arc(#1:#2:#3);}}}



\usepackage{array}
\setlength{\extrarowheight}{+.1cm}
\newdimen\digitwidth
\settowidth\digitwidth{9}
\def\divrule#1#2{
\noalign{\moveright#1\digitwidth
\vbox{\hrule width#2\digitwidth}}}






\DeclareMathOperator{\arccot}{arccot}
\DeclareMathOperator{\arcsec}{arcsec}
\DeclareMathOperator{\arccsc}{arccsc}

















%%This is to help with formatting on future title pages.
\newenvironment{sectionOutcomes}{}{}


\outcome{Determine when a function is a composition of two or more functions.}
\outcome{Calculate indefinite and definite integrals requiring complicated substitutions.}
\outcome{Recognize common patterns in substitutions.}
\outcome{Evaluate indefinite and definite integrals through a change of variables.}

\author{Nela Lakos \and Kyle Parsons}

\usetikzlibrary{fillbetween}

\begin{document}
\begin{exercise}

Consider the two regions depicted in the graphs below.

\begin{image}
  \begin{tikzpicture}
    \begin{axis}[
        xmin=-0.3,xmax=1.8,ymin=-0.3,ymax=1.3,
        clip=true,
        unit vector ratio*=1 1 1,
        axis lines=center,
        grid = major,
        ytick={-1,-0.5,...,36},
        xtick={0,0.5,...,10},
        xlabel=$x$, ylabel=$y$,
        every axis y label/.style={at=(current axis.above origin),anchor=south},
        every axis x label/.style={at=(current axis.right of origin),anchor=west},
      ]
      \addplot[fill=penColor2,draw=none,fill opacity=0.3,domain=0:{sqrt(pi/2)},samples=50] {x*cos(x^2*180/pi)};   
      \addplot[ultra thick,penColor,domain=0:{sqrt(pi/2)},samples=50] {x*cos(x^2*180/pi)};   

	  \node at (axis cs:0.75,0.25) {\Large $A$};         
      \node at (axis cs:0.75,0.75) {$y=x \cos(x^2)$};
      \end{axis}`
  \end{tikzpicture}
  \begin{tikzpicture}
	  \begin{axis}[
        xmin=-0.3,xmax=1.8,ymin=-0.3,ymax=1.3,
        clip=true,
        unit vector ratio*=1 1 1,
        axis lines=center,
        grid = major,
        ytick={-1,-0.5,...,36},
        xtick={0,0.5,...,10},
        xlabel=$u$, ylabel=$y$,
        every axis y label/.style={at=(current axis.above origin),anchor=south},
        every axis x label/.style={at=(current axis.right of origin),anchor=west},
      ]
      \addplot[fill=penColor2,fill opacity=0.3,draw=none,domain=0:{pi/2},samples=50] {cos(x*180/pi)/2} \closedcycle;
      \addplot[ultra thick,draw=penColor,domain=0:{pi/2},samples=50] {cos(x*180/pi)/2};

	  \node at (axis cs:0.25,0.25) {\Large $B$};         
      \node at (axis cs:0.75,0.75) {$y=\frac{1}{2}\cos(u)$};
      \end{axis}`
  \end{tikzpicture}
\end{image}

We will see that the regions $A$ and $B$ have the same area.

First the area of $A$ is given by the integral
\[
\int_{\answer{0}}^{\answer{\sqrt{\pi/2}}} \answer{x \cos(x^2)} dx.
\]
And the area of $B$ is given by 
\[
\int_{\answer{0}}^{\answer{\pi/2}} \answer{\frac{1}{2}\cos(u)} du.
\]

Now if we wanted to compute the area of $A$ we could perform a substitution with $u=\answer{x^2}$. In this case $du=\answer{2x}dx$.  The lower bound is $g(0)=\answer{0}$ and our upper bound   is $g(\sqrt{\pi/2})=\answer{\pi/2}$. Making all these substitutions, we can rewrite our integral that computes the area of $A$ as
\[
\int_{\answer{0}}^{\answer{\pi/2}} \answer{\frac{1}{2}\cos(u)}du
\]
which is exactly the integral that computes the area of $B$.  Thus we see that $A$ and $B$ have the same area.

\end{exercise}
\end{document}
