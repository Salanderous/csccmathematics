\documentclass{ximera}


\graphicspath{
  {./}
  {ximeraTutorial/}
  {basicPhilosophy/}
}

\newcommand{\mooculus}{\textsf{\textbf{MOOC}\textnormal{\textsf{ULUS}}}}

\usepackage{tkz-euclide}\usepackage{tikz}
\usepackage{tikz-cd}
\usetikzlibrary{arrows}
\tikzset{>=stealth,commutative diagrams/.cd,
  arrow style=tikz,diagrams={>=stealth}} %% cool arrow head
\tikzset{shorten <>/.style={ shorten >=#1, shorten <=#1 } } %% allows shorter vectors

\usetikzlibrary{backgrounds} %% for boxes around graphs
\usetikzlibrary{shapes,positioning}  %% Clouds and stars
\usetikzlibrary{matrix} %% for matrix
\usepgfplotslibrary{polar} %% for polar plots
\usepgfplotslibrary{fillbetween} %% to shade area between curves in TikZ
\usetkzobj{all}
\usepackage[makeroom]{cancel} %% for strike outs
%\usepackage{mathtools} %% for pretty underbrace % Breaks Ximera
%\usepackage{multicol}
\usepackage{pgffor} %% required for integral for loops



%% http://tex.stackexchange.com/questions/66490/drawing-a-tikz-arc-specifying-the-center
%% Draws beach ball
\tikzset{pics/carc/.style args={#1:#2:#3}{code={\draw[pic actions] (#1:#3) arc(#1:#2:#3);}}}



\usepackage{array}
\setlength{\extrarowheight}{+.1cm}
\newdimen\digitwidth
\settowidth\digitwidth{9}
\def\divrule#1#2{
\noalign{\moveright#1\digitwidth
\vbox{\hrule width#2\digitwidth}}}






\DeclareMathOperator{\arccot}{arccot}
\DeclareMathOperator{\arcsec}{arcsec}
\DeclareMathOperator{\arccsc}{arccsc}

















%%This is to help with formatting on future title pages.
\newenvironment{sectionOutcomes}{}{}


\outcome{}

\title[Break-Ground:]{Geometry and substitution}

\begin{document}
\begin{abstract}
Two students consider substitution geometrically.
\end{abstract}
\maketitle


Check out this dialogue between two calculus students (based on a true
story):

\begin{dialogue}
\item[Devyn] Riley! We should be able to figure some integrals
  geometrically using transformations of functions.
\item[Riley] That sounds like a cool idea.  Maybe, since we know the
  graph of $f(x) = \sqrt{1-x^2}$ is a semicircle, we get an ellipse
  defined on $[-2,2]$ just by stretching the graph of $f$ by a factor
  of $2$ horizontally.  The equation of this ellipse would be
  \[
  g(x) =\sqrt{1-\left(\frac{x}{2}\right)^2}
  \]
\item[Devyn] Exactly!  So since we know that
  \[
  \int_{-1}^1 \sqrt{1-x^2} dx = \frac{\pi}{2}
  \]
  geometrically\dots
\item[Riley] And we know that the area under $g$ from $[-2,2]$ is
  twice the under $f$\dots
\item[Devyn and Riley] We must have
  \[
  \int_{-2}^2 \sqrt{1-(x/2)^2} dx =\pi !
  \]
\item[Devyn and Riley] Jinx!
\item[Devyn] It is kind of like we just stretched out our whole
  coordinate system, and that helped us solve an integral.
\item[Riley] In this case, everything got stretched out by a constant
  factor of $2$ in the horizontal direction.  I wonder if we could
  ever say anything useful about cases where we stretch the $x$-axis
  by a different amount at each point?
\item[Devyn] Whao, that is a wild thought.  That seems really hard.
  Since derivatives measure how much a function stretches a little
  piece of the domain, maybe the derivative will come into play here?
\item[Riley] Hmmmm, but I do not see exactly how.  Maybe we should ask
  our TA about this?
\end{dialogue}

%%Q stretch ellipse just like above

\begin{problem}
  Say we know that
  \[
  \int_1^4 f(x) dx = 5.
  \]
  Then, using this transformation idea, we can evaluate
  \[
  \int_a^b f(3x+1) dx
  \]
  if $a= \answer{0}$ and $b=\answer{1}$.  The value of the integral on
  this interval is $\answer{5/3}$.
  \begin{feedback}
    Since we are ``squishing'' the integrand, the integral has a value
    of $5/3$. As food for thought, how do we change the integrand so
    that the two integrals above are equal?
  \end{feedback}
\end{problem}

%\input{../leveledQuestions.tex}


\end{document}
