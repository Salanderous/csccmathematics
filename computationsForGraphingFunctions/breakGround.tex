\documentclass{ximera}


\graphicspath{
  {./}
  {ximeraTutorial/}
  {basicPhilosophy/}
}

\newcommand{\mooculus}{\textsf{\textbf{MOOC}\textnormal{\textsf{ULUS}}}}

\usepackage{tkz-euclide}\usepackage{tikz}
\usepackage{tikz-cd}
\usetikzlibrary{arrows}
\tikzset{>=stealth,commutative diagrams/.cd,
  arrow style=tikz,diagrams={>=stealth}} %% cool arrow head
\tikzset{shorten <>/.style={ shorten >=#1, shorten <=#1 } } %% allows shorter vectors

\usetikzlibrary{backgrounds} %% for boxes around graphs
\usetikzlibrary{shapes,positioning}  %% Clouds and stars
\usetikzlibrary{matrix} %% for matrix
\usepgfplotslibrary{polar} %% for polar plots
\usepgfplotslibrary{fillbetween} %% to shade area between curves in TikZ
\usetkzobj{all}
\usepackage[makeroom]{cancel} %% for strike outs
%\usepackage{mathtools} %% for pretty underbrace % Breaks Ximera
%\usepackage{multicol}
\usepackage{pgffor} %% required for integral for loops



%% http://tex.stackexchange.com/questions/66490/drawing-a-tikz-arc-specifying-the-center
%% Draws beach ball
\tikzset{pics/carc/.style args={#1:#2:#3}{code={\draw[pic actions] (#1:#3) arc(#1:#2:#3);}}}



\usepackage{array}
\setlength{\extrarowheight}{+.1cm}
\newdimen\digitwidth
\settowidth\digitwidth{9}
\def\divrule#1#2{
\noalign{\moveright#1\digitwidth
\vbox{\hrule width#2\digitwidth}}}






\DeclareMathOperator{\arccot}{arccot}
\DeclareMathOperator{\arcsec}{arcsec}
\DeclareMathOperator{\arccsc}{arccsc}

















%%This is to help with formatting on future title pages.
\newenvironment{sectionOutcomes}{}{}


\outcome{Determine how the graph of a function looks without using a calculator.}

% Synthesis of everything we've learned

% We've learned so much stuff!
% Can't we put this all together somehow?

% Let's make a concept map of everything we've learned so far.

% Holes, point discontinuities, review from limits at the beginning


\title[Break-Ground:]{Wanted: graphing procedure}

\begin{document}
\begin{abstract}
Two young mathematicians discuss how to sketch the graphs of functions.
\end{abstract}
\maketitle

Check out this dialogue between two calculus students (based on a true
story):

\begin{dialogue}
\item[Devyn] Riley, OK I know how to plot something if I'm given a description.
\item[Riley] Yes, it's kinda fun right?
\item[Devyn] I know!  But now I'm not sure how to get the information I need.
\item[Riley] You know, I'd like to make up a procedure based on all
  these facts, that would tell me what the graph of any function would look like.
\item[Devyn] Me too! Let's get to work!
\end{dialogue}



\begin{problem}%% BADBAD This is a good problem. How do we do it?
  Below is a list of features of a graph of a function.
  \begin{enumerate}
  \item Find the numbers $x=a$ where
    $f(x)$ goes to infinity as $x$ goes to $a$ (from the right, left,
    or both). These are the numbers where the graph of $f$ has a  vertical asymptote.
  \item Find the critical numbers (the numbers where $f'(x) = 0$ or
    $f'(x)$ is undefined).
 
  \item Identify inflection points and concavity ofr the graph.
  \item Determine an interval that shows all relevant behavior.
  \item Find the candidates for inflection points, the points on the graph corresponding to where
    $f''(x) = 0$ or $f''(x)$ is undefined.
  \item If possible, find the $x$-intercepts, the points corresponding to where $f(x) =
    0$. Place these points on your graph.
       \item Compute $f'$ and $f''$.
  \item Analyze end behavior:  as $x \to \pm \infty$, what happens to the graph of $f$?  Does it  have horizontal asymptotes, increase or decrease without bound, or have some other kind of behavior?
 \end{enumerate}
   \item Use either the first or second derivative test to identify local extrema and/or
    find the intervals where your function is increasing/decreasing.
  In what order should we take these steps? For example, one must compute
   $f'$ before computing $f''$. Also, one must compute $f'$ before 
   finding the critical numbers. There is more than one correct answer.
  \begin{freeResponse}
  Here is one possible answer to this question.  Compare it with yours!
  \begin{enumerate}
  \item Find the $y$-intercept, this is the point $(0,f(0))$. Place this
    point on your graph.
  \item Find any vertical asymptotes, these correspond to numbers $x=a$ where
    $f(x)$ goes to infinity as $x$ goes to $a$ (from the right, left,
    or both).
  \item If possible, find the $x$-intercepts, the points corresponding to where $f(x) =
    0$. Place these points on your graph.
  \item Analyze end behavior:  as $x \to \pm \infty$, what happens to the graph of $f$?  Does it  have horizontal asymptotes, increase or decrease without bound, or have some other kind of behavior?
    \item Compute $f'(x)$ and $f''(x)$.
    \item Find the critical numbers (the numbers where $f'(x) = 0$ or
    $f'(x)$ is undefined).
      \item Use either the first or second derivative test to identify local extrema and/or
    find the intervals where your function is increasing/decreasing.
      \item Find the candidates for inflection points, the points corresponding to where
    $f''(x) = 0$ or $f''(x)$ is undefined.
  \item Identify inflection points and concavity.
  \item Determine an interval that shows all relevant behavior
  \end{enumerate}
  \end{freeResponse}
\end{problem}


%\input{../leveledQuestions.tex}
\end{document}
