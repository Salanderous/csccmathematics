\documentclass{ximera}


\graphicspath{
  {./}
  {ximeraTutorial/}
  {basicPhilosophy/}
}

\newcommand{\mooculus}{\textsf{\textbf{MOOC}\textnormal{\textsf{ULUS}}}}

\usepackage{tkz-euclide}\usepackage{tikz}
\usepackage{tikz-cd}
\usetikzlibrary{arrows}
\tikzset{>=stealth,commutative diagrams/.cd,
  arrow style=tikz,diagrams={>=stealth}} %% cool arrow head
\tikzset{shorten <>/.style={ shorten >=#1, shorten <=#1 } } %% allows shorter vectors

\usetikzlibrary{backgrounds} %% for boxes around graphs
\usetikzlibrary{shapes,positioning}  %% Clouds and stars
\usetikzlibrary{matrix} %% for matrix
\usepgfplotslibrary{polar} %% for polar plots
\usepgfplotslibrary{fillbetween} %% to shade area between curves in TikZ
\usetkzobj{all}
\usepackage[makeroom]{cancel} %% for strike outs
%\usepackage{mathtools} %% for pretty underbrace % Breaks Ximera
%\usepackage{multicol}
\usepackage{pgffor} %% required for integral for loops



%% http://tex.stackexchange.com/questions/66490/drawing-a-tikz-arc-specifying-the-center
%% Draws beach ball
\tikzset{pics/carc/.style args={#1:#2:#3}{code={\draw[pic actions] (#1:#3) arc(#1:#2:#3);}}}



\usepackage{array}
\setlength{\extrarowheight}{+.1cm}
\newdimen\digitwidth
\settowidth\digitwidth{9}
\def\divrule#1#2{
\noalign{\moveright#1\digitwidth
\vbox{\hrule width#2\digitwidth}}}






\DeclareMathOperator{\arccot}{arccot}
\DeclareMathOperator{\arcsec}{arcsec}
\DeclareMathOperator{\arccsc}{arccsc}

















%%This is to help with formatting on future title pages.
\newenvironment{sectionOutcomes}{}{}


\outcome{State the Intermediate Value Theorem including hypotheses.}

\author{Nela Lakos \and Kyle Parsons}


\begin{document}
\begin{exercise}


The figure below shows the graph of a function $f$.

\begin{image}
  \begin{tikzpicture}
    \begin{axis}[
        xmin=0,xmax=5.3,ymin=-1.3,ymax=4.3,
        clip=false,
        unit vector ratio*=1 1 1,
        axis lines=center,
        grid = major,
        ytick={-1,-0,...,4},
    xtick={1,2,...,5},
        xlabel=$x$, ylabel=$y$,
        every axis y label/.style={at=(current axis.above origin),anchor=south},
        every axis x label/.style={at=(current axis.right of origin),anchor=west},
      ]
      \addplot[very thick, penColor, domain=0.7:2] {x+2};
      \addplot[very thick, penColor, domain=2:3] {4-2*sqrt(x-2)};
      \addplot[very thick, penColor, domain=3:4] {sqrt(x-3)+2};
      \addplot[very thick, penColor, domain=4:5.3] {-2*x+11};

      \node[penColor] at (axis cs:2, 1.2) [penColor] {$y=f(x)$};
      \end{axis}`
  \end{tikzpicture}
\end{image}





Select all the following statements that are correct.

\begin{selectAll}
\choice [correct]{When we apply the Intermediate value theorem to $f$ on the interval $\left[1,5\right]$ with $L=2$, we are guaranteed the existence of at least one point $c$ such that $f(c)=L$.}
\choice {When we apply the Intermediate value theorem to $f$ on the interval $\left[1,4\right]$ with $L=2$, we are guaranteed the existence of at least one point $c$ such that $f(c)=L$.}
\choice {When we apply the Intermediate value theorem to $f$ on the interval $\left[1,5\right]$ with $L=4$, we are guaranteed the existence of at least one point $c$ such that $f(c)=L$.}
\choice {When we apply the Intermediate value theorem to $f$ on the interval $\left[3,5\right]$ with $L=3$, we are guaranteed the existence of at least one point $c$ such that $f(c)=L$.}
\choice[correct] {When we apply the Intermediate value theorem to $f$ on the interval $\left[2,5\right]$ with $L=\pi$, we are guaranteed the existence of at least one point $c$ such that $f(c)=L$.}
\choice[correct] {When we apply the Intermediate value theorem to $f$ on the interval $\left[2,4\right]$ with $L=\pi$, we are guaranteed the existence of at least one point $c$ such that $f(c)=L$.}
\choice {When we apply the Intermediate value theorem to $f$ on the interval $\left[1,5\right]$ with $L=\pi$, we are guaranteed the existence of at least one point $c$ such that $f(c)=L$.}
\end{selectAll}



\end{exercise}
\end{document}