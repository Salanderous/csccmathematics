\documentclass{ximera}


\graphicspath{
  {./}
  {ximeraTutorial/}
  {basicPhilosophy/}
}

\newcommand{\mooculus}{\textsf{\textbf{MOOC}\textnormal{\textsf{ULUS}}}}

\usepackage{tkz-euclide}\usepackage{tikz}
\usepackage{tikz-cd}
\usetikzlibrary{arrows}
\tikzset{>=stealth,commutative diagrams/.cd,
  arrow style=tikz,diagrams={>=stealth}} %% cool arrow head
\tikzset{shorten <>/.style={ shorten >=#1, shorten <=#1 } } %% allows shorter vectors

\usetikzlibrary{backgrounds} %% for boxes around graphs
\usetikzlibrary{shapes,positioning}  %% Clouds and stars
\usetikzlibrary{matrix} %% for matrix
\usepgfplotslibrary{polar} %% for polar plots
\usepgfplotslibrary{fillbetween} %% to shade area between curves in TikZ
\usetkzobj{all}
\usepackage[makeroom]{cancel} %% for strike outs
%\usepackage{mathtools} %% for pretty underbrace % Breaks Ximera
%\usepackage{multicol}
\usepackage{pgffor} %% required for integral for loops



%% http://tex.stackexchange.com/questions/66490/drawing-a-tikz-arc-specifying-the-center
%% Draws beach ball
\tikzset{pics/carc/.style args={#1:#2:#3}{code={\draw[pic actions] (#1:#3) arc(#1:#2:#3);}}}



\usepackage{array}
\setlength{\extrarowheight}{+.1cm}
\newdimen\digitwidth
\settowidth\digitwidth{9}
\def\divrule#1#2{
\noalign{\moveright#1\digitwidth
\vbox{\hrule width#2\digitwidth}}}






\DeclareMathOperator{\arccot}{arccot}
\DeclareMathOperator{\arcsec}{arcsec}
\DeclareMathOperator{\arccsc}{arccsc}

















%%This is to help with formatting on future title pages.
\newenvironment{sectionOutcomes}{}{}


\outcome{Simplify expression}

\author{Nela Lakos}
\begin{document}
\begin{exercise}

Simplify the  expression $$ \frac{\frac{4}{x-2}+1}{\frac{x}{x+3}-\frac{4}{x-2}}.$$ Show your work as indicated below.

First, we multiply the numerator and the denominator by  the lowest common denominator

	 \[
	 \frac{\frac{4}{x-2}+1}{\frac{x}{x+3}-\frac{4}{x-2}}= \frac{\frac{4}{x-2}+1}{\frac{x}{x+3}-\frac{4}{x-2}}\cdot \frac{(x+3)(\answer{x-2})}{(x+3)(\answer{x-2})},
	 \]
	and obtain that  
\[
\frac{\frac{4}{x-2}+1}{\frac{x}{x+3}-\frac{4}{x-2}}= \frac{4(\answer{x+3})+1(x+3)(\answer{x-2})}{x(\answer{x-2})-4(\answer{x+3})}.
\]
Then we simplify the numerator and the denominator.
\[
\frac{\frac{4}{x-2}+1}{\frac{x}{x+3}-\frac{4}{x-2}}= \frac{\answer{1}x^2+\answer{5}x+\answer{6}}{\answer{1}x^2+\answer{-6}x+\answer{-12}}.
\]
\end{exercise}
\end{document}
