\documentclass{ximera}


\graphicspath{
  {./}
  {ximeraTutorial/}
  {basicPhilosophy/}
}

\newcommand{\mooculus}{\textsf{\textbf{MOOC}\textnormal{\textsf{ULUS}}}}

\usepackage{tkz-euclide}\usepackage{tikz}
\usepackage{tikz-cd}
\usetikzlibrary{arrows}
\tikzset{>=stealth,commutative diagrams/.cd,
  arrow style=tikz,diagrams={>=stealth}} %% cool arrow head
\tikzset{shorten <>/.style={ shorten >=#1, shorten <=#1 } } %% allows shorter vectors

\usetikzlibrary{backgrounds} %% for boxes around graphs
\usetikzlibrary{shapes,positioning}  %% Clouds and stars
\usetikzlibrary{matrix} %% for matrix
\usepgfplotslibrary{polar} %% for polar plots
\usepgfplotslibrary{fillbetween} %% to shade area between curves in TikZ
\usetkzobj{all}
\usepackage[makeroom]{cancel} %% for strike outs
%\usepackage{mathtools} %% for pretty underbrace % Breaks Ximera
%\usepackage{multicol}
\usepackage{pgffor} %% required for integral for loops



%% http://tex.stackexchange.com/questions/66490/drawing-a-tikz-arc-specifying-the-center
%% Draws beach ball
\tikzset{pics/carc/.style args={#1:#2:#3}{code={\draw[pic actions] (#1:#3) arc(#1:#2:#3);}}}



\usepackage{array}
\setlength{\extrarowheight}{+.1cm}
\newdimen\digitwidth
\settowidth\digitwidth{9}
\def\divrule#1#2{
\noalign{\moveright#1\digitwidth
\vbox{\hrule width#2\digitwidth}}}






\DeclareMathOperator{\arccot}{arccot}
\DeclareMathOperator{\arcsec}{arcsec}
\DeclareMathOperator{\arccsc}{arccsc}

















%%This is to help with formatting on future title pages.
\newenvironment{sectionOutcomes}{}{}


\outcome{Understand the connection between continuity of a function and the value of a limit.}

\author{Nela Lakos \and Kyle Parsons}

\begin{document}
\begin{exercise}

For the following statements select True if the statement is always true, and select False if the statement can be false.

Let $f$ and $g$ be two functions defined on $\left(-1,1\right)$. If $f$ is continuous on $\left(-1,1\right)$ and if $\lim_{x\to0}g(x)=0$ then $\lim_{x\to0}\left(f(x)g(x)\right) = 0$.

\begin{multipleChoice}
\choice[correct]{True}
\choice{False}
\end{multipleChoice}

\begin{feedback}
Since $f$ is continuous on $\left(-1,1\right)$ we know $\lim_{x\to0}f(x)$ exists.  This allows us to use the limit product law to compute the desired limit.
\end{feedback}

\begin{exercise}

Let $f$ and $g$ be two functions defined on $\left(-1,1\right)$. If $f$ is continuous on $\left(-1,1\right)$ and if $\lim_{x\to0}g(x)=0$ then $\lim_{x\to0}\frac{g(x)}{f(x)} = 0$.

\begin{multipleChoice}
\choice{True}
\choice[correct]{False}
\end{multipleChoice}

\begin{feedback}
For example, take $f(x)=g(x)$ in which case the limit is 1.
\end{feedback}

\begin{exercise}

Let $f$ and $g$ be two functions defined on $\left(-1,1\right)$. If $f$ is continuous on $\left(-1,1\right)$ and if $\lim_{x\to0}g(x)=0$ then $\lim_{x\to0}g(f(x)) = 0$.

\begin{multipleChoice}
\choice{True}
\choice[correct]{False}
\end{multipleChoice}

\begin{feedback}
For example, take $g(x) = x$ and $f(x) = \frac{1}{2}$ in which case the limit is $\frac{1}{2}$.
\end{feedback}

\begin{exercise}

If $f$ is continuous on $\left[1,3\right]$, and if $f(1) = 4$ and $f(3) = 0$, the the equation $f(x) = e$ has a solution in $\left(1,3\right)$.

\begin{multipleChoice}
\choice[correct]{True}
\choice{False}
\end{multipleChoice}

\begin{feedback}
$f$ is continuous on $\left[1,3\right]$ and $e$ is between $f(1)$ and $f(3)$ so we can apply the Intermediate Value Theorem.
\end{feedback}

\begin{exercise}

If $f$ is continuous on $\left[1,3\right]$, and if $f(1) = e$ and $f(3) = 0$, the the equation $f(x) = \pi$ has a solution in $\left(1,3\right)$.

\begin{multipleChoice}
\choice{True}
\choice[correct]{False}
\end{multipleChoice}

\begin{feedback}
$\pi$ is not between $f(1)$ and $f(3)$ so we cannot apply the intermediate value theorem.  Consider, as a counterexample, $f$ being the line between $(1,e)$ and $(3,0)$.
\end{feedback}

\end{exercise}
\end{exercise}
\end{exercise}
\end{exercise}
\end{exercise}
\end{document}