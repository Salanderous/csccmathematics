\documentclass{ximera}


\graphicspath{
  {./}
  {ximeraTutorial/}
  {basicPhilosophy/}
}

\newcommand{\mooculus}{\textsf{\textbf{MOOC}\textnormal{\textsf{ULUS}}}}

\usepackage{tkz-euclide}\usepackage{tikz}
\usepackage{tikz-cd}
\usetikzlibrary{arrows}
\tikzset{>=stealth,commutative diagrams/.cd,
  arrow style=tikz,diagrams={>=stealth}} %% cool arrow head
\tikzset{shorten <>/.style={ shorten >=#1, shorten <=#1 } } %% allows shorter vectors

\usetikzlibrary{backgrounds} %% for boxes around graphs
\usetikzlibrary{shapes,positioning}  %% Clouds and stars
\usetikzlibrary{matrix} %% for matrix
\usepgfplotslibrary{polar} %% for polar plots
\usepgfplotslibrary{fillbetween} %% to shade area between curves in TikZ
\usetkzobj{all}
\usepackage[makeroom]{cancel} %% for strike outs
%\usepackage{mathtools} %% for pretty underbrace % Breaks Ximera
%\usepackage{multicol}
\usepackage{pgffor} %% required for integral for loops



%% http://tex.stackexchange.com/questions/66490/drawing-a-tikz-arc-specifying-the-center
%% Draws beach ball
\tikzset{pics/carc/.style args={#1:#2:#3}{code={\draw[pic actions] (#1:#3) arc(#1:#2:#3);}}}



\usepackage{array}
\setlength{\extrarowheight}{+.1cm}
\newdimen\digitwidth
\settowidth\digitwidth{9}
\def\divrule#1#2{
\noalign{\moveright#1\digitwidth
\vbox{\hrule width#2\digitwidth}}}






\DeclareMathOperator{\arccot}{arccot}
\DeclareMathOperator{\arcsec}{arcsec}
\DeclareMathOperator{\arccsc}{arccsc}

















%%This is to help with formatting on future title pages.
\newenvironment{sectionOutcomes}{}{}


\outcome{Understand the necessity of continuity for the Intermediate Value Theorem.}
\outcome{Determine if the Intermediate Value Theorem applies.}

\title[Break-Ground:]{Roxy and Yuri like food}

\begin{document}
\begin{abstract}
Two young mathematicians discuss the eating habits of their cats.
\end{abstract}
\maketitle

Check out this dialogue between two calculus students (based on a true
story):

\begin{dialogue}
\item[Devyn] Yo Riley, I was watching my two cats
  \textit{Roxy} and \textit{Yuri} eat their dry cat food last night.
\item[Riley] Cats love food!  It's so weird that they swallow the pieces whole!
\item[Devyn] I know! I noticed something else kinda funny though:
  Both Roxy and Yuri start and finish eating at the same times; and
  while I gave Roxy a little more food than Yuri, less food was left
  in Roxy's bowl when they stopped eating.

  I wonder, is there is a point in time when Roxy and Yuri have the
  exact same amount of \textbf{dry cat food} in their bowls?
\item[Riley] Hmmmmm. Do Roxy and Yuri both start and finish
  drinking their water at the same times?  And does Roxy start with a
  little more water than Yuri, and finish with less water left than
  Yuri?
\item[Devyn] Yes!
\item[Riley] Interesting. I wonder, is there is a point in
  time when Roxy and Yuri have the exact same amount of \textbf{water}
  in their bowls?
\end{dialogue}

\begin{problem}
  Is there a time when Roxy and Yuri have the same amount of dry cat
  food in their bowls? Make the following assumptions:
  \begin{itemize}
  \item They start and finish eating at the same times.
  \item Roxy starts with more food than Yuri, and leaves less food uneaten than Yuri. 
  \end{itemize}
  \begin{hint}
  	You might want to try drawing a graph of this situation.
  \end{hint}
  \begin{prompt}
  \begin{multipleChoice}
    \choice{yes}
    \choice{no}
    \choice[correct]{There is no way to tell.}
  \end{multipleChoice}
  \end{prompt}
\end{problem}

\begin{problem}
  Is there a time when Roxy and Yuri have the same amount of water in
  their bowls? Make the following assumptions:
  \begin{itemize}
  \item They start and finish drinking at the same times.
  \item Roxy starts with more water than Yuri, and leaves less water
    left in her bowl than Yuri.
  \end{itemize}
    \begin{hint}
  	You might want to try drawing a graph of this situation.
    \end{hint}
    \begin{prompt}
    \begin{multipleChoice}
    \choice[correct]{yes}
    \choice{no}
    \choice{There is no way to tell.}
    \end{multipleChoice}
    \end{prompt}
\end{problem}



\begin{problem}
  Within the context of the two problems above, what is the difference
  between ``dry cat food'' and ``water?''
  \begin{freeResponse}
    If we write the amount of dry cat food as a function of time, this function
    is not continuous.  The reason it isn't continuous is that the dry cat food
    is a collection of individual kibbles, which are eaten whole.
    
    On the other hand, if we write the amount of water as a function of time, 
    this function is continuous.
  \end{freeResponse}
\end{problem}

%\input{../leveledQuestions.tex}


\end{document}
