\documentclass{ximera}

\outcome{Calculate limits of the form number over zero.}
\outcome{Identify determinate and indeterminate forms.}
\outcome{Distinguish between determinate and indeterminate forms}


\graphicspath{
  {./}
  {ximeraTutorial/}
  {basicPhilosophy/}
}

\newcommand{\mooculus}{\textsf{\textbf{MOOC}\textnormal{\textsf{ULUS}}}}

\usepackage{tkz-euclide}\usepackage{tikz}
\usepackage{tikz-cd}
\usetikzlibrary{arrows}
\tikzset{>=stealth,commutative diagrams/.cd,
  arrow style=tikz,diagrams={>=stealth}} %% cool arrow head
\tikzset{shorten <>/.style={ shorten >=#1, shorten <=#1 } } %% allows shorter vectors

\usetikzlibrary{backgrounds} %% for boxes around graphs
\usetikzlibrary{shapes,positioning}  %% Clouds and stars
\usetikzlibrary{matrix} %% for matrix
\usepgfplotslibrary{polar} %% for polar plots
\usepgfplotslibrary{fillbetween} %% to shade area between curves in TikZ
\usetkzobj{all}
\usepackage[makeroom]{cancel} %% for strike outs
%\usepackage{mathtools} %% for pretty underbrace % Breaks Ximera
%\usepackage{multicol}
\usepackage{pgffor} %% required for integral for loops



%% http://tex.stackexchange.com/questions/66490/drawing-a-tikz-arc-specifying-the-center
%% Draws beach ball
\tikzset{pics/carc/.style args={#1:#2:#3}{code={\draw[pic actions] (#1:#3) arc(#1:#2:#3);}}}



\usepackage{array}
\setlength{\extrarowheight}{+.1cm}
\newdimen\digitwidth
\settowidth\digitwidth{9}
\def\divrule#1#2{
\noalign{\moveright#1\digitwidth
\vbox{\hrule width#2\digitwidth}}}






\DeclareMathOperator{\arccot}{arccot}
\DeclareMathOperator{\arcsec}{arcsec}
\DeclareMathOperator{\arccsc}{arccsc}

















%%This is to help with formatting on future title pages.
\newenvironment{sectionOutcomes}{}{}


\title[Dig-In:]{Limits of the form nonzero over zero}

\begin{document}
\begin{abstract}
  What can be said about limits that have the form nonzero over zero?
\end{abstract}

\maketitle

Let's cut to the chase:

\begin{definition}
  A limit
  \[
  \lim_{x\to a} \frac{f(x)}{g(x)}
  \]
  is said to be of the form $\boldsymbol{\tfrac{\#}{0}}$\ if
  \[
  \lim_{x\to a} f(x) = k\qquad\text{and}\qquad \lim_{x\to a} g(x) =0.
  \]
  where $k$ is some nonzero constant.
\end{definition}

\begin{question}
  Which of the following limits are of the form $\boldsymbol{\tfrac{\#}{0}}$?
  \begin{selectAll}
    \choice[correct]{$\lim\limits_{x\to -1} \frac{1}{(x+1)^2}$}
    \choice{$\lim\limits_{x\to 2}\frac{x^2-3x+2}{x-2}$}
    \choice{$\lim\limits_{x\to 0}\frac{\sin(x)}{x}$}
    \choice[correct]{$\lim\limits_{x\to 2}\frac{x^2-3x-2}{x-2}$}
    \choice[correct]{$\lim\limits_{x\to 1}\frac{e^x}{\ln(x)}$}
  \end{selectAll}
\end{question}

In our next example, let's see what is going on with limits of the
form $\boldsymbol{\tfrac{\#}{0}}$.

\begin{example}
Consider the function
  \[
  f(x) = \frac{1}{(x+1)^2}.
  \]
  Use a table of values to investigate $\lim\limits_{x\to -1} \frac{1}{(x+1)^2}$.
  \begin{explanation}
    Fill in the table below:
    \[
    \begin{array}{c|c|c}
      x & (x+1)^{2} & f(x) = \frac{1}{(x+1)^2}\\ \hline
      -1.1    & 0.01       & 100 \\
      -1.01   & 0.0001     & 10000 \\
      -1.001  & 0.000001   & \answer{1000000} \\
      -1.0001 & 0.00000001 & \answer{100000000} \\
      -0.9    & 0.01       & \answer{100} \\
      -0.99   & 0.0001     & \answer{10000} \\
      -0.999  & 0.000001   & \answer{1000000} \\
      -0.9999 & 0.00000001 & \answer{100000000} \\
    \end{array}
    \]
    What does the table tell us about 
    \[
    \lim_{x\to -1} \frac{1}{(x+1)^2}?
    \]
    It appears that the limit  does not exist, since the expression
    \[
    \frac{1}{(x+1)^2}
    \]
    becomes larger and larger as  $x$ approaches $-1$ . So,
    \[
    \lim_{x\to -1} \frac{1}{(x+1)^2}\qquad\text{is of the form }\boldsymbol{\tfrac{\#}{0}}
    \]
    as
    \[
    \lim_{x\to -1} 1 = 1 \qquad\text{and}\qquad \lim_{x\to -1}(x+1)^2 = 0.
    \]
    Moreover, as $x$ approaches $-1$:
    \begin{itemize}
    \item The numerator is positive.
    \item The denominator approaches zero and is positive.
    \end{itemize}
    Hence, the expression
    \[
    \frac{1}{(x+1)^2}
    \]
    will become arbitrarily large as $x$ approaches $-1$.  We can see this
    in the graph of $f$.
    \begin{image}
      \begin{tikzpicture}
	\begin{axis}[
            domain=-2:1,
            ymax=100,
            samples=100,
            axis lines =middle, xlabel=$x$, ylabel=$y$,
            every axis y label/.style={at=(current axis.above origin),anchor=south},
            every axis x label/.style={at=(current axis.right of origin),anchor=west}
          ]
	  \addplot [very thick, penColor, smooth, domain=(-2:-1.1)] {1/(x+1)^2};
          \addplot [very thick, penColor, smooth, domain=(-.9:1)] {1/(x+1)^2};
          \addplot [textColor, dashed] plot coordinates {(-1,0) (-1,100)};
        \end{axis}
      \end{tikzpicture}
    \end{image}
  \end{explanation}
\end{example}

We are now ready for our next definition.

\begin{definition}
If $f(x)$ grows arbitrarily large as $x$ approaches $a$, we write
\[
\lim_{x\to a} f(x) = \infty
\]
and say that the limit of $f(x)$ is  \textbf{infinity} as $x$
goes to $a$.


If $|f(x)|$ grows arbitrarily large as $x$ approaches $a$ and $f(x)$ is
negative, we write
\[
\lim_{x\to a} f(x) = -\infty
\]
and say that the limit of $f(x)$ is  \textbf{ negative infinity}
as $x$ goes to $a$.

\end{definition}

Note: Saying ``the limit is equal to infinity''  describes more precisely the behavior of the function $f$ near $a$, than just saying "the limit does not exist".

Let's consider a few more examples.

\begin{example}
  Compute:
  \[
  \lim_{x\to -2} \frac{e^x}{(x+2)^4}
  \]
  \begin{explanation}
    First let's look at the form of this limit. We do this by taking the limits of both the numerator and denominator:
    \[
    \lim_{x\to -2} e^x = \answer[given]{\frac{1}{e^2}}\qquad\text{and}\lim_{x\to-2}\left((x+2)^4\right) = 0.
    \]
    So, this limit is of the form $\boldsymbol{\tfrac{\#}{0}}$. This form is \textbf{determinate}, since it implies that  the limit does not exist.\\
    But, we can do better than that!
     As $x$ approaches $-2$:
    \begin{itemize}
    \item The numerator is a \wordChoice{\choice[correct]{positive}\choice{negative}} number. 
    \item The denominator is \wordChoice{\choice[correct]{positive}\choice{negative}} and is approaching zero.
    \end{itemize}
    This means that 
    \[
    \lim_{x\to -2} \frac{e^x}{(x+2)^4} = \infty.
    \]
  
  \end{explanation}
\end{example}


\begin{example}
  Compute:
  \[
  \lim_{x\to 3^+} \frac{x^2-9x+14}{x^2-5x+6}
  \]
  \begin{explanation}
    First let's look at the form of this limit, which we do by taking the limits of both the numerator and denominator.
    \[
    \lim_{x\to 3^+} \left(x^2-9x+14\right) = \answer[given]{-4}\qquad\text{and}\lim_{x\to3^+}\left(x^2-5x+6\right) = 0
    \]
    This limit is of the form $\boldsymbol{\tfrac{\#}{0}}$. Next, we should factor the numerator and denominator to see if we can simplify the problem at all. 
    \begin{align*}
      \lim_{x\to 3^+}\frac{x^2-9x+14}{x^2-5x+6} &= \lim_{x\to 3^+}\frac{\cancel{(x-2)}(x-7)}{\cancel{(x-2)}(x-3)}\\
      &= \lim_{x\to 3^+}\frac{x-7}{x-3}
    \end{align*}
    Canceling a factor of $x-2$ in the numerator and denominator
    means we can more easily check the behavior of this limit.  As $x$
    approaches $3$ from the right:
    \begin{itemize}
    \item The numerator is a \wordChoice{\choice{positive}\choice[correct]{negative}} number. 
    \item The denominator is \wordChoice{\choice[correct]{positive}\choice{negative}} and approaching zero.
    \end{itemize}
    This means that
    \[
    \lim_{x\to 3^+} \frac{x^2-9x+14}{x^2-5x+6} = -\infty.
    \]
   \end{explanation}
\end{example}

Here is our final example.

\begin{example}
  Compute:
  \[
  \lim_{x\to 3} \frac{x^2-9x+14}{x^2-5x+6}
  \]
  \begin{explanation}
    We've already considered part of this example, but now we consider the two-sided limit. We already know that
    \[
    \lim_{x\to 3} \frac{x^2-9x+14}{x^2-5x+6} = \lim_{x\to
      3}\frac{x-7}{x-3},
    \]
    and that this limit is of the form $\boldsymbol{\tfrac{\#}{0}}$.
    We also know that as $x$ approaches $3$ from the right,
    \begin{itemize}
    \item The numerator is a negative number. 
    \item The denominator is positive and approaching zero.
    \end{itemize}
    Hence our function is approaching $-\infty$ from the right.
    
    As $x$ approaches $3$ from the left,
    \begin{itemize}
    \item The numerator is negative.
    \item The denominator is negative and approaching zero.
    \end{itemize}
    Hence our function is approaching $\infty$ from the left.
    This means
    \[
    \lim_{x\to 3} \frac{x^2-9x+14}{x^2-5x+6} = \answer[format=string,given]{DNE}.
    \]
    \begin{onlineOnly}
     We can confirm our results of the previous two examples by looking at the graph of $y=\frac{x^2-9x+14}{x^2-5x+6}$:
     \[
     \graph[xmin=-10,xmax=15,ymin=-10,ymax=10]{\frac{x^2-9x+14}{x^2-5x+6}}
     \]
   \end{onlineOnly}
  \end{explanation}
\end{example}

Some people worry that the mathematicians are passing into mysticism
when we talk about infinity and negative infinity. However, when we write
\[
\lim_{x\to a} f(x) = \infty \qquad\text{and}\qquad \lim_{x\to a} g(x) = -\infty
\]
all we mean is that as $x$ approaches $a$, $f(x)$ becomes arbitrarily
large and $|g(x)|$ becomes arbitrarily large, with $g(x)$ taking
negative values.
\end{document}
