\documentclass{ximera}


\graphicspath{
  {./}
  {ximeraTutorial/}
  {basicPhilosophy/}
}

\newcommand{\mooculus}{\textsf{\textbf{MOOC}\textnormal{\textsf{ULUS}}}}

\usepackage{tkz-euclide}\usepackage{tikz}
\usepackage{tikz-cd}
\usetikzlibrary{arrows}
\tikzset{>=stealth,commutative diagrams/.cd,
  arrow style=tikz,diagrams={>=stealth}} %% cool arrow head
\tikzset{shorten <>/.style={ shorten >=#1, shorten <=#1 } } %% allows shorter vectors

\usetikzlibrary{backgrounds} %% for boxes around graphs
\usetikzlibrary{shapes,positioning}  %% Clouds and stars
\usetikzlibrary{matrix} %% for matrix
\usepgfplotslibrary{polar} %% for polar plots
\usepgfplotslibrary{fillbetween} %% to shade area between curves in TikZ
\usetkzobj{all}
\usepackage[makeroom]{cancel} %% for strike outs
%\usepackage{mathtools} %% for pretty underbrace % Breaks Ximera
%\usepackage{multicol}
\usepackage{pgffor} %% required for integral for loops



%% http://tex.stackexchange.com/questions/66490/drawing-a-tikz-arc-specifying-the-center
%% Draws beach ball
\tikzset{pics/carc/.style args={#1:#2:#3}{code={\draw[pic actions] (#1:#3) arc(#1:#2:#3);}}}



\usepackage{array}
\setlength{\extrarowheight}{+.1cm}
\newdimen\digitwidth
\settowidth\digitwidth{9}
\def\divrule#1#2{
\noalign{\moveright#1\digitwidth
\vbox{\hrule width#2\digitwidth}}}






\DeclareMathOperator{\arccot}{arccot}
\DeclareMathOperator{\arcsec}{arcsec}
\DeclareMathOperator{\arccsc}{arccsc}

















%%This is to help with formatting on future title pages.
\newenvironment{sectionOutcomes}{}{}


\outcome{Understand what is meant by the form of a limit.}
\outcome{Calculate limits of the form zero over zero.}
\outcome{Identify determinate and indeterminate forms.}
\outcome{Distinguish between determinate and indeterminate forms.}

\author{Nela Lakos \and Kyle Parsons}

\begin{document}
\begin{exercise}

Let $g(x) = 2\left|x-1\right|$, and $h(x) = -(x-1)^2$.\\

For the following limits determine whether the form of the limit is determinate or indeterminate, determine the form of the limit and compute the value of the limit.  Possible answers include a number, $+\infty$, $-\infty$ and $DNE$.

\[
\lim_{x\to1}\frac{h(x)}{g(x)} = \answer{0}
\]


Choose all correct statements.
\begin{selectAll} 
\choice{The limit is of determinate form.}
\choice[correct]{The limit is of indeterminate form.}
\choice[correct]{The limit is of the form $\dfrac{0}{0}$.}
\choice{The limit is of the form $\dfrac{\#}{0}$.}
\end{selectAll}

\noindent\rule[0.5ex]{\linewidth}{0.2pt}
\[
\lim_{x\to1^{+}}\frac{g(x)}{h(x)} = \answer{-\infty}
\]


Choose all correct statements.
\begin{selectAll} 
\choice{The limit is of determinate form.}
\choice[correct]{The limit is of indeterminate form.}
\choice[correct]{The limit is of the form $\dfrac{0}{0}$.}
\choice{The limit is of the form $\dfrac{\#}{0}$.}
\end{selectAll}

\noindent\rule[0.5ex]{\linewidth}{0.2pt}

\[
\lim_{x\to1^{-}}\frac{g(x)}{h(x)} = \answer{-\infty}
\]


Choose all correct statements.
\begin{selectAll} 
\choice{The limit is of determinate form.}
\choice[correct]{The limit is of indeterminate form.}
\choice[correct]{The limit is of the form $\dfrac{0}{0}$.}
\choice{The limit is of the form $\dfrac{\#}{0}$.}
\end{selectAll}

\noindent\rule[0.5ex]{\linewidth}{0.2pt}
\[
\lim_{x\to1}\frac{g(x)}{h(x)} = \answer{-\infty}
\]


Choose all correct statements.
\begin{selectAll} 
\choice{The limit is of determinate form.}
\choice[correct]{The limit is of indeterminate form.}
\choice[correct]{The limit is of the form $\dfrac{0}{0}$.}
\choice{The limit is of the form $\dfrac{\#}{0}$.}
\end{selectAll}

\noindent\rule[0.5ex]{\linewidth}{0.2pt}


\[
\lim_{x\to1^-}\frac{g(x)-g(1)}{x-1} = \answer{-2}
\]



Choose all correct statements.
\begin{selectAll} 
\choice{The limit is of determinate form.}
\choice[correct]{The limit is of indeterminate form.}
\choice[correct]{The limit is of the form $\dfrac{0}{0}$.}
\choice{The limit is of the form $\dfrac{\#}{0}$.}
\end{selectAll}
\noindent\rule[0.5ex]{\linewidth}{0.2pt}

\[
\lim_{x\to1^{+}}\frac{g(x)-g(1)}{x-1} = \answer{2}
\]


Choose all correct statements.
\begin{selectAll} 
\choice{The limit is of determinate form.}
\choice[correct]{The limit is of indeterminate form.}
\choice[correct]{The limit is of the form $\dfrac{0}{0}$.}
\choice{The limit is of the form $\dfrac{\#}{0}$.}
\end{selectAll}
\noindent\rule[0.5ex]{\linewidth}{0.2pt}

\[
\lim_{x\to1}\frac{g(x)-g(1)}{x-1} = \answer{DNE}
\]



Choose all correct statements.
\begin{selectAll} 
\choice{The limit is of determinate form.}
\choice[correct]{The limit is of indeterminate form.}
\choice[correct]{The limit is of the form $\dfrac{0}{0}$.}
\choice{The limit is of the form $\dfrac{\#}{0}$.}
\end{selectAll}


\end{exercise}
\end{document}