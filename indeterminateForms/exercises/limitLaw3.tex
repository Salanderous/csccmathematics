\documentclass{ximera}

\graphicspath{
  {./}
  {ximeraTutorial/}
  {basicPhilosophy/}
}

\newcommand{\mooculus}{\textsf{\textbf{MOOC}\textnormal{\textsf{ULUS}}}}

\usepackage{tkz-euclide}\usepackage{tikz}
\usepackage{tikz-cd}
\usetikzlibrary{arrows}
\tikzset{>=stealth,commutative diagrams/.cd,
  arrow style=tikz,diagrams={>=stealth}} %% cool arrow head
\tikzset{shorten <>/.style={ shorten >=#1, shorten <=#1 } } %% allows shorter vectors

\usetikzlibrary{backgrounds} %% for boxes around graphs
\usetikzlibrary{shapes,positioning}  %% Clouds and stars
\usetikzlibrary{matrix} %% for matrix
\usepgfplotslibrary{polar} %% for polar plots
\usepgfplotslibrary{fillbetween} %% to shade area between curves in TikZ
\usetkzobj{all}
\usepackage[makeroom]{cancel} %% for strike outs
%\usepackage{mathtools} %% for pretty underbrace % Breaks Ximera
%\usepackage{multicol}
\usepackage{pgffor} %% required for integral for loops



%% http://tex.stackexchange.com/questions/66490/drawing-a-tikz-arc-specifying-the-center
%% Draws beach ball
\tikzset{pics/carc/.style args={#1:#2:#3}{code={\draw[pic actions] (#1:#3) arc(#1:#2:#3);}}}



\usepackage{array}
\setlength{\extrarowheight}{+.1cm}
\newdimen\digitwidth
\settowidth\digitwidth{9}
\def\divrule#1#2{
\noalign{\moveright#1\digitwidth
\vbox{\hrule width#2\digitwidth}}}






\DeclareMathOperator{\arccot}{arccot}
\DeclareMathOperator{\arcsec}{arcsec}
\DeclareMathOperator{\arccsc}{arccsc}

















%%This is to help with formatting on future title pages.
\newenvironment{sectionOutcomes}{}{}

\outcome{Calculate limits using the limit laws.}
\outcome{Calculate limits by replacing a function with a continuous
  function.}
\tag{limit}
\author{Steven Gubkin}
\license{Creative Commons 3.0 By-NC}
\begin{document}

\begin{exercise}

	A student's attempt to evaluate the limit $\lim_{x \to 0} \frac{x^3+x}{x}$ using limit laws is recorded below.  From one line to the next, only one limit law should be utilized.  Which of the following options best describes the student's work?
	
	\begin{align*}
		\lim_{x \to 0} \frac{x^3 + x}{x} &= \lim_{x \to 0} \frac{x(x^2+1)}{x}\\
		&= x^2+1
	\end{align*}
	
	\begin{multipleChoice}
		\choice{The work is perfect}
		\choice{The answer is correct, but the student skipped some steps, or made a mistake along the way}
		\choice[correct]{The answer is incorrect}
	\end{multipleChoice}
	
	\begin{feedback}
		It never makes sense for the limiting variable to appear in the answer to a limit.
	\end{feedback}
	
\end{exercise}
\end{document}
