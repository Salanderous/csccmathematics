\documentclass{ximera}


\graphicspath{
  {./}
  {ximeraTutorial/}
  {basicPhilosophy/}
}

\newcommand{\mooculus}{\textsf{\textbf{MOOC}\textnormal{\textsf{ULUS}}}}

\usepackage{tkz-euclide}\usepackage{tikz}
\usepackage{tikz-cd}
\usetikzlibrary{arrows}
\tikzset{>=stealth,commutative diagrams/.cd,
  arrow style=tikz,diagrams={>=stealth}} %% cool arrow head
\tikzset{shorten <>/.style={ shorten >=#1, shorten <=#1 } } %% allows shorter vectors

\usetikzlibrary{backgrounds} %% for boxes around graphs
\usetikzlibrary{shapes,positioning}  %% Clouds and stars
\usetikzlibrary{matrix} %% for matrix
\usepgfplotslibrary{polar} %% for polar plots
\usepgfplotslibrary{fillbetween} %% to shade area between curves in TikZ
\usetkzobj{all}
\usepackage[makeroom]{cancel} %% for strike outs
%\usepackage{mathtools} %% for pretty underbrace % Breaks Ximera
%\usepackage{multicol}
\usepackage{pgffor} %% required for integral for loops



%% http://tex.stackexchange.com/questions/66490/drawing-a-tikz-arc-specifying-the-center
%% Draws beach ball
\tikzset{pics/carc/.style args={#1:#2:#3}{code={\draw[pic actions] (#1:#3) arc(#1:#2:#3);}}}



\usepackage{array}
\setlength{\extrarowheight}{+.1cm}
\newdimen\digitwidth
\settowidth\digitwidth{9}
\def\divrule#1#2{
\noalign{\moveright#1\digitwidth
\vbox{\hrule width#2\digitwidth}}}






\DeclareMathOperator{\arccot}{arccot}
\DeclareMathOperator{\arcsec}{arcsec}
\DeclareMathOperator{\arccsc}{arccsc}

















%%This is to help with formatting on future title pages.
\newenvironment{sectionOutcomes}{}{}


\outcome{Understand what is meant by the form of a limit.}
\outcome{Calculate limits of the form zero over zero.}
\outcome{Identify determinate and indeterminate forms.}
\outcome{Distinguish between determinate and indeterminate forms.}
\outcome{Discuss why infinity is not a number}

\title[Dig-In:]{Limits of the form zero over zero}

\begin{document}
\begin{abstract}
  We want to evaluate limits for which the Limit Laws do not apply. 
\end{abstract}

\maketitle

In the last section we computed  limits 
using continuity and the limit laws. What about limits that cannot be
directly computed using these methods?  Consider the following limit,
\[
\lim_{x\to 2}\frac{x^2-3x+2}{x-2}.
\]
Here 
\[
\lim_{x\to 2}\left(x^2-3x+2\right) = 0\qquad\text{and}\qquad \lim_{x\to
  2}\left(x-2\right) = 0
\]
in light of this, you may think that the limit is one or
zero. \textbf{Not so fast}. This limit is of an \textit{indeterminate
  form}. What does this mean? Read on, young mathematician.

\begin{definition}
  A limit
  \[
  \lim_{x\to a} \frac{f(x)}{g(x)}
  \]
  is said to be of the form $\boldsymbol{\tfrac{0}{0}}$\ if
  \[
  \lim_{x\to a} f(x) = 0\quad\text{and}\quad \lim_{x\to a} g(x) =0.
  \]
\end{definition}

\begin{question}
  Which of the following limits are of the form $\boldsymbol{\tfrac{0}{0}}$?
  \begin{selectAll}
    \choice[correct]{$\lim_{x\to 0}\frac{\sin(x)}{x}$}
    \choice{$\lim_{x\to 0}\frac{\cos(x)}{x}$}
    \choice{$\lim_{x\to 0}\frac{x^2-3x+2}{x-2}$}
    \choice[correct]{$\lim_{x\to 2}\frac{x^2-3x+2}{x-2}$}
    \choice{$\lim_{x\to 3}\frac{x^2-3x+2}{x-3}$}
  \end{selectAll}
\end{question}

\begin{warning}
  The symbol $\boldsymbol{\tfrac{0}{0}}$\ is \textbf{not} the number $0$ divided by
  $0$. It is simply short-hand and means that a limit $\lim_{x\to a}
  \frac{f(x)}{g(x)}$ has the property that
  \[
  \lim_{x\to a} f(x) = 0\quad\text{and}\quad \lim_{x\to a} g(x) =0.
  \]
\end{warning}


Let's finish the example with the function above. 


\begin{example}
  Compute:
  \[
  \lim_{x\to 2}\frac{x^2-3x+2}{x-2}
  \]
  \begin{explanation}
  This limit is of the form $\boldsymbol{\tfrac{0}{0}}$. However, note that if we
  assume $x\ne 2$, then we can write
    \[
    \frac{x^2-3x+2}{x-2} = \frac{(x-2)(\answer[given]{x-1})}{(x-2)} = \answer[given]{x-1}.
    \]
  
    \begin{image}
      \begin{tikzpicture}
        \begin{axis}[
            domain=-2:4,
            width=2.5in,
            axis lines =middle, xlabel=$x$, ylabel=$y$,
            every axis y label/.style={at=(current axis.above origin),anchor=south},
            every axis x label/.style={at=(current axis.right of origin),anchor=west},
            xtick={-2,...,4},
            ytick={-3,...,3},
          ]
	  \addplot [very thick, penColor, smooth] {x-1};
        \end{axis}
        \node [penColor] at (2,-.75) {$y= x-1$};
      \end{tikzpicture}
      \quad
      \begin{tikzpicture}
	\begin{axis}[
            domain=-2:4,
            width=2.5in,
            axis lines =middle, xlabel=$x$, ylabel=$y$,
            every axis y label/.style={at=(current axis.above origin),anchor=south},
            every axis x label/.style={at=(current axis.right of origin),anchor=west},
            xtick={-2,...,4},
            ytick={-3,...,3},
          ]
	  \addplot [very thick, penColor, smooth] {x-1};
          \addplot[color=penColor,fill=background,only marks,mark=*] coordinates{(2,1)};  %% open hole
        \end{axis}
        \node [penColor] at (2,-.5) {$y=\frac{x^2-3x+2}{x-2}$};
      \end{tikzpicture}
    \end{image}
      So, ``our function" is equal to the polynomial $x-1$ everywhere, except at $x=2$. Therefore, for all values of $x$ near $2$, ``our function" is equal to the polynomial $x-1$!
    This means that
    \[
    \lim_{x\to 2}\frac{x^2-3x+2}{x-2} = \lim_{x\to 2} (x-1).
    \]
    But now, we have to take the limit of a polynomial,
    $\lim_{x\to 2} (x-1) =\answer[given]{1}$.
    
     Hence
    \[
    \lim_{x\to 2}\frac{x^2-3x+2}{x-2} = \answer[given]{1}.
    \]
  \end{explanation}
\end{example} 


Let's consider a few more examples of the form $\boldsymbol{\tfrac{0}{0}}$.

\begin{example}
  Compute:
  \[
  \lim_{x\to1}\frac{x-1}{x^2+2x-3}.
  \]
  \begin{explanation}
    First note that
    \[
    \lim_{x\to1}\left(x-1\right)=0 \quad\text{and}\quad  \lim_{x\to1}\left(x^2+2x-3\right) = 0.
    \]
    Hence, this limit is of the form \boldsymbol{\tfrac{0}{0}}. Again, we cannot apply the Quotient Law or any other Limit Law. We cannot use continuity, either. Namely, ``our function" is not continuous at $x=1$, since it is not defined 
    at $x=1$. 
    
    What can be done? We can hope to be able to  cancel a factor going to $0$ out of the numerator and
    denominator.  Since $\answer[given]{(x-1)}$ is a factor going to $0$ in the
    numerator, let's see if we can factor a $\answer[given]{(x-1)}$ out of the
    denominator as well.
    \begin{align*}
      \lim_{x\to1}\frac{x-1}{x^2+2x-3}&=\lim_{x\to1}\frac{\cancel{x-1}}{\cancel{(x-1)}\answer[given]{(x+3)}} \\
      &=\lim_{x\to1}\frac{1}{\answer[given]{x+3}}\\
      &=\frac{1}{4}.
    \end{align*}
  \end{explanation}
\end{example}


\begin{example}
  Compute:
  \[
  \lim_{x\to 1} \frac{\frac{1}{x+1}-\frac{3}{x+5}}{x-1}.
  \]
\begin{explanation}
  We find the form of this limit by looking at the limits of the
  numerator and denominator separately. The limit of the numerator is:
  \begin{align*}
    \lim_{x\to 1}\left(\frac{1}{x+1}-\frac{3}{x+5}\right)&=  \lim_{x\to 1}\frac{x+5-3(x+1)}{(x+1)(x+5)}\\
    &= \lim_{x\to 1}\frac{-2x+2}{(x+1)(x+5)}\\
    &= \frac{0}{12}\\
    &=0
  \end{align*}
  The limit of the denominator is:
  \[
  \lim_{x\to 1}\left(x-1\right)=0
  \]
  Our limit is therefore of the form $\boldsymbol{\tfrac{0}{0}}$\ and we can
  probably factor a term going to $0$ out of both the numerator and
  denominator.
  \[
  \lim_{x\to 1} \frac{\frac{1}{x+1}-\frac{3}{x+5}}{x-1}
  \]
  When looking at the denominator, we hope that this
  term is $(x-1)$.  Unfortunately, it is not immediately obvious how to
  factor an $(x-1)$ out of the numerator. So, we should first simplify the complex fraction
  by multiplying it by
  \[
  1 = \frac{(x+1)(x+5)}{(x+1)(x+5)}
  \]
  this will allow us to cancel immediately
\begin{align*}
  \lim_{x\to 1}& \frac{\frac{1}{x+1}-\frac{3}{x+5}}{x-1}  \cdot \frac{(x+1)(x+5)}{(x+1)(x+5)} \\
  &= \lim_{x\to 1}\frac{(x+5)-3(x+1)}{(x+1)(x+5)(x-1)}.
\end{align*}

Now we will multiply out the numerator.  Note that we do not want to
multiply out the denominator because we already have an $(x-1)$
factored out of the denominator and that was the goal.
\[
\lim_{x\to 1}\frac{(x+5)-3(x+1)}{(x+1)(x+5)(x-1)}
\]
\begin{align*}
  &= \lim_{x\to 1}\frac{x+5-3x-3}{(x+1)(x+5)(x-1)} \\
  &= \lim_{x\to 1}\frac{-2x+2}{(x+1)(x+5)(x-1)}\\
  &= \lim_{x\to 1}\frac{-2\cancel{(x-1)}}{(x+1)(x+5)\cancel{(x-1)}}\\
  &= \lim_{x\to 1}\frac{-2}{(x+1)(x+5)}.
\end{align*}

Now, we can see that the limit of ``our function" is equal to the limit of a rational function $\frac{-2}{(x+1)(x+5)}$. This rational function is continuous on its domain, and, therefore, at $x=1$.  Hence
\begin{align*}
\lim_{x\to 1} \frac{\frac{1}{x+1}-\frac{3}{x+5}}{x-1}&=\lim_{x\to
  1}\frac{-2}{(x+1)(x+5)}\\
&= \frac{-2}{(1+1)(1+5)} \\
&=\answer[given]{\frac{-1}{6}}.
\end{align*}
\end{explanation}
\end{example}

We will  look at one more example.

\begin{example}
  Compute:
  \[
  \lim_{x\to-1} \frac{\sqrt{x+5}-2}{x+1}.
  \]

\begin{explanation} 
  Note that 
  \[
  \lim_{x\to-1} \left(\sqrt{x+5}-2\right)=0\quad\text{and}\quad\lim_{x\to -1} \left(x+1\right) =0.
  \]
  Our limit is, again, of the form $\boldsymbol{\tfrac{0}{0}}$\ and we
  can probably factor a term going to $0$ out of both the numerator
  and denominator.  We suspect from looking at the denominator that
  this term is $(x+1)$.  Unfortunately, it is not immediately obvious
  how to factor an $(x+1)$ out of the numerator.
 
  We will use an algebraic technique called \textbf{multiplying by the
    conjugate}.  This technique is useful when you are trying to
  simplify an expression that looks like
  \[
  \sqrt{\text{something}} \pm \text{something else}.
  \]
  It takes advantage of the difference of squares rule 
  \[
  a^2-b^2=(a-b)(a+b).
  \]
  In our case, we will use $a=\sqrt{x+5}$ and $b=2$.  Write
  \[
  \lim_{x\to-1} \frac{\sqrt{x+5}-2}{x+1}
  \]
  \begin{align*}
    &= \lim_{x\to-1} \frac{\left(\sqrt{x+5}-2\right)}{(x+1)} \cdot \frac{\left(\sqrt{x+5}+2\right)}{\left(\sqrt{x+5}+2\right)} \\
&=\lim_{x\to-1} \frac{\answer[given]{\left(\sqrt{x+5}\right)^2-2^2}}{(x+1)\left(\sqrt{x+5}+2\right)} \\
&=\lim_{x\to-1} \frac{x+5-4}{(x+1)\left(\sqrt{x+5}+2\right)} \\
&=\lim_{x\to-1} \frac{\cancel{(x+1)}}{\cancel{(x+1)}\left(\sqrt{x+5}+2\right)} \\
&=\lim_{x\to-1} \frac{1}{\sqrt{x+5}+2}\\
&= \frac{1}{\sqrt{-1+5}+2}\\
&=\answer[given]{\frac{1}{4}}.
\end{align*}
\end{explanation}
\end{example}

All of the examples in this section are limits of the form $\boldsymbol{\tfrac{0}{0}}$.
When you come across a limit of the form $\boldsymbol{\tfrac{0}{0}}$, you should try
to use algebraic techniques to come up with a continuous
function whose limit you can evaluate.

Notice that we solved multiple examples of limits of the form
$\boldsymbol{\tfrac{0}{0}}$\ and we got different answers each time.  This tells us
that just knowing that the form of the limit is $\boldsymbol{\tfrac{0}{0}}$\ is not enough
to compute the limit. The moral of the story is
\begin{center}
  \textbf{Limits of the form $\boldsymbol{\tfrac{0}{0}}$\ can take any value.}
\end{center}

\begin{definition}
A form that gives us no information about whether the limit exists or not, and if the limit exists, no information about the value of the limit, is
called an \textbf{indeterminate form}.

A form that gives information about whether the limit exists or not, and if it exists gives information about the value of the limit, is called a
\dfn{determinate form}.
\end{definition}  

Finally, you may find it distressing that we introduced a form, namely
\boldsymbol{\tfrac{0}{0}}, only to end up saying they give no information on the
value of the limit. But this is precisely what makes
indeterminate forms interesting\dots~they're a mystery!



\end{document}
