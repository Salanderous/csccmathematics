\documentclass{ximera}


\graphicspath{
  {./}
  {ximeraTutorial/}
  {basicPhilosophy/}
}

\newcommand{\mooculus}{\textsf{\textbf{MOOC}\textnormal{\textsf{ULUS}}}}

\usepackage{tkz-euclide}\usepackage{tikz}
\usepackage{tikz-cd}
\usetikzlibrary{arrows}
\tikzset{>=stealth,commutative diagrams/.cd,
  arrow style=tikz,diagrams={>=stealth}} %% cool arrow head
\tikzset{shorten <>/.style={ shorten >=#1, shorten <=#1 } } %% allows shorter vectors

\usetikzlibrary{backgrounds} %% for boxes around graphs
\usetikzlibrary{shapes,positioning}  %% Clouds and stars
\usetikzlibrary{matrix} %% for matrix
\usepgfplotslibrary{polar} %% for polar plots
\usepgfplotslibrary{fillbetween} %% to shade area between curves in TikZ
\usetkzobj{all}
\usepackage[makeroom]{cancel} %% for strike outs
%\usepackage{mathtools} %% for pretty underbrace % Breaks Ximera
%\usepackage{multicol}
\usepackage{pgffor} %% required for integral for loops



%% http://tex.stackexchange.com/questions/66490/drawing-a-tikz-arc-specifying-the-center
%% Draws beach ball
\tikzset{pics/carc/.style args={#1:#2:#3}{code={\draw[pic actions] (#1:#3) arc(#1:#2:#3);}}}



\usepackage{array}
\setlength{\extrarowheight}{+.1cm}
\newdimen\digitwidth
\settowidth\digitwidth{9}
\def\divrule#1#2{
\noalign{\moveright#1\digitwidth
\vbox{\hrule width#2\digitwidth}}}






\DeclareMathOperator{\arccot}{arccot}
\DeclareMathOperator{\arcsec}{arcsec}
\DeclareMathOperator{\arccsc}{arccsc}

















%%This is to help with formatting on future title pages.
\newenvironment{sectionOutcomes}{}{}



\outcome{Calculate limits of the form zero over zero.}


\title[Break-Ground:]{Could it be anything?}

\begin{document}
\begin{abstract}
Two young mathematicians investigate the arithmetic of large
and small numbers.
\end{abstract}
\maketitle


Check out this dialogue between two calculus students (based on a true
story):


\begin{dialogue}
\item[Devyn] Hey Riley, remember
  \[
  \lim_{\theta\to 0}\frac{\sin(\theta)}{\theta}?
  \]
\item[Riley] It is equal to $1$!
\item[Devyn] But was that crazy proof with all the triangles really
  necessary? I mean, just plug in zero. 
  \[
  \eval{\frac{\sin(\theta)}{\theta}}_{\theta=0} = \frac{\sin(0)}{0}=\frac{0}{0}\dots
  \]
  \item[Riley] You were going to say ``$1$,'' right? 
  \item[Devyn] Yeah, but now I'm not sure I was right.
  \item[Riley] Dividing by zero is usually a bad idea.
  \item[Devyn] You are right. I will never do it again! Also, don't
    tell anyone about this conversation.
  \item[Riley] What conversation?
  \item[Devyn] Exactly.
\end{dialogue}



\begin{problem}
  Consider the function
  \[
  f(x) = \frac{x}{x}.
  \]
  \[
  f(0) = \answer{DNE}\qquad\lim_{x\to 0} f(x) = \answer{1}.
  \]
\end{problem}

\begin{problem}
  Consider the function
  \[
  f(x) = \frac{4x}{x}.
  \]
  \[
  f(0) = \answer{DNE}\qquad\lim_{x\to 0} f(x) = \answer{4}.
  \]
\end{problem}

\begin{problem}
  Consider the function
  \[
  f(x) = \frac{x}{-3x}.
  \]
  \[
  f(0) = \answer{DNE}\qquad\lim_{x\to 0} f(x) = \answer{-1/3}.
  \]
\end{problem}

%\input{../leveledQuestions.tex}


\end{document}
