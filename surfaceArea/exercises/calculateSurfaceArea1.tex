\documentclass{ximera}


\graphicspath{
  {./}
  {ximeraTutorial/}
  {basicPhilosophy/}
}

\newcommand{\mooculus}{\textsf{\textbf{MOOC}\textnormal{\textsf{ULUS}}}}

\usepackage{tkz-euclide}\usepackage{tikz}
\usepackage{tikz-cd}
\usetikzlibrary{arrows}
\tikzset{>=stealth,commutative diagrams/.cd,
  arrow style=tikz,diagrams={>=stealth}} %% cool arrow head
\tikzset{shorten <>/.style={ shorten >=#1, shorten <=#1 } } %% allows shorter vectors

\usetikzlibrary{backgrounds} %% for boxes around graphs
\usetikzlibrary{shapes,positioning}  %% Clouds and stars
\usetikzlibrary{matrix} %% for matrix
\usepgfplotslibrary{polar} %% for polar plots
\usepgfplotslibrary{fillbetween} %% to shade area between curves in TikZ
\usetkzobj{all}
\usepackage[makeroom]{cancel} %% for strike outs
%\usepackage{mathtools} %% for pretty underbrace % Breaks Ximera
%\usepackage{multicol}
\usepackage{pgffor} %% required for integral for loops



%% http://tex.stackexchange.com/questions/66490/drawing-a-tikz-arc-specifying-the-center
%% Draws beach ball
\tikzset{pics/carc/.style args={#1:#2:#3}{code={\draw[pic actions] (#1:#3) arc(#1:#2:#3);}}}



\usepackage{array}
\setlength{\extrarowheight}{+.1cm}
\newdimen\digitwidth
\settowidth\digitwidth{9}
\def\divrule#1#2{
\noalign{\moveright#1\digitwidth
\vbox{\hrule width#2\digitwidth}}}






\DeclareMathOperator{\arccot}{arccot}
\DeclareMathOperator{\arcsec}{arcsec}
\DeclareMathOperator{\arccsc}{arccsc}

















%%This is to help with formatting on future title pages.
\newenvironment{sectionOutcomes}{}{}


\author{Jim Talamo and Nicholas Hemleben}
\license{Creative Commons 3.0 By-NC}


\outcome{Set up an integral that gives the area of a surface of revolution}
\outcome{Find  the area of a surface of revolution}

\begin{document}
\begin{exercise}

The portion of the curve $y=2x^3$ from $x=0$ to $x=1$ is shown below:

 \begin{image}
      \begin{tikzpicture}
        \begin{axis}[
            xmin=-.3, xmax=1.2,
            domain=-1:1,
            ymin=-.3, ymax=2.5,
            clip=false,
            xtick = {-1,1},
            ytick = {-1,1,2},
            axis lines =center,
            xlabel=$x$, ylabel=$y$, every axis y label/.style={at=(current axis.above origin),anchor=south},
            every axis x label/.style={at=(current axis.right of origin),anchor=west},
            axis on top,
          ]
                              
         \addplot [penColor,thick,smooth,domain=0:1.1]{2*x^3};
          
         % ds and points
          	\addplot[color=penColor2,fill=penColor2,only marks,mark=*] coordinates{(.4,.128)};
		\addplot[color=penColor2,fill=penColor2,only marks,mark=*] coordinates{(.6,.44)};
		\addplot[ultra thick, penColor2] plot coordinates {(.4,.128) (.6,.44)};
          	\node[anchor=north, penColor2] at (axis cs:.45,.6) {$\Delta s$};
	
          %r and point 
           	\addplot[thick, penColor2] plot coordinates {(.5,0) (.5,.25)};
		\node[anchor=north, penColor2] at (axis cs:.55,.25) {$r$};
          
          
        \addplot[color=penColor,fill=penColor,only marks,mark=*] coordinates{(0,0)};
	\addplot[color=penColor,fill=penColor,only marks,mark=*] coordinates{(1,2)};
          
          \node[penColor] at (axis cs:.6,1.4) {$y=2x^3$};
        \end{axis}
      \end{tikzpicture}
    \end{image}

 The curve is then revolved about the $x$-axis to form a surface of revolution.  The resulting frustum resulting from revolving the slice is shown below:
    
      \begin{image}
      \begin{tikzpicture}
        \begin{axis}[
            xmin=-.3, xmax=1.2,
            domain=-1:1,
            ymin=-2.1, ymax=2.1,
            clip=false,
            xtick = {-1,1},
            ytick = {-1,1,2},
            axis lines =center,
            xlabel=$x$, ylabel=$y$, every axis y label/.style={at=(current axis.above origin),anchor=south},
            every axis x label/.style={at=(current axis.right of origin),anchor=west},
            axis on top,
          ]
                              
         \addplot [penColor,thick,smooth,domain=0:1]{2*x^3};
          
         % ds and points
          	\addplot[color=penColor2,fill=penColor2,only marks,mark=*] coordinates{(.4,.128)};
		\addplot[color=penColor2,fill=penColor2,only marks,mark=*] coordinates{(.6,.44)};
		\addplot[thick, penColor2] plot coordinates {(.4,.128) (.6,.44)};
		\addplot[thick, penColor2] plot coordinates {(.4,-.128) (.6,-.44)};
          	\node[anchor=north, penColor2] at (axis cs:.45,.6) {$ds$};
          
        \addplot[color=penColor,fill=penColor,only marks,mark=*] coordinates{(0,0)};
	\addplot[color=penColor,fill=penColor,only marks,mark=*] coordinates{(1,2)};
          
          \node[penColor] at (axis cs:.6,1.4) {$y=2x^3$};
          
           %ellipses
           \addplot [penColor2,thick,smooth,domain=.39:.4,samples=100]{sqrt(.016-.016/.0001*(x-.4)^2)};
           \addplot [penColor2,thick,smooth,domain=.39:.4,samples=100]{-sqrt(.016-.016/.0001*(x-.4)^2)};
            \addplot [penColor2,thick,smooth,domain=.59:.61,samples=100]{sqrt(.194-.194/.0001*(x-.6)^2)};
           \addplot [penColor2,thick,smooth,domain=.59:.61,samples=100]{-sqrt(.194-.194/.0001*(x-.6)^2)};
           
           
        \end{axis}
      \end{tikzpicture}
    \end{image}    
    
The function and its limits are in terms of $x$.  Also, note that if we solve for $x$, we have $x= \sqrt[3]{\frac{y}{2}}$, which is not differentiable at $y=0$.  Thus, we should:

\begin{multipleChoice}
\choice[correct]{integrate with respect to $x$.}
\choice{integrate with respect to $y$.}
\end{multipleChoice}



\begin{exercise}
Since we have chosen this, we use the result:

\[ SA = \int_{x=a}^{x=b} 2 \pi r ds\]

and we must express $r$ in terms of $x$ and $ds$ in terms of $x$ and $dx$.  

\begin{exercise}
For $ds$, we use $ds \sqrt{1+\left(\frac{dy}{dx}\right)^2} dx$.

Here, $\frac{dy}{dx}=\answer{6x^2}$ so $ds= \answer{\sqrt{1+ 36 x^4}} dx$. 
\end{exercise}

\begin{exercise}
Note that is $r$ is the distance from the axis to the curve. This is a:

\begin{multipleChoice}
\choice[correct]{vertical distance}
\choice{horizontal distance}
\end{multipleChoice} 
Thus $r=y_{top}-y_{bot}$.  

We note that the slice is at a location $(x,y)$, which happens to be on the curve.  This allows us to use the curve to express $x$ in terms of $y$ or $y$ in terms of $x$ if necessary.  

The quantity $2x^3$ expresses:
\begin{multipleChoice}
\choice[correct]{The $y$-value on the curve if the $x$-value of the slice is specified.}
\choice{The $x$-value on the curve.}
\end{multipleChoice} 

The quantity $\sqrt[3]{\frac{y}{2}}$ expresses:
\begin{multipleChoice}
\choice[correct]{The $x$-value on the curve if the $y$-value of the slice is specified.}
\choice{The $y$-value on the curve.}
\end{multipleChoice} 

Since we have to express $r$ in terms of $x$, and we note that $y_{top}$ is on the curve, we must express it in terms of $x$.  Hence, $y_{top} = \answer{2x^3}$.

Since $y_{bot}$ lies on the $x$-axis, $y_{bot} = \answer{0}$, and $r= \answer{2x^3}$.

\end{exercise}

\begin{exercise}
Now we see that an integral that gives the surface area is: 
\[
SA= \int_{x=\answer{0}}^{x=\answer{1}} 2 \pi r ds = \int_0^1 \answer{4 \pi x^3} \answer{\sqrt{1+36x^4}} dx 
\]

\begin{exercise}
Evaluating this integral, we find that the surface area of the surface of revolution is $\answer{\frac \pi {54} ((37)^{3/2} -1)}$ square units. 

\begin{hint} You can use a substitution to evaluate the integral with $u= \answer{1+36x^4}$.  Then, $du= \answer{144x^3} dx$. 

For the limits of integration,  when $x=0$, $u= \answer{1}$ and when $x=1$, $u=\answer{37}$.

Thus:

\begin{align*}
SA= \int_{x=0}^{x=1} 4 \pi x^3 \sqrt{1+36x^4} dx &=  \int_{u=\answer{1}}^{u=\answer{37}} \answer{\frac{\pi}{36}} \answer{u^{1/2}} du \\
&=  \bigg[\answer{\frac{\pi}{54} u^{3/2 } }\bigg]_{u=1}^{u=37}
&= \answer{\frac \pi {54} ((37)^{3/2} -1)}
\end{align*}
\end{hint}


\end{exercise}
\end{exercise}
\end{exercise}
\end{exercise}





\end{document}
