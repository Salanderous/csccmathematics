\documentclass{ximera}


\graphicspath{
  {./}
  {ximeraTutorial/}
  {basicPhilosophy/}
}

\newcommand{\mooculus}{\textsf{\textbf{MOOC}\textnormal{\textsf{ULUS}}}}

\usepackage{tkz-euclide}\usepackage{tikz}
\usepackage{tikz-cd}
\usetikzlibrary{arrows}
\tikzset{>=stealth,commutative diagrams/.cd,
  arrow style=tikz,diagrams={>=stealth}} %% cool arrow head
\tikzset{shorten <>/.style={ shorten >=#1, shorten <=#1 } } %% allows shorter vectors

\usetikzlibrary{backgrounds} %% for boxes around graphs
\usetikzlibrary{shapes,positioning}  %% Clouds and stars
\usetikzlibrary{matrix} %% for matrix
\usepgfplotslibrary{polar} %% for polar plots
\usepgfplotslibrary{fillbetween} %% to shade area between curves in TikZ
\usetkzobj{all}
\usepackage[makeroom]{cancel} %% for strike outs
%\usepackage{mathtools} %% for pretty underbrace % Breaks Ximera
%\usepackage{multicol}
\usepackage{pgffor} %% required for integral for loops



%% http://tex.stackexchange.com/questions/66490/drawing-a-tikz-arc-specifying-the-center
%% Draws beach ball
\tikzset{pics/carc/.style args={#1:#2:#3}{code={\draw[pic actions] (#1:#3) arc(#1:#2:#3);}}}



\usepackage{array}
\setlength{\extrarowheight}{+.1cm}
\newdimen\digitwidth
\settowidth\digitwidth{9}
\def\divrule#1#2{
\noalign{\moveright#1\digitwidth
\vbox{\hrule width#2\digitwidth}}}






\DeclareMathOperator{\arccot}{arccot}
\DeclareMathOperator{\arcsec}{arcsec}
\DeclareMathOperator{\arccsc}{arccsc}

















%%This is to help with formatting on future title pages.
\newenvironment{sectionOutcomes}{}{}


\author{Jim Talamo}
\license{Creative Commons 3.0 By-NC}


\outcome{Set up an integral that gives the area of a surface of revolution with respect to both $x$ and $y$.}
\outcome{Find  the area of a surface of revolution}

\begin{document}
\begin{exercise}

The portion of the curve $x=4e^{3y}$ from $y=\ln(2)$ to $y=1$ is revolved around the line $y=2$.

 \begin{image}
      \begin{tikzpicture}
        \begin{axis}[
            xmin=-3, xmax=90,
            domain=-1:1,
            ymin=-.3, ymax=2.4,
            clip=false,
            xtick = {10,20,30,40,50,60,70,80},
            ytick = {-1,1,2,3},
            axis lines =center,
            xlabel=$y$, ylabel=$y$, every axis y label/.style={at=(current axis.above origin),anchor=south},
            every axis x label/.style={at=(current axis.right of origin),anchor=west},
            axis on top,
          ]
                              
         \addplot [penColor,thick,smooth,domain=32:80]{1/3*ln(x/4)};
          
         % ds and points
          	\addplot[color=penColor2,fill=penColor2,only marks,mark=*] coordinates{(40,.767)};
		\addplot[color=penColor2,fill=penColor2,only marks,mark=*] coordinates{(45,.806)};
		\addplot[ultra thick, penColor2] plot coordinates {(40,.767) (45,.806)};
          	\node[anchor=north, penColor2] at (axis cs:45,.75) {$\Delta s$};
	%axis
	\addplot[ultra thick, penColor5, dotted] plot coordinates {(.3,2) (90,2)};
	
          %r and point 
           	\addplot[thick, penColor2] plot coordinates {(42.5,2) (42.5,.79)};
		\node[anchor=north, penColor2] at (axis cs:39,1.4) {$r$};
          
          %endpoints
        \addplot[color=penColor,fill=penColor,only marks,mark=*] coordinates{(32,.69)};
	\addplot[color=penColor,fill=penColor,only marks,mark=*] coordinates{(80,1)};
          
          \node[penColor] at (axis cs:65,1.15) {$x=4e^{3y}$};
        \end{axis}
      \end{tikzpicture}
    \end{image}
 
To set up an integral with respect to $x$ that gives the area of the surface of revolution, do the following:  

Since we have chosen to integrate with respect to $x$, we use the result:

\[ SA = \int_{x=a}^{x=b} 2 \pi r ds\]

and we must express $r$ in terms of $x$ and $ds$ in terms of $x$ and $dx$.  


Let's start by describing the curve as a function of $x$.  Since $x=4e^{3y}$, we find:

\[
y= \answer{\frac{1}{3}\ln\left(\frac{x}{4}\right)}
\]

Calculating $ds$ gives: 

\[
ds = \sqrt{1+ \answer{\frac{1}{9x^2}}} dx
\]


\begin{exercise}
Note that is $r$ is the distance from the axis to the curve. This is a:

\begin{multipleChoice}
\choice[correct]{vertical distance}
\choice{horizontal distance}
\end{multipleChoice} 
Thus $r=y_{top}-y_{bot} = \answer{2-\frac{1}{3}\ln\left(\frac{x}{4}\right)}.$


\begin{hint}
Since we have to express $r$ in terms of $x$, and we note that $y_{top}$ is on the axis of rotation, we must express it in terms of $x$.  Hence, $y_{top} = \answer{2}$.

Similarly, $y_{bot} =\answer{ \frac{1}{3}\ln\left(\frac{x}{4}\right)}$ since $y_{bot}$ is on the curve.
\end{hint}



\end{exercise}

\begin{exercise}
Now we see that an integral that gives the surface area is: 
\[
SA= \int_{x=a}^{x=b} 2 \pi r ds = \int_{x=\answer{32}}^{x=\answer{4e^3}} \answer{2 \pi \left(2-\frac{1}{3}\ln\left(\frac{x}{4}\right)\right)} \sqrt{\answer{1+\frac{1}{9x^2}}} dx
\]

\begin{hint}
When $y=\ln(2)$, we have $x= 4e^{3 \ln(2)}$.  There are two ways to simplify this:

Way 1: Use the properties of logarithms to write: $3 \ln(2) = \ln\left(2^3\right)$ so $4e^{3 \ln(2)} = 4e^{\ln(8)} =\answer{32}$.

Way 2: Use the properties of exponents to write $4e^{3 \ln(2)} = 4\left(e^{\ln(2)}\right)^3 = 4(2)^3 = \answer{32}$.

Now, make sure that your limits of integration are in increasing order.
\end{hint}

\begin{exercise}
Using computational software of your choice, the integral to 4 decimal places shows that the surface area is $\answer[tolerance=.001]{56756.2279}$ square units.  
\end{exercise}
\end{exercise}


%%%%%%%%%%%%%%%%%

To set up an integral with respect to $y$ that gives the area of the surface of revolution, do the following:  

Since we have chosen to integrate with respect to $y$, we use the result:

\[ SA = \int_{x=a}^{x=b} 2 \pi r ds\]

and we must express $r$ in terms of $y$ and $ds$ in terms of $y$ and $dy$.  


Let's start by finding $ds$.  Since we integrate with respect to $y$, we use $ds = \sqrt{1+\left(\frac{dx}{dy}\right)^2} dy$. So: 

\[
ds = \sqrt{\answer{1+144e^{6y}}} dy
\]


\begin{exercise}
For the radius $r$, we find: $r=\answer{2-y}$.  


\begin{exercise}
Now we see that an integral that gives the surface area is: 
\[
SA= \int_{y=c}^{y=d} 2 \pi r ds = \int_{y=\answer{\ln(2)}}^{y=\answer{1}} \answer{ 2\pi (2-y)} \sqrt{\answer{1+144e^{6y}}} dy
\]

\begin{exercise}
Using computational software of your choice, the integral to 4 decimal places shows that the surface area is $\answer[tolerance=.001]{56756.2279}$ square units.  

Does this agree with he previous result?

\begin{multipleChoice}
\choice[correct]{Yes}
\choice{No}
\end{multipleChoice}

\end{exercise}



\end{exercise}
\end{exercise}
\end{exercise}
\end{document}
