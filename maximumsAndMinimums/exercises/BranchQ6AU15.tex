\documentclass{ximera}


\graphicspath{
  {./}
  {ximeraTutorial/}
  {basicPhilosophy/}
}

\newcommand{\mooculus}{\textsf{\textbf{MOOC}\textnormal{\textsf{ULUS}}}}

\usepackage{tkz-euclide}\usepackage{tikz}
\usepackage{tikz-cd}
\usetikzlibrary{arrows}
\tikzset{>=stealth,commutative diagrams/.cd,
  arrow style=tikz,diagrams={>=stealth}} %% cool arrow head
\tikzset{shorten <>/.style={ shorten >=#1, shorten <=#1 } } %% allows shorter vectors

\usetikzlibrary{backgrounds} %% for boxes around graphs
\usetikzlibrary{shapes,positioning}  %% Clouds and stars
\usetikzlibrary{matrix} %% for matrix
\usepgfplotslibrary{polar} %% for polar plots
\usepgfplotslibrary{fillbetween} %% to shade area between curves in TikZ
\usetkzobj{all}
\usepackage[makeroom]{cancel} %% for strike outs
%\usepackage{mathtools} %% for pretty underbrace % Breaks Ximera
%\usepackage{multicol}
\usepackage{pgffor} %% required for integral for loops



%% http://tex.stackexchange.com/questions/66490/drawing-a-tikz-arc-specifying-the-center
%% Draws beach ball
\tikzset{pics/carc/.style args={#1:#2:#3}{code={\draw[pic actions] (#1:#3) arc(#1:#2:#3);}}}



\usepackage{array}
\setlength{\extrarowheight}{+.1cm}
\newdimen\digitwidth
\settowidth\digitwidth{9}
\def\divrule#1#2{
\noalign{\moveright#1\digitwidth
\vbox{\hrule width#2\digitwidth}}}






\DeclareMathOperator{\arccot}{arccot}
\DeclareMathOperator{\arcsec}{arcsec}
\DeclareMathOperator{\arccsc}{arccsc}

















%%This is to help with formatting on future title pages.
\newenvironment{sectionOutcomes}{}{}


\begin{document}
\author{Nela Lakos}
\outcome{Local maximums and minimums.}
\outcome{Critical numbers.}
\outcome{First derivative test.}
\outcome{Second derivative test.}


\begin{exercise}

Let $f(x) = 20 + 8x^2 - x^4$.\\
(a) Find all the critical number of $f$ and classify them, i.e., for each critical number of $f$ decide whether it is a local minimum, local maximum or neither.
(b) Find all inflection points of $f$.

(a) First, we have to compute $f'(x)$.

$f'(x) = \answer{16x-4x^3}$.

Complete the statement below.

The  x-coordinates of all critical numbers of $f$ (from left to right) are $a=\answer{-2}$, $b=\answer{0}$, and $c=\answer{2}$.

%The absolute maximum of $f$ on the interval $[-1,3]$ is $\answer{36}$ and it occurs at $x=\answer{2}$.

%The absolute minimum of $f$ on the interval $[-1,3]$ is $\answer{11}$ and it occurs at $x=\answer{3}$.


\begin{exercise}
Now compute
\[
f''(x)=\answer{16-12x^2}
\]

Now, we will evaluate the second derivative at all the critical numbers of $f$ and apply the second derivative test to determine whether the function $f$ has a local extremum at any of those points.
\[
f''(a)=\answer{-32}
\]
\begin{exercise}
Because the value of $f''(a)$ is \wordChoice{\choice{positive}\choice[correct]{negative}\choice{zero}}, the function $f$ has  \wordChoice{\choice[correct]{a local maximum}\choice{a local minimum}\choice{neither a local maximum nor a local minimum}\choice{possibly a local maximum, a local minimum or neither}} at $x=a$ by the second derivative test.
\[
f''(b)=\answer{16}
\]
\begin{exercise}
Because the value of $f''(b)$ is \wordChoice{\choice[correct]{positive}\choice{negative}\choice{zero}}, the function $f$ has\wordChoice{\choice{a local maximum}\choice[correct]{a local minimum}\choice{neither a local maximum nor a local minimum}\choice{possibly a local maximum, a local minimum or neither}}  at $x=b$ by the second derivative test.

\[
f''(c)=\answer{-32}
\]
\begin{exercise}
Because the value of $f''(c)$ is \wordChoice{\choice{positive}\choice[correct]{negative}\choice{zero}}, the function $f$ has\wordChoice{\choice[correct]{a local maximum}\choice{a local minimum}\choice{neither a local maximum nor a local minimum}\choice{possibly a local maximum, a local minimum or neither}} at $x=c$  by the second derivative test.\\


(b) Complete the statement below.
\begin{exercise}

Since $f''(x)=\answer{16-12x^2}$, it follows that inflection point might occur at  (from left to right) $m=\answer{-\frac{2}{\sqrt{3}}}$ and $n=\answer{\frac{2}{\sqrt{3}}}$.\\

In order to check whether the function $f$ has an inflection point at either $x=m$ or $x=n$, we have to check the sign of $f''(x)$ on the intervals $(-\infty,m)$, $(m,n)$, and $(n,+\infty)$.

Since $a < m < b < n < c$, we can use the values $f''(a)$, $f''(b)$, and $f''(c)$ to determine the sign of $f''(x)$ on these intervals.
\begin{exercise}


 $f''(x)$ is \wordChoice{\choice{positive}\choice[correct]{negative}\choice{zero}} on the interval $(-\infty,m)$.\\
 
  $f''(x)$ is \wordChoice{\choice[correct]{positive}\choice{negative}\choice{zero}} on the interval $(m,n)$.\\
  
    $f''(x)$ is \wordChoice{\choice{positive}\choice[correct]{negative}\choice{zero}} on the interval  $(n,+\infty)$.\\
  \begin{exercise}
 Make a correct choice.\\
 
  The function $f$ \wordChoice{\choice[correct]{has}\choice{does not have}} an inflection point at $x=m$, because the sign of $f''(x)$ \wordChoice{\choice[correct]{changes}\choice{does not change} } at $x=m$.\\
  The function $f$ \wordChoice{\choice[correct]{has}\choice{does not have}} an inflection point at $x=n$, because the sign of $f''(x)$ \wordChoice{\choice[correct]{changes}\choice{does not change} } at $x=n$.\\
 
  \end{exercise}
 \end{exercise}
\end{exercise}
\end{exercise}
\end{exercise}
\end{exercise}
\end{exercise}
\end{exercise}
\end{document}
