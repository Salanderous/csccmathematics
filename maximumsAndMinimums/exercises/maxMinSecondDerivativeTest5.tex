\documentclass{ximera}

\graphicspath{
  {./}
  {ximeraTutorial/}
  {basicPhilosophy/}
}

\newcommand{\mooculus}{\textsf{\textbf{MOOC}\textnormal{\textsf{ULUS}}}}

\usepackage{tkz-euclide}\usepackage{tikz}
\usepackage{tikz-cd}
\usetikzlibrary{arrows}
\tikzset{>=stealth,commutative diagrams/.cd,
  arrow style=tikz,diagrams={>=stealth}} %% cool arrow head
\tikzset{shorten <>/.style={ shorten >=#1, shorten <=#1 } } %% allows shorter vectors

\usetikzlibrary{backgrounds} %% for boxes around graphs
\usetikzlibrary{shapes,positioning}  %% Clouds and stars
\usetikzlibrary{matrix} %% for matrix
\usepgfplotslibrary{polar} %% for polar plots
\usepgfplotslibrary{fillbetween} %% to shade area between curves in TikZ
\usetkzobj{all}
\usepackage[makeroom]{cancel} %% for strike outs
%\usepackage{mathtools} %% for pretty underbrace % Breaks Ximera
%\usepackage{multicol}
\usepackage{pgffor} %% required for integral for loops



%% http://tex.stackexchange.com/questions/66490/drawing-a-tikz-arc-specifying-the-center
%% Draws beach ball
\tikzset{pics/carc/.style args={#1:#2:#3}{code={\draw[pic actions] (#1:#3) arc(#1:#2:#3);}}}



\usepackage{array}
\setlength{\extrarowheight}{+.1cm}
\newdimen\digitwidth
\settowidth\digitwidth{9}
\def\divrule#1#2{
\noalign{\moveright#1\digitwidth
\vbox{\hrule width#2\digitwidth}}}






\DeclareMathOperator{\arccot}{arccot}
\DeclareMathOperator{\arcsec}{arcsec}
\DeclareMathOperator{\arccsc}{arccsc}

















%%This is to help with formatting on future title pages.
\newenvironment{sectionOutcomes}{}{}

\author{Steven Gubkin}
\license{Creative Commons 3.0 By-NC}
\outcome{ Define a critical number.}
\outcome{ Find critical numbers.}
\outcome{ Define absolute maximum and absolute minimum.}
\outcome{ Find the absolute max or min of a continuous function on a closed interval.}
\outcome{ Define local maximum and local minimum.}
\outcome{ Compare and contrast local and absolute maxima and minima.}
\outcome{ Identify situations in which an absolute maximum or minimum is guaranteed.}
\outcome{ Classify critical numbers.}
\outcome{ State the First Derivative Test.}
\outcome{ Apply the First Derivative Test.}
\outcome{ State the Second Derivative Test.}
\outcome{ Apply the Second Derivative Test.}
\begin{document}

\begin{exercise}

The function $f(x) = x^4-4x^3+16x-3$ has two critical numbers. If we
call these critical numbers $a$ and $b$, and order them such that $a <
b$, then

$$
a = \answer{-1}
$$

$$
b=\answer{2}
$$

At $x=a$, the second derivative test
\begin{multipleChoice}
\choice{Indicates a local maxima}
\choice[correct]{Indicates a local minima}
\choice{Fails, but the First derivative test indicates a local max}
\choice{Fails, but the First derivative test indicates a local min}
\choice{Fails, and the First derivative test indicates that it is not a local extrema}
\end{multipleChoice}

At $x=b$, the second derivative test
\begin{multipleChoice}
\choice{Indicates a local maxima}
\choice{Indicates a local minima}
\choice{Fails, but the First derivative test indicates a local max}
\choice{Fails, but the First derivative test indicates a local min}
\choice[correct]{Fails, and the First derivative test indicates that it is not a local extrema}
\end{multipleChoice}

\end{exercise}
\end{document}

