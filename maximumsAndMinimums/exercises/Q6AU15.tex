\documentclass{ximera}


\graphicspath{
  {./}
  {ximeraTutorial/}
  {basicPhilosophy/}
}

\newcommand{\mooculus}{\textsf{\textbf{MOOC}\textnormal{\textsf{ULUS}}}}

\usepackage{tkz-euclide}\usepackage{tikz}
\usepackage{tikz-cd}
\usetikzlibrary{arrows}
\tikzset{>=stealth,commutative diagrams/.cd,
  arrow style=tikz,diagrams={>=stealth}} %% cool arrow head
\tikzset{shorten <>/.style={ shorten >=#1, shorten <=#1 } } %% allows shorter vectors

\usetikzlibrary{backgrounds} %% for boxes around graphs
\usetikzlibrary{shapes,positioning}  %% Clouds and stars
\usetikzlibrary{matrix} %% for matrix
\usepgfplotslibrary{polar} %% for polar plots
\usepgfplotslibrary{fillbetween} %% to shade area between curves in TikZ
\usetkzobj{all}
\usepackage[makeroom]{cancel} %% for strike outs
%\usepackage{mathtools} %% for pretty underbrace % Breaks Ximera
%\usepackage{multicol}
\usepackage{pgffor} %% required for integral for loops



%% http://tex.stackexchange.com/questions/66490/drawing-a-tikz-arc-specifying-the-center
%% Draws beach ball
\tikzset{pics/carc/.style args={#1:#2:#3}{code={\draw[pic actions] (#1:#3) arc(#1:#2:#3);}}}



\usepackage{array}
\setlength{\extrarowheight}{+.1cm}
\newdimen\digitwidth
\settowidth\digitwidth{9}
\def\divrule#1#2{
\noalign{\moveright#1\digitwidth
\vbox{\hrule width#2\digitwidth}}}






\DeclareMathOperator{\arccot}{arccot}
\DeclareMathOperator{\arcsec}{arcsec}
\DeclareMathOperator{\arccsc}{arccsc}

















%%This is to help with formatting on future title pages.
\newenvironment{sectionOutcomes}{}{}


\begin{document}
\author{Kyle Parson\and Nela Lakos}
\outcome{Classify critical numbers.}
\outcome{Apply the First Derivative Test.}
\outcome{Apply the Second Derivative Test.}
\outcome{Find inflection points.}


\begin{exercise}
Let $f(x)=x\ln x$. The domain of $f$ is
$(\answer{0},\answer{\infty})$.
\begin{exercise}
\begin{selectAll}
\choice{$f$ is not everywhere continuous on $(0,\infty)$}
\choice[correct]{$f$ is continuous on $[M,\infty)$ for some $M>0$}
\choice{There is some $M>0$ such that $f$ is not continuous on $[M,\infty)$.}
\choice{$f$ is nowhere continuous on $(0,\infty)$}
\choice[correct]{$f$ is continuous on $(\varepsilon,\infty)$ for every $\varepsilon>0$.}
\choice[correct]{$f$ is continuous on $(0,\infty)$}
\end{selectAll}
\begin{exercise}
The preceding statements regarding $f$ are true because
\begin{multipleChoice}
\choice{$f$ is the product of two functions whose intervals of continuity have union equal to $(0,\infty)$.}
\choice{the natural logarithm is not continuous.}
\choice{the limit $\lim_{x\to 0^+}f(x)$ exists and is equal to $0$.}
\choice[correct]{$f$ is the product of two functions which are continuous on $(0,\infty)$.}
\end{multipleChoice}
\begin{exercise}
Next, we shall classify the critical numbers of $f$.

First, compute 
\[
f'(x)=\answer{1+\ln(x)}
\]
\begin{exercise}
The function $f$ has $\answer{1}$ critical number(s). 
\begin{exercise}
The single critical number of $f$ is located at
\[
x=\answer{1/e}
\]
\begin{exercise}
Now compute
\[
f''(x)=\answer{\frac{1}{x}}
\]
\begin{exercise}
Because the value of $f''(\frac{1}{e})=\frac{1}{1/e}=e$ is \wordChoice{\choice[correct]{positive}\choice{negative}\choice{zero}}, the critical number the function  $f$ has at $x=1/e$ is \wordChoice{\choice{a local maximum}\choice[correct]{a local minimum}\choice{neither a local maximum nor a local minimum}\choice{possibly a local maximum, a local minimum or neither}} by the second derivative test.

\begin{exercise}
Finally, we shall identify the inflection points of $f$ and the intervals on which $f$ is concave up or concave down.

The function $f$ has $\answer{0}$ inflection point(s).
\begin{exercise}
Since
\[
\frac{d^2}{dx^2}f(x)=\answer{1/x}
\]
$f''(x)$ is \wordChoice{\choice[correct]{positive}\choice{negative}} on $(0,\infty)$. 
\begin{exercise}
Since $f''(x)>0$ for all $x>0$, $f$ is \wordChoice{\choice[correct]{concave up}\choice{concave down}} on $(\answer{0},\answer{\infty})$.
\end{exercise}
\end{exercise}
\end{exercise}
\end{exercise}
\end{exercise}
\end{exercise}
\end{exercise}
\end{exercise}
\end{exercise}
\end{exercise}
\end{exercise}
\end{document}
