\documentclass{ximera}


\graphicspath{
  {./}
  {ximeraTutorial/}
  {basicPhilosophy/}
}

\newcommand{\mooculus}{\textsf{\textbf{MOOC}\textnormal{\textsf{ULUS}}}}

\usepackage{tkz-euclide}\usepackage{tikz}
\usepackage{tikz-cd}
\usetikzlibrary{arrows}
\tikzset{>=stealth,commutative diagrams/.cd,
  arrow style=tikz,diagrams={>=stealth}} %% cool arrow head
\tikzset{shorten <>/.style={ shorten >=#1, shorten <=#1 } } %% allows shorter vectors

\usetikzlibrary{backgrounds} %% for boxes around graphs
\usetikzlibrary{shapes,positioning}  %% Clouds and stars
\usetikzlibrary{matrix} %% for matrix
\usepgfplotslibrary{polar} %% for polar plots
\usepgfplotslibrary{fillbetween} %% to shade area between curves in TikZ
\usetkzobj{all}
\usepackage[makeroom]{cancel} %% for strike outs
%\usepackage{mathtools} %% for pretty underbrace % Breaks Ximera
%\usepackage{multicol}
\usepackage{pgffor} %% required for integral for loops



%% http://tex.stackexchange.com/questions/66490/drawing-a-tikz-arc-specifying-the-center
%% Draws beach ball
\tikzset{pics/carc/.style args={#1:#2:#3}{code={\draw[pic actions] (#1:#3) arc(#1:#2:#3);}}}



\usepackage{array}
\setlength{\extrarowheight}{+.1cm}
\newdimen\digitwidth
\settowidth\digitwidth{9}
\def\divrule#1#2{
\noalign{\moveright#1\digitwidth
\vbox{\hrule width#2\digitwidth}}}






\DeclareMathOperator{\arccot}{arccot}
\DeclareMathOperator{\arcsec}{arcsec}
\DeclareMathOperator{\arccsc}{arccsc}

















%%This is to help with formatting on future title pages.
\newenvironment{sectionOutcomes}{}{}


\title{Sums of sequences}

\begin{document}

\begin{abstract}
%Stuff can go here later if we want!
\end{abstract}

\maketitle




We have noted that decimal approximations or expansions can be viewed as a sequence of numbers, where a new digit is added for each term.


\begin{itemize}
\item $4.1$
\item $4.12$
\item $4.123$
\item $4.1231$
\item $4.12310$
\item $4.123105$
\item $4.1231056$
\item $4.12310562$
\item $4.123105626$
\end{itemize}


For decimal expansions, this pattern continues with no end. \\

We can view this decimnal expansion in a different way: as a sum. \\


\[
4 + \frac{1}{10^1} + \frac{2}{10^2} + \frac{3}{10^3} + \frac{1}{10^4} + \frac{0}{10^5} + \frac{5}{10^6} + \frac{6}{10^7} + \frac{2}{10^8} + \frac{6}{10^9} + \cdots
\]



We would like to connect these two views.


One way is to make a sequence of the place value digits:  

\[
\frac{4}{10^0}, \frac{41}{10^1},  \frac{412}{10^2}, \frac{4123}{10^3}, \frac{41231}{10^4}, \frac{412310}{10^5}, \frac{4123105}{10^6}, \frac{41231056}{10^7}, \frac{412310562}{10^8}, \frac{4123105626}{10^9}, \cdots
\]


and then add them up:



\[
4 + \frac{1}{10^1} + \frac{2}{10^2} + \frac{3}{10^3} + \frac{1}{10^4} + \frac{0}{10^5} + \frac{5}{10^6} + \frac{6}{10^7} + \frac{2}{10^8} + \frac{6}{10^9} + \cdots
\]

When summing up the terms in a sequence we obtain a \textbf{series}, a new way to represent numbers. \\


$\blacktriangleright$  $\sqrt{17} = 0.123105626\cdots$ \\



$\blacktriangleright$ $\sqrt{17} =  \lim\limits_{\infty} \, \{ \frac{4}{10^0}, \frac{41}{10^1},  \frac{412}{10^2}, \frac{4123}{10^3}, \frac{41231}{10^4}, \frac{412310}{10^5}, \frac{4123105}{10^6}, \frac{41231056}{10^7}, \frac{412310562}{10^8}, \frac{4123105626}{10^9}, \cdots \}$ \\



$\blacktriangleright$ $\sqrt{17} = \lim\limits_{\infty} \, 4 + \frac{1}{10^1} + \frac{2}{10^2} + \frac{3}{10^3} + \frac{1}{10^4} + \frac{0}{10^5} + \frac{5}{10^6} + \frac{6}{10^7} + \frac{2}{10^8} + \frac{6}{10^9} + \cdots$ \\




















\begin{sectionOutcomes}

After completing this section, students will be aqainted with

\begin{itemize}
\item defining a series.
\item recognizing a geometric series.
\item recognizing a telescoping series.
\item computing the sum of a geometric series.
\item computing the sum of a telescoping series.
\end{itemize}

\end{sectionOutcomes}

\end{document}
