\documentclass{ximera}


\graphicspath{
  {./}
  {ximeraTutorial/}
  {basicPhilosophy/}
}

\newcommand{\mooculus}{\textsf{\textbf{MOOC}\textnormal{\textsf{ULUS}}}}

\usepackage{tkz-euclide}\usepackage{tikz}
\usepackage{tikz-cd}
\usetikzlibrary{arrows}
\tikzset{>=stealth,commutative diagrams/.cd,
  arrow style=tikz,diagrams={>=stealth}} %% cool arrow head
\tikzset{shorten <>/.style={ shorten >=#1, shorten <=#1 } } %% allows shorter vectors

\usetikzlibrary{backgrounds} %% for boxes around graphs
\usetikzlibrary{shapes,positioning}  %% Clouds and stars
\usetikzlibrary{matrix} %% for matrix
\usepgfplotslibrary{polar} %% for polar plots
\usepgfplotslibrary{fillbetween} %% to shade area between curves in TikZ
\usetkzobj{all}
\usepackage[makeroom]{cancel} %% for strike outs
%\usepackage{mathtools} %% for pretty underbrace % Breaks Ximera
%\usepackage{multicol}
\usepackage{pgffor} %% required for integral for loops



%% http://tex.stackexchange.com/questions/66490/drawing-a-tikz-arc-specifying-the-center
%% Draws beach ball
\tikzset{pics/carc/.style args={#1:#2:#3}{code={\draw[pic actions] (#1:#3) arc(#1:#2:#3);}}}



\usepackage{array}
\setlength{\extrarowheight}{+.1cm}
\newdimen\digitwidth
\settowidth\digitwidth{9}
\def\divrule#1#2{
\noalign{\moveright#1\digitwidth
\vbox{\hrule width#2\digitwidth}}}






\DeclareMathOperator{\arccot}{arccot}
\DeclareMathOperator{\arcsec}{arcsec}
\DeclareMathOperator{\arccsc}{arccsc}

















%%This is to help with formatting on future title pages.
\newenvironment{sectionOutcomes}{}{}


\title{Absolute Value}

\begin{document}

\begin{abstract}
traits
\end{abstract}
\maketitle





Absolute value makes negative numbers positive and leaves nonnegative numbers alone.  Of course, there is no algebra called "make positive".  Instead, we need a piecewise defined function.






\begin{definition} \textbf{\textcolor{green!50!black}{Absolute Value}} 
\[
|k| = 
\begin{cases}
  -k & \text{if $k<0$,}\\
   k & \text{if $k\ge 0$}.
\end{cases}
\]
\end{definition}

The symbol for the absolute value function are two vertical bars surrounding the domain value.





Graph of $y = |k|$.


\begin{image}
\begin{tikzpicture} 
  \begin{axis}[
            domain=-10:10, ymax=10, xmax=10, ymin=-10, xmin=-10,
            axis lines =center, xlabel=$k$, ylabel=$y$,
            ytick={-10,-8,-6,-4,-2,2,4,6,8,10},
            xtick={-10,-8,-6,-4,-2,2,4,6,8,10},
            ticklabel style={font=\scriptsize},
            every axis y label/.style={at=(current axis.above origin),anchor=south},
            every axis x label/.style={at=(current axis.right of origin),anchor=west},
            axis on top
          ]
          
          \addplot [line width=2, penColor, smooth, samples=200, domain=(-9:9),<->] {abs(x)};
        

  \end{axis}
\end{tikzpicture}
\end{image}









\begin{example}  Shifting

Graph of $y = P(t) = |t+4|$.


\begin{image}
\begin{tikzpicture} 
  \begin{axis}[
            domain=-10:10, ymax=10, xmax=10, ymin=-10, xmin=-10,
            axis lines =center, xlabel=$t$, ylabel=$y$,
            ytick={-10,-8,-6,-4,-2,2,4,6,8,10},
            xtick={-10,-8,-6,-4,-2,2,4,6,8,10},
            ticklabel style={font=\scriptsize},
            every axis y label/.style={at=(current axis.above origin),anchor=south},
            every axis x label/.style={at=(current axis.right of origin),anchor=west},
            axis on top
          ]
          
          \addplot [line width=2, penColor, smooth, samples=200, domain=(-9:6),<->] {abs(x+4)};
        

  \end{axis}
\end{tikzpicture}
\end{image}

The graph is positioned by its corner, which occurs when the inside of the absolute value formula equals $0$.  $t+4=0$, when $t=-4$.


$P$ has a global minimum of $\answer{0}$, which occurs at $\answer{-4}$. $P$ has no maximums.

$P$ decreases on $(-\infty, -4]$ and increases on $[-4, \infty)$ 




\[
P(t) = |t+4| = 
\begin{cases}
  -(t+4) & \text{if $t<\answer{-4}$,}\\
   t+4 & \text{if $t\ge \answer{-4}$}.
\end{cases}
\]





\end{example}


















\begin{example}  Solving Equations

Let $B(r) = |r-3| + 2$.


\begin{image}
\begin{tikzpicture} 
  \begin{axis}[
            domain=-10:10, ymax=10, xmax=10, ymin=-10, xmin=-10,
            axis lines =center, xlabel=$r$, ylabel=$y$,
            ytick={-10,-8,-6,-4,-2,2,4,6,8,10},
            xtick={-10,-8,-6,-4,-2,2,4,6,8,10},
            ticklabel style={font=\scriptsize},
            every axis y label/.style={at=(current axis.above origin),anchor=south},
            every axis x label/.style={at=(current axis.right of origin),anchor=west},
            axis on top
          ]
          
          \addplot [line width=2, penColor, smooth, samples=200, domain=(-5:9),<->] {abs(x-3)+2};
        

  \end{axis}
\end{tikzpicture}
\end{image}



$\blacktriangleright$ Solve $B(r) = 5$.

We can see from the graph that there are two solutions to this equation. Our first goal in solving this equation is to get the absolute term by itself.


$|r-3| + 2 = 5$


$|r-3| = 3$

There are only two numbers whose absolute value is $3$.  They are $\answer{-3}$ or $\answer{3}$.

We have two possibilities, either the inside of the absolute value, $r-3=-3$ or $r-3=3$.

If $r-3=-3$, then $r=0$. If $r-3=3$, then $r=6$. 











\begin{image}
\begin{tikzpicture} 
  \begin{axis}[
            domain=-10:10, ymax=10, xmax=10, ymin=-10, xmin=-10,
            axis lines =center, xlabel=$r$, ylabel=$y$,
            every axis y label/.style={at=(current axis.above origin),anchor=south},
            every axis x label/.style={at=(current axis.right of origin),anchor=west},
            axis on top
          ]
          
           \addplot [line width=2, penColor, smooth, samples=200, domain=(-5:9),<->] {abs(x-3)+2};
           \addplot [line width=2, penColor3, smooth, samples=200, domain=(-9:9)] {5};

           \addplot[color=penColor,fill=penColor3,only marks,mark=*] coordinates{(0,5)};
           \addplot[color=penColor,fill=penColor3,only marks,mark=*] coordinates{(6,5)};
        

  \end{axis}
\end{tikzpicture}
\end{image}



We can also turn to the definition of the absolute value function.


The function $|r-3| + 2$ can be written as 



\[
|r-3| + 2 = 
\begin{cases}
  -(r-3) + 2  & \text{if $t<\answer{3}$,}\\
   r-3 + 2  & \text{if $t\ge \answer{3}$}.
\end{cases}
\]


We are looking for where $-(r-3) + 2 = 5$ and $ t < 3$  \\

or \\

where $(r-3) + 2 = 5$ and $t \ge 3$.  \\


These produce the solutions $r=0$ and $r=6$










\end{example}



















\begin{center}
\textbf{\textcolor{green!50!black}{ooooo=-=-=-=-=-=-=-=-=-=-=-=-=ooOoo=-=-=-=-=-=-=-=-=-=-=-=-=ooooo}} \\

more examples can be found by following this link\\ \link[More Examples of Analysis]{https://ximera.osu.edu/csccmathematics/precalculus1/precalculus1/libraryAnalysis2/examples/exampleList}

\end{center}




\end{document}
