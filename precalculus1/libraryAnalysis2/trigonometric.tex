\documentclass{ximera}


\graphicspath{
  {./}
  {ximeraTutorial/}
  {basicPhilosophy/}
}

\newcommand{\mooculus}{\textsf{\textbf{MOOC}\textnormal{\textsf{ULUS}}}}

\usepackage{tkz-euclide}\usepackage{tikz}
\usepackage{tikz-cd}
\usetikzlibrary{arrows}
\tikzset{>=stealth,commutative diagrams/.cd,
  arrow style=tikz,diagrams={>=stealth}} %% cool arrow head
\tikzset{shorten <>/.style={ shorten >=#1, shorten <=#1 } } %% allows shorter vectors

\usetikzlibrary{backgrounds} %% for boxes around graphs
\usetikzlibrary{shapes,positioning}  %% Clouds and stars
\usetikzlibrary{matrix} %% for matrix
\usepgfplotslibrary{polar} %% for polar plots
\usepgfplotslibrary{fillbetween} %% to shade area between curves in TikZ
\usetkzobj{all}
\usepackage[makeroom]{cancel} %% for strike outs
%\usepackage{mathtools} %% for pretty underbrace % Breaks Ximera
%\usepackage{multicol}
\usepackage{pgffor} %% required for integral for loops



%% http://tex.stackexchange.com/questions/66490/drawing-a-tikz-arc-specifying-the-center
%% Draws beach ball
\tikzset{pics/carc/.style args={#1:#2:#3}{code={\draw[pic actions] (#1:#3) arc(#1:#2:#3);}}}



\usepackage{array}
\setlength{\extrarowheight}{+.1cm}
\newdimen\digitwidth
\settowidth\digitwidth{9}
\def\divrule#1#2{
\noalign{\moveright#1\digitwidth
\vbox{\hrule width#2\digitwidth}}}






\DeclareMathOperator{\arccot}{arccot}
\DeclareMathOperator{\arcsec}{arcsec}
\DeclareMathOperator{\arccsc}{arccsc}

















%%This is to help with formatting on future title pages.
\newenvironment{sectionOutcomes}{}{}


\title{Trigonometric}

\begin{document}

\begin{abstract}
facets
\end{abstract}
\maketitle




Rational functions were quotients of polynomials.  We can do the same with trigonometric functions.


\begin{definition}  \textbf{\textcolor{green!50!black}{Tangent}} 


\textbf{Tangent} is the quotient of sine and cosine


\[   \tan(\theta) = \frac{\sin(\theta)}{\cos(\theta)}      \]



Its domain is all real numbers except the zeros of cosine, $\left\{  \frac{(2k+1)\pi}{2}   \, | \,   k \in \textbf{Z}   \right\}$.

\end{definition}


At each of these singularities, the graph has a vertical asymptote, where the function becomes unbounded.




Graph of $z = \tan(\theta)$.





\begin{image}
\begin{tikzpicture}
  \begin{axis}[
            xmin=-6.75,xmax=6.75,ymin=-5,ymax=5,
            axis lines=center,
            width=6in,
            height=3in,
            xtick={-6.28, -4.71, -3.14, -1.57, 0, 1.57, 3.142, 4.71, 6.28},
            xticklabels={$-2\pi$,$\frac{-3\pi}{2}$,$-\pi$, $\frac{-\pi}{2}$, $0$, $\frac{\pi}{2}$, $\pi$, $\frac{3\pi}{2}$, $2\pi$},       
            unit vector ratio*=1 1 1,
            xlabel=$\theta$, ylabel=$z$,
            ticklabel style={font=\scriptsize},
            every axis y label/.style={at=(current axis.above origin),anchor=south},
            every axis x label/.style={at=(current axis.right of origin),anchor=west},
          ]        
          \addplot [line width=2, penColor, samples=200,smooth, domain=(-1.37:1.37),<->] {tan(deg(x))};
          \addplot [line width=2, penColor, samples=100,smooth, domain=(-4.51:-1.76),<->] {tan(deg(x))};
          \addplot [line width=2, penColor, samples=100,smooth, domain=(-6.08:-4.9),<->] {tan(deg(x))};
          \addplot [line width=2, penColor, samples=100,smooth, domain=(1.76:4.51),<->] {tan(deg(x))};
          \addplot [line width=2, penColor, samples=100,smooth, domain=(4.9:6.75),<->] {tan(deg(x))};
          
          \addplot [textColor,dashed] plot coordinates {(-4.71,-5) (-4.71,5)};
          \addplot [textColor,dashed] plot coordinates {(-1.57,-5) (-1.57,5)};
          \addplot [textColor,dashed] plot coordinates {(1.57,-5) (1.57,5)};
          \addplot [textColor,dashed] plot coordinates {(4.71,-5) (4.71,5)};
          
          %\node at (axis cs:.4,1.25) [penColor3] {$\tan(\theta)$};    
        \end{axis}
\end{tikzpicture}

\end{image}



Tangent takes on every value in $\mathbb{R}$ exactly once inside the open interval $\left( \frac{-\pi}{2}, \frac{\pi}{2} \right)$.  It is then periodic with a period of $\pi$.

Tangent is always \wordChoice{\choice[correct]{increasing} \choice{decreasing}} inside its domain.

Tangent is unbounded and has no global or local maximums or minimums.

Tangent has an infinite number of zeros. They coincide with the zeros of \wordChoice{\choice[correct]{sine} \choice{cosine}} : $\{  k \pi    \, | \,   k \in \textbf{Z}        \}$























\begin{center}
\textbf{\textcolor{green!50!black}{ooooo=-=-=-=-=-=-=-=-=-=-=-=-=ooOoo=-=-=-=-=-=-=-=-=-=-=-=-=ooooo}} \\

more examples can be found by following this link\\ \link[More Examples of Analysis]{https://ximera.osu.edu/csccmathematics/precalculus1/precalculus1/libraryAnalysis2/examples/exampleList}

\end{center}




\end{document}
