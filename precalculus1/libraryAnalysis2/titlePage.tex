\documentclass{ximera}


\graphicspath{
  {./}
  {ximeraTutorial/}
  {basicPhilosophy/}
}

\newcommand{\mooculus}{\textsf{\textbf{MOOC}\textnormal{\textsf{ULUS}}}}

\usepackage{tkz-euclide}\usepackage{tikz}
\usepackage{tikz-cd}
\usetikzlibrary{arrows}
\tikzset{>=stealth,commutative diagrams/.cd,
  arrow style=tikz,diagrams={>=stealth}} %% cool arrow head
\tikzset{shorten <>/.style={ shorten >=#1, shorten <=#1 } } %% allows shorter vectors

\usetikzlibrary{backgrounds} %% for boxes around graphs
\usetikzlibrary{shapes,positioning}  %% Clouds and stars
\usetikzlibrary{matrix} %% for matrix
\usepgfplotslibrary{polar} %% for polar plots
\usepgfplotslibrary{fillbetween} %% to shade area between curves in TikZ
\usetkzobj{all}
\usepackage[makeroom]{cancel} %% for strike outs
%\usepackage{mathtools} %% for pretty underbrace % Breaks Ximera
%\usepackage{multicol}
\usepackage{pgffor} %% required for integral for loops



%% http://tex.stackexchange.com/questions/66490/drawing-a-tikz-arc-specifying-the-center
%% Draws beach ball
\tikzset{pics/carc/.style args={#1:#2:#3}{code={\draw[pic actions] (#1:#3) arc(#1:#2:#3);}}}



\usepackage{array}
\setlength{\extrarowheight}{+.1cm}
\newdimen\digitwidth
\settowidth\digitwidth{9}
\def\divrule#1#2{
\noalign{\moveright#1\digitwidth
\vbox{\hrule width#2\digitwidth}}}






\DeclareMathOperator{\arccot}{arccot}
\DeclareMathOperator{\arcsec}{arcsec}
\DeclareMathOperator{\arccsc}{arccsc}

















%%This is to help with formatting on future title pages.
\newenvironment{sectionOutcomes}{}{}


\title{Library Analysis 2}

\begin{document}

\begin{abstract}
%Stuff can go here later if we want!
\end{abstract}
\maketitle


Our goal in this course is to map out our own plan for analyzing functions.  Our starting point is the elementary functions, however, we also want to analyze new functions made from combinations of the elementary functions.  We will add, subtract, and multiply, them; create quotients of them; glue pieces of them together; repeat them; and other concoctions we will think up.

We want exact analysis.  We want to know exact information.  We want exact values and exact descriptions where these occur in the domain.

We can't always get exactness. 

Our overall plan includes a lot of algebra to help us track down exact values.  This algebra will slowly make its appearance in these courses.  So, what we approximate now, we may be able to calculate exactly later.  We would like to transition all of our approximating to exact values through algebra, although this will not always be possible.

This plan will continue through Calculus.  We will continue to develop algebraic tools and reasoning to make more exact values possible.

In this section, we are back to the library of elementary functions and another level of analysis.  How much can we describe exactly? How much do we need to approximate?



What do we want to know when we analyze functions?

We want to know the 
\begin{itemize}
\item \textbf{\textcolor{red!80!black}{domain}} 
\item \textbf{\textcolor{red!80!black}{range}} 
\item \textbf{\textcolor{red!80!black}{zeros}} 
\item \textbf{\textcolor{red!80!black}{discontinuities}} 
\item \textbf{\textcolor{red!80!black}{singularities}} 
\item \textbf{\textcolor{red!80!black}{intervals of increasing and decreasing}} 
\item \textbf{\textcolor{red!80!black}{global maximum and minimum}} 
\item \textbf{\textcolor{red!80!black}{local maximums and minimums}} 
\item \textbf{\textcolor{red!80!black}{symmetry}} 
\item \textbf{\textcolor{red!80!black}{endbehavior}}  \\
\item \textbf{\textcolor{purple!85!blue}{and we would like a nice graph}} 
\end{itemize}


\subsection{Expectations}

\begin{sectionOutcomes}
After completing this section, students should 

\begin{itemize}
\item recongize library functions inside formulas.
\item locate intervals of increasing and decreasing.
\item locate global and local maximums and minimums.
\item identify zeros.
\item identify strategic points for a graph.
\end{itemize}
\end{sectionOutcomes}










\begin{center}
\textbf{\textcolor{green!50!black}{ooooo=-=-=-=-=-=-=-=-=-=-=-=-=ooOoo=-=-=-=-=-=-=-=-=-=-=-=-=ooooo}} \\

more examples can be found by following this link\\ \link[More Examples of Analysis]{https://ximera.osu.edu/csccmathematics/precalculus1/precalculus1/libraryAnalysis2/examples/exampleList}

\end{center}







\end{document}
