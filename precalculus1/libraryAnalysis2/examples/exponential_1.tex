\documentclass{ximera}


\graphicspath{
  {./}
  {ximeraTutorial/}
  {basicPhilosophy/}
}

\newcommand{\mooculus}{\textsf{\textbf{MOOC}\textnormal{\textsf{ULUS}}}}

\usepackage{tkz-euclide}\usepackage{tikz}
\usepackage{tikz-cd}
\usetikzlibrary{arrows}
\tikzset{>=stealth,commutative diagrams/.cd,
  arrow style=tikz,diagrams={>=stealth}} %% cool arrow head
\tikzset{shorten <>/.style={ shorten >=#1, shorten <=#1 } } %% allows shorter vectors

\usetikzlibrary{backgrounds} %% for boxes around graphs
\usetikzlibrary{shapes,positioning}  %% Clouds and stars
\usetikzlibrary{matrix} %% for matrix
\usepgfplotslibrary{polar} %% for polar plots
\usepgfplotslibrary{fillbetween} %% to shade area between curves in TikZ
\usetkzobj{all}
\usepackage[makeroom]{cancel} %% for strike outs
%\usepackage{mathtools} %% for pretty underbrace % Breaks Ximera
%\usepackage{multicol}
\usepackage{pgffor} %% required for integral for loops



%% http://tex.stackexchange.com/questions/66490/drawing-a-tikz-arc-specifying-the-center
%% Draws beach ball
\tikzset{pics/carc/.style args={#1:#2:#3}{code={\draw[pic actions] (#1:#3) arc(#1:#2:#3);}}}



\usepackage{array}
\setlength{\extrarowheight}{+.1cm}
\newdimen\digitwidth
\settowidth\digitwidth{9}
\def\divrule#1#2{
\noalign{\moveright#1\digitwidth
\vbox{\hrule width#2\digitwidth}}}






\DeclareMathOperator{\arccot}{arccot}
\DeclareMathOperator{\arcsec}{arcsec}
\DeclareMathOperator{\arccsc}{arccsc}

















%%This is to help with formatting on future title pages.
\newenvironment{sectionOutcomes}{}{}



\author{Lee Wayand}

\begin{document}
\begin{exercise}



Completely analyze $y(x) = \left(\frac{4}{3}\right)^x - 3$ with its natural domain.



\textbf{\textcolor{blue!55!black}{Domain:}}  \\

$y$ is a shifted exponential function, therefore its natural domain is $(-\infty, \infty)$.




\textbf{\textcolor{blue!55!black}{Continuity:}}  \\

$y$ is a shifted exponential function, therefore it is continuous on its domain and has no discontinuities.


$y$ is a shifted exponential function, therefore it has no singularities.




\textbf{\textcolor{blue!55!black}{End-Behavior:}}  \\



$y$ is a shifted exponential function. The leading coefficient is $1$, which is positive.  The base is $\frac{4}{3}$, which is greater than $1$.  


$\blacktriangleright$ As $x$ becomes big and negative, then the exponent becomes big and negative.  We have a number greater than $1$ raised to big negative numbers.  That will approach $0$ and then we subtract $3$

\[
\lim\limits_{x \to -\infty} y(x) = -3
\]






$\blacktriangleright$ As $x$ becomes big and positive, then the exponent becomes big and positive.  We have a number greater than $1$ raised to big positive numbers. That becomes unbounded positively.

\[
\lim\limits_{x \to \infty} y(x) = \infty
\]


















\textbf{\textcolor{blue!55!black}{Behavior:}}  \\

$y$ is a shifted exponential function. The leading coefficient is $1$, which is positive.  The base is $\frac{4}{3}$, which is greater than $1$.  Therefore, $y$ is an increasing function.

Shifted exponential functions have no global or local maximums or minimums.


Exponential functions are either increasing or decreasing and since the end-behavior is $-3$ to $\infty$, we have that $y$ is an increasing function.




\textbf{\textcolor{blue!55!black}{Zeros:}}  \\


Since we have a continuous exponential function with negative and positive values, it will have a single zero.


The zero will occur when 

\[
\left(\frac{4}{3}\right)^x = 3
\]

$x$ is the thing you raise $\frac{4}{3}$ to, to get $3$.

\[
x = log_{\tfrac{4}{3}}(3)
\]




This all agrees with the graph.


\begin{image}
\begin{tikzpicture} 
  \begin{axis}[
            domain=-10:10, ymax=10, xmax=10, ymin=-10, xmin=-10,
            axis lines =center, xlabel=$x$, ylabel=$y$, 
            ytick={-10,-8,-6,-4,-2,2,4,6,8,10},
            xtick={-10,-8,-6,-4,-2,2,4,6,8,10},
            ticklabel style={font=\scriptsize},
            every axis y label/.style={at=(current axis.above origin),anchor=south},
            every axis x label/.style={at=(current axis.right of origin),anchor=west},
            axis on top
          ]
          
          \addplot [line width=1, gray, dashed,domain=(-9:9),<->] ({x},{-3});
          \addplot [line width=2, penColor, smooth,samples=200,domain=(-9:8.5),<->] {(1.333)^x - 3};
          

           

  \end{axis}
\end{tikzpicture}
\end{image}















\end{exercise}
\end{document}