\documentclass{ximera}


\graphicspath{
  {./}
  {ximeraTutorial/}
  {basicPhilosophy/}
}

\newcommand{\mooculus}{\textsf{\textbf{MOOC}\textnormal{\textsf{ULUS}}}}

\usepackage{tkz-euclide}\usepackage{tikz}
\usepackage{tikz-cd}
\usetikzlibrary{arrows}
\tikzset{>=stealth,commutative diagrams/.cd,
  arrow style=tikz,diagrams={>=stealth}} %% cool arrow head
\tikzset{shorten <>/.style={ shorten >=#1, shorten <=#1 } } %% allows shorter vectors

\usetikzlibrary{backgrounds} %% for boxes around graphs
\usetikzlibrary{shapes,positioning}  %% Clouds and stars
\usetikzlibrary{matrix} %% for matrix
\usepgfplotslibrary{polar} %% for polar plots
\usepgfplotslibrary{fillbetween} %% to shade area between curves in TikZ
\usetkzobj{all}
\usepackage[makeroom]{cancel} %% for strike outs
%\usepackage{mathtools} %% for pretty underbrace % Breaks Ximera
%\usepackage{multicol}
\usepackage{pgffor} %% required for integral for loops



%% http://tex.stackexchange.com/questions/66490/drawing-a-tikz-arc-specifying-the-center
%% Draws beach ball
\tikzset{pics/carc/.style args={#1:#2:#3}{code={\draw[pic actions] (#1:#3) arc(#1:#2:#3);}}}



\usepackage{array}
\setlength{\extrarowheight}{+.1cm}
\newdimen\digitwidth
\settowidth\digitwidth{9}
\def\divrule#1#2{
\noalign{\moveright#1\digitwidth
\vbox{\hrule width#2\digitwidth}}}






\DeclareMathOperator{\arccot}{arccot}
\DeclareMathOperator{\arcsec}{arcsec}
\DeclareMathOperator{\arccsc}{arccsc}

















%%This is to help with formatting on future title pages.
\newenvironment{sectionOutcomes}{}{}



\author{Lee Wayand}

\begin{document}
\begin{exercise}



Completely analyze $y(k) = -2 ln(-k+5)$ with its natural domain.



\textbf{\textcolor{blue!55!black}{Domain:}}  \\

$y$ is a logarithmic function, therefore its natural domain contains those numbers that make the inside positive. 

\[
-k + 5 > 0
\]

The naatural domain is $(-\infty, 5)$.




\textbf{\textcolor{blue!55!black}{Continuity:}}  \\

$y$ is a logarithmic function, therefore it is continuous on its domain and has no discontinuities.


$y$ is a logarithmic function, therefore it has a singularity, when the inside is $0$, which is when $k = 5$.




\textbf{\textcolor{blue!55!black}{Behavior Near a Singularity:}}  \\



$\blacktriangleright$ $y$ is a logarithmic function. The leading coefficient is $-2$, which is negative.  When the domain gets near the singularity $ln(-k+5)$ becomes unbounded negatively.  Since the leading coefficient is negative, we have

\[
\lim\limits_{k \to 5^-} y(k) = \infty
\]





\textbf{\textcolor{blue!55!black}{End-Behavior:}}  \\


$\blacktriangleright$ As $x$ becomes big and negative, then the inside becomes big and positive.  $ln(big positive)$  is big and positive.  Since the leading coefficient is negative, we have

\[
\lim\limits_{k \to \infty} y(x) = -\infty
\]








\textbf{\textcolor{blue!55!black}{Behavior:}}  \\

$y$ is a logarithm function, therefore, is is either increasing or decreasing.  The leading coefficient is $-2$, which is negative.  


Logarithmic functions have no global or local maximums or minimums. Since the extreme behavior is $\infty$ to $-\infty$, we know that $y$ is a decreasing function.




\textbf{\textcolor{blue!55!black}{Zeros:}}  \\


Since we have a continuous logarithmic function with negative and positive values, it will have a single zero.


The zero will occur when 

\[
-2 ln(-k+5) = 0
\]

\[
ln(-k+5) = 2
\]



$ln(-k+5)$ is the thing you raise $e$ to, to get $-k+5$. Therefore, $2$ is the thing you raise $e$ to, to get $-k+5$. 

\[
e^2 = -k + 5
\]


\[
k = 5 - e^2 
\]



This all agrees with the graph.


\begin{image}
\begin{tikzpicture} 
  \begin{axis}[
            domain=-10:10, ymax=10, xmax=10, ymin=-10, xmin=-10,
            axis lines =center, xlabel=$x$, ylabel=$y$, 
            ytick={-10,-8,-6,-4,-2,2,4,6,8,10},
            xtick={-10,-8,-6,-4,-2,2,4,6,8,10},
            ticklabel style={font=\scriptsize},
            every axis y label/.style={at=(current axis.above origin),anchor=south},
            every axis x label/.style={at=(current axis.right of origin),anchor=west},
            axis on top
          ]
          
          \addplot [line width=1, gray, dashed,domain=(-9:9),<->] ({5},{x});
          \addplot [line width=2, penColor, smooth,samples=200,domain=(-9:4.99),<->] {-2 * ln(-x + 5)};
          

           

  \end{axis}
\end{tikzpicture}
\end{image}















\end{exercise}
\end{document}