\documentclass{ximera}


\graphicspath{
  {./}
  {ximeraTutorial/}
  {basicPhilosophy/}
}

\newcommand{\mooculus}{\textsf{\textbf{MOOC}\textnormal{\textsf{ULUS}}}}

\usepackage{tkz-euclide}\usepackage{tikz}
\usepackage{tikz-cd}
\usetikzlibrary{arrows}
\tikzset{>=stealth,commutative diagrams/.cd,
  arrow style=tikz,diagrams={>=stealth}} %% cool arrow head
\tikzset{shorten <>/.style={ shorten >=#1, shorten <=#1 } } %% allows shorter vectors

\usetikzlibrary{backgrounds} %% for boxes around graphs
\usetikzlibrary{shapes,positioning}  %% Clouds and stars
\usetikzlibrary{matrix} %% for matrix
\usepgfplotslibrary{polar} %% for polar plots
\usepgfplotslibrary{fillbetween} %% to shade area between curves in TikZ
\usetkzobj{all}
\usepackage[makeroom]{cancel} %% for strike outs
%\usepackage{mathtools} %% for pretty underbrace % Breaks Ximera
%\usepackage{multicol}
\usepackage{pgffor} %% required for integral for loops



%% http://tex.stackexchange.com/questions/66490/drawing-a-tikz-arc-specifying-the-center
%% Draws beach ball
\tikzset{pics/carc/.style args={#1:#2:#3}{code={\draw[pic actions] (#1:#3) arc(#1:#2:#3);}}}



\usepackage{array}
\setlength{\extrarowheight}{+.1cm}
\newdimen\digitwidth
\settowidth\digitwidth{9}
\def\divrule#1#2{
\noalign{\moveright#1\digitwidth
\vbox{\hrule width#2\digitwidth}}}






\DeclareMathOperator{\arccot}{arccot}
\DeclareMathOperator{\arcsec}{arcsec}
\DeclareMathOperator{\arccsc}{arccsc}

















%%This is to help with formatting on future title pages.
\newenvironment{sectionOutcomes}{}{}



\author{Lee Wayand}

\begin{document}
\begin{exercise}



Completely analyze $G(t) = -2 | t - 3 | + 6$ with its natural domain.



\textbf{\textcolor{blue!55!black}{Domain:}}  \\

$G$ is an absolute value function, therefore its natural domain is $(-\infty, \infty)$.




\textbf{\textcolor{blue!55!black}{Continuity:}}  \\

$G$ is an absolute value function, therefore it is continuous on its domain and has no discontinuities.


$G$ is an absolute value function, therefore it has no singularities.








\textbf{\textcolor{blue!55!black}{End-Behavior:}}  \\




$\blacktriangleright$ As $t$ becomes big, then the inside becomes big positively or negatively.  Either way, the $| t - 3|$  is big and positive.  Since the leading coefficient is negative, we have

\[
\lim\limits_{t \to \infty} G(t) = -\infty
\]


\[
\lim\limits_{t \to -\infty} G(t) = -\infty
\]





\textbf{\textcolor{blue!55!black}{Behavior:}}  \\

$G$ is an absolute value function, therefore, it is a piecewise function.  The pieces are linear functions.






\[
G(t) = 
\begin{cases}
  -2 ( t - 3 ) + 6 & (-\infty, 5)\\
  -2 (3 - t) + 6 & (5, \infty)
\end{cases}
\]




\[
G(t) = 
\begin{cases}
  -2 t + 12 & (-\infty, 5)\\
  2t  & (5, \infty)
\end{cases}
\]




$-2 t + 12$ is a decreasing linear function.  $G(t)$ decreases on $(-\infty, 5)$.

$2 t$ is an increasing linear function.  $G(t)$ increases on $(5, \infty)$.





\textbf{\textcolor{blue!55!black}{Zeros:}}  \\


We can find the zeros of each piece of the piecewise function.




$-2 t + 12 $ has a zero at $6$.


$-2 t$ has a zero at $0$.





\textbf{\textcolor{blue!55!black}{Extema:}}  \\


From the end-behavior, we know that $G$ has no global or local minimum.


Since $G$ is an absolute value funciton, $G$ has a global and local maximum when the inside is $0$, which is at $3$.


$G$ is a global and local maximum of $G(3) + 6$ at $3$.












This all agrees with the graph.


\begin{image}
\begin{tikzpicture} 
  \begin{axis}[
            domain=-10:10, ymax=10, xmax=10, ymin=-10, xmin=-10,
            axis lines =center, xlabel=$x$, ylabel=$y$, 
            ytick={-10,-8,-6,-4,-2,2,4,6,8,10},
            xtick={-10,-8,-6,-4,-2,2,4,6,8,10},
            ticklabel style={font=\scriptsize},
            every axis y label/.style={at=(current axis.above origin),anchor=south},
            every axis x label/.style={at=(current axis.right of origin),anchor=west},
            axis on top
          ]
          
          %\addplot [line width=1, gray, dashed,domain=(-9:9),<->] ({5},{x});
          \addplot [line width=2, penColor, smooth,samples=200,domain=(-4:9),<->] {-2 * abs(x-3) + 6};
          

           

  \end{axis}
\end{tikzpicture}
\end{image}















\end{exercise}
\end{document}