\documentclass{ximera}


\graphicspath{
  {./}
  {ximeraTutorial/}
  {basicPhilosophy/}
}

\newcommand{\mooculus}{\textsf{\textbf{MOOC}\textnormal{\textsf{ULUS}}}}

\usepackage{tkz-euclide}\usepackage{tikz}
\usepackage{tikz-cd}
\usetikzlibrary{arrows}
\tikzset{>=stealth,commutative diagrams/.cd,
  arrow style=tikz,diagrams={>=stealth}} %% cool arrow head
\tikzset{shorten <>/.style={ shorten >=#1, shorten <=#1 } } %% allows shorter vectors

\usetikzlibrary{backgrounds} %% for boxes around graphs
\usetikzlibrary{shapes,positioning}  %% Clouds and stars
\usetikzlibrary{matrix} %% for matrix
\usepgfplotslibrary{polar} %% for polar plots
\usepgfplotslibrary{fillbetween} %% to shade area between curves in TikZ
\usetkzobj{all}
\usepackage[makeroom]{cancel} %% for strike outs
%\usepackage{mathtools} %% for pretty underbrace % Breaks Ximera
%\usepackage{multicol}
\usepackage{pgffor} %% required for integral for loops



%% http://tex.stackexchange.com/questions/66490/drawing-a-tikz-arc-specifying-the-center
%% Draws beach ball
\tikzset{pics/carc/.style args={#1:#2:#3}{code={\draw[pic actions] (#1:#3) arc(#1:#2:#3);}}}



\usepackage{array}
\setlength{\extrarowheight}{+.1cm}
\newdimen\digitwidth
\settowidth\digitwidth{9}
\def\divrule#1#2{
\noalign{\moveright#1\digitwidth
\vbox{\hrule width#2\digitwidth}}}






\DeclareMathOperator{\arccot}{arccot}
\DeclareMathOperator{\arcsec}{arcsec}
\DeclareMathOperator{\arccsc}{arccsc}

















%%This is to help with formatting on future title pages.
\newenvironment{sectionOutcomes}{}{}


\title{Function Behavior}

\begin{document}

\begin{abstract}
%Stuff can go here later if we want!
\end{abstract}
\maketitle




We have been examining function characteristics near individual numbers. We have noticed that, for the most part, all of the action happens within some interval.  Action includes zeros, singularities, hills andvalleys in the graph, maximums and minimums.  $[-10, 10]$  is a very popular interval to investigate, because of technology.  Of course, functions can have zeros outside this interval.  

Perhaps we should be investigating the interval $[-1000, 1000]$, $[-50000, 50000]$, $[-750000, 750000]$, or $[-1000000, 1000000]$. Each of our elementary functions has a finite interval which holds all of its interesting places.  Even periodic functions have an interval containing a full single wave, which then describes the whole function.

Eventually, outside some interval, the function doesn't do anything.  It settles down into a simple pattern.  We call this eventual pattern the \textbf{end-behavior}.



What are the simple patterns on $(-\infty, a) \cup (a, \infty)$?  Where $a$ is "big enough".











\begin{sectionOutcomes}
In this section, students will 

\begin{itemize}
\item examine endbehavior.
\item use limit notation.
\item examine horizontal asymptotes.
\item examine functon dominance.
\end{itemize}
\end{sectionOutcomes}

\end{document}
