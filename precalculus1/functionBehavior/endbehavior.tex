\documentclass{ximera}


\graphicspath{
  {./}
  {ximeraTutorial/}
  {basicPhilosophy/}
}

\newcommand{\mooculus}{\textsf{\textbf{MOOC}\textnormal{\textsf{ULUS}}}}

\usepackage{tkz-euclide}\usepackage{tikz}
\usepackage{tikz-cd}
\usetikzlibrary{arrows}
\tikzset{>=stealth,commutative diagrams/.cd,
  arrow style=tikz,diagrams={>=stealth}} %% cool arrow head
\tikzset{shorten <>/.style={ shorten >=#1, shorten <=#1 } } %% allows shorter vectors

\usetikzlibrary{backgrounds} %% for boxes around graphs
\usetikzlibrary{shapes,positioning}  %% Clouds and stars
\usetikzlibrary{matrix} %% for matrix
\usepgfplotslibrary{polar} %% for polar plots
\usepgfplotslibrary{fillbetween} %% to shade area between curves in TikZ
\usetkzobj{all}
\usepackage[makeroom]{cancel} %% for strike outs
%\usepackage{mathtools} %% for pretty underbrace % Breaks Ximera
%\usepackage{multicol}
\usepackage{pgffor} %% required for integral for loops



%% http://tex.stackexchange.com/questions/66490/drawing-a-tikz-arc-specifying-the-center
%% Draws beach ball
\tikzset{pics/carc/.style args={#1:#2:#3}{code={\draw[pic actions] (#1:#3) arc(#1:#2:#3);}}}



\usepackage{array}
\setlength{\extrarowheight}{+.1cm}
\newdimen\digitwidth
\settowidth\digitwidth{9}
\def\divrule#1#2{
\noalign{\moveright#1\digitwidth
\vbox{\hrule width#2\digitwidth}}}






\DeclareMathOperator{\arccot}{arccot}
\DeclareMathOperator{\arcsec}{arcsec}
\DeclareMathOperator{\arccsc}{arccsc}

















%%This is to help with formatting on future title pages.
\newenvironment{sectionOutcomes}{}{}


\title{End Behavior}

\begin{document}

\begin{abstract}
approaching infinity
\end{abstract}
\maketitle







Eventually, outside some interval, the function doesn't do anything surprising.  It settles down into a simple pattern.  We call this eventual pattern the \textbf{\textcolor{purple!85!blue}{end-behavior}}.



What are the simple patterns on $(-\infty, a) \cup (a, \infty)$?  Where $a$ is ``big enough''.  What does ``big enough'' mean?




\begin{example} big enough





Compare the graphs of $y = f(x) = \frac{1}{2}(3x-2)(x+5)(x+3)$ and $y=g(x) = \frac{3}{2}x^3$.



\begin{center}
\desmos{hldmst701o}{400}{300}
\end{center}







\begin{center}
\desmos{kfhf8jbk3s}{400}{300}
\end{center}







These two functions are either completely different or almost the same.  It depends on your viewpoint.

Inside $[-10, 10]$, they behave much differently. Their zeros are different.  $f(x)$ has local maximums and minimums.  Its graph has hills and valleys.  $g(x)$ is always increasing.

However, if you change your view point to $(-\infty, -50) \cup (50, \infty)$, then they are almost identical.


We would say $g(x) = \frac{3}{2}x^3$ is the \textbf{end-behavior} of $f(x) = \frac{1}{2}(3x-2)(x+5)(x+3)$.






\end{example}


End-behavior occurs only for very large numbers.   Eventually, the numbers are so large that the major pieces of the function just overshadow everything thing else.  

For polynomials, the major piece is the leading term, consisting of the leading coefficient with the highest power term.





\section{Rational Functions} 

Rational functions are quotients of polynomials.  Therefore, their end-behavior is revealed by the quotient of their leading terms.




graph of $y = g(x) = \frac{x+1}{(x+3)(x-4)}$


\begin{image}
\begin{tikzpicture}
  \begin{axis}[
            domain=-10:10, ymax=10, xmax=10, ymin=-10, xmin=-10,
            axis lines =center, xlabel=$x$, ylabel={$y=g(x)$}, grid = major,
            ytick={-10,-8,-6,-4,-2,2,4,6,8,10},
            xtick={-10,-8,-6,-4,-2,2,4,6,8,10},
            yticklabels={$-10$,$-8$,$-6$,$-4$,$-2$,$2$,$4$,$6$,$8$,$10$}, xticklabels={$-10$,$-8$,$-6$,$-4$,$-2$,$2$,$4$,$6$,$8$,$10$},
            ticklabel style={font=\scriptsize},
            every axis y label/.style={at=(current axis.above origin),anchor=south},
            every axis x label/.style={at=(current axis.right of origin),anchor=west},
            axis on top
          ]
          
          \addplot [line width=1, gray, dashed, domain=(-9.5:9.5),<->] ({-3},{x});
          \addplot [line width=1, gray, dashed, domain=(-9.5:9.5),<->] ({4},{x});

          \addplot [line width=2, penColor, smooth, domain=(-9:-3.1),<->] {(x-1)/((x+3)*(x-4))};
          \addplot [line width=2, penColor, smooth, domain=(-2.9:3.9),<->] {(x-1)/((x+3)*(x-4))};
          \addplot [line width=2, penColor, smooth, domain=(4.1:9),<->] {(x-1)/((x+3)*(x-4))};

          


           

  \end{axis}
\end{tikzpicture}
\end{image}




The end-behavior would come from 


\[ \frac{x+1}{(x+3)(x-4)}    \sim    \frac{x}{x^2} = \frac{1}{x}  \]


This approaches $0$ as $x \to \infty$ or $x \to -\infty$ 


$\blacktriangleright$   For a rational function, if the degree of the denominator is greater than the degree of the numerator, then the end-behavior of a rational function is the constant function $0$ and the horizontal axis is a horizontal asymptote on the graph.




$\blacktriangleright$   For a rational function, if the degrees of the denominator and numerator are equal, then the end-behavior of a rational function is again a constant function.


\[ \frac{(3t+1)(5t-5)}{(t+3)(2t-4)}    \sim    \frac{15t^2}{2t^2} = \frac{15}{2}  \]




The graph of $y = h(t) = \frac{(3t+1)(5t-5)}{(t+3)(2t-4)} $ has a horizontal asymptote with the equation $y = \frac{15}{2}$.




$\blacktriangleright$   If the degree of the numerator and is one more than the degree of the denominator, then the end-behavior is a linear function.


\[ \frac{(w+8)(2w-7)(3w-1)}{(3w-5)(w+2)}    \sim   \frac{6w^3 + 11w^2}{3w^2 + w} \sim 2w +3 \]
















\begin{center}
\desmos{avpix89xzs}{400}{300}
\end{center}







\begin{center}
\desmos{khtwpb7qdx}{400}{300}
\end{center}



However, since we are experts at linear functions, we might want a little more detail.  We might also want the constant term of this linear function.  

We are looking for a linear function, $A \cdot w + B$, such that 










\[ \frac{(w+8)(2w-7)(3w-1)}{(3w-5)(w+2)}    \sim    A \cdot w + B  \]




We are looking for 



\[ (w+8)(2w-7)(3w-1)   \sim   (3w-5)(w+2) \cdot  (A \cdot w + B)  \]



If we multiply these out and compare...


\[
6 w^3 + 25 w^2 - 177 w + 56 = 3A w^2 + (A + 3B) w^2 + \cdots
\]


This tells us that $A = 2$ and $B = \frac{23}{3}$




\begin{center}
\desmos{b23uvbvp8n}{400}{300}
\end{center}

This line is called an \textbf{oblique} asymptote.






















\begin{center}
\textbf{\textcolor{green!50!black}{ooooo=-=-=-=-=-=-=-=-=-=-=-=-=ooOoo=-=-=-=-=-=-=-=-=-=-=-=-=ooooo}} \\

more examples can be found by following this link\\ \link[More Examples of Function Behavior]{https://ximera.osu.edu/csccmathematics/precalculus1/precalculus1/functionBehavior/examples/exampleList}

\end{center}




\end{document}
