\documentclass{ximera}


\graphicspath{
  {./}
  {ximeraTutorial/}
  {basicPhilosophy/}
}

\newcommand{\mooculus}{\textsf{\textbf{MOOC}\textnormal{\textsf{ULUS}}}}

\usepackage{tkz-euclide}\usepackage{tikz}
\usepackage{tikz-cd}
\usetikzlibrary{arrows}
\tikzset{>=stealth,commutative diagrams/.cd,
  arrow style=tikz,diagrams={>=stealth}} %% cool arrow head
\tikzset{shorten <>/.style={ shorten >=#1, shorten <=#1 } } %% allows shorter vectors

\usetikzlibrary{backgrounds} %% for boxes around graphs
\usetikzlibrary{shapes,positioning}  %% Clouds and stars
\usetikzlibrary{matrix} %% for matrix
\usepgfplotslibrary{polar} %% for polar plots
\usepgfplotslibrary{fillbetween} %% to shade area between curves in TikZ
\usetkzobj{all}
\usepackage[makeroom]{cancel} %% for strike outs
%\usepackage{mathtools} %% for pretty underbrace % Breaks Ximera
%\usepackage{multicol}
\usepackage{pgffor} %% required for integral for loops



%% http://tex.stackexchange.com/questions/66490/drawing-a-tikz-arc-specifying-the-center
%% Draws beach ball
\tikzset{pics/carc/.style args={#1:#2:#3}{code={\draw[pic actions] (#1:#3) arc(#1:#2:#3);}}}



\usepackage{array}
\setlength{\extrarowheight}{+.1cm}
\newdimen\digitwidth
\settowidth\digitwidth{9}
\def\divrule#1#2{
\noalign{\moveright#1\digitwidth
\vbox{\hrule width#2\digitwidth}}}






\DeclareMathOperator{\arccot}{arccot}
\DeclareMathOperator{\arcsec}{arcsec}
\DeclareMathOperator{\arccsc}{arccsc}

















%%This is to help with formatting on future title pages.
\newenvironment{sectionOutcomes}{}{}


\title{Limiting Behavior}

\begin{document}

\begin{abstract}
limit notation
\end{abstract}
\maketitle




End-behavior is a simplier description of approximate function values as we move way out in the domain to the very very very large numbers.  Our phrases for this movement in the domain are 

\begin{itemize}
\item tending to infinity
\item tending to negative infinity
\end{itemize}


We also refer to this as \textbf{limiting behavior}. \\


Our shorthand notation for " the limiting behavior of" is $\lim\limits_{x \to \infty}$.  This is placed to the left of the function. \\






We use limit notation to describe end-behavior, when the endbehavior is a constant or unbounded.






$\blacktriangleright$  We have seen limiting or end-behavior of exponential functions.


\[   \lim_{x \to -\infty} 1.5^x   = 0      \]


\begin{image}
\begin{tikzpicture} 
  \begin{axis}[
            domain=-10:10, ymax=10, xmax=10, ymin=-10, xmin=-10,
            axis lines =center, xlabel=$x$, ylabel=$y$,
            ticklabel style={font=\scriptsize},
            every axis y label/.style={at=(current axis.above origin),anchor=south},
            every axis x label/.style={at=(current axis.right of origin),anchor=west},
            axis on top
          ]
          
          \addplot [line width=2, penColor, smooth,samples=200,domain=(-9:5.5),<->] {1.5^x};
          \addplot [line width=1, gray, dashed,domain=(-9:9),<->] ({x},{0});

          %\addplot[color=penColor,only marks,mark=*] coordinates{(-4,0)}; 

           

  \end{axis}
\end{tikzpicture}
\end{image}



as $x \to \infty$, the function values become unbounded and our notation for that looks like 


\[   \lim_{x \to \infty} 1.5^x   = \infty      \]






$\blacktriangleright$  Same with logaarithms.  

\[   \lim_{t \to \infty} log_2(t+5)   = \infty      \]

\begin{image}
\begin{tikzpicture} 
  \begin{axis}[
            domain=-10:10, ymax=10, xmax=10, ymin=-10, xmin=-10,
            axis lines =center, xlabel=$t$, ylabel=$y$,
            ticklabel style={font=\scriptsize},
            every axis y label/.style={at=(current axis.above origin),anchor=south},
            every axis x label/.style={at=(current axis.right of origin),anchor=west},
            axis on top
          ]
          
          \addplot [line width=2, penColor, smooth,samples=200,domain=(-4.97:9),<->] {ln(x+5)/ln(2)};
          \addplot [line width=1, gray, dashed,domain=(-9:9),<->] ({-5},{x});

          \addplot[color=penColor,only marks,mark=*] coordinates{(-4,0)}; 

           

  \end{axis}
\end{tikzpicture}
\end{image}




$\blacktriangleright$ Some rational functions approach a constant, which we see on their graphs as a horizontal asymptote.








graph of $y = f(x) = \frac{2x-4}{x+3}$


\begin{image}
\begin{tikzpicture}
  \begin{axis}[
            domain=-20:20, ymax=20, xmax=20, ymin=-20, xmin=-20,
            axis lines =center, xlabel=$x$, ylabel={$y=f(x)$}, grid = major,
            ytick={-20,-15,-10,-5,5,10,15,20},
            xtick={-20,-15,-10,-5,5,10,15,20},
            yticklabels={$-10$,$-15$,$-10$,$-5$,$5$,$10$,$15$,$20$}, 
            xticklabels={$-10$,$-15$,$-10$,$-5$,$5$,$10$,$15$,$20$},
            ticklabel style={font=\scriptsize},
            every axis y label/.style={at=(current axis.above origin),anchor=south},
            every axis x label/.style={at=(current axis.right of origin),anchor=west},
            axis on top
          ]
          
          \addplot [line width=2, penColor, smooth, samples=300, domain=(-19.9:-3.58),<->] {(2*x-4)/(x+3)};
          \addplot [line width=2, penColor, smooth, samples=300, domain=(-2.53:19.9),<->] {(2*x-4)/(x+3)};


          \addplot [line width=1, gray, dashed, domain=(-19.5:19.5),<->] ({-3},{x});
          \addplot [line width=2, gray, dashed, domain=(-19.9:19.9),<->] {2};



           

  \end{axis}
\end{tikzpicture}
\end{image}





\[   \lim_{x \to -\infty} f(x)   = 2     \]


\[   \lim_{x \to \infty} f(x)   = 2     \]




\begin{example} Limits 



Let $G(m)$ be defined by the following graph. 

\begin{image}
\begin{tikzpicture}
  \begin{axis}[
            domain=-10:10, ymax=10, xmax=10, ymin=-10, xmin=-10,
            axis lines =center, xlabel=$t$, ylabel=$y$, grid = major,
            ytick={-10,-8,-6,-4,-2,2,4,6,8,10},
            xtick={-10,-8,-6,-4,-2,2,4,6,8,10},
            ticklabel style={font=\scriptsize},
            every axis y label/.style={at=(current axis.above origin),anchor=south},
            every axis x label/.style={at=(current axis.right of origin),anchor=west},
            axis on top
          ]
          
          \addplot [line width=2, penColor, smooth, samples=300, domain=(-9.9:-3.58),<->] {(6*x-4)/(3*(x+3))+3};
          \addplot [line width=2, penColor, smooth, samples=300, domain=(-2.4:9.9),<->] {(-4*x-4)/(x+3)};


          \addplot [line width=1, gray, dashed, domain=(-9.5:9.5),<->] ({-3},{x});
          \addplot [line width=2, gray, dashed, domain=(-9.9:-1),<->] {6};
          \addplot [line width=2, gray, dashed, domain=(1:9.9),<->] {-4};



           

  \end{axis}
\end{tikzpicture}
\end{image}





\[   \lim_{x \to -\infty} f(x)   = \answer{6}     \]


\[   \lim_{x \to \infty} f(x)   = \answer{-4}     \]



\end{example}

















\begin{example} Limits 



Let $T(y)$ be defined by 



\[
  T(y) = \frac{2 \sin(3y)}{y} \, \text{ defined on } \, (2, \infty)
\]



\begin{image}
\begin{tikzpicture}
  \begin{axis}[
            domain=0:20, ymax=2, xmax=20, ymin=-2, xmin=0,
            axis lines =center, xlabel=$y$, ylabel=$z$, grid = major,
            ytick={-1,1},
            xtick={2,4,6,8,10,12,14,16,18},
            ticklabel style={font=\scriptsize},
            every axis y label/.style={at=(current axis.above origin),anchor=south},
            every axis x label/.style={at=(current axis.right of origin),anchor=west},
            axis on top
          ]
          
          \addplot [line width=2, penColor, smooth, samples=300, domain=(2:18.5),->] {2*sin(deg(3*x))/x};




           

  \end{axis}
\end{tikzpicture}
\end{image}





\[   \lim_{y \to \infty} T(y)   = \answer{0}     \]






\end{example}


















\begin{example} Limits 



Let $R(t)$, with its natural domain, be defined by 



\[
  R(t) = \frac{(2t+5)(t-3)(3t-7)}{(t-8)(t+1)(2t+9)} \,
\]





\[   \lim_{\answer{t} \to \infty} R(t)   = \answer{3}     \]






\end{example}












\end{document}
