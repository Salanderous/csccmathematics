\documentclass{ximera}


\graphicspath{
  {./}
  {ximeraTutorial/}
  {basicPhilosophy/}
}

\newcommand{\mooculus}{\textsf{\textbf{MOOC}\textnormal{\textsf{ULUS}}}}

\usepackage{tkz-euclide}\usepackage{tikz}
\usepackage{tikz-cd}
\usetikzlibrary{arrows}
\tikzset{>=stealth,commutative diagrams/.cd,
  arrow style=tikz,diagrams={>=stealth}} %% cool arrow head
\tikzset{shorten <>/.style={ shorten >=#1, shorten <=#1 } } %% allows shorter vectors

\usetikzlibrary{backgrounds} %% for boxes around graphs
\usetikzlibrary{shapes,positioning}  %% Clouds and stars
\usetikzlibrary{matrix} %% for matrix
\usepgfplotslibrary{polar} %% for polar plots
\usepgfplotslibrary{fillbetween} %% to shade area between curves in TikZ
\usetkzobj{all}
\usepackage[makeroom]{cancel} %% for strike outs
%\usepackage{mathtools} %% for pretty underbrace % Breaks Ximera
%\usepackage{multicol}
\usepackage{pgffor} %% required for integral for loops



%% http://tex.stackexchange.com/questions/66490/drawing-a-tikz-arc-specifying-the-center
%% Draws beach ball
\tikzset{pics/carc/.style args={#1:#2:#3}{code={\draw[pic actions] (#1:#3) arc(#1:#2:#3);}}}



\usepackage{array}
\setlength{\extrarowheight}{+.1cm}
\newdimen\digitwidth
\settowidth\digitwidth{9}
\def\divrule#1#2{
\noalign{\moveright#1\digitwidth
\vbox{\hrule width#2\digitwidth}}}






\DeclareMathOperator{\arccot}{arccot}
\DeclareMathOperator{\arcsec}{arcsec}
\DeclareMathOperator{\arccsc}{arccsc}

















%%This is to help with formatting on future title pages.
\newenvironment{sectionOutcomes}{}{}


\title{Graph Movement}

\begin{document}

\begin{abstract}
horizontal
\end{abstract}
\maketitle

Composition with a linear function as the inside function results in horizontal transformations.



Let $W(t) = 3 |t-1| - 3$ with domain $[-1, 5)$.





\begin{image}
\begin{tikzpicture}
  \begin{axis}[
            domain=-10:10, ymax=10, xmax=10, ymin=-10, xmin=-10,
            axis lines =center, xlabel=$t$, ylabel={$y=W(t)$}, grid = major, grid style={dashed},
            ytick={-10,-8,-6,-4,-2,2,4,6,8,10},
            xtick={-10,-8,-6,-4,-2,2,4,6,8,10},
            yticklabels={$-10$,$-8$,$-6$,$-4$,$-2$,$2$,$4$,$6$,$8$,$10$}, 
            xticklabels={$-10$,$-8$,$-6$,$-4$,$-2$,$2$,$4$,$6$,$8$,$10$},
            ticklabel style={font=\scriptsize},
            every axis y label/.style={at=(current axis.above origin),anchor=south},
            every axis x label/.style={at=(current axis.right of origin),anchor=west},
            axis on top
          ]
          
            
      		\addplot [line width=2, penColor, smooth,samples=200,domain=(-1:5)] {3*abs(x-1) - 3};

      		\addplot[color=penColor,fill=penColor,only marks,mark=*] coordinates{(-1,3)};
      		\addplot[color=penColor,fill=white,only marks,mark=*] coordinates{(5,9)};






  \end{axis}
\end{tikzpicture}
\end{image}



\textbf{Briefly:} $t$ moves along the real line left to right from $-1$ to $5$. We first encounter an included endpoint and then a short line segment, then a corner at $1$, then a longer line segment with an excluded endpoint. \\







And, now compose it with $L(x) = -\frac{1}{2}x$ \\



$(W \circ L)(x) = W(L(x))$


The implied domain for $L$ is \textbf{$\mathbb{R}$}.  The implied range for $L$ is \textbf{$\mathbb{R}$}. This is too much for the domain of $W$.  The domain of $W$ is only $[1, 5)$. Therefore, we must align those.

But, before doing that, let's just thnk about traversing the real number line. \\

We can imagine $x$ moving along the real line left to right from $-\infty$ to $\infty$. These numbers are input into $L$ and the output runs right to left from $\infty$ to $-\infty$.  The direction is reversed because of the $-\frac{1}{2}$ coefficient in $L$.  It is changing the sign of the numbers.


Therefore, in the composition, the domain of $W$ is running backwards.  We will first encounter the longer line segment with the excluded endpoint, then a corner at $2$, then the shorter line segment with the included endpoint.

The graph is reflected horizontally. \\


That was due to the negative sign of the coefficient.  Now to figure out the effect of $\frac{1}{2}$, which is less than $1$.


The domain of $W$ is $[-1, 5)$.  So, when do the values of $L(x)$ equal $-1$ and $5$.  Since all of the domain numbers of $L$ are multiplied by $-\frac{1}{2}$, these function values of $L$ occur at $2$ and $-10$ in the domain of $L$. \\



The domain of $W \circ L$ is $(-10, 2]$. \\



$x$ will move across the interval $(-10, 2]$, from left to right.  \\

The function values of $L$ will move across the interval $[-1, 5)$ from right to left.  These values are going into $W$ backwards from their normal order.

The graph of $W \circ L$ will first encounter the longer line segment with the excluded endpoint, then it will cross the corner.  The corner occurs at $1$ in the domain of $W$, which is the range of $L$, which occurs at $-2$, in the domain of $L$.  Then the graph of $W \circ L$ encounters the shorter line segment with the included endpoint. \\





Let $(W \circ L)(x) = 3 |-\frac{1}{2}x-1| - 3$ with domain $(-10, 2]$.


\begin{image}
\begin{tikzpicture}
  \begin{axis}[
            domain=-10:10, ymax=10, xmax=10, ymin=-10, xmin=-10,
            axis lines =center, xlabel=$x$, ylabel={$y=(W \circ L)(x)$}, grid = major, grid style={dashed},
            ytick={-10,-8,-6,-4,-2,2,4,6,8,10},
            xtick={-10,-8,-6,-4,-2,2,4,6,8,10},
            yticklabels={$-10$,$-8$,$-6$,$-4$,$-2$,$2$,$4$,$6$,$8$,$10$}, 
            xticklabels={$-10$,$-8$,$-6$,$-4$,$-2$,$2$,$4$,$6$,$8$,$10$},
            ticklabel style={font=\scriptsize},
            every axis y label/.style={at=(current axis.above origin),anchor=south},
            every axis x label/.style={at=(current axis.right of origin),anchor=west},
            axis on top
          ]
          
            
      		\addplot [line width=2, penColor, smooth,samples=200,domain=(-10:2)] {3*abs(-0.5*x-1) - 3};

      		\addplot[color=penColor,fill=penColor,only marks,mark=*] coordinates{(2,3)};
      		\addplot[color=penColor,fill=white,only marks,mark=*] coordinates{(-10,9)};






  \end{axis}
\end{tikzpicture}
\end{image}


The graph has transformed horizontally. \\


\begin{itemize}
\item The negative coefficient in $L$ has reflected the graph horizontally.
\item The $\frac{1}{2}$ coefficient means the domain needs to expand by a factor of $2$, so that the values of $L$ match the normal inputs to $W$.
\item This agrees with setting the inside of the absolute value equal to $0$:  \\
$-\frac{1}{2}x-1 = 0$, when $x = -2$.
\end{itemize}




\end{document}
