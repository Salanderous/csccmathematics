\documentclass{ximera}


\graphicspath{
  {./}
  {ximeraTutorial/}
  {basicPhilosophy/}
}

\newcommand{\mooculus}{\textsf{\textbf{MOOC}\textnormal{\textsf{ULUS}}}}

\usepackage{tkz-euclide}\usepackage{tikz}
\usepackage{tikz-cd}
\usetikzlibrary{arrows}
\tikzset{>=stealth,commutative diagrams/.cd,
  arrow style=tikz,diagrams={>=stealth}} %% cool arrow head
\tikzset{shorten <>/.style={ shorten >=#1, shorten <=#1 } } %% allows shorter vectors

\usetikzlibrary{backgrounds} %% for boxes around graphs
\usetikzlibrary{shapes,positioning}  %% Clouds and stars
\usetikzlibrary{matrix} %% for matrix
\usepgfplotslibrary{polar} %% for polar plots
\usepgfplotslibrary{fillbetween} %% to shade area between curves in TikZ
\usetkzobj{all}
\usepackage[makeroom]{cancel} %% for strike outs
%\usepackage{mathtools} %% for pretty underbrace % Breaks Ximera
%\usepackage{multicol}
\usepackage{pgffor} %% required for integral for loops



%% http://tex.stackexchange.com/questions/66490/drawing-a-tikz-arc-specifying-the-center
%% Draws beach ball
\tikzset{pics/carc/.style args={#1:#2:#3}{code={\draw[pic actions] (#1:#3) arc(#1:#2:#3);}}}



\usepackage{array}
\setlength{\extrarowheight}{+.1cm}
\newdimen\digitwidth
\settowidth\digitwidth{9}
\def\divrule#1#2{
\noalign{\moveright#1\digitwidth
\vbox{\hrule width#2\digitwidth}}}






\DeclareMathOperator{\arccot}{arccot}
\DeclareMathOperator{\arcsec}{arcsec}
\DeclareMathOperator{\arccsc}{arccsc}

















%%This is to help with formatting on future title pages.
\newenvironment{sectionOutcomes}{}{}


\title{Equivalent Enough}

\begin{document}

\begin{abstract}
watching the domain
\end{abstract}
\maketitle





Our main strategy for solving equations is guessing (knowing what the solution looks like) and factoring.




If you are willing to be a bit careful, then we have some additional ideas.

The object is to maintain equality, but change the way things look.



\begin{observation}


You can replace equal things with equal things and get equal things.


If 
\begin{itemize}
\item $a=b$, and
\item $c=d$, and
\item $a=c$
\end{itemize}
then you can conclude that $b=d$.

You can replace $a$ and $c$ with things equal to $a$ and $c$ and the equality is maintained.


\end{observation}





\begin{observation}


If you ``do'' equal things to equal things, then you get equal things


If $a=b$, then $f(a) = f(b)$, where you are ``doing'' $f$ to both.



\end{observation}




\textbf{\textcolor{red!80!black}{Watch Out!}} The last rule doesn't go backwards.




\begin{warning}  \textbf{\textcolor{red!80!black}{Doesn't Reverse}}  



$3^2 = (-3)^2$, but that doesn't mean $3=-3$.


You can do the same thing to different numbers and get an equality.  But, it doesn't go backwards.  Here we squared both numbers and got an equality.  But, that doesn't mean the original numbers are equal.


\end{warning}



Squaring hides signs.


\begin{example} Squaring


Solve $\sqrt{x} = -9$

\begin{explanation}

Square both sides.


$(\sqrt{x})^2 = (-9)^2$

$x = 81$



The problem here is that  $(\sqrt{x})^2 \ne x$.  For instance, $(\sqrt{-3})^2 \ne -3$.


Instead, we need to keep domains and ranges in mind.  The range of $\sqrt{x}$ is $[0, \infty)$. It cannot give the value $-9$.


There is no solution.


\end{explanation}

\end{example}









\begin{example} Logarithms


Solve $\log_2(t-1) - \log_2(t) = 4$


\begin{explanation}


We can solve simple logarithmic equations - like, just one logarithm.  Therefore, we will use a logarithm rule to combine the two logarithms into one.




\begin{align*}
\log_2(t-1) - \log_2(t) & = 4   \\
\log_2 \left( \frac{t-1}{t} \right)  & = 4  \\
\frac{t-1}{t} & = 2^4  \\
 t-1 & = 16t  \\
-1 & = 15t  \\
-\frac{1}{15} & = t
\end{align*}



However, the original equation included the function $\log_2(t)$ and the domain of $\log_2(t)$ does not include $-\frac{1}{15}$


There is no solution.

\end{explanation}

\end{example}


What happened?


The problem in the above example is that the logarithm rule covered up a domain issue.

If we let $t = -\frac{1}{15}$ in the orginal equation, then it would look like 


\[    \log_2\left(-\frac{1}{15}-1\right) - \log_2\left(-\frac{1}{15}\right)   \]

We can see the domain issue here.  However, when we apply the logarithm rule we get



\[   \log_2\left(\frac{-\frac{1}{15}-1}{-\frac{1}{15}}\right)    = \log_2(16)  \]


When the rule combined the insides together into a fraction, the numerator and denominator both became negative, and a negative over a neagtive equals a positive - problem disappears.





\begin{definition}  \textbf{\textcolor{green!50!black}{Extraneous Solutions}}

\textbf{Extraneous solutions} are non-solutions that appear during the solving process, because you didn't keep track of the domain.

\end{definition}

\textbf{Hence:} Always check your solutions in the original equation!







\begin{example} Extraneous Solutions


Solve $\frac{1}{y-2} + \frac{1}{y+2} = \frac{4}{(y-2)(y+2)}$


\begin{explanation}



First combine the fractions on the left, by getting a common denominator.



\[    \frac{1}{y-2} + \frac{1}{y+2}        \]

\[    \frac{1}{y-2} \cdot \frac{y+2}{y+2} + \frac{1}{y+2}  \cdot \frac{y-2}{y-2}       \]

\[    \frac{y+2}{(y-2)(y+2)} + \frac{y-2}{(y-2)(y+2)}      \]

\[    \frac{2y}{(y-2)(y+2)}     \]

Our equivalent equation is


\[    \frac{2y}{(y-2)(y+2)}   = \frac{4}{(y-2)(y+2)}   \]


This can only happen if $2y = 4$ or $y = 2$.

But, this is not possible.  $y=2$ would make a denominator in the orginal equation equal to $0$.  Therefore, $2$ is an extraneous solution and this equation has no solutions.


\end{explanation}
\end{example}



The solving process presents solution \textbf{candidates}.  These must be checked.












\begin{example} Extraneous Solutions

Solve $\sqrt{w+4} = w - 2$


\begin{explanation}

Square both sides:



$(\sqrt{w+4})^2 = (w - 2)^2$


$w + 4 = \answer{w^2 - 4w + 4}$


$0 = \answer{w^2 - 5w}$

$0 = w(w-5) $


Zero product property:  either $w=0$ or $w-5=0$ and $w=5$.



However, $w$ cannot equal $0$.  If we replace $w$ with $0$ in the original equation, then we get $\sqrt{0+4} = 0 - 2 = -2$.

And, $-2$ is not in the range if the square root function. $-2$ is an extraneous solution.

The solution set is $\{ 5 \}$.

\end{explanation}

\end{example}


The problem again is squaring.

In the example above, $0$ doesn't work, because it gives us $\sqrt{4} = -2$, but squaring hid that fact.  


\[ (\sqrt{4})^2 = (-2)^2 \, \text{ gives } \, 4 = 4\]


After you square both sides you get $4 = 4$, which is fine.





When you do the same thing to both sides, watch out!  You might be hiding domain or range issues.  Check your solutions! \\




All of this is due to the fact that the functions we are applying are not \textbf{one-to-one}.  More on that later.















\begin{center}
\textbf{\textcolor{green!50!black}{ooooo=-=-=-=-=-=-=-=-=-=-=-=-=ooOoo=-=-=-=-=-=-=-=-=-=-=-=-=ooooo}} \\

more examples can be found by following this link\\ \link[More Examples of Equivalent Forms]{https://ximera.osu.edu/csccmathematics/precalculus1/precalculus1/equivalentForms/examples/exampleList}

\end{center}





\end{document}
