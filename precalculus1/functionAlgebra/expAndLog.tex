\documentclass{ximera}


\graphicspath{
  {./}
  {ximeraTutorial/}
  {basicPhilosophy/}
}

\newcommand{\mooculus}{\textsf{\textbf{MOOC}\textnormal{\textsf{ULUS}}}}

\usepackage{tkz-euclide}\usepackage{tikz}
\usepackage{tikz-cd}
\usetikzlibrary{arrows}
\tikzset{>=stealth,commutative diagrams/.cd,
  arrow style=tikz,diagrams={>=stealth}} %% cool arrow head
\tikzset{shorten <>/.style={ shorten >=#1, shorten <=#1 } } %% allows shorter vectors

\usetikzlibrary{backgrounds} %% for boxes around graphs
\usetikzlibrary{shapes,positioning}  %% Clouds and stars
\usetikzlibrary{matrix} %% for matrix
\usepgfplotslibrary{polar} %% for polar plots
\usepgfplotslibrary{fillbetween} %% to shade area between curves in TikZ
\usetkzobj{all}
\usepackage[makeroom]{cancel} %% for strike outs
%\usepackage{mathtools} %% for pretty underbrace % Breaks Ximera
%\usepackage{multicol}
\usepackage{pgffor} %% required for integral for loops



%% http://tex.stackexchange.com/questions/66490/drawing-a-tikz-arc-specifying-the-center
%% Draws beach ball
\tikzset{pics/carc/.style args={#1:#2:#3}{code={\draw[pic actions] (#1:#3) arc(#1:#2:#3);}}}



\usepackage{array}
\setlength{\extrarowheight}{+.1cm}
\newdimen\digitwidth
\settowidth\digitwidth{9}
\def\divrule#1#2{
\noalign{\moveright#1\digitwidth
\vbox{\hrule width#2\digitwidth}}}






\DeclareMathOperator{\arccot}{arccot}
\DeclareMathOperator{\arcsec}{arcsec}
\DeclareMathOperator{\arccsc}{arccsc}

















%%This is to help with formatting on future title pages.
\newenvironment{sectionOutcomes}{}{}


\title{Exponential}

\begin{document}

\begin{abstract}
exponential and logarithmic
\end{abstract}
\maketitle



Exponential and logarithmic funcitons are examples of inverse functions.



First, let's notice that both types of functions are one-to-one and their graphs pass the hoirizontal line test.







Graph of $y = E(t) = 0.5 \, e^{-t-3} - 4$.


\begin{image}
\begin{tikzpicture}
  \begin{axis}[
            domain=-10:10, ymax=10, xmax=10, ymin=-10, xmin=-10,
            axis lines =center, xlabel=$t$, ylabel=$y$, grid = major,
            ytick={-10,-8,-6,-4,-2,2,4,6,8,10},
            xtick={-10,-8,-6,-4,-2,2,4,6,8,10},
            yticklabels={$-10$,$-8$,$-6$,$-4$,$-2$,$2$,$4$,$6$,$8$,$10$}, 
            xticklabels={$-10$,$-8$,$-6$,$-4$,$-2$,$2$,$4$,$6$,$8$,$10$},
            ticklabel style={font=\scriptsize},
            every axis y label/.style={at=(current axis.above origin),anchor=south},
            every axis x label/.style={at=(current axis.right of origin),anchor=west},
            axis on top
          ]
          

            \addplot [line width=2, gray, dashed,samples=100,domain=(-9:10),<->] {-4};
            \addplot [line width=2, penColor, smooth,samples=100,domain=(-6.2:9),<->] {0.5* (e^(-x-3)) - 4};
            \addplot[color=penColor,fill=penColor,only marks,mark=*] coordinates{(-3,-3.5)};


  \end{axis}
\end{tikzpicture}
\end{image}



Graph of $z = L(x) = ln(x+5) - 1$.





\begin{image}
\begin{tikzpicture}
  \begin{axis}[
            domain=-10:10, ymax=10, xmax=10, ymin=-10, xmin=-10,
            axis lines =center, xlabel=$x$, ylabel=$z$, grid = major,
            ytick={-10,-8,-6,-4,-2,2,4,6,8,10},
            xtick={-10,-8,-6,-4,-2,2,4,6,8,10},
            yticklabels={$-10$,$-8$,$-6$,$-4$,$-2$,$2$,$4$,$6$,$8$,$10$}, 
            xticklabels={$-10$,$-8$,$-6$,$-4$,$-2$,$2$,$4$,$6$,$8$,$10$},
            ticklabel style={font=\scriptsize},
            every axis y label/.style={at=(current axis.above origin),anchor=south},
            every axis x label/.style={at=(current axis.right of origin),anchor=west},
            axis on top
          ]
          

            \addplot [line width=1, gray, dashed,samples=200,domain=(-10:10),<->] ({-5},{x});
            \addplot [line width=2, penColor, smooth,samples=300,domain=(-4.999:9),<->] {ln(x+5) - 1};
            \addplot[color=penColor,fill=penColor,only marks,mark=*] coordinates{(-4,-1)};


  \end{axis}
\end{tikzpicture}
\end{image}













\begin{example} Inverse

Let $E(t) = 0.5 \, e^{-t-3} - 4$.




Graph of $y = E(t) = 0.5 \, e^{-t-3} - 4$.


\begin{image}
\begin{tikzpicture}
  \begin{axis}[
            domain=-10:10, ymax=10, xmax=10, ymin=-10, xmin=-10,
            axis lines =center, xlabel=$t$, ylabel=$y$, grid = major,
            ytick={-10,-8,-6,-4,-2,2,4,6,8,10},
            xtick={-10,-8,-6,-4,-2,2,4,6,8,10},
            yticklabels={$-10$,$-8$,$-6$,$-4$,$-2$,$2$,$4$,$6$,$8$,$10$}, 
            xticklabels={$-10$,$-8$,$-6$,$-4$,$-2$,$2$,$4$,$6$,$8$,$10$},
            ticklabel style={font=\scriptsize},
            every axis y label/.style={at=(current axis.above origin),anchor=south},
            every axis x label/.style={at=(current axis.right of origin),anchor=west},
            axis on top
          ]
          


            \addplot [line width=2, gray, dashed,samples=100,domain=(-9:10),<->] {-4};
            \addplot [line width=2, penColor, smooth,samples=100,domain=(-6.2:9),<->] {0.5*(e^(-x-3)) - 4};
            \addplot[color=penColor,fill=penColor,only marks,mark=*] coordinates{(-3,-3.5)};


  \end{axis}
\end{tikzpicture}
\end{image}




We have a function called $E(t)$.  Its pairs look like $(a, E(a))$.  The inverse of this function will have pairs of the form $(E(a), a)$. The domain and range just switch roles.


In the formula, $E$ is representing the range and $t$ the domain.  They just switch roled for the inverse. Same equation.  Just a different interpretation.

\[    E = 0.5 \, e^{-t(E)-3} - 4   \]


Now, $t$ is the function name and $E$ is the variable.  We can just solve this for $t(E)$.

\begin{align*}
E           & = 0.5 \, e^{-t(E)-3} - 4 \\
E + 4       & = 0.5 \, e^{-t(E)-3}  \\
2(E + 4)    & =  e^{-t(E)-3}   \\
ln(2(E+4))  & = -t(E)-3   \\
-(ln(2(E+4)))  & = t(E)+3   \\
-(ln(2(E+4))) - 3   & = t(E)
\end{align*}







Graphs of $y = E(t) = 0.5 \, e^{-t-3} - 4$ and $z = L(x) = -ln(2(x+4)) - 3$.


\begin{image}
\begin{tikzpicture}
  \begin{axis}[
            domain=-10:10, ymax=10, xmax=10, ymin=-10, xmin=-10,
            axis lines =center, xlabel={$t,x$}, ylabel={$y,z$}, grid = major,
            ytick={-10,-8,-6,-4,-2,2,4,6,8,10},
            xtick={-10,-8,-6,-4,-2,2,4,6,8,10},
            yticklabels={$-10$,$-8$,$-6$,$-4$,$-2$,$2$,$4$,$6$,$8$,$10$}, 
            xticklabels={$-10$,$-8$,$-6$,$-4$,$-2$,$2$,$4$,$6$,$8$,$10$},
            ticklabel style={font=\scriptsize},
            every axis y label/.style={at=(current axis.above origin),anchor=south},
            every axis x label/.style={at=(current axis.right of origin),anchor=west},
            axis on top
          ]
          
			\addplot [line width=1, gray, dashed,samples=200,domain=(-10:10),<->] ({-4},{x});
			\addplot [line width=1, gray, dashed,samples=200,domain=(-10:10),<->] ({x},{x});

            \addplot [line width=2, gray, dashed,samples=100,domain=(-9:10),<->] {-4};
            \addplot [line width=2, penColor, smooth,samples=100,domain=(-6.2:9),<->] {0.5* (e^(-x-3)) - 4};
            \addplot[color=penColor,fill=penColor,only marks,mark=*] coordinates{(-3,-3.5)};

            \addplot [line width=2, penColor2, smooth,samples=300,domain=(-3.999:9),<->] {-ln(2*(x+4)) - 3};
            \addplot[color=penColor,fill=penColor2,only marks,mark=*] coordinates{(-3.5,-3)};




  \end{axis}
\end{tikzpicture}
\end{image}




Since, these are inverses of each other, the pairs of each function are just reversed and the graphs are mirror images of eaach other over the graph of the idenity function: $\{ (r,r) \, | \, r \in \mathbb{R}   \}$.


\begin{itemize}
\item The domain of $E(t)$ is \textbf{$\mathbb{R}$}. The range of $L(x)$ is \textbf{$\mathbb{R}$}.
\item The range of $E(t)$ is $(-4, \infty)$. The domain of $L(x)$ is $(-4, \infty)$.
\end{itemize}




\begin{itemize}
\item The graph of $E(t)$ has a horizontal asymptote at $-4$.
\item The graph of $L(x)$ has a vertical asymptote at $-4$.
\end{itemize}






\end{example}





















\end{document}
