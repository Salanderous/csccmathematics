\documentclass{ximera}


\graphicspath{
  {./}
  {ximeraTutorial/}
  {basicPhilosophy/}
}

\newcommand{\mooculus}{\textsf{\textbf{MOOC}\textnormal{\textsf{ULUS}}}}

\usepackage{tkz-euclide}\usepackage{tikz}
\usepackage{tikz-cd}
\usetikzlibrary{arrows}
\tikzset{>=stealth,commutative diagrams/.cd,
  arrow style=tikz,diagrams={>=stealth}} %% cool arrow head
\tikzset{shorten <>/.style={ shorten >=#1, shorten <=#1 } } %% allows shorter vectors

\usetikzlibrary{backgrounds} %% for boxes around graphs
\usetikzlibrary{shapes,positioning}  %% Clouds and stars
\usetikzlibrary{matrix} %% for matrix
\usepgfplotslibrary{polar} %% for polar plots
\usepgfplotslibrary{fillbetween} %% to shade area between curves in TikZ
\usetkzobj{all}
\usepackage[makeroom]{cancel} %% for strike outs
%\usepackage{mathtools} %% for pretty underbrace % Breaks Ximera
%\usepackage{multicol}
\usepackage{pgffor} %% required for integral for loops



%% http://tex.stackexchange.com/questions/66490/drawing-a-tikz-arc-specifying-the-center
%% Draws beach ball
\tikzset{pics/carc/.style args={#1:#2:#3}{code={\draw[pic actions] (#1:#3) arc(#1:#2:#3);}}}



\usepackage{array}
\setlength{\extrarowheight}{+.1cm}
\newdimen\digitwidth
\settowidth\digitwidth{9}
\def\divrule#1#2{
\noalign{\moveright#1\digitwidth
\vbox{\hrule width#2\digitwidth}}}






\DeclareMathOperator{\arccot}{arccot}
\DeclareMathOperator{\arcsec}{arcsec}
\DeclareMathOperator{\arccsc}{arccsc}

















%%This is to help with formatting on future title pages.
\newenvironment{sectionOutcomes}{}{}


\title{Function Algebra}

\begin{document}

\begin{abstract}

\end{abstract}
\maketitle
















We have \textbf{operations} for our numbers.  Operations usually take two numbers and exchange them for a third number.

\begin{center}
Addition, Subtraction, Multiplication, Division, etc.
\end{center}



Just like numbers, functions have an arithmetic.  They have all of the usual number operations, plus one more.  Functions also have an operation called \textbf{\textcolor{purple!85!blue}{composition}}.  Composition takes two functions and exchanges them for a third function.

Just like number operations have an identity number, $0$ and $1$, composition has an identity function - the identity function: $Id(x) = x$.


The identity for an operation is an object which appears to be unaffected by the operation.  


\begin{itemize}
\item Adding $0$ with numbers appears to do nothing.
\item Multiplying $1$ with numbers appears to do nothing.
\item Composing $Id$ with functions appears to do nothing.
\end{itemize}




We call this \textbf{symmetry}.



The identity is extremely important for each operation.  The identity can be viewed as the center of the operational structure.  We like to know how move from it and move toward it without operation.  Pairs of items which produce the identity are called \\textbf{\textcolor{purple!85!blue}{inverses}}.



\begin{itemize}
\item If $a + b = 0$, then $a$ and $b$ are called additive inverses.
\item If $a \cdot b = 1$, then $a$ and $b$ are called multiplicative inverses.
\item If $a \circ b = Id$, then $a$ and $b$ are called compositional inverses.
\end{itemize}


Given a function, we would like to know its compositional inverse, which we just call \textbf{the inverse function}.







\subsection{Learning Outcomes}


\begin{sectionOutcomes}
In this section, students will 

\begin{itemize}
\item explore function arithmetic.
\item investigate inverse functions.
\end{itemize}
\end{sectionOutcomes}













\begin{center}
\textbf{\textcolor{green!50!black}{ooooo=-=-=-=-=-=-=-=-=-=-=-=-=ooOoo=-=-=-=-=-=-=-=-=-=-=-=-=ooooo}} \\

more examples can be found by following this link\\ \link[More Examples of Function Algebra]{https://ximera.osu.edu/csccmathematics/precalculus1/precalculus1/functionAlgebra/examples/exampleList}

\end{center}








\end{document}
