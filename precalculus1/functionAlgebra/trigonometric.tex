\documentclass{ximera}


\graphicspath{
  {./}
  {ximeraTutorial/}
  {basicPhilosophy/}
}

\newcommand{\mooculus}{\textsf{\textbf{MOOC}\textnormal{\textsf{ULUS}}}}

\usepackage{tkz-euclide}\usepackage{tikz}
\usepackage{tikz-cd}
\usetikzlibrary{arrows}
\tikzset{>=stealth,commutative diagrams/.cd,
  arrow style=tikz,diagrams={>=stealth}} %% cool arrow head
\tikzset{shorten <>/.style={ shorten >=#1, shorten <=#1 } } %% allows shorter vectors

\usetikzlibrary{backgrounds} %% for boxes around graphs
\usetikzlibrary{shapes,positioning}  %% Clouds and stars
\usetikzlibrary{matrix} %% for matrix
\usepgfplotslibrary{polar} %% for polar plots
\usepgfplotslibrary{fillbetween} %% to shade area between curves in TikZ
\usetkzobj{all}
\usepackage[makeroom]{cancel} %% for strike outs
%\usepackage{mathtools} %% for pretty underbrace % Breaks Ximera
%\usepackage{multicol}
\usepackage{pgffor} %% required for integral for loops



%% http://tex.stackexchange.com/questions/66490/drawing-a-tikz-arc-specifying-the-center
%% Draws beach ball
\tikzset{pics/carc/.style args={#1:#2:#3}{code={\draw[pic actions] (#1:#3) arc(#1:#2:#3);}}}



\usepackage{array}
\setlength{\extrarowheight}{+.1cm}
\newdimen\digitwidth
\settowidth\digitwidth{9}
\def\divrule#1#2{
\noalign{\moveright#1\digitwidth
\vbox{\hrule width#2\digitwidth}}}






\DeclareMathOperator{\arccot}{arccot}
\DeclareMathOperator{\arcsec}{arcsec}
\DeclareMathOperator{\arccsc}{arccsc}

















%%This is to help with formatting on future title pages.
\newenvironment{sectionOutcomes}{}{}


\title{Trigonometric}

\begin{document}

\begin{abstract}
coordinates to angles
\end{abstract}
\maketitle



We have discovered that not all functions have an inverse.  Only one-to-one functions have inverses.  We have also seen that we can salvage pieces and parts of other funcitons, by restricting their domain and range.  This is necessary for trigonometric functions.





\section{Sine}

The Sine function defined as the vertical coordinate of a p[oint on the unit circle at a given angle.  The domain represented angles measureed counterclockwise from the positive horizontal axis. The value of $sin(\theta)$ is the vertical coordinate of the point on the unit circle at the angle $\theta$. Therfore, the range of sine is $[-1, 1]$.


Graph of $y - sin(\theta)$.

\begin{image}
\begin{tikzpicture}
  \begin{axis}[
            domain=-10:10, ymax=1.5, xmax=10, ymin=-1.5, xmin=-10,
            axis lines =center, xlabel={$\theta$}, ylabel=$y$, grid = major, grid style={dashed},
            ytick={-1.5,-1,-0.5,0.5,1,1.5},
            xtick={-7.85, -6.28, -4.71, -3.14, -1.57, 0, 1.57, 3.142, 4.71, 6.28, 7.85},
            xticklabels={$\tfrac{-5\pi}{2}$,$-2\pi$,$\tfrac{-3\pi}{2}$,$-\pi$, $\tfrac{-\pi}{2}$, $0$, $\tfrac{\pi}{2}$, $\pi$, $\tfrac{3\pi}{2}$, $2\pi$, $\tfrac{5\pi}{2}$},
            yticklabels={$1.5$,$-1$,$-0.5$,$0.5$,$1$,$1.5$}, 
            ticklabel style={font=\scriptsize},
            every axis y label/.style={at=(current axis.above origin),anchor=south},
            every axis x label/.style={at=(current axis.right of origin),anchor=west},
            axis on top
          ]
          

            \addplot [line width=2, penColor, smooth,samples=300,domain=(-10:10),<->] {sin(deg(x)};



  \end{axis}
\end{tikzpicture}
\end{image}




The sine function is not one-to-one.  If we want an inverse, then we'll have to restrict the domain.  The most common restriction is $\left[ \frac{\pi}{2}, \frac{\pi}{2} \right]$.




\begin{image}
\begin{tikzpicture}
  \begin{axis}[
            domain=-10:10, ymax=1.5, xmax=10, ymin=-1.5, xmin=-10,
            axis lines =center, xlabel={$\theta$}, ylabel=$y$, grid = major, grid style={dashed},
            ytick={-1.5,-1,-0.5,0.5,1,1.5},
            xtick={-7.85, -6.28, -4.71, -3.14, -1.57, 0, 1.57, 3.142, 4.71, 6.28, 7.85},
            xticklabels={$\tfrac{-5\pi}{2}$,$-2\pi$,$\tfrac{-3\pi}{2}$,$-\pi$, $\tfrac{-\pi}{2}$, $0$, $\tfrac{\pi}{2}$, $\pi$, $\tfrac{3\pi}{2}$, $2\pi$, $\tfrac{5\pi}{2}$},
            yticklabels={$-1.5$,$-1$,$-0.5$,$0.5$,$1$,$1.5$}, 
            ticklabel style={font=\scriptsize},
            every axis y label/.style={at=(current axis.above origin),anchor=south},
            every axis x label/.style={at=(current axis.right of origin),anchor=west},
            axis on top
          ]
          

            \addplot [line width=2, penColor, smooth,samples=300,domain=(-10:10),<->] {sin(deg(x)};

            \addplot [line width=2, penColor2, smooth,samples=300,domain=(-1.57:1.57)] {sin(deg(x)};
            \addplot[color=penColor2,fill=penColor2,only marks,mark=*] coordinates{(-1.57,-1)}; 
            \addplot[color=penColor2,fill=penColor2,only marks,mark=*] coordinates{(1.57,1)};



  \end{axis}
\end{tikzpicture}
\end{image}




With this restriction, the inverse of sine is called \textbf{arcsine}, abbreviated \textbf{arcsin}.  Of course you can use the general inverse notation: $sin^{-1}(x)$.


\begin{itemize}
\item The domain of arcsine is $[-1, 1]$.  
\item The range of arcsine is $\left[ \frac{-\pi}{2}, \frac{\pi}{2} \right]$.
\end{itemize}


The range of arcsine now represents angles, just as the domain of sine does.







\begin{image}
\begin{tikzpicture}
  \begin{axis}[
            domain=-1.6:1.6, ymax=1.6, xmax=1.6, ymin=-1.6, xmin=-1.6,
            axis lines =center, xlabel={$\theta$}, ylabel=$y$, grid = major, grid style={dashed},
            ytick={-1.57, -0.785, 0, 0.785, 1.57},
            xtick={-1.5,-1,-0.5,0.5,1,1.5},
            yticklabels={$\tfrac{-\pi}{2}$, $\tfrac{-\pi}{4}$, $0$, $\tfrac{-\pi}{4}$, $\tfrac{\pi}{2}$},
            xticklabels={$-1.5$,$-1$,$-0.5$,$0.5$,$1$,$1.5$}, 
            ticklabel style={font=\scriptsize},
            every axis y label/.style={at=(current axis.above origin),anchor=south},
            every axis x label/.style={at=(current axis.right of origin),anchor=west},
            axis on top
          ]
          

            \addplot [line width=2, penColor, smooth,samples=300,domain=(-1:1)] {asin(x)*pi/180};
            \addplot[color=penColor,fill=penColor,only marks,mark=*] coordinates{(-1,-1.57)}; 
            \addplot[color=penColor,fill=penColor,only marks,mark=*] coordinates{(1,1.57)};



  \end{axis}
\end{tikzpicture}
\end{image}



















\end{document}
