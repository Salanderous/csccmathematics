\documentclass{ximera}


\graphicspath{
  {./}
  {ximeraTutorial/}
  {basicPhilosophy/}
}

\newcommand{\mooculus}{\textsf{\textbf{MOOC}\textnormal{\textsf{ULUS}}}}

\usepackage{tkz-euclide}\usepackage{tikz}
\usepackage{tikz-cd}
\usetikzlibrary{arrows}
\tikzset{>=stealth,commutative diagrams/.cd,
  arrow style=tikz,diagrams={>=stealth}} %% cool arrow head
\tikzset{shorten <>/.style={ shorten >=#1, shorten <=#1 } } %% allows shorter vectors

\usetikzlibrary{backgrounds} %% for boxes around graphs
\usetikzlibrary{shapes,positioning}  %% Clouds and stars
\usetikzlibrary{matrix} %% for matrix
\usepgfplotslibrary{polar} %% for polar plots
\usepgfplotslibrary{fillbetween} %% to shade area between curves in TikZ
\usetkzobj{all}
\usepackage[makeroom]{cancel} %% for strike outs
%\usepackage{mathtools} %% for pretty underbrace % Breaks Ximera
%\usepackage{multicol}
\usepackage{pgffor} %% required for integral for loops



%% http://tex.stackexchange.com/questions/66490/drawing-a-tikz-arc-specifying-the-center
%% Draws beach ball
\tikzset{pics/carc/.style args={#1:#2:#3}{code={\draw[pic actions] (#1:#3) arc(#1:#2:#3);}}}



\usepackage{array}
\setlength{\extrarowheight}{+.1cm}
\newdimen\digitwidth
\settowidth\digitwidth{9}
\def\divrule#1#2{
\noalign{\moveright#1\digitwidth
\vbox{\hrule width#2\digitwidth}}}






\DeclareMathOperator{\arccot}{arccot}
\DeclareMathOperator{\arcsec}{arcsec}
\DeclareMathOperator{\arccsc}{arccsc}

















%%This is to help with formatting on future title pages.
\newenvironment{sectionOutcomes}{}{}


\title{A New Operation}

\begin{document}

\begin{abstract}
composition
\end{abstract}
\maketitle









Just like numbers, functions have an arithmetic.  They have all of the usual number operations, plus one more.  Functions also have composition as an operation.

Just like number operations have an identiy number, $0$ and $1$, composition has an identiy function - the identity function: $I(x) = x$.


\begin{itemize}
\item Adding $0$ with numbers appears to do nothing.
\item Multiplying $1$ with numbers appears to do nothing.
\item Composing the identity function with functions appears to do nothing.
\end{itemize}







\begin{example} Composition


\begin{itemize}
\item Let $K(u) = \frac{u}{u-1}$ with its implied domain: $(-\infty, 1) \cup (1. \infty)$. \\

\item Let $T(w) = w^2 + 5w + 7$ with its implied domain: \textbf{\mathbb{R}}.



Form the composition $K \circ T$.

\[        (K \circ T)(w) =     \frac{w^2 + 5w + 7}{(w^2 + 5w + 7)-1}   =    \frac{w^2 + 5w + 7}{w^2 + 5w + 6}  \]


$K$ can accept any number, except $1$.  Therefore we need to find out when $T = 1$ and take out the domain numbers where $1$ occurs.



\begin{align*}
T(w) & = 1   \\
w^2 + 5w + 7 & = 1 \\
w^2 + 5w + 6 & = 0   \\
(w+2)(w+3) & = 0
\end{align*}


We need to remove $-2$ and $-3$ from the domain of $T$, because $T(-2)=1$ and $T(-3)=1$ and $1$ cannot be an input into $K$.


The domain for $K \circ T$ is $(-\infty, -3) \cup (-3, -2) \cup (-2, \infty)$




\[        (K \circ T)(w)  =    \frac{w^2 + 5w + 7}{w^2 + 5w + 6}  =    \frac{w^2 + 5w + 7}{(w+2)(w+3)} \]





\end{itemize}




The composition produces a function. \\


This is a good reminder that a formula is not a function.  In the case of a composition, we need to restrict the domain of the inner function to avoid inner function values that are not in the domain of the outer function.










\end{example}
















\end{document}
