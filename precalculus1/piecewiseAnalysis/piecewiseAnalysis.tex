\documentclass{ximera}


\graphicspath{
  {./}
  {ximeraTutorial/}
  {basicPhilosophy/}
}

\newcommand{\mooculus}{\textsf{\textbf{MOOC}\textnormal{\textsf{ULUS}}}}

\usepackage{tkz-euclide}\usepackage{tikz}
\usepackage{tikz-cd}
\usetikzlibrary{arrows}
\tikzset{>=stealth,commutative diagrams/.cd,
  arrow style=tikz,diagrams={>=stealth}} %% cool arrow head
\tikzset{shorten <>/.style={ shorten >=#1, shorten <=#1 } } %% allows shorter vectors

\usetikzlibrary{backgrounds} %% for boxes around graphs
\usetikzlibrary{shapes,positioning}  %% Clouds and stars
\usetikzlibrary{matrix} %% for matrix
\usepgfplotslibrary{polar} %% for polar plots
\usepgfplotslibrary{fillbetween} %% to shade area between curves in TikZ
\usetkzobj{all}
\usepackage[makeroom]{cancel} %% for strike outs
%\usepackage{mathtools} %% for pretty underbrace % Breaks Ximera
%\usepackage{multicol}
\usepackage{pgffor} %% required for integral for loops



%% http://tex.stackexchange.com/questions/66490/drawing-a-tikz-arc-specifying-the-center
%% Draws beach ball
\tikzset{pics/carc/.style args={#1:#2:#3}{code={\draw[pic actions] (#1:#3) arc(#1:#2:#3);}}}



\usepackage{array}
\setlength{\extrarowheight}{+.1cm}
\newdimen\digitwidth
\settowidth\digitwidth{9}
\def\divrule#1#2{
\noalign{\moveright#1\digitwidth
\vbox{\hrule width#2\digitwidth}}}






\DeclareMathOperator{\arccot}{arccot}
\DeclareMathOperator{\arcsec}{arcsec}
\DeclareMathOperator{\arccsc}{arccsc}

















%%This is to help with formatting on future title pages.
\newenvironment{sectionOutcomes}{}{}


\title{Analyzing}

\begin{document}

\begin{abstract}
characteristics and features
\end{abstract}
\maketitle



Piecewise functions are defined by using pieces of other functions. Since we are now familiar with linear and quadratic functions, let's build the piecewise function, $V(h)$, from their pieces.




\[
V(h) = 
\begin{cases}
  -2h-3 & \text{ if } [-6, -2]   \\
  -(h+3)(h-3) & \text{ if } (-2, 4]  \\
  \frac{7h}{4} - 8 & \text{ if } (4,6)
\end{cases}
\]




\begin{idea}

When evaluating $V(d)$, 

\begin{itemize}
  \item if $-6 \leq d \leq -2$, then use the formula $V(h) = -2h - 3$
  \item if $-2 < d \leq 4$, then use the formula $V(h) = -(h+3)(h-3)$
  \item if $4 < d < 6$, then use the formula $V(h) = \frac{7h}{4} - 8$
\end{itemize}

\end{idea}

A graph always helps our thinking. Here is the graph of $y = V(h)$.








\begin{image}
\begin{tikzpicture} 
  \begin{axis}[
            domain=-10:10, ymax=10, xmax=10, ymin=-10, xmin=-10,
            axis lines =center, xlabel=$h$, ylabel=$y$, grid = major,
            ytick={-10,-8,-6,-4,-2,2,4,6,8,10},
            xtick={-10,-8,-6,-4,-2,2,4,6,8,10},
            ticklabel style={font=\scriptsize},
            every axis y label/.style={at=(current axis.above origin),anchor=south},
            every axis x label/.style={at=(current axis.right of origin),anchor=west},
            axis on top
          ]
          
			\addplot [line width=2, penColor, smooth,samples=100,domain=(-6:-2)] {-2*x-3};
       		\addplot [line width=2, penColor, smooth,samples=100,domain=(-2:4)] {-1*(x+3)*(x-3))};
       		\addplot [line width=2, penColor, smooth,samples=100,domain=(4:6)] {1.75*x-8};




			\addplot[color=penColor,fill=penColor,only marks,mark=*] coordinates{(-6,9)};
			\addplot[color=penColor,fill=penColor,only marks,mark=*] coordinates{(-2,1)};

			\addplot[color=penColor,fill=white,only marks,mark=*] coordinates{(-2,5)};
			\addplot[color=penColor,fill=penColor,only marks,mark=*] coordinates{(4,-7)};

			\addplot[color=penColor,fill=white,only marks,mark=*] coordinates{(4,-1)};
			\addplot[color=penColor,fill=white,only marks,mark=*] coordinates{(6,2.5)};


           

  \end{axis}
\end{tikzpicture}
\end{image}




\subsection{Domain} 

From the definition of $V(h)$, we can see the domain is $[-6, -2] \cup (-2, 4] \cup (4,6) = [-6, 6)$.



\subsection{Zeros} 

We can see from the graph that usually the component pieces of $V$ would total four intercepts and zeros. One each from the linear pieces and two from the quadratic piece.  However, the domain restrictions only admit two of them.  The greater zero from the quadratic piece and the zero from the right linear function. These are both represented by intercepts on the graph.

The left line would have an intercept, except the domain does not allow it.  The middle parabola has a left intercept, except the domain does allow it. 

From the graph, we might estimate our two zeros to be around $3$ and $4.5$.  What are the exact values?







$\blacktriangleright$ The middle piece. 


\begin{explanation}


We are solving $-(h+3)(h-3) = 0$.  Since this is in factored form, we can quickly see the candidates are $-3$ and $3$.  $\answer{-3}$ is not in the domain for this piece of $V$, which is $(-2, 4]$.

Therefore, $V$ has $\answer{3}$ as its only zero contributed from the quadratic piece.
\end{explanation}



$\blacktriangleright$ The right piece. 


\begin{explanation}


We are solving $\frac{7h}{4} - 8 = 0$.  This give $\frac{32}{7}$ as a zero of $V$, which agrees with our extimation from the graph.

\end{explanation}







\subsection{Increasing and Decreasing} 



\begin{itemize}
\item \textbf{$-2h-3$} is a linear function.  Its constant rate of change is $-2$.  This tells us it is a decreasing function.  $V$ is decreasing on $[-6, -2]$.

\item \textbf{$-(h+3)(h-3)$} is a quadratic on $(-2, 4]$. Here, $iRoC_V(h) = -2h$. The critical number occurs when $iRoC_V(h) = -2h = 0$, which happens when $h=0$. This agrees with the midpoint of the zeros, which are $-3$ and $3$.  Therefore, $0$ is where the function switches behavior.  

\begin{itemize}
\item On $(-2, 0)$, $iRoC_V(h) = -2h > 0$, therefore, $V$ is increasing.  
\item On $(0,4)$, $iRoC_V(h) = -2h < 0$, therefore, $V$ is decreasing.
\end{itemize}

This agrees with the graph.

\end{itemize}


\textbf{Technically:} We also need to account for the endpoints. $V$ is increasing on $[-2, 4]$, since the dot at $-2$ is lower than the left side of the included parabola.


\begin{question}
\begin{itemize}
\item \textbf{$\frac{7h}{4} - 8$} is a linear function with a constant rate of change of $\answer{\frac{7}{4}}$ telling us that $V$ is \wordChoice{\choice[correct]{increasing} \choice{decreasing}} on $(4, 6)$.
\end{itemize}
\end{question}

\textbf{Technically:} $V$ is increasing on $[4, 6)$, since the dot at $4$ is lower than the left side of the included line segment.


The intervals where $V$ is monotonic are

\begin{itemize}
\item $V$ is decreasing on $[-6,-2]$.
\item $V$ is increasing on $[-2,0]$.
\item $V$ is decreasing on $[0,4]$.
\item $V$ is increasing on $[4,6)$.
\end{itemize}









\subsection{Extreme Values} 


We have three pieces to our function and the graph will help our algebra. We'll gather information about the pieces indiviudally and then combine them for $V$.

\begin{itemize}
  \item \textbf{$-2h-3$} is a decreasing linear function on $[-6, -2]$.  Its maximum would occur at the left endpoint, which is included. Therefore, the maximum is $-2(-6)-3 = 9$. Its minimum would be at the right endpoint, which is included. Therefore, the minimum is $-2(-2)-3 = 1$.

  \item \textbf{$-(h+3)(h-3)$} is a quadratic on $(-2, 4]$.  Its graph is a parabola opening down and we have seen that $V$ increases and then decreases.  

  Therefore, its maximum value corresponds to the vertex, $(0, 9)$, which is included.  We get a maximum value of $9$.  The candidates for minimum value would occur at the endpoints, except only the right endpoint is included.  $-(4-3)(4+3) = -7$ is a possible minimum value.


\begin{observation}


We could have expanded the factored form: $-(h+3)(h-3) = -h^2 + 9$.  This is now in vertex form, $-(h-0)^2 + 9$. From this expression we can see that the vertex is $(0, 9)$, and the negative leading coefficient tells us that we have a maximum of $9$ occuring at $0$.


We could have observed the expanded form as a standard form, $-h^2 + 0 h + 9$.  Then the quadratic formula would give us $\frac{-b}{2a} = \frac{-0}{-2} = 0$ as the location of the vertex and the place where the behavior changes.


And, we can look at the instantaneous rate of change. The middle piece of $V$ is a quadratic, therefore, the vertex form $-(h-0)^2 + 9$ gives us $iRoC(h) = -2(h-0) = -2h$.  This equals $0$ at $0$, which means $0$ is a critical number and a possible location of changing behavior.

\end{observation}
Lastly, \\


  \item \textbf{$\frac{7h}{4} - 8$} is an increasing linear function on $[4, 6)$.  Its maximum would be encoded in the right endpoint, which is excluded. Its minimum would be encoded in the left endpoint, which is also excluded. This piece is contributing no extreme values to $V$.
\end{itemize}


Our analysis of the pieces leaves us with a global maximum value of $9$ for $V$.  $9$ occurs twice.  Once at $-6$ and then again at $0$.

Our analysis of the pieces leaves us with a global minimum value of $-7$ for $V$. $-7$ occurs once at $4$.



Each of these global extrema are also local extrema.  In addition we have $1$, which is a local minimum occuring at $-2$.  

\begin{explanation} Local Minimum


Consider the open interval $(2 - 0.1, 2 + 0.1)$. $V$ is a decreasing function on $(2 - 0.1, 2)$. $V$ is an increasing function on $(2, 2 + 0.1)$.  This tells us that $V(2)$ is a minimum value on this neighborhood of $2$.  Thus, $V(2)=1$ is a local minimum of $V$.





\end{explanation}









\subsection{Critical Numbers} 


$\blacktriangleright$ We know that the instantaneous rate of change of the middle quadratic is $iRoC_V(h) = -2h$ 

$\blacktriangleright$ We know that the instantaneous rate of change of the left linear is $iRoC_V(h) = -2$

$\blacktriangleright$ We know that the instantaneous rate of change of the right linear is $iRoC_V(h) = \frac{7}{4}$



The only place where the $iRoC_V(h) = 0$ is at $0$, therefore, $0$ is a critical number.


We also include domain numbers where the $iRoC_V$ does not exist.  That would be domain numbers where there is no corresponding tangent line.  That would include $-2$ and $4$.



The set of critical numbers is $\{ -2, 0, 4 \}$.




This doesn't include all of the places where we were investigating for extreme values. We also looked at $-6$.  There is a tangent line at $(-6, 9)$, which means $-6$ is not a critical number. Instead, $-6$ is an endpoint of a maximal interval of the domain. That automatically makes $-6$ a candidate for the location of an extreme value of $V$.




Critical numbers and endpoints: $\{ -6, -2, 0, 4 \}$. \\

These are exactly the places we have been investigating for possible maximums and minimums.






\subsection{Range} 


Since each piece of $V$ is continuous, the ranges of the three pieces of $V$ are $[1, 9]$, $[-7, 9]$, and $\left( -1, \frac{2}{5} \right)$. \\

The range of $V$ is their union.

\[
[1, 9] \cup [-7, 9] \cup \left( -1, \frac{2}{5} \right) = [-7, 9]
\]






\subsection{Continuity} 


Linear and quadratic functions are continuous functions.  $V(h)$ is continuous on $[-6, -2]$, $(-2, 4]$, and $(4, 6)$.  These correspond to the intervals in the definition of $V(h)$, because there were no isolated points. \\

Therefore, we only need to check the endpoints of maximal intervals.  Here, those would be $\{ -6, -2, 4, 6 \}$.



\begin{observation} 
From the graph, we can see that 

\begin{itemize}
\item $-6$ is an included endpoint of a maximal interval.  
\item there is a jump discontinuity at $-2$.
\item there is a jump discontinuity at $4$.
\item $6$ is not in the domain.
\end{itemize}



\end{observation}


This will help our algebraic explanations. \\









\begin{explanation} Discontinuity at $-2$


Let $\epsilon > 0$ be a small positive number.  Consider the corresponding interval, $(-2 - \epsilon, -2 + \epsilon)$.

\begin{itemize}
\item On $(-2 - \epsilon, -2)$, the values of $V$ are greater than $1$, since $V$ is decreasing here.
\item On $(-2, -2 + \epsilon)$, the values of $V$ are greater than $2$.  Let's verify this.



We have $V(h) = 9 - h^2$. \\

For $V < 2$, we would need $h < -\sqrt{7}$ or $h > \sqrt{7}$.  Neither of these is true, since $h \in (-2 - \epsilon, -2 + \epsilon)$


\begin{idea}

How come $-\sqrt{7} \notin (-2 - \epsilon, -2 + \epsilon)$? \\


$\sqrt{7} \approx 2.64575$, which is only $0.64575$ away from $2$. That seems pretty small. \\


\textbf{\textcolor{red!80!black}{Nope!}} \\


$\epsilon$ is a veeeeeeeeeeery smaaaaaaaaaaaaaaall postive number.  $0.64575$ is HUGE compared to $\epsilon$. \\


This part of our new way of thinking.  We are trying to get away from specific numbers and, instead, think of \textbf{types} of numbers.  Our first encounter with this type of thinking is $\epsilon$.  It is representing a very small type of number, much smaller than the usual small numbers we have encountered in the past.  But, it is not any particular small number.  It is representing all of them. Kind of weird.







\end{idea}


\end{itemize}



Following the definition of discontinuity, let's pick a distance of $1$.


No matter how small $\epsilon$ is, $(-2 - \epsilon, -2 + \epsilon)$ \textbf{\textcolor{red!90!darkgray}{ALWAYS}} contains a domain number, $a \in (-2, -2 + \epsilon)$, such that the distance between $V(a)$ and $V(-2)$ is greater than $1$.

This tells us that there is a discontinuity at $-2$.

\end{explanation}





\begin{explanation} Discontinuity at $4$


Let $\epsilon > 0$ be a small positive number.  Consider the corresponding interval, $(4 - \epsilon, 4 + \epsilon)$.

\begin{itemize}
\item On $(4 - \epsilon, 4)$, the values of $V$ are less than $-6$.
\item On $(4, 4 + \epsilon)$, the values of $V$ are greater than $-2$.
\end{itemize}


Following the definition of discontinuity, let's pick a distance of $1$.


No matter how small $\epsilon$ is, $(4 - \epsilon, 4 + \epsilon)$ \textbf{\textcolor{red!90!darkgray}{ALWAYS}} contains (at least) one domain number, $a \in (4, 4 + \epsilon)$, such that the distance between $V(a)$ and $V(4)$ is greater than $1$.

This tells us that there is a discontinuity at $4$.

\end{explanation}




\begin{notation}


Graphs are inherently inaccurate.  We cannot be exact with them.  They hide stuff all of the time. We need better ways, algebraic ways, of talking about types of discontinuities: ``jump'', ``removeable'', and ``asymptotic''.

Our algebraic notation to talk about function behavior around discontinuities and singularities will be called \textbf{\textcolor{purple!85!blue}{limits}}.  Limits have already made an appearance helping us describe end-behavior, algebraically.
\end{notation}





















\begin{center}
\textbf{\textcolor{green!50!black}{ooooo=-=-=-=-=-=-=-=-=-=-=-=-=ooOoo=-=-=-=-=-=-=-=-=-=-=-=-=ooooo}} \\

more examples can be found by following this link\\ \link[More Examples of Piecewise-Defined Functions]{https://ximera.osu.edu/csccmathematics/precalculus1/precalculus1/piecewiseAnalysis/examples/exampleList}

\end{center}





\end{document}
