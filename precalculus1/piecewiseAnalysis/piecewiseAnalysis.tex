\documentclass{ximera}


\graphicspath{
  {./}
  {ximeraTutorial/}
  {basicPhilosophy/}
}

\newcommand{\mooculus}{\textsf{\textbf{MOOC}\textnormal{\textsf{ULUS}}}}

\usepackage{tkz-euclide}\usepackage{tikz}
\usepackage{tikz-cd}
\usetikzlibrary{arrows}
\tikzset{>=stealth,commutative diagrams/.cd,
  arrow style=tikz,diagrams={>=stealth}} %% cool arrow head
\tikzset{shorten <>/.style={ shorten >=#1, shorten <=#1 } } %% allows shorter vectors

\usetikzlibrary{backgrounds} %% for boxes around graphs
\usetikzlibrary{shapes,positioning}  %% Clouds and stars
\usetikzlibrary{matrix} %% for matrix
\usepgfplotslibrary{polar} %% for polar plots
\usepgfplotslibrary{fillbetween} %% to shade area between curves in TikZ
\usetkzobj{all}
\usepackage[makeroom]{cancel} %% for strike outs
%\usepackage{mathtools} %% for pretty underbrace % Breaks Ximera
%\usepackage{multicol}
\usepackage{pgffor} %% required for integral for loops



%% http://tex.stackexchange.com/questions/66490/drawing-a-tikz-arc-specifying-the-center
%% Draws beach ball
\tikzset{pics/carc/.style args={#1:#2:#3}{code={\draw[pic actions] (#1:#3) arc(#1:#2:#3);}}}



\usepackage{array}
\setlength{\extrarowheight}{+.1cm}
\newdimen\digitwidth
\settowidth\digitwidth{9}
\def\divrule#1#2{
\noalign{\moveright#1\digitwidth
\vbox{\hrule width#2\digitwidth}}}






\DeclareMathOperator{\arccot}{arccot}
\DeclareMathOperator{\arcsec}{arcsec}
\DeclareMathOperator{\arccsc}{arccsc}

















%%This is to help with formatting on future title pages.
\newenvironment{sectionOutcomes}{}{}


\title{Analyzing}

\begin{document}

\begin{abstract}
characteristics and features
\end{abstract}
\maketitle



Piecewise functions are defined by using pieces of other functions. Since we are now familiar with linear and quadratic functions, let's build some piecewise functions from their pieces.




\[
V(h) = 
\begin{cases}
  -2h-3 & \text{ if } [-6, -2]   \\
  -(h+3)(h-3) & \text{ if } (-2, 4]  \\
  \frac{7h}{4} - 8 & \text{ if } (4,6)
\end{cases}
\]



A graph always helps our thinking. Here is the graph of $y = V(h)$.








\begin{image}
\begin{tikzpicture} 
  \begin{axis}[
            domain=-10:10, ymax=10, xmax=10, ymin=-10, xmin=-10,
            axis lines =center, xlabel=$h$, ylabel=$y$, grid = major,
            ytick={-10,-8,-6,-4,-2,2,4,6,8,10},
            xtick={-10,-8,-6,-4,-2,2,4,6,8,10},
            ticklabel style={font=\scriptsize},
            every axis y label/.style={at=(current axis.above origin),anchor=south},
            every axis x label/.style={at=(current axis.right of origin),anchor=west},
            axis on top
          ]
          
			\addplot [line width=2, penColor, smooth,samples=100,domain=(-6:-2)] {-2*x-3};
       		\addplot [line width=2, penColor, smooth,samples=100,domain=(-2:4)] {-1*(x+3)*(x-3))};
       		\addplot [line width=2, penColor, smooth,samples=100,domain=(4:8)] {1.75*x-8};




			\addplot[color=penColor,fill=penColor,only marks,mark=*] coordinates{(-6,9)};
			\addplot[color=penColor,fill=penColor,only marks,mark=*] coordinates{(-2,1)};

			\addplot[color=penColor,fill=white,only marks,mark=*] coordinates{(-2,5)};
			\addplot[color=penColor,fill=penColor,only marks,mark=*] coordinates{(4,-7)};

			\addplot[color=penColor,fill=white,only marks,mark=*] coordinates{(4,-1)};
			\addplot[color=penColor,fill=white,only marks,mark=*] coordinates{(8,6)};


           

  \end{axis}
\end{tikzpicture}
\end{image}




\subsection{Domain} 

From the definition of $V(h)$, we can see the domain is $[-6, 6)$.



\subsection{Zeros} 

We can see from the graph that $V$ has a single zero and it comes from the middle piece. 

We are solving $-(h+3)(h-3) = 0$.  Since this is in factored for, we can quickly see the cadidates are $-3$ and $3$.  $-3$ is not in the domain for this piece of $V$.

Therefore, $V$ has $3$ as its only zero.



\subsection{Extreme Values} 


We have three pieces to our graph.

\begin{itemize}
  \item \textbf{$-2h-3$} is a decreasing linear function on $[-6, -2]$.  Its maximum would be at the left endpoint, which is included. Therefore, the maximum is $-2(-6)-3 = 9$. Its minimum would be at the right endpoint, which is included. Therefore, the maximum is $-2(-2)-3 = 1$.

  \item \textbf{$-(h+3)(h-3)$} is a quadratic on $(-2, 4]$.  Its graph is a parabola opening down.  Therefore, its maximum value corresponds to the vertex, $(0, 9)$, which is included.  We get a maximum value of $9$.  The candidates for minimum value would occur at the endpoints, except only the right endpoint is included.  $-(4-3)(4+3) = -7$ is a possible minimum value.

  \item \textbf{$\frac{7h}{4} - 8$} is an increasing linear function on $(4, 6)$.  Its maximum would be at the left endpoint, which is excluded. Its maximum would be at the right endpoint, which is excluded. This piece is contributing no extreme values to $V$.
\end{itemize}


Our analysis of the pieces leaves us with a global maximum value of $9$ for $V$.  $9$ occurs twice.  Once at $-6$ and then again at $0$.

Our analysis of the pieces leaves us with a global minimum value of $-7$ for $V$. $-7$ occurs once.  Once at $-6$ and then again at $0$.



Each of these global extrema are also local extrema.  In addition we have $1$, which is a local minimum occuring at $-2$.  



\subsection{Range} 

From our analysis of extrema and the graph, we deduce that the range is $[-7, -9]$.





\subsection{Continuity} 


Linear and quadratic functions are continuous functions.  Therefore, we only need to check the endpoints of maximal intervals.  Here, those would be $\{ -6, -2, 4, 6 \}$.

From the graph, we can see that 

\begin{itemize}
\item $-6$ is an included endpoint of a maximal interval.  
\item there is a jump discontinuity at $-2$.
\item there is a jump discontinuity at $4$.
\item $6$ is not in the domain.
\end{itemize}


$V(h)$ is continuous on $[-6, -2]$, $(-2, 4]$, and $(4, 6)$.  These correspond to the intervals in the definition of $V(h)$, because there were not isolated points.













\subsection{Monotonic Intervals} 



\begin{itemize}
\item $V$ is decreasing on $[-8,-2]$.
\item $V$ is increasing on $[-2,0]$.
\item $V$ is decreasing on $[0,4]$.
\item $V$ is increasing on $[4,6)$.
\end{itemize}




















\end{document}
