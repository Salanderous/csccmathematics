\documentclass{ximera}


\graphicspath{
  {./}
  {ximeraTutorial/}
  {basicPhilosophy/}
}

\newcommand{\mooculus}{\textsf{\textbf{MOOC}\textnormal{\textsf{ULUS}}}}

\usepackage{tkz-euclide}\usepackage{tikz}
\usepackage{tikz-cd}
\usetikzlibrary{arrows}
\tikzset{>=stealth,commutative diagrams/.cd,
  arrow style=tikz,diagrams={>=stealth}} %% cool arrow head
\tikzset{shorten <>/.style={ shorten >=#1, shorten <=#1 } } %% allows shorter vectors

\usetikzlibrary{backgrounds} %% for boxes around graphs
\usetikzlibrary{shapes,positioning}  %% Clouds and stars
\usetikzlibrary{matrix} %% for matrix
\usepgfplotslibrary{polar} %% for polar plots
\usepgfplotslibrary{fillbetween} %% to shade area between curves in TikZ
\usetkzobj{all}
\usepackage[makeroom]{cancel} %% for strike outs
%\usepackage{mathtools} %% for pretty underbrace % Breaks Ximera
%\usepackage{multicol}
\usepackage{pgffor} %% required for integral for loops



%% http://tex.stackexchange.com/questions/66490/drawing-a-tikz-arc-specifying-the-center
%% Draws beach ball
\tikzset{pics/carc/.style args={#1:#2:#3}{code={\draw[pic actions] (#1:#3) arc(#1:#2:#3);}}}



\usepackage{array}
\setlength{\extrarowheight}{+.1cm}
\newdimen\digitwidth
\settowidth\digitwidth{9}
\def\divrule#1#2{
\noalign{\moveright#1\digitwidth
\vbox{\hrule width#2\digitwidth}}}






\DeclareMathOperator{\arccot}{arccot}
\DeclareMathOperator{\arcsec}{arcsec}
\DeclareMathOperator{\arccsc}{arccsc}

















%%This is to help with formatting on future title pages.
\newenvironment{sectionOutcomes}{}{}


\title{Piecewise Intersection}

\begin{document}

\begin{abstract}
equations
\end{abstract}
\maketitle





Below are two piecewise defined functions and their graphs: $V(h)$ and $K(m)$. 







\[
V(h) = 
\begin{cases}
  -2h-3 & \text{ if } [-6, -2]   \\
  -(h+3)(h-3) & \text{ if } (-2, 4]  \\
  \frac{7h}{4} - 8 & \text{ if } (4,8)
\end{cases}
\]







\begin{image}
\begin{tikzpicture}
  \begin{axis}[
            domain=-10:10, ymax=10, xmax=10, ymin=-10, xmin=-10,
            axis lines =center, xlabel=$h$, ylabel={$y=V(h)$}, grid = major,
            ytick={-10,-8,-6,-4,-2,2,4,6,8,10},
            xtick={-10,-8,-6,-4,-2,2,4,6,8,10},
            ticklabel style={font=\scriptsize},
            every axis y label/.style={at=(current axis.above origin),anchor=south},
            every axis x label/.style={at=(current axis.right of origin),anchor=west},
            axis on top
          ]
          
			\addplot [line width=2, penColor, smooth,samples=100,domain=(-6:-2)] {-2*x-3};
       		\addplot [line width=2, penColor, smooth,samples=100,domain=(-2:4)] {-1*(x+3)*(x-3))};
       		\addplot [line width=2, penColor, smooth,samples=100,domain=(4:8)] {1.75*x-8};




			\addplot[color=penColor,fill=penColor,only marks,mark=*] coordinates{(-6,9)};
			\addplot[color=penColor,fill=penColor,only marks,mark=*] coordinates{(-2,1)};

			\addplot[color=penColor,fill=white,only marks,mark=*] coordinates{(-2,5)};
			\addplot[color=penColor,fill=penColor,only marks,mark=*] coordinates{(4,-7)};

			\addplot[color=penColor,fill=white,only marks,mark=*] coordinates{(4,-1)};
			\addplot[color=penColor,fill=white,only marks,mark=*] coordinates{(8,6)};


           

  \end{axis}
\end{tikzpicture}
\end{image}




\[
K(m) = 
\begin{cases}
  m+4 & \text{ if } (-7, 0]   \\
  m^2-7m+9 & \text{ if } (1,6]
\end{cases}
\]



\begin{image}
\begin{tikzpicture} 
  \begin{axis}[
            domain=-10:10, ymax=10, xmax=10, ymin=-10, xmin=-10,
            axis lines =center, xlabel=$m$, ylabel={$y=K(m)$}, grid = major,
            ytick={-10,-8,-6,-4,-2,2,4,6,8,10},
        	xtick={-10,-8,-6,-4,-2,2,4,6,8,10},
            every axis y label/.style={at=(current axis.above origin),anchor=south},
            every axis x label/.style={at=(current axis.right of origin),anchor=west},
            axis on top
          ]
          
			\addplot [line width=2, penColor, smooth,samples=100,domain=(-7:0)] {x+4};
       		\addplot [line width=2, penColor, smooth,samples=100,domain=(1:6)] {(x-1)*(x-6)+3};



			\addplot[color=penColor,fill=white,only marks,mark=*] coordinates{(-7,-3)};
			\addplot[color=penColor,fill=penColor,only marks,mark=*] coordinates{(0,4)};

			\addplot[color=penColor,fill=white,only marks,mark=*] coordinates{(1,3)};
			\addplot[color=penColor,fill=penColor,only marks,mark=*] coordinates{(6,3)};

       

  \end{axis}
\end{tikzpicture}
\end{image}



Find the points of intersection.


From the graphs there appear to be three points of intersection.

\section{First Point}




\begin{image}
\begin{tikzpicture} 
  \begin{axis}[
            domain=-10:10, ymax=10, xmax=10, ymin=-10, xmin=-10,
            axis lines =center, xlabel=$x$, ylabel=$y$, grid = major,
            ytick={-10,-8,-6,-4,-2,2,4,6,8,10},
        	xtick={-10,-8,-6,-4,-2,2,4,6,8,10},
            every axis y label/.style={at=(current axis.above origin),anchor=south},
            every axis x label/.style={at=(current axis.right of origin),anchor=west},
            axis on top
          ]
          
			\addplot [line width=2, penColor, smooth,samples=100,domain=(-7:0)] {x+4};
       		\addplot [line width=2, penColor, smooth,samples=100,domain=(-6:-2)] {-2*x-3};



			\addplot[color=penColor,fill=white,only marks,mark=*] coordinates{(-7,-3)};
			\addplot[color=penColor,fill=penColor,only marks,mark=*] coordinates{(0,4)};
			\addplot[color=penColor,fill=penColor,only marks,mark=*] coordinates{(-6,9)};
			\addplot[color=penColor,fill=penColor,only marks,mark=*] coordinates{(-2,1)};

       

  \end{axis}
\end{tikzpicture}
\end{image}



\[ -2x-3 =   \answer{x+4}       \]


\[ \answer{-3x} =  7      \]

\[ x =  \frac{7}{-3}   =  -\frac{7}{3}      \]


Check....
\begin{itemize}
\item $V(-\tfrac{7}{3}) = -2 \cdot -\tfrac{7}{3} - 3 = \tfrac{14}{3} - \tfrac{9}{3} = \tfrac{5}{3}$
\item $K(-\tfrac{7}{3}) =   -\tfrac{7}{3} + 4 =  -\tfrac{7}{3} + \frac{12}{3} = \tfrac{5}{3} $
\end{itemize}


The intersection point is $\left( -\frac{7}{3}, \frac{5}{3} \right)$.








\section{Second Point}




\begin{image}
\begin{tikzpicture}
  \begin{axis}[
            domain=-10:10, ymax=10, xmax=10, ymin=-10, xmin=-10,
            axis lines =center, xlabel=$t$, ylabel=$y$, grid = major,
            ytick={-10,-8,-6,-4,-2,2,4,6,8,10},
            xtick={-10,-8,-6,-4,-2,2,4,6,8,10},
            ticklabel style={font=\scriptsize},
            every axis y label/.style={at=(current axis.above origin),anchor=south},
            every axis x label/.style={at=(current axis.right of origin),anchor=west},
            axis on top
          ]
          

       		\addplot [line width=2, penColor, smooth,samples=100,domain=(-2:4)] {-1*(x+3)*(x-3))};
			\addplot [line width=2, penColor, smooth,samples=100,domain=(1:6)] {(x-1)*(x-6)+3};



			\addplot[color=penColor,fill=white,only marks,mark=*] coordinates{(1,3)};
			\addplot[color=penColor,fill=penColor,only marks,mark=*] coordinates{(6,3)};

			\addplot[color=penColor,fill=white,only marks,mark=*] coordinates{(-2,5)};
			\addplot[color=penColor,fill=penColor,only marks,mark=*] coordinates{(4,-7)};

		

           

  \end{axis}
\end{tikzpicture}
\end{image}





\begin{align*}
\answer{-(t+3)(t-3)} &= t^2 - 7 t + 9  \\
\answer{-t^2 + 9}    &= t^2 - 7 t + 9 \\
0           &= 2 t^2 - 7t \\
0           &= t(2t - 7)  
\end{align*}






This gives two solutions, $0$ and $\frac{7}{2}$.  


We can see from the graph that the full graphs of $-(t+3)(t-3)$ and $t^2 - 7 t + 9$ would intersect at $t - 0$.  However, our piecewise defined functions do not share these formulas at $0$.  The definitions of the piecewise functions do not allow $0$ as a common solution.

After considering the domains of $V$ and $H$, the only solution here is $\frac{7}{2}$.






Check....
\begin{itemize}
\item $V(\tfrac{7}{2}) = -\left(\tfrac{7}{2} + 3\right) \left(\tfrac{7}{2} - 3\right) = -\tfrac{13}{2}  \cdot \tfrac{1}{2} = -\frac{13}{4}$
\item $K(\tfrac{7}{2}) =   \left(\tfrac{7}{2}\right)^2 - 7\left(\tfrac{7}{2}\right) + 9 = -\frac{13}{4}$
\end{itemize}


The intersection point is $\left(\tfrac{7}{2}, -\frac{13}{4}\right)$.

















\section{Third Point}




\begin{image}
\begin{tikzpicture}
  \begin{axis}[
            domain=-10:10, ymax=10, xmax=10, ymin=-10, xmin=-10,
            axis lines =center, xlabel=$w$, ylabel=$y$, grid = major,
            ytick={-10,-8,-6,-4,-2,2,4,6,8,10},
            xtick={-10,-8,-6,-4,-2,2,4,6,8,10},
            ticklabel style={font=\scriptsize},
            every axis y label/.style={at=(current axis.above origin),anchor=south},
            every axis x label/.style={at=(current axis.right of origin),anchor=west},
            axis on top
          ]
          

       		\addplot [line width=2, penColor, smooth,samples=100,domain=(4:8)] {1.75*x-8};


			\addplot[color=penColor,fill=white,only marks,mark=*] coordinates{(4,-1)};
			\addplot[color=penColor,fill=white,only marks,mark=*] coordinates{(8,6)};

			\addplot [line width=2, penColor, smooth,samples=100,domain=(1:6)] {(x-1)*(x-6)+3};



			\addplot[color=penColor,fill=white,only marks,mark=*] coordinates{(1,3)};
			\addplot[color=penColor,fill=penColor,only marks,mark=*] coordinates{(6,3)};

           

  \end{axis}
\end{tikzpicture}
\end{image}






\begin{align*}
\answer{\frac{7w}{4}}    &= w^2 - 7 w + 9     \\
0               &= w^2 - \answer{\frac{35w}{4}} + 17  \\
0           &= \frac{-(- \tfrac{35}{4}) \pm  \sqrt{(- \tfrac{35}{4})^2 - 4 \cdot 1 \cdot 17}}{2 \cdot 1} t \\
0           &= \frac{\tfrac{35}{4} \pm  \sqrt{\tfrac{137}{16}}}{2}  \\
0           &=   \frac{35}{8} \pm  \frac{\sqrt{137}}{8} \\
0          &= \frac{35 \pm \sqrt{137}}{8} 
\end{align*}







The graph shows us that we only have one of these solutions because the line segment doesn't extend further to the left to intercept the parabola twice.


The only soution here is $w = \frac{35 + \sqrt{137}}{8} $.








Check....
\begin{itemize}
\item $V\left(\frac{35 + \sqrt{137}}{8}\right) =  \frac{1}{32}(7 \sqrt{137} - 11)  $




\item $K\left(\frac{35 + \sqrt{137}}{8}\right) =  \frac{1}{32}(7 \sqrt{137} - 11)  $
\end{itemize}


The intersection point is $\left(\frac{35 + \sqrt{137}}{8}, \answer{\frac{1}{32}(7 \sqrt{137} - 11)} \right)$.












\end{document}
