\documentclass{ximera}


\graphicspath{
  {./}
  {ximeraTutorial/}
  {basicPhilosophy/}
}

\newcommand{\mooculus}{\textsf{\textbf{MOOC}\textnormal{\textsf{ULUS}}}}

\usepackage{tkz-euclide}\usepackage{tikz}
\usepackage{tikz-cd}
\usetikzlibrary{arrows}
\tikzset{>=stealth,commutative diagrams/.cd,
  arrow style=tikz,diagrams={>=stealth}} %% cool arrow head
\tikzset{shorten <>/.style={ shorten >=#1, shorten <=#1 } } %% allows shorter vectors

\usetikzlibrary{backgrounds} %% for boxes around graphs
\usetikzlibrary{shapes,positioning}  %% Clouds and stars
\usetikzlibrary{matrix} %% for matrix
\usepgfplotslibrary{polar} %% for polar plots
\usepgfplotslibrary{fillbetween} %% to shade area between curves in TikZ
\usetkzobj{all}
\usepackage[makeroom]{cancel} %% for strike outs
%\usepackage{mathtools} %% for pretty underbrace % Breaks Ximera
%\usepackage{multicol}
\usepackage{pgffor} %% required for integral for loops



%% http://tex.stackexchange.com/questions/66490/drawing-a-tikz-arc-specifying-the-center
%% Draws beach ball
\tikzset{pics/carc/.style args={#1:#2:#3}{code={\draw[pic actions] (#1:#3) arc(#1:#2:#3);}}}



\usepackage{array}
\setlength{\extrarowheight}{+.1cm}
\newdimen\digitwidth
\settowidth\digitwidth{9}
\def\divrule#1#2{
\noalign{\moveright#1\digitwidth
\vbox{\hrule width#2\digitwidth}}}






\DeclareMathOperator{\arccot}{arccot}
\DeclareMathOperator{\arcsec}{arcsec}
\DeclareMathOperator{\arccsc}{arccsc}

















%%This is to help with formatting on future title pages.
\newenvironment{sectionOutcomes}{}{}


\title{Piecewise Evaluation}

\begin{document}

\begin{abstract}
selecting formulas
\end{abstract}
\maketitle



Below are two piecewise defined functions and their graphs: $V(h)$ an $K(m)$. Use the definitions to evaluate the following expressions.








\[
V(h) = 
\begin{cases}
  -2h-3 & \text{ if } [-6, -2]   \\
  -(h+3)(h-3) & \text{ if } (-2, 4]  \\
  \frac{7h}{4} - 8 & \text{ if } (4,8)
\end{cases}
\]







\begin{image}
\begin{tikzpicture}
  \begin{axis}[
            domain=-10:10, ymax=10, xmax=10, ymin=-10, xmin=-10,
            axis lines =center, xlabel=$h$, ylabel={$y=V(h)$}, grid = major,
            ytick={-10,-8,-6,-4,-2,2,4,6,8,10},
            xtick={-10,-8,-6,-4,-2,2,4,6,8,10},
            ticklabel style={font=\scriptsize},
            every axis y label/.style={at=(current axis.above origin),anchor=south},
            every axis x label/.style={at=(current axis.right of origin),anchor=west},
            axis on top
          ]
          
			\addplot [line width=2, penColor, smooth,samples=100,domain=(-6:-2)] {-2*x-3};
       		\addplot [line width=2, penColor, smooth,samples=100,domain=(-2:4)] {-1*(x+3)*(x-3))};
       		\addplot [line width=2, penColor, smooth,samples=100,domain=(4:8)] {1.75*x-8};




			\addplot[color=penColor,fill=penColor,only marks,mark=*] coordinates{(-6,9)};
			\addplot[color=penColor,fill=penColor,only marks,mark=*] coordinates{(-2,1)};

			\addplot[color=penColor,fill=white,only marks,mark=*] coordinates{(-2,5)};
			\addplot[color=penColor,fill=penColor,only marks,mark=*] coordinates{(4,-7)};

			\addplot[color=penColor,fill=white,only marks,mark=*] coordinates{(4,-1)};
			\addplot[color=penColor,fill=white,only marks,mark=*] coordinates{(8,6)};


           

  \end{axis}
\end{tikzpicture}
\end{image}




\[
K(m) = 
\begin{cases}
  m+4 & \text{ if } (-7, 0]   \\
  m^2-7m+9 & \text{ if } (1,6]
\end{cases}
\]



\begin{image}
\begin{tikzpicture} 
  \begin{axis}[
            domain=-10:10, ymax=10, xmax=10, ymin=-10, xmin=-10,
            axis lines =center, xlabel=$m$, ylabel={$y=K(m)$}, grid = major,
            ytick={-10,-8,-6,-4,-2,2,4,6,8,10},
            tick={-10,-8,-6,-4,-2,2,4,6,8,10},
            ticklabel style={font=\scriptsize},
            every axis y label/.style={at=(current axis.above origin),anchor=south},
            every axis x label/.style={at=(current axis.right of origin),anchor=west},
            axis on top
          ]
          
			\addplot [line width=2, penColor, smooth,samples=100,domain=(-7:0)] {x+4};
       		\addplot [line width=2, penColor, smooth,samples=100,domain=(1:6)] {(x-1)*(x-6)+3};



			\addplot[color=penColor,fill=white,only marks,mark=*] coordinates{(-7,-3)};
			\addplot[color=penColor,fill=penColor,only marks,mark=*] coordinates{(0,4)};

			\addplot[color=penColor,fill=white,only marks,mark=*] coordinates{(1,3)};
			\addplot[color=penColor,fill=penColor,only marks,mark=*] coordinates{(6,3)};

       

  \end{axis}
\end{tikzpicture}
\end{image}





\begin{question}


\[
V(h) = 
\begin{cases}
  -2h-3 & \text{ if } [-6, -2]   \\
  -(h+3)(h-3) & \text{ if } (-2, 4]  \\
  \frac{7h}{4} - 8 & \text{ if } (4,8)
\end{cases}
\]


\[
K(m) = 
\begin{cases}
  m+4 & \text{ if } (-7, 0]   \\
  m^2-7m+9 & \text{ if } (1,6]
\end{cases}
\]



Calculate the following expressions using the definitions above. Type \textbf{DNE} for "does not exist''.


\[
\begin{array}{rrl|rrl}
(1) & V(-2)  & = \answer{1} &  (2) &  K(0) & =\answer{4} \\
(3) & V(4)  & = \answer{-7} &  (4) &  K(2) & =\answer{-1} \\
(5) & V(6-2)  & = \answer{-7} &  (6) &  3-K(3) & =\answer{6} \\
(7) & (V(1))^2  & = \answer{64} &  (8) &  K(-6)K(0) & =\answer{-8} \\
    &           &               &      &            & \\
(9) & V(-4)  & = \answer{5} &  (10) &  K(V(-4)) & =\answer{-1} \\
(11) & V(-3)  & = \answer{3} &  (12) &  K(V(-3)) & =\answer{-3} \\
(13) & V(5)  & = \answer{\frac{3}{4}} &  (14) &  K(V(5)) & =\answer{DNE} \\
(15) & K(-3)  & = \answer{1} &  (16) &  V(K(-3)) & =\answer{8} \\
     &        &              &       &           & \\
(17) & V(-5)  & = \answer{7} &  (18) &  V(V(-5)) & =\answer{\frac{17}{4}} \\
(19) & V(-2)  & = \answer{1} &  (20) &  V(V(-2)) & =\answer{8} \\
(21) & K(0)  & = \answer{4} &  (22) &  K(K(0)) & =\answer{-3} \\
(23) & K(-4)  & = \answer{0} &  (24) &  K(K(0)) & =\answer{-3} 
\end{array}
\]


\end{question}



















\begin{center}
\textbf{\textcolor{green!50!black}{ooooo=-=-=-=-=-=-=-=-=-=-=-=-=ooOoo=-=-=-=-=-=-=-=-=-=-=-=-=ooooo}} \\

more examples can be found by following this link\\ \link[More Examples of Piecewise-Defined Functions]{https://ximera.osu.edu/csccmathematics/precalculus1/precalculus1/piecewiseAnalysis/examples/exampleList}

\end{center}





\end{document}
