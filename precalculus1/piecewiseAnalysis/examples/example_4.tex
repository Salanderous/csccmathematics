\documentclass{ximera}


\graphicspath{
  {./}
  {ximeraTutorial/}
  {basicPhilosophy/}
}

\newcommand{\mooculus}{\textsf{\textbf{MOOC}\textnormal{\textsf{ULUS}}}}

\usepackage{tkz-euclide}\usepackage{tikz}
\usepackage{tikz-cd}
\usetikzlibrary{arrows}
\tikzset{>=stealth,commutative diagrams/.cd,
  arrow style=tikz,diagrams={>=stealth}} %% cool arrow head
\tikzset{shorten <>/.style={ shorten >=#1, shorten <=#1 } } %% allows shorter vectors

\usetikzlibrary{backgrounds} %% for boxes around graphs
\usetikzlibrary{shapes,positioning}  %% Clouds and stars
\usetikzlibrary{matrix} %% for matrix
\usepgfplotslibrary{polar} %% for polar plots
\usepgfplotslibrary{fillbetween} %% to shade area between curves in TikZ
\usetkzobj{all}
\usepackage[makeroom]{cancel} %% for strike outs
%\usepackage{mathtools} %% for pretty underbrace % Breaks Ximera
%\usepackage{multicol}
\usepackage{pgffor} %% required for integral for loops



%% http://tex.stackexchange.com/questions/66490/drawing-a-tikz-arc-specifying-the-center
%% Draws beach ball
\tikzset{pics/carc/.style args={#1:#2:#3}{code={\draw[pic actions] (#1:#3) arc(#1:#2:#3);}}}



\usepackage{array}
\setlength{\extrarowheight}{+.1cm}
\newdimen\digitwidth
\settowidth\digitwidth{9}
\def\divrule#1#2{
\noalign{\moveright#1\digitwidth
\vbox{\hrule width#2\digitwidth}}}






\DeclareMathOperator{\arccot}{arccot}
\DeclareMathOperator{\arcsec}{arcsec}
\DeclareMathOperator{\arccsc}{arccsc}

















%%This is to help with formatting on future title pages.
\newenvironment{sectionOutcomes}{}{}



\author{Lee Wayand}

\begin{document}
\begin{exercise}





\[
H(k) = 
\begin{cases}
  -(k+4)(k-2)   & \text{ on } [-5, 2)   \\
  (k-2)(k-7)      & \text{ on } [2, 8]  
\end{cases}
\]






Graph of $y = H(k)$.



\begin{image}
\begin{tikzpicture} 
  \begin{axis}[
            domain=-10:10, ymax=10, xmax=10, ymin=-10, xmin=-10,
            axis lines =center, xlabel=$k$, ylabel=$y$, grid = major,
            ytick={-10,-8,-6,-4,-2,2,4,6,8,10},
            xtick={-10,-8,-6,-4,-2,2,4,6,8,10},
            ticklabel style={font=\scriptsize},
            every axis y label/.style={at=(current axis.above origin),anchor=south},
            every axis x label/.style={at=(current axis.right of origin),anchor=west},
            axis on top
          ]
          
          \addplot [line width=2, penColor, smooth,samples=100,domain=(-5:2)] {-(x+4)*(x-2)};
       	  \addplot [line width=2, penColor, smooth,samples=100,domain=(2:8)] {(x-2)*(x-7)};
       		%\addplot [line width=2, penColor, smooth,samples=100,domain=(4:8)] {3*x-16};




			\addplot[color=penColor,fill=penColor,only marks,mark=*] coordinates{(-5,-7)};

			%\addplot[color=penColor,fill=penColor,only marks,mark=*] coordinates{(2,6)};
			%\addplot[color=penColor,fill=white,only marks,mark=*] coordinates{(2,0)};

			\addplot[color=penColor,fill=penColor,only marks,mark=*] coordinates{(8,6)};
			%\addplot[color=penColor,fill=white,only marks,mark=*] coordinates{(4,-4)};

			%\addplot[color=penColor,fill=white,only marks,mark=*] coordinates{(8,8)};

  \end{axis}
\end{tikzpicture}
\end{image}






$H(k)$ is a continuous piecewise function.  The pieces change at $k = 2$.  Is it a smooth transition?  That is, the funciton values match at $k = 2$, but do the rates of change match? \\


Both pieces of $H$ are quadratics, which, by themselves, would have domains of all real numbers.  Both parabolas have tangent lines at $k = 2$.  We need to compare their slopes. \\



Let's rename the pieces. \\

\begin{itemize}
\item   $H_L(k) = -(k+4)(k-2) = -k^2 - 2k + 8$. $iRoC(k) = \answer{-2k - 2}$.
\item   $H_R(k) = (k-2)(k-7) = k^2 - 9k + 14$. $iRoC(k) = \answer{2k - 9}$.
\end{itemize}



At $k = 2$, the two slopes are $-6$ and $-5$.  Close, but not equal. \\

There is not a smooth transition.









\end{exercise}
\end{document}