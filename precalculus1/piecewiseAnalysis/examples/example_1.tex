\documentclass{ximera}


\graphicspath{
  {./}
  {ximeraTutorial/}
  {basicPhilosophy/}
}

\newcommand{\mooculus}{\textsf{\textbf{MOOC}\textnormal{\textsf{ULUS}}}}

\usepackage{tkz-euclide}\usepackage{tikz}
\usepackage{tikz-cd}
\usetikzlibrary{arrows}
\tikzset{>=stealth,commutative diagrams/.cd,
  arrow style=tikz,diagrams={>=stealth}} %% cool arrow head
\tikzset{shorten <>/.style={ shorten >=#1, shorten <=#1 } } %% allows shorter vectors

\usetikzlibrary{backgrounds} %% for boxes around graphs
\usetikzlibrary{shapes,positioning}  %% Clouds and stars
\usetikzlibrary{matrix} %% for matrix
\usepgfplotslibrary{polar} %% for polar plots
\usepgfplotslibrary{fillbetween} %% to shade area between curves in TikZ
\usetkzobj{all}
\usepackage[makeroom]{cancel} %% for strike outs
%\usepackage{mathtools} %% for pretty underbrace % Breaks Ximera
%\usepackage{multicol}
\usepackage{pgffor} %% required for integral for loops



%% http://tex.stackexchange.com/questions/66490/drawing-a-tikz-arc-specifying-the-center
%% Draws beach ball
\tikzset{pics/carc/.style args={#1:#2:#3}{code={\draw[pic actions] (#1:#3) arc(#1:#2:#3);}}}



\usepackage{array}
\setlength{\extrarowheight}{+.1cm}
\newdimen\digitwidth
\settowidth\digitwidth{9}
\def\divrule#1#2{
\noalign{\moveright#1\digitwidth
\vbox{\hrule width#2\digitwidth}}}






\DeclareMathOperator{\arccot}{arccot}
\DeclareMathOperator{\arcsec}{arcsec}
\DeclareMathOperator{\arccsc}{arccsc}

















%%This is to help with formatting on future title pages.
\newenvironment{sectionOutcomes}{}{}



\author{Lee Wayand}

\begin{document}
\begin{exercise}





\[
f(x) = 
\begin{cases}
  -\frac{x}{2} - 4   & \text{ if } [-8, -2)   \\
  (x+2)(x-3) - 1      & \text{ if } [-2, 4]  \\
  3x-16              & \text{ if } (4,8)
\end{cases}
\]






Graph of $y = f(x)$.



\begin{image}
\begin{tikzpicture} 
  \begin{axis}[
            domain=-10:10, ymax=10, xmax=10, ymin=-10, xmin=-10,
            axis lines =center, xlabel=$x$, ylabel=$y$, grid = major,
            ytick={-10,-8,-6,-4,-2,2,4,6,8,10},
            xtick={-10,-8,-6,-4,-2,2,4,6,8,10},
            ticklabel style={font=\scriptsize},
            every axis y label/.style={at=(current axis.above origin),anchor=south},
            every axis x label/.style={at=(current axis.right of origin),anchor=west},
            axis on top
          ]
          
          \addplot [line width=2, penColor, smooth,samples=100,domain=(-8:-2)] {-0.5*x-4};
       	  \addplot [line width=2, penColor, smooth,samples=100,domain=(-2:4)] {(x+2)*(x-3)-1};
       		\addplot [line width=2, penColor, smooth,samples=100,domain=(4:8)] {3*x-16};




			\addplot[color=penColor,fill=penColor,only marks,mark=*] coordinates{(-8,0)};

			\addplot[color=penColor,fill=penColor,only marks,mark=*] coordinates{(-2,-1)};
			\addplot[color=penColor,fill=white,only marks,mark=*] coordinates{(-2,-3)};

			\addplot[color=penColor,fill=penColor,only marks,mark=*] coordinates{(4,5)};
			\addplot[color=penColor,fill=white,only marks,mark=*] coordinates{(4,-4)};

			\addplot[color=penColor,fill=white,only marks,mark=*] coordinates{(8,8)};

  \end{axis}
\end{tikzpicture}
\end{image}







\begin{question} Domain


The domain is given in the definition of the function.

\[
[ \answer{-8}, \answer{-2} ) \cup [  \answer{-2}, \answer{4} ] \cup (  \answer{4}, \answer{8} ) = \left[ \answer{-8}, \answer{8} \right)
\]


\end{question}







\begin{question} Continuity


The three pieces in the definition of $f(x)$ are linear, quadratic, and linear.  These are continuous functions.  Therefore, the only candidates for discontinuities or singularities are the endpoints of the defining intervals.

$-8$ is included in $[-8, -2)$ and $-\frac{x}{2} - 4$ is continuous on this interval. \\

$-2$ is a discontinuity of $f$.  Let $d = 1$. For any small $\epsilon > 0$, the interval $(-2 - \epsilon, -2)$ always contains the number $-2 - \frac{\epsilon}{2}$.  


\[
f\left( -2 - \frac{\epsilon}{2} \right) = 1 + \frac{\epsilon}{4} - 4 =  \frac{\epsilon}{4} - 3
\]


\[
f(-2) - f\left( -2 - \frac{\epsilon}{2} \right) = -1 - \frac{\epsilon}{4} + 3 = 2 - \frac{\epsilon}{4} > 1 = d
\]


Every small open interval around $-2$ contains a domain number where the value of $f$ is more than 1 away from $f(-2)$.





Similarly, $4$ is also a discontinuity.  \\



$8$ is not in the domain.  It cannot be a singularity since it is not a hole in the domain.  For the number $a$ to be a hole, there would need to be an open interval inside the domain, except for $a$:  $(c, a) \cup (a, b)$. 


\end{question}





\begin{question} Zeros


The defininng pieces of $f$ have zeros.  The linear pieces have one zero and the quadratic piece has two zeros.  We'll determine these zeros and then see if they are in the domain of $f$. \\



The zero of $f_1(x) = -\frac{x}{2} - 4$ is $-8$, which is included in the domain of $f$. \\

There are two zeros for $f_2(x) = (x+2)(x-3) - 1$. \\


$ 0 = (x+2)(x-3) - 1 = x^2 - x - 7$ \\

\[
x = \frac{1 \pm \sqrt{(-1)^2 - 4 \cdot 1 \cdot (-7)}}{2} = \frac{1 \pm \sqrt{29}}{2}
\]


Only $\frac{1 \pm \sqrt{29}}{2}$ is contained in the domain of $f$. \\


The zero of $f_3(x) = 3x-16$ is $\frac{16}{3}$, which is included in the domain of $f$. \\




$f$ has three zeros:  $-8, \frac{1 \pm \sqrt{29}}{2}, \frac{16}{3}$.







\end{question}




\end{exercise}
\end{document}