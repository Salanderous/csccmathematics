\documentclass{ximera}


\graphicspath{
  {./}
  {ximeraTutorial/}
  {basicPhilosophy/}
}

\newcommand{\mooculus}{\textsf{\textbf{MOOC}\textnormal{\textsf{ULUS}}}}

\usepackage{tkz-euclide}\usepackage{tikz}
\usepackage{tikz-cd}
\usetikzlibrary{arrows}
\tikzset{>=stealth,commutative diagrams/.cd,
  arrow style=tikz,diagrams={>=stealth}} %% cool arrow head
\tikzset{shorten <>/.style={ shorten >=#1, shorten <=#1 } } %% allows shorter vectors

\usetikzlibrary{backgrounds} %% for boxes around graphs
\usetikzlibrary{shapes,positioning}  %% Clouds and stars
\usetikzlibrary{matrix} %% for matrix
\usepgfplotslibrary{polar} %% for polar plots
\usepgfplotslibrary{fillbetween} %% to shade area between curves in TikZ
\usetkzobj{all}
\usepackage[makeroom]{cancel} %% for strike outs
%\usepackage{mathtools} %% for pretty underbrace % Breaks Ximera
%\usepackage{multicol}
\usepackage{pgffor} %% required for integral for loops



%% http://tex.stackexchange.com/questions/66490/drawing-a-tikz-arc-specifying-the-center
%% Draws beach ball
\tikzset{pics/carc/.style args={#1:#2:#3}{code={\draw[pic actions] (#1:#3) arc(#1:#2:#3);}}}



\usepackage{array}
\setlength{\extrarowheight}{+.1cm}
\newdimen\digitwidth
\settowidth\digitwidth{9}
\def\divrule#1#2{
\noalign{\moveright#1\digitwidth
\vbox{\hrule width#2\digitwidth}}}






\DeclareMathOperator{\arccot}{arccot}
\DeclareMathOperator{\arcsec}{arcsec}
\DeclareMathOperator{\arccsc}{arccsc}

















%%This is to help with formatting on future title pages.
\newenvironment{sectionOutcomes}{}{}



\author{Lee Wayand}

\begin{document}
\begin{exercise}





\[
H(k) = 
\begin{cases}
  -(k+4)(k-2)   & \text{ on } [-5, 2)   \\
  6              & \text{ if } k = 2 \\
  (k-2)(k-7)      & \text{ on } (2, 8]  
\end{cases}
\]






Graph of $y = H(k)$.



\begin{image}
\begin{tikzpicture} 
  \begin{axis}[
            domain=-10:10, ymax=10, xmax=10, ymin=-10, xmin=-10,
            axis lines =center, xlabel=$k$, ylabel=$y$, grid = major,
            ytick={-10,-8,-6,-4,-2,2,4,6,8,10},
            xtick={-10,-8,-6,-4,-2,2,4,6,8,10},
            ticklabel style={font=\scriptsize},
            every axis y label/.style={at=(current axis.above origin),anchor=south},
            every axis x label/.style={at=(current axis.right of origin),anchor=west},
            axis on top
          ]
          
          \addplot [line width=2, penColor, smooth,samples=100,domain=(-8:-2)] {-(x+4)*(x-2)};
       	  \addplot [line width=2, penColor, smooth,samples=100,domain=(-2:4)] {(x-2)*(x-7)};
       		%\addplot [line width=2, penColor, smooth,samples=100,domain=(4:8)] {3*x-16};




			\addplot[color=penColor,fill=penColor,only marks,mark=*] coordinates{(-5,-7)};

			\addplot[color=penColor,fill=penColor,only marks,mark=*] coordinates{(2,6)};
			\addplot[color=penColor,fill=white,only marks,mark=*] coordinates{(2,0)};

			\addplot[color=penColor,fill=penColor,only marks,mark=*] coordinates{(8,6)};
			%\addplot[color=penColor,fill=white,only marks,mark=*] coordinates{(4,-4)};

			%\addplot[color=penColor,fill=white,only marks,mark=*] coordinates{(8,8)};

  \end{axis}
\end{tikzpicture}
\end{image}







\begin{question} Domain


The domain is given in the definition of the function.

\[
\left[ \answer{-5}, \answer{2} \right) \cup \left\{ \answer{2} \right\} \cup \left(  \answer{2}, \answer{8} \right] = \left[ \answer{-5}, \answer{8} \right]
\]


\end{question}







\begin{question} Continuity


The two pieces in the definition of $H(k)$ are quadratic functions.  These are continuous functions.  Therefore, the only candidate for discontinuity or singularity is $x = 2$.

The graph suggests that $x = 2$ is a discontinuity. We will show this is true. \\

Let $d = 1$. \\

For any small $\frac{1}{100} > \epsilon > 0$, the interval $(2 - \epsilon, 2)$ always contains the number $2 - \frac{\epsilon}{2}$.  


\[
H\left( 2 - \frac{\epsilon}{2} \right) = H\left( \frac{4 - \epsilon}{2} \right) = -\left( \frac{4 - \epsilon}{2} + 4 \right) \left(  \frac{4 - \epsilon}{2} - 2 \right)
\]

\[
= -\left( \frac{12 - \epsilon}{2} \right) \left(  \frac{- \epsilon}{2} \right) = \left( \frac{12 - \epsilon}{2} \right) \left(  \frac{\epsilon}{2} \right)
\]


Since, $\frac{1}{100} > \epsilon > 0$, $\frac{12 - \epsilon}{2} < 12$ and $\frac{\epsilon}{2} < \frac{1}{200}$


Therefore, 

\[
H\left( 2 - \frac{\epsilon}{2} \right) < \frac{12}{200} 
\]



On the interval, $(2 - \epsilon, 2)$, $H$ always has a value less than $\frac{12}{200}$.  Its distance from $H(2) = 6$ is greater than $d = 1$.



Perhaps the other side of $2$ would have been easier. \\

On the interval $(2, 2 + \epsilon)$, the second piece of $H$ is negative, which is more than $1$ away from $H(2) = 6$. \\


Every small open interval around $2$ contains a domain number where the value of $H$ is more than $1$ away from $H(2)$. \\


$x = 2$ is a discontinuity.



\end{question}





\begin{question} Zeros


The defininng pieces of $H$ have two zeros each.  The graph suggests that $x=2$ would be a zero for each quadratic piece, however, $H$ has moved this one value away from $0$.  That leaves two zeros for $H$.


\begin{explanation}



The left piece is a quadratic, which we have in factored form, $-(k+4)(k-2)$.  The zeros are $-4$ and $2$.  $H$ has redefined $H(2)$ to be $6$. \\


The right piece is a quadratic, which we have in factored form, $(k-2)(k-7)$.  The zeros are $2$ and $7$.  $H$ has redefined $H(2)$ to be $6$. \\


The zeros of $H$ are $\{ -4, 7 \}$.


\end{explanation}


\end{question}





\begin{question} End-Behavior



End-behavior is explicitly for unbounded domains.  Therefore, there is no end-behavior for $H$.


\end{question}






\begin{question} Behavior



From the graph, we can see that $H$ increases, then decreases, and then increases.  We want algebraic and functional reasoning. $iRoC_H(k)$ will give us behavior information.   \\


Avoiding the discontinuity, we have

\[
iRoC_H(k) = 
\begin{cases}
  \answer{-2k - 2}   & \text{ on } [-5, 2)   \\
  \answer{2k - 9}     & \text{ on } (2, 8]  
\end{cases}
\]



$-2k - 2$ gives us a critical number $\answer{-1}$. \\


$2k - 9$ gives us a critical number $\answer{\frac{9}{2}}$. \\





\begin{itemize}
\item $-2k - 2$ \wordChoice{\choice{<} [correct]\choice{>}}  $0$ on  $[-5, -1)$ where $H$ is increasing.
\item $-2k - 2$ \wordChoice{\choice[correct]{<} \choice{>}}  $0$ on  $(-1, 2)$ where $H$ is decreasing.
\item $2k - 9$ \wordChoice{\choice[correct]{<} \choice{>}}  $0$ on  $\left( 2, \frac{9}{2} \right)$ where $H$ is decreasing.
\item $2k - 9$ \wordChoice{\choice{<} [correct]\choice{>}}  $0$ on  $\left( \frac{9}{2}, 8 \right]$ where $H$ is increasing.
\end{itemize}







\begin{warning}


$H$ decreases on $[-1, 2)$ and on $\left( \frac{9}{2}, 8 \right]$, however, $H$ does not decrease on their union $[-1, 2) \cup \left( \frac{9}{2}, 8 \right]$.



We can show this algebraically, by providing a counterexample to the definition of decreasing. Select two domain numbers $a < b$, yet $f(a) < f(b)$.  Our two numbers will be $\frac{7}{4} < \frac{15}{2}$.


\begin{align*}
H\left( \frac{7}{4} \right & = -\left( \frac{7}{4} + 4 \right) \left( \frac{7}{4} - 2 \right) \\
& = -\left( \frac{23}{4} \right) \left( -\frac{1}{4} \right) \\
& = \frac{23}{16}  
\end{align*}


\begin{align*}
H\left( \frac{15}{2} \right & = \left( \frac{15}{2} - 2 \right) \left( \frac{15}{2} - 7 \right) \\
& = \left( \frac{11}{2} \right) \left( \frac{1}{2} \right) \\
& = \frac{11}{4}  
\end{align*}


\[
\frac{23}{16}  < \frac{32}{16} = 2 = \frac{8}{4} < \frac{11}{4} 
\]

\end{warning}


We now have a counterexample to the statement that $H$ is decreasing on the set $[-1, 2) \cup \left( \frac{9}{2}, 8 \right]$.





\end{question}








\end{exercise}
\end{document}