\documentclass{ximera}


\graphicspath{
  {./}
  {ximeraTutorial/}
  {basicPhilosophy/}
}

\newcommand{\mooculus}{\textsf{\textbf{MOOC}\textnormal{\textsf{ULUS}}}}

\usepackage{tkz-euclide}\usepackage{tikz}
\usepackage{tikz-cd}
\usetikzlibrary{arrows}
\tikzset{>=stealth,commutative diagrams/.cd,
  arrow style=tikz,diagrams={>=stealth}} %% cool arrow head
\tikzset{shorten <>/.style={ shorten >=#1, shorten <=#1 } } %% allows shorter vectors

\usetikzlibrary{backgrounds} %% for boxes around graphs
\usetikzlibrary{shapes,positioning}  %% Clouds and stars
\usetikzlibrary{matrix} %% for matrix
\usepgfplotslibrary{polar} %% for polar plots
\usepgfplotslibrary{fillbetween} %% to shade area between curves in TikZ
\usetkzobj{all}
\usepackage[makeroom]{cancel} %% for strike outs
%\usepackage{mathtools} %% for pretty underbrace % Breaks Ximera
%\usepackage{multicol}
\usepackage{pgffor} %% required for integral for loops



%% http://tex.stackexchange.com/questions/66490/drawing-a-tikz-arc-specifying-the-center
%% Draws beach ball
\tikzset{pics/carc/.style args={#1:#2:#3}{code={\draw[pic actions] (#1:#3) arc(#1:#2:#3);}}}



\usepackage{array}
\setlength{\extrarowheight}{+.1cm}
\newdimen\digitwidth
\settowidth\digitwidth{9}
\def\divrule#1#2{
\noalign{\moveright#1\digitwidth
\vbox{\hrule width#2\digitwidth}}}






\DeclareMathOperator{\arccot}{arccot}
\DeclareMathOperator{\arcsec}{arcsec}
\DeclareMathOperator{\arccsc}{arccsc}

















%%This is to help with formatting on future title pages.
\newenvironment{sectionOutcomes}{}{}



\author{Lee Wayand}

\begin{document}


\begin{exercise} 




Let $M(k) = 3(k+2)(k-5)$ with its natural domain. \\


Describe the graph of $y = M(k)$.






\begin{question}  \textbf{\textcolor{blue!55!black}{Shape}}


$M$ is a \wordChoice{\choice{constant} \choice{linear} \choice[correct]{quadratic}} function, which means its graph is a \wordChoice{\choice{line} \choice[correct]{parabola}}. \\

The leading coefficient of $M$ is $\answer{3}$. Therefore, the parabola opens \wordChoice{\choice[correct]{up} \choice{down}}



\end{question}








\begin{question}  \textbf{\textcolor{blue!55!black}{Intercepts}}


The real zeros of $M$ are $-2$ and $5$, which make the intercepts

\[
\left( \answer{-2}, \answer{0}) \, \text{ and } \, (\answer{5}, \answer{0} \right)
\]

\end{question}







\begin{question}  \textbf{\textcolor{blue!55!black}{Breaks}}


Because the graph is a parabola, it has no breaks.

\end{question}







\begin{question}  \textbf{\textcolor{blue!55!black}{Axis of Symmetry}}


The midpoint of $-2$ and $5$ is $\answer{\frac{3}{2}}$, therefore, the axis of symmetry has the equation

\[
k = \answer{\frac{3}{2}}
\]

\end{question}







\begin{question}  \textbf{\textcolor{blue!55!black}{Vertex}}


\[
M\left( \frac{3}{2} \right) = -\frac{147}{4}
\]


Therefore, the vertex is

 \[  
 \left( \answer{\frac{3}{2}}, \answer{-\frac{147}{4}} \right)
\]


\end{question}











\end{exercise}


\end{document}