\documentclass{ximera}


\graphicspath{
  {./}
  {ximeraTutorial/}
  {basicPhilosophy/}
}

\newcommand{\mooculus}{\textsf{\textbf{MOOC}\textnormal{\textsf{ULUS}}}}

\usepackage{tkz-euclide}\usepackage{tikz}
\usepackage{tikz-cd}
\usetikzlibrary{arrows}
\tikzset{>=stealth,commutative diagrams/.cd,
  arrow style=tikz,diagrams={>=stealth}} %% cool arrow head
\tikzset{shorten <>/.style={ shorten >=#1, shorten <=#1 } } %% allows shorter vectors

\usetikzlibrary{backgrounds} %% for boxes around graphs
\usetikzlibrary{shapes,positioning}  %% Clouds and stars
\usetikzlibrary{matrix} %% for matrix
\usepgfplotslibrary{polar} %% for polar plots
\usepgfplotslibrary{fillbetween} %% to shade area between curves in TikZ
\usetkzobj{all}
\usepackage[makeroom]{cancel} %% for strike outs
%\usepackage{mathtools} %% for pretty underbrace % Breaks Ximera
%\usepackage{multicol}
\usepackage{pgffor} %% required for integral for loops



%% http://tex.stackexchange.com/questions/66490/drawing-a-tikz-arc-specifying-the-center
%% Draws beach ball
\tikzset{pics/carc/.style args={#1:#2:#3}{code={\draw[pic actions] (#1:#3) arc(#1:#2:#3);}}}



\usepackage{array}
\setlength{\extrarowheight}{+.1cm}
\newdimen\digitwidth
\settowidth\digitwidth{9}
\def\divrule#1#2{
\noalign{\moveright#1\digitwidth
\vbox{\hrule width#2\digitwidth}}}






\DeclareMathOperator{\arccot}{arccot}
\DeclareMathOperator{\arcsec}{arcsec}
\DeclareMathOperator{\arccsc}{arccsc}

















%%This is to help with formatting on future title pages.
\newenvironment{sectionOutcomes}{}{}



\author{Lee Wayand}

\begin{document}


\begin{exercise} 




Let $p(x) = 2x^2 - x + 3$ with its natural domain. \\


Completely analyze $p$. \\


\begin{question}  \textbf{\textcolor{blue!55!black}{Domain}}


$p$ is a \wordChoice{\choice{constant} \choice{linear} \choice[correct]{quadratic}} function, which means its natural domain is all real numbers.

\end{question}








\begin{question}  \textbf{\textcolor{blue!55!black}{Zeros}}


$p(x)$ doesn't factor easily. factor.  We'll use the quadratic formula.\\

$2x^2 - x + 3 = $ \\


$a = \answer{2}$ \\

$b = \answer{-1}$ \\

$c = \answer{3}$ \\


\[
x = \frac{-(-1) \pm \sqrt{(-1)^2 - 4 \cdot 2 \cdot 3}}{2 \cdot 2}
\]


\[
x = \frac{1 \pm \sqrt{-23}}{4}
\]




$p$ has no real zeros.


\end{question}







\begin{question}  \textbf{\textcolor{blue!55!black}{Continuity}}


$p$ is a \wordChoice{\choice{constant} \choice{linear} \choice[correct]{quadratic}} function, which means it is continuous.

\end{question}









\begin{question}  \textbf{\textcolor{blue!55!black}{End-Behavior}}


The leading coefficient of $p$ is $\answer{2}$. \\


Because $p$ is a quadratic function with a positive leading coefficient,

$\lim\limits_{x \to -\infty} p(x) = \answer{\infty}$ \\


$\lim\limits_{x \to \infty} p(x) = \answer{\infty}$ \\

\end{question}




\begin{question}  \textbf{\textcolor{blue!55!black}{Behavior}}

Let's rewrite $p$ as a completed square.


\begin{align*}
p(x) & = 2x^2 - x + 3 \\
& = 2 (x^2 - \frac{1}{2} x) + 3  \\
& = 2 (x^2 - \frac{1}{2} x + \frac{1}{16} - \frac{1}{16}) + 3  \\
& = 2 (x - \frac{1}{4})^2 - \frac{1}{16} + 3  \\
& = 2 (x - \frac{1}{4})^2 + \frac{47}{16} + 3  
\end{align*}



Because $p$ is a quadratic function with a positive leading coefficient, $p$ decreases on $\left( \answer{-\infty}, \answer{\frac{1}{4}} \right]$ and increases on $\left[ \answer{\frac{1}{4}}, \answer{\infty} \right)$. 

\end{question}









\begin{question}  \textbf{\textcolor{blue!55!black}{Global Extrema}}

Because $p$ decreases on $\left( -\infty, \frac{1}{4} \right]$ and increases on $\left[ \frac{1}{4}, \infty \right)$, $p$ has a global \wordChoice{\choice[correct]{minimum} \choice{maximum}}  of $\answer{\frac{47}{16}}$ at $\answer{\frac{1}{4}}$.



Because $\lim\limits_{x \to \infty} p(x) = \answer{\infty}$, $p$ has no global \wordChoice{\choice{minimum} \choice[correct]{maximum}}.

\end{question}












\begin{question}  \textbf{\textcolor{blue!55!black}{Local Extrema}}

The only local minimum value of $p$ is the global minimum of $\answer{\frac{47}{16}}$ at $\answer{\frac{1}{4}}$.

\end{question}







\begin{question}  \textbf{\textcolor{blue!55!black}{Range}}

$p$ is continuous with no maximum value and a minimum value of $\frac{47}{16}$.  Therefore, the range of $p$ is

\[
\left[ \answer{\frac{47}{16}}, \answer{\infty} \right)
\]

\end{question}


\end{exercise}


\end{document}