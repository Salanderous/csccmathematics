\documentclass{ximera}


\graphicspath{
  {./}
  {ximeraTutorial/}
  {basicPhilosophy/}
}

\newcommand{\mooculus}{\textsf{\textbf{MOOC}\textnormal{\textsf{ULUS}}}}

\usepackage{tkz-euclide}\usepackage{tikz}
\usepackage{tikz-cd}
\usetikzlibrary{arrows}
\tikzset{>=stealth,commutative diagrams/.cd,
  arrow style=tikz,diagrams={>=stealth}} %% cool arrow head
\tikzset{shorten <>/.style={ shorten >=#1, shorten <=#1 } } %% allows shorter vectors

\usetikzlibrary{backgrounds} %% for boxes around graphs
\usetikzlibrary{shapes,positioning}  %% Clouds and stars
\usetikzlibrary{matrix} %% for matrix
\usepgfplotslibrary{polar} %% for polar plots
\usepgfplotslibrary{fillbetween} %% to shade area between curves in TikZ
\usetkzobj{all}
\usepackage[makeroom]{cancel} %% for strike outs
%\usepackage{mathtools} %% for pretty underbrace % Breaks Ximera
%\usepackage{multicol}
\usepackage{pgffor} %% required for integral for loops



%% http://tex.stackexchange.com/questions/66490/drawing-a-tikz-arc-specifying-the-center
%% Draws beach ball
\tikzset{pics/carc/.style args={#1:#2:#3}{code={\draw[pic actions] (#1:#3) arc(#1:#2:#3);}}}



\usepackage{array}
\setlength{\extrarowheight}{+.1cm}
\newdimen\digitwidth
\settowidth\digitwidth{9}
\def\divrule#1#2{
\noalign{\moveright#1\digitwidth
\vbox{\hrule width#2\digitwidth}}}






\DeclareMathOperator{\arccot}{arccot}
\DeclareMathOperator{\arcsec}{arcsec}
\DeclareMathOperator{\arccsc}{arccsc}

















%%This is to help with formatting on future title pages.
\newenvironment{sectionOutcomes}{}{}



\author{Lee Wayand}

\begin{document}


















\begin{exercise} 


Complete the square to write $7 - T^2 - 6T$ in vertex form. \\



$7 - T^2 - 6T$ is not quite in standard form. Let's rearrange terms. $-T^2 - 6T + 7$ is in standard form, therefore



\begin{itemize}
\item  a = $\answer{-1}$ \\
\item  b = $\answer{-6}$ \\
\item  c = $\answer{7}$ \\
\end{itemize}




$a \ne 1$, so let's factor it out of the quadratic and linear terms.



\[   -T^2 - 6T + 7 = -(T^2 + 6T) + 7   \]




\begin{procedure} \textbf{Refocus:}  


Now, we are just focusing on the quadratic inside the parentheses.  It doesn't have a constant term.  The $7$ will just float around outside the parentheses for a moment.



Our inside quadratic is $T^2 + 6T$ or $T^2 + 6T + 0$.


\begin{itemize}
\item  a = $\answer{1}$ \\
\item  b = $\answer{6}$ \\
\item  c = $\answer{0}$ \\
\end{itemize}



This makes $\frac{b}{2} = \answer{3}$, and its square is $\left( \frac{b}{2} \right)^2 = \answer{9}$, which is added and subtracted to the expression.  That way we have only add $0$ to the expression and not changed any values. 


\[ T^2 + 6T + \answer{9} - \answer{9} \]



The ``added'' number is grouped together with the quadratic and linear terms to form a perfect square.



\[ ( T + \answer{3} )^2  - 9 \]





\end{procedure}


That was happening inside the parentheses, so let's replace the old inside with the new inside.


\[ -( ( T + 3 )^2  - 9 ) + 7 \]




Distribute and combine the constant terms.



\[ -( T + 3 )^2  + 9 + 7 \]


\[ -( T + 3 )^2  + 16  \]



\end{exercise}










\end{document}