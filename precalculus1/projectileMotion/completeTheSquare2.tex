\documentclass{ximera}


\graphicspath{
  {./}
  {ximeraTutorial/}
  {basicPhilosophy/}
}

\newcommand{\mooculus}{\textsf{\textbf{MOOC}\textnormal{\textsf{ULUS}}}}

\usepackage{tkz-euclide}\usepackage{tikz}
\usepackage{tikz-cd}
\usetikzlibrary{arrows}
\tikzset{>=stealth,commutative diagrams/.cd,
  arrow style=tikz,diagrams={>=stealth}} %% cool arrow head
\tikzset{shorten <>/.style={ shorten >=#1, shorten <=#1 } } %% allows shorter vectors

\usetikzlibrary{backgrounds} %% for boxes around graphs
\usetikzlibrary{shapes,positioning}  %% Clouds and stars
\usetikzlibrary{matrix} %% for matrix
\usepgfplotslibrary{polar} %% for polar plots
\usepgfplotslibrary{fillbetween} %% to shade area between curves in TikZ
\usetkzobj{all}
\usepackage[makeroom]{cancel} %% for strike outs
%\usepackage{mathtools} %% for pretty underbrace % Breaks Ximera
%\usepackage{multicol}
\usepackage{pgffor} %% required for integral for loops



%% http://tex.stackexchange.com/questions/66490/drawing-a-tikz-arc-specifying-the-center
%% Draws beach ball
\tikzset{pics/carc/.style args={#1:#2:#3}{code={\draw[pic actions] (#1:#3) arc(#1:#2:#3);}}}



\usepackage{array}
\setlength{\extrarowheight}{+.1cm}
\newdimen\digitwidth
\settowidth\digitwidth{9}
\def\divrule#1#2{
\noalign{\moveright#1\digitwidth
\vbox{\hrule width#2\digitwidth}}}






\DeclareMathOperator{\arccot}{arccot}
\DeclareMathOperator{\arcsec}{arcsec}
\DeclareMathOperator{\arccsc}{arccsc}

















%%This is to help with formatting on future title pages.
\newenvironment{sectionOutcomes}{}{}


\title{Quadratic Forms}

\begin{document}

\begin{abstract}
vertex form
\end{abstract}
\maketitle






The previous viewpoint was an algebraic viewpoint.  \\

It presented a step-by-step procedure for applying algebra that would transform a quadratic polynomial in standard form to one in verxtex form.  The procedure gather terms that could be rewritten as a square.  \\

The result of the procedure was an equivalent form, called the vertex form. \\


\[
a (x - h)^2 + k
\]




The procedure looked like this:



\begin{example} \textit{Completing the Square}



Completing the square for $2 \, t^2 + 4 \, t + 6$.


\begin{explanation}

$\blacktriangleright$ Factor out the leading coefficent, $2$.\\

\[     2 \, t^2 + 4 \, t + 6 = 2 \left( \answer{t^2 + 2 \, t + 3} \right)   \] 


Now we have a monic to work with inside the parentheses. \\


$\blacktriangleright$ Let's move inside the parentheses.

\[ t^2 + 2 t + 3 \]

Take half of the linear coefficient, $\frac{2}{2} = 1$, square that $1^2 = 1$, and add and subtract it, so that we have just added $0$ to the expression and not changed its value.


\[ t^2 + 2 t + 1 - 1 +3 \]


Now, group.

\[ (t^2 + 2 t + 1)- 1 +3 \]

the grouped part is a square.

\[ \left( \answer{t+1} \right)^2- 1 +3 \]

\[ (t+1)^2 + 2 \]

Remember, this was inside parentheses.

\[     2 \, t^2 + 4 \, t + 6 = 2 (t^2 + 2 \, t + 3)  = 2 ((t+1)^2 + 2) =  2 (t+1)^2 + \answer{4}\] 


$2 (t+1)^2 + 4$ is the completed square form of $2 \, t^2 + 4 \, t + 6$ \\


$2 (t+1)^2 + 4 = 2 \, t^2 + 4 \, t + 6$

\end{explanation}



\end{example}














The whole point was to change from an expression with two occurences of the variable to an expression with only one occurence of the variable.

This is helpful in graphing and helpful in solving quadratic equations.


However, the procedure isn't the point of all of this.  It is just a procedure to obtain the vertex form.  If we take a funcitonal viewpoint instead of an algebraic viewpoint, we can get there much faster.






\subsection{a Functional Viewpoint}



Oue goal is to convert $a x^2 + b x + c$ into $a (x - h)^2 + k$. \\

Let's think of these as function.


\begin{itemize}
\item $S(x) = a x^2 + b x + c$.
\item $V(x) = a (x - h)^2 + k$.
\end{itemize}



The $a$ is the leading coefficient in both forms.  Therefore, we only need $h$ and $k$ in terms of $a$, $b$, and $c$. \\




\textbf{\textcolor{blue!55!black}{$\blacktriangleright$}}  The extrema \\



\begin{itemize}
\item From the standard form, we know that the vertex's first coordinate is $\frac{-b}{2 a}$.
\item From the vertex form, we know that the vertex's first coordinate is $h$.
\end{itemize}


These must be equal.

\[
h = \frac{-b}{2 a}
\]



If we evaluate each function at $\frac{-b}{2 a}$, we must get the same value.  But, $\frac{-b}{2 a} = h$.  And, $V(h) = a (h - h)^2 + k = k$

Therefore, if we evaluate $S(x) = a x^2 + b x + c$ at $\frac{-b}{2 a}$, we must get $h$.



\[
h = S\left( \frac{-b}{2 a} \right) = a \left( \frac{-b}{2 a} \right)^2 + b \left( \frac{-b}{2 a} \right) + c 
\]







This makes the example above much quicker.





\begin{example} \textit{Vertex Form}



Completing the square for $2 \, t^2 + 4 \, t + 6$.


\begin{explanation}


$S(t) = 2 \, t^2 + 4 \, t + 6$


\[
\frac{-b}{2 a} = \frac{-4}{2 \cdot 2} = -1
\]


$S(-1) = 2 \, (-1)^2 + 4 \, (-1) + 6 = 4$



$a (x - h)^2 + k$ give us $2 (t+1)^2 + 4$

which is what we got through the procedure of completing the square.

\end{explanation}



\end{example}




\begin{center}
\textbf{\textcolor{green!50!black}{ooooo=-=-=-=-=-=-=-=-=-=-=-=-=ooOoo=-=-=-=-=-=-=-=-=-=-=-=-=ooooo}} \\

more examples can be found by following this link\\ \link[More Examples of Quadratics]{https://ximera.osu.edu/csccmathematics/precalculus1/precalculus1/projectileMotion/examples/exampleList}

\end{center}











\end{document}
