\documentclass{ximera}


\graphicspath{
  {./}
  {ximeraTutorial/}
  {basicPhilosophy/}
}

\newcommand{\mooculus}{\textsf{\textbf{MOOC}\textnormal{\textsf{ULUS}}}}

\usepackage{tkz-euclide}\usepackage{tikz}
\usepackage{tikz-cd}
\usetikzlibrary{arrows}
\tikzset{>=stealth,commutative diagrams/.cd,
  arrow style=tikz,diagrams={>=stealth}} %% cool arrow head
\tikzset{shorten <>/.style={ shorten >=#1, shorten <=#1 } } %% allows shorter vectors

\usetikzlibrary{backgrounds} %% for boxes around graphs
\usetikzlibrary{shapes,positioning}  %% Clouds and stars
\usetikzlibrary{matrix} %% for matrix
\usepgfplotslibrary{polar} %% for polar plots
\usepgfplotslibrary{fillbetween} %% to shade area between curves in TikZ
\usetkzobj{all}
\usepackage[makeroom]{cancel} %% for strike outs
%\usepackage{mathtools} %% for pretty underbrace % Breaks Ximera
%\usepackage{multicol}
\usepackage{pgffor} %% required for integral for loops



%% http://tex.stackexchange.com/questions/66490/drawing-a-tikz-arc-specifying-the-center
%% Draws beach ball
\tikzset{pics/carc/.style args={#1:#2:#3}{code={\draw[pic actions] (#1:#3) arc(#1:#2:#3);}}}



\usepackage{array}
\setlength{\extrarowheight}{+.1cm}
\newdimen\digitwidth
\settowidth\digitwidth{9}
\def\divrule#1#2{
\noalign{\moveright#1\digitwidth
\vbox{\hrule width#2\digitwidth}}}






\DeclareMathOperator{\arccot}{arccot}
\DeclareMathOperator{\arcsec}{arcsec}
\DeclareMathOperator{\arccsc}{arccsc}

















%%This is to help with formatting on future title pages.
\newenvironment{sectionOutcomes}{}{}


\title{Parabolas}

\begin{document}

\begin{abstract}
quadratic formula
\end{abstract}
\maketitle



\subsection{Graphs of Quadratic Functions}

We have two types of elementary functions, so far:

\begin{itemize}
\item Linear functions
\item Quadratic functions
\end{itemize}

One of our goals for analysis is to identify zeros of functions. \\

Unless the linear function is actually a constant function (which is a linear function), the linear function has exactly one zero. We can identify this inique zero by setting the formulas equal to $0$ and solving.  This is accomplished by combining like terms and isolating the variable on one side of the equation.

We have seen one approach to identifying zeros of quadratic functions.
\begin{itemize}
\item Completing the Square
\end{itemize}

We have a second approach. \\












\begin{image}
\begin{tikzpicture}
     \begin{axis}[
                domain=-5:10, ymax=10, xmax=10, ymin=-5, xmin=-5,
                axis lines =center, xlabel=$x$, ylabel=$y$,
                ytick={-4,-2,2,4,6,8,10},
                xtick={-4,-2,2,4,6,8,10},
                ticklabel style={font=\scriptsize},
                every axis y label/.style={at=(current axis.above origin),anchor=south},
                every axis x label/.style={at=(current axis.right of origin),anchor=west},
                axis on top,
                ]


        \addplot [draw=penColor, very thick, smooth, domain=(-1:7),<->] {0.5*(x-3)^2 + 2};
        \addplot [line width=1, gray, dashed,samples=100,domain=(-9.5:9.5)] ({3},{x});
        


        \addplot [color=penColor,only marks,mark=*] coordinates{(3,2)};
        \node[penColor] at (axis cs:4,1.5) {$(h, k)$};
        %\node[penColor] at (axis cs:5,-9) {$-0.5 x^2 - 5 x + 15.5$};



    \end{axis}
\end{tikzpicture}
\end{image}























\end{document}
