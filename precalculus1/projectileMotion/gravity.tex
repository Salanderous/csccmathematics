\documentclass{ximera}


\graphicspath{
  {./}
  {ximeraTutorial/}
  {basicPhilosophy/}
}

\newcommand{\mooculus}{\textsf{\textbf{MOOC}\textnormal{\textsf{ULUS}}}}

\usepackage{tkz-euclide}\usepackage{tikz}
\usepackage{tikz-cd}
\usetikzlibrary{arrows}
\tikzset{>=stealth,commutative diagrams/.cd,
  arrow style=tikz,diagrams={>=stealth}} %% cool arrow head
\tikzset{shorten <>/.style={ shorten >=#1, shorten <=#1 } } %% allows shorter vectors

\usetikzlibrary{backgrounds} %% for boxes around graphs
\usetikzlibrary{shapes,positioning}  %% Clouds and stars
\usetikzlibrary{matrix} %% for matrix
\usepgfplotslibrary{polar} %% for polar plots
\usepgfplotslibrary{fillbetween} %% to shade area between curves in TikZ
\usetkzobj{all}
\usepackage[makeroom]{cancel} %% for strike outs
%\usepackage{mathtools} %% for pretty underbrace % Breaks Ximera
%\usepackage{multicol}
\usepackage{pgffor} %% required for integral for loops



%% http://tex.stackexchange.com/questions/66490/drawing-a-tikz-arc-specifying-the-center
%% Draws beach ball
\tikzset{pics/carc/.style args={#1:#2:#3}{code={\draw[pic actions] (#1:#3) arc(#1:#2:#3);}}}



\usepackage{array}
\setlength{\extrarowheight}{+.1cm}
\newdimen\digitwidth
\settowidth\digitwidth{9}
\def\divrule#1#2{
\noalign{\moveright#1\digitwidth
\vbox{\hrule width#2\digitwidth}}}






\DeclareMathOperator{\arccot}{arccot}
\DeclareMathOperator{\arcsec}{arcsec}
\DeclareMathOperator{\arccsc}{arccsc}

















%%This is to help with formatting on future title pages.
\newenvironment{sectionOutcomes}{}{}


\title{Gravity}

\begin{document}

\begin{abstract}
inverse square law
\end{abstract}
\maketitle




Gravity affects the trajectory of every projectile.




\section{Newton}

Newton's law of gravitation is an inverse square law

\[ F_g = \frac{G m_1 m_2}{r^2}    \]




where $G = 6.67 \times 10^{-11} \frac{m^3}{kg \cdot s^2}$ is the gravitational constant. \\


$F_g$ is the gravitational force between point-masses $m_1$ and $m_2$, which are a distance $r$ apart. \\



This tells us that the gravity we feel from the Earth is given by


\[ g = \frac{F_g}{m_1}  = \frac{G m_e}{(r_e)^2}    \]


\[ g = \frac{(6.67 \times 10^{-11} \frac{m^3}{kg \cdot s^2}) \cdot (5.972 \times 10^{24} kg)}{(6356 km)^2} = 9.81 \frac{m}{s^2}   \]


This is the acceleration we feel on Earth, a downward acceleration, $-9.81 \frac{m}{s^2}$.  Compared to the Earth's radius we don't add very much height (or weight), so we can consider $r$ to be constant for most of our investigations.  That means, for most of our investigations, the acceleration due to the Earth can be considered a constant.




$\blacktriangleright$ Gravity is constant acceleration.


\[ a(t) = -9.81 \frac{m}{s^2}  \]



Acceleration is meters per second PER second.  It is a constant rate of change of velocity.  Thus, it is the rate of change of a linear function.




\[ v(t) = v_0 - 9.81 t  \]



Velocity is meters per second. This is the rate of change of distance.  It is a linear function. Thus, it is the rate of change of a quadratic function.




\[ h(t) = s_0 + v_0 t - \frac{9.81}{2} t^2  \]



\begin{itemize}
\item $v_0$ is the initial velocity for a projectile thrown into the air. \\

\item $h_0$ is the initial height for a projectile thrown into the air. \\

\item $h(t)$ is height (vertical distance) as a function of time.  Seconds is the domain unit and meters is the range unit.
\end{itemize}



\section{Projectile Motion}


Actually, all of those quantities are vectors.  

\begin{observation}
\item There is a vertical position, a vertical velocity, and a vertical acceleration
\item There is a horizontal position, a horizontal velocity, and a horizontal acceleration
\end{observation}



Both the vertical, $y$, and horizontal, $x$, components of position are functions of time.

We have already seen this equation for the height of a projectile.
\[ D_y(t) = s_{y_0} + v_{y_0} t - \frac{9.81}{2} t^2  \]

Horizontal motion isn't effected by the Earth's gravity, since gravity is a downward acceleration. Therefore, the horizontal distance is just affected by the initial horizontal velocity.


\[ D_x(t) = s_{x_0} + v_{x_0} t  \]








\section{Parabolas}

Parabolas are geometric curves.  They are defined as the collection of points which are equidistant from a point, called the \textbf{focus}, and a line, called the \textbf{directrix}.  In the diagram below,  we have rotated out viewpoint so that the directrix is the horizontal line $y = c$ and the focus is $(0,c)$.









\begin{image}
\begin{tikzpicture}
     \begin{axis}[
            	domain=-10:10, ymax=10, xmax=10, ymin=-10, xmin=-10,
            	axis lines =center, xlabel=$x$, ylabel=$y$,
            	every axis y label/.style={at=(current axis.above origin),anchor=south},
            	every axis x label/.style={at=(current axis.right of origin),anchor=west},
            	axis on top,
          		]


	
        \addplot[color=penColor2,fill=penColor2,only marks,mark=*] coordinates{(0,3)};
        
        \addplot [draw=penColor, very thick, smooth, domain=(-8:8),<->] {0.125*x^2};
        \addplot [draw=gray, very thick, dashed, domain=(-8:8),<->] {-3};

        \addplot[color=penColor2,fill=penColor2,only marks,mark=*] coordinates{(7,6.125)};
        \addplot[color=penColor2,fill=penColor2,only marks,mark=*] coordinates{(7,-3)};


        




		\node[penColor] at (axis cs:8.5,6.125) {$(x,y)$};
		\node[penColor] at (axis cs:7,-4) {$(x,-c)$};
		\node[penColor] at (axis cs:-2,3) {$(0,c)$};


        \addplot [ultra thick,penColor2] plot coordinates {(7,6.125) (7,-3)};
        \addplot [ultra thick,penColor2] plot coordinates {(7,6.125) (0,3)};
    





    \end{axis}
\end{tikzpicture}
\end{image}



If we select a random point, $(x,y)$, on the parabola then the distance from this point to the focus has to equal the distance to the directrix.


\[  \sqrt{x^2 + (y-c)^2} = y+c   \]


Solving this for $y$ gives


\[  x^2 + (y-c)^2 = (y+c)^2   \]

\[  x^2 + y^2 - 2 y c + c^2 = y^2 + 2 y c + c^2   \]

\[  x^2  - 2 y c  =  2 y c    \]

\[  x^2   =  4 y c    \]


The coordinates of the point on the parabola satisfy a quadratic equation.  Parabolas are the graphs of quadratic equations.












\section{Projectile Trajectory}


We have descriptions of the vertical height and the horizontal distance for a projectile under the influence of gravity.  Let's put those together and get height as a function of distance.  That will coorespond to our normal experience standing on the ground watching a projectile fly.


\begin{itemize}
\item $h_y(t) = s_{y_0} + v_{y_0} t - \frac{9.81}{2} t^2$


\item $d_x(t) = s_{x_0} + v_{x_0} t$
\end{itemize}




First, let's go back to our separate equations and let's just assume we are firing the projectile off the ground.  Then our initial position will be $0$.  

And, let's use the more common symbol $g$, for the acceleration due to gravity, $g = \frac{9.81}{2} \frac{m^2}{s}$, downward.




\begin{itemize}
\item $y = v_{y_0} t - g \, t^2$


\item $x = v_{x_0} t$
\end{itemize}


Solve for $t$ in the horizonal equation.


\[ t = \frac{x}{v_{x_0}} \]

Substitute this into the vertical equation.


\[  y = v_{y_0} \, \frac{x}{v_{x_0}} - g \left(\frac{x}{v_{x_0}}\right)^2  \]



\[  y = \frac{v_{y_0}}{v_{x_0}} \, x  - \frac{g}{(v_{x_0})^2} \, x^2 \]



A quadratic! 

The projectile itself follows a parabola tractory in the air. \\


The projectile's height is described by a quadratic function in time. \\


The projectile's horizontal distance is not affected by gravity, so it is described by a linear function in time. \\









\end{document}
