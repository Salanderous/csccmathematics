\documentclass{ximera}


\graphicspath{
  {./}
  {ximeraTutorial/}
  {basicPhilosophy/}
}

\newcommand{\mooculus}{\textsf{\textbf{MOOC}\textnormal{\textsf{ULUS}}}}

\usepackage{tkz-euclide}\usepackage{tikz}
\usepackage{tikz-cd}
\usetikzlibrary{arrows}
\tikzset{>=stealth,commutative diagrams/.cd,
  arrow style=tikz,diagrams={>=stealth}} %% cool arrow head
\tikzset{shorten <>/.style={ shorten >=#1, shorten <=#1 } } %% allows shorter vectors

\usetikzlibrary{backgrounds} %% for boxes around graphs
\usetikzlibrary{shapes,positioning}  %% Clouds and stars
\usetikzlibrary{matrix} %% for matrix
\usepgfplotslibrary{polar} %% for polar plots
\usepgfplotslibrary{fillbetween} %% to shade area between curves in TikZ
\usetkzobj{all}
\usepackage[makeroom]{cancel} %% for strike outs
%\usepackage{mathtools} %% for pretty underbrace % Breaks Ximera
%\usepackage{multicol}
\usepackage{pgffor} %% required for integral for loops



%% http://tex.stackexchange.com/questions/66490/drawing-a-tikz-arc-specifying-the-center
%% Draws beach ball
\tikzset{pics/carc/.style args={#1:#2:#3}{code={\draw[pic actions] (#1:#3) arc(#1:#2:#3);}}}



\usepackage{array}
\setlength{\extrarowheight}{+.1cm}
\newdimen\digitwidth
\settowidth\digitwidth{9}
\def\divrule#1#2{
\noalign{\moveright#1\digitwidth
\vbox{\hrule width#2\digitwidth}}}






\DeclareMathOperator{\arccot}{arccot}
\DeclareMathOperator{\arcsec}{arcsec}
\DeclareMathOperator{\arccsc}{arccsc}

















%%This is to help with formatting on future title pages.
\newenvironment{sectionOutcomes}{}{}


\title{Quadratic Forms}

\begin{document}

\begin{abstract}
complete the square
\end{abstract}
\maketitle



We have seen that the trajectories of projectiles are described by quadratic equations.  In fact, quadratic equations and formulas are used everywhere.


\begin{itemize}
\item \link[101 Uses of a Quadratic Equation Part 1]{https://plus.maths.org/content/101-uses-quadratic-equation}
\item \link[101 Uses of a Quadratic Equation Part 2]{https://plus.maths.org/content/101-uses-quadratic-equation-part-ii}
\end{itemize}







\section{Quadratic Functions}


\begin{definition}
\textbf{Quadratic functions} are functions which can be described with a formula of the form

\[  Q(t) = a \, t^2 + b \, t + c  \]

where $a$, $b$, and $c$ are real numbers with $a \ne 0$



\begin{itemize}
\item $a$ is called the \textbf{leading coefficient} 
\item $a \, t^2$ is called the \textbf{leading term} 
\item $b \, t$ is called the \textbf{linear term} 
\item $c$ is called the \textbf{constant term} 
\end{itemize}

\end{definition}




As we have seen the graph of a quadratic equation is a parabola. The sign of the leading coefficient determines if the parabola opens up or down.





\begin{image}
\begin{tikzpicture}
     \begin{axis}[
            	domain=-10:10, ymax=10, xmax=10, ymin=-10, xmin=-10,
            	axis lines =center, xlabel=$x$, ylabel=$y$,
            	every axis y label/.style={at=(current axis.above origin),anchor=south},
            	every axis x label/.style={at=(current axis.right of origin),anchor=west},
            	axis on top,
          		]


        \addplot [draw=penColor, very thick, smooth, domain=(-7:-1),<->] {0.5*(x+4)^2 -2};
        \addplot [draw=penColor2, very thick, smooth, domain=(1:9),<->] {-0.5*(x-5)^2 +3};

        

		\node[penColor] at (axis cs:-6.5,6.125) {$0.5 x^2 + 4 x + 6$};
		\node[penColor] at (axis cs:5,-9) {$-0.5 x^2 - 5 x + 15.5$};



    \end{axis}
\end{tikzpicture}
\end{image}


\section{Analysis}

What do we want to know when we analyze functions?

We want to know the domain, range, zeros, intervals of increasing and decreasing, global maximum and minimum, local maximums and minimums, discontinuities, singularities, symmetry, endbehavior, and a nice graph.



\textbf{Domain and Range} \\

The implied domain of a quadratic function is all real numbers.  Any quadratic funciton could come with a stated domain that restricted this.

From the graphs above we can see that the implied range of a quadratic comes in two types.  

\begin{itemize}
\item The range could be all real numbers greater than or equal to some particular.
\item The range could be all real numbers less than or equal to some particular.
\end{itemize}

If there is a stated domain, then the range will be restricted similarly.




\textbf{Zeros} \\

Zero has a unique property appropriately named as the \textbf{zero product property}.  



\begin{fact} \textit {Zero Product Property}

If $a$ and $b$ are real numbers and $a\cdot b = 0$, then either $a=0$ or $b=0$.

\end{fact}


Zero is the ONLY number with such an identifiable test, therefore, we use it...A LOT! 


Unless we happen to know something special, when solving equations our general approach is to set everything equal tp zero and then chnage the expression into a product.  This process is called \textbf{factoring}.  The whole idea is to make our situation look like the zero product property.

Thus, it comes as no surprise that identifying zeros of functions is a top priority.  Factoring is a top priority.

As we can see from the graph below, quadratics can have $0$, $1$, or $2$ real zeros.  How do we find them?







\begin{image}
\begin{tikzpicture}
     \begin{axis}[
            	domain=-10:10, ymax=10, xmax=10, ymin=-10, xmin=-10,
            	axis lines =center, xlabel=$x$, ylabel=$y$,
            	every axis y label/.style={at=(current axis.above origin),anchor=south},
            	every axis x label/.style={at=(current axis.right of origin),anchor=west},
            	axis on top,
          		]


        \addplot [draw=penColor, very thick, smooth, domain=(-8:-4),<->] {2*(x+6)^2 + 2};
        \addplot [draw=penColor2, very thick, smooth, domain=(-1:3),<->] {2*(x-1)^2 };
        \addplot [draw=penColor4, very thick, smooth, domain=(5:9),<->] {2*(x-7)^2 - 3};
        

		%\node[penColor] at (axis cs:-6.5,6.125) {$0.5 x^2 + 4 x + 6$};
		%\node[penColor] at (axis cs:5,-9) {$-0.5 x^2 - 5 x + 15.5$};



    \end{axis}
\end{tikzpicture}
\end{image}



\section{Completing the Square}

Zeros of linear functions were easy to identify.  Simply apply a little algebra to the equation to get the variable by itself on one side of the equation.  But, this is not always possible with equations of the form $0 = a \, t^2 + b \, t + c $, because there are two occurrences of the variable and they are to different degrees. They cannot be combined to leave just one occurrence.

Therefore, an idea is to rearrange the quadratic expression so that there is only one occurrence of hte varible.  This process is called \textbf{completing the square}.

The idea is to rearrange  $0 = a \, t^2 + b \, t + c$ to look like $0 = A \, (t-B)^2 + C$.  Then there will only be  one occurrence of the $t$ and we can solve for it.


To accomplish this, we need to investigate $A \, (t-B)^2 + C$. For the moment, let's multiply it back out.


\begin{align*}
A \, (t-B)^2 + C & = A \, (t-B)(t-B) + C \\
& = A \, (t^2 - 2 t B + B^2) + C  \\
& = A \, t^2 - 2 t A B + A \, B^2 + C
\end{align*}

This last line $A \, t^2 - 2 t A B + A \, B^2 + C$ should turn out to be $a \, t^2 + b \, t + c$, our original quadratic.

The only way that is going to happen is if $A = a$. As long as we know $A = a$, then let's start over.

\textbf{Step 1}  Factor out the leading coefficent.


Starting over...starting with a leading coefficient of $1$...



We want to complete the square on $t^2 + b \, t + c$.  When the leading coefficent is $1$, then we call this quadratic \textbf{monic}.



\begin{align*}
(t-B)^2 + C & = (t-B)(t-B) + C \\
& = t^2 - 2 t B + B^2 + C  \\
& = t^2 - 2 t B +  B^2 + C
\end{align*}


This last line $t^2 - 2 t B + B^2 + C$ should turn out to be $t^2 + b \, t + c$.



The leading term is $1$ in both. Now, the linear terms need to be the same: $ - 2 t B = b \, t$.  That tells us that $B = \frac{b}{-2}$.


To complete the square with a monic quadratic, we need $B = \frac{b}{-2}$.  Let's put that in.


$t^2 - 2 t B + B^2 + C = t^2 - 2 t \frac{b}{-2} + \left(\frac{b}{-2}\right)^2 + C = t^2 + b t + \left(\frac{b}{-2}\right)^2 + C$


We want this to be $t^2 + b \, t + c$. \\

leading terms match...check \\
linear terms match...check \\

Now there is a mess for the constant term.  We have $\left(\frac{b}{-2}\right)^2 + C$, when we just wanted $c$.  Then let's just pick $C$ to be $-\left(\frac{b}{-2}\right)^2 + c = -\left(\frac{b}{2}\right)^2 + c$



Whew!

Let's see some examples








\begin{procedure} \textit{Completing the Square}



Completing the square on $2 \, t^2 + 4 \, t + 6$.


\textbf{Step 1} Factor out the leading coefficent, $2$. \\

\[     2 \, t^2 + 4 \, t + 6 = 2 (t^2 + 2 \, t + 3)   \] 


Now we have a monic to work with inside the parentheses. Let's move inside the parentheses.

\[ t^2 + 2 t + 3 \]

Take half of the linear coefficient, $\frac{2}{2} = 1$, squre that $1^2 = 1$, and add and subtract it, so thatwe have just added $0$ to the expression and nbt changed its value.


\[ t^2 + 2 t + 1 - 1 +3 \]


Now, group.

\[ (t^2 + 2 t + 1)- 1 +3 \]

the grouped part is a square.

\[ (t+1)^2- 1 +3 \]

\[ (t+1)^2 + 2 \]

Remember, this was inside parentheses.

\[     2 \, t^2 + 4 \, t + 6 = 2 (t^2 + 2 \, t + 3)  = 2 ((t+1)^2 + 2) =  2 (t+1)^2 + 4\] 


$2 (t+1)^2 + 4$ is the completed square form of $2 \, t^2 + 4 \, t + 6$


\end{procedure}



\begin{example} \textit{Two Real Solutions}

Solve $4  t^2 - 4 t - 8 = 0$ \\


First complete the square.



\[ 4 t^2 -  4 t - 8 = 0 \]

\[ 4 (t^2 - t - 2) = 0 \]

\[ 4 (t^2 - t + \frac{1}{4} - \frac{1}{4} - 2) = 0 \]


\[ 4 \left(\left(t - \frac{1}{2}\right)^2 - \frac{1}{4} - 2\right) = 0 \]


\[ 4 \left(\left(t - \frac{1}{2}\right)^2 - \frac{1}{4} - 2\right) = 0 \]

\[ 4 \left(t - \frac{1}{2}\right)^2 - 1 - 8 = 0 \]

\[ 4 \left(t - \frac{1}{2}\right)^2 - 9 = 0 \]


One occurrence of $t$, good. Now to get $t$ by itself.


\[ 4 \left(t - \frac{1}{2}\right)^2 = 9  \]

\[  \left(t - \frac{1}{2}\right)^2 = \frac{9}{4}  \]

The only way this can happen is if \\



either   $t - \frac{1}{2} = \frac{3}{2}$  or  $t - \frac{1}{2} = -\frac{3}{2}$ \\

The first choice gives us $t = 2$ and the second choice gives us $t = -1$



Let's check those solutions.

\begin{itemize}
\item $4 (2)^2 - 4 (2) - 8 = 0$ ... check
\item $4 (-1)^2 - 4 (-1) - 8 = 0$ ... check
\end{itemize}



We have already seen that a quadratic equation can have at most two solutions.  So, we must have all of the solutions.



\end{example}









\begin{example} \textit{One Real Solution}

Solve $2 x^2 - 12x + 21 = 3$ \\


First, get everything to one side and $0$ on the other side.



\[  2 x^2 - 12x + 18 = 0  \]

Factor out $a$,

\[  2 (x^2 - 6x + 9) = 0  \]


Square half of $b$, $\left(\frac{-6}{2}\right) = (-3)^2 = 9$.  But, $9$ is alrerady there.  This must already be a square.



\[  2 (x - 3)^2 = 0  \]


Now, get $x$ by itself.

\[  (x - 3)^2 = 0  \]


Unlike $\frac{9}{4}$ in the previous example, the only number you can square and get $0$ is $0$.

\[  x - 3 = 0  \]

\[  x = 3  \]


This equation only has one solution.



\end{example}




There is another viewpoint on the example above.  We arrived at this equation

\[  (x - 3)^2 = 0  \]

which could be viewed as 

\[  (x - 3) (x - 3) = 0  \]


If we proceed with the zero product property we would create two new equaitons.  One for each factor.

Either $x - 3 = 0$   or $x - 3 = 0$. \\

Either $x = 3$ or $x = 3$.  \\

$3$ is the only solution to the equation, but the equation has this solutoin twice.


The previous two examples illustrated that quadratic equations can have two or one solutions.  The next example illustrates that a quadratic equation can have no real solutions.










\begin{example} \textit{No Real Solutions}

Solve $2 x^2 - 12x + 21 = 1$ \\


First, get everything to one side and $0$ on the other side.



\[  2 x^2 - 12x + 20 = 0  \]

Factor out $a$,

\[  2 (x^2 - 6x + 10) = 0  \]


Square half of $b$, $\left(\frac{-6}{2}\right) = (-3)^2 = 9$.  Change the $10$ to be $9+1$.



\[  2 (x^2 - 6x + 9 + 1) = 0  \]


\[  2 ((x-3)^2 + 1) = 0  \]


\[  2 (x-3)^2 + 2 = 0  \]

$2 (x-3)^2)$ is never negative and $2$ is positive.  They cannot add to $0$. \\


This equation has no real solutions.



\end{example}



The above example illustrates that there are numbers missing.  It feels like that equation should have a solution.  The real numbers are not enough for our equations.  Eventually, we will fill in the missing pieces with the \textbf{Complex Numbers}.  But, for now we note that there are no real solutions.




























\end{document}
