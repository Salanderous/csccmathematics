\documentclass{ximera}


\graphicspath{
  {./}
  {ximeraTutorial/}
  {basicPhilosophy/}
}

\newcommand{\mooculus}{\textsf{\textbf{MOOC}\textnormal{\textsf{ULUS}}}}

\usepackage{tkz-euclide}\usepackage{tikz}
\usepackage{tikz-cd}
\usetikzlibrary{arrows}
\tikzset{>=stealth,commutative diagrams/.cd,
  arrow style=tikz,diagrams={>=stealth}} %% cool arrow head
\tikzset{shorten <>/.style={ shorten >=#1, shorten <=#1 } } %% allows shorter vectors

\usetikzlibrary{backgrounds} %% for boxes around graphs
\usetikzlibrary{shapes,positioning}  %% Clouds and stars
\usetikzlibrary{matrix} %% for matrix
\usepgfplotslibrary{polar} %% for polar plots
\usepgfplotslibrary{fillbetween} %% to shade area between curves in TikZ
\usetkzobj{all}
\usepackage[makeroom]{cancel} %% for strike outs
%\usepackage{mathtools} %% for pretty underbrace % Breaks Ximera
%\usepackage{multicol}
\usepackage{pgffor} %% required for integral for loops



%% http://tex.stackexchange.com/questions/66490/drawing-a-tikz-arc-specifying-the-center
%% Draws beach ball
\tikzset{pics/carc/.style args={#1:#2:#3}{code={\draw[pic actions] (#1:#3) arc(#1:#2:#3);}}}



\usepackage{array}
\setlength{\extrarowheight}{+.1cm}
\newdimen\digitwidth
\settowidth\digitwidth{9}
\def\divrule#1#2{
\noalign{\moveright#1\digitwidth
\vbox{\hrule width#2\digitwidth}}}






\DeclareMathOperator{\arccot}{arccot}
\DeclareMathOperator{\arcsec}{arcsec}
\DeclareMathOperator{\arccsc}{arccsc}

















%%This is to help with formatting on future title pages.
\newenvironment{sectionOutcomes}{}{}


\title{Quadratic Products}

\begin{document}

\begin{abstract}
factoring
\end{abstract}
\maketitle







\subsection{Elementary Functions}

We have two types of elementary functions, so far:

\begin{itemize}
\item Linear functions
\item Quadratic functions
\end{itemize}

One of our goals for analysis is to identify zeros of functions. \\

Unless the linear function is actually a constant function (which is a linear function), the linear function has exactly one zero. We can identify this unique zero by setting the formulas equal to $0$ and solving.  This is accomplished by combining like terms and isolating the variable on one side of the equation.

We have seen two approaches to identifying zeros of quadratic functions.
\begin{itemize}
\item Completing the Square
\item {The Quadratic Formula}
\end{itemize}

We have a third approach. \\









We have seen that there are at most two real solutions to $a \, t^2 + b \, t + c = 0$, with $a \ne 0$. 

Let's look at this from a function viewpoint.  The quadratic function  $Q(t) = a \, t^2 + b \, t + c$ has at most two real zeros.  

The quadratic formula gives explicit expressions for these roots.


\[   \frac{-b + \sqrt{b^2 - 4 a c}}{2a}     \text{ and }    \frac{-b - \sqrt{b^2 - 4 a c}}{2a}   \]





$\blacktriangleright$  Are there other quadratic equations, besides $a \, t^2 + b \, t + c = 0$, that have these two solutions?















Perhaps, there are other quadratic functions, which have these two zeros. Let's create a quadratic function from these two zeros.



\[ f(t) =  \left(t - \frac{-b + \sqrt{b^2 - 4 a c}}{2a}\right)   \left(t -  \frac{-b - \sqrt{b^2 - 4 a c}}{2a}\right)   \]


If we multiply this out, we get



\[ f(t) =   t^2 \, - \, \frac{-b + \sqrt{b^2 - 4 a c}}{2a} t \, - \, \frac{-b - \sqrt{b^2 - 4 a c}}{2a}  t \, + \, \left(\frac{-b + \sqrt{b^2 - 4 a c}}{2a}\right) \left(\frac{-b - \sqrt{b^2 - 4 a c}}{2a}\right) \]


\[ f(t) = t^2  \, - \, 2 \cdot \frac{-b}{2a} + \left(    \frac{b^2 - (b^2 - 4 a c)}{4 a^2}     \right)        \]


\[ f(t) = t^2  \, + \frac{b}{a} + \left(    \frac{4 a c}{4 a^2}     \right)        \]

\[ f(t) = t^2  \, +  \frac{b}{a} + \left(    \frac{c}{a}     \right)        \]


This quadratic function has the same two zeros as $Q(t)$.  


\textbf{Note:} If we multiply this by $a$, we get 


\[ a \, f(t) = a \, t^2 + b \, t + c \]

which is $Q(t)$.


$\blacktriangleright$ If two quadratics have the same two roots, then they must be multiples of each other.






We have discovered a new expression or formula or form for our quadratic functions.





\[ f(t) =  \left(t - \frac{-b + \sqrt{b^2 - 4 a c}}{2a}\right)   \left(t -  \frac{-b - \sqrt{b^2 - 4 a c}}{2a}\right)   \]







\section{Factored Form}

\textbf{\textcolor{red!90!darkgray}{$\blacktriangleright$ }} Every quadratic function can be written as a sum, $Q(t) = a \, t^2 + b \, t + c$, with $a \ne 0$. \\



\textbf{\textcolor{red!90!darkgray}{$\blacktriangleright$ }} Every quadratic function can be written with one occurrence of the variable via a square, $Q(t) = a (t - h)^2 + k$, with $a \ne 0$. \\



\textbf{\textcolor{red!90!darkgray}{$\blacktriangleright$ }} Every quadratic function can be written as a product, $Q(t) = a (t - r_1)(t - r_2)$, where $r_1$ and $r_2$ are the real zeros and $a \ne 0$ - provided teh quadratic funciton has two real zeros. \\





The real zeros can be obtained via the quadratic formula - provided the discrimanant is positve.  If the discriminant is $0$, then $r_1 = r_2$ and we get a square.  If the discriminant is negative, then $Q(t)$ doesn't factor with real numbers.  


The process of writing the function as a product is called \textbf{factoring}. $(t - r_1)$  and $(t - r_2)$ are the \textbf{factors}.



\begin{example} \textit{Two Real Solutions} 

Factor $Q(t) = 4 t^2 - 4 t - 8$ 

\begin{explanation}

The leading coefficient is $a=4$ and the quadratic formula gives us $2$ and $-1$ as zeros.  

That gives us a \textbf{factorization} of $Q(t)$:



\[    Q(t) = 4 t^2 - 4 t - 8 =  4 (t-2)(t-(-1))    = 4 (t-2)(t+1)         \]

\end{explanation}

\end{example}



The quadratic formula always works.  But, many times it is slow with lots of steps and reducing.  Sometimes it is just easier to guess the factors.




















\section{Guessing Factors}


Let's begin with a quadratic function written as a sum: $g(x) = a \, x^2 + b \, x + c$, a.k.a \textbf{standard form}.  


\textbf{\textcolor{blue!75!black}{Step 1:}} If there are any common numeric factors among the three coefficients then factor them out.


\textbf{\textcolor{blue!75!black}{Step 2:}} We are looking for a factorization $g(x) = a \, x^2 + b \, x + c = (A \, x + B)(C \, x + D)$.

for this to happen, we need

\begin{itemize}
\item $a = A \cdot C $
\item $b = A\cdot D + B \cdot C$
\item $c = B \cdot D$
\end{itemize}

Therefore, look for pairs of factors of $a$ and pairs of factors of $c$.






\textbf{\textcolor{blue!75!black}{Step 3:}} Step through your pairs of factors and look for $b = A \cdot D + B \cdot C$.










\begin{example} \textit{Factoring}

Factor $k(x) = 2 x^2 - 5 x - 12$.


\begin{explanation}

First, the coefficients, $2$, $-5$, and $-21$, have no common factors.


Second, $2$ is a prime number, therefore, it's factors are $2 \cdot 1$ or $-2 \cdot -1$.

From this, we know there are only two possibilities. The factorization looks like $(2 x \, + \, ?) (x \, + \, ?)$ or $(-2 x \, + \, ?) (-x \, + \, ?)$.






$-12$ factors as 
\begin{itemize}
\item $-12 \cdot \answer{1}$
\item $12 \cdot \answer{-1}$
\item $-6 \cdot \answer{2}$
\item $6 \cdot \answer{-2}$
\item $-4 \cdot \answer{3}$
\item $4 \cdot \answer{-3}$  
\end{itemize}



Let's start stepping through our list.

Since $-5 x$ seems far away from $12$, let's skip $12$ in our list and begin stepping through our list with $-6$ and $2$.

\begin{itemize}

\item $-12 = -6 \cdot 2$ gives $(2 x + (-6)) (x + 2) = \answer{2 x^2 - 2 x - 12}$
\item $-12 = 6 \cdot -2$ gives $(2 x + 6) (x - 2) = \answer{2 x^2 + 2 x -12}$
\item $-12 = -3 \cdot 4$ gives $(2 x + (-3)) (x + 4) = \answer{2 x^2 + 5 x - 12}$
\item $-12 = 3 \cdot -4$ gives $(2 x + 3) (x + (-4)) = \answer{2 x^2 - 5 x -12}$
\end{itemize}

$2 x^2 - 5 x - 12$ factors as $(2 x + 3) (x + (-4)) = (2 x + 3)(x - 4)$


\end{explanation}
\end{example}

As you gain more experience, this process becomes a mental process.






\begin{example} \textit{Factoring}

Factor $p(y) = 6 y^2 - 16 y - 22$.

\begin{explanation}

First, the coefficients have a common factor.  Factor out $2$ to get $p(y) = 2 \left( \answer{3 y^2 - 8 y - 11} \right)$.


Now factor $3 y^2 - 8 y - 11$.



$3$ is a prime number, therefore, it's factors are $3 \cdot 1$ or $-3 \cdot -1$. However, $-3 \cdot -1$ is actually redundant.  If we picked $-3 \cdot -1$, then we could just factor out $-1$ from both and get back to $3 \cdot 1$.


$3 y^2 - 8 y - 11 = (3y \, \pm \, ?) (y \, \pm \, ?) $


$11$ is also a prime number. We only need to think about $11$ and $1$. And, since we need $-11$, then we know to pick opposite signs for $11$ and $1$.



\begin{itemize}
\item $(3y + 11) (y - 1) = \answer{3 y^2 + 8 y - 11}$
\end{itemize}

Everything is correct except the sign of $-8$. That tells us we are almost correct. We just need to reverse the signs.


\begin{itemize}
\item $(3y - 11) (y + 1) = 3 y^2 - 8 y - 11$
\end{itemize}




\end{explanation}
\end{example}



























\begin{center}
\textbf{\textcolor{green!50!black}{ooooo=-=-=-=-=-=-=-=-=-=-=-=-=ooOoo=-=-=-=-=-=-=-=-=-=-=-=-=ooooo}} \\

more examples can be found by following this link\\ \link[More Examples of Projectile Motion]{https://ximera.osu.edu/csccmathematics/precalculus1/precalculus1/projectileMotion/examples/exampleList}

\end{center}












\end{document}
