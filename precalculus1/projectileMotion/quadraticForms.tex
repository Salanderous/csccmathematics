\documentclass{ximera}


\graphicspath{
  {./}
  {ximeraTutorial/}
  {basicPhilosophy/}
}

\newcommand{\mooculus}{\textsf{\textbf{MOOC}\textnormal{\textsf{ULUS}}}}

\usepackage{tkz-euclide}\usepackage{tikz}
\usepackage{tikz-cd}
\usetikzlibrary{arrows}
\tikzset{>=stealth,commutative diagrams/.cd,
  arrow style=tikz,diagrams={>=stealth}} %% cool arrow head
\tikzset{shorten <>/.style={ shorten >=#1, shorten <=#1 } } %% allows shorter vectors

\usetikzlibrary{backgrounds} %% for boxes around graphs
\usetikzlibrary{shapes,positioning}  %% Clouds and stars
\usetikzlibrary{matrix} %% for matrix
\usepgfplotslibrary{polar} %% for polar plots
\usepgfplotslibrary{fillbetween} %% to shade area between curves in TikZ
\usetkzobj{all}
\usepackage[makeroom]{cancel} %% for strike outs
%\usepackage{mathtools} %% for pretty underbrace % Breaks Ximera
%\usepackage{multicol}
\usepackage{pgffor} %% required for integral for loops



%% http://tex.stackexchange.com/questions/66490/drawing-a-tikz-arc-specifying-the-center
%% Draws beach ball
\tikzset{pics/carc/.style args={#1:#2:#3}{code={\draw[pic actions] (#1:#3) arc(#1:#2:#3);}}}



\usepackage{array}
\setlength{\extrarowheight}{+.1cm}
\newdimen\digitwidth
\settowidth\digitwidth{9}
\def\divrule#1#2{
\noalign{\moveright#1\digitwidth
\vbox{\hrule width#2\digitwidth}}}






\DeclareMathOperator{\arccot}{arccot}
\DeclareMathOperator{\arcsec}{arcsec}
\DeclareMathOperator{\arccsc}{arccsc}

















%%This is to help with formatting on future title pages.
\newenvironment{sectionOutcomes}{}{}


\title{Quadratic Forms}

\begin{document}

\begin{abstract}
zeros
\end{abstract}
\maketitle



We have seen that the trajectories of projectiles are described by quadratic equations.  In fact, quadratic equations and formulas are used everywhere.


\begin{itemize}
\item \link[101 Uses of a Quadratic Equation Part 1]{https://plus.maths.org/content/101-uses-quadratic-equation}
\item \link[101 Uses of a Quadratic Equation Part 2]{https://plus.maths.org/content/101-uses-quadratic-equation-part-ii}
\end{itemize}







\section{Quadratic Functions}


\begin{definition}
\textbf{Quadratic functions} are functions which can be described with a formula of the form

\[  Q(t) = a \, t^2 + b \, t + c  \]

where $a$, $b$, and $c$ are real numbers with $a \ne 0$



\begin{itemize}
\item $a$ is called the \textbf{leading coefficient} 
\item $a \, t^2$ is called the \textbf{leading term} 
\item $b \, t$ is called the \textbf{linear term} 
\item $c$ is called the \textbf{constant term} 
\end{itemize}

\end{definition}




As we have seen the graph of a quadratic equation is a parabola. The sign of the leading coefficient determines if the parabola opens up or down.





\begin{image}
\begin{tikzpicture}
     \begin{axis}[
            	domain=-10:10, ymax=10, xmax=10, ymin=-10, xmin=-10,
            	axis lines =center, xlabel=$x$, ylabel=$y$,
            	every axis y label/.style={at=(current axis.above origin),anchor=south},
            	every axis x label/.style={at=(current axis.right of origin),anchor=west},
            	axis on top,
          		]


        \addplot [draw=penColor, very thick, smooth, domain=(-7:-1),<->] {0.5*(x+4)^2 -2};
        \addplot [draw=penColor2, very thick, smooth, domain=(1:9),<->] {-0.5*(x-5)^2 +3};

        

		\node[penColor] at (axis cs:-6.5,6.125) {$0.5 x^2 + 4 x + 6$};
		\node[penColor] at (axis cs:5,-9) {$-0.5 x^2 - 5 x + 15.5$};



    \end{axis}
\end{tikzpicture}
\end{image}


\section{Analysis}

What do we want to know when we analyze functions?

We want to know the domain, range, zeros, intervals of increasing and decreasing, global maximum and minimum, local maximums and minimums, discontinuities, singularities, symmetry, endbehavior, and a nice graph.



\textbf{Domain and Range} \\

The implied domain of a quadratic function is all real numbers.  Any quadratic funciton could come with a stated domain that restricted this.

From the graphs above we can see that the implied range of a quadratic comes in two types.  

\begin{itemize}
\item The range could be all real numbers greater than or equal to some particular.
\item The range could be all real numbers less than or equal to some particular.
\end{itemize}

If there is a stated domain, then the range will be restricted similarly.




\textbf{Zeros} \\





\begin{image}
\begin{tikzpicture}
     \begin{axis}[
            	domain=-10:10, ymax=10, xmax=10, ymin=-10, xmin=-10,
            	axis lines =center, xlabel=$x$, ylabel=$y$,
            	every axis y label/.style={at=(current axis.above origin),anchor=south},
            	every axis x label/.style={at=(current axis.right of origin),anchor=west},
            	axis on top,
          		]


        \addplot [draw=penColor, very thick, smooth, domain=(-8:-4),<->] {2*(x+6)^2 + 2};
        \addplot [draw=penColor2, very thick, smooth, domain=(-1:3),<->] {2*(x-1)^2 };
        \addplot [draw=penColor4, very thick, smooth, domain=(5:9),<->] {2*(x-7)^2 - 3};
        

		%\node[penColor] at (axis cs:-6.5,6.125) {$0.5 x^2 + 4 x + 6$};
		%\node[penColor] at (axis cs:5,-9) {$-0.5 x^2 - 5 x + 15.5$};



    \end{axis}
\end{tikzpicture}
\end{image}
























\end{document}
