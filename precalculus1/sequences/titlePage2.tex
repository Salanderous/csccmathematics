\documentclass{ximera}


\graphicspath{
  {./}
  {ximeraTutorial/}
  {basicPhilosophy/}
}

\newcommand{\mooculus}{\textsf{\textbf{MOOC}\textnormal{\textsf{ULUS}}}}

\usepackage{tkz-euclide}\usepackage{tikz}
\usepackage{tikz-cd}
\usetikzlibrary{arrows}
\tikzset{>=stealth,commutative diagrams/.cd,
  arrow style=tikz,diagrams={>=stealth}} %% cool arrow head
\tikzset{shorten <>/.style={ shorten >=#1, shorten <=#1 } } %% allows shorter vectors

\usetikzlibrary{backgrounds} %% for boxes around graphs
\usetikzlibrary{shapes,positioning}  %% Clouds and stars
\usetikzlibrary{matrix} %% for matrix
\usepgfplotslibrary{polar} %% for polar plots
\usepgfplotslibrary{fillbetween} %% to shade area between curves in TikZ
\usetkzobj{all}
\usepackage[makeroom]{cancel} %% for strike outs
%\usepackage{mathtools} %% for pretty underbrace % Breaks Ximera
%\usepackage{multicol}
\usepackage{pgffor} %% required for integral for loops



%% http://tex.stackexchange.com/questions/66490/drawing-a-tikz-arc-specifying-the-center
%% Draws beach ball
\tikzset{pics/carc/.style args={#1:#2:#3}{code={\draw[pic actions] (#1:#3) arc(#1:#2:#3);}}}



\usepackage{array}
\setlength{\extrarowheight}{+.1cm}
\newdimen\digitwidth
\settowidth\digitwidth{9}
\def\divrule#1#2{
\noalign{\moveright#1\digitwidth
\vbox{\hrule width#2\digitwidth}}}






\DeclareMathOperator{\arccot}{arccot}
\DeclareMathOperator{\arcsec}{arcsec}
\DeclareMathOperator{\arccsc}{arccsc}

















%%This is to help with formatting on future title pages.
\newenvironment{sectionOutcomes}{}{}


\title{Sequences}

\begin{document}

\begin{abstract}
%Stuff can go here later if we want!
\end{abstract}

\maketitle





The elementary functions are but a drop in the bucket of the collection of all functions.  We just don't have formulas for most functions.  Therefore, we invent new ways of describing them. \\


One method of describing a function is to mimic how we use decimal approximations to describe irrational numbers. \\


For example $\sqrt{17}$. \\

Follow this list of decimal numbers:


\begin{itemize}
\item $4.1$
\item $4.12$
\item $4.123$
\item $4.1231$
\item $4.12310$
\item $4.123105$
\item $4.1231056$
\item $4.12310562$
\item $4.123105626$
\end{itemize}


This lists keeps going.  There is no end to the list.  $\sqrt{17}$ is the distinct number that this list is getting closer and closer to. \\



We could help our analysis of this list by numbering the items:

\begin{itemize}
\item 1) $4.1$
\item 2) $4.12$
\item 3) $4.123$
\item 4) $4.1231$
\item 5) $4.12310$
\item 6) $4.123105$
\item 7) $4.1231056$
\item 8) $4.12310562$
\item 9) $4.123105626$
\end{itemize}





This suggest a different type of description, a function description where the domain is the natural numbers.



\begin{itemize}
\item $f(1) = 4.1$
\item $f(2) = 4.12$
\item $f(3) = 4.123$
\item $f(4) = 4.1231$
\item $f(5) = 4.12310$
\item $f(6) = 4.123105$
\item $f(7) = 4.1231056$
\item $f(8) = 4.12310562$
\item $f(9) = 4.123105626$
\end{itemize}



There are actually logical reasons to switch the domain to the whole numbers and include $0$.





\begin{itemize}
\item $f(0) = 4.1$
\item $f(1) = 4.12$
\item $f(2) = 4.123$
\item $f(3) = 4.1231$
\item $f(4) = 4.12310$
\item $f(5) = 4.123105$
\item $f(6) = 4.1231056$
\item $f(7) = 4.12310562$
\item $f(8) = 4.123105626$
\end{itemize}




There are actually logical reasons to switch the domain to an interval of natural numbers.





\begin{itemize}
\item $f(6) = 4.1$
\item $f(7) = 4.12$
\item $f(8) = 4.123$
\item $f(9) = 4.1231$
\item $f(10) = 4.12310$
\item $f(11) = 4.123105$
\item $f(12) = 4.1231056$
\item $f(13) = 4.12310562$
\item $f(14) = 4.123105626$
\end{itemize}








These types of functions are called \textbf{\textcolor{purple!85!blue}{sequences}}. \\







\subsection{Learning Outcomes}


\begin{sectionOutcomes}

In this section, students will

\begin{itemize}
\item define a sequence.
\item write the first several terms of a sequence using an explicit formula.
\item write the first several terms of a sequence using a recurrence relation.
\item find an explicit formula for a sequence given recursively.
\item find a recurrence relation for a sequence given explicitly.
\end{itemize}

\end{sectionOutcomes}















\begin{center}
\textbf{\textcolor{green!50!black}{ooooo=-=-=-=-=-=-=-=-=-=-=-=-=ooOoo=-=-=-=-=-=-=-=-=-=-=-=-=ooooo}} \\

more examples can be found by following this link\\ \link[More Examples of Sequences]{https://ximera.osu.edu/csccmathematics/precalculus1/precalculus1/sequences/examples/exampleList}

\end{center}





\end{document}
