\documentclass{ximera}


\graphicspath{
  {./}
  {ximeraTutorial/}
  {basicPhilosophy/}
}

\newcommand{\mooculus}{\textsf{\textbf{MOOC}\textnormal{\textsf{ULUS}}}}

\usepackage{tkz-euclide}\usepackage{tikz}
\usepackage{tikz-cd}
\usetikzlibrary{arrows}
\tikzset{>=stealth,commutative diagrams/.cd,
  arrow style=tikz,diagrams={>=stealth}} %% cool arrow head
\tikzset{shorten <>/.style={ shorten >=#1, shorten <=#1 } } %% allows shorter vectors

\usetikzlibrary{backgrounds} %% for boxes around graphs
\usetikzlibrary{shapes,positioning}  %% Clouds and stars
\usetikzlibrary{matrix} %% for matrix
\usepgfplotslibrary{polar} %% for polar plots
\usepgfplotslibrary{fillbetween} %% to shade area between curves in TikZ
\usetkzobj{all}
\usepackage[makeroom]{cancel} %% for strike outs
%\usepackage{mathtools} %% for pretty underbrace % Breaks Ximera
%\usepackage{multicol}
\usepackage{pgffor} %% required for integral for loops



%% http://tex.stackexchange.com/questions/66490/drawing-a-tikz-arc-specifying-the-center
%% Draws beach ball
\tikzset{pics/carc/.style args={#1:#2:#3}{code={\draw[pic actions] (#1:#3) arc(#1:#2:#3);}}}



\usepackage{array}
\setlength{\extrarowheight}{+.1cm}
\newdimen\digitwidth
\settowidth\digitwidth{9}
\def\divrule#1#2{
\noalign{\moveright#1\digitwidth
\vbox{\hrule width#2\digitwidth}}}






\DeclareMathOperator{\arccot}{arccot}
\DeclareMathOperator{\arcsec}{arcsec}
\DeclareMathOperator{\arccsc}{arccsc}

















%%This is to help with formatting on future title pages.
\newenvironment{sectionOutcomes}{}{}


\title{Sequences as functions}

\begin{document}

\begin{abstract}
%Stuff can go here later if we want!
\end{abstract}

\maketitle




We have seen a common structure in several mathematical situations - the idea of a limit.


$\blacktriangleright$ Decimal numbers or expansions are a sequence of digits that go on forever.  The sequence gets closer to a number. If you stop at any place, then you don't have the limiting number. \\


$\blacktriangleright$ Rational functions and exponential functions have an end-behavior, which appears as a horizontal asymptote on the graph.  The function values get closer and closer to the asymptotic value.  If you stop at any place, then you don't have the limiting value. \\


$\blacktriangleright$ Sequences do the same thing. \\


We can create a sequence of numbers. The numbers might get closer and closer to a limiting number.  They might not. \\




In some cases, the sequence does not head toward a limiting number. In some cases, we might know the limiting number is there, but have little information about it.  So, we end up with two questions.



\begin{itemize}
\item Does the sequence have a limiting value?
\item If so, what is the number?
\end{itemize}













\begin{sectionOutcomes}

After completing this section, students should be able to do the following.

\begin{itemize}
\item{Recognize sequences can be generated by functions.}
\item{Compute limits of sequences.}
\item{Understand growth rates of basic sequences.}
\item{Introduce important terminology for sequences.}
\item{Apply the monotone convergence theorem.}
\end{itemize}

\end{sectionOutcomes}

\end{document}
