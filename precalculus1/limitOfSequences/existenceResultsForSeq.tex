\documentclass{ximera}


\graphicspath{
  {./}
  {ximeraTutorial/}
  {basicPhilosophy/}
}

\newcommand{\mooculus}{\textsf{\textbf{MOOC}\textnormal{\textsf{ULUS}}}}

\usepackage{tkz-euclide}\usepackage{tikz}
\usepackage{tikz-cd}
\usetikzlibrary{arrows}
\tikzset{>=stealth,commutative diagrams/.cd,
  arrow style=tikz,diagrams={>=stealth}} %% cool arrow head
\tikzset{shorten <>/.style={ shorten >=#1, shorten <=#1 } } %% allows shorter vectors

\usetikzlibrary{backgrounds} %% for boxes around graphs
\usetikzlibrary{shapes,positioning}  %% Clouds and stars
\usetikzlibrary{matrix} %% for matrix
\usepgfplotslibrary{polar} %% for polar plots
\usepgfplotslibrary{fillbetween} %% to shade area between curves in TikZ
\usetkzobj{all}
\usepackage[makeroom]{cancel} %% for strike outs
%\usepackage{mathtools} %% for pretty underbrace % Breaks Ximera
%\usepackage{multicol}
\usepackage{pgffor} %% required for integral for loops



%% http://tex.stackexchange.com/questions/66490/drawing-a-tikz-arc-specifying-the-center
%% Draws beach ball
\tikzset{pics/carc/.style args={#1:#2:#3}{code={\draw[pic actions] (#1:#3) arc(#1:#2:#3);}}}



\usepackage{array}
\setlength{\extrarowheight}{+.1cm}
\newdimen\digitwidth
\settowidth\digitwidth{9}
\def\divrule#1#2{
\noalign{\moveright#1\digitwidth
\vbox{\hrule width#2\digitwidth}}}






\DeclareMathOperator{\arccot}{arccot}
\DeclareMathOperator{\arcsec}{arcsec}
\DeclareMathOperator{\arccsc}{arccsc}

















%%This is to help with formatting on future title pages.
\newenvironment{sectionOutcomes}{}{}

\author{Bart Snapp and Jim Talamo (edited)}

\outcome{Recognize sequences can be generated by functions.}
\outcome{Compute limits of sequences.}
\outcome{Understand growth rates of basic sequences.}
\outcome{Introduce important terminology for sequences.}
\outcome{Apply the monotone convergence theorem.}

\title{Existence}

\begin{document}
\begin{abstract}
convergence vs. divergence
\end{abstract}
\maketitle







\section{Existence results for limits}

Perhaps the best way to determine whether the limit of a sequence exists is to compute it.  Even though we've been working with sequences that are generated by an explicit formula in this section thus far, not all sequences are defined this way.  Sometimes, we'll only have a recursive description of a sequence rather than an explicit one, and sometimes we will have neither.  It is common to only have a recursive description of a sequence, so we want to determine a good approach for determining whether a limit exists without having to compute it directly.  To do this, we introduce some terminology focused on the relationships between the terms of a sequence.

\begin{definition} \textbf{\textcolor{green!50!black}{Growth}} 


  A sequence is called
  \begin{itemize}
    \item \textbf{increasing} if $a_n<a_{n+1}$ for all $n$,
    \item \textbf{nondecreasing} if $a_n\le a_{n+1}$ for all $n$,
    \item \textbf{decreasing} if $a_n>a_{n+1}$ for all $n$,
    \item \textbf{nonincreasing} if $a_n\ge a_{n+1}$ for all $n$.
  \end{itemize}
\end{definition}

Lots of facts are true for sequences which are either increasing or
decreasing; to talk about this situation without constantly saying
``either increasing or decreasing,'' we can introduce a single word to
cover both cases.
\begin{definition} \textbf{\textcolor{green!50!black}{Monotonic Sequences}} 


  If a sequence is always increasing, or  always nondecreasing, or always decreasing, or always nonincreasing, it is said to be \textbf{monotonic}\index{sequence!monotonic}.
\end{definition}


\begin{example}
If $a_n = 2n^2+1$, then $\{a_n\}_{n=1}^{\infty}$ is \wordChoice{\choice[correct]{increasing}\choice{decreasing}}, so it is
      \wordChoice{\choice[correct]{monotonic}\choice{not monotonic}}
\end{example}

\begin{question}
  If an arithmetic sequence $a_n = m\cdot n + b$ is monotonic, what
  must be true about $m$ and $b$?
  \begin{prompt}
    \begin{quote}
      The sign of $m$ \wordChoice{\choice{is positive}\choice{is
          negative}\choice[correct]{does not matter}}, and the sign of $b$ is
      \wordChoice{\choice{is positive}\choice{is negative}\choice[correct]{does
          not matter}}
    \end{quote}
  \end{prompt}
  \begin{feedback}
   We can model an arithmetic sequence $a_n = m\cdot n + b$ with the line $f(x) = mx+b$.  Can a line ever increase then decrease or vice-versa? 
  \end{feedback}
\end{question}

\begin{question}
  If a geometric sequence $a_n = a_1 \cdot r^{n-1}$ is monotonic, what
  must be true about $a_1$ and $r$?
  \begin{prompt}
    \begin{quote}
      The sign of $a_1$ \wordChoice{\choice{is positive}\choice{is
          negative}\choice[correct]{does not matter}}, and the sign of
      $r$ is \wordChoice{\choice[correct]{is positive}\choice{is
          negative}\choice{does not matter}}
    \end{quote}
  \end{prompt}
    \begin{feedback}
    From our examples earlier in the section, a geometric sequence $a_n = a_1 \cdot r^{n-1}$ can be modeled by an exponential function (which is always increasing or always decreasing) if the sign of $r$ is positive.  If $r$ is negative, the signs of each successive term is different from the last.
  \end{feedback}
\end{question}












Sometimes we want to classify sequences for which the terms do not get too big or too
small.  

\begin{definition}  \textbf{\textcolor{green!50!black}{Bounded Sequences}}   


  \label{definition:sequence-bounded}
  A sequence $\{a_n\}$ is:
  
  \begin{itemize}
  \item \textbf{bounded above} if there is some number $M$ so
  that for all $n$, we have $a_n\le M$.
    \item \textbf{bounded below} if there is some number $m$ so
  that for all $n$, we have $a_n\ge m$.
  \item \textbf{bounded} if it is both bounded above and bounded below.
  \end{itemize}
\end{definition}

So what does this definition actually say? Essentially, we say that a sequence is bounded above if its terms cannot become too large, bounded below if its terms cannot become too large and negative, and bounded if the terms cannot become too large and positive or too large and negative.

\begin{question}
  True or False: If a sequence $(a_n)_{n=0}^\infty$ is nondecreasing
  it is bounded below by $a_0$.
  \begin{prompt}
    \begin{multipleChoice}
    \choice[correct]{True}
    \choice{False}
    \end{multipleChoice}
  \end{prompt}
  \begin{feedback}
    If a sequence is nondecreasing, then its smallest value is its
    first element.
  \end{feedback}
\end{question}


\begin{question}
  True or False: If a sequence $(a_n)_{n=0}^\infty$ is nonincreasing
  it is bounded above by $a_0$.
  \begin{prompt}
    \begin{multipleChoice}
    \choice[correct]{True}
    \choice{False}
    \end{multipleChoice}
  \end{prompt}
  \begin{feedback}
    If a sequence is nonincreasing, then its largest value is its
    first element.
  \end{feedback}
\end{question}

So, what do these previous definitions have to do with the idea of a limit?  Essentially, there are three reasons that a sequence may diverge:

\begin{itemize}
\item the terms eventually are either always positive or always negative but become arbitrarily large in magnitude.
\item the terms are never eventually monotonic.
\item the terms are never eventually monotonic \emph{and} become arbitrarily large in magnitude.
\end{itemize}

Let's think about the terminology we introduced.

\begin{question}
Think about the following statements and choose the correct option.
\begin{itemize}
\item If we know that a sequence is monotonic and its limit does not exist, then  \wordChoice{\choice[correct]{the terms become too large in magnitude}\choice{the terms are never eventually monotonic}\choice{the terms  are never eventually monotonic and become arbitrarily large in magnitude}}.

\item If we know that a sequence is bounded and its limit does not exist, then  \wordChoice{\choice{the terms become arbitrarily large in magnitude.}\choice[correct]{the terms  are never eventually monotonic}\choice{the terms  are never eventually monotonic and become arbitrarily large in magnitude}}.

\end{itemize}
\end{question}

We can now state an important theorem:

\begin{theorem}[Bounded-monotone convergence theorem]
  If the sequence $\{a_n\}_{n=1}^{\infty}$ is bounded and monotonic, then $\lim\limits_{n \to  \infty} a_n $ exists.
\end{theorem}

To think about the statement of the theorem, if we have a sequence that is bounded, the only way it could diverge is if the terms are never eventually monotonic.  However, if we know the sequence is also monotonic, this cannot happen!  Thus, the series cannot diverge, so it must have a limit.









In short, bounded monotonic sequences always converge, though we can't
necessarily describe the number to which they converge.  Let's try
some examples.

\begin{example}
  Given the sequence $a_n=\frac{2^n-1}{2^n}$ for $n=1,2,3,\dots$,
  explain how you know that $\lim\limits_{n\to\infty} a_n$ converges to a
  finite value without computing its limit.
  \begin{explanation}
    To start, note that $a_n=\frac{2^n-1}{2^n} = \frac{2^n}{2^n} - \frac{1}{2^n} = 1 - \frac{1}{2^n} $.  Thus, $\{a_n\}$ is
    \wordChoice{\choice[correct]{monotonic}\choice{not monotonic}}.
    Moreover, all of the elements $a_n = (2^n-1)/2^n$ are less than
    $2$  and greater than zero.   So, the sequence is bounded above by $2$ and below by $0$, so it is bounded.  By the bounded-monotone
    convergence theorem, $\lim\limits_{n\to\infty} a_n$ must converge to a finite value.
  \end{explanation}
\end{example}



\begin{remark}
We don't actually need to know that a sequence is always monotonic to apply
the bounded-monotone convergence theorem. It is enough to know that
the sequence is eventually monotonic.  More formally, this means that there is 
some integer $N$ for which the sequence $\{a_n\}_{n=N}$ 
is always increasing or always decreasing.
\end{remark}













In the previous examples, we could write down a function $f(x)$ corresponding to each series and apply the theorem from earlier in the section.  However, this is not always possible.

\begin{example}
Suppose that $a_n = 2 + \sin(n)$, and let $s_n = \sum\limits_{k=1}^{n} a_k$.  Determine whether the sequences $\{a_n\}$ and $\{s_n\}$ are bounded or monotonic, and explain whether either has a limit.

\begin{explanation}
The sequence $a_n$ is certainly not monotonic, but it  is bounded since $3 \geq a_n \geq 1$ for all $n$.  We also can see that $\lim\limits_{n \to \infty} a_n$ does not exist since the terms oscillate.

For $s_n$, note that since $a_n \geq 1$ for all $n$, each term $a_n$ is positive.  Thus, $s_n$ is increasing and hence monotonic.  

However, since $a_n\geq1$ for all $n$, we have the following inequality.

\[s_n = \sum\limits_{k=1}^n a_n \geq \sum\limits_{k=1}^n 1 = n\]

Hence, $\{s_n\}$ is not bounded and $\lim\limits_{n \to \infty} s_n$ does not exist.
\end{explanation}
\end{example}

\begin{remark}
We had a way to analyze $\{a_n\}$ in the above example because we had an explicit formula for $a_n$, but how would we find such a formula for $s_n$?  As it turned out, we didn't have to do so in order to determine that $\lim\limits_{n \to \infty} s_n$ does not exist.  We will make arguments in the coming sections that allow us to determine whether limits of sequences exist without relying on having explicit formulas for the sequences.  
\end{remark}



\end{document}
