\documentclass{ximera}


\graphicspath{
  {./}
  {ximeraTutorial/}
  {basicPhilosophy/}
}

\newcommand{\mooculus}{\textsf{\textbf{MOOC}\textnormal{\textsf{ULUS}}}}

\usepackage{tkz-euclide}\usepackage{tikz}
\usepackage{tikz-cd}
\usetikzlibrary{arrows}
\tikzset{>=stealth,commutative diagrams/.cd,
  arrow style=tikz,diagrams={>=stealth}} %% cool arrow head
\tikzset{shorten <>/.style={ shorten >=#1, shorten <=#1 } } %% allows shorter vectors

\usetikzlibrary{backgrounds} %% for boxes around graphs
\usetikzlibrary{shapes,positioning}  %% Clouds and stars
\usetikzlibrary{matrix} %% for matrix
\usepgfplotslibrary{polar} %% for polar plots
\usepgfplotslibrary{fillbetween} %% to shade area between curves in TikZ
\usetkzobj{all}
\usepackage[makeroom]{cancel} %% for strike outs
%\usepackage{mathtools} %% for pretty underbrace % Breaks Ximera
%\usepackage{multicol}
\usepackage{pgffor} %% required for integral for loops



%% http://tex.stackexchange.com/questions/66490/drawing-a-tikz-arc-specifying-the-center
%% Draws beach ball
\tikzset{pics/carc/.style args={#1:#2:#3}{code={\draw[pic actions] (#1:#3) arc(#1:#2:#3);}}}



\usepackage{array}
\setlength{\extrarowheight}{+.1cm}
\newdimen\digitwidth
\settowidth\digitwidth{9}
\def\divrule#1#2{
\noalign{\moveright#1\digitwidth
\vbox{\hrule width#2\digitwidth}}}






\DeclareMathOperator{\arccot}{arccot}
\DeclareMathOperator{\arcsec}{arcsec}
\DeclareMathOperator{\arccsc}{arccsc}

















%%This is to help with formatting on future title pages.
\newenvironment{sectionOutcomes}{}{}


\title{Theory}

\begin{document}

\begin{abstract}
the whole story
\end{abstract}
\maketitle




We have investigated quadratic functions from several viewpoints.  Time to collect all of our thoughts and charcterize quadratic functions.





\begin{definition} Quadratic Function


A \textbf{quadratic function} is a function that can be represented by a formula of the form


\[   Q(t) = a \, t^2 + b \, t + c         \]

where $a$, $b$, $c$ are real numbers and $a \ne 0$.


\end{definition}



$\blacktriangleright$ \textbf{Domain:} The implied domain of a quadratic function is all real numbers.  However, a particular quadratic function may be defined with a subset of the real numbers.




\begin{formula} Forms

Formulas for quadratic functions have three basic forms:



\begin{itemize}
\item \textbf{Standard Form:}  The standard form for a quadratic function looks like $a \, t^2 + b \, t + c $.
\item \textbf{Factored Form:}  The factored form for a quadratic function looks like $a(t - r_1)(t - r_2)$.
\item \textbf{Vertex Form:}  The vertex form for a quadratic function looks like $a(t - h)^2 + k$.
\end{itemize}



Each form serves a different purpose.  For analysis purposes, the factored form reveals the zeros of the quadratic function.  The vertex form reveals the extreme values of the function.



\end{formula}

\textbf{Note:} We are restricting our investigation to quadratics with real roots (or zeros).  Therefore, we many encounter quadratics that do not factor for us.  We will need complex numbers in the next course to complete our characterization of quadratic functions.




$\blacktriangleright$ \textbf{Graph:}  The graph of a quadratic function is a parabola, which opens up or down.  The direction depends on the sign of the leading coefficient, $a$.








\begin{image}
\begin{tikzpicture}
  \begin{axis}[name = leftgraph,
            domain=-10:10, ymax=10, xmax=10, ymin=-10, xmin=-10,
            axis lines =center, xlabel=$x$, ylabel=$y$, grid = major, grid style={dashed},
            ytick={-10,-8,-6,-4,-2,2,4,6,8,10},
            xtick={-10,-8,-6,-4,-2,2,4,6,8,10},
            yticklabels={$-10$,$-8$,$-6$,$-4$,$-2$,$2$,$4$,$6$,$8$,$10$}, 
            xticklabels={$-10$,$-8$,$-6$,$-4$,$-2$,$2$,$4$,$6$,$8$,$10$},
            ticklabel style={font=\scriptsize},
            every axis y label/.style={at=(current axis.above origin),anchor=south},
            every axis x label/.style={at=(current axis.right of origin),anchor=west},
            axis on top
          ]
          

            \addplot [line width=2, penColor, smooth,samples=200,domain=(-8.5:-3.25),<->] {(x+6)^2+3};
            \addplot [line width=2, penColor2, smooth,samples=200,domain=(-5:1.1),<->] {(x+2)^2};
            \addplot [line width=2, penColor3, smooth,samples=200,domain=(0:8),<->] {(x-4)^2-6};



           

  \end{axis}
  \begin{axis}[at={(leftgraph.outer east)},anchor=outer west,
            domain=-10:10, ymax=10, xmax=10, ymin=-10, xmin=-10,
            axis lines =center, xlabel=$x$, ylabel=$y$, grid = major, grid style={dashed},
            ytick={-10,-8,-6,-4,-2,2,4,6,8,10},
            xtick={-10,-8,-6,-4,-2,2,4,6,8,10},
            yticklabels={$-10$,$-8$,$-6$,$-4$,$-2$,$2$,$4$,$6$,$8$,$10$}, 
            xticklabels={$-10$,$-8$,$-6$,$-4$,$-2$,$2$,$4$,$6$,$8$,$10$},
            ticklabel style={font=\scriptsize},
            every axis y label/.style={at=(current axis.above origin),anchor=south},
            every axis x label/.style={at=(current axis.right of origin),anchor=west},
            axis on top
          ]
          
			\addplot [line width=2, penColor, smooth,samples=200,domain=(-8.5:-3.25),<->] {-(x+6)^2-3};
            \addplot [line width=2, penColor2, smooth,samples=200,domain=(-5:1.1),<->] {-(x+2)^2};
            \addplot [line width=2, penColor3, smooth,samples=200,domain=(0:8),<->] {-(x-4)^2+6};


           

  \end{axis}
\end{tikzpicture}
\end{image}





$\blacktriangleright$ \textbf{Zeros (1):}  Quadratic function always have two roots or zeros.  However, these roots can be complex numbers, which we are postpoing to the second course.  Our quadratics, in this course, will have $0$, $1$, or $2$ real roots.  These correspond to  $0$, $1$, or $2$ intercepts on the graph.

The two roots can be diffeernt numbers corresponding to two different intercepts on the graph.  The two roots can be the same number.  In this case the factorization is a square: $Q(t) = a(t-r)^2$.  The root has an even multiplicity and the vertex of the parabola is $(r, 0)$



$\blacktriangleright$ \textbf{Range:} From the graphs, we can see that the there are two types of ranges: 
\begin{itemize}
\item $(-\infty, k]$ if the leading coefficient is negative and the parabola opens down.
\item $[k, -\infty)$ if the leading coefficient is positive and the parabola opens up.
\end{itemize}

$k$ is the constant term in the vector form.



$\blacktriangleright$ \textbf{Extrema:} From the graphs, we can see that 
\begin{itemize}
\item The function has a single maximum value of $k$, if the leading coefficient is negative and the parabola opens down.
\item The function has a single minimum value of $k$, if the leading coefficient is positive and the parabola opens up.
\end{itemize}

$k$ is the constant term in the vector form.




$\blacktriangleright$ \textbf{Rate-of-change:} 
\begin{itemize}
\item If the leading coefficient is negative and the parabola opens down, then the function increases on $(-\infty, h]$ and decreases on $[h, \infty)$, where $h$ is from the vertex form.
\item If the leading coefficient is positive and the parabola opens up, then the function decreases on $(-\infty, h]$ and increases on $[h, \infty)$, where $h$ is from the vertex form.
\end{itemize}





$\blacktriangleright$ \textbf{Axis of Symmetry:} The vector form, $a(t-h)^2 + k$ shows us that the graph is symmetric about the vertical line, $t=h$.  This is the line of symmetry or the axis of symmetry.  This should be drawn on the graph.

















\begin{image}
\begin{tikzpicture}
  \begin{axis}[
            domain=-10:10, ymax=10, xmax=10, ymin=-10, xmin=-10,
            axis lines =center, xlabel=$t$, ylabel={$y=Q(t)$}, grid = major, grid style={dashed},
            ytick={-10,-8,-6,-4,-2,2,4,6,8,10},
            xtick={-10,-8,-6,-4,-2,2,4,6,8,10},
            yticklabels={$-10$,$-8$,$-6$,$-4$,$-2$,$2$,$4$,$6$,$8$,$10$}, 
            xticklabels={$-10$,$-8$,$-6$,$-4$,$-2$,$2$,$4$,$6$,$8$,$10$},
            ticklabel style={font=\scriptsize},
            every axis y label/.style={at=(current axis.above origin),anchor=south},
            every axis x label/.style={at=(current axis.right of origin),anchor=west},
            axis on top
          ]
          

			\addplot [line width=1, gray, dashed, domain=(-9.5:9.5),<->] ({4},{x});
			\addplot [line width=2, penColor, smooth,samples=200,domain=(0:8),<->] {(x-4)^2 - 6};

          %\addplot[color=penColor,fill=penColor2,only marks,mark=*] coordinates{(-6,9)};
          %\addplot[color=penColor,fill=penColor2,only marks,mark=*] coordinates{(2,-7)};

			\addplot[color=penColor,fill=penColor,only marks,mark=*] coordinates{(1.551,0)};
			\addplot[color=penColor,fill=penColor,only marks,mark=*] coordinates{(6.449,0)};


           

  \end{axis}
\end{tikzpicture}
\end{image}






$\blacktriangleright$ \textbf{Zeros (2):}  Our quadratic functions have $0$, $1$, or $2$ roots or zeros.  These can be obtained from any of the three forms.


\begin{itemize}

\item \textbf{Standard Form:}  Given the standard form, $Q(t) = a \, t^2 + b \, t + c$, we can use the \textbf{Qudratic Formula}.

The zeros of $Q(t) = a \, t^2 + b \, t + c$ are

\[   \frac{-b + \sqrt{b^2 - 4 \, a \, c}}{2a}     \, \text{ and } \,       \frac{-b - \sqrt{b^2 - 4 \, a \, c}}{2a}    \]



$b^2 - 4 \, a \, c < 0$ is known as the \textbf{discriminant}. If $b^2 - 4 \, a \, c < 0$, then there are no real zeros.  They will be complex numbers.


\item \textbf{Factored Form:} Given the factored form, $Q(t) = a (t - r_1)(t - r_2)$, we can read off the roots or zeros as $r_1$ and $r_2$. Mnay times the leading coefficient, $a$, is not factored out.  Then we can use the zero product property and set each factor equals to $0$ and solve.






\item \textbf{Vertex Form:}  Given the vertex form, $Q(t) = a (t - h)^2 + k$, we can set equal to $0$ and solve.



\begin{align*}
a (t - h)^2 + k    & = 0  \\
a (t - h)^2        & = -k  \\
(t - h)^2        & = -\frac{k}{a}  \\
t - h        & = \pm \sqrt{-\frac{k}{a}}  \\
t        & = \pm \sqrt{-\frac{k}{a}}  + h
\end{align*}

From $a (t - h)^2 + k  = 0$, we can see that if $a$ and $k$ have the same sign, then ther are no solutions.  The zeros are complex numbers.



\end{itemize}











\end{document}
