\documentclass{ximera}


\graphicspath{
  {./}
  {ximeraTutorial/}
  {basicPhilosophy/}
}

\newcommand{\mooculus}{\textsf{\textbf{MOOC}\textnormal{\textsf{ULUS}}}}

\usepackage{tkz-euclide}\usepackage{tikz}
\usepackage{tikz-cd}
\usetikzlibrary{arrows}
\tikzset{>=stealth,commutative diagrams/.cd,
  arrow style=tikz,diagrams={>=stealth}} %% cool arrow head
\tikzset{shorten <>/.style={ shorten >=#1, shorten <=#1 } } %% allows shorter vectors

\usetikzlibrary{backgrounds} %% for boxes around graphs
\usetikzlibrary{shapes,positioning}  %% Clouds and stars
\usetikzlibrary{matrix} %% for matrix
\usepgfplotslibrary{polar} %% for polar plots
\usepgfplotslibrary{fillbetween} %% to shade area between curves in TikZ
\usetkzobj{all}
\usepackage[makeroom]{cancel} %% for strike outs
%\usepackage{mathtools} %% for pretty underbrace % Breaks Ximera
%\usepackage{multicol}
\usepackage{pgffor} %% required for integral for loops



%% http://tex.stackexchange.com/questions/66490/drawing-a-tikz-arc-specifying-the-center
%% Draws beach ball
\tikzset{pics/carc/.style args={#1:#2:#3}{code={\draw[pic actions] (#1:#3) arc(#1:#2:#3);}}}



\usepackage{array}
\setlength{\extrarowheight}{+.1cm}
\newdimen\digitwidth
\settowidth\digitwidth{9}
\def\divrule#1#2{
\noalign{\moveright#1\digitwidth
\vbox{\hrule width#2\digitwidth}}}






\DeclareMathOperator{\arccot}{arccot}
\DeclareMathOperator{\arcsec}{arcsec}
\DeclareMathOperator{\arccsc}{arccsc}

















%%This is to help with formatting on future title pages.
\newenvironment{sectionOutcomes}{}{}


\title{Quadratic Analysis}

\begin{document}

\begin{abstract}
behavior
\end{abstract}
\maketitle





\section{Analysis}

What do we want to know when we analyze functions?

We want to know the 


\begin{itemize}
\item domain
\item range
\item zeros
\item intervals of increasing and decreasing
\item global maximum and minimum
\item local maximums and minimums
\item discontinuities
\item singularities
\item symmetry
\item endbehavior \\
\item and we would like a nice graph
\end{itemize}


For a quadratic function much of this information is connected to vertex of the graph.







\textbf{Vertex Form} \\

The graph of a quadratic function is a parabola, which is easily connected to the completed square form of the formula.

Below is the graph of $y = f(x) = a (x - h)^2 + k$, with $a$, and $h$, and $k$ all real numbers and $a > 0$. The extreme point is called the \textbf{vertex}. If $a<0$, then the parabola would have opened downward and the extreme point would be at the top.

\begin{image}
\begin{tikzpicture}
     \begin{axis}[
                domain=-10:10, ymax=10, xmax=10, ymin=-10, xmin=-10,
                axis lines =center, xlabel=$x$, ylabel=$y$,
                ytick={-10,-8,-6,-4,-2,2,4,6,8,10},
                xtick={-10,-8,-6,-4,-2,2,4,6,8,10},
                ticklabel style={font=\scriptsize},
                every axis y label/.style={at=(current axis.above origin),anchor=south},
                every axis x label/.style={at=(current axis.right of origin),anchor=west},
                axis on top,
                ]


        \addplot [draw=penColor, very thick, smooth, domain=(1:5),<->] {2*(x-3)^2 - 4};
        \addplot [line width=1, gray, dashed,samples=100,domain=(-9.5:9.5)] ({3},{x});
        \addplot [color=penColor2,only marks,mark=*] coordinates{(3,-4)};
        


        \node[penColor] at (axis cs:5,-4) {$(h, k)$};
        %\node[penColor] at (axis cs:5,-9) {$-0.5 x^2 - 5 x + 15.5$};



    \end{axis}
\end{tikzpicture}
\end{image}

The vertex visually encodes the  minimum (or maximum) value of the function. 

We can see from the formula $f(x) = a (x - h)^2 + k$, that since $(x - h)$ is squared and $a>0$, the value of $f(x)$ is greater than or equal to $k$. This corresponds to the graph opening up.  The only way to get the least value possible for $f$ is to select $x = h$. That makes the vertex $(k, h)$.


If $a<0$, then everything is reveresed.







\begin{image}
\begin{tikzpicture}
     \begin{axis}[
                domain=-10:10, ymax=10, xmax=10, ymin=-10, xmin=-10,
                axis lines =center, xlabel=$x$, ylabel=$y$, 
                ytick={-10,-8,-6,-4,-2,2,4,6,8,10},
                xtick={-10,-8,-6,-4,-2,2,4,6,8,10},
                ticklabel style={font=\scriptsize},
                every axis y label/.style={at=(current axis.above origin),anchor=south},
                every axis x label/.style={at=(current axis.right of origin),anchor=west},
                axis on top,
                ]


        \addplot [draw=penColor, very thick, smooth, domain=(1:5),<->] {-2*(x-3)^2 + 4};
        \addplot [line width=1, gray, dashed,samples=100,domain=(-9.5:9.5)] ({3},{x});
        \addplot [color=penColor2,only marks,mark=*] coordinates{(3,4)};
        


        \node[penColor] at (axis cs:5,4) {$(h, k)$};
        %\node[penColor] at (axis cs:5,-9) {$-0.5 x^2 - 5 x + 15.5$};



    \end{axis}
\end{tikzpicture}
\end{image}










From the graphs above we can see that the implied range of a quadratic comes in two types.  

\begin{itemize}
\item The range could be all real numbers greater than or equal to some particular number:  $\{ r \in \textbf{R} \, | \, r \geq k \}$.
\item The range could be all real numbers less than or equal to some particular number:  $\{ r \in \textbf{R} \, | \, r \leq k \}$.
\end{itemize}

If there is a stated domain, then the range will be restricted appropriately. \\






\textbf{Symmetry} \\


If we take two domain numbers equidistant from $h$, like $h - \epsilon$ and $h + \epsilon$, and evaluate $f(x)$ at these two numbers we get the same value.


\begin{align*}
f(h - \epsilon) & = a (h - \epsilon - h)^2 + k \\
& = a (-\epsilon)^2 + k  \\
& = a (\epsilon)^2 + k  \\
& = a (h + \epsilon - h)^2 + k \\
& = f(h + \epsilon)
\end{align*}


The graph is symmetric about the line $x = h$.


When developing the quadratic formula, we had the opportunity to complete the square and that gave us a squared term that looked like 

\[  a \, \left(x - \frac{b}{2a}\right)^2       \]


The domain coordinate of the vertex is $h = \frac{b}{2a}$.


The graph is symmetric about the line $x = \frac{b}{2a}$.






The zeros must also be symmetric about $\frac{b}{2a}$, which we can see in the quadratic formula. \\





\textbf{Zeros} \\

The quadratic formula is

\[ t  =   \frac{-b \pm \sqrt{b^2 - 4 a c}}{2a}      \]



which we can separate into 



\[ t  =   \frac{b}{2a} \pm \frac{\sqrt{b^2 - 4 a c}}{2a}      \]


to see that the zeros are symmetric about $\frac{b}{2a}$ and the intercepts are symmetric about the line $x = \frac{b}{2a}$.





\begin{image}
\begin{tikzpicture}
     \begin{axis}[
                domain=-10:10, ymax=10, xmax=10, ymin=-10, xmin=-10,
                axis lines =center, xlabel=$x$, ylabel=$y$,
                ytick={-10,-8,-6,-4,-2,2,4,6,8,10},
                xtick={-10,-8,-6,-4,-2,2,4,6,8,10},
                ticklabel style={font=\scriptsize},
                every axis y label/.style={at=(current axis.above origin),anchor=south},
                every axis x label/.style={at=(current axis.right of origin),anchor=west},
                axis on top,
                ]


        \addplot [draw=penColor, very thick, smooth, domain=(1:5),<->] {-2*(x-3)^2 + 4};
        \addplot [line width=1, gray, dashed,samples=100,domain=(-9.5:9.5)] ({3},{x});
        \addplot [color=penColor2,only marks,mark=*] coordinates{(3,4)};

        \addplot [color=penColor2,only marks,mark=*] coordinates{(1.585,0)};
        \addplot [color=penColor2,only marks,mark=*] coordinates{(4.414,0)};
        


        %\node[penColor] at (axis cs:5,4) {$(h, k))$};
        %\node[penColor] at (axis cs:5,-9) {$-0.5 x^2 - 5 x + 15.5$};



    \end{axis}
\end{tikzpicture}
\end{image}





Working backwards, we can see that if we have the zeros, like from factoring, then we have the intercepts, and the \textbf{line of symmetry} must run in the middle. \\

























\section{Increasing and Decreasing}



We can see from the graph that a quadratic function switches from increasing to decreasing at the symmetric number in the domain.


\begin{itemize}
\item If the graph is opening up, then the function is \\

decreasing on $\left( -\infty, \frac{b}{2a} \right)$ and increasing on $\left( \frac{b}{2a}, \infty \right)$

\item If the graph is opening down, then the function is \\

increasing on $\left( -\infty, \frac{b}{2a} \right)$ and decreasing on $\left( \frac{b}{2a}, \infty \right)$
\end{itemize}





Increasing and decreasing refer to the rate of change.


\begin{itemize}
\item Increasing is a positive rate of change.
\item Decreasing is a negative rate of change.
\end{itemize}


We have already seen if we write a quadratic function as $f(x) = a (x - h)^2 + k$, then the instantaneous rate of change of $f$ is the linear function $iRoC_f(x) = 2 a (x - h)$. The values of $iRoC$ are the slopes of the tangent lines.










\begin{image}
\begin{tikzpicture}
     \begin{axis}[
                domain=-10:10, ymax=10, xmax=10, ymin=-10, xmin=-10,
                axis lines =center, xlabel=$x$, ylabel=$y$,
                ytick={-10,-8,-6,-4,-2,2,4,6,8,10},
                xtick={-10,-8,-6,-4,-2,2,4,6,8,10},
                ticklabel style={font=\scriptsize},
                every axis y label/.style={at=(current axis.above origin),anchor=south},
                every axis x label/.style={at=(current axis.right of origin),anchor=west},
                axis on top,
                ]


        \addplot [draw=penColor, very thick, smooth, domain=(1:5),<->] {2*(x-3)^2 - 4};
        \addplot [line width=1, gray, dashed,samples=100,domain=(-9.5:9.5)] ({3},{x});
        \addplot [color=penColor,only marks,mark=*] coordinates{(3,-4)};
        
        \addplot [draw=penColor2, very thick, smooth, domain=(-1:7),<->] {4*(x-3)};

        \node[penColor] at (axis cs:5,-4) {$(h, k)$};
        %\node[penColor] at (axis cs:5,-9) {$-0.5 x^2 - 5 x + 15.5$};



    \end{axis}
\end{tikzpicture}
\end{image}
Our linear rate of change function now informs us about the behavior of $f$. \\


$\blacktriangleright$ When the linear rate of change function $iRoC_f(x) < 0$, then $f(x)$ is decreasing. \\

$\blacktriangleright$ When the linear rate of change function $iRoC_f(x) > 0$, then $f(x)$ is increasing. \\

$\blacktriangleright$ When the linear rate of change function $iRoC_f(x) = 0$, then $f(x)$ is neither increasing nor decreasing and the graph of $y = f(x)$ is flat.















\section{Maximum and Minimum}


The maximum and minimum values of a quadratic function are visually encoded in the highest and lowest points on the  graph, which is the vertex here.

Depending on the sign of $a$, the maximum or minimum value of $f(x) = a \, x^2 + b \, x + c$ occurs at $\frac{b}{2a}$. The maximum or minimum value is $f\left( \frac{b}{2a} \right)$


Depending on the sign of $a$, the maximum or minimum value of $f(x) = a \, (x - h)^2 + k$ is $k$ and occurs at $h$. 



We can also look at the linear $iRoC_f(x)$ function.  Where $iRoC_f(x) = 0$ is where the vertex is located, which encodes the maximum or minimum value of $f$.




\section{Discontinuities}


A quadratic function has no discontinuities.  It is continuous for all real numbers.










\end{document}




