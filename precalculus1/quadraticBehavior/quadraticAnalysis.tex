\documentclass{ximera}


\graphicspath{
  {./}
  {ximeraTutorial/}
  {basicPhilosophy/}
}

\newcommand{\mooculus}{\textsf{\textbf{MOOC}\textnormal{\textsf{ULUS}}}}

\usepackage{tkz-euclide}\usepackage{tikz}
\usepackage{tikz-cd}
\usetikzlibrary{arrows}
\tikzset{>=stealth,commutative diagrams/.cd,
  arrow style=tikz,diagrams={>=stealth}} %% cool arrow head
\tikzset{shorten <>/.style={ shorten >=#1, shorten <=#1 } } %% allows shorter vectors

\usetikzlibrary{backgrounds} %% for boxes around graphs
\usetikzlibrary{shapes,positioning}  %% Clouds and stars
\usetikzlibrary{matrix} %% for matrix
\usepgfplotslibrary{polar} %% for polar plots
\usepgfplotslibrary{fillbetween} %% to shade area between curves in TikZ
\usetkzobj{all}
\usepackage[makeroom]{cancel} %% for strike outs
%\usepackage{mathtools} %% for pretty underbrace % Breaks Ximera
%\usepackage{multicol}
\usepackage{pgffor} %% required for integral for loops



%% http://tex.stackexchange.com/questions/66490/drawing-a-tikz-arc-specifying-the-center
%% Draws beach ball
\tikzset{pics/carc/.style args={#1:#2:#3}{code={\draw[pic actions] (#1:#3) arc(#1:#2:#3);}}}



\usepackage{array}
\setlength{\extrarowheight}{+.1cm}
\newdimen\digitwidth
\settowidth\digitwidth{9}
\def\divrule#1#2{
\noalign{\moveright#1\digitwidth
\vbox{\hrule width#2\digitwidth}}}






\DeclareMathOperator{\arccot}{arccot}
\DeclareMathOperator{\arcsec}{arcsec}
\DeclareMathOperator{\arccsc}{arccsc}

















%%This is to help with formatting on future title pages.
\newenvironment{sectionOutcomes}{}{}


\title{Quadratic Analysis}

\begin{document}

\begin{abstract}
behavior
\end{abstract}
\maketitle





\section{Analysis}

What do we want to know when we analyze functions?

We want to know the 
\begin{itemize}
\item \textbf{\textcolor{red!80!black}{domain}} 
\item \textbf{\textcolor{red!80!black}{range}} 
\item \textbf{\textcolor{red!80!black}{zeros}} 
\item \textbf{\textcolor{red!80!black}{discontinuities}} 
\item \textbf{\textcolor{red!80!black}{singularities}} 
\item \textbf{\textcolor{red!80!black}{intervals of increasing and decreasing}} 
\item \textbf{\textcolor{red!80!black}{global maximum and minimum}} 
\item \textbf{\textcolor{red!80!black}{local maximums and minimums}} 
\item \textbf{\textcolor{red!80!black}{symmetry}} 
\item \textbf{\textcolor{red!80!black}{endbehavior}}  \\
\item \textbf{\textcolor{purple!85!blue}{and we would like a nice graph}} 
\end{itemize}


For a quadratic function much of this information is connected to vertex of the graph, which is why we like the vertex form, which is why we like completing the square.



\subsection{Domain} 

Quadratic functions are defined for all real numbers.  Their natural domain is $\mathbb{R}$. \\




\subsection{Range}

$\blacktriangleright$  \textbf{Vertex Form} \\

The graph of a quadratic function is a parabola, which is easily connected to the completed square form of the formula.

Below is the graph of $y = f(x) = a (x - h)^2 + k$, with $a$, and $h$, and $k$ all real numbers and $a > 0$. The extreme point is called the \textbf{vertex}. If $a<0$, then the parabola would have opened downward and the extreme point would be at the top.

\begin{image}
\begin{tikzpicture}
     \begin{axis}[
                domain=-10:10, ymax=10, xmax=10, ymin=-10, xmin=-10,
                axis lines =center, xlabel=$x$, ylabel=$y$,
                ytick={-10,-8,-6,-4,-2,2,4,6,8,10},
                xtick={-10,-8,-6,-4,-2,2,4,6,8,10},
                ticklabel style={font=\scriptsize},
                every axis y label/.style={at=(current axis.above origin),anchor=south},
                every axis x label/.style={at=(current axis.right of origin),anchor=west},
                axis on top,
                ]


        \addplot [draw=penColor, very thick, smooth, domain=(1:5),<->] {2*(x-3)^2 - 4};
        \addplot [line width=1, gray, dashed,samples=100,domain=(-9.5:9.5)] ({3},{x});
        \addplot [color=penColor2,only marks,mark=*] coordinates{(3,-4)};
        


        \node[penColor] at (axis cs:5,-4) {$(h, k)$};
        %\node[penColor] at (axis cs:5,-9) {$-0.5 x^2 - 5 x + 15.5$};



    \end{axis}
\end{tikzpicture}
\end{image}

The vertex visually encodes the  minimum (or maximum) value of the function. 




If $a<0$, then everything is reveresed.







\begin{image}
\begin{tikzpicture}
     \begin{axis}[
                domain=-10:10, ymax=10, xmax=10, ymin=-10, xmin=-10,
                axis lines =center, xlabel=$x$, ylabel=$y$, 
                ytick={-10,-8,-6,-4,-2,2,4,6,8,10},
                xtick={-10,-8,-6,-4,-2,2,4,6,8,10},
                ticklabel style={font=\scriptsize},
                every axis y label/.style={at=(current axis.above origin),anchor=south},
                every axis x label/.style={at=(current axis.right of origin),anchor=west},
                axis on top,
                ]


        \addplot [draw=penColor, very thick, smooth, domain=(1:5),<->] {-2*(x-3)^2 + 4};
        \addplot [line width=1, gray, dashed,samples=100,domain=(-9.5:9.5)] ({3},{x});
        \addplot [color=penColor2,only marks,mark=*] coordinates{(3,4)};
        


        \node[penColor] at (axis cs:5,4) {$(h, k)$};
        %\node[penColor] at (axis cs:5,-9) {$-0.5 x^2 - 5 x + 15.5$};



    \end{axis}
\end{tikzpicture}
\end{image}











We can see from the formula $f(x) = a (x - h)^2 + k$, that since $(x - h)$ is squared, and thus nonnegative, the range of $f$ depends on the sign of $a$. \\

When $a>0$, the values of $f(x)$ are greater than or equal to $k$. This corresponds to the graph opening up.  The only way to get the least value possible for $f$ is to select $x = h$. That corresponds to the vertex $(k, h)$. \\



When $a<0$, the values of $f(x)$ are less than or equal to $k$. This corresponds to the graph opening down.  The only way to get the greatest value possible for $f$ is to select $x = h$. That corresponds to the vertex $(k, h)$.  \\

We can see that the implied range of a quadratic comes in two types.  

\begin{itemize}
\item The range could be all real numbers greater than or equal to some particular number:  $\{ r \in \textbf{R} \, | \, r \geq k \}$.
\item The range could be all real numbers less than or equal to some particular number:  $\{ r \in \textbf{R} \, | \, r \leq k \}$.
\end{itemize}

If there is a stated domain, then the range will be restricted appropriately. \\






$\blacktriangleright$ \textbf{Symmetry}   \\


If we take two domain numbers equidistant from $h$, like $h - \epsilon$ and $h + \epsilon$, and evaluate $f(x)$ at these two numbers we get the same value.


\begin{align*}
f(h - \epsilon) & = a (h - \epsilon - h)^2 + k \\
& = a (-\epsilon)^2 + k  \\
& = a (\epsilon)^2 + k  \\
& = a (h + \epsilon - h)^2 + k \\
& = f(h + \epsilon)
\end{align*}


The graph is symmetric about the line $x = h$.


When developing the quadratic formula, we had the opportunity to complete the square and that gave us a squared term that looked like 

\[  a \, \left(x + \frac{b}{2a}\right)^2       \]


The domain coordinate of the vertex is $h = \frac{-b}{2a}$.


The graph is symmetric about the line $x = \frac{-b}{2a}$.






The zeros must also be symmetric about $\frac{-b}{2a}$, which we can see in the quadratic formula. \\




\subsection{Zeros}


The quadratic formula is

\[ t  =   \frac{-b \pm \sqrt{b^2 - 4 a c}}{2a}      \]



which we can separate into 



\[ t  =   \frac{-b}{2a} \pm \frac{\sqrt{b^2 - 4 a c}}{2a}      \]


The $\pm$ shows that the zeros are symmetric about $\frac{-b}{2a}$ and the intercepts are symmetric about the line $x = \frac{-b}{2a}$.





\begin{image}
\begin{tikzpicture}
     \begin{axis}[
                domain=-10:10, ymax=10, xmax=10, ymin=-10, xmin=-10,
                axis lines =center, xlabel=$x$, ylabel=$y$,
                ytick={-10,-8,-6,-4,-2,2,4,6,8,10},
                xtick={-10,-8,-6,-4,-2,2,4,6,8,10},
                ticklabel style={font=\scriptsize},
                every axis y label/.style={at=(current axis.above origin),anchor=south},
                every axis x label/.style={at=(current axis.right of origin),anchor=west},
                axis on top,
                ]


        \addplot [draw=penColor, very thick, smooth, domain=(1:5),<->] {-2*(x-3)^2 + 4};
        \addplot [line width=1, gray, dashed,samples=100,domain=(-9.5:9.5)] ({3},{x});
        \addplot [color=penColor2,only marks,mark=*] coordinates{(3,4)};

        \addplot [color=penColor2,only marks,mark=*] coordinates{(1.585,0)};
        \addplot [color=penColor2,only marks,mark=*] coordinates{(4.414,0)};
        


        %\node[penColor] at (axis cs:5,4) {$(h, k))$};
        %\node[penColor] at (axis cs:5,-9) {$-0.5 x^2 - 5 x + 15.5$};



    \end{axis}
\end{tikzpicture}
\end{image}





Working backwards, we can see that if we have the zeros, like from factoring, then we have the intercepts, and the \textbf{line of symmetry} must run in the middle. \\








\subsection{Continuity}

Quadratic functions are continuous functions.  They have no discontinuities or singularities. \\














\subsection{Behavior}



\textbf{\textcolor{blue!55!black}{Increasing and Decreasing}}






The graph vividly suggests that quadratic functions switch from increasing to decreasing (or vice versa) at the symmetric/vertex number in the domain.


\begin{itemize}
\item If the graph is opening up, then the quadratic function is \\

decreasing on $\left( -\infty, \frac{-b}{2a} \right)$ and increasing on $\left( \frac{-b}{2a}, \infty \right)$

\item If the graph is opening down, then the quadratic function is \\

increasing on $\left( -\infty, \frac{-b}{2a} \right)$ and decreasing on $\left( \frac{-b}{2a}, \infty \right)$
\end{itemize}





Increasing and decreasing refer to the rate of change.


\begin{itemize}
\item Increasing is a positive rate of change.
\item Decreasing is a negative rate of change.
\end{itemize}



Now, we can replace our graphical intuition with algebraic rigor. \\ 

We have seen if we write a quadratic function as $f(x) = a (x - h)^2 + k$, then the instantaneous rate of change of $f$ is the linear function $iRoC_f(x) = 2 a (x - h)$. The values of $iRoC$ are the slopes of the lines tangent to the parabola.


Since $iRoC_f$ is a linear function, its graph is a line.


Here is a graph of both the parabola for $f$ and the line for $iRoC_f$.







\begin{image}
\begin{tikzpicture}
     \begin{axis}[
                domain=-10:10, ymax=10, xmax=10, ymin=-10, xmin=-10,
                axis lines =center, xlabel=$x$, ylabel=$y$,
                ytick={-10,-8,-6,-4,-2,2,4,6,8,10},
                xtick={-10,-8,-6,-4,-2,2,4,6,8,10},
                ticklabel style={font=\scriptsize},
                every axis y label/.style={at=(current axis.above origin),anchor=south},
                every axis x label/.style={at=(current axis.right of origin),anchor=west},
                axis on top,
                ]


        \addplot [draw=penColor, very thick, smooth, domain=(1:5),<->] {2*(x-3)^2 - 4};
        \addplot [line width=1, gray, dashed,samples=100,domain=(-9.5:9.5)] ({3},{x});
        \addplot [color=penColor,only marks,mark=*] coordinates{(3,-4)};
        
        \addplot [draw=penColor2, very thick, smooth, domain=(1:5),<->] {4*(x-3)};

        \node[penColor] at (axis cs:5,-4) {$(h, k)$};
        %\node[penColor] at (axis cs:5,-9) {$-0.5 x^2 - 5 x + 15.5$};



    \end{axis}
\end{tikzpicture}
\end{image}
Our linear rate of change function now informs us about the behavior of $f$. \\



\begin{conclusion}  \textbf{\textcolor{green!50!black}{Behavior}}

$\blacktriangleright$ When the linear rate of change function $iRoC_f(x) < 0$, then $f(x)$ is decreasing. \\

$\blacktriangleright$ When the linear rate of change function $iRoC_f(x) > 0$, then $f(x)$ is increasing. \\

$\blacktriangleright$ When the linear rate of change function $iRoC_f(x) = 0$, then $f(x)$ is neither increasing nor decreasing and the graph of $y = f(x)$ is flat. 

\end{conclusion}


As we can see, the behavior of our function, $f(x)$, can change drastically where $iRoC_f(x) = 0$.  Such domain numbers deserve a special name.



\begin{definition} \textbf{\textcolor{green!50!black}{Critical Number}}  


Let $f$ be a function. Let $x_0$ be a number in the domain of $f$ such that $iRoC_f(x_0) = 0$ or $iRoC_f(x_0)$ does not exist.

Then $x_0$ is called a \textbf{critical number}.


\end{definition}
\textbf{Note: } Domain numbers where $iRoC_f$ doesn't exist are also places where a function's behavior can change drastically.



\begin{procedure} \textbf{\textcolor{blue!75!black}{iRoC for Quadratic Functions}} 



Let $Q$ be a quadratic function.

Then $Q(x) = a (x - h)^2 + k$, for some $a$, $h$, and $k$ with $a \ne 0$. \\

Then, $iRoC_Q(x) = 2 a (x - h)$. \\


\textbf{Procedure:}. It appears that the $2$ in the exponent has slid down in front of the leading coefficient and the constant term $k$ has been removed.



\end{procedure}



We have a procedure for obtaining the $iRoC$ of a quadratic function, when the formula is in vertex form (completed square form). \\

What about standard form? \\



\begin{align*}
Q(x) & = a (x - h)^2 + k \\
     & = a \, x^2 - 2 \, a \, h \, x + a \, h^2 \, k \\
     & = a \, x^2  + (- 2 \, a \, h) x + (a \, h^2 \, k) 
\end{align*}


\begin{align*}
iRoC_Q(x) &= 2 a (x - h) \\
          & = 2 \, a \, x - 2 \, a \, h  \\
\end{align*}





\begin{procedure} \textbf{\textcolor{blue!75!black}{iRoC for Quadratic Functions}} 



Let $Q$ be a quadratic function.

Then $Q(x) = a \, x^2 + b \, x + c$, for some $a$, $b$, and $c$ with $a \ne 0$. \\

Then, $iRoC_Q(x) = 2 \, a \, x + b$. \\


\textbf{Procedure:}. It appears that the $2$ in the exponent has slid down in front of the leading coefficient, the linear coefficent has remained, and the constant term as been removed. 



\end{procedure}

There is probably an overall pattern going on here. \\










\textbf{\textcolor{blue!55!black}{Maximum and Minimum}}



The maximum and minimum values of a quadratic function, $f$, are visually encoded in the highest and lowest points on the  graph, which is the vertex of the parabola.

Depending on the sign of $a$, the maximum or minimum value of $f(x) = a \, x^2 + b \, x + c$ occurs at $\frac{-b}{2a}$. The maximum or minimum value is $f\left( \frac{-b}{2a} \right)$


Depending on the sign of $a$, the maximum or minimum value of $f(x) = a \, (x - h)^2 + k$ is $k$ and occurs at $h = \frac{-b}{2a}$. 







We can also look at the linear $iRoC_f(x)$ function.  Where $iRoC_f(x) = 0$ is where the vertex is located, which encodes the maximum or minimum value of $f$.





\begin{align*}
iRoC_Q(x)       &= 0  \\
2 \, a \, x + b  & = 0  \\
x     &=  \frac{-b}{2a}
\end{align*}




\textbf{\textcolor{blue!55!black}{$\blacktriangleright$}}  $\frac{-b}{2a}$ is the critical number for a quadratic function given in standard form: $a \, x^2 + b \, x + c$.



\textbf{\textcolor{blue!55!black}{$\blacktriangleright$}}  $h$ is the critical number for a quadratic function given in vertex form: $a \, (x - h)^2 + c$.



\textbf{\textcolor{blue!55!black}{$\blacktriangleright$}}  $\frac{r_1 + r_2}{2}$ is the critical number for a quadratic function given in factored form: $a \, (x - r_1) (x - r_2)$.







\begin{center}
\textbf{\textcolor{green!50!black}{ooooo=-=-=-=-=-=-=-=-=-=-=-=-=ooOoo=-=-=-=-=-=-=-=-=-=-=-=-=ooooo}} \\

more examples can be found by following this link\\ \link[More Examples of Quadratic Behavior]{https://ximera.osu.edu/csccmathematics/precalculus1/precalculus1/quadraticBehavior/examples/exampleList}

\end{center}




\end{document}




