\documentclass{ximera}


\graphicspath{
  {./}
  {ximeraTutorial/}
  {basicPhilosophy/}
}

\newcommand{\mooculus}{\textsf{\textbf{MOOC}\textnormal{\textsf{ULUS}}}}

\usepackage{tkz-euclide}\usepackage{tikz}
\usepackage{tikz-cd}
\usetikzlibrary{arrows}
\tikzset{>=stealth,commutative diagrams/.cd,
  arrow style=tikz,diagrams={>=stealth}} %% cool arrow head
\tikzset{shorten <>/.style={ shorten >=#1, shorten <=#1 } } %% allows shorter vectors

\usetikzlibrary{backgrounds} %% for boxes around graphs
\usetikzlibrary{shapes,positioning}  %% Clouds and stars
\usetikzlibrary{matrix} %% for matrix
\usepgfplotslibrary{polar} %% for polar plots
\usepgfplotslibrary{fillbetween} %% to shade area between curves in TikZ
\usetkzobj{all}
\usepackage[makeroom]{cancel} %% for strike outs
%\usepackage{mathtools} %% for pretty underbrace % Breaks Ximera
%\usepackage{multicol}
\usepackage{pgffor} %% required for integral for loops



%% http://tex.stackexchange.com/questions/66490/drawing-a-tikz-arc-specifying-the-center
%% Draws beach ball
\tikzset{pics/carc/.style args={#1:#2:#3}{code={\draw[pic actions] (#1:#3) arc(#1:#2:#3);}}}



\usepackage{array}
\setlength{\extrarowheight}{+.1cm}
\newdimen\digitwidth
\settowidth\digitwidth{9}
\def\divrule#1#2{
\noalign{\moveright#1\digitwidth
\vbox{\hrule width#2\digitwidth}}}






\DeclareMathOperator{\arccot}{arccot}
\DeclareMathOperator{\arcsec}{arcsec}
\DeclareMathOperator{\arccsc}{arccsc}

















%%This is to help with formatting on future title pages.
\newenvironment{sectionOutcomes}{}{}


\title{Quadratic Behavior}

\begin{document}

\begin{abstract}
%Stuff can go here later if we want!
\end{abstract}
\maketitle







We have two different thoughts about functions.


$\blacktriangleright$ \textbf{Numbers:}  Functions consist of individual pairs of individual numbers.  When we evaluate functions, we think of each pair one at a time.  The graphs of functions are individual dots plotted to represent points, which visually encode the function pairs. 







$\blacktriangleright$ \textbf{Intervals:} As we examine functions, we notice patterns, which we call behavior. We naturally group many pairs together where the function has similar behavior.  Collecting pairs forms domain intervals and we begin thinking in terms of intervals.





In particular, we calculate rate of change over intervals. 



The rate of change of the function $f$ over the interval $[a, b]$ is given by $\frac{f(b) - f(a)}{b - a}$.




However, intervals are just collections of individual numbers.  Is there a way to translate this rate of change idea down to single numbers?

The answer is yes.  \\


We will begin this thought here with quadratic functions. We'll extend this thought a bit throughout the course. Then Calculus will answer the question fully with the derivative.













\subsection{Learning Outcomes}

\begin{sectionOutcomes}
In this section, students will 

\begin{itemize}
\item develop the $iRoC$ function.
\item analyze quadratic functions.
\end{itemize}
\end{sectionOutcomes}















\begin{center}
\textbf{\textcolor{green!50!black}{ooooo=-=-=-=-=-=-=-=-=-=-=-=-=ooOoo=-=-=-=-=-=-=-=-=-=-=-=-=ooooo}} \\

more examples can be found by following this link\\ \link[More Examples of Quadratic Behavior]{https://ximera.osu.edu/csccmathematics/precalculus1/precalculus1/quadraticBehavior/examples/exampleList}

\end{center}





\end{document}
