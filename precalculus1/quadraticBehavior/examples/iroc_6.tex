\documentclass{ximera}


\graphicspath{
  {./}
  {ximeraTutorial/}
  {basicPhilosophy/}
}

\newcommand{\mooculus}{\textsf{\textbf{MOOC}\textnormal{\textsf{ULUS}}}}

\usepackage{tkz-euclide}\usepackage{tikz}
\usepackage{tikz-cd}
\usetikzlibrary{arrows}
\tikzset{>=stealth,commutative diagrams/.cd,
  arrow style=tikz,diagrams={>=stealth}} %% cool arrow head
\tikzset{shorten <>/.style={ shorten >=#1, shorten <=#1 } } %% allows shorter vectors

\usetikzlibrary{backgrounds} %% for boxes around graphs
\usetikzlibrary{shapes,positioning}  %% Clouds and stars
\usetikzlibrary{matrix} %% for matrix
\usepgfplotslibrary{polar} %% for polar plots
\usepgfplotslibrary{fillbetween} %% to shade area between curves in TikZ
\usetkzobj{all}
\usepackage[makeroom]{cancel} %% for strike outs
%\usepackage{mathtools} %% for pretty underbrace % Breaks Ximera
%\usepackage{multicol}
\usepackage{pgffor} %% required for integral for loops



%% http://tex.stackexchange.com/questions/66490/drawing-a-tikz-arc-specifying-the-center
%% Draws beach ball
\tikzset{pics/carc/.style args={#1:#2:#3}{code={\draw[pic actions] (#1:#3) arc(#1:#2:#3);}}}



\usepackage{array}
\setlength{\extrarowheight}{+.1cm}
\newdimen\digitwidth
\settowidth\digitwidth{9}
\def\divrule#1#2{
\noalign{\moveright#1\digitwidth
\vbox{\hrule width#2\digitwidth}}}






\DeclareMathOperator{\arccot}{arccot}
\DeclareMathOperator{\arcsec}{arcsec}
\DeclareMathOperator{\arccsc}{arccsc}

















%%This is to help with formatting on future title pages.
\newenvironment{sectionOutcomes}{}{}



\author{Lee Wayand}

\begin{document}




\begin{exercise} 









Let $M(k) = -3(k+2)^2 - 5$ be a quadratic function. \\




Since $M(k)$ is a quadratic function, its domain is $\left( \answer{-\infty}, \answer{\infty} \right)$. \\


Since $M(k)$ is a quadratic function, it is continuous and has no discontinuities or singularities. \\





\begin{question} Zeros



In numerical order, the zeros of $M(k)$ are

\[
\answer{-2 - \sqrt{\frac{5}{3}}} \, \text{ and } \, \answer{-2 + \sqrt{\frac{5}{3}}}
\]

\end{question}






\begin{question} End-Behavior



Since $M(k)$ is a quadratic function with a negative leading coefficient,

\[
\lim\limits_{k \to -\infty}M(k) = \answer{-\infty} \, \text{ and } \, \lim\limits_{k \to \infty}M(k) = \answer{-\infty}
\]


This means $M(k)$ has no global \wordChoice{\choice[correct]{maximum} \choice{minimum}} 

\end{question}




\begin{question}   Rate of Change



$iRoC_M(k) = \answer{-6(k+2)}$. \\



$iRoC_M(k) = 0$  at  $k = \answer{-2}$. \\


$\answer{-2}$ is the only critical number of $M(k)$. \\


$iRoC_M(k) > 0$ on $\left( \answer{-\infty}, \answer{-2} \right)$. \\


$iRoC_M(k) < 0$ on $\left( \answer{-2}, \answer{\infty} \right)$. \\

\end{question}





\begin{question} Behavior



$M(k)$ increases on $\left( \answer{-\infty}, \answer{-2} \right)$. \\


$M(k)$ decreases on $\left( \answer{-2}, \answer{\infty} \right)$. \\


The maximum of $M(k)$ is located at  $k = \answer{-2}$. \\


The maximum of $M(k)$ is $\answer{-5}$. \\


This global maximum is also a local maximum.

\end{question}










\begin{question} Range



The range of $M(k)$ is 

\[
\left( \answer{-\infty}, -5 \right]
\]



\end{question}









\end{exercise}


\end{document}