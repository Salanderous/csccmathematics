\documentclass{ximera}


\graphicspath{
  {./}
  {ximeraTutorial/}
  {basicPhilosophy/}
}

\newcommand{\mooculus}{\textsf{\textbf{MOOC}\textnormal{\textsf{ULUS}}}}

\usepackage{tkz-euclide}\usepackage{tikz}
\usepackage{tikz-cd}
\usetikzlibrary{arrows}
\tikzset{>=stealth,commutative diagrams/.cd,
  arrow style=tikz,diagrams={>=stealth}} %% cool arrow head
\tikzset{shorten <>/.style={ shorten >=#1, shorten <=#1 } } %% allows shorter vectors

\usetikzlibrary{backgrounds} %% for boxes around graphs
\usetikzlibrary{shapes,positioning}  %% Clouds and stars
\usetikzlibrary{matrix} %% for matrix
\usepgfplotslibrary{polar} %% for polar plots
\usepgfplotslibrary{fillbetween} %% to shade area between curves in TikZ
\usetkzobj{all}
\usepackage[makeroom]{cancel} %% for strike outs
%\usepackage{mathtools} %% for pretty underbrace % Breaks Ximera
%\usepackage{multicol}
\usepackage{pgffor} %% required for integral for loops



%% http://tex.stackexchange.com/questions/66490/drawing-a-tikz-arc-specifying-the-center
%% Draws beach ball
\tikzset{pics/carc/.style args={#1:#2:#3}{code={\draw[pic actions] (#1:#3) arc(#1:#2:#3);}}}



\usepackage{array}
\setlength{\extrarowheight}{+.1cm}
\newdimen\digitwidth
\settowidth\digitwidth{9}
\def\divrule#1#2{
\noalign{\moveright#1\digitwidth
\vbox{\hrule width#2\digitwidth}}}






\DeclareMathOperator{\arccot}{arccot}
\DeclareMathOperator{\arcsec}{arcsec}
\DeclareMathOperator{\arccsc}{arccsc}

















%%This is to help with formatting on future title pages.
\newenvironment{sectionOutcomes}{}{}


\title{Tangents}

\begin{document}

\begin{abstract}
intervals to points
\end{abstract}
\maketitle





\subsection{Intervals and Secants}



Suppose we have a qudratic function $Q(x) = a (x - h)^2 + k$, with $a > 0$. 


Then its graph is a parabola.







\begin{image}
\begin{tikzpicture}
     \begin{axis}[
                domain=-5:10, ymax=10, xmax=10, ymin=-5, xmin=-5,
                axis lines =center, xlabel=$x$, ylabel=$y$,
                ytick={-4,-2,2,4,6,8,10},
                xtick={-4,-2,2,4,6,8,10},
                ticklabel style={font=\scriptsize},
                every axis y label/.style={at=(current axis.above origin),anchor=south},
                every axis x label/.style={at=(current axis.right of origin),anchor=west},
                axis on top,
                ]


        \addplot [draw=penColor, very thick, smooth, domain=(-1:7),<->] {0.5*(x-3)^2 + 2};
        \addplot [line width=1, gray, dashed,samples=100,domain=(-9.5:9.5)] ({3},{x});
        


        \addplot [color=penColor,only marks,mark=*] coordinates{(3,2)};
        \node[penColor] at (axis cs:4,1.5) {$(h, k)$};
        %\node[penColor] at (axis cs:5,-9) {$-0.5 x^2 - 5 x + 15.5$};



    \end{axis}
\end{tikzpicture}
\end{image}
We can examine the rate of change over any interval.


Let's examine the interval $[2, 5]$.


$\blacktriangleright$ \textbf{\textcolor{blue!55!black}{Algebraically}}, the rate of change of $Q(x)$ over the interval $[2,5]$ is given by 

\[
\frac{Q(5) - Q(2)}{5 - 2} 
\]



$\blacktriangleright$ \textbf{\textcolor{blue!55!black}{Geometrically}}, this is the slope of the \textbf{secant} line running through the points $(2, Q(2))$ and $(5, Q(5))$.














\begin{image}
\begin{tikzpicture}
     \begin{axis}[
                domain=-5:10, ymax=10, xmax=10, ymin=-5, xmin=-5,
                axis lines =center, xlabel=$x$, ylabel=$y$,
                ytick={-4,-2,2,4,6,8,10},
                xtick={-4,-2,2,4,6,8,10},
                ticklabel style={font=\scriptsize},
                every axis y label/.style={at=(current axis.above origin),anchor=south},
                every axis x label/.style={at=(current axis.right of origin),anchor=west},
                axis on top,
                ]


        \addplot [draw=penColor, very thick, smooth, domain=(-1:7),<->] {0.5*(x-3)^2 + 2};
        \addplot [line width=1, gray, dashed,samples=100,domain=(-9.5:9.5)] ({3},{x});
        

        \addplot [color=penColor2,only marks,mark=*] coordinates{(5,4)};
        \addplot [color=penColor2,only marks,mark=*] coordinates{(2,2.5)};
        
        \addplot [draw=penColor2, very thick, smooth, domain=(-2:9),<->] {0.5*(x-5) + 4};

        \addplot [color=penColor,only marks,mark=*] coordinates{(3,2)};
        \node[penColor] at (axis cs:4,1.5) {$(h, k)$};
        %\node[penColor] at (axis cs:5,-9) {$-0.5 x^2 - 5 x + 15.5$};



    \end{axis}
\end{tikzpicture}
\end{image}

Intervals and secants are algebraic and geometric partners. Secants give a picture of rates of change over an interval.



What if we push the secant over a little bit until it becomes a tangent line?







\begin{image}
\begin{tikzpicture}
     \begin{axis}[
                domain=-5:10, ymax=10, xmax=10, ymin=-5, xmin=-5,
                axis lines =center, xlabel=$x$, ylabel=$y$,
                ytick={-4,-2,2,4,6,8,10},
                xtick={-4,-2,2,4,6,8,10},
                ticklabel style={font=\scriptsize},
                every axis y label/.style={at=(current axis.above origin),anchor=south},
                every axis x label/.style={at=(current axis.right of origin),anchor=west},
                axis on top,
                ]


        \addplot [draw=penColor, very thick, smooth, domain=(-1:7),<->] {0.5*(x-3)^2 + 2};
        \addplot [line width=1, gray, dashed,samples=100,domain=(-9.5:9.5)] ({3},{x});
        

        \addplot [color=penColor2,only marks,mark=*] coordinates{(3.5,2.125)};
        
        \addplot [draw=penColor2, very thick, smooth, domain=(-2:9),<->] {0.5*(x-3.5) + 2.125};

        %\addplot [color=penColor,only marks,mark=*] coordinates{(3,2)};
        %\node[penColor] at (axis cs:4,1.5) {$(h, k)$};
        %\node[penColor] at (axis cs:5,-9) {$-0.5 x^2 - 5 x + 15.5$};



    \end{axis}
\end{tikzpicture}
\end{image}

Now it is a picture of the rate of change over the interval $\left[ \tfrac{7}{2}, \tfrac{7}{2} \right]$. \\


How should we interpret this? \\


What is the rate of change \textbf{\textcolor{red!90!darkgray}{AT}} a point? \\


You cannot calculate the rate of change over an interval with $0$ length.  Therefore, we will invent an interpretation for this rate of change at a point, i.e. \textit{an instantaneous rate of change}.















\subsection{Tangent Lines}

Suppose $f$ is a function.  The graph of $f$ is the collection of points whose coordinates are pairs in $f$.  That is, their coordinates look like $(d, f(d))$.

Suppose $a$ and $b$ are two distinct domain numbers of $f$.  Then the line through $(a, f(a))$ and $(b, f(b))$ is called a secant line.  The slope of this secant line equals the rate of change of $f$ over the interval $[a, b]$.

Tangent lines are degenerate secants. Secant lines need two points on the graph of a function.  A tangent line is a secant line where the two points are the same point. Tangent lines are secant lines created from an interval of length $0$.  But tangent lines are still lines.  They still have a slope.


If the slope of a secant line corresponds to the function's rate of change over an interval, then the slope of a tangent line corresponds to the function's rate of change at a single number.


We'll call this the \textbf{\textcolor{purple!85!blue}{instantaneous rate of change}} at the domain number corresponding to the tangent point.




\begin{definition} \textbf{\textcolor{green!50!black}{Instantaneous Rate of Change}}  


Let $f$ be a function. Let $a$ be a number in the domain of $f$.

If the graph of $y = f(x)$ has a non-vertical tangent line at the point $(a, f(a))$, then the slope of this tangent line is the \textbf{instantaneous rate of change} of $f$ \textbf{at} a.


\end{definition}














\subsection{Quadratics}

We'll begin our investigation into instantaneous rate of change with quadratic functions, whose graphs are parabolas (which never have vertical tangent lines). \\


We need a method of obtaining the slope of tangent lines to parabolas.







$\blacktriangleright$ \textbf{Basic Quadratic}


We'll begin with our basic quadratic function: \textbf{\textcolor{purple!85!blue}{$Q(x) = x^2$}} and its graph \textbf{\textcolor{purple!85!blue}{$y = x^2$}}. \\


We are investigating the graph of $y = x^2$, which is a parabola whose vertex is $(0, 0)$. \\

Let's select a domain number of $Q$, call it $x_0$.   This corresponds to the point $(x_0, y_0) = (x_0, (x_0)^2)$ on the graph of $Q$.

Let's add a picture of the tangent line at $(x_0, y_0) = (x_0, (x_0)^2)$ to our graph of $Q$.


\begin{image}
\begin{tikzpicture}
     \begin{axis}[
                domain=-5:5, ymax=10, xmax=5, ymin=-5, xmin=-5,
                axis lines =center, xlabel=$x$, ylabel=$y$,
                ytick={-4,-2,2,4,6,8,10},
                xtick={-4,-2,2,4,6,8,10},
                ticklabel style={font=\scriptsize},
                every axis y label/.style={at=(current axis.above origin),anchor=south},
                every axis x label/.style={at=(current axis.right of origin),anchor=west},
                axis on top,
                ]


        \addplot [draw=penColor, very thick, smooth, domain=(-3:3),<->] {x^2};
        \addplot [line width=1, gray, dashed,samples=100,domain=(-5:9.5)] ({0},{x});
        

        \addplot [color=penColor2,only marks,mark=*] coordinates{(2,4)};
        \node[penColor2] at (axis cs:3,4) {$(x_0, y_0)$};
        \addplot [draw=penColor2, very thick, smooth, domain=(0.5:3),<->] {4*x - 4};
        


        %\addplot [color=penColor,only marks,mark=*] coordinates{(3,2)};
        %\node[penColor] at (axis cs:2,1.5) {$(h, k)$};




    \end{axis}
\end{tikzpicture}
\end{image}

The tangent line is the graph of a linear function. Let's call this linear function $T$.


$T$ is a linear function. Therefore, the formula for $T$, would look like $T(x) = m(x - x_0) + y_0$, for some $m$.  According to our definition, $m$ is the instantaneous rate of change of $Q$ at $x_0$.  How do we determine its exact value?


$\blacktriangleright$ We would like a way to obtain the exact value of $m$. \\



To get the exact value of $m$, we are going to create a new quadratic function.



\begin{explanation}


\[
Q_{new}(x) = Q(x) - T(x) = x^2 - (m(x - x_0) + y_0)
\]



$Q_{new}(x)$ is a quadratic function and $Q_{new}(x_0) = Q(x_0) - T(x_0) = y_0 - y_0 = 0$. \\

We can also see from the graph above that $Q(x) > T(x)$ when $x \ne x_0$, which means $Q_{new}(x) \geq 0$\\


So, its graph is a parabola with vertex, $(x_0, 0)$, on the $x$-axis.  The graph of $y = Q(x)$ has slide to the left a distance of $x_0$.  \\


Therefore, the vertex form of $Q_{new}(x)$ is $Q_{new}(x) = (x - x_0)^2 = (x - x_0)(x - x_0)$  \\






\[
Q_{new}(x) = Q(x) - T(x) = x^2 - (m(x - x_0) + y_0) = (x - x_0)^2
\]




\[
x^2 - (m(x - x_0) + y_0) = (x - x_0)^2 = x^2 - 2 x_0 x + x_0^2
\]



\[
 -m \, x + m \, x_0 - y_0) =   - 2 \, x_0 \, x + x_0^2
\]


For this to happen, we must have $-m \, x = - 2 \, x_0 \, x$ or \textbf{$m = 2 \, x_0$}. \\


\end{explanation}


At any point $(x_0, y_0)$ on the graph of $y = x^2$, the slope of the tangent line is $2 \, x_0$.  The slope of the tangent line is always twice the $x$-coordinate of the tangent point.




\begin{conclusion} \textbf{\textcolor{green!50!black}{Slope of a Tangent Line}} 


The graph of $y = x^2$ is a parabola.

Let $(x_0, y_0)$ be a point on this parabola.

Then the slope of the tangent line to the parabola at $(x_0, y_0)$ is given by 



\[ \text{slope } \, = \, 2 x_0  \]


\end{conclusion}
This for any point on the parabola.



























$\blacktriangleright$ \textbf{Stretching}



We can vertically stretch our basic quadratic parabola by multiplying all of the $y$-coordinates by a constant, $a \ne 0$. \\


The new parabola has an equation of the form $y = a \, x^2$. \\

These are graphs of quadratic functions of the form $Q(x) = a \, x^2$. \\


Let's go through the same process. \\


\begin{explanation}

The tangent line at $(x_0, y_0)$ is the graph of a linear function. Let's call this linear function $T$.


$T$ is a linear function. Therefore, the formula for $T$, would look like $T(x) = m(x - x_0) + y_0$, for some $m$.  According to our definition, $m$ is the instantaneous rate of change of $Q$ at $x_0$.  How do we determine its exact value?


$\blacktriangleright$ We would like a way to obtain the exact value of $m$. \\



To get the exact value of $m$, we are going to create a new quadratic function.



\[
Q_{new}(x) = Q(x) - T(x) = a \, x^2 - (m(x - x_0) + y_0)
\]



$Q_{new}(x)$ is a quadratic function and $Q_{new}(x_0) = Q(x_0) - T(x_0) = y_0 - y_0 = 0$. \\

We can also see from the graph that $Q(x) > T(x)$ when $x \ne x_0$, which means $Q_{new}(x) \geq 0$.\\


So, its graph is a parabola with vertex, $(x_0, 0)$, on the $x$-axis.  \\


Therefore, the vertex form of $Q_{new}(x)$ is $Q_{new}(x) = a \, (x - x_0)^2 = (x - x_0)(x - x_0)$  \\






\[
Q_{new}(x) = Q(x) - T(x) = a \, x^2 - (m(x - x_0) + y_0) = a \, (x - x_0)^2
\]




\[
a \, x^2 - (m(x - x_0) + y_0) = a \, (x - x_0)^2 = a \, x^2 - 2a \, x_0 x + a \, x_0^2
\]



\[
 -m \, x + m \, x_0 - y_0) =   - 2a \, x_0 \, x + a \, x_0^2
\]


For this to happen, we must have $-m \, x = - 2a \, x_0 \, x$ or \textbf{$m = 2a \, x_0$}. \\


\end{explanation}

At any point $(x_0, y_0)$ on the graph of $y = a \, x^2$, the slope of the tangent line is $2 \, a \, x_0$.  The slope of the tangent line is always $2a$ times the $x$-coordinate of the tangent point.










\begin{conclusion} \textbf{\textcolor{green!50!black}{Slope of a Tangent Line}} 


The graph of $y = a \, x^2$ is a parabola.

Let $(x_0, y_0)$ be a point on this parabola.

Then the slope of the tangent line to the parabola at $(x_0, y_0)$ is given by 



\[ \text{slope } \, = \, 2 \, a \, x_0  \]


\end{conclusion}
This for any point on the parabola.




















$\blacktriangleright$ \textbf{Shifting}




Beginning with $Q(x) = a \, x^2$ and its parabola $y = a \, x^2$, we can shift horizontally and vertically. \\





If we keep the shape of the parabola, $y = a \, x^2$, intact and shift its position, then the new parabola has an equation of the form $y = a \, (x - h)^2 + k$.

These are graphs of quadratic functions of the form $Q_{new}(x) = a \, (x - h)^2 + k$. \\


If we pick any point, $(x_0, y_0)$, on this parabola, then the slope of the tangent line is the same as the slope of the tangent line for $y = a \, x^2$ at the point $(x_0 - h, y_0 - k)$.  \\

From above, we know this slope is \textbf{$2 \, a \, (x_0 - h)$}.






\begin{conclusion} \textbf{\textcolor{green!50!black}{Slope of a Tangent Line}} 


The graph of $y = a \, (x - h)^2 + k$ is a parabola.

Let $(x_0, y_0)$ be a point on this parabola.

Then the slope of the tangent line to the parabola at $(x_0, y_0)$ is given by 



\[ \text{slope } \, = \, 2 \, a \, (x_0 - h)  \]


\end{conclusion}
This for any point on the parabola.




































\begin{theorem} \textbf{\textcolor{green!50!black}{Slope of a Tangent Line (Parabola)}} 


The graph of $y = a (x - h)^2 + k$ is a parabola.

Let $(x_0, y_0)$ be a point on this parabola.

Then the slope of the tangent line to the parabola at $(x_0, y_0)$ is given by 



\[ 2 \, a \, (x_0 - h) \]


\end{theorem}
This for any point on the parabola.


\textbf{Note:} If we look at the vertex $(x_0, y_0) = (h, k)$, then $2 a (x_0 - h) = 2 a (h - h) = 0$ and the slope of the tangent line is $0$, just like we had reasoned before.





\section{iRoC}

Let $Q(x) = a (x - h)^2 + k$ be any quadratic function.

Every point, $(x_0, y_0)$, on the graph of $y = Q(x)$ has a tangent line.

Each tangent line has a slope, which we are calling the instantaneous rate of change of $Q$ at $x_0$.


\textbf{\textcolor{red!90!darkgray}{$\blacktriangleright$}} We can create a new function from this.



\begin{definition} \textbf{\textcolor{green!50!black}{iRoC}}  


Given a quadratic function, $Q(x) = a (x - h)^2 + k$, we define \textbf{the instantaneous rate of change of Q} to be the slope of the tangent line on the graph of $y = Q(x)$ at the point $(x, y)$.

$iRoC_Q(x)$ is a linear function given by 

\[  iRoC_Q(x) = 2 a (x-h) \]

\end{definition}


We now have a function, with a formula, for measuring the instantaneous rate of change of any quadratic function at any domain number. \\


\textbf{\textcolor{red!90!darkgray}{$\blacktriangleright$}} The instantaneous rate of change of the quadratic function, $Q(x) = a \, (x - h)^2 + k$, is the linear function, $iRoC_Q(x) = 2 \, a \, (x-h)$. \\




We already have a function, with a formula, for measuring the instantaneous rate of change of any linear function at any domain number. \\ 


\textbf{\textcolor{red!90!darkgray}{$\blacktriangleright$}} The instantaneous rate of change of the linear function, $L(x) = m \, x + b$, is the constant function, $iRoC_L(x) = m$. \\



\textbf{\textcolor{red!90!darkgray}{$\blacktriangleright$}} The instantaneous rate of change of the constant function, $C(x) = c$, is the zero function, $iRoC_C(x) = 0$. \\



















\begin{center}
\textbf{\textcolor{green!50!black}{ooooo=-=-=-=-=-=-=-=-=-=-=-=-=ooOoo=-=-=-=-=-=-=-=-=-=-=-=-=ooooo}} \\

more examples can be found by following this link\\ \link[More Examples of Quadratic Behavior]{https://ximera.osu.edu/csccmathematics/precalculus1/precalculus1/quadraticBehavior/examples/exampleList}

\end{center}






\end{document}




