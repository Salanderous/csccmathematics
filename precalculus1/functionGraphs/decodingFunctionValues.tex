\documentclass{ximera}


\graphicspath{
  {./}
  {ximeraTutorial/}
  {basicPhilosophy/}
}

\newcommand{\mooculus}{\textsf{\textbf{MOOC}\textnormal{\textsf{ULUS}}}}

\usepackage{tkz-euclide}\usepackage{tikz}
\usepackage{tikz-cd}
\usetikzlibrary{arrows}
\tikzset{>=stealth,commutative diagrams/.cd,
  arrow style=tikz,diagrams={>=stealth}} %% cool arrow head
\tikzset{shorten <>/.style={ shorten >=#1, shorten <=#1 } } %% allows shorter vectors

\usetikzlibrary{backgrounds} %% for boxes around graphs
\usetikzlibrary{shapes,positioning}  %% Clouds and stars
\usetikzlibrary{matrix} %% for matrix
\usepgfplotslibrary{polar} %% for polar plots
\usepgfplotslibrary{fillbetween} %% to shade area between curves in TikZ
\usetkzobj{all}
\usepackage[makeroom]{cancel} %% for strike outs
%\usepackage{mathtools} %% for pretty underbrace % Breaks Ximera
%\usepackage{multicol}
\usepackage{pgffor} %% required for integral for loops



%% http://tex.stackexchange.com/questions/66490/drawing-a-tikz-arc-specifying-the-center
%% Draws beach ball
\tikzset{pics/carc/.style args={#1:#2:#3}{code={\draw[pic actions] (#1:#3) arc(#1:#2:#3);}}}



\usepackage{array}
\setlength{\extrarowheight}{+.1cm}
\newdimen\digitwidth
\settowidth\digitwidth{9}
\def\divrule#1#2{
\noalign{\moveright#1\digitwidth
\vbox{\hrule width#2\digitwidth}}}






\DeclareMathOperator{\arccot}{arccot}
\DeclareMathOperator{\arcsec}{arcsec}
\DeclareMathOperator{\arccsc}{arccsc}

















%%This is to help with formatting on future title pages.
\newenvironment{sectionOutcomes}{}{}


\title{Decoding Visually}


\begin{document}

\begin{abstract}
dots to pairs
\end{abstract}
\maketitle



















The graph below is the graph of $y = m(x)$.  One of the dots has been highlighted. This dot visually encodes a function pair.  How do we decipher the pair represented by this dot?
\begin{image}
\begin{tikzpicture}
\begin{axis}[
            domain=0:4, ymax=5, xmax=5, ymin=-5, xmin=-5,
            axis lines =center, xlabel=$x$, ylabel=${y = m(x)}$,
            every axis y label/.style={at=(current axis.above origin),anchor=south},
            every axis x label/.style={at=(current axis.right of origin),anchor=west},
            axis on top,
          ]

        \addplot [very thick,penColor, smooth,domain=(-3:4)] {3*sin(deg(x)) - 1};
        \addplot [color=penColor,only marks,mark=*] coordinates{(-3,-1.42)};
        \addplot [color=penColor,only marks,mark=*] coordinates{(4,-3.27)};
          
        \addplot [color=penColor,only marks,mark=*] coordinates{(-1,-3.52)};


\end{axis}
\end{tikzpicture}
\end{image}

We need the coordinates of this dot.  To obtain these, we move perpendicularly from the dot to the axes.


\begin{image}
\begin{tikzpicture}
\begin{axis}[
            domain=0:4, ymax=5, xmax=5, ymin=-5, xmin=-5,
            axis lines =center, xlabel=$x$, ylabel=${y = m(x)}$,
            every axis y label/.style={at=(current axis.above origin),anchor=south},
            every axis x label/.style={at=(current axis.right of origin),anchor=west},
            axis on top,
          ]

        \addplot [very thick,penColor, smooth,domain=(-3:4)] {3*sin(deg(x)) - 1};
        \addplot [color=penColor,only marks,mark=*] coordinates{(-3,-1.42)};
        \addplot [color=penColor,only marks,mark=*] coordinates{(4,-3.27)};
          
        \addplot [color=penColor,only marks,mark=*] coordinates{(-1,-3.52)};
        \addplot [line width=1, penColor2, smooth,samples=100,domain=(-1:0.1)] ({x},{-3.52});
        \addplot [line width=1, penColor2, smooth,samples=100,domain=(-3.52:0.1)] ({-1},{x});

        \node[anchor=east] at (axis cs:-0.7,0.5) {$-1$};
        \node[anchor=east] at (axis cs:1.4,-3.5) {$-3.5$};



       %% \addplot[color=penColor,fill=white,only marks,mark=*] coordinates{(2,-2)};  %% open hole      


\end{axis}
\end{tikzpicture}
\end{image}


From there we can identify the domain number, $-1$, and its range partner (function value), $-3.52$. We now know that $m(-1)=-3.5$.




\begin{question}

According to the graph for $y = m(x)$ above, $m(2) = \answer[tolerance=0.3]{1.8}$

\end{question}



\begin{question}

According to the graph for $y = m(x)$ above, $m(0)$ is \wordChoice{\choice{positive}\choice[correct]{negative}}.

\end{question}



\begin{question}

According to the graph for $y = m(x)$ above, $m(1.8)$  \wordChoice{\choice{$<$}\choice[correct]{$>$}}   $m(3.2)$.

\end{question}




\begin{question}

Below is the graph for $k = T(w)$. Use this graph to answer the following questions.

\begin{image}
\begin{tikzpicture}
\begin{axis}[
            domain=-10:10, ymax=10, xmax=10, ymin=-10, xmin=-10,
            axis lines =center, xlabel=$w$, ylabel=${k = T(w)}$,
            every axis y label/.style={at=(current axis.above origin),anchor=south},
            every axis x label/.style={at=(current axis.right of origin),anchor=west},
            axis on top,
          ]

        \addplot [very thick,penColor, smooth,domain=(-7:-1)] {-2*x-8};
        \addplot [color=penColor,only marks,mark=*] coordinates{(-7,6)};
        \addplot [color=penColor,only marks,mark=*] coordinates{(-1,-6)};

        \addplot [very thick,penColor, smooth,domain=(3:8)] {x-7};
        \addplot [color=penColor,fill=white,only marks,mark=*] coordinates{(3,-4)};
        \addplot [color=penColor,only marks,mark=*] coordinates{(8,1)};
          


\end{axis}
\end{tikzpicture}
\end{image}


\begin{itemize}
\item $T(-5) = \answer[tolerance=0.3]{10}$
\item $T(-1) = \answer[tolerance=0.3]{-6}$
\item $T(4) = \answer[tolerance=0.3]{-3}$
\item $T(8) = \answer[tolerance=0.3]{1}$
\end{itemize}


\end{question}








\begin{question}

Below is the graph for $k = T(w)$. Use this graph to answer the following questions.

\begin{image}
\begin{tikzpicture}
\begin{axis}[
            domain=-10:10, ymax=10, xmax=10, ymin=-10, xmin=-10,
            axis lines =center, xlabel=$w$, ylabel=${k = T(w)}$,
            every axis y label/.style={at=(current axis.above origin),anchor=south},
            every axis x label/.style={at=(current axis.right of origin),anchor=west},
            axis on top,
          ]

        \addplot [very thick,penColor, smooth,domain=(-7:-1)] {-2*x-8};
        \addplot [color=penColor,only marks,mark=*] coordinates{(-7,6)};
        \addplot [color=penColor,only marks,mark=*] coordinates{(-1,-6)};

        \addplot [very thick,penColor, smooth,domain=(3:8)] {x-7};
        \addplot [color=penColor,fill=white,only marks,mark=*] coordinates{(3,-4)};
        \addplot [color=penColor,only marks,mark=*] coordinates{(8,1)};
          


\end{axis}
\end{tikzpicture}
\end{image}


Which of the expressions below best describes the domain of $T$?
\begin{multipleChoice}
\choice  {$[-7,8]$}
\choice  {$[-6,6]$}
\choice  {$[-7,-1] \cup [3,8]$}
\choice [correct] {$[-7,-1] \cup (3,8]$}
\end{multipleChoice}




Which of the expressions below best describes the range of $T$?
\begin{multipleChoice}
\choice [correct] {$[-6,6]$}
\choice  {$(-4, 1]$}
\choice  {$[-6,6] \cup (-4,1]$}
\choice  {$[-6,6] \cup [-4,1]$}
\end{multipleChoice}


\end{question}











\begin{question}

Below is the graph for $z = x(y)$.

\begin{image}
\begin{tikzpicture}
\begin{axis}[
            domain=-10:10, ymax=10, xmax=10, ymin=-10, xmin=-10,
            axis lines =center, xlabel=$y$, ylabel=${z = x(y)}$,
            every axis y label/.style={at=(current axis.above origin),anchor=south},
            every axis x label/.style={at=(current axis.right of origin),anchor=west},
            axis on top,
          ]

        \addplot [very thick,penColor, smooth,domain=(-7:-1)] {2*x+8};
        \addplot [color=penColor,only marks,mark=*] coordinates{(-7,-6)};
        \addplot [color=penColor,only marks,mark=*] coordinates{(-1,6)};

        \addplot [very thick,penColor, smooth,domain=(3:8)] {x-3};
        \addplot [color=penColor,fill=white,only marks,mark=*] coordinates{(3,0)};
        \addplot [color=penColor,only marks,mark=*] coordinates{(8,5)};
          


\end{axis}
\end{tikzpicture}
\end{image}

\begin{itemize}
\item According to the graph, how many solutions are there to the equation $x(y) = 2$?
\begin{multipleChoice}
\choice  {$0$}
\choice  {$1$}
\choice [correct] {$2$}
\choice  {$4$}
\end{multipleChoice}


 
\item According to the graph, how many solutions are there to the equation $x(y) = 0$?
\begin{multipleChoice}
\choice  {$0$}
\choice [correct] {$1$}
\choice  {$2$}
\choice  {$4$}
\end{multipleChoice}


\item According to the graph, how many solutions are there to the equation $x(y) = -3$?
\begin{multipleChoice}
\choice  {$0$}
\choice [correct] {$1$}
\choice  {$2$}
\choice  {$4$}
\end{multipleChoice}
\end{itemize}
\end{question}












\end{document}
