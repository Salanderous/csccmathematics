\documentclass{ximera}


\graphicspath{
  {./}
  {ximeraTutorial/}
  {basicPhilosophy/}
}

\newcommand{\mooculus}{\textsf{\textbf{MOOC}\textnormal{\textsf{ULUS}}}}

\usepackage{tkz-euclide}\usepackage{tikz}
\usepackage{tikz-cd}
\usetikzlibrary{arrows}
\tikzset{>=stealth,commutative diagrams/.cd,
  arrow style=tikz,diagrams={>=stealth}} %% cool arrow head
\tikzset{shorten <>/.style={ shorten >=#1, shorten <=#1 } } %% allows shorter vectors

\usetikzlibrary{backgrounds} %% for boxes around graphs
\usetikzlibrary{shapes,positioning}  %% Clouds and stars
\usetikzlibrary{matrix} %% for matrix
\usepgfplotslibrary{polar} %% for polar plots
\usepgfplotslibrary{fillbetween} %% to shade area between curves in TikZ
\usetkzobj{all}
\usepackage[makeroom]{cancel} %% for strike outs
%\usepackage{mathtools} %% for pretty underbrace % Breaks Ximera
%\usepackage{multicol}
\usepackage{pgffor} %% required for integral for loops



%% http://tex.stackexchange.com/questions/66490/drawing-a-tikz-arc-specifying-the-center
%% Draws beach ball
\tikzset{pics/carc/.style args={#1:#2:#3}{code={\draw[pic actions] (#1:#3) arc(#1:#2:#3);}}}



\usepackage{array}
\setlength{\extrarowheight}{+.1cm}
\newdimen\digitwidth
\settowidth\digitwidth{9}
\def\divrule#1#2{
\noalign{\moveright#1\digitwidth
\vbox{\hrule width#2\digitwidth}}}






\DeclareMathOperator{\arccot}{arccot}
\DeclareMathOperator{\arcsec}{arcsec}
\DeclareMathOperator{\arccsc}{arccsc}

















%%This is to help with formatting on future title pages.
\newenvironment{sectionOutcomes}{}{}


\title{Function Features}


\begin{document}

\begin{abstract}
characteristics
\end{abstract}
\maketitle





Functions are packages containing three sets: codomain, codomain, and pairs.  The pairs connect the range vales to domain values and its is this relationship we want to investigate.   We would like to know how the range values are affected by the domain values. 

We are interested in the function values - the range values.  But they are controlled by the domain values.

We ask questions about the function values, but the answers are in the domain.







$\blacktriangleright$ What is the maximum value of the function? 

Translation: Which domain value results in the maximum value of the function? \\


$\blacktriangleright$ What is the minimum value of the function? 

Translation: Which domain value results in the minimum value of the function? \\


$\blacktriangleright$ where is the function increasing? 

Translation: On which set of domain values do the function increase? \\


$\blacktriangleright$ where is the function decreasing? 

Translation: On which set of domain values do the function decrease? \\







\section{Domain vs. Graph}



A graph provides a global view of all of the pairs, which reveals many of the patterns, features, and characteristics we seek.   

However, we must separate the graph from the function.  The graph helps us answer the questions, but the graph doesn't hold the answers.  The answers are in the domain. 












\begin{example}

Below is the graph of $y=f(x)$.  Use it to answer the following questions.

\begin{image}
\begin{tikzpicture} 
  \begin{axis}[
            domain=-10:10, ymax=10, xmax=10, ymin=-10, xmin=-10,
            axis lines =center, xlabel=$x$, ylabel={$y=f(x)$},
            ticklabel style={font=\scriptsize},
            every axis y label/.style={at=(current axis.above origin),anchor=south},
            every axis x label/.style={at=(current axis.right of origin),anchor=west},
            axis on top
          ]
          
          \addplot [line width=2, penColor, smooth,samples=100,domain=(-5:0)] ({x},{(x+5)*(1-x)});
          \addplot [line width=2, penColor, smooth,samples=100,domain=(1:5)] ({x},{-x-1});

          \addplot [color=penColor,fill=white,only marks,mark=*] coordinates{(0,5) (1,-2)};
          \addplot [color=penColor,only marks,mark=*] coordinates{(-5,0) (5,-6)};


           

  \end{axis}
\end{tikzpicture}
\end{image}


\end{example}






















\section{Visually Encoding}

Our plan is to visually encode the pair $(d,f(d))$ as a dot in the Cartesian plane with coordinates $(d,f(d))$.

\begin{center}
\large{That's it!!!!}  
\end{center}








\end{document}
