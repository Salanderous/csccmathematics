\documentclass{ximera}


\graphicspath{
  {./}
  {ximeraTutorial/}
  {basicPhilosophy/}
}

\newcommand{\mooculus}{\textsf{\textbf{MOOC}\textnormal{\textsf{ULUS}}}}

\usepackage{tkz-euclide}\usepackage{tikz}
\usepackage{tikz-cd}
\usetikzlibrary{arrows}
\tikzset{>=stealth,commutative diagrams/.cd,
  arrow style=tikz,diagrams={>=stealth}} %% cool arrow head
\tikzset{shorten <>/.style={ shorten >=#1, shorten <=#1 } } %% allows shorter vectors

\usetikzlibrary{backgrounds} %% for boxes around graphs
\usetikzlibrary{shapes,positioning}  %% Clouds and stars
\usetikzlibrary{matrix} %% for matrix
\usepgfplotslibrary{polar} %% for polar plots
\usepgfplotslibrary{fillbetween} %% to shade area between curves in TikZ
\usetkzobj{all}
\usepackage[makeroom]{cancel} %% for strike outs
%\usepackage{mathtools} %% for pretty underbrace % Breaks Ximera
%\usepackage{multicol}
\usepackage{pgffor} %% required for integral for loops



%% http://tex.stackexchange.com/questions/66490/drawing-a-tikz-arc-specifying-the-center
%% Draws beach ball
\tikzset{pics/carc/.style args={#1:#2:#3}{code={\draw[pic actions] (#1:#3) arc(#1:#2:#3);}}}



\usepackage{array}
\setlength{\extrarowheight}{+.1cm}
\newdimen\digitwidth
\settowidth\digitwidth{9}
\def\divrule#1#2{
\noalign{\moveright#1\digitwidth
\vbox{\hrule width#2\digitwidth}}}






\DeclareMathOperator{\arccot}{arccot}
\DeclareMathOperator{\arcsec}{arcsec}
\DeclareMathOperator{\arccsc}{arccsc}

















%%This is to help with formatting on future title pages.
\newenvironment{sectionOutcomes}{}{}


\title{Encoding Visually}


\begin{document}

\begin{abstract}
pairs to dots
\end{abstract}
\maketitle


Function notation allows us to talk about individual pairs inside a function.


\[
(d, F(d))
\]

$d$ is sitting in the left position of our ordered pair, therefore it represents a domain number. $F(d)$ represents the value of the function at $d$ and is written on the right in the ordered pair.


\begin{example} Ordered Pairs

Suppose $4$ is a member of the domain of the function $H$. \\
Then $H(4)$ represents the value of $H$ at $4$. \\
$H(4)$ is a member of the range of $H$. \\
$(4, H(4))$ is a pair in the function $H$.

If we happen to know that $(4, 9)$ is a pair in the function $H$, then we know $H(4) = 9$.

\end{example}
Two expressions for the same value. \\


Aside from talking about individual pairs, we might also like to talk about the whole collection at once.  Our first attempt at this is via pictures. We need a way to visually represent a single pair and then convert all pairs to a picture.  With this picture, we can analyze the function as a whole, identify important places in the domain, detect trands in the data, and quickly estimate information about the whole function.














\section{Visually Encoding}

Our plan is to visually encode the pair $(d,f(d))$ as a dot in the Cartesian plane with coordinates $(d,f(d))$.

\begin{center}
\large{That's it!!!!}  
\end{center}








\begin{image}
\begin{tikzpicture}
        \begin{axis}[
            domain=0:4, ymax=5, xmax=5, ymin=-5, xmin=-5,
            axis lines =center, xlabel=$\tiny{\text{domain}}$, ylabel=$\tiny{\text{codomain}}$,
            ticklabel style={font=\scriptsize},
            every axis y label/.style={at=(current axis.above origin),anchor=south},
            every axis x label/.style={at=(current axis.right of origin),anchor=west},
            axis on top
          ]
          
        \addplot [color=penColor2,only marks,mark=*] coordinates{(3,2)};


        \draw[decoration={brace,raise=.2cm,mirror},decorate,thin] (axis cs:3.05,0)--(axis cs:3.05,2);
        \draw[decoration={brace,raise=.2cm},decorate,thin] (axis cs:0,2.05)--(axis cs:3,2.05);
        \node[anchor=east] at (axis cs:1.85,3) {$d$};
        \node[anchor=east] at (axis cs:4.7,1) {$f(d)$};
                     
        \node at (axis cs:4,2.7) [penColor] {$(d,f(d))$};

        \end{axis}
\end{tikzpicture}
\end{image}


From the origin, we measure horizontally a distance of $d$.  Then we measure a vertical distance of $f(d)$. Plot a dot. The horizontal and vertical measurements for a point are called its \textit{coordinates}.



\begin{example}

Suppose we have a function $f$ and we know that $f(3)=2$. This information can be encoded visually with a dot plotted at $(3, 2)$.

\begin{image}
\begin{tikzpicture}
\begin{axis}[
            domain=0:4, ymax=5, xmax=5, ymin=-5, xmin=-5,
            axis lines =center, xlabel=$\scriptsize{range}$, ylabel=$\scriptsize{codomain}$,
            every axis y label/.style={at=(current axis.above origin),anchor=south},
            every axis x label/.style={at=(current axis.right of origin),anchor=west},
            axis on top
          ]
        
          \addplot [color=penColor2,only marks,mark=*] coordinates{(3,2)};


          \draw[decoration={brace,raise=.2cm,mirror},decorate,thin] (axis cs:3.05,0)--(axis cs:3.05,2);
          \draw[decoration={brace,raise=.2cm},decorate,thin] (axis cs:0,2.05)--(axis cs:3,2.05);
          \node[anchor=east] at (axis cs:1.85,3) {$3$};
          \node[anchor=east] at (axis cs:4,1) {$2$};
                    
          
          \node at (axis cs:3.7,2.7) [penColor] {$(3,2)$};

        \end{axis}
\end{tikzpicture}
\end{image}

\end{example}



\begin{question}
Here is a graph of $y=H(t)$. Which point represents $H(-3) = -2$?
\begin{image}
\begin{tikzpicture}
\begin{axis}[
            domain=0:4, ymax=5, xmax=5, ymin=-5, xmin=-5,
            axis lines =center, xlabel=$t$, ylabel=$y$,
            every axis y label/.style={at=(current axis.above origin),anchor=south},
            every axis x label/.style={at=(current axis.right of origin),anchor=west},
            axis on top
          ]

          
        \addplot [color=penColor2,only marks,mark=*] coordinates{(-3,2)};
        \addplot [color=penColor2,only marks,mark=*] coordinates{(3,-2)};
        \addplot [color=penColor2,only marks,mark=*] coordinates{(-3,-2)};
        \addplot [color=penColor2,only marks,mark=*] coordinates{(3,2)};

        \node at (axis cs:-3.4,2.4) [penColor] {$A$};
        \node at (axis cs:3.4,2.4) [penColor] {$B$};
        \node at (axis cs:-3.4,-2.4) [penColor] {$C$};
        \node at (axis cs:3.4,-2.4) [penColor] {$D$};

\end{axis}
\end{tikzpicture}
\end{image}


\begin{multipleChoice}
\choice {$A$}
\choice {$B$}
\choice[correct] {$C$}
\choice {$D$}
\end{multipleChoice}


\textbf{Note: } The horizontal axis is named $t$, because the function notation, $H(t)$, tells us that $t$ represents the domain numbers.  The vertical axis is named $y$, because the equation $y=H(t)$ tells us that $y$ is representing funciton values in the codomain.
\end{question}



\begin{question}
Suppose $k$ is a funnction with $k(-1)>0$. When we consider the graph of $y=k(t)$, there will be a point corresponding to $k(-1)$. Which quadrant will the corresponding dot be plotted?
\begin{image}
\begin{tikzpicture}
\begin{axis}[
            domain=0:4, ymax=5, xmax=5, ymin=-5, xmin=-5,
            axis lines =center, xlabel=$t$, ylabel=$y$,
            every axis y label/.style={at=(current axis.above origin),anchor=south},
            every axis x label/.style={at=(current axis.right of origin),anchor=west},
            axis on top
          ]

        
        \node at (axis cs:3.4,2.4) [penColor] {$I$};
        \node at (axis cs:-3.4,2.4) [penColor] {$II$};
        \node at (axis cs:-3.4,-2.4) [penColor] {$III$};
        \node at (axis cs:3.4,-2.4) [penColor] {$IV$};

\end{axis}
\end{tikzpicture}
\end{image}


\begin{multipleChoice}
\choice {$I$}
\choice[correct] {$II$}
\choice {$II$}
\choice {$IV$}
\end{multipleChoice}

\end{question}
FOur quadrants for hte four possible combinations of domain codomain signs. \\


Graphs of functions are just a collection of a bunch of dots.  Each dot represents a pair in the function.  Each dot has two coordinates.  The horizontal (first/left) coordinate is the domain number and the vertical (second/right) coordinate is the function value at that domain value.

The functions we are most interested in have millions of billions of trillions of dots.  



\begin{image}
\begin{tikzpicture} 
  \begin{axis}[
            domain=-4:4, ymax=5, xmax=4, ymin=-5, xmin=-4,
            axis lines =center, xlabel=$t$, ylabel={$y=g(t)$},
            every axis y label/.style={at=(current axis.above origin),anchor=south},
            every axis x label/.style={at=(current axis.right of origin),anchor=west},
            axis on top
          ]
          
    \addplot [color=penColor2,only marks,mark=*] coordinates{(-2,5) (-1,0) (0,-3) (1,-4) (2,-3) (3,0)};
           

  \end{axis}
\end{tikzpicture}
\end{image}


As we add more and more dots....


\begin{image}
\begin{tikzpicture} 
  \begin{axis}[
            domain=-4:4, ymax=5, xmax=4, ymin=-5, xmin=-4,
            axis lines =center, xlabel=$t$, ylabel={$y=g(t)$},
            every axis y label/.style={at=(current axis.above origin),anchor=south},
            every axis x label/.style={at=(current axis.right of origin),anchor=west},
            axis on top
          ]
          
    \addplot [color=penColor2,only marks,mark=*] coordinates{(-2.5,8.25) (-2,5) (-1.5,2.25) (-1,0) (-0.5,-1.75) (0,-3) (0.5,-3.75) (1,-4) (1.5,-3.75) (2,-3) (2.5,-1.75) (3,0) (3.5,2.25)};
           

  \end{axis}
\end{tikzpicture}
\end{image}


It soon becomes difficult to distinguish the individual dots.

\begin{image}
\begin{tikzpicture} 
  \begin{axis}[
            domain=-4:4, ymax=5, xmax=4, ymin=-5, xmin=-4,
            axis lines =center, xlabel=$t$, ylabel={$y=g(t)$},
            every axis y label/.style={at=(current axis.above origin),anchor=south},
            every axis x label/.style={at=(current axis.right of origin),anchor=west},
            axis on top
          ]
          
      \foreach \x in {-3,-2.75,...,3}
      \addplot [color=penColor2,only marks,mark=*] coordinates{(\x,\x*\x-2*\x-3)};
           

  \end{axis}
\end{tikzpicture}
\end{image}


They begin to overlap...

\begin{image}
\begin{tikzpicture} 
  \begin{axis}[
            domain=-4:4, ymax=5, xmax=4, ymin=-5, xmin=-4,
            axis lines =center, xlabel=$t$, ylabel={$y=g(t)$},
            every axis y label/.style={at=(current axis.above origin),anchor=south},
            every axis x label/.style={at=(current axis.right of origin),anchor=west},
            axis on top
          ]
          
      \foreach \x in {-3,-2.9,...,3}
      \addplot [color=penColor2,only marks,mark=*] coordinates{(\x,\x*\x-2*\x-3)};
           

  \end{axis}
\end{tikzpicture}
\end{image}




Eventually, your eyes play tricks on you.  You think you see a single object.  A thing - like a wire twisted on a piece of paper.



\begin{image}
\begin{tikzpicture} 
  \begin{axis}[
            domain=-4:4, ymax=5, xmax=4, ymin=-5, xmin=-4,
            axis lines =center, xlabel=$t$, ylabel={$y=g(t)$},
            every axis y label/.style={at=(current axis.above origin),anchor=south},
            every axis x label/.style={at=(current axis.right of origin),anchor=west},
            axis on top
          ]
          
    \addplot [draw=penColor2,very thick,smooth] {(x+1)*(x-3)};
           

  \end{axis}
\end{tikzpicture}
\end{image}



\begin{center}
\large{It isn't!!!!}  
\end{center}




It is a bunch of individual dots very close together. They are forming a pattern that your eye likes. Your eye glues them together.  





\begin{question}
The domain of the function, $p$, only contains positive numbers.  Which quadrants cannot contain points on the graph of $y=p(k)$?

\begin{selectAll}
\choice {$I$}
\choice[correct] {$II$}
\choice[correct] {$III$}
\choice {$IV$}
\end{selectAll}

\end{question}




















\section{Dots}



Of course, the dots on a graph of a function are wrong. They are too big. Dots, even the size of a pencil point, cover millions of billions of points.

Does the graph below tell us that $K(1) = -4$ or $K(1.03) = -4.1$ or $K(0.95) = -3.97$ or $K(1.01) = -3.95$?

\begin{image}
\begin{tikzpicture} 
  \begin{axis}[
            domain=-4:4, ymax=5, xmax=4, ymin=-5, xmin=-4,
            axis lines =center, xlabel=$v$, ylabel={$y=K(v)$},
            every axis y label/.style={at=(current axis.above origin),anchor=south},
            every axis x label/.style={at=(current axis.right of origin),anchor=west},
            axis on top
          ]
          
    \addplot [color=penColor2,only marks,mark=*] coordinates{(-2,5) (-1,0) (0,-3) (1,-4) (2,-3) (3,0)};
           

  \end{axis}
\end{tikzpicture}
\end{image}

The answer is yes.  \\


It says all of those, because the dot is too big.  Our graph should consist of points, but points are dimensionless.  We wouldn't be able to see them.  So, we draw dots that we can see and accept the inaccuracy of the graph.

Graphs are inherently inaccurate and there is nothing you can do about that.  They are communication tools. We have to keep this in mind when discussing mathematics with other people. Algebra is our tool for exactness.  Graphs give us the overall picture, but at the expense of accuracy.

We want to use both Algebra and graphs.  They each tell us what the other should be doing.  We just have to remember what each tool provides us.










\section{Domain and Range}


From our visual encoding plan, we see that the first coordinate of dots will be numbers from the domain of the function and the second coordinate of dots will be numbers from the range of the function.  More specifically, the second coordinate is the function value at the first coordinate.  The horizontal and vertical axis are seen as holding the domain and range.

\begin{itemize}
\item \textbf{DOMAIN:} We think of the domain as sitting on the horizontal axis.
\item \textbf{RANGE:} We think of the range as sitting on the vertical axis.
\end{itemize}


The two axes in our Cartesian plane are performing double duty.  We need double vision to see them correctly. If a dot on the graph happens to land on an axis, then we have function value information.

If the graph of $K(v)$ included three dots on the axes...

\begin{image}
\begin{tikzpicture} 
  \begin{axis}[
            domain=-4:4, ymax=5, xmax=4, ymin=-5, xmin=-4,
            axis lines =center, xlabel=$v$, ylabel={$y=K(v)$},
            every axis y label/.style={at=(current axis.above origin),anchor=south},
            every axis x label/.style={at=(current axis.right of origin),anchor=west},
            axis on top
          ]
          
    \addplot [color=penColor2,only marks,mark=*] coordinates{(-2,5) (-1,0) (0,-3) (1,-4) (2,-3) (3,0)};
           

  \end{axis}
\end{tikzpicture}
\end{image}


... on one hand...
\begin{itemize}
\item The dot at $(-1,0)$ tells us that $K(-1) = 0$.
\item The dot at $(0,-3)$ tells us that $K(0) = -3$.
\item The dot at $(3,0)$ tells us that $K(3) = 0$.
\end{itemize}


...on the other hand, from all of the dots, we can see that the domain of $K$ includes $\{ -2, -1, 0, 1, 2, 3 \}$. 


What if we want visualize the domain and the range of the function, by themselves?  When visualizing the domain and range, we would view the axes as individual number lines.  Rather than thinking of the axes as holding dots with two coordinates, we would think that the axes are number lines, holding numbers, and we would color in the intervals and numbers, like on a number line.

We need our brains to jump back and forth between these views.  Dots represent pairs in a function.  We decipher these paies of numbers by the location of the dot.  At the same time, locations on the axes also represent domain and range numbers - depending on the plotted dots.





\begin{example} 


Below is the graph of $y=h(f)$. We can identify its domain, by collecting all of the first coordinates of all of the dots.   We can identify its range, by collecting all of the second coordinates of all of the dots.  


\begin{image}
\begin{tikzpicture} 
  \begin{axis}[
            domain=-10:10, ymax=10, xmax=10, ymin=-10, xmin=-10,
            axis lines =center, xlabel=$\scriptsize{f}$, ylabel={$\scriptsize{y=h(f)}$},
            every axis y label/.style={at=(current axis.above origin),anchor=south},
            every axis x label/.style={at=(current axis.right of origin),anchor=west},
            axis on top
          ]
          
          \addplot [line width=2, penColor, smooth,samples=100,domain=(-4:0)] ({x},{-0.5*x+1});
          \addplot [line width=2, penColor, smooth,samples=100,domain=(1:7)] ({x},{0.5*x-7});
          \addplot [color=penColor,only marks,mark=*] coordinates{(-4,3) (0,1) (1,-6.5) (7,-3.5)};


  \end{axis}
\end{tikzpicture}
\end{image}



We could think of the horizontal axis as the domain number line and draw in the intervals of the domain.
\[
domain = [-4,0] \cup [1,7]
\]

\begin{image}
\begin{tikzpicture} 
  \begin{axis}[
            domain=-10:10, ymax=10, xmax=10, ymin=-10, xmin=-10,
            axis lines =center, xlabel=$f$, ylabel={$y=h(f)$},
            every axis y label/.style={at=(current axis.above origin),anchor=south},
            every axis x label/.style={at=(current axis.right of origin),anchor=west},
            axis on top
          ]
          
          \addplot [line width=2, penColor, smooth,samples=100,domain=(-4:0)] ({x},{-0.5*x+1});
          \addplot [line width=2, penColor, smooth,samples=100,domain=(1:7)] ({x},{0.5*x-7});
          \addplot [color=penColor,only marks,mark=*] coordinates{(-4,3) (0,1) (1,-6.5) (7,-3.5)};

          \addplot [line width=2, penColor2, smooth,samples=100,domain=(-4:0)] ({x},{0});
          \addplot [line width=2, penColor2, smooth,samples=100,domain=(1:7)] ({x},{0});
          \addplot [color=penColor2,only marks,mark=*] coordinates{(-4,0) (0,0) (1,0) (7,0)};
           

  \end{axis}
\end{tikzpicture}
\end{image}



We could think of the vertical axis as the range number line and draw in the intervals of the range.

\[
range = [-6.5, -3.5] \cup [1, 4]
\]

\begin{image}
\begin{tikzpicture} 
  \begin{axis}[
            domain=-10:10, ymax=10, xmax=10, ymin=-10, xmin=-10,
            axis lines =center, xlabel=$f$, ylabel={$y=h(f)$},
            every axis y label/.style={at=(current axis.above origin),anchor=south},
            every axis x label/.style={at=(current axis.right of origin),anchor=west},
            axis on top
          ]
          
          \addplot [line width=2, penColor, smooth,samples=100,domain=(-4:0)] ({x},{-0.5*x+1});
          \addplot [line width=2, penColor, smooth,samples=100,domain=(1:7)] ({x},{0.5*x-7});
          \addplot [color=penColor,only marks,mark=*] coordinates{(-4,3) (0,1) (1,-6.5) (7,-3.5)};

          \addplot [line width=2, penColor2, smooth,samples=100,domain=(-6.5:-3.5)] ({0},{x});
          \addplot [line width=2, penColor2, smooth,samples=100,domain=(1:4)] ({0},{x});
          \addplot [color=penColor2,only marks,mark=*] coordinates{(0, -6.5) (0,-3.5) (0,1) (0,4)};
           

  \end{axis}
\end{tikzpicture}
\end{image}


\end{example}






\end{document}
