\documentclass{ximera}


\graphicspath{
  {./}
  {ximeraTutorial/}
  {basicPhilosophy/}
}

\newcommand{\mooculus}{\textsf{\textbf{MOOC}\textnormal{\textsf{ULUS}}}}

\usepackage{tkz-euclide}\usepackage{tikz}
\usepackage{tikz-cd}
\usetikzlibrary{arrows}
\tikzset{>=stealth,commutative diagrams/.cd,
  arrow style=tikz,diagrams={>=stealth}} %% cool arrow head
\tikzset{shorten <>/.style={ shorten >=#1, shorten <=#1 } } %% allows shorter vectors

\usetikzlibrary{backgrounds} %% for boxes around graphs
\usetikzlibrary{shapes,positioning}  %% Clouds and stars
\usetikzlibrary{matrix} %% for matrix
\usepgfplotslibrary{polar} %% for polar plots
\usepgfplotslibrary{fillbetween} %% to shade area between curves in TikZ
\usetkzobj{all}
\usepackage[makeroom]{cancel} %% for strike outs
%\usepackage{mathtools} %% for pretty underbrace % Breaks Ximera
%\usepackage{multicol}
\usepackage{pgffor} %% required for integral for loops



%% http://tex.stackexchange.com/questions/66490/drawing-a-tikz-arc-specifying-the-center
%% Draws beach ball
\tikzset{pics/carc/.style args={#1:#2:#3}{code={\draw[pic actions] (#1:#3) arc(#1:#2:#3);}}}



\usepackage{array}
\setlength{\extrarowheight}{+.1cm}
\newdimen\digitwidth
\settowidth\digitwidth{9}
\def\divrule#1#2{
\noalign{\moveright#1\digitwidth
\vbox{\hrule width#2\digitwidth}}}






\DeclareMathOperator{\arccot}{arccot}
\DeclareMathOperator{\arcsec}{arcsec}
\DeclareMathOperator{\arccsc}{arccsc}

















%%This is to help with formatting on future title pages.
\newenvironment{sectionOutcomes}{}{}


\title{Function Features}


\begin{document}

\begin{abstract}
change
\end{abstract}
\maketitle





Functions are packages containing three sets: codomain, codomain, and pairs.  The pairs connect the range vales to domain values and its is this relationship we want to investigate.   We would like to know how the range values are affected by the domain values. 

We are interested in the function values - the range values.  But they are controlled by the domain values.

We ask questions about the function values, but the answers are in the domain.





\begin{idea}

$\blacktriangleright$ Where is the function increasing? 

Translation: On which set of domain values do the function increase? \\


$\blacktriangleright$ Where is the function decreasing? 

Translation: On which set of domain values do the function decrease? \\


\end{idea}








A graph provides a global view of all of the pairs, which reveals many of the patterns, features, and characteristics we seek.   

However, we must separate the graph from the function.  The graph helps us answer the questions, but the graph doesn't hold the answers.  The answers are in the domain and range. 





\section{Increasing}



\begin{definition} Maximum


Let $f$ be a function defined on the domain $D$.

Then $M$ is the maximum value of $f$ on $D$ if    


\begin{itemize}
\item There exists $d \in D$, such that $f(d) = M$.   i.e., $M$ has to be a function value \\

\item $f(d) \leq M$ for all $d \in D$. i.e., all function values are less than or equal to $M$. 

\end{itemize}


If there is no such $M$, then $F$ has no maximum value.

\end{definition}



\begin{observation}
The maximum value is visually encoded as the highest point on the graph of the function.  The second coordinate is the maximum value and the first coordinate is the domain number where the maximum occurs.
\end{observation}










\begin{example}

Below is the graph of $y=f(x)$.  

\begin{image}
\begin{tikzpicture} 
  \begin{axis}[
            domain=-10:10, ymax=10, xmax=10, ymin=-10, xmin=-10,
            axis lines =center, xlabel=$x$, ylabel={$y=f(x)$},
            ticklabel style={font=\scriptsize},
            every axis y label/.style={at=(current axis.above origin),anchor=south},
            every axis x label/.style={at=(current axis.right of origin),anchor=west},
            axis on top
          ]
          
          \addplot [line width=2, penColor, smooth,samples=100,domain=(-5:0)] ({x},{(x+5)*(1-x)});
          \addplot [line width=2, penColor, smooth,samples=100,domain=(1:5)] ({x},{-x-1});
          \addplot [line width=2, penColor, smooth,samples=100,domain=(5:8)] ({x},{2*x-16});

          \addplot [color=penColor,fill=white,only marks,mark=*] coordinates{(0,5) (1,-2) (8,0)};
          \addplot [color=penColor,only marks,mark=*] coordinates{(-5,0) (5,-6)};


           

  \end{axis}
\end{tikzpicture}
\end{image}



\begin{question}

The highest point on the graph of $f$ is


\begin{multipleChoice}
\choice {$(-5,0)$}
\choice [correct]{$(-2,9)$}
\choice {$(0,5)$}
\choice {$(-1,-2)$}
\choice {$(5,-6)$}
\choice {$(8,0)$}
\end{multipleChoice}
\end{question}


\begin{question}
The maximum value of $f$ is $\answer{9}$, which occurs at $\answer{-2}$.
\end{question}







\end{example}










\begin{example}

Below is the graph of $y=g(t)$.  

\begin{image}
\begin{tikzpicture} 
  \begin{axis}[
            domain=-10:10, ymax=10, xmax=10, ymin=-10, xmin=-10,
            axis lines =center, xlabel=$t$, ylabel={$y=g(t)$},
            ticklabel style={font=\scriptsize},
            every axis y label/.style={at=(current axis.above origin),anchor=south},
            every axis x label/.style={at=(current axis.right of origin),anchor=west},
            axis on top
          ]
          

          \addplot [line width=2, penColor, smooth,samples=100,domain=(-6:5)] ({x},{-x-1});
          \addplot [line width=2, penColor, smooth,samples=100,domain=(5:8)] ({x},{2*x-16});

          \addplot [color=penColor,fill=white,only marks,mark=*] coordinates{(-6,5)};
          \addplot [color=penColor,only marks,mark=*] coordinates{(5,-6) (8,0)};


           

  \end{axis}
\end{tikzpicture}
\end{image}



There is no highest point on the graph.  Therefore, $g$ has no maximum value.

\end{example}


































\section{Minimum}



\begin{definition} Minimum


Let $f$ be a function defined on the domain $D$.

Then $M$ is the mainimum value of $f$ on $D$ if    


\begin{itemize}
\item There exists $d \in D$, such that $f(d) = M$.   i.e., $M$ has to be a function value \\

\item $f(d) \geq M$ for all $d \in D$. i.e., all function values are greater than or equal to $M$. 

\end{itemize}


If there is no such $M$, then $F$ has no minimum value.

\end{definition}



\begin{observation}
The minimum value is visually encoded as the lowest point on the graph of the function.  The second coordinate is the minimum value and the first coordinate is the domain number where the minimum occurs.
\end{observation}










\begin{example}

Below is the graph of $y=f(x)$.  

\begin{image}
\begin{tikzpicture} 
  \begin{axis}[
            domain=-10:10, ymax=10, xmax=10, ymin=-10, xmin=-10,
            axis lines =center, xlabel=$x$, ylabel={$y=f(x)$},
            ticklabel style={font=\scriptsize},
            every axis y label/.style={at=(current axis.above origin),anchor=south},
            every axis x label/.style={at=(current axis.right of origin),anchor=west},
            axis on top
          ]
          
          \addplot [line width=2, penColor, smooth,samples=100,domain=(-5:0)] ({x},{(x+5)*(1-x)});
          \addplot [line width=2, penColor, smooth,samples=100,domain=(1:5)] ({x},{-x-1});
          \addplot [line width=2, penColor, smooth,samples=100,domain=(5:8)] ({x},{2*x-16});

          \addplot [color=penColor,fill=white,only marks,mark=*] coordinates{(0,5) (1,-2) (8,0)};
          \addplot [color=penColor,only marks,mark=*] coordinates{(-5,0) (5,-6)};


           

  \end{axis}
\end{tikzpicture}
\end{image}




\begin{question}

The lowest point on the graph of $f$ is


\begin{multipleChoice}
\choice {$(-5,0)$}
\choice [correct]{$(-2,9)$}
\choice {$(0,5)$}
\choice {$(-1,-2)$}
\choice {$(5,-6)$}
\choice {$(8,0)$}
\end{multipleChoice}
\end{question}


\begin{question}
The minimum value of $f$ is $\answer{-6}$, which occurs at $\answer{5}$.
\end{question}







\end{example}










\begin{example}

Below is the graph of $y=H(k)$.  

\begin{image}
\begin{tikzpicture} 
  \begin{axis}[
            domain=-10:10, ymax=10, xmax=10, ymin=-10, xmin=-10,
            axis lines =center, xlabel=$k$, ylabel={$y=H(k)$},
            ticklabel style={font=\scriptsize},
            every axis y label/.style={at=(current axis.above origin),anchor=south},
            every axis x label/.style={at=(current axis.right of origin),anchor=west},
            axis on top
          ]
          

          \addplot [line width=2, penColor, smooth,samples=100,domain=(-6:5)] ({x},{-x-1});
          \addplot [line width=2, penColor, smooth,samples=100,domain=(5:8)] ({x},{2*x-16});

          \addplot [color=penColor,fill=white,only marks,mark=*] coordinates{(-6,5) (5,-6)};
          \addplot [color=penColor,only marks,mark=*] coordinates{(8,0)};


           

  \end{axis}
\end{tikzpicture}
\end{image}



There is no lowest point on the graph.  Therefore, $H$ has no minimum value.

\end{example}





















\end{document}
