\documentclass{ximera}


\graphicspath{
  {./}
  {ximeraTutorial/}
  {basicPhilosophy/}
}

\newcommand{\mooculus}{\textsf{\textbf{MOOC}\textnormal{\textsf{ULUS}}}}

\usepackage{tkz-euclide}\usepackage{tikz}
\usepackage{tikz-cd}
\usetikzlibrary{arrows}
\tikzset{>=stealth,commutative diagrams/.cd,
  arrow style=tikz,diagrams={>=stealth}} %% cool arrow head
\tikzset{shorten <>/.style={ shorten >=#1, shorten <=#1 } } %% allows shorter vectors

\usetikzlibrary{backgrounds} %% for boxes around graphs
\usetikzlibrary{shapes,positioning}  %% Clouds and stars
\usetikzlibrary{matrix} %% for matrix
\usepgfplotslibrary{polar} %% for polar plots
\usepgfplotslibrary{fillbetween} %% to shade area between curves in TikZ
\usetkzobj{all}
\usepackage[makeroom]{cancel} %% for strike outs
%\usepackage{mathtools} %% for pretty underbrace % Breaks Ximera
%\usepackage{multicol}
\usepackage{pgffor} %% required for integral for loops



%% http://tex.stackexchange.com/questions/66490/drawing-a-tikz-arc-specifying-the-center
%% Draws beach ball
\tikzset{pics/carc/.style args={#1:#2:#3}{code={\draw[pic actions] (#1:#3) arc(#1:#2:#3);}}}



\usepackage{array}
\setlength{\extrarowheight}{+.1cm}
\newdimen\digitwidth
\settowidth\digitwidth{9}
\def\divrule#1#2{
\noalign{\moveright#1\digitwidth
\vbox{\hrule width#2\digitwidth}}}






\DeclareMathOperator{\arccot}{arccot}
\DeclareMathOperator{\arcsec}{arcsec}
\DeclareMathOperator{\arccsc}{arccsc}

















%%This is to help with formatting on future title pages.
\newenvironment{sectionOutcomes}{}{}


\title{Zero Property}

\begin{document}

\begin{abstract}
factoring
\end{abstract}
\maketitle




Locating the zeros of a function is an important skill.  Functions switch signs at zeros which can reflect a lot of important information  - especially where functions switch between increasing and decreasing.

We don't have many ways of locating zeros for complicated functions.


\textbf{\textcolor{blue!55!black}{$\blacktriangleright$}} Zeros of linear functions are easy to locate - just solve the equation for the variable. \\


\textbf{\textcolor{blue!55!black}{$\blacktriangleright$}} Basic exponential functions don't have zeros.  Basic Logarthmic functions have a zero when the inside of the formula equals $1$. For more complicated exponential and logarithmic functions, we can use transformations and the algebra rules. \\

\textbf{\textcolor{blue!55!black}{$\blacktriangleright$}} Sine and Cosine have repeating zeros. So, we can identify them for one period and then use transformations to locate the others. \\

\textbf{\textcolor{blue!55!black}{$\blacktriangleright$}} We have several methods for Quadratic functions.  We can complete the square and use the quadratic formula. \\



We have only one more method.  It is specifc and yet mysterious at the same time.



\begin{quote}
\textbf{Factoring: } Write the expression as a product and use the Zero Product Property of real numbers.
\end{quote}



The Zero Product Property is one of the few simple facts we know about the arithmetic of the real numbers.


\begin{definition} \textbf{\textcolor{green!50!black}{Zero Product Property}}  

Let $a$ and $b$ be real numbers. \\

If $a \cdot b = 0$, then either $a=0$ or $b=0$.


\end{definition}



That is one of our best tool for locating zeros of functions. \\



\begin{example} Factoring

Solve $k^2 - k - 20 = 0$ \\

\begin{explanation}



Rewrite $k^2 - k - 20$ as a product.

If $k^2 - k - 20 = (factor) \cdot (factor)$, then the factors will have to be linear or constants.  If one of the factors was a constant, then we would have recognized a constant factor in all three coefficients of the terms of the expression.  We can look beyond constant factors for this quadratic. \\

Therefore, we can begin with a template involving two linear factors, like $k^2 - k - 20 = (k + ?) \cdot (k + ?)$

When we multiply the factors out, the product of the two constant terms will have to be $\answer{-20}$.

Possibilities:
\begin{itemize}
\item $-20 \cdot 1$
\item $20 \cdot \answer{-1}$
\item $-10 \cdot 2$
\item $10 \cdot \answer{-2}$
\item $-5 \cdot 4$
\item $5 \cdot \answer{-4}$
\end{itemize}



Only one of these options gives a linear term of $-k$.


$k^2 - k - 20 = (k + 4) \cdot (k + (-5))$

$k^2 - k - 20 = (k + 4) \cdot (k - 5)$



Now we can replace the original equation with an equivalent one:


$(k + 4) \left( \answer{k - 5} \right) = 0$



Applying the zero product property tells us that either $k + 4 = 0$ or $k - 5 = 0$.  \\

The solution set is $\{ -4, 5 \}$



\end{explanation}

\end{example}



















\begin{example} Common Factor


Solve $3t e^t - (t-1) e^t = 0$

\begin{explanation}



This expression is a difference of two terms and each term has $e^t$ as a factor.  Therefore, we will apply the Distributive Property.



$e^t (3t - (t-1))= 0$


We can simplify the parentheses.



$e^t \left( \answer{2t + 1} \right)= 0$



The zero product property tells us that either $\answer{e^t} = 0$ or $2t + 1 = 0$.  \\

The function $f(t) = e^t$ has no zeros. 


$2t + 1 = 0$, when $t = -\frac{1}{2}$

We have one solution: $\left\{ -\frac{1}{2} \right\}$

\end{explanation}
\end{example}















\begin{center}
\textbf{\textcolor{green!50!black}{ooooo=-=-=-=-=-=-=-=-=-=-=-=-=ooOoo=-=-=-=-=-=-=-=-=-=-=-=-=ooooo}} \\

more examples can be found by following this link\\ \link[More Examples of Function Zeros]{https://ximera.osu.edu/csccmathematics/precalculus1/precalculus1/zeros/examples/exampleList}

\end{center}








\end{document}
