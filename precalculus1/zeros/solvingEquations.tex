\documentclass{ximera}


\graphicspath{
  {./}
  {ximeraTutorial/}
  {basicPhilosophy/}
}

\newcommand{\mooculus}{\textsf{\textbf{MOOC}\textnormal{\textsf{ULUS}}}}

\usepackage{tkz-euclide}\usepackage{tikz}
\usepackage{tikz-cd}
\usetikzlibrary{arrows}
\tikzset{>=stealth,commutative diagrams/.cd,
  arrow style=tikz,diagrams={>=stealth}} %% cool arrow head
\tikzset{shorten <>/.style={ shorten >=#1, shorten <=#1 } } %% allows shorter vectors

\usetikzlibrary{backgrounds} %% for boxes around graphs
\usetikzlibrary{shapes,positioning}  %% Clouds and stars
\usetikzlibrary{matrix} %% for matrix
\usepgfplotslibrary{polar} %% for polar plots
\usepgfplotslibrary{fillbetween} %% to shade area between curves in TikZ
\usetkzobj{all}
\usepackage[makeroom]{cancel} %% for strike outs
%\usepackage{mathtools} %% for pretty underbrace % Breaks Ximera
%\usepackage{multicol}
\usepackage{pgffor} %% required for integral for loops



%% http://tex.stackexchange.com/questions/66490/drawing-a-tikz-arc-specifying-the-center
%% Draws beach ball
\tikzset{pics/carc/.style args={#1:#2:#3}{code={\draw[pic actions] (#1:#3) arc(#1:#2:#3);}}}



\usepackage{array}
\setlength{\extrarowheight}{+.1cm}
\newdimen\digitwidth
\settowidth\digitwidth{9}
\def\divrule#1#2{
\noalign{\moveright#1\digitwidth
\vbox{\hrule width#2\digitwidth}}}






\DeclareMathOperator{\arccot}{arccot}
\DeclareMathOperator{\arcsec}{arcsec}
\DeclareMathOperator{\arccsc}{arccsc}

















%%This is to help with formatting on future title pages.
\newenvironment{sectionOutcomes}{}{}


\title{Solving Equations}

\begin{document}

\begin{abstract}
break up
\end{abstract}
\maketitle





\begin{example}  Common Factor


Solve $\sqrt{x+9} + x \cdot \frac{1}{2} (x+9)^{-\tfrac{1}{2}} = 0$



\begin{explanation}



A common factor is a factor of the least degree in all of the terms.  We have one here. $(x+9)$ is a factor in both terms.  The powers are $\frac{1}{2}$ and $-\frac{1}{2}$. The least of which is $-\frac{1}{2}$.

We will use the distributive property to factor out $(x+9)^{\answer{-\frac{1}{2}}}$.


$(x+9)^{-\tfrac{1}{2}} \left((x+9) + x \cdot \frac{1}{2}\right)  = 0$


We could, in turn, write this as a fraction


\[  \frac{\left((x+9) + x \cdot \frac{1}{2}\right)}{\sqrt{x+9}} = 0 \]


\[  \frac{ \frac{3}{2} x + 9}{\sqrt{x+9}} = 0 \]



This is a fraction, so it equals $0$ when the numerator equals $0$ and the denominator does not equal $0$.


$\answer{\frac{3}{2} x + 9} = 0$



$x = 6$, which is in the domain of $\sqrt{x+9} + x \cdot \frac{1}{2} (x+9)^{-\tfrac{1}{2}}$.


\end{explanation}
\end{example}
















\begin{example}  Common Factor


Solve 

\[  \frac{x^3 (18x) - 9(x^2-3)(3x^2)}{(x^3)^2} = 0 \]



\begin{explanation}


The numerator is a difference of two terms and $9 x^2$ is a common factor.  We can use the Distributive Property to factor it out.



\[  \frac{9x^2 (2x^2 - 3(x^2-3))}{x^6} = 0  \]


\[  \frac{9 \left( \answer{9 - x^2} \right)}{x^4} = 0 \]



This is a fraction, so it equals $0$ when the numerator equals $0$ and the denominator does not equal $0$.


$9 - x^2 = 0$

This has two solutions: $\{ -3, 3  \}$, both of which are in the domain.






\end{explanation}

\end{example}

















\begin{example}  Common Factor


Solve $12 x^3 - 12 x^2 = 0 $



\begin{explanation}


The Distributive Property (factoring) gives us the product $12 x^2 (x-1)$.




We have two solutions: $\{ 0, 1  \}$, both of which are in the domain.


\end{explanation}
\end{example}





















\begin{example}  Fractions


Solve $2 - \frac{2}{\sqrt[3]{x}} = 0$



\begin{explanation}


Converting to exponents gives us $2 - \frac{2}{x^{\tfrac{1}{3}}} = 0$


$2 = \frac{2}{x^{\tfrac{1}{3}}}$

The only way this can happen is if $x^{\tfrac{1}{3}} = \answer{1}$.

This has one solution: $\{  1  \}$





\end{explanation}


\end{example}











\end{document}
