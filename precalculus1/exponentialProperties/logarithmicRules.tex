\documentclass{ximera}


\graphicspath{
  {./}
  {ximeraTutorial/}
  {basicPhilosophy/}
}

\newcommand{\mooculus}{\textsf{\textbf{MOOC}\textnormal{\textsf{ULUS}}}}

\usepackage{tkz-euclide}\usepackage{tikz}
\usepackage{tikz-cd}
\usetikzlibrary{arrows}
\tikzset{>=stealth,commutative diagrams/.cd,
  arrow style=tikz,diagrams={>=stealth}} %% cool arrow head
\tikzset{shorten <>/.style={ shorten >=#1, shorten <=#1 } } %% allows shorter vectors

\usetikzlibrary{backgrounds} %% for boxes around graphs
\usetikzlibrary{shapes,positioning}  %% Clouds and stars
\usetikzlibrary{matrix} %% for matrix
\usepgfplotslibrary{polar} %% for polar plots
\usepgfplotslibrary{fillbetween} %% to shade area between curves in TikZ
\usetkzobj{all}
\usepackage[makeroom]{cancel} %% for strike outs
%\usepackage{mathtools} %% for pretty underbrace % Breaks Ximera
%\usepackage{multicol}
\usepackage{pgffor} %% required for integral for loops



%% http://tex.stackexchange.com/questions/66490/drawing-a-tikz-arc-specifying-the-center
%% Draws beach ball
\tikzset{pics/carc/.style args={#1:#2:#3}{code={\draw[pic actions] (#1:#3) arc(#1:#2:#3);}}}



\usepackage{array}
\setlength{\extrarowheight}{+.1cm}
\newdimen\digitwidth
\settowidth\digitwidth{9}
\def\divrule#1#2{
\noalign{\moveright#1\digitwidth
\vbox{\hrule width#2\digitwidth}}}






\DeclareMathOperator{\arccot}{arccot}
\DeclareMathOperator{\arcsec}{arcsec}
\DeclareMathOperator{\arccsc}{arccsc}

















%%This is to help with formatting on future title pages.
\newenvironment{sectionOutcomes}{}{}


\title{Logarithmic Rules}

\begin{document}

\begin{abstract}
rules
\end{abstract}
\maketitle
















\section{Exponents}

Logarithms are exponents. \\


$log_a(b)$ is what you \textbf{\textcolor{red!90!darkgray}{raise}} $a$ to, to get $b$.

\[    a^{log_a(b)}  = b    \, \text{ for } \,  a,b > 0     \]


We can read it right to left as well.  Any positive number, $b$, can be written with any base, $a$, like $b = a^{log_a(b)}$.






Logarithms are exponents. Therefore, they should follow all of the exponent rules. \\



$\blacktriangleright$  Let $M$ and $N$ be two positive real numbers.

We can write them as $M = a^{log_a(M)}$ and $N = a^{log_a(N)}$


This allows us to write the product $M \cdot N$ in two different ways.



\[   M \cdot N = a^{log_a(M)} \cdot a^{log_a(N)}                    \]

\[   M \cdot N = a^{log_a(M \cdot N)}                  \]


Therefore, these must be equal.


\[    a^{log_a(M)} \cdot a^{log_a(N)}     =   a^{log_a(M \cdot N)}                \]


Apply an exponent rule:


\[    a^{log_a(M)+log_a(N)}    =   a^{log_a(M \cdot N)}                \]



Since exponential functions are one-to-one, we have 


\[    log_a(M)+log_a(N)    =   log_a(M \cdot N)               \]




\begin{template} 

\[    log_a(M)+log_a(N)    =   log_a(M \cdot N)      \, \text{ for } \, a, M, N > 0        \]


\end{template}




$\blacktriangleright$  Let's write the quotient $\frac{M}{N}$ in two different ways.



\[   \frac{M}{N} = \frac{a^{log_a(M)}}{a^{log_a(N)}}                    \]

\[   \frac{M}{N} = a^{log_a\left(\frac{M}{N}\right)}                  \]


Therefore, these must be equal.


\[    \frac{a^{log_a(M)}}{a^{log_a(N)}}    =   a^{log_a\left(\frac{M}{N}\right)}                \]


Apply an exponent rule:


\[    a^{log_a(M) - log_a(N)}    =   a^{log_a\left(\frac{M}{N}\right)}                \]



Since exponential functions are one-to-one, we have 


\[    log_a(M)-log_a(N)    =   log_a\left(\frac{M}{N}\right)             \]








\begin{template} 

\[    log_a(M)-log_a(N)    =   log_a\left(\frac{M}{N}\right)        \, \text{ for } \, a, M, N > 0        \]


\end{template}




















$\blacktriangleright$  Let's write $M^N$ in two different ways.



\[   M^N = a^{log_a(M^N)}                  \]

\[   M^N = (a^{log_a(M)})^N =     (a^{N \cdot log_a(M)})             \]


Therefore, these must be equal.


\[    a^{log_a(M^N)}      =    (a^{N \cdot log_a(M)})                \]





Since exponential functions are one-to-one, we have 


\[    log_a(M^N)    =   N \cdot log_a(M)            \]








\begin{template} 

\[    log_a(M^N)    =   N \cdot log_a(M)       \, \text{ for } \, a M, N  > 0        \]


\end{template}


















\section{One-to-One}







Since exponential functions are one-to-one, and logarithmic functions are just the reverse, logarithmic functions must be \textbf{one-to-one} as well. One-to-one means that each range number is paired with a unique domain number.





\begin{image}
\begin{tikzpicture}
  \begin{axis}[
            domain=-10:10, ymax=10, xmax=10, ymin=-10, xmin=-10,
            axis lines =center, xlabel=$t$, ylabel=$y$, grid = major,
            ytick={-10,-8,-6,-4,-2,2,4,6,8,10},
            xtick={-10,-8,-6,-4,-2,2,4,6,8,10},
            ticklabel style={font=\scriptsize},
            every axis y label/.style={at=(current axis.above origin),anchor=south},
            every axis x label/.style={at=(current axis.right of origin),anchor=west},
            axis on top
          ]
          

			\addplot [line width=2, penColor, smooth,samples=200,domain=(-2.95:8),<->] {ln(x+3)/ln(2)-4};
			%\addplot [line width=2, penColor2, smooth,samples=200,domain=(-8:-0.25),<->] {2^(x+4)-3};


            %\addplot [line width=1, gray, dashed,samples=200,domain=(-10:10),<->] ({x},{-3});
            %\addplot [line width=1, gray, dashed,samples=200,domain=(-10:10),<->] ({x},{x});


			\addplot[color=penColor,fill=penColor,only marks,mark=*] coordinates{(-2,-4)};
			%\addplot[color=penColor2,fill=penColor,only marks,mark=*] coordinates{(-4,-2)};

      \addplot [line width=1, gray, dashed,samples=200,domain=(-10:10),<->] ({-3},{x});






           

  \end{axis}
\end{tikzpicture}
\end{image}




In other words, each function value in a basic logarithmic function occurs exactly once.


If you know that $log_a(r) = log_a(t)$, then $r=t$ follows. \\


This is an important rule.


\begin{template} Uniqueness


\[     \text{If } \,  log_a(r) = log_a(t), \,  \text{ then }  \, r=t    \]


\end{template}





















\begin{example} Solving Equations


Solve $log_5(21 + x) = 2$


\begin{explanation}


$log_5(21 + x) = 2$

$5^{log_5(21 + x)} = 5^2$

$21 + x = 5^2 = 25$, using the definition of logarithm.

$x = 4$
\end{explanation}
\end{example}










\begin{example} Solving Equations


Solve $log_2(y) + log_2(y-4) = log_2(y+6)$

\begin{explanation}


$log_2(y) + log_2(y-4) = log_2(y+6)$

$log_2(y(y-4)) = log_2(y+6)$

$y(y-4) = y+6$    

$y^2 - 4y - y - 6 = 0$

$y^2 - 5y - 6 = 0$

$(y-6) \left( \answer{y+1} \right) = 0$


Either $y-6 = 0$ or $y+1 = 0$

Either $y = 6$ or $y = -1$

However, $y \ne -1$, since $\answer{-1}$ is not in the domain of $log_2(y)$.

Therefore, just one solution: $6$.

The solution set is $\{ 6 \}$
\end{explanation}
\end{example}






\begin{example} Solving Equations


Solve $7^k = 5$


\begin{explanation}


We are looking for the number that you raise $7$ to, to get $5$.  That number  is $log_7(5)$.
\end{explanation}
\end{example}















\section{Change of Base}


Logarithms are exponenets.  We use them to write expressions in exponential form.



\begin{quote}

Any positive number, $b$, can be written as a power of $a$, any other positive number.

\[    b = a^{log_a(b)}  \]

\end{quote}

For this to be useful, we will need to write multiple expressions with the same base. Therefore, changing the base becomes important.

How do we write $log_a(b)$ in terms of $log_c$?




First, $a$ and $b$ can be written in terms of some third base, $c$, using logarithms.


\[    b = c^{log_c(b)} \,   \text{ and } \,      a = c^{log_c(a)}      \]



Second, substitute these in for the $a$ and $b$ bases in $b = a^{log_a(b)}$.



\[   c^{log_c(b)} = \left(c^{log_c(a)}\right)^{log_a(b)}  \]



\[   c^{log_c(b)} = c^{log_c(a) \cdot log_a(b)}  \]



\[   log_c(b) = log_c(a) \cdot log_a(b)  \]


\[   \frac{log_c(b)}{log_c(a)} =  log_a(b)  \]


This is known as the \textbf{Change of Base Formula}.








\begin{template}  Change of Base Formula

\[   log_a(b)  =  \frac{log_c(b)}{log_c(a)}        \, \text{ for } \, a, b, c  > 0        \]


\end{template}











\begin{example}


Write $log_3(5)$ in terms of logarithms base $7$.


\begin{explanation}


Following the change of base template: $log_a(b)  =  \frac{log_c(b)}{log_c(a)} $ gives



\[   log_3(5)  =  \frac{log_7(5)}{log_7(3)}         \]



\end{explanation}
\end{example}


















$\blacktriangleright$ It occured to people: If we can write any power in terms of any base, and any logarithm in terms of any base, then let's just pick one base to write everything in.


We have such a base: $e$


\section{e}



$e$ was defined in a very weird way.

\[   e = \lim_{x \to \infty}  \left(\answer{1 + \frac{1}{x}}\right)^x      \]


It is weird now.  But as you continue through Calculus, you will see $e$ pop up all over the place.  It seems to have a connection to everything.  So much so that scientists, engineers, and mathematicians have  adopted $e$ as the prefered base for everything.


This means that we encounter $log_e(x)$ a lot.  And, any time something appears a lot in mathematics, it usually gets a shortcut abbreviation.




\begin{definition}  \textbf{\textcolor{green!50!black}{Natural Logarithm}} 


\[     ln(x) = log_e(x) \]


That is a lowerscase "el" and a lowercase "en".


\end{definition}


That is not an uppercase "eye".  It is not In.  It is ln.





\begin{example}  base e


$log_7(5) = \frac{ln(5)}{ln(7)}$


\end{example}
Everything can be written in terms of base $e$.


Calculators usually have a button titled "ln" or "LN".  When approximating values of logarithms with other bases, we convert them to natural logarithms.  Then we can use the calculator.









\begin{example}  


Approximate $log_{13}(61)$


\begin{explanation}


Using the LN button on the calculator, we get $log_{13}(61) = \frac{ln\left(\answer{61}\right)}{ln\left(\answer{13}\right)} \approx 1.602711512$
\end{explanation}
\end{example}








\begin{example}  


Rewrite the expression $3 \cdot 5^{2x + 1}$ in the form $e^{f(x)}$.


\begin{explanation}


$3 = e^{\answer{ln(3)}}$ \\


$5 = e^{\answer{ln(5)}}$ \\


\[   3 \cdot 5^{2x + 1} = e^{ln(3)} \cdot (e^{ln(5)})^{2x + 1}  =  e^{ln(3)} \cdot e^{ln(5) \cdot (2x + 1)}  = e^{ln(3) + ln(5) \cdot (2x + 1)} \]


\[  e^{ln(3) + ln(5) \cdot (2x + 1)} \] 

\end{explanation}
\end{example}












\end{document}
