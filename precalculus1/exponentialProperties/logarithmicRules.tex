\documentclass{ximera}


\graphicspath{
  {./}
  {ximeraTutorial/}
  {basicPhilosophy/}
}

\newcommand{\mooculus}{\textsf{\textbf{MOOC}\textnormal{\textsf{ULUS}}}}

\usepackage{tkz-euclide}\usepackage{tikz}
\usepackage{tikz-cd}
\usetikzlibrary{arrows}
\tikzset{>=stealth,commutative diagrams/.cd,
  arrow style=tikz,diagrams={>=stealth}} %% cool arrow head
\tikzset{shorten <>/.style={ shorten >=#1, shorten <=#1 } } %% allows shorter vectors

\usetikzlibrary{backgrounds} %% for boxes around graphs
\usetikzlibrary{shapes,positioning}  %% Clouds and stars
\usetikzlibrary{matrix} %% for matrix
\usepgfplotslibrary{polar} %% for polar plots
\usepgfplotslibrary{fillbetween} %% to shade area between curves in TikZ
\usetkzobj{all}
\usepackage[makeroom]{cancel} %% for strike outs
%\usepackage{mathtools} %% for pretty underbrace % Breaks Ximera
%\usepackage{multicol}
\usepackage{pgffor} %% required for integral for loops



%% http://tex.stackexchange.com/questions/66490/drawing-a-tikz-arc-specifying-the-center
%% Draws beach ball
\tikzset{pics/carc/.style args={#1:#2:#3}{code={\draw[pic actions] (#1:#3) arc(#1:#2:#3);}}}



\usepackage{array}
\setlength{\extrarowheight}{+.1cm}
\newdimen\digitwidth
\settowidth\digitwidth{9}
\def\divrule#1#2{
\noalign{\moveright#1\digitwidth
\vbox{\hrule width#2\digitwidth}}}






\DeclareMathOperator{\arccot}{arccot}
\DeclareMathOperator{\arcsec}{arcsec}
\DeclareMathOperator{\arccsc}{arccsc}

















%%This is to help with formatting on future title pages.
\newenvironment{sectionOutcomes}{}{}


\title{Logarithmic Rules}

\begin{document}

\begin{abstract}
rules
\end{abstract}
\maketitle
































\section{Exponents}

Logarithms are exponents. \\


$log_a(b)$ is what you raise $a$ to, to get $b$.

\[    a^{log_a(b)}  = b    \, \text{ for } \,  a,b > 0     \]


We can read it right to left as well.  Any positive number, $b$, can be written with any base, $a$, like $a^{log_a(b)}  = b$.






Logarithms are exponents. Therefore, they should follow all of the exponent rules. \\



$\blacktriangleright$  Let $M$ and $N$ be two positive real numbers.

We can write them as $M = a^{log_a(M)}$ and $N = a^{log_a(N)}$


This allows us to write the product $M \cdot N$ in two different ways.



\[   M \cdot N = a^{log_a(M)} \cdot a^{log_a(N)}                    \]

\[   M \cdot N = a^{log_a(M \cdot N)}                  \]


Therefore, these must be equal.


\[    a^{log_a(M)} \cdot a^{log_a(N)}     =   a^{log_a(M \cdot N)}                \]


Apply an exponent rule:


\[    a^{log_a(M)+log_a(N)}    =   a^{log_a(M \cdot N)}                \]



Since exponential functions are one-to-one, we have 


\[    log_a(M)+log_a(N)    =   log_a(M \cdot N)               \]




\begin{fact} 

\[    log_a(M)+log_a(N)    =   log_a(M \cdot N)      \, \text{ for } \, a, M, N > 0        \]


\end{fact}




$\blacktriangleright$  Let's write the quotient $\frac{M}{N}$ in two different ways.



\[   \frac{M}{N} = \frac{a^{log_a(M)}}{a^{log_a(N)}}                    \]

\[   \frac{M}{N} = a^{log_a\left(\frac{M}{N}\right)}                  \]


Therefore, these must be equal.


\[    \frac{a^{log_a(M)}}{a^{log_a(N)}}    =   a^{log_a\left(\frac{M}{N}\right)}                \]


Apply an exponent rule:


\[    a^{log_a(M) - log_a(N)}    =   a^{log_a\left(\frac{M}{N}\right)}                \]



Since exponential functions are one-to-one, we have 


\[    log_a(M)-log_a(N)    =   log_a\left(\frac{M}{N}\right)             \]








\begin{fact} 

\[    log_a(M)-log_a(N)    =   log_a\left(\frac{M}{N}\right)        \, \text{ for } \, a, M, N > 0        \]


\end{fact}




















$\blacktriangleright$  Let's write $M^N$ in two different ways.



\[   M^N = a^{log_a(M^N)}                  \]

\[   M^N = (a^{log_a(M)})^N =     (a^{N \cdot log_a(M)})             \]


Therefore, these must be equal.


\[    a^{log_a(M^N)}      =    (a^{N \cdot log_a(M)})                \]





Since exponential functions are one-to-one, we have 


\[    log_a(M^N)    =   N \cdot log_a(M)            \]








\begin{fact} 

\[    log_a(M^N)    =   N \cdot log_a(M)       \, \text{ for } \, a M, N  > 0        \]


\end{fact}

















\section{One-to-One}









We have also seen that logarithmic functions are \textbf{one-to-one}. One-to-one means that each range number is paired with a unique domain number.





\begin{image}
\begin{tikzpicture}
  \begin{axis}[
            domain=-10:10, ymax=10, xmax=10, ymin=-10, xmin=-10,
            axis lines =center, xlabel=$t$, ylabel=$y$, grid = major,
            ytick={-10,-8,-6,-4,-2,2,4,6,8,10},
            xtick={-10,-8,-6,-4,-2,2,4,6,8,10},
            ticklabel style={font=\scriptsize},
            every axis y label/.style={at=(current axis.above origin),anchor=south},
            every axis x label/.style={at=(current axis.right of origin),anchor=west},
            axis on top
          ]
          

			\addplot [line width=2, penColor, smooth,samples=200,domain=(-2.95:8),<->] {ln(x+3)/ln(2)-4};
			%\addplot [line width=2, penColor2, smooth,samples=200,domain=(-8:-0.25),<->] {2^(x+4)-3};

			\addplot [line width=1, gray, dashed,samples=200,domain=(-10:10),<->] ({-3},{x});
            %\addplot [line width=1, gray, dashed,samples=200,domain=(-10:10),<->] ({x},{-3});
            %\addplot [line width=1, gray, dashed,samples=200,domain=(-10:10),<->] ({x},{x});


			\addplot[color=penColor,fill=penColor,only marks,mark=*] coordinates{(-2,-4)};
			%\addplot[color=penColor2,fill=penColor,only marks,mark=*] coordinates{(-4,-2)};






           

  \end{axis}
\end{tikzpicture}
\end{image}


Our definition of functions includes a similar rule for domain numbers.  Each domain number is in exactly one pair.  One-to-one puts the same restrictions on range numbers.  Each range number is in exactly one pair. Logarithmic functions have this feature.

In other words, each function value occurs exactly once.


If you know that $log_a(r) = log_a(t)$, then $r=t$ follows. \\


This is an important rule.


\begin{fact} Uniqueness


\[     \text{If } \,  log_a(r) = log_a(t), \,  \text{ then }  \, r=t    \]


\end{fact}





















\begin{example} Solving Equations


Solve $log_5(21 + x) = 2$


\textbf{\textcolor{purple!50!blue!90!black}{SOLUTION}}


$log_5(21 + x) = 2$

$5^{log_5(21 + x)} = 5^2$

$21 + x = 5^2 = 25$    Since, $log_5(x)$ is a one-to-one function.

$x = 4$

\end{example}










\begin{example} Solving Equations


Solve $log_2(y) + log_2(y-4) = log_2(y+6)$


\textbf{\textcolor{purple!50!blue!90!black}{SOLUTION}}


$log_2(y) + log_2(y-4) = log_2(y+6)$

$log_2(y(y-4)) = log_2(y+6)$

$y(y-4) = y+6$    Since, $log_2(x)$ is a one-to-one function.

$y^2 - 4y - y - 6 = 0$

$y^2 - 5y - 6 = 0$

$(y-6)(\answer{y+1}) = 0$


Either $y-6 = 0$ or $y+1 = 0$

Either $y = 6$ or $y = -1$

However, $y \ne -1$, since $\answer{-1}$ is not in the domain of $log_2(y)$.

Therefore, just one solution: $6$.

The solution set is $\{ 6 \}$

\end{example}






\begin{example} Solving Equations


Solve $7^k = 5$


\textbf{\textcolor{purple!50!blue!90!black}{SOLUTION}}


We are looking for the number that you raise $7$ to, to get $5$.  That number  is $log_7(5)$.

\end{example}















\section{Change of Base}


Any positive number can be written as a power of any other positive number.  We can change the base. That is what logarithms do.


\begin{quote}

Any positive number, $b$, can be written as a power of $a$, any other positive number.

\[    b = a^{log_a(b)}  \]

\end{quote}

We can also write any logarithm in terms of any base.  We can write $log_a(b)$ in terms of $log_c$.




First, $a$ and $b$ can be written in terms of some third base, $c$, using logarithms.


\[    b = c^{log_c(b)} \,   \text{ and } \,      a = c^{log_c(a)}      \]



Second, substitute these in for $a$ and $b$ bases in $b = a^{log_a(b)}$.



\[   c^{log_c(b)} = \left(c^{log_c(a)}\right)^{log_a(b)}  \]



\[   c^{log_c(b)} = c^{log_c(a) \cdot log_a(b)}  \]



\[   log_c(b) = log_c(a) \cdot log_a(b)  \]


\[   \frac{log_c(b)}{log_c(a)} =  log_a(b)  \]


This is known as the \textbf{Change of Base Formula}.








\begin{fact}  Change of Base Formula

\[   log_a(b)  =  \frac{log_c(b)}{log_c(a)}        \, \text{ for } \, a, b, c  > 0        \]


\end{fact}









If we can write any power in terms of any base, and any logarithm in terms of any base, then let's just pick one base to write everything in.


We have such a base: $e$


\begin{example}


Write $log_3(5)$ in terms of logarithms base $7$.


\textbf{\textcolor{purple!50!blue!90!black}{SOLUTION}}


Following the change of base template: $log_a(b)  =  \frac{log_c(b)}{log_c(a)} $ gives



\[   log_3(5)  =  \frac{log_7(5)}{log_7(3)}         \]




\end{example}




















\section{e}



$e$ was defined in a very weird way.

\[   e = \lim_{x \to \infty}  \left(\answer{1 + \frac{1}{x}}\right)^x      \]


It is weird now.  But as you continue through Calculus, you will see $e$ pop up all over the place.  It seems to have a connection to everything.  So much so that scientists, engineers, and mathematicians have  adopted $e$ as the prefered base for everything.


This means that we encounter $log_e(x)$ a lot.  And, any time something appears a lot in mathematics, it usually gets a shortcut abbreviation.




\begin{definition}  Natural Logarithm


\[     ln(x) = log_e(x) \]


That is a lowerscase "el" and a lowercase "en".


\end{definition}


That is not an uppercase "eye".  It is not IN.  It is ln.





\begin{example}  base e


$log_7(5) = \frac{ln(5)}{ln(7)}$


\end{example}
Everything can be written in terms of base $e$.


Calculators usually have a button titled "ln" or "LN".  When approximating values of logarithms with other bases, we convert them to natural logarithms.  Then we can use the calculator.









\begin{example}  


Approximate $log_13(61)$


\textbf{\textcolor{purple!50!blue!90!black}{SOLUTION}}


Using the LN button on the calculator, we get $log_13(61) = \frac{ln\left(\answer{61}\right)}{ln\left(\answer{13}\right)} \approx 1.602711512$

\end{example}








\begin{example}  


Rewrite the expression $3 \cdot 5^{2x + 1}$ in the form $e^{f(x)}$.


\textbf{\textcolor{purple!50!blue!90!black}{SOLUTION}}


$3 = e^{\answer{ln(3)}}$ \\


$5 = e^{\answer{ln(5)}}$ \\


\[   3 \cdot 5^{2x + 1} = e^{ln(3)} \cdot (e^{ln(5)})^{2x + 1}  =  e^{ln(3)} \cdot e^{ln(5) \cdot (2x + 1)}  = e^{ln(3) + ln(5) \cdot (2x + 1)} \]




\end{example}












\end{document}
