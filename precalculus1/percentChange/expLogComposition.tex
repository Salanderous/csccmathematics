\documentclass{ximera}


\graphicspath{
  {./}
  {ximeraTutorial/}
  {basicPhilosophy/}
}

\newcommand{\mooculus}{\textsf{\textbf{MOOC}\textnormal{\textsf{ULUS}}}}

\usepackage{tkz-euclide}\usepackage{tikz}
\usepackage{tikz-cd}
\usetikzlibrary{arrows}
\tikzset{>=stealth,commutative diagrams/.cd,
  arrow style=tikz,diagrams={>=stealth}} %% cool arrow head
\tikzset{shorten <>/.style={ shorten >=#1, shorten <=#1 } } %% allows shorter vectors

\usetikzlibrary{backgrounds} %% for boxes around graphs
\usetikzlibrary{shapes,positioning}  %% Clouds and stars
\usetikzlibrary{matrix} %% for matrix
\usepgfplotslibrary{polar} %% for polar plots
\usepgfplotslibrary{fillbetween} %% to shade area between curves in TikZ
\usetkzobj{all}
\usepackage[makeroom]{cancel} %% for strike outs
%\usepackage{mathtools} %% for pretty underbrace % Breaks Ximera
%\usepackage{multicol}
\usepackage{pgffor} %% required for integral for loops



%% http://tex.stackexchange.com/questions/66490/drawing-a-tikz-arc-specifying-the-center
%% Draws beach ball
\tikzset{pics/carc/.style args={#1:#2:#3}{code={\draw[pic actions] (#1:#3) arc(#1:#2:#3);}}}



\usepackage{array}
\setlength{\extrarowheight}{+.1cm}
\newdimen\digitwidth
\settowidth\digitwidth{9}
\def\divrule#1#2{
\noalign{\moveright#1\digitwidth
\vbox{\hrule width#2\digitwidth}}}






\DeclareMathOperator{\arccot}{arccot}
\DeclareMathOperator{\arcsec}{arcsec}
\DeclareMathOperator{\arccsc}{arccsc}

















%%This is to help with formatting on future title pages.
\newenvironment{sectionOutcomes}{}{}


\title{Composed}

\begin{document}

\begin{abstract}
reversed functions
\end{abstract}
\maketitle







We have seen that given an exponential function, we can reverse the pairs and get a logarithmic function.  If the pairs of the exponential function look like $(D,R)$, then the pairs of the logarithmic function look like $(R,D)$.  The same numbers are paired together.  The just swap their roles between domain and range.

\section{Composition}

What if we were to compose these two functions?  That is we make a new function whereby we take a domain number for the exponential function, get the function value, and then take this function value as the input into the logarithmic funciton, and get is output. \\





\textbf{Setup}

We have two partnered exponential and logarithmic functions: $E(x)$ and $L(t)$.  \\

\begin{enumerate}
\item Let $d$ be a domain number of $E(x)$.
\item $d$ is in a pair: $(d, E(d))$.
\item View $E(d)$ as a number in the domain of $L(t)$.
\item We already know that $(E(d),d)$ is a pair in $L(t)$, because $E$ and $L$ are partnered.  Their pairs are reversed.
\item That tells us that $L(t)=d$
\end{enumerate}



$d$ was partnered with $E(d)$.  then we reversed the pair and $E(d)$ is partnered with $d$.

The composition $L \circ E$ pairs $d$ with itself: $(d,d)$ for all of the domain numbers. \\



\[   (L \circ E)(x) = x        \]




\begin{example}  Composition

We have seen that $E(x) = -2 \, \log_3(4-x)$ and $L(t) = 4 - 3^{-\frac{t}{2}}$ are two partnered exponential and logarithmic functions. \\


Map out the pairs for the composition, $(L \circ E)(d)$.

\begin{explanation}


Let $d$ be in the domain of $E(x)$, then $E(d) = -2 \, \log_3(4-d)$. \\

Now, let's put this number into $L$. \\



\[    (L \circ E)(d)  =  L(-2 \, \log_3(4-d)) = 4 - 3^{-\frac{-2 \, \log_3(4-d)}{2}}              \]


\[    (L \circ E)(d)  =  4 - 3^{ \log_3(4-d)}              \]


\[    (L \circ E)(d)  =  4 - \left( \answer{4-d} \right)             \]


\[    (L \circ E)(d)  = d             \]

\end{explanation}
\end{example}




Let's try it the other way.










\begin{example}  Composition

We have seen that $E(x) = -2 \, \log_3(4-x)$ and $L(t) = 4 - 3^{-\frac{t}{2}}$ are two partnered exponential and logarithmic functions. \\



Map out the pairs for the composition, $(E \circ L)(d)$.

\begin{explanation}



Let $d$ be in the domain of $L(t)$, then $L(d) = 4 - 3^{-\frac{d}{2}}$. \\

Now, let's put this number into $E$. \\



\[    (E \circ L)(d)  =  E(4 - 3^{-\frac{d}{2}}) = -2 \, \log_3(4 - (4 - 3^{-\frac{d}{2}}))            \]


\[    (E \circ L)(d)  =  -2 \, \log_3( 3^{-\frac{d}{2}})              \]


\[    (E \circ L)(d)  = -2 \, \left( \answer{-\frac{d}{2}} \right)              \]


\[    (E \circ L)(d)  = d            \]

\end{explanation}
\end{example}

























\begin{center}
\textbf{\textcolor{green!50!black}{ooooo=-=-=-=-=-=-=-=-=-=-=-=-=ooOoo=-=-=-=-=-=-=-=-=-=-=-=-=ooooo}} \\

more examples can be found by following this link\\ \link[More Examples of Percent Change]{https://ximera.osu.edu/csccmathematics/precalculus1/precalculus1/percentChange/examples/exampleList}

\end{center}




\end{document}
