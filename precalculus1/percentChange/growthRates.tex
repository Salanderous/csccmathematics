\documentclass{ximera}


\graphicspath{
  {./}
  {ximeraTutorial/}
  {basicPhilosophy/}
}

\newcommand{\mooculus}{\textsf{\textbf{MOOC}\textnormal{\textsf{ULUS}}}}

\usepackage{tkz-euclide}\usepackage{tikz}
\usepackage{tikz-cd}
\usetikzlibrary{arrows}
\tikzset{>=stealth,commutative diagrams/.cd,
  arrow style=tikz,diagrams={>=stealth}} %% cool arrow head
\tikzset{shorten <>/.style={ shorten >=#1, shorten <=#1 } } %% allows shorter vectors

\usetikzlibrary{backgrounds} %% for boxes around graphs
\usetikzlibrary{shapes,positioning}  %% Clouds and stars
\usetikzlibrary{matrix} %% for matrix
\usepgfplotslibrary{polar} %% for polar plots
\usepgfplotslibrary{fillbetween} %% to shade area between curves in TikZ
\usetkzobj{all}
\usepackage[makeroom]{cancel} %% for strike outs
%\usepackage{mathtools} %% for pretty underbrace % Breaks Ximera
%\usepackage{multicol}
\usepackage{pgffor} %% required for integral for loops



%% http://tex.stackexchange.com/questions/66490/drawing-a-tikz-arc-specifying-the-center
%% Draws beach ball
\tikzset{pics/carc/.style args={#1:#2:#3}{code={\draw[pic actions] (#1:#3) arc(#1:#2:#3);}}}



\usepackage{array}
\setlength{\extrarowheight}{+.1cm}
\newdimen\digitwidth
\settowidth\digitwidth{9}
\def\divrule#1#2{
\noalign{\moveright#1\digitwidth
\vbox{\hrule width#2\digitwidth}}}






\DeclareMathOperator{\arccot}{arccot}
\DeclareMathOperator{\arcsec}{arcsec}
\DeclareMathOperator{\arccsc}{arccsc}

















%%This is to help with formatting on future title pages.
\newenvironment{sectionOutcomes}{}{}


\title{Growth Rates}

\begin{document}

\begin{abstract}
constant percentage
\end{abstract}
\maketitle










$\blacktriangleright$ \textbf{\textcolor{purple!85!blue}{Linear Functions}}   \\


\begin{quote}
The defining characteristic of linear functions is that the have a constant growth rate.
\end{quote}





\[   \frac{L(b)-L(a)}{b-a} = m       \]

This led to the formula for linear functions:  $L(x) = m(x-a) + L(a)$


This tells us that no matter where you are in the domain, of $L$, if you move the same distance inside the domain, then the value of $L$ changes by the same \textbf{\textcolor{purple!85!blue}{amount}}. \\


Exponential functions have a similar growth, but with percentages. \\



$\blacktriangleright$ \textbf{\textcolor{purple!85!blue}{Exponential Functions}} \\



Suppose we have a function with this property:  No mater where you are in the domain, if you move the same amount in the domain, then the function value changes by the same \textbf{\textcolor{purple!85!blue}{percent}}.  Then what does the formula look like?



\begin{observation}

\begin{itemize}
\item Let's call our function $E$.
\item Let's call our original position in the domain, $d$.
\item Let's call $D$ the distance moved from $d$, inside the domain.
\item Let's say the percentage change is $p \cdot 100\%$. (Think of $p$ as a decimal number.)
\end{itemize}



Inside the domain we are moving from $D$ to $d + D$, a change of $D$.  How does $E$ change?



In this story, the starting function value is $E(d)$ and the finishing function value is $E(d+D)$.  Therefore, the change in the function value is given by $E(d+D) - E(d)$.

The supposed property tells us that this change is equal to percentage $p$ of the original value $E(d)$.



\[
E(d+D) - E(d) = p \, E(d)
\]

\[
E(d+D) = E(d) + p \, E(d)
\]


\[
E(d+D) = E(d) (1+p)
\]



To get the value of $E(d+D)$, we multiply $E(d)$ by $1+p$. \\









If we travel another $D$, then we move from $d+D$ to $d+2D$, inside the domain. \\

We have travelled another distance of $D$. Our supposed property says that we should get another $p$ percent.



This time our starting value is $E(d+D) = E(d) (1+p)$ and we would like to calculate the value of $E(d+2D)$. \\




\[
E(d + 2D) = E(d+D)(1+p)  = E(d) (1+p) \cdot (1+p)
\]



\[
E(d + 2D) =  E(d) (1+p)^2
\]


If we travel another $D$, then we get another $p$ percent of our original starting value, $E(d)(1+p)^2$.


\[
E(d + 3D) = E(d)(1+p)^2 \cdot (1+p) = E(d) (1+p)^3
\]



If we travel "$x$" number of these $D$ distances from $d$, then we get

\[
E(d + x \cdot D) = E(d)(1+p)^x 
\]

An exponential function.

\end{observation}











\begin{definition} \textbf{\textcolor{green!50!black}{Exponential Functions}}


\textbf{Exponential functions} are those functions that experience a constant percentage growth rate. \\


Their formulas look like

\[
exp(x) = a \cdot b^x
\]


where $a \ne 0$ and $b > 0$.

$a$ is called the \textbf{coefficient} and $b$ is called the \textbf{base}.

\end{definition}

















\begin{fact}  Not Constant Growth



Exponential functions do not have a constant growth rate. \\

\begin{explanation}

Let $G(t) = 5 \cdot \left(\frac{3}{2}\right)^t$ with domain \textbf{$R$}. 






\begin{itemize}
\item $G(0) = 5$
\item $G(1) = \frac{15}{2}$
\item $G(2) = \frac{45}{4}$
\end{itemize}



Each time we moved $1$ in the domain, $G(t)$ DID NOT increase by the same amount.

$G(1) - G(0) = \answer{\frac{5}{2}}$ \\

$G(2) - G(1) = \answer{\frac{15}{4}}$ \\

$ \frac{5}{2}  \ne  \frac{15}{4} $, therefore $G(t)$ does not have a constant rate-of-change. \\


\end{explanation}

\end{fact}




This example function, $G$, has a constant percentage rate-of-change. \\

No matter where you are in the domain, when you move $1$, the value of $G$ increases by the same percentage (is multiplied by the same factor).


Let $a$ be a real number.  The percentage rate-of-change of $G$ over the interval $[a,a+1]$ is 



\[    \frac{5 \left(\frac{3}{2}\right)^{a+1} - 5 \left(\frac{3}{2}\right)^a}{5 \left(\frac{3}{2}\right)^a}       \]


\[   \frac{5 \left(\frac{3}{2}\right)^a (\left(\frac{3}{2}\right) - 1) }{5 \left(\frac{3}{2}\right)^a}    \]


\[   \left(\frac{3}{2}\right) - 1 = \frac{1}{2}   \]

Every time we move $1$ in the domain, $G$ \textbf{\textcolor{blue!55!black}{increases}} by a factor of $\frac{1}{2}$.  $G$ \textbf{\textcolor{blue!55!black}{increases}} by $50\%$.  Therefore, we multiply by $1 + \frac{1}{2} = \frac{3}{2}$, which is the base.





The general template for an exponential function is 

\[   exp(t) = a \cdot b^t   \, \text{ where } \,  a \ne 0  \, \text{ and } \,    b > 0   \]



\begin{example} Percentage Change



Let $H(t) = 3 \, \cdot \, 2^t$ \\

$H$ is an exponential function, which means it has a constant percentage change. \\

Values of $H$:

\begin{itemize}
\item $H(0) = \answer{3}$

\item $H(1) = \answer{6}$

\item $H(2) = \answer{12}$
\end{itemize}



Change in $H$:

\begin{itemize}
\item $H(1) - H(0) = \answer{3}$

\item $H(2) - H(1) = \answer{6}$

\end{itemize}





Percentage change in $H$:

\begin{itemize}
\item $\frac{H(1) - H(0)}{H(0)} = \answer{1}$, which is $\answer{100}\%$

\item $\frac{H(2) - H(1)}{H(1)} = \answer{1}$, which is $\answer{100}\%$

\end{itemize}






\end{example}



Every time $t$ increases by $1$, the value of $H$ gets multiplied by another $2$, which means its value is doubled, which means it \textbf{changes} by its current value, which means it grows by $100\%$.











\begin{example} Percentage Change



Let $k(x) = 5 \, \left(\frac{3}{4}\right)^x$ \\

$k$ is an exponential function, which means it has a constant percentage change. \\

Values of $k$:

\begin{itemize}
\item $k(0) = \answer{5}$

\item $k(1) = \answer{\frac{15}{4}}$

\item $k(2) = \answer{\frac{45}{16}}$
\end{itemize}



Change in $k$:

\begin{itemize}
\item $k(1) - k(0) = \answer{-\frac{5}{4}}$

\item $k(2) - k(1) = \answer{-\frac{15}{16}}$

\end{itemize}





Percentage change in $k$:

\begin{itemize}
\item $\frac{k(1) - k(0)}{k(0)} = \answer{-\frac{1}{4}}$, which is $\answer{-25}\%$

\item $\frac{k(2) - k(1)}{k(1)} = \answer{-\frac{1}{4}}$, which is $\answer{-25}\%$

\end{itemize}






\end{example}

Every time $x$ increases by $1$, the value of $k$ gets multiplied by another $\frac{3}{4}$, which means its value decreases. It loses $\frac{1}{4}$ of its value. It has a negative growth rate. It grows by $-25\%$.








\section{A New Number: e}


Consider this function 

\[
e(x) = \left(1 + \frac{1}{x}\right)^x
\]



Here is a graph


\begin{center}
\desmos{0tuazdb3kc}{400}{300}
\end{center}


The graph of this function has a horizontal asymptote.  The horizontal asymptote intersects the $y$-axis at a number.  This number is called $e$.   It is a special number and gets its own symbol, much like $\pi$ gets its own symbol.



\[
e \approx  2.718281828459045235360287471352662497757247093699959574966967627724076630353547594571382178525166427
\]


$e$ is an irrational number that has many connections to natural growth.  The importance of $e$ will show up in Calculus.



















\begin{example} Percentage Change



Let $p(x) = 3 \, e^x$ \\

$p$ is an exponential function, which means it has a constant percentage change. \\



\begin{explanation}

Values of $p$:

\begin{itemize}
\item $p(0) = 3$

\item $p(1) = 3 e$

\item $p(2) = 3 e^2$
\end{itemize}



Change in $p$:

\begin{itemize}
\item $p(1) - p(0) = 3 (e-1)$

\item $p(2) - p(1) = 3 e (e-1)$

\end{itemize}





Percentage change in $p$:

\begin{itemize}
\item $\frac{p(1) - p(0)}{p(0)} = e - 1$, which is about $171.8\%$

\item $\frac{p(2) - p(1)}{p(1)} = e-1 $, which is about $171.8\%$

\end{itemize}

\end{explanation}

\end{example}


















\begin{example} Percentage Change



Let $R(t) = a \cdot r^t$, with $a$ and $r$ real numbers and $r>0$ \\

$R$ is an exponential function, which means it has a constant percentage change. \\

Values of $R$:

\begin{itemize}
\item $R(0) = a$

\item $R(1) = a \, r$

\item $R(2) = a \, r^2$
\end{itemize}



Change in $R$:

\begin{itemize}
\item $R(1) - R(0) = \answer{a (r-1)}$

\item $R(2) - R(1) = \answer{a r (r-1)}$

\end{itemize}





Percentage change in $R$:

\begin{itemize}
\item $\frac{R(1) - R(0)}{R(0)} = \answer{r-1}$

\item $\frac{R(2) - R(1)}{R(1)} = \answer{r-1}$

\end{itemize}


\end{example}













\begin{center}
\textbf{\textcolor{green!50!black}{ooooo=-=-=-=-=-=-=-=-=-=-=-=-=ooOoo=-=-=-=-=-=-=-=-=-=-=-=-=ooooo}} \\

more examples can be found by following this link\\ \link[More Examples of Percent Change]{https://ximera.osu.edu/csccmathematics/precalculus1/precalculus1/percentChange/examples/exampleList}

\end{center}




\end{document}
