\documentclass{ximera}


\graphicspath{
  {./}
  {ximeraTutorial/}
  {basicPhilosophy/}
}

\newcommand{\mooculus}{\textsf{\textbf{MOOC}\textnormal{\textsf{ULUS}}}}

\usepackage{tkz-euclide}\usepackage{tikz}
\usepackage{tikz-cd}
\usetikzlibrary{arrows}
\tikzset{>=stealth,commutative diagrams/.cd,
  arrow style=tikz,diagrams={>=stealth}} %% cool arrow head
\tikzset{shorten <>/.style={ shorten >=#1, shorten <=#1 } } %% allows shorter vectors

\usetikzlibrary{backgrounds} %% for boxes around graphs
\usetikzlibrary{shapes,positioning}  %% Clouds and stars
\usetikzlibrary{matrix} %% for matrix
\usepgfplotslibrary{polar} %% for polar plots
\usepgfplotslibrary{fillbetween} %% to shade area between curves in TikZ
\usetkzobj{all}
\usepackage[makeroom]{cancel} %% for strike outs
%\usepackage{mathtools} %% for pretty underbrace % Breaks Ximera
%\usepackage{multicol}
\usepackage{pgffor} %% required for integral for loops



%% http://tex.stackexchange.com/questions/66490/drawing-a-tikz-arc-specifying-the-center
%% Draws beach ball
\tikzset{pics/carc/.style args={#1:#2:#3}{code={\draw[pic actions] (#1:#3) arc(#1:#2:#3);}}}



\usepackage{array}
\setlength{\extrarowheight}{+.1cm}
\newdimen\digitwidth
\settowidth\digitwidth{9}
\def\divrule#1#2{
\noalign{\moveright#1\digitwidth
\vbox{\hrule width#2\digitwidth}}}






\DeclareMathOperator{\arccot}{arccot}
\DeclareMathOperator{\arcsec}{arcsec}
\DeclareMathOperator{\arccsc}{arccsc}

















%%This is to help with formatting on future title pages.
\newenvironment{sectionOutcomes}{}{}



\author{Lee Wayand}

\begin{document}
\begin{exercise}



The best way to analyze a rationa function is with its factored form. \\


\textbf{Analyze R(x)} \\

\[
R(x) = \frac{(x+5)(x-3)}{x+2}, \text { with its natural domain } 
\]




We want to say as much as we can exactly.  Then, we'll turn to the graph for approximations. \\






\textbf{\textcolor{blue!55!black}{$\blacktriangleright$ Domain: }} We are given that the domain is the natural domain. Therefore, we need to exclude any zero of the denominator. $(-\infty, -2) \cup (-2, \infty)$.


\textbf{\textcolor{blue!55!black}{$\blacktriangleright$ Continuity: }}  All rational functions are continuous on their domain.  So, $R$ has no discontinuities and just one singularity.




\textbf{\textcolor{blue!55!black}{$\blacktriangleright$ End-Behavior: }} 


$R$ is a rational function. The degree of the numerator is $2$, which is greater than than the degree of the denominator.  Therefore, $R$ is unbounded as $x$ tends to $-\infty$ or $\infty$.  We just have to figure out the sign.


When $x$ is very large and negative, we have 

\begin{itemize}
\item $x+5 < 0$
\item $x-3 < 0$
\item $x+2 < 0$
\end{itemize}


That makes $R < 0$

\[
\lim\limits_{x \to -\infty} R(x) = -\infty
\]


When $x$ is very large and positive, then all three factors are positive and $R$ is positive.


\[
\lim\limits_{x \to \infty} R(x) = \infty
\]





\textbf{\textcolor{blue!55!black}{$\blacktriangleright$ Zeros: }}  


We are given the factored form.  So, we can see that the zeros are $-5$ and $3$.



Both zeros and the singularity have a multiplicity of $1$, which is odd.  Therefore, $R(x)$ changes signs over each one.




\textbf{\textcolor{blue!55!black}{$\blacktriangleright$ Behavior Around the Singularity: }}  


When $x$ is close to $-2$, but less than $-2$, then we have 


\begin{itemize}
\item $x+5 ~ 3 > 0$
\item $x-3 ~ -5 < 0$
\item $x+2 ~ 0 < 0$
\end{itemize}


The numerator will be around $-15$ and the denominator will be heading to $0$ through negative numbers.  $R$ will be positive and unbounded.




\[
\lim\limits_{x \to -2^-} R(x) = \infty
\]






When $x$ is close to $-2$, but greater than $-2$, then we have 


\begin{itemize}
\item $x+5 ~ 3 > 0$
\item $x-3 ~ -5 < 0$
\item $x+2 ~ 0 > 0$
\end{itemize}


The numerator will be around $-15$ and the denominator will be heading to $0$ through positive numbers.  $R$ will be negative and unbounded.




\[
\lim\limits_{x \to -2^+} R(x) = -\infty
\]





\textbf{\textcolor{blue!55!black}{$\blacktriangleright$ Range: }}


$B$ is continuous on $(-\infty, -2)$ and  $\lim\limits_{x \to -\infty} R(x) = -\infty$  and $\lim\limits_{x \to -2^-} R(x) = \infty$.  This tells us that the range is $(-\infty, \infty)$.







So far, all of our algebraic reasoning agrees with the graph.




\begin{center}
\desmos{mfppvb2aoo}{400}{300}
\end{center}










\textbf{\textcolor{blue!55!black}{$\blacktriangleright$ Behavior: }}  




Now, we want to figure out where $R$ is increasing and decreasing and then use that information to identify any local maximums or minimums.



\textbf{\textcolor{red!90!darkgray}{$\blacktriangleright$}} Our algebra cannot do that. We'll need Calculus to do that. \\










At this point, we can only turn to the graph for approximations.




The graph suggests that $R$ is always increasing and there are no local maximums or minimums.





\subsection{with Calculus...}



Calculus will give us some new algebra tools by which we can obtain a formula for a derivative.



\[
B'(x) \frac{x^2 + 4x + 19}{(x+2)^2}
\]



\begin{itemize}
\item Where $B'(x) < 0$, $B$ is increasing.
\item Where $B'(x) > 0$, $B$ is decreasing.
\item Where $B'(x) = 0$ or DNE identifies critical numbers.
\end{itemize}





The quadratic formula tells us that the zeros of the numerator are not real numbers.


\[
\frac{-4 \pm \sqrt{16 - 4 (1) (19)}}{2}
\]


Therefore, the numerator is always the same sign.  It never changes sign.  Since, the numerator equals $19$ when $x=0$, we know that the numerator is always positive.  

The denominator is a square, so it is never negative.


This shows that $B'(x) > 0$ on the whole domain and $B(x)$ is always increasing.




\end{exercise}
\end{document}