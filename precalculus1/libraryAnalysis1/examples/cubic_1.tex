\documentclass{ximera}


\graphicspath{
  {./}
  {ximeraTutorial/}
  {basicPhilosophy/}
}

\newcommand{\mooculus}{\textsf{\textbf{MOOC}\textnormal{\textsf{ULUS}}}}

\usepackage{tkz-euclide}\usepackage{tikz}
\usepackage{tikz-cd}
\usetikzlibrary{arrows}
\tikzset{>=stealth,commutative diagrams/.cd,
  arrow style=tikz,diagrams={>=stealth}} %% cool arrow head
\tikzset{shorten <>/.style={ shorten >=#1, shorten <=#1 } } %% allows shorter vectors

\usetikzlibrary{backgrounds} %% for boxes around graphs
\usetikzlibrary{shapes,positioning}  %% Clouds and stars
\usetikzlibrary{matrix} %% for matrix
\usepgfplotslibrary{polar} %% for polar plots
\usepgfplotslibrary{fillbetween} %% to shade area between curves in TikZ
\usetkzobj{all}
\usepackage[makeroom]{cancel} %% for strike outs
%\usepackage{mathtools} %% for pretty underbrace % Breaks Ximera
%\usepackage{multicol}
\usepackage{pgffor} %% required for integral for loops



%% http://tex.stackexchange.com/questions/66490/drawing-a-tikz-arc-specifying-the-center
%% Draws beach ball
\tikzset{pics/carc/.style args={#1:#2:#3}{code={\draw[pic actions] (#1:#3) arc(#1:#2:#3);}}}



\usepackage{array}
\setlength{\extrarowheight}{+.1cm}
\newdimen\digitwidth
\settowidth\digitwidth{9}
\def\divrule#1#2{
\noalign{\moveright#1\digitwidth
\vbox{\hrule width#2\digitwidth}}}






\DeclareMathOperator{\arccot}{arccot}
\DeclareMathOperator{\arcsec}{arcsec}
\DeclareMathOperator{\arccsc}{arccsc}

















%%This is to help with formatting on future title pages.
\newenvironment{sectionOutcomes}{}{}



\author{Lee Wayand}

\begin{document}
\begin{exercise}



The best way to analyze a polynomial is with its factored form. \\


\textbf{Analyze p(x)} \\

\[
p(x) = 3 x^2 + 5 x^2 - 107 x + 35 \, \text { with its natural domain } 
\]






\subsection{Algebraic Language}



\textbf{\textcolor{blue!55!black}{$\blacktriangleright$ Domain: }} We are given that the domain is the natural domain and the natural domain of all polynomials is $(-\infty, \infty)$.


\textbf{\textcolor{blue!55!black}{$\blacktriangleright$ Continuity: }}  All polynomials are continuous on their domain.  So, $Q$ has no discontinuities.  Since, the domain is all real numbers, there can be no singularities.



\textbf{\textcolor{blue!55!black}{$\blacktriangleright$ Zeros: }}  


There are several approaches to factoring.  Identifying zeros or roots is one.  For this we need a graph.




\begin{center}
\desmos{wgi6pkoje5}{400}{300}
\end{center}



The DESMOS graph is suggesting that perhaps $-7$ and $5$ might be roots and then there is a third root.  Let's check:



\[
p(-7) = 3 (-7)^2 + 5 (-7)^2 - 107 (-7) + 35 = 0
\]



\[
p(5) = 3 (5)^2 + 5 (5)^2 - 107 (5) + 35 = 0
\]



We can now factor $Q$. \\


We know that $(x+7)$ and $x-5$ are both factors of $p$. We just need to figure out the third factor.




\[
p(x) = 3 x^2 + 5 x^2 - 107 x + 35 = (x+7)(x-5)(A x + B)
\]




\[
p(x) = 3 x^2 + 5 x^2 - 107 x + 35 = A x^3 + (2 A + B) x^2 + (-35 A + 2 B) x - 35 B 
\]


Comparing the constant terms tells us that $B = -1$ \\



Comparing leading terms tells us that $A = 3$.




\[
p(x) = 3 x^2 + 5 x^2 - 107 x + 35 = (x+7)(x-5)(3 x - 1)
\]





The zeros or roots are $-7$, $\frac{1}{3}$, and $5$.  All three have multiplicity of $1$, which is odd.  Therefore, $p$ changes signs across all three.










\textbf{\textcolor{blue!55!black}{$\blacktriangleright$ End-Behavior: }} Polynomials with odd degree (like cubics) have the different end-behavior on either side.  Since $p$ has a positive leading coefficient, $Q$ is unbounded negatively as $w$ tends to $-\infty$. and $Q$ is unbounded positively as $w$ tends to $\infty$.

\[
\limlimits_{x \to -\infty} p(x) = -\infty
\]


\[
\limlimits_{x \to \infty} p(x) = \infty
\]


There is no global maximum or global minimum. \\



\textbf{\textcolor{blue!55!black}{$\blacktriangleright$ Behavior: }}  The end-behavior for this cubic tells us that 



\begin{itemize}
\item $p$ increases on $(-\infty, -7)$ 
\item $p$ decreases on $\left( -7, \frac{1}{3} \right)$
\item $p$ increases on $\left( \frac{1}{3}, \infty \right)$
\end{itemize}




\textbf{\textcolor{blue!55!black}{$\blacktriangleright$ Extrema: }}  We'll need Calculus to know how to get a derivative for cubic polynomials.  So, we cannot get the criticval numbers for $p$.  That means we cannot get exact information about the local maximum or minimum.  However, we can establish some information.





Since the cubic polynomial $p$ changes sign from negative to positive across $-7$, $p$ must be increasing around $-7$. Since the cubic polynomial $p$ changes sign from positive to negative across $\frac{1}{3}$, $p$ must be decreasing around $\frac{1}{3}$. 


We also know that $p(-7) = 0$ and $p\left( \frac{1}{3} \right) = 0$ and that $p$ is continuous on $\left[-7, \frac{1}{3}  \right]$.  There must be a local maximum somewhere on this interval.


Similar reasoning tells us that must be a local minimum somewhere on $\left[ \frac{1}{3}, 5 \right]$.





At this point, we can only turn to the graph for approximations.



$p$ has a local maximum of approximately $351.072$ at approximately $-4.048$.  $-4.048$ is a critical number. \\





$p$ has a local minimum of approximately $-160.126$ at approximately $2.937$.  $2.937$ is a critical number. \\









\textbf{\textcolor{blue!55!black}{$\blacktriangleright$ Range: }}


$p$ is continuous and  $\lim\limits_{x \to -\infty} p(x) = -\infty$  and $\lim\limits_{x \to \infty} p(x) = \infty$.  This tells us that the range is $(-\infty, \infty)$.








\subsection{Graphical Language}







\textbf{\textcolor{blue!55!black}{$\blacktriangleright$ }}  The graph of $y = p(x)$ is that of a cubic with one hill and one valley. It has no holes or breaks. \\

\textbf{\textcolor{blue!55!black}{$\blacktriangleright$ }}  The graph has three intercepts:


\[
\left( -7, 0  \right) \,     \left( \frac{\sqrt{1}}{3}, 0  \right)  \, \text{ and } \,  (5, 0)
\]









\textbf{\textcolor{blue!55!black}{$\blacktriangleright$ }}  The graph slopes up to the right until it hits the top of a hill at approximately the point, $(-4.048, 351.072)$, then it slopes down to the right until it hits the bottom of a valley at approximately the point, $(2.937, -160.126)$, then is slopes up tot he right.



The graph has no highest or lowest points.






\end{exercise}
\end{document}