\documentclass{ximera}


\graphicspath{
  {./}
  {ximeraTutorial/}
  {basicPhilosophy/}
}

\newcommand{\mooculus}{\textsf{\textbf{MOOC}\textnormal{\textsf{ULUS}}}}

\usepackage{tkz-euclide}\usepackage{tikz}
\usepackage{tikz-cd}
\usetikzlibrary{arrows}
\tikzset{>=stealth,commutative diagrams/.cd,
  arrow style=tikz,diagrams={>=stealth}} %% cool arrow head
\tikzset{shorten <>/.style={ shorten >=#1, shorten <=#1 } } %% allows shorter vectors

\usetikzlibrary{backgrounds} %% for boxes around graphs
\usetikzlibrary{shapes,positioning}  %% Clouds and stars
\usetikzlibrary{matrix} %% for matrix
\usepgfplotslibrary{polar} %% for polar plots
\usepgfplotslibrary{fillbetween} %% to shade area between curves in TikZ
\usetkzobj{all}
\usepackage[makeroom]{cancel} %% for strike outs
%\usepackage{mathtools} %% for pretty underbrace % Breaks Ximera
%\usepackage{multicol}
\usepackage{pgffor} %% required for integral for loops



%% http://tex.stackexchange.com/questions/66490/drawing-a-tikz-arc-specifying-the-center
%% Draws beach ball
\tikzset{pics/carc/.style args={#1:#2:#3}{code={\draw[pic actions] (#1:#3) arc(#1:#2:#3);}}}



\usepackage{array}
\setlength{\extrarowheight}{+.1cm}
\newdimen\digitwidth
\settowidth\digitwidth{9}
\def\divrule#1#2{
\noalign{\moveright#1\digitwidth
\vbox{\hrule width#2\digitwidth}}}






\DeclareMathOperator{\arccot}{arccot}
\DeclareMathOperator{\arcsec}{arcsec}
\DeclareMathOperator{\arccsc}{arccsc}

















%%This is to help with formatting on future title pages.
\newenvironment{sectionOutcomes}{}{}



\author{Lee Wayand}

\begin{document}
\begin{exercise}



We encouter quadratic function more than any other in Calculus.  We need to know them backwards and forwards. 


\textbf{Analyze Q(w)} \\

\[
Q(w) = 3 w^2 -5 w + 1 \, \text { with its natural domain } 
\]






\subsection{Algebraic Language}



\textbf{\textcolor{blue!55!black}{$\blacktriangleright$ Domain: }} We are given that the domain is the natural domain and the natural domain of all polynomials is $(-\infty, \infty)$.


\textbf{\textcolor{blue!55!black}{$\blacktriangleright$ Continuity: }}  All polynomials are continuous on their domain.  So, $Q$ has no discontinuities.  Since, the domain is all real numbers, there can be no singularities.



\textbf{\textcolor{blue!55!black}{$\blacktriangleright$ Zeros: }}  $Q$ doesn't seem to factor easily, so the quadratic formula will give the zeros.


\[
\frac{-(-5) \pm \sqrt{(-5)^2 - 4 (3) (1)}}{2(3)} = \frac{5 \pm \sqrt{13}}{6}
\]


There are two real zeros $\frac{5 - \sqrt{13}}{6}$ and $\frac{5 + \sqrt{13}}{6}$



We can now factor $Q$.

\[
Q(w) = \left(w - \left( \frac{5 - \sqrt{13}}{6} \right) \right) \left(w - \left( \frac{5 + \sqrt{13}}{6} \right) \right)
\]

Both roots have an odd multipicity, which means the sign of $Q$ will change over them.




\textbf{\textcolor{blue!55!black}{$\blacktriangleright$ End-Behavior: }} Polynomials with even degree (like quadratics) have the same end-behavior on either side.  Since $Q$ has a positive leading coefficient, $Q$ is unbounded positively as $w$ tends to $\infty$ or $-\infty$.

\[
\lim\limits_{w \to -\infty} Q(w) = \infty
\]


\[
\lim\limits_{w \to \infty} Q(w) = \infty
\]





\textbf{\textcolor{blue!55!black}{$\blacktriangleright$ Behavior: }}


Since the leading coefficient of $Q$ is positive, we know $Q$ has a global maximum at the critical number $\frac{-b}{2 a}= \frac{5}{6}$.  We also know that $Q$ decreases and then increases.







$Q$ decreases on $\left( -\infty, \frac{5}{6} \right)$


$Q$ increases on $\left( \frac{5}{6}, \infty \right)$





\textbf{\textcolor{blue!55!black}{$\blacktriangleright$ Extrema: }}


Since $Q$ decreases and then increase, $Q$ has a global (and one local) minimum of $Q\left( \frac{5}{6} \right) = - \frac{13}{12}$ at $\frac{5}{6}$.


Since $Q$ is unbounded positively, it has not global maximum.  Since there is only one critical number, $Q$ has no local maximums.



\textbf{\textcolor{blue!55!black}{$\blacktriangleright$ Range: }}


That range of $Q$ is $\left[ -\frac{13}{12}, \infty \right)$.








\subsection{Graphical Language}



\begin{image}
\begin{tikzpicture} 
  \begin{axis}[
            domain=-10:10, ymax=10, xmax=10, ymin=-10, xmin=-10,
            axis lines =center, xlabel=$w$, ylabel=$y$, grid = major,
            ytick={-10,-8,-6,-4,-2,2,4,6,8,10},
            xtick={-10,-8,-6,-4,-2,2,4,6,8,10},
            ticklabel style={font=\scriptsize},
            every axis y label/.style={at=(current axis.above origin),anchor=south},
            every axis x label/.style={at=(current axis.right of origin),anchor=west},
            axis on top
          ]
          
			\addplot [line width=2, penColor, smooth,samples=100,domain=(-6:-2)] {-2*x-3};
       		\addplot [line width=2, penColor, smooth,samples=100,domain=(-2:4)] {-1*(x+3)*(x-3))};
       		\addplot [line width=2, penColor, smooth,samples=100,domain=(4:6)] {1.75*x-8};




			\addplot[color=penColor,fill=penColor,only marks,mark=*] coordinates{(-6,9)};
			\addplot[color=penColor,fill=penColor,only marks,mark=*] coordinates{(-2,1)};

			\addplot[color=penColor,fill=white,only marks,mark=*] coordinates{(-2,5)};
			\addplot[color=penColor,fill=penColor,only marks,mark=*] coordinates{(4,-7)};

			\addplot[color=penColor,fill=white,only marks,mark=*] coordinates{(4,-1)};
			\addplot[color=penColor,fill=white,only marks,mark=*] coordinates{(6,2.5)};


           

  \end{axis}
\end{tikzpicture}
\end{image}





\textbf{\textcolor{blue!55!black}{$\blacktriangleright$ }}  The graph of $y = Q(w)$ is a parabola and has no holes or breaks. \\

\textbf{\textcolor{blue!55!black}{$\blacktriangleright$ }}  The graph has two intercepts:


\[
\left( \frac{5 - \sqrt{13}}{6}, 0  \right) \, \text{ and } \,    \left( \frac{5 + \sqrt{13}}{6}, 0  \right)
\]









\textbf{\textcolor{blue!55!black}{$\blacktriangleright$ }}  The graph slopes down to the right until it hits its lowest point, $\left( \frac {5}{6}, -\frac{12}{13} \right)$, then it slopes up to the right.  The graph has no highest point and tends to infinity as $w$ tends to $-\infty$ or $\infty$.






\end{exercise}
\end{document}