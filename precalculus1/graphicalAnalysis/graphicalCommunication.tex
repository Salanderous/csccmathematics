\documentclass{ximera}


\graphicspath{
  {./}
  {ximeraTutorial/}
  {basicPhilosophy/}
}

\newcommand{\mooculus}{\textsf{\textbf{MOOC}\textnormal{\textsf{ULUS}}}}

\usepackage{tkz-euclide}\usepackage{tikz}
\usepackage{tikz-cd}
\usetikzlibrary{arrows}
\tikzset{>=stealth,commutative diagrams/.cd,
  arrow style=tikz,diagrams={>=stealth}} %% cool arrow head
\tikzset{shorten <>/.style={ shorten >=#1, shorten <=#1 } } %% allows shorter vectors

\usetikzlibrary{backgrounds} %% for boxes around graphs
\usetikzlibrary{shapes,positioning}  %% Clouds and stars
\usetikzlibrary{matrix} %% for matrix
\usepgfplotslibrary{polar} %% for polar plots
\usepgfplotslibrary{fillbetween} %% to shade area between curves in TikZ
\usetkzobj{all}
\usepackage[makeroom]{cancel} %% for strike outs
%\usepackage{mathtools} %% for pretty underbrace % Breaks Ximera
%\usepackage{multicol}
\usepackage{pgffor} %% required for integral for loops



%% http://tex.stackexchange.com/questions/66490/drawing-a-tikz-arc-specifying-the-center
%% Draws beach ball
\tikzset{pics/carc/.style args={#1:#2:#3}{code={\draw[pic actions] (#1:#3) arc(#1:#2:#3);}}}



\usepackage{array}
\setlength{\extrarowheight}{+.1cm}
\newdimen\digitwidth
\settowidth\digitwidth{9}
\def\divrule#1#2{
\noalign{\moveright#1\digitwidth
\vbox{\hrule width#2\digitwidth}}}






\DeclareMathOperator{\arccot}{arccot}
\DeclareMathOperator{\arcsec}{arcsec}
\DeclareMathOperator{\arccsc}{arccsc}

















%%This is to help with formatting on future title pages.
\newenvironment{sectionOutcomes}{}{}


\title{Graph Symbols}

\begin{document}

\begin{abstract}
communication
\end{abstract}
\maketitle



We can easily define a function via a graph.  However, it is not so easy to decipher information from a graph, especially when drawn by hand.  Therefore, in addition to the points on the graph, we include other symbols to help people read the graph.  These are  communication symbols.  They help the reader understand what is drawn and what is not drawn.



\section{Endpoints}


Points are dimensionless objects.  They are too small to actually see.  WHen we draw a graph, the thinkness of the graph is already an exaggeration of the underlying points. But, even with the thickness of the drawing the endpoints are a mystery.  If you leave off the very last point or keep it, how would you tell?  Therefore, we exaggerate even more for the endpoints, so that the reader immediately understands whether or not they are plotted.

We use big solid dots to illustrate an endpoint is a part of the drawing. We use big hollow dots when the actual last point is not plotted. 


In the graph on the left, it is difficult to tell if the endpoints are included or not.  The graph on the right makes this clear.

\begin{image}
\begin{tikzpicture}
    \begin{axis}[name = without,
            	domain=-5:5, ymax=5, xmax=5, ymin=-5, xmin=-5,
            	axis lines =center, xlabel=$x$, ylabel=$y$,
            	every axis y label/.style={at=(current axis.above origin),anchor=south},
            	every axis x label/.style={at=(current axis.right of origin),anchor=west},
            	axis on top,
          		]

        \addplot [draw=penColor, very thick, smooth, domain=(-3:4)] {0.5*x-1};
    \end{axis}





     \begin{axis}[
            	at={(without.outer east)}, anchor=outer west, domain=-5:5, ymax=5, xmax=5, ymin=-5, xmin=-5,
            	axis lines =center, xlabel=$x$, ylabel=$y$,
            	every axis y label/.style={at=(current axis.above origin),anchor=south},
            	every axis x label/.style={at=(current axis.right of origin),anchor=west},
            	axis on top,
          		]

        \addplot [draw=penColor, very thick, smooth, domain=(-3:4)] {0.5*x-1};
        \addplot[color=penColor,fill=penColor,only marks,mark=*] coordinates{(-3, -2.5)};
        \addplot[color=penColor,fill=white,only marks,mark=*] coordinates{(4, 1)};
    \end{axis}



\end{tikzpicture}
\end{image}


The big exaggerated dots are just for the single endpoints.  The graph on the right does not include a circle as part of the graph. It is a single hollow dot indicating thatthe point $(4, 1)$ is not part of the graph.







\section{Arrows}


When we use graphs to define functions, we need to communicate the domain. This is difficult when the domain is large, such as $(-\infty, \infty)$. The paper we draw on is only inches wide.  So, we include several additional graph symbols to help the reader understand what is happening out of view.

Arrows at the ends of graphs signal that the graph continues its current pattern.


In the graph on the left, it is difficult to decide if the domain is $(-3, 4)$ or not.  The graph on the right makes this clear.

\begin{image}
\begin{tikzpicture}
    \begin{axis}[name = without,
            	domain=-5:5, ymax=5, xmax=5, ymin=-5, xmin=-5,
            	axis lines =center, xlabel=$x$, ylabel=$y$,
            	every axis y label/.style={at=(current axis.above origin),anchor=south},
            	every axis x label/.style={at=(current axis.right of origin),anchor=west},
            	axis on top,
          		]

        \addplot [draw=penColor, very thick, smooth, domain=(-3:4)] {0.5*x-1};
    \end{axis}





     \begin{axis}[
            	at={(without.outer east)}, anchor=outer west, domain=-5:5, ymax=5, xmax=5, ymin=-5, xmin=-5,
            	axis lines =center, xlabel=$x$, ylabel=$y$,
            	every axis y label/.style={at=(current axis.above origin),anchor=south},
            	every axis x label/.style={at=(current axis.right of origin),anchor=west},
            	axis on top,
          		]

        \addplot [draw=penColor, very thick, smooth, domain=(-3:4), <->] {0.5*x-1};
    \end{axis}



\end{tikzpicture}
\end{image}


The arrows tell the reader that the graph continues its current pattern of a line indefinitely.  The domain is $(-\infty, \infty)$.











\section{Asymptotes}


Arrows at the ends of graphs signal that the graph continues with its current pattern. However, this may not not be enough.  Many functions have graphs that continue moving up and to the right, however, their movement to the right has a boundary. We use dashed lines to indicate that a graph is approaching this line pattern - getting closer


\begin{image}
\begin{tikzpicture}
     \begin{axis}[
            	domain=-10:10, ymax=10, xmax=10, ymin=-10, xmin=-10,
            	axis lines =center, xlabel=$x$, ylabel=$y$,
            	every axis y label/.style={at=(current axis.above origin),anchor=south},
            	every axis x label/.style={at=(current axis.right of origin),anchor=west},
            	axis on top,
          		]

        
        \addplot [draw=penColor, very thick, smooth, domain=(-6:3), <->] {1/(x-3) + 2};
        \addplot [draw=penColor, very thick, smooth, domain=(3:8), <->] {1/(x-3) + 2};

        \addplot [line width=1, gray, dashed,samples=100,domain=(-9:9)] ({3},{x});
        \addplot [line width=1, gray, dashed,samples=100,domain=(-9:9)] ({x},{2});


    \end{axis}



\end{tikzpicture}
\end{image}


The arrows tell the reader that the graph continues its current pattern, for example the left piece of the graph continues to move up and to the left.  The dashed asymptote tells us that the leftward movement does not go past $3$.  The graph continues to go up indefinitely while getting closer and closer to the asyptte, without hitting the asymptote. 











\section{Ellipsis}


Sometimes an arrow at the end of the  graph may give the wrong impression.  If the pattern involves some kind of shape, then ellipses may be used.



\begin{image}
\begin{tikzpicture}
	\begin{axis}[
            domain=-10:10, ymax=3, xmax=10, ymin=-2, xmin=-10,
            axis lines =center, xlabel=$t$, ylabel=$y$,
            every axis y label/.style={at=(current axis.above origin),anchor=south},
            every axis x label/.style={at=(current axis.right of origin),anchor=west},
            axis on top
          ]
          
	\addplot [draw=penColor,very thick,smooth,domain=(-2:-1)] {x+2};
	\addplot [draw=penColor,very thick,smooth,domain=(-1:0)] {-x};

	\addplot [draw=penColor,very thick,smooth,domain=(0:1)] {x};
	\addplot [draw=penColor,very thick,smooth,domain=(1:2)] {-x+2};

	\addplot [draw=penColor,very thick,smooth,domain=(2:3)] {x-2};
	\addplot [draw=penColor,very thick,smooth,domain=(3:4)] {-x+4};



	\addplot[color=penColor,only marks,mark=*] coordinates{(-3,0.5)}; 
	\addplot[color=penColor,only marks,mark=*] coordinates{(-4,0.5)}; 
	\addplot[color=penColor,only marks,mark=*] coordinates{(-5,0.5)}; 

	\addplot[color=penColor,only marks,mark=*] coordinates{(5,0.5)}; 
	\addplot[color=penColor,only marks,mark=*] coordinates{(6,0.5)}; 
	\addplot[color=penColor,only marks,mark=*] coordinates{(7,0.5)}; 




    \end{axis}
\end{tikzpicture}
\end{image}





A graph is a communication tool. We already know that the accuracy has limitations. There are also other limitations that we must deal with.  To that end, a nice graph displays other symbols besides just the points.  These other graphical features help the reader understand the thoughts of the drawer.




\begin{question} Reading Graphs

\begin{image}
\begin{tikzpicture}
     \begin{axis}[
            	domain=-10:10, ymax=10, xmax=10, ymin=-10, xmin=-10,
            	axis lines =center, xlabel=$x$, ylabel=$y$,
            	every axis y label/.style={at=(current axis.above origin),anchor=south},
            	every axis x label/.style={at=(current axis.right of origin),anchor=west},
            	axis on top,
          		]

        
        \addplot [draw=penColor, very thick, smooth, domain=(-6:3), <->] {1/(x-3) + 2};
        \addplot [draw=penColor, very thick, smooth, domain=(3:8)] {1/(x-3) + 2};

        \addplot [line width=1, gray, dashed,samples=100,domain=(-9:9)] ({3},{x});
        \addplot [line width=1, gray, dashed,samples=100,domain=(-9:9)] ({x},{2});

        \addplot[color=penColor,only marks,mark=*] coordinates{(3.2,7)}; 
        \addplot[color=penColor,only marks,mark=*] coordinates{(8,2.2)}; 


    \end{axis}



\end{tikzpicture}
\end{image}

\begin{itemize}
\item The graph above has points with $y$-coordinates greater than $5,000$.
\begin{multipleChoice}
\choice {True}
\choice [correct]{False}
\end{multipleChoice}


\item The graph above has points with $y$-coordinates less than $-5,000$.
\begin{multipleChoice}
\choice [correct]{True}
\choice {False}
\end{multipleChoice}



\item The graph above has points with $x$-coordinates greater than $5,000$.
\begin{multipleChoice}
\choice {True}
\choice [correct]{False}
\end{multipleChoice}



\item The graph above has points with $x$-coordinates less than $-5,000$.
\begin{multipleChoice}
\choice [correct]{True}
\choice {False}
\end{multipleChoice}
\end{itemize}

\end{question}










\end{document}
