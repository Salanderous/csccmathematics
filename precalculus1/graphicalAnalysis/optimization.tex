\documentclass{ximera}


\graphicspath{
  {./}
  {ximeraTutorial/}
  {basicPhilosophy/}
}

\newcommand{\mooculus}{\textsf{\textbf{MOOC}\textnormal{\textsf{ULUS}}}}

\usepackage{tkz-euclide}\usepackage{tikz}
\usepackage{tikz-cd}
\usetikzlibrary{arrows}
\tikzset{>=stealth,commutative diagrams/.cd,
  arrow style=tikz,diagrams={>=stealth}} %% cool arrow head
\tikzset{shorten <>/.style={ shorten >=#1, shorten <=#1 } } %% allows shorter vectors

\usetikzlibrary{backgrounds} %% for boxes around graphs
\usetikzlibrary{shapes,positioning}  %% Clouds and stars
\usetikzlibrary{matrix} %% for matrix
\usepgfplotslibrary{polar} %% for polar plots
\usepgfplotslibrary{fillbetween} %% to shade area between curves in TikZ
\usetkzobj{all}
\usepackage[makeroom]{cancel} %% for strike outs
%\usepackage{mathtools} %% for pretty underbrace % Breaks Ximera
%\usepackage{multicol}
\usepackage{pgffor} %% required for integral for loops



%% http://tex.stackexchange.com/questions/66490/drawing-a-tikz-arc-specifying-the-center
%% Draws beach ball
\tikzset{pics/carc/.style args={#1:#2:#3}{code={\draw[pic actions] (#1:#3) arc(#1:#2:#3);}}}



\usepackage{array}
\setlength{\extrarowheight}{+.1cm}
\newdimen\digitwidth
\settowidth\digitwidth{9}
\def\divrule#1#2{
\noalign{\moveright#1\digitwidth
\vbox{\hrule width#2\digitwidth}}}






\DeclareMathOperator{\arccot}{arccot}
\DeclareMathOperator{\arcsec}{arcsec}
\DeclareMathOperator{\arccsc}{arccsc}

















%%This is to help with formatting on future title pages.
\newenvironment{sectionOutcomes}{}{}


\title{Optimization}

\begin{document}

\begin{abstract}
minimum and maximum
\end{abstract}
\maketitle



When using functions to analyze situations, we are often interested in the maximum and minimum values that a function takes on. These are often called \textbf{textreme} values. We would like to know the extreme values and where, in the domain, they occur.



There are two views on extreme values.

\textbf{Global Extreme Values} 

\begin{itemize}
\item The \textbf{global maximum value} is the greatest value of the function.  This maximum may occur at mulitple domain numbers.  A function can also not have a maximum value.

\[  f(c) \text{ is a global maximum if } f(x) \leq f(c) \text{ for all } x \text{ in the domain of } f \]

\item The \textbf{global minimum value} is the least value of the function.  This mnimum may occur at mulitple domain numbers.  A function can also not have a minimum value.

\[  f(c) \text{ is a global minimum if } f(c) \leq f(x) \text{ for all } x \text{ in the domain of } f \]
\end{itemize}

A function has at most, one maximum value and one minimum value.

In contrast to a sinle maximum or minimum value, a function also values which are the greatest value in their own neighborhood.  We see these as tops of hills and bottom of valleys on the graph.  A function may have many of these \textbf{local} or \textbf{relative} extrema.







\textbf{Local Extreme Values} \\
\begin{itemize}
\item $f(c)$ is a \textbf{local maximum value} of the function $f$ if there exists an $0 < \epsilon$ such that $f(x) \leq f(c)$ for all $x$ within a distance of $\epsilon$ of $c$.

\item $f(c)$ is a \textbf{local minimu value} of the function $f$ if there exists an $0 < \epsilon$ such that $f(c) \leq f(x)$ for all $x$ within a distance of $\epsilon$ of $c$.
\end{itemize}














\begin{example} Extrema \\

Let $g(k)$ be a function.  The graph of $y= g(k)$ is displayed below. 

\begin{image}
\begin{tikzpicture}
     \begin{axis}[
            	domain=-10:10, ymax=10, xmax=10, ymin=-10, xmin=-10,
            	axis lines =center, xlabel=$k$, ylabel=$y$,
            	every axis y label/.style={at=(current axis.above origin),anchor=south},
            	every axis x label/.style={at=(current axis.right of origin),anchor=west},
            	axis on top,
          		]

        
        \addplot [draw=penColor, very thick, smooth, domain=(-8:5), ->] {0.05*(x+7)*(x+2)*(x-3)};
        %\addplot [draw=penColor, very thick, smooth, domain=(-1:3)] {0.5*x-1};
        %\addplot [draw=penColor, very thick, smooth, domain=(4:8)] {-2*x+10};


        \addplot[color=penColor,fill=penColor,only marks,mark=*] coordinates{(-8,-3.3)};
        %\addplot[color=penColor,fill=white,only marks,mark=*] coordinates{(-3,3)};


    \end{axis}
\end{tikzpicture}
\end{image}

\begin{itemize}
\item $g$ has no global maximum.
\item The global minimum of $g$ is $-3.3$, which occurs at $-8$.
\item $g$ has a local minimum of $-3.3$, which occurs at $-8$ and a local minimum of $-2.5$, which occurs at $0.9$.
\item $g$ has a local maximum of $2.4$, which occurs at $-4.9$
\end{itemize}

\end{example}




\textit{Local Explanations} \\

Let $0 < \epsilon = 0.5$.  $g(-8) \leq g(k)$ for all $k$ in the domain within a distance of $0.5$ of $-8$, which would be the interval $[-8, -7.5)$.

Let $0 < \epsilon = 0.3$.  $g(0.9) \leq g(k)$ for all $k$ in the domain within a distance of $0.53$ of $0.9$, which would be the interval $(0.6, 1.2)$.

Let $0 < \epsilon = 0.4$.  $g(k) \leq g(-4.9)$ for all $k$ in the domain within a distance of $0.4$ of $-4.9$, which would be the interval $(-5.3, -4.6)$.




\textbf{Note:} As the previous example illustrates, global extrema are also local extrema.



\begin{example} Extrema \\

Let $B(w)$ be a function.  The graph of $y = B(w)$ is displayed below. 

\begin{image}
\begin{tikzpicture}
     \begin{axis}[
            	domain=-10:10, ymax=10, xmax=10, ymin=-10, xmin=-10,
            	axis lines =center, xlabel=$w$, ylabel=$y$,
            	every axis y label/.style={at=(current axis.above origin),anchor=south},
            	every axis x label/.style={at=(current axis.right of origin),anchor=west},
            	axis on top,
          		]

        
        \addplot [draw=penColor, very thick, smooth, domain=(-8:4), <->] {0.01*(x+4)*(x+7)*(x+2)*(x-3)};
        %\addplot [draw=penColor, very thick, smooth, domain=(-1:3)] {0.5*x-1};
        %\addplot [draw=penColor, very thick, smooth, domain=(4:8)] {-2*x+10};


        %\addplot[color=penColor,fill=penColor,only marks,mark=*] coordinates{(-8,4.3)};
        %\addplot[color=penColor,fill=white,only marks,mark=*] coordinates{(-3,3)};


    \end{axis}
\end{tikzpicture}
\end{image}

\begin{itemize}
\item $B$ has no global maximum.
\item The global minimum of $B$ is $-2.5$, which occurs at $1.4$.
\item $B$ has a local minimum of $-2.5$, which occurs at $1.4$ and a local minimum of $-0.7$, which occurs at $-5.9$.
\item $B$ has a local maximum of $0.25$, which occurs at $-3$
\end{itemize}

\end{example}


\textit{Local Explanations} \\

Let $0 < \epsilon = 0.6$.  $B(1.4) \leq B(w)$ for all $w$ in the domain within a distance of $0.6$ of $1.4$, which would be the interval $(0.8, 2)$.

Let $0 < \epsilon = 0.25$.  $B(-5.9) \leq B(w)$ for all $w$ in the domain within a distance of $0.25$ of $-5.9$, which would be the interval $(0.6, 1.2)$.

Let $0 < \epsilon = 0.5$.  $B(w) \leq B(-3)$ for all $w$ in the domain within a distance of $0.5$ of $-3$, which would be the interval $(-3.5, -2.5)$.








\begin{example} Extrema \\

Let $T(p)$ be a function.  The graph of $y=  T(p)$ is displayed below. 

\begin{image}
\begin{tikzpicture}
     \begin{axis}[
            	domain=-10:10, ymax=10, xmax=10, ymin=-10, xmin=-10,
            	axis lines =center, xlabel=$p$, ylabel=$y$,
            	every axis y label/.style={at=(current axis.above origin),anchor=south},
            	every axis x label/.style={at=(current axis.right of origin),anchor=west},
            	axis on top,
          		]

        
        \addplot [draw=penColor, very thick, smooth, domain=(-8:-1)] {-x};
        \addplot [draw=penColor, very thick, smooth, domain=(-1:3)] {0.5*x-1};
        \addplot [draw=penColor, very thick, smooth, domain=(3:8), ->] {-2*x+10};


        \addplot[color=penColor,fill=white,only marks,mark=*] coordinates{(-8,8)};
        \addplot[color=penColor,fill=penColor,only marks,mark=*] coordinates{(-8,-2)};

        \addplot[color=penColor,fill=white,only marks,mark=*] coordinates{(-1,1)};
        \addplot[color=penColor,fill=white,only marks,mark=*] coordinates{(-1,-1.5)};
        \addplot[color=penColor,fill=penColor,only marks,mark=*] coordinates{(-1,6)};

        \addplot[color=penColor,fill=white,only marks,mark=*] coordinates{(3,0.5)};
        \addplot[color=penColor,fill=white,only marks,mark=*] coordinates{(3,4)};
        \addplot[color=penColor,fill=penColor,only marks,mark=*] coordinates{(3,-6)};

    \end{axis}
\end{tikzpicture}
\end{image}

\begin{itemize}
\item $T$ has no global maximum.
\item $T$ has no global minimum.
\item $T$ has a local minimum of $-2$, which occurs at $-8$.
\item $T$ has a local minimum of $-6$, which occurs at $3$.
\item $T$ has a local maximum of $6$, which occurs at $-1$.
\end{itemize}

\end{example}





It appears that $8$ would have been the global maximum of $T$, however, the point $(-8, 8)$ is missing from the graph and $8$ is not in the range.  And, there is no real number "just below $8$".  If you choose any real number, $r$, below $8$ to be the possible candidate for global maximum, there is the number $\frac{8+r}{2}$. This number is in the domain and $T(r) < T\left(\frac{8+r}{2}\right)$.




\textit{Local Explanations} \\

Let $0 < \epsilon = 0.5$.  $B(-8) \leq B(w)$ for all $w$ in the domain within a distance of $0.5$ of $-8$, which would be the interval $[-8, -7.5)$.

Let $0 < \epsilon = 0.4$.  $B(3) \leq B(w)$ for all $w$ in the domain within a distance of $0.4$ of $3$, which would be the interval $(3.4, 2.6)$.

Let $0 < \epsilon = 0.3$.  $B(w) \leq B(-1)$ for all $w$ in the domain within a distance of $0.3$ of $-1$, which would be the interval $(-1.3, -0.7)$.
















\end{document}
