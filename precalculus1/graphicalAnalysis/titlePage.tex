\documentclass{ximera}


\graphicspath{
  {./}
  {ximeraTutorial/}
  {basicPhilosophy/}
}

\newcommand{\mooculus}{\textsf{\textbf{MOOC}\textnormal{\textsf{ULUS}}}}

\usepackage{tkz-euclide}\usepackage{tikz}
\usepackage{tikz-cd}
\usetikzlibrary{arrows}
\tikzset{>=stealth,commutative diagrams/.cd,
  arrow style=tikz,diagrams={>=stealth}} %% cool arrow head
\tikzset{shorten <>/.style={ shorten >=#1, shorten <=#1 } } %% allows shorter vectors

\usetikzlibrary{backgrounds} %% for boxes around graphs
\usetikzlibrary{shapes,positioning}  %% Clouds and stars
\usetikzlibrary{matrix} %% for matrix
\usepgfplotslibrary{polar} %% for polar plots
\usepgfplotslibrary{fillbetween} %% to shade area between curves in TikZ
\usetkzobj{all}
\usepackage[makeroom]{cancel} %% for strike outs
%\usepackage{mathtools} %% for pretty underbrace % Breaks Ximera
%\usepackage{multicol}
\usepackage{pgffor} %% required for integral for loops



%% http://tex.stackexchange.com/questions/66490/drawing-a-tikz-arc-specifying-the-center
%% Draws beach ball
\tikzset{pics/carc/.style args={#1:#2:#3}{code={\draw[pic actions] (#1:#3) arc(#1:#2:#3);}}}



\usepackage{array}
\setlength{\extrarowheight}{+.1cm}
\newdimen\digitwidth
\settowidth\digitwidth{9}
\def\divrule#1#2{
\noalign{\moveright#1\digitwidth
\vbox{\hrule width#2\digitwidth}}}






\DeclareMathOperator{\arccot}{arccot}
\DeclareMathOperator{\arcsec}{arcsec}
\DeclareMathOperator{\arccsc}{arccsc}

















%%This is to help with formatting on future title pages.
\newenvironment{sectionOutcomes}{}{}


\title{Graphical Analysis}

\begin{document}

\begin{abstract}
%Stuff can go here later if we want!
\end{abstract}
\maketitle




Functions are packages containing three sets, one of which is a set of pairs.  That is the struture of a function.  However, the sets are often measurements and the function is how we compare those measurements.

We would like to know things like:

\begin{itemize}
\item When one measurement increases, does the other also increase?
\item When one measurement increases, does the other decrease?
\item When one measurement changes quickly, does the other change quickly or slowly?
\item Which measurements are connected to maximums and minimums in the other measurements?
\item If one measurement is smooth is the other smooth or does it have sudden changes?
\end{itemize}


In this section, we will introduce this type of function analysis via graphs. Once we gain a familiarity with this type of analysis, we will extend our thinking to include formulas.  Our eventual goal is to be exact with our analysis.  This will require both our graphical and algebraic tools working together.









\subsection{Expectations}

\begin{sectionOutcomes}
In this section, students will...

\begin{itemize}
\item identify intervals where functions are increasing and decreasing.
\item locate maximums and minimums.
\item locate zeros.
\end{itemize}
\end{sectionOutcomes}

\end{document}
