\documentclass{ximera}


\graphicspath{
  {./}
  {ximeraTutorial/}
  {basicPhilosophy/}
}

\newcommand{\mooculus}{\textsf{\textbf{MOOC}\textnormal{\textsf{ULUS}}}}

\usepackage{tkz-euclide}\usepackage{tikz}
\usepackage{tikz-cd}
\usetikzlibrary{arrows}
\tikzset{>=stealth,commutative diagrams/.cd,
  arrow style=tikz,diagrams={>=stealth}} %% cool arrow head
\tikzset{shorten <>/.style={ shorten >=#1, shorten <=#1 } } %% allows shorter vectors

\usetikzlibrary{backgrounds} %% for boxes around graphs
\usetikzlibrary{shapes,positioning}  %% Clouds and stars
\usetikzlibrary{matrix} %% for matrix
\usepgfplotslibrary{polar} %% for polar plots
\usepgfplotslibrary{fillbetween} %% to shade area between curves in TikZ
\usetkzobj{all}
\usepackage[makeroom]{cancel} %% for strike outs
%\usepackage{mathtools} %% for pretty underbrace % Breaks Ximera
%\usepackage{multicol}
\usepackage{pgffor} %% required for integral for loops



%% http://tex.stackexchange.com/questions/66490/drawing-a-tikz-arc-specifying-the-center
%% Draws beach ball
\tikzset{pics/carc/.style args={#1:#2:#3}{code={\draw[pic actions] (#1:#3) arc(#1:#2:#3);}}}



\usepackage{array}
\setlength{\extrarowheight}{+.1cm}
\newdimen\digitwidth
\settowidth\digitwidth{9}
\def\divrule#1#2{
\noalign{\moveright#1\digitwidth
\vbox{\hrule width#2\digitwidth}}}






\DeclareMathOperator{\arccot}{arccot}
\DeclareMathOperator{\arcsec}{arcsec}
\DeclareMathOperator{\arccsc}{arccsc}

















%%This is to help with formatting on future title pages.
\newenvironment{sectionOutcomes}{}{}


\title{Change}

\begin{document}

\begin{abstract}
increasing and decreasing
\end{abstract}
\maketitle



We use functions to compare information. One aspect we are interest in is how the information changes compare to each other, relatively. 


\begin{itemize}
\item When the supply of pineapples goes up does the price go down?
\item When medicine dosage goes up does the pain go down?
\item When the speed limit goes down does the number of accidents go down?
\end{itemize}


Our words for this comparison are \textbf{increase} and \textbf{decrease}.


\begin{itemize}
\item When the supply of pineapples increases does the price decrease?
\item When medicine dosage increases does the pain decrease?
\item When the speed limit decreases does the number of accidents decrease?
\end{itemize}




\begin{definition} \textit{Increase}

The function $f$ increases on the set $S$, if for EVERY pair of numbers $a < b$ in $S$ we have  $f(a) < f(b)$.


\end{definition}



\begin{definition} \textit{Decrease}

The function $f$ decreases on the set $S$, if for EVERY pair of numbers $a < b$ in $S$ we have  $f(a) > f(b)$.


\end{definition}





Graphically, increasing appears as dots to the right being higher than dots to the left.  Decreasing appears as dots on the curve being lower to the right. 











\begin{example} Increasing and Decreasing \\

Let $g(k)$ be a function.  The graph of $y= g(k)$ is displayed below. 

\begin{image}
\begin{tikzpicture}
     \begin{axis}[
            	domain=-10:10, ymax=10, xmax=10, ymin=-10, xmin=-10,
            	axis lines =center, xlabel=$k$, ylabel=$y$,
            	every axis y label/.style={at=(current axis.above origin),anchor=south},
            	every axis x label/.style={at=(current axis.right of origin),anchor=west},
            	axis on top,
          		]

        
        \addplot [draw=penColor, very thick, smooth, domain=(-8:-3)] {-x};
        \addplot [draw=penColor, very thick, smooth, domain=(-1:3)] {0.5*x-1};
        \addplot [draw=penColor, very thick, smooth, domain=(4:8)] {-2*x+10};


        \addplot[color=penColor,fill=white,only marks,mark=*] coordinates{(-8,8)};
        \addplot[color=penColor,fill=white,only marks,mark=*] coordinates{(-3,3)};

        \addplot[color=penColor,fill=penColor,only marks,mark=*] coordinates{(-1,-1.5)};
        \addplot[color=penColor,fill=white,only marks,mark=*] coordinates{(3,0.5)};

        \addplot[color=penColor,fill=penColor,only marks,mark=*] coordinates{(4,2)};
        \addplot[color=penColor,fill=white,only marks,mark=*] coordinates{(8,-6)};

    \end{axis}
\end{tikzpicture}
\end{image}

\begin{itemize}
\item $g$ is \wordChoice{\choice{increasing} \choice[correct]{decreasing} \choice{neither}} on the interval $(-8,-3)$.  
\item $g$ is \wordChoice{\choice[correct]{increasing} \choice{decreasing} \choice{neither}} on the interval $[-1,3)$. 
\item $g$ is \wordChoice{\choice{increasing} \choice[correct]{decreasing} \choice{neither}} on the interval $[4, 8)$. 
\end{itemize}
\end{example}


\begin{question}
In the example above, $g$ is decreasing on $(-8,-3)$.  Is $g$ decreasing on any and every subinterval of $(-8,-3)$. 
\begin{multipleChoice}
\choice [correct]{Yes}
\choice {No}
\end{multipleChoice}
\end{question}








\begin{example} Decreasing \\

Let $m(t)$ be a function.  The graph of $y = m(t)$ is displayed below. 

\begin{image}
\begin{tikzpicture}
     \begin{axis}[
            	domain=-10:10, ymax=10, xmax=10, ymin=-10, xmin=-10,
            	axis lines =center, xlabel=$t$, ylabel=$y$,
            	every axis y label/.style={at=(current axis.above origin),anchor=south},
            	every axis x label/.style={at=(current axis.right of origin),anchor=west},
            	axis on top,
          		]

        
        \addplot [draw=penColor, very thick, smooth, domain=(-8:-3)] {-x};
        %\addplot [draw=penColor, very thick, smooth, domain=(-1:3)] {0.5*x-1};
        \addplot [draw=penColor, very thick, smooth, domain=(-3:8)] {-x+3};


        \addplot[color=penColor,fill=white,only marks,mark=*] coordinates{(-8,8)};
        \addplot[color=penColor,fill=white,only marks,mark=*] coordinates{(-3,3)};

        %\addplot[color=penColor,fill=penColor,only marks,mark=*] coordinates{(-1,-1.5)};
        %\addplot[color=penColor,fill=white,only marks,mark=*] coordinates{(3,0.5)};

        \addplot[color=penColor,fill=penColor,only marks,mark=*] coordinates{(-3,6)};
        \addplot[color=penColor,fill=white,only marks,mark=*] coordinates{(8,-5)};

    \end{axis}
\end{tikzpicture}
\end{image}


\begin{itemize}
\item $m$ is \wordChoice{\choice{increasing} \choice[correct]{decreasing} \choice{neither}} on the interval $(-8,-3)$.  
\item $m$ is \wordChoice{\choice{increasing} \choice[correct]{decreasing} \choice{neither}} on the interval $[-3, 8)$. 
\end{itemize}



However, $m$ is not decreasing on the interval $(-8, 8)$.  To show that a function is not decreasing on an interval, we merely have to give ONE counterexample. That is we need to come up with ONE pair of numbers $a < b$ in the interval where $f(a) \nless f(b)$.  $a = -6$ and $b = 5$ will work.

\[ -6 < 5    \text{ but, }   f(-6) = 6  \nless f(5) = -2 \]




\end{example}


A \textbf{counterexample} is one example where all of the conditions of a statement are met, yet the conclusion is not true.  One single counterexample shows a statement to be false.












\begin{example} Decreasing \\

Let $N(z)$ be a function.  The graph of $y = N(z)$ is displayed below. 

\begin{image}
\begin{tikzpicture}
     \begin{axis}[
            	domain=-10:10, ymax=10, xmax=10, ymin=-10, xmin=-10,
            	axis lines =center, xlabel=$z$, ylabel=$y$,
                xtick={-10,-8,-6,-4,-2,2,4,6,8,10},
                xticklabels={$-10$,$-8$,$-6$,$-4$,$-2$,$2$,$4$,$6$,$8$,$10$},
                ticklabel style={font=\scriptsize},
            	every axis y label/.style={at=(current axis.above origin),anchor=south},
            	every axis x label/.style={at=(current axis.right of origin),anchor=west},
            	axis on top,
          		]

        
        \addplot [draw=penColor, very thick, smooth, domain=(-8:-3), <-] {(x+7)*(x+2)};
        \addplot [draw=penColor, very thick, smooth, domain=(-3:2)] {-x};
        \addplot [draw=penColor, very thick, smooth, domain=(2:8)] {-0.25*x-4.5};


        \addplot[color=penColor,fill=white,only marks,mark=*] coordinates{(-3,-4)};
        \addplot[color=penColor,fill=white,only marks,mark=*] coordinates{(-3,3)};
        \addplot[color=penColor,fill=penColor,only marks,mark=*] coordinates{(-3,5)};
       

        \addplot[color=penColor,fill=penColor,only marks,mark=*] coordinates{(2,-5)};
        \addplot[color=penColor,fill=white,only marks,mark=*] coordinates{(2,-2)};
        \addplot[color=penColor,fill=penColor,only marks,mark=*] coordinates{(8,-8)};
        \addplot[color=penColor,fill=white,only marks,mark=*] coordinates{(8,-6.5)};

    \end{axis}
\end{tikzpicture}
\end{image}







\begin{question}
Select all of the correct statements
\begin{selectAll}
    \choice [correct]{$N$ is decreasing on the interval $(-3,2)$}
    \choice [correct]{$N$ is decreasing on the interval $(2,8)$}
    \choice [correct]{$N$ is decreasing on the interval $(-3,8)$}
    \choice [correct]{$N$ is decreasing on the interval $[-3,8)$}
    \choice [correct]{$N$ is decreasing on the interval $(-3,8]$}
    \choice [correct]{$N$ is decreasing on the interval $[-3,8]$}
\end{selectAll}



\end{question}




\end{example}
































\section{Measuring Rates}
Comparing how the connected values in the domain and range change is called a \textbf{rate}. Our symbol for a change in an amount is a Greek Delta, $\Delta$.  And, we use fractions to compare the changes.

\[ rate = \frac{\Delta \, pineapples}{\Delta \, price}\]




\begin{itemize}
\item If the changes in pineapples and price are both positive, then this rate is positive.
\item If the changes in pineapples and price are both negative, then this rate is again positive.
\item If the changes in pineapples and price are of different sign, one increases while the other decreases, then this rate is negative.
\end{itemize}


Using this, we can compare and measure changes in function values across a domain interval.


\begin{definition}
The \textbf{rate-of-change} of a function $f$ across $[a, b]$ in the domain is given by

\[  \frac{f(b) - f(a)}{b - a}  \]


\end{definition}

The rate-of-change measures how the function values change relative to changes in the domain.









\begin{example} Rates \\

Let $V(x)$ be a function.  The graph of $y = V(x)$ is displayed below. 

\begin{image}
\begin{tikzpicture}
     \begin{axis}[
            	domain=-10:10, ymax=10, xmax=10, ymin=-10, xmin=-10,
            	axis lines =center, xlabel=$x$, ylabel=$y$,
            	every axis y label/.style={at=(current axis.above origin),anchor=south},
            	every axis x label/.style={at=(current axis.right of origin),anchor=west},
            	axis on top,
          		]

        
        \addplot [draw=penColor, very thick, smooth, domain=(-8:-3), <-] {(x+7)*(x+2)};
        \addplot [draw=penColor, very thick, smooth, domain=(-3:2)] {-x};
        \addplot [draw=penColor, very thick, smooth, domain=(2:8)] {-0.25*x-4.5};


        \addplot[color=penColor,fill=white,only marks,mark=*] coordinates{(-3,-4)};
        \addplot[color=penColor,fill=white,only marks,mark=*] coordinates{(-3,3)};
        \addplot[color=penColor,fill=penColor,only marks,mark=*] coordinates{(-3,5)};
       

        \addplot[color=penColor,fill=penColor,only marks,mark=*] coordinates{(2,-5)};
        \addplot[color=penColor,fill=white,only marks,mark=*] coordinates{(2,-2)};
        \addplot[color=penColor,fill=penColor,only marks,mark=*] coordinates{(8,-8)};
        \addplot[color=penColor,fill=white,only marks,mark=*] coordinates{(8,-6.5)};

    \end{axis}
\end{tikzpicture}
\end{image}



From the graph, we can estimate that over the interval $[-7, -5]$, $V$ had a rate-of-change of $\frac{-6 - 0}{-5 - (-7)} = \frac{-6}{2} = -3$.


From the graph, we can estimate that over the interval $[1, 8]$, $V$ had a rate-of-change of $\frac{-8 - (-1)}{8 - 1} = \frac{-7}{7} = -1$.



\end{example}
























This might sound familiar. You have probably seen this when working with slopes of lines, $\frac{rise}{run}$.  The \textit{rise} is the vertical change (change in the value of the function). The \textit{run} is the horizontal change (change in the domain).

Rate-of-change is a function idea that we will connect up to the geometric idea of slope.





The rate of change over an interval is an overall measurement.  It compares the changes occuring over the whole interval. 











\begin{example} 



Let $M(t)$ be a function.  The graph of $y = M(t)$ is displayed below. 

\begin{image}
\begin{tikzpicture}
     \begin{axis}[
                domain=-10:10, ymax=10, xmax=10, ymin=-10, xmin=-10,
                axis lines =center, xlabel=$t$, ylabel=$y$,
                every axis y label/.style={at=(current axis.above origin),anchor=south},
                every axis x label/.style={at=(current axis.right of origin),anchor=west},
                axis on top,
                ]

        
        \addplot [draw=penColor, very thick, smooth, domain=(-8:-3), <-] {(x+8)*(x+2)+2};
        \addplot [draw=penColor, very thick, smooth, domain=(-3:2)] {-x};
        \addplot [draw=penColor, very thick, smooth, domain=(2:8)] {0.25*x + 1};


        \addplot[color=penColor,fill=white,only marks,mark=*] coordinates{(-3,-3)};
        \addplot[color=penColor,fill=white,only marks,mark=*] coordinates{(-3,3)};
        \addplot[color=penColor,fill=penColor,only marks,mark=*] coordinates{(-3,5)};
       

        \addplot[color=penColor,fill=penColor,only marks,mark=*] coordinates{(2,1.5)};
        \addplot[color=penColor,fill=white,only marks,mark=*] coordinates{(2,-2)};
        \addplot[color=penColor,fill=penColor,only marks,mark=*] coordinates{(8,-1)};
        \addplot[color=penColor,fill=white,only marks,mark=*] coordinates{(8,3)};

    \end{axis}
\end{tikzpicture}
\end{image}


\begin{question}
The rate of change of $M$ over the interval $[-5,5]$ is
    \begin{multipleChoice}
        \choice [correct]{positive}
        \choice {negative}
    \end{multipleChoice}
\end{question}



\begin{question}
The rate of change of $M$ over the interval $[-2,1]$ is
    \begin{multipleChoice}
        \choice {positive}
        \choice [correct]{negative}
    \end{multipleChoice}
\end{question}


\end{example}




\begin{question}
If the rate of change of a function is positive over the interval $[a,b]$ then the rate of change over any subinterval is aslo positive.
    \begin{multipleChoice}
        \choice {True}
        \choice [correct]{False}
    \end{multipleChoice}
\end{question}










\end{document}
