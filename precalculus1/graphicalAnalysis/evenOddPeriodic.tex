\documentclass{ximera}


\graphicspath{
  {./}
  {ximeraTutorial/}
  {basicPhilosophy/}
}

\newcommand{\mooculus}{\textsf{\textbf{MOOC}\textnormal{\textsf{ULUS}}}}

\usepackage{tkz-euclide}\usepackage{tikz}
\usepackage{tikz-cd}
\usetikzlibrary{arrows}
\tikzset{>=stealth,commutative diagrams/.cd,
  arrow style=tikz,diagrams={>=stealth}} %% cool arrow head
\tikzset{shorten <>/.style={ shorten >=#1, shorten <=#1 } } %% allows shorter vectors

\usetikzlibrary{backgrounds} %% for boxes around graphs
\usetikzlibrary{shapes,positioning}  %% Clouds and stars
\usetikzlibrary{matrix} %% for matrix
\usepgfplotslibrary{polar} %% for polar plots
\usepgfplotslibrary{fillbetween} %% to shade area between curves in TikZ
\usetkzobj{all}
\usepackage[makeroom]{cancel} %% for strike outs
%\usepackage{mathtools} %% for pretty underbrace % Breaks Ximera
%\usepackage{multicol}
\usepackage{pgffor} %% required for integral for loops



%% http://tex.stackexchange.com/questions/66490/drawing-a-tikz-arc-specifying-the-center
%% Draws beach ball
\tikzset{pics/carc/.style args={#1:#2:#3}{code={\draw[pic actions] (#1:#3) arc(#1:#2:#3);}}}



\usepackage{array}
\setlength{\extrarowheight}{+.1cm}
\newdimen\digitwidth
\settowidth\digitwidth{9}
\def\divrule#1#2{
\noalign{\moveright#1\digitwidth
\vbox{\hrule width#2\digitwidth}}}






\DeclareMathOperator{\arccot}{arccot}
\DeclareMathOperator{\arcsec}{arcsec}
\DeclareMathOperator{\arccsc}{arccsc}

















%%This is to help with formatting on future title pages.
\newenvironment{sectionOutcomes}{}{}


\title{Symmetry}

\begin{document}

\begin{abstract}
even and odd
\end{abstract}
\maketitle



One of the benefits of graphs is that you can see a lot of information simultaneously.  The visual patterns often suggest properties of the functions.  Many times you can see how to move the graph so that the graph lands back on itself.  Any time you can do something and it \textit{appears} that nothing has been done, that is called \textbf{symmetry}.





\section{Even Symmetry}


\begin{definition} \textit{Even Functions}

A function is an \textbf{even} function if $f(x) = f(-x)$ for all $x$ in the domain.

\end{definition}


If you flip the graph of an even function along the vertical axis, the graph lands on itself - the graph is a mirror image across the vertical axis. \\

You flip and it \textit{appears} that nothing has happened - symmetry. \\






\begin{example} \textbf{\textcolor{green!50!black}{Even Functions}} \\

Let $g(k)$ be a function.  The graph of $y= g(k)$ is displayed below. 

\begin{image}
\begin{tikzpicture}
     \begin{axis}[
            	domain=-10:10, ymax=10, xmax=10, ymin=-10, xmin=-10,
            	axis lines =center, xlabel=$k$, ylabel=$y$,
                ytick={-10,-8,-6,-4,-2,2,4,6,8,10},
                xtick={-10,-8,-6,-4,-2,2,4,6,8,10},
                ticklabel style={font=\scriptsize},
            	every axis y label/.style={at=(current axis.above origin),anchor=south},
            	every axis x label/.style={at=(current axis.right of origin),anchor=west},
            	axis on top,
          		]

        
        \addplot [draw=penColor, very thick, smooth, domain=(-6:6), <->] {0.5*(x+3)*(x-3)-5)};
        %\addplot [draw=penColor, very thick, smooth, domain=(-1:3)] {0.5*x-1};
        %\addplot [draw=penColor, very thick, smooth, domain=(4:8)] {-2*x+10};


        %\addplot[color=penColor,fill=penColor,only marks,mark=*] coordinates{(-8,-3.3)};
        %\addplot[color=penColor,fill=white,only marks,mark=*] coordinates{(-3,3)};


    \end{axis}
\end{tikzpicture}
\end{image}


\end{example}






\begin{example} \textit{Cosine} \\

Cosine is an even function.  The graph of $y= cos(\theta)$ is displayed below. 

\begin{image}
\begin{tikzpicture}
     \begin{axis}[
            	domain=-10:10, ymax=3, xmax=10, ymin=-3, xmin=-10,
            	axis lines =center, xlabel={$\theta$}, ylabel=$y$,
                ytick={-2,-1,1,2},
                xtick={-10,-8,-6,-4,-2,2,4,6,8,10},
                ticklabel style={font=\scriptsize},
            	every axis y label/.style={at=(current axis.above origin),anchor=south},
            	every axis x label/.style={at=(current axis.right of origin),anchor=west},
            	axis on top,
          		]

        
        %\addplot [draw=penColor, very thick, smooth, domain=(-3.7, 3.7), <->] {x};
        \addplot [draw=penColor, very thick, smooth, domain=(-8.3:8.3), <->] {cos(deg(x)};
        %\addplot [draw=penColor, very thick, smooth, domain=(4:8)] {-2*x+10};


        %\addplot[color=penColor,fill=penColor,only marks,mark=*] coordinates{(-8,-3.3)};
        %\addplot[color=penColor,fill=white,only marks,mark=*] coordinates{(-3,3)};


    \end{axis}
\end{tikzpicture}
\end{image}


\end{example}











\section{Odd Symmetry}


\begin{definition} \textit{Odd Functions}

A function is an \textbf{odd} function if $-f(x) = f(-x)$ for all $x$ in the domain.

\end{definition}


This means the graph when the graph rotated $180^\circ$ it lands back on itself.






\begin{example} \textit{Polynomials} \\
 
Polynomials with only odd exponents are odd functions.  The graph of $y= 0.1x^3-1.6x$ is displayed below. 

\begin{image}
\begin{tikzpicture}
     \begin{axis}[
            	domain=-10:10, ymax=10, xmax=10, ymin=-10, xmin=-10,
            	axis lines =center, xlabel=$x$, ylabel=$y$,
                ytick={-10,-8,-6,-4,-2,2,4,6,8,10},
                xtick={-10,-8,-6,-4,-2,2,4,6,8,10},
                ticklabel style={font=\scriptsize},
            	every axis y label/.style={at=(current axis.above origin),anchor=south},
            	every axis x label/.style={at=(current axis.right of origin),anchor=west},
            	axis on top,
          		]

        
        \addplot [draw=penColor, very thick, smooth, domain=(-5.5:5.5), <->] {0.1*x*(x+4)*(x-4)};
        %\addplot [draw=penColor, very thick, smooth, domain=(-1:3)] {0.5*x-1};
        %\addplot [draw=penColor, very thick, smooth, domain=(4:8)] {-2*x+10};


        %\addplot[color=penColor,fill=penColor,only marks,mark=*] coordinates{(-8,-3.3)};
        %\addplot[color=penColor,fill=white,only marks,mark=*] coordinates{(-3,3)};


    \end{axis}
\end{tikzpicture}
\end{image}


\end{example}






\begin{example} \textit{Sine} \\

Sine is an odd function.  The graph of $y= sin(\theta)$ is displayed below. 

\begin{image}
\begin{tikzpicture}
     \begin{axis}[
            	domain=-10:10, ymax=3, xmax=10, ymin=-3, xmin=-10,
            	axis lines =center, xlabel={$\theta$}, ylabel=$y$,
                ytick={-2,-1,1,2},
                xtick={-10,-8,-6,-4,-2,2,4,6,8,10},
                ticklabel style={font=\scriptsize},
            	every axis y label/.style={at=(current axis.above origin),anchor=south},
            	every axis x label/.style={at=(current axis.right of origin),anchor=west},
            	axis on top,
          		]

        
        %\addplot [draw=penColor, very thick, smooth, domain=(-3.7, 3.7), <->] {x};
        \addplot [draw=penColor, very thick, smooth, domain=(-8.3:8.3), <->] {sin(deg(x)};
        %\addplot [draw=penColor, very thick, smooth, domain=(4:8)] {-2*x+10};


        %\addplot[color=penColor,fill=penColor,only marks,mark=*] coordinates{(-8,-3.3)};
        %\addplot[color=penColor,fill=white,only marks,mark=*] coordinates{(-3,3)};


    \end{axis}
\end{tikzpicture}
\end{image}


\end{example}















\section{Shift Symmetry}


\begin{definition} \textit{Periodic}

If you shift a periodic function by its wave length (or period), then you get the original function back: $f(\theta + \omega) = f(\theta)$ for all $\theta$ in the domain, where $\omega$ is the wave length.

\end{definition}



\textit{Sine} and \textit{Cosine} have shift symmetry.  If you shift them by $2\pi$ radians, then the graph lands back on itself. YOu cannot tell that the graph has been shifted.






\begin{example} \textit{Cosine} \\

The graph of $y= cos(\theta + 2\pi)$ is displayed below. 

\begin{image}
\begin{tikzpicture}
     \begin{axis}[
            	domain=-10:10, ymax=3, xmax=10, ymin=-3, xmin=-10,
            	axis lines =center, xlabel={$\theta$}, ylabel=$y$,
                ytick={-2,-1,1,2},
                xtick={-10,-8,-6,-4,-2,2,4,6,8,10},
                ticklabel style={font=\scriptsize},
            	every axis y label/.style={at=(current axis.above origin),anchor=south},
            	every axis x label/.style={at=(current axis.right of origin),anchor=west},
            	axis on top,
          		]

        
        %\addplot [draw=penColor, very thick, smooth, domain=(-3.7, 3.7), <->] {x};
        \addplot [draw=penColor, very thick, smooth, domain=(-8.3:8.3), <->] {cos(deg(x)};
        %\addplot [draw=penColor, very thick, smooth, domain=(4:8)] {-2*x+10};


        %\addplot[color=penColor,fill=penColor,only marks,mark=*] coordinates{(-8,-3.3)};
        %\addplot[color=penColor,fill=white,only marks,mark=*] coordinates{(-3,3)};


    \end{axis}
\end{tikzpicture}
\end{image}


\end{example}







\begin{example} \textit{Sine} \\

The graph of $y= sin(\theta + 2\pi)$ is displayed below. 

\begin{image}
\begin{tikzpicture}
     \begin{axis}[
            	domain=-10:10, ymax=3, xmax=10, ymin=-3, xmin=-10,
            	axis lines =center, xlabel={$\theta$}, ylabel=$y$,
                ytick={-2,-1,1,2},
                xtick={-10,-8,-6,-4,-2,2,4,6,8,10},
                ticklabel style={font=\scriptsize},
            	every axis y label/.style={at=(current axis.above origin),anchor=south},
            	every axis x label/.style={at=(current axis.right of origin),anchor=west},
            	axis on top,
          		]

        
        %\addplot [draw=penColor, very thick, smooth, domain=(-3.7, 3.7), <->] {x};
        \addplot [draw=penColor, very thick, smooth, domain=(-8.3:8.3), <->] {sin(deg(x)};
        %\addplot [draw=penColor, very thick, smooth, domain=(4:8)] {-2*x+10};


        %\addplot[color=penColor,fill=penColor,only marks,mark=*] coordinates{(-8,-3.3)};
        %\addplot[color=penColor,fill=white,only marks,mark=*] coordinates{(-3,3)};


    \end{axis}
\end{tikzpicture}
\end{image}


\end{example}















\begin{example} \textit{Even} \\

Half of the graph of $y = K(t)$ is displayed below.   If $K(t)$ is an even function, then think of what the other half of the graph would look like.



\begin{image}
\begin{tikzpicture}
     \begin{axis}[
                domain=-10:10, ymax=10, xmax=10, ymin=-10, xmin=-10,
                axis lines =center, xlabel=$t$, ylabel=${y=g(t)}$,
                ytick={-10,-8,-6,-4,-2,2,4,6,8,10},
                xtick={-10,-8,-6,-4,-2,2,4,6,8,10},
                ticklabel style={font=\scriptsize},
                every axis y label/.style={at=(current axis.above origin),anchor=south},
                every axis x label/.style={at=(current axis.right of origin),anchor=west},
                axis on top,
                ]

        
        %\addplot [draw=penColor, very thick, smooth, domain=(-7.5:3), <->] {1/(x-3) + 2};
        \addplot [draw=penColor, very thick, smooth, domain=(3:6), <-] {1/(x-3) + 2};
        \addplot [draw=penColor, very thick, smooth, domain=(6:8), ->] {2*x-14};

        \addplot [line width=1, gray, dashed,samples=100,domain=(-9:9)] ({3},{x});
        %\addplot [line width=1, gray, dashed,samples=100,domain=(-9:4)] ({x},{2});

        %\addplot[color=penColor,fill=penColor,only marks,mark=*] coordinates{(-6,5)};
        %\addplot[color=penColor,fill=white,only marks,mark=*] coordinates{(-6,1.88)};
        \addplot[color=penColor,fill=penColor,only marks,mark=*] coordinates{(6,2.33)};
        \addplot[color=penColor,fill=white,only marks,mark=*] coordinates{(6,-2)};

    \end{axis}



\end{tikzpicture}
\end{image}

Visualize what the full graph looks like, then click the arrow.


\begin{expandable}


\begin{image}
\begin{tikzpicture}
     \begin{axis}[
                domain=-10:10, ymax=10, xmax=10, ymin=-10, xmin=-10,
                axis lines =center, xlabel=$t$, ylabel=${y=g(t)}$,
                ytick={-10,-8,-6,-4,-2,2,4,6,8,10},
                xtick={-10,-8,-6,-4,-2,2,4,6,8,10},
                ticklabel style={font=\scriptsize},
                every axis y label/.style={at=(current axis.above origin),anchor=south},
                every axis x label/.style={at=(current axis.right of origin),anchor=west},
                axis on top,
                ]

        
        \addplot [draw=penColor, very thick, smooth, samples=300, domain=(-6:-3.15), ->] {-1/(x+3) + 2};
        \addplot [draw=penColor, very thick, smooth, samples=300, domain=(-8:-6), <-] {-2*x-14};
        \addplot [draw=penColor, very thick, smooth, samples=300, domain=(3.15:6), <-] {1/(x-3) + 2};
        \addplot [draw=penColor, very thick, smooth, samples=300, domain=(6:8), ->] {2*x-14};

        \addplot [line width=1, gray, dashed,samples=100,domain=(-9:9)] ({3},{x});
        \addplot [line width=1, gray, dashed,samples=100,domain=(-9:9)] ({-3},{x});

        \addplot[color=penColor,fill=penColor,only marks,mark=*] coordinates{(-6,2.33)};
        \addplot[color=penColor,fill=white,only marks,mark=*] coordinates{(-6,-2)};
        \addplot[color=penColor,fill=penColor,only marks,mark=*] coordinates{(6,2.33)};
        \addplot[color=penColor,fill=white,only marks,mark=*] coordinates{(6,-2)};

    \end{axis}



\end{tikzpicture}
\end{image}

\end{expandable}

\end{example}














\begin{example} \textit{Odd} \\

Half of the graph of $y = K(t)$ is displayed below.   If $K(t)$ is an odd function, then think of what the other half of the graph would look like.



\begin{image}
\begin{tikzpicture}
     \begin{axis}[
                domain=-10:10, ymax=10, xmax=10, ymin=-10, xmin=-10,
                axis lines =center, xlabel=$t$, ylabel=${y=g(t)}$,
                ytick={-10,-8,-6,-4,-2,2,4,6,8,10},
                xtick={-10,-8,-6,-4,-2,2,4,6,8,10},
                ticklabel style={font=\scriptsize},
                every axis y label/.style={at=(current axis.above origin),anchor=south},
                every axis x label/.style={at=(current axis.right of origin),anchor=west},
                axis on top,
                ]

        
        %\addplot [draw=penColor, very thick, smooth, domain=(-7.5:3), <->] {1/(x-3) + 2};
        \addplot [draw=penColor, very thick, smooth, domain=(3:6), <-] {1/(x-3) + 2};
        \addplot [draw=penColor, very thick, smooth, domain=(6:8), ->] {2*x-14};

        \addplot [line width=1, gray, dashed,samples=100,domain=(-9:9)] ({3},{x});
        %\addplot [line width=1, gray, dashed,samples=100,domain=(-9:4)] ({x},{2});

        %\addplot[color=penColor,fill=penColor,only marks,mark=*] coordinates{(-6,5)};
        %\addplot[color=penColor,fill=white,only marks,mark=*] coordinates{(-6,1.88)};
        \addplot[color=penColor,fill=penColor,only marks,mark=*] coordinates{(6,2.33)};
        \addplot[color=penColor,fill=white,only marks,mark=*] coordinates{(6,-2)};

    \end{axis}



\end{tikzpicture}
\end{image}

Visualize what the full graph looks like, then click the arrow.


\begin{expandable}


\begin{image}
\begin{tikzpicture}
     \begin{axis}[
                domain=-10:10, ymax=10, xmax=10, ymin=-10, xmin=-10,
                axis lines =center, xlabel=$t$, ylabel=${y=g(t)}$,
                ytick={-10,-8,-6,-4,-2,2,4,6,8,10},
                xtick={-10,-8,-6,-4,-2,2,4,6,8,10},
                ticklabel style={font=\scriptsize},
                every axis y label/.style={at=(current axis.above origin),anchor=south},
                every axis x label/.style={at=(current axis.right of origin),anchor=west},
                axis on top,
                ]

        
        \addplot [draw=penColor, very thick, smooth, samples=300, domain=(-6:-3.15), ->] {1/(x+3) - 2};
        \addplot [draw=penColor, very thick, smooth, samples=300, domain=(-8:-6), <-] {2*x+14};
        \addplot [draw=penColor, very thick, smooth, samples=300, domain=(3.15:6), <-] {1/(x-3) + 2};
        \addplot [draw=penColor, very thick, smooth, samples=300, domain=(6:8), ->] {2*x-14};

        \addplot [line width=1, gray, dashed,samples=100,domain=(-9:9)] ({3},{x});
        \addplot [line width=1, gray, dashed,samples=100,domain=(-9:9)] ({-3},{x});

        \addplot[color=penColor,fill=penColor,only marks,mark=*] coordinates{(-6,-2.33)};
        \addplot[color=penColor,fill=white,only marks,mark=*] coordinates{(-6,2)};
        \addplot[color=penColor,fill=penColor,only marks,mark=*] coordinates{(6,2.33)};
        \addplot[color=penColor,fill=white,only marks,mark=*] coordinates{(6,-2)};

    \end{axis}



\end{tikzpicture}
\end{image}

\end{expandable}

\end{example}














\end{document}
