\documentclass{ximera}


\graphicspath{
  {./}
  {ximeraTutorial/}
  {basicPhilosophy/}
}

\newcommand{\mooculus}{\textsf{\textbf{MOOC}\textnormal{\textsf{ULUS}}}}

\usepackage{tkz-euclide}\usepackage{tikz}
\usepackage{tikz-cd}
\usetikzlibrary{arrows}
\tikzset{>=stealth,commutative diagrams/.cd,
  arrow style=tikz,diagrams={>=stealth}} %% cool arrow head
\tikzset{shorten <>/.style={ shorten >=#1, shorten <=#1 } } %% allows shorter vectors

\usetikzlibrary{backgrounds} %% for boxes around graphs
\usetikzlibrary{shapes,positioning}  %% Clouds and stars
\usetikzlibrary{matrix} %% for matrix
\usepgfplotslibrary{polar} %% for polar plots
\usepgfplotslibrary{fillbetween} %% to shade area between curves in TikZ
\usetkzobj{all}
\usepackage[makeroom]{cancel} %% for strike outs
%\usepackage{mathtools} %% for pretty underbrace % Breaks Ximera
%\usepackage{multicol}
\usepackage{pgffor} %% required for integral for loops



%% http://tex.stackexchange.com/questions/66490/drawing-a-tikz-arc-specifying-the-center
%% Draws beach ball
\tikzset{pics/carc/.style args={#1:#2:#3}{code={\draw[pic actions] (#1:#3) arc(#1:#2:#3);}}}



\usepackage{array}
\setlength{\extrarowheight}{+.1cm}
\newdimen\digitwidth
\settowidth\digitwidth{9}
\def\divrule#1#2{
\noalign{\moveright#1\digitwidth
\vbox{\hrule width#2\digitwidth}}}






\DeclareMathOperator{\arccot}{arccot}
\DeclareMathOperator{\arcsec}{arcsec}
\DeclareMathOperator{\arccsc}{arccsc}

















%%This is to help with formatting on future title pages.
\newenvironment{sectionOutcomes}{}{}


\title{Piecewise Functions}

\begin{document}

\begin{abstract}
%Stuff can go here later if we want!
\end{abstract}
\maketitle



The goal of Precalculus is to become familiar with the \textit{Elementary Functions}. You already know some of these.


\begin{summary} The Elementary Functions
	\begin{itemize}
		\item Constant, Linear, Quadratic - Polynomials
		\item Rational
		\item Roots and Radicals
		\item Exponential
		\item Logarithmic
		\item Trigonometric
	\end{itemize}
\end{summary}




These are our basic building blocks for functions.  We will examine their formulas and graphs.  We will identify important features of each type. We will measure their rates of change. 

From these building blocks we build some very sophisticated and complex functions through the usual arithmetic operations and a new operation called \textit{composition}. 

Separate from those types of combinations, we also create new functions by cutting the Elementary Functions into pieces and gluing them together into odd shapes.   These are called \textbf{piecewise defined} functions.  Piecewise defined functions are especially handy for illustrating the rich and weird structure of functions.


We have several well-known functions that are piecewise defined functions. Among these are the \textbf{absolute value function} and the \textbf{greatest integer function}.

We will explore these as well as more general piecewise defined functions in this section. \\






\subsection{Learning Outcomes}



\begin{sectionOutcomes}
In this section, students will 

\begin{itemize}
\item evaluate piecewise functions.
\item graph piecewise functions.
\item solve equations involving piecewise functions.
\item solve inequalities involving piecewise functions.
\end{itemize}
\end{sectionOutcomes}


















\begin{center}
\textbf{\textcolor{green!50!black}{ooooo=-=-=-=-=-=-=-=-=-=-=-=-=ooOoo=-=-=-=-=-=-=-=-=-=-=-=-=ooooo}} \\

more examples can be found by following this link\\ \link[More Examples of Piecewise-Defined Functions]{https://ximera.osu.edu/csccmathematics/precalculus1/precalculus1/piecewise/examples/exampleList}

\end{center}












\end{document}
