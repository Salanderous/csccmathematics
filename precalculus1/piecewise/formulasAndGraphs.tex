\documentclass{ximera}


\graphicspath{
  {./}
  {ximeraTutorial/}
  {basicPhilosophy/}
}

\newcommand{\mooculus}{\textsf{\textbf{MOOC}\textnormal{\textsf{ULUS}}}}

\usepackage{tkz-euclide}\usepackage{tikz}
\usepackage{tikz-cd}
\usetikzlibrary{arrows}
\tikzset{>=stealth,commutative diagrams/.cd,
  arrow style=tikz,diagrams={>=stealth}} %% cool arrow head
\tikzset{shorten <>/.style={ shorten >=#1, shorten <=#1 } } %% allows shorter vectors

\usetikzlibrary{backgrounds} %% for boxes around graphs
\usetikzlibrary{shapes,positioning}  %% Clouds and stars
\usetikzlibrary{matrix} %% for matrix
\usepgfplotslibrary{polar} %% for polar plots
\usepgfplotslibrary{fillbetween} %% to shade area between curves in TikZ
\usetkzobj{all}
\usepackage[makeroom]{cancel} %% for strike outs
%\usepackage{mathtools} %% for pretty underbrace % Breaks Ximera
%\usepackage{multicol}
\usepackage{pgffor} %% required for integral for loops



%% http://tex.stackexchange.com/questions/66490/drawing-a-tikz-arc-specifying-the-center
%% Draws beach ball
\tikzset{pics/carc/.style args={#1:#2:#3}{code={\draw[pic actions] (#1:#3) arc(#1:#2:#3);}}}



\usepackage{array}
\setlength{\extrarowheight}{+.1cm}
\newdimen\digitwidth
\settowidth\digitwidth{9}
\def\divrule#1#2{
\noalign{\moveright#1\digitwidth
\vbox{\hrule width#2\digitwidth}}}






\DeclareMathOperator{\arccot}{arccot}
\DeclareMathOperator{\arcsec}{arcsec}
\DeclareMathOperator{\arccsc}{arccsc}

















%%This is to help with formatting on future title pages.
\newenvironment{sectionOutcomes}{}{}


\title{Stitching Functions}

\begin{document}

\begin{abstract}
pieces and parts
\end{abstract}
\maketitle



Piecewise defined functions are defined by using pieces of other functions. We begin with several functions with their own defining sets of pairs.  Then we take some pairs from each of those sets and collect them into a new set of pairs defining a new function.  Of course, we have to make sure that we do not duplicate domain numbers, otherwise we will not have a function.  This new collection of pairs will automatically give us a domain by collecting all of the first coordinates.

For example, we could start with two constant functions: $Zero$ and $One$.

\begin{itemize}
\item $Zero(r) = 0$ with $(-\infty, \infty)$ as its domain.
\item $One(r) = 1$ with $(-\infty, \infty)$ as its domain.
\end{itemize}

We can create a new function called \textit{step}, by selecting some pairs from each of these functions.  
\begin{itemize}
\item From $Zero$ we'll take pairs with negative domain values.
\item From $One$ we'll take pairs with nonnegative domain values.
\end{itemize}

\textbf{*} Nonnegative means positive or 0.




\begin{example} Step Function


The \textit{step} function uses different formulas depending on the domain number.  If the domain number is negative, then the value of $step$ is $0$. If the value of the domain number is nonnegative, then the value of $step$ is $1$.



\begin{itemize}
\item $step(-5.3) = 0$
\item $step(-0.7) = 0$
\item $step(0) = 0$
\item $step(1.2) = 1$
\item $step(142) = 1$
\end{itemize}



Graph of $y = step(x)$.
\begin{image}
\begin{tikzpicture}
	\begin{axis}[
            domain=-10:10, ymax=5, xmax=10, ymin=-5, xmin=-10,
            axis lines =center, xlabel=$x$, ylabel=$y$,
            ytick={-10,-8,-6,-4,-2,2,4,6,8,10},
            xtick={-10,-8,-6,-4,-2,2,4,6,8,10},
            ticklabel style={font=\scriptsize},
            every axis y label/.style={at=(current axis.above origin),anchor=south},
            every axis x label/.style={at=(current axis.right of origin),anchor=west},
            axis on top
          ]
          
	\addplot [draw=penColor,very thick,smooth,domain=(-9:0),<-] {0};
	\addplot [draw=penColor,very thick,smooth,domain=(0:9),->] {1};
	\addplot[color=penColor,only marks,mark=*] coordinates{(0,1)}; 
	\addplot[color=penColor,fill=white,only marks,mark=*] coordinates{(0,0)}; 

    \end{axis}
\end{tikzpicture}
\end{image}


\end{example}





The step function uses two formulas, but only one at a time.  $step(x) = 0$, if $x < 0$. $step(x) = 1$, if $x \geq 0$.  Either you use the formula $0$ or your use the formula $1$.  The domain number at which you are evaluating $step$ tells you which formula to use.


The traditional way to write this formula looks like

\[
step(x) = 
\begin{cases}
  0 & \text{ if } x < 0 \\
  1 & \text{ if } 0 \leq x
\end{cases}
\]

The formulas are listed in the left column and the domain conditions are listed in the right column.  When evaluating a piecewise defined function, you don't look for the formula first.  First, you decide which domain condition in the right column your domain number satisfies.  Then you choose the corresponding formula and evaluate with your domain number.











\begin{example}

\[
G(t) = 
\begin{cases}
  2t-1 & \text{ if } t \leq -3 \\
  t^2 & \text{ if } t > -3
\end{cases}
\]


\begin{itemize}
\item $G(-5) = 2(-5) - 1 = -11$  
\item $G(-3) = \answer{-7}$ 
\item $G(-2) = 2^2 = 4$ 
\item $G(0) = \answer{0}$ 
\item $G(3) = \answer{9}$ 
\end{itemize}


$-5$ and $-3$ satisfy $t \leq -3$, therefore we use $2t-1$ as their formula. \\
$-2$ and $0$ and $3$ satisfy $t > -3$, therefore we use $t^2$ as their formula. \\

\end{example}


We can have any number of pieces defining our function.

\begin{question}

\[
p(k) = 
\begin{cases}
  k^2 + 1 & \text{ if } -7 < k \leq -4 \\
  -4k + 3 & \text{ if } -4 < k \leq 1 \\
  1 - 3k & \text{ if } k > 3
\end{cases}
\]


\begin{itemize}
\item $p(-6) = \answer{37}$  
\item $p(-2) = \answer{11}$ 
\item $p(-1) = \answer{7}$ 
\item $p(0) = \answer{1}$ 
\item $p(1) = \answer{-2}$ 
\item $p(2) = \answer{DNE}$ 
\item $p(3) = \answer{DNE}$ 
\item $p(4) = \answer{-11}$ 
\end{itemize}

\end{question}



From the formula above, we can see that the domain of $p$ is $(-7, 1] \cup (3, \infty)$.




\begin{example}

\[
T(v) = 
\begin{cases}
  2v-1 & \text{ if }  -4 < v \leq -1 \\
  -v+3 & \text{ if } 1 \leq v < 7
\end{cases}
\]


On the interval $(-4, -1]$, the graph should be a line for $T(v) = 2v-1$. On the interval $[1, 7)$, the graph should be another line for $T(v) = -v+3$




Graph of $y = T(v)$.
\begin{image}
\begin{tikzpicture}
	\begin{axis}[
            domain=-10:10, ymax=10, xmax=10, ymin=-10, xmin=-10,
            axis lines =center, xlabel=$v$, ylabel=$y$,
            ytick={-10,-8,-6,-4,-2,2,4,6,8,10},
            xtick={-10,-8,-6,-4,-2,2,4,6,8,10},
            ticklabel style={font=\scriptsize},
            every axis y label/.style={at=(current axis.above origin),anchor=south},
            every axis x label/.style={at=(current axis.right of origin),anchor=west},
            axis on top
          ]
          
	\addplot [draw=penColor,very thick,smooth,domain=(-4:-1)] {2*x-1};
	\addplot [draw=penColor,very thick,smooth,domain=(1:7)] {-x+3};
	\addplot[color=penColor,only marks,mark=*] coordinates{(-1,-3)}; 
	\addplot[color=penColor,fill=white,only marks,mark=*] coordinates{(-4,-9)}; 
	\addplot[color=penColor,only marks,mark=*] coordinates{(1,2)}; 
	\addplot[color=penColor,fill=white,only marks,mark=*] coordinates{(7,-4)}; 


    \end{axis}
\end{tikzpicture}
\end{image}



The line graph for $y = 2v - 1$ is only drawn over the domain interval $(-4, -1]$. \\
The line graph for $y = -v+3$ is only drawn over the domain interval $[1, 7)$. 




The function $T$ has no minimum value. The maximum value of $T$ is $\answer{2}$.

\end{example}























\begin{example}



Define the function $D(x)$ by the graph below.

\begin{image}
\begin{tikzpicture}
  \begin{axis}[
            domain=-10:10, ymax=10, xmax=10, ymin=-10, xmin=-10,
            axis lines =center, xlabel=$x$, ylabel=$D(x)$,
            ytick={-10,-8,-6,-4,-2,2,4,6,8,10},
            xtick={-10,-8,-6,-4,-2,2,4,6,8,10},
            ticklabel style={font=\scriptsize},
            every axis y label/.style={at=(current axis.above origin),anchor=south},
            every axis x label/.style={at=(current axis.right of origin),anchor=west},
            axis on top
          ]
          
  \addplot [draw=penColor,very thick,smooth,domain=(-8:3)] {3*sin(deg(x)) + 2};
  \addplot [draw=penColor,very thick,smooth,domain=(3:8)] {(1.75^(x-4) - 3};
  \addplot[color=penColor,only marks,mark=*] coordinates{(-8,-0.968) (3,8) (8,6.378)}; 
  \addplot[color=penColor,fill=white,only marks,mark=*] coordinates{(3,2.423) (3,-2.428)}; 
  %\addplot[color=penColor,only marks,mark=*] coordinates{(1,2)}; 
  %\addplot[color=penColor,fill=white,only marks,mark=*] coordinates{(7,-4)}; 


    \end{axis}
\end{tikzpicture}
\end{image}




\begin{question}
The domain of $D$ is

\begin{multipleChoice}
  \choice {$(-8,3) \cup (3,8)$}
  \choice {$[-8,3) \cup (3,8]$}
  \choice [correct]{$[-8,8]$}
\end{multipleChoice}
\end{question}





\begin{question}
The range of $D$ is approximately

\begin{multipleChoice}
  \choice {$[0.97, 8]$}
  \choice {$(-2.43, 8]$}
  \choice {$(-2.43, 6.38) \cup \{ 8 \}$}
  \choice [correct]{$(-2.43, 6.38] \cup \{ 8 \}$}
\end{multipleChoice}
\end{question}



\begin{question}
The function $D$ has no minimum value
\begin{multipleChoice}
  \choice [correct]{True}
  \choice {False}
\end{multipleChoice}
\end{question}


\begin{question}
The function $D$ has no maximum value
\begin{multipleChoice}
  \choice {True}
  \choice [correct]{False}
\end{multipleChoice}
\end{question}




\end{example}


















\end{document}
