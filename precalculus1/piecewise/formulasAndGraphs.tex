\documentclass{ximera}


\graphicspath{
  {./}
  {ximeraTutorial/}
  {basicPhilosophy/}
}

\newcommand{\mooculus}{\textsf{\textbf{MOOC}\textnormal{\textsf{ULUS}}}}

\usepackage{tkz-euclide}\usepackage{tikz}
\usepackage{tikz-cd}
\usetikzlibrary{arrows}
\tikzset{>=stealth,commutative diagrams/.cd,
  arrow style=tikz,diagrams={>=stealth}} %% cool arrow head
\tikzset{shorten <>/.style={ shorten >=#1, shorten <=#1 } } %% allows shorter vectors

\usetikzlibrary{backgrounds} %% for boxes around graphs
\usetikzlibrary{shapes,positioning}  %% Clouds and stars
\usetikzlibrary{matrix} %% for matrix
\usepgfplotslibrary{polar} %% for polar plots
\usepgfplotslibrary{fillbetween} %% to shade area between curves in TikZ
\usetkzobj{all}
\usepackage[makeroom]{cancel} %% for strike outs
%\usepackage{mathtools} %% for pretty underbrace % Breaks Ximera
%\usepackage{multicol}
\usepackage{pgffor} %% required for integral for loops



%% http://tex.stackexchange.com/questions/66490/drawing-a-tikz-arc-specifying-the-center
%% Draws beach ball
\tikzset{pics/carc/.style args={#1:#2:#3}{code={\draw[pic actions] (#1:#3) arc(#1:#2:#3);}}}



\usepackage{array}
\setlength{\extrarowheight}{+.1cm}
\newdimen\digitwidth
\settowidth\digitwidth{9}
\def\divrule#1#2{
\noalign{\moveright#1\digitwidth
\vbox{\hrule width#2\digitwidth}}}






\DeclareMathOperator{\arccot}{arccot}
\DeclareMathOperator{\arcsec}{arcsec}
\DeclareMathOperator{\arccsc}{arccsc}

















%%This is to help with formatting on future title pages.
\newenvironment{sectionOutcomes}{}{}


\title{Piecewise Functions}

\begin{document}

\begin{abstract}
pieces and parts
\end{abstract}
\maketitle



Piecewise functions are defined by using pieces of other functions.




\begin{example} Step Function


The step function uses different formulas depending on the domain number.  If the domain number is positive, then the value of $step$ is $1$. If the value of the domain number is nonpositive, then the value of $step$ is $0$.



\begin{itemize}
\item $step(-5.3) = 0$
\item $step(-0.7) = 0$
\item $step(0) = 0$
\item $step(1.2) = 1$
\item $step(142) = 1$
\end{itemize}



Graph of $y = step(x)$.
\begin{image}
\begin{tikzpicture}
	\begin{axis}[
            domain=-10:10, ymax=5, xmax=10, ymin=-5, xmin=-10,
            axis lines =center, xlabel=$x$, ylabel=$y$,
            every axis y label/.style={at=(current axis.above origin),anchor=south},
            every axis x label/.style={at=(current axis.right of origin),anchor=west},
            axis on top
          ]
          
	\addplot [draw=penColor,very thick,smooth,domain=(-9:0),<-] {0};
	\addplot [draw=penColor,very thick,smooth,domain=(0:9),->] {1};
	\addplot[color=penColor,only marks,mark=*] coordinates{(0,0)}; 
	\addplot[color=penColor,fill=white,only marks,mark=*] coordinates{(0,1)}; 

    \end{axis}
\end{tikzpicture}
\end{image}


\end{example}





The step function uses two formulas, but only one at a time.  $step(x) = 0$, if $x <= 0$. $step(x) = 0$, if $x > 1$.  Either you use the formula $0$ or your use the formula $1$.  The domain number at which you are evaluating $step$ tells y ou which formula to use.


The traditional way to write this formula looks like

\[
step(x) = 
\begin{cases}
  0 & \text{ if } x \leq 0 \\
  1 & \text{ if } x > 0
\end{cases}
\]

The formulas are list in the left column and the domain conditions are listed in the right column.  When evaluating a piecewise defined function, you don;t look for the formula first.  First, you decide which domain condition in the right column your domain number satisfies.  Then you choose the corresponding formula and evaluate with your domain number.



\begin{example}

\[
G(t) = 
\begin{cases}
  2t-1 & \text{ if } t \leq -3 \\
  t^2 & \text{ if } t > -3
\end{cases}
\]


\begin{itemize}
\item $G(-5) = 2(-5) - 1 = -11$  
\item $G(-3) = 2(-3) - 1 = -7$ 
\item $G(-2) = 2^2 = 4$ 
\item $G(0) = 0^2 = 0$ 
\item $G(3) = 3^2 = 9$ 
\end{itemize}

\end{example}




\begin{question}

\[
p(k) = 
\begin{cases}
  k^2 + 1 & \text{ if } -7 < k \leq -4 \\
  -4k + 3 & \text{ if } -4 < k \leq 1 \\
  1 - 3k & \text{ if } k > 3
\end{cases}
\]


\begin{itemize}
\item $p(-6) = \answer{37}$  
\item $p(-2) = \answer{11}$ 
\item $p(-1) = \answer{7}$ 
\item $p(0) = \answer{1}$ 
\item $p(1) = \answer{-2}$ 
\item $p(2) = \answer{DNE}$ 
\item $p(3) = \answer{DNE}$ 
\item $p(4) = \answer{-11}$ 
\end{itemize}

\end{question}



From the formula above, we can see that the domain of $p$ is $(-7, 1] \cup (3, \infty)$.



here





\end{document}
