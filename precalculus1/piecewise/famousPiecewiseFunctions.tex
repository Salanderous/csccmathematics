\documentclass{ximera}


\graphicspath{
  {./}
  {ximeraTutorial/}
  {basicPhilosophy/}
}

\newcommand{\mooculus}{\textsf{\textbf{MOOC}\textnormal{\textsf{ULUS}}}}

\usepackage{tkz-euclide}\usepackage{tikz}
\usepackage{tikz-cd}
\usetikzlibrary{arrows}
\tikzset{>=stealth,commutative diagrams/.cd,
  arrow style=tikz,diagrams={>=stealth}} %% cool arrow head
\tikzset{shorten <>/.style={ shorten >=#1, shorten <=#1 } } %% allows shorter vectors

\usetikzlibrary{backgrounds} %% for boxes around graphs
\usetikzlibrary{shapes,positioning}  %% Clouds and stars
\usetikzlibrary{matrix} %% for matrix
\usepgfplotslibrary{polar} %% for polar plots
\usepgfplotslibrary{fillbetween} %% to shade area between curves in TikZ
\usetkzobj{all}
\usepackage[makeroom]{cancel} %% for strike outs
%\usepackage{mathtools} %% for pretty underbrace % Breaks Ximera
%\usepackage{multicol}
\usepackage{pgffor} %% required for integral for loops



%% http://tex.stackexchange.com/questions/66490/drawing-a-tikz-arc-specifying-the-center
%% Draws beach ball
\tikzset{pics/carc/.style args={#1:#2:#3}{code={\draw[pic actions] (#1:#3) arc(#1:#2:#3);}}}



\usepackage{array}
\setlength{\extrarowheight}{+.1cm}
\newdimen\digitwidth
\settowidth\digitwidth{9}
\def\divrule#1#2{
\noalign{\moveright#1\digitwidth
\vbox{\hrule width#2\digitwidth}}}






\DeclareMathOperator{\arccot}{arccot}
\DeclareMathOperator{\arcsec}{arcsec}
\DeclareMathOperator{\arccsc}{arccsc}

















%%This is to help with formatting on future title pages.
\newenvironment{sectionOutcomes}{}{}


\title{Famous Piecewise}

\begin{document}

\begin{abstract}
examples
\end{abstract}
\maketitle



We have many famous functions, which are piecewise defined function.  We have already seen the step function.  It acts as a mathematical on/off switch.

Perhaps more famous are the absolute value funciton and greatest integer function.







\begin{example} \textit{Absolute Value Function}
The Absolute Value function is made from pieces of two linear functions: $L_1(x) = x$ and $L_2(x) = -x$. 

\begin{itemize}
\item From $L_1$ we'll take pairs with nonnegative domain values.
\item From $L_2$ we'll take pairs with negative domain values.
\end{itemize}

The traditional way of notating the absolute value function is with little vertical bars.


\[
|x| = 
\begin{cases}
  -x & \text{ if }  x < 0 \\
  x & \text{ if } x \geq 0
\end{cases}
\]



Graph of $y = |x|$.
\begin{image}
\begin{tikzpicture}
  \begin{axis}[
            domain=-10:10, ymax=10, xmax=10, ymin=-10, xmin=-10,
            axis lines =center, xlabel=$x$, ylabel=$y$,
            every axis y label/.style={at=(current axis.above origin),anchor=south},
            every axis x label/.style={at=(current axis.right of origin),anchor=west},
            axis on top
          ]
          
  \addplot [draw=penColor,very thick,smooth,domain=(-6:0),<-] {-x};
  \addplot [draw=penColor,very thick,smooth,domain=(0:6),->] {x};


    \end{axis}
\end{tikzpicture}
\end{image}


\end{example}















\begin{example} \textit{Greatest Integer Function}
Also known as the \textit{floor function}, the \textit{Greatest Integer function} (\textit{GIF}) pairs a domain number with the greatest integer less than or equal to the domain number.

If you think of a number line, then you start at the domain on go down (or left) to the next integer.  If you domina number is already an integer, then you don't need to go anywhere.


Like the absolute value function, the greatest integer function has its own symbol:  $GIF(x) = \lfloor x\rfloor$. The little feet remind us to go down to the next integer.

\begin{itemize}
\item $GIF(4.5) = 4$
\item $GIF(2) = 2$
\item $GIF(0.5) = 0$
\item $GIF(0) = 0$
\item $GIF(-4.5) = -5$
\item $GIF(-7) = -7$
\end{itemize}




Let $N$ be any integer. On the interval $[N, N+1)$, we have $GIF(x) = N$.

\textit{GIF} is made from an infinite number of pieces of constant functions.






Graph of $y = \lfloor x\rfloor$ is below, except the steps keep going up to the right and down to the left.
\begin{image}
\begin{tikzpicture}
  \begin{axis}[
            domain=-2:4,
            width=6in,
            height=3in,
            axis lines =middle, xlabel=$x$, ylabel=$y$,
            every axis y label/.style={at=(current axis.above origin),anchor=south},
            every axis x label/.style={at=(current axis.right of origin),anchor=west},
            clip=false,
            %axis on top,
          ]
          \addplot [textColor, very thin, domain=(0:2.3)] {0}; % puts the axis back, axis on top clobbers our open holes
          \addplot [textColor, very thin] plot coordinates {(0,0) (0,2)}; % puts the axis back, axis on top clobbers our open holes
          \addplot [very thick, penColor, domain=(-2:-1)] {-2};
          \addplot [very thick, penColor, domain=(-1:0)] {-1};
          \addplot [very thick, penColor, domain=(0:1)] {0};
          \addplot [very thick, penColor, domain=(1:2)] {1};
          \addplot [very thick, penColor, domain=(2:3)] {2};
          \addplot [very thick, penColor, domain=(3:4)] {3};
          \addplot[color=penColor,fill=penColor,only marks,mark=*] coordinates{(-2,-2)};  %% closed hole          
          \addplot[color=penColor,fill=penColor,only marks,mark=*] coordinates{(-1,-1)};  %% closed hole          
          \addplot[color=penColor,fill=penColor,only marks,mark=*] coordinates{(0,0)};  %% closed hole          
          \addplot[color=penColor,fill=penColor,only marks,mark=*] coordinates{(1,1)};  %% closed hole          
          \addplot[color=penColor,fill=penColor,only marks,mark=*] coordinates{(2,2)};  %% closed hole  
          \addplot[color=penColor,fill=penColor,only marks,mark=*] coordinates{(3,3)};  %% closed hole                  
          \addplot[color=penColor,fill=background,only marks,mark=*] coordinates{(-1,-2)};  %% open hole
          \addplot[color=penColor,fill=background,only marks,mark=*] coordinates{(0,-1)};  %% open hole
          \addplot[color=penColor,fill=background,only marks,mark=*] coordinates{(1,0)};  %% open hole
          \addplot[color=penColor,fill=background,only marks,mark=*] coordinates{(2,1)};  %% open hole
          \addplot[color=penColor,fill=background,only marks,mark=*] coordinates{(3,2)};  %% open hole
          \addplot[color=penColor,fill=background,only marks,mark=*] coordinates{(4,3)};  %% open hole
        \end{axis}
\end{tikzpicture}
\end{image}


Notice that we have many domain numbers with the same function value. That is ok. To be a function, we just need that each domain number only has one function value.  The \textit{GIF} passes this test.




\end{example}







\[
GIF{x} = 
\begin{cases}
   etc. &    \\
  -2 & \text{ on } [-2, -1)] \\
  -1 & \text{ on } [-1, 0) \\ 
   0 & \text{ on } [0, 1)] \\
  1 & \text{ on } [1, 2) \\
   2 & \text{ on } [2, 3)] \\
  etc. &  
\end{cases}
\]





\begin{question}
  Calculate:
  \[
  \lfloor 2.4 \rfloor
  \begin{prompt}
    =\answer{2}
  \end{prompt}
  \]
  \end{question}

  \begin{question}
  Calculate:
  \[
  \lfloor -2.4 \rfloor
  \begin{prompt}
    =\answer{-3}
  \end{prompt}
  \]
\end{question}





Notice that both the functions described above pass the so-called \textit{vertical line test}.

\begin{theorem} \textit{Vertical Line Test}
The curve $y=f(x)$ represents $y$ as a function of $x$ at $x=a$ if and only if the vertical line $x=a$ intersects the curve $y=f(x)$ at
exactly one point. This is called the \textbf{vertical line test}.
\end{theorem}



\end{document}
