\documentclass{ximera}


\graphicspath{
  {./}
  {ximeraTutorial/}
  {basicPhilosophy/}
}

\newcommand{\mooculus}{\textsf{\textbf{MOOC}\textnormal{\textsf{ULUS}}}}

\usepackage{tkz-euclide}\usepackage{tikz}
\usepackage{tikz-cd}
\usetikzlibrary{arrows}
\tikzset{>=stealth,commutative diagrams/.cd,
  arrow style=tikz,diagrams={>=stealth}} %% cool arrow head
\tikzset{shorten <>/.style={ shorten >=#1, shorten <=#1 } } %% allows shorter vectors

\usetikzlibrary{backgrounds} %% for boxes around graphs
\usetikzlibrary{shapes,positioning}  %% Clouds and stars
\usetikzlibrary{matrix} %% for matrix
\usepgfplotslibrary{polar} %% for polar plots
\usepgfplotslibrary{fillbetween} %% to shade area between curves in TikZ
\usetkzobj{all}
\usepackage[makeroom]{cancel} %% for strike outs
%\usepackage{mathtools} %% for pretty underbrace % Breaks Ximera
%\usepackage{multicol}
\usepackage{pgffor} %% required for integral for loops



%% http://tex.stackexchange.com/questions/66490/drawing-a-tikz-arc-specifying-the-center
%% Draws beach ball
\tikzset{pics/carc/.style args={#1:#2:#3}{code={\draw[pic actions] (#1:#3) arc(#1:#2:#3);}}}



\usepackage{array}
\setlength{\extrarowheight}{+.1cm}
\newdimen\digitwidth
\settowidth\digitwidth{9}
\def\divrule#1#2{
\noalign{\moveright#1\digitwidth
\vbox{\hrule width#2\digitwidth}}}






\DeclareMathOperator{\arccot}{arccot}
\DeclareMathOperator{\arcsec}{arcsec}
\DeclareMathOperator{\arccsc}{arccsc}

















%%This is to help with formatting on future title pages.
\newenvironment{sectionOutcomes}{}{}



\author{Alan Yang}

\begin{document}





The study of the real numbers often separates into knowledge about the whole part of a number and the fractional part.  This generates the idea of the floor and ceiling functions.





\begin{definition}  \textbf{\textcolor{green!50!black}{Floor}} \\

The \textbf{floor function} maps the real numbers onto the integers.

\[
\lfloor \, \rfloor : \mathbb{R} \mapsto \mathbb{Z}
\]


\[
\lfloor r \rfloor = N \text{, where } N \in \mathbb{Z} \text{ and } N \leq r < N+1  
\]


The \textbf{floor function} returns the greatest integer less than or equal to the argument.

\end{definition}








\begin{definition}  \textbf{\textcolor{green!50!black}{Ceiling}} \\

The \textbf{ceiling function} maps the real numbers onto the integers.

\[
\lceil \, \rceil : \mathbb{R} \mapsto \mathbb{Z}
\]


\[
\lceil r \rceil = N \text{, where } N \in \mathbb{Z} \text{ and } N-1 < r \leq N  
\]


The \textbf{ceiling function} returns the least integer greater than or equal to the argument.

\end{definition}









\begin{exercise} Evaluating \\

\begin{align*}
\lfloor 1.5 \rfloor  &=  \answer{1}       & \, | \, &    \lceil 1.5 \rceil  &=  \answer{2}  \\
\lfloor -1.5 \rfloor  &=  \answer{-2}       & \, | \, &    \lceil -1.5 \rceil  &=  \answer{-1}  \\
\lfloor \sqrt{5} \rfloor  &=  \answer{2}  & \, | \, &    \lceil \sqrt{5} \rceil  &=  \answer{3}  \\
\lfloor -\sqrt{5} \rfloor  &=  \answer{-3}  & \, | \, &    \lceil -\sqrt{5} \rceil  &=  \answer{-2}  \\
\lfloor e \rfloor  &=  \answer{2}  & \, | \, &    \lceil e \rceil  &=  \answer{3}  \\
\lfloor  7 \rfloor  &=  \answer{7}  & \, | \, &    \lceil 7 \rceil  &=  \answer{7}  \\
\lfloor \frac{\pi}{3} \rfloor  &=  \answer{1}  & \, | \, &    \lceil \frac{\pi}{3} \rceil  &=  \answer{2}  \\
\lfloor -4 \rfloor  &=  \answer{-4}  & \, | \, &    \lceil -4 \rceil  &=  \answer{-4}  
\end{align*}


\end{exercise}





















\end{document}