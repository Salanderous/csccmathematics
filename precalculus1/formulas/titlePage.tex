\documentclass{ximera}


\graphicspath{
  {./}
  {ximeraTutorial/}
  {basicPhilosophy/}
}

\newcommand{\mooculus}{\textsf{\textbf{MOOC}\textnormal{\textsf{ULUS}}}}

\usepackage{tkz-euclide}\usepackage{tikz}
\usepackage{tikz-cd}
\usetikzlibrary{arrows}
\tikzset{>=stealth,commutative diagrams/.cd,
  arrow style=tikz,diagrams={>=stealth}} %% cool arrow head
\tikzset{shorten <>/.style={ shorten >=#1, shorten <=#1 } } %% allows shorter vectors

\usetikzlibrary{backgrounds} %% for boxes around graphs
\usetikzlibrary{shapes,positioning}  %% Clouds and stars
\usetikzlibrary{matrix} %% for matrix
\usepgfplotslibrary{polar} %% for polar plots
\usepgfplotslibrary{fillbetween} %% to shade area between curves in TikZ
\usetkzobj{all}
\usepackage[makeroom]{cancel} %% for strike outs
%\usepackage{mathtools} %% for pretty underbrace % Breaks Ximera
%\usepackage{multicol}
\usepackage{pgffor} %% required for integral for loops



%% http://tex.stackexchange.com/questions/66490/drawing-a-tikz-arc-specifying-the-center
%% Draws beach ball
\tikzset{pics/carc/.style args={#1:#2:#3}{code={\draw[pic actions] (#1:#3) arc(#1:#2:#3);}}}



\usepackage{array}
\setlength{\extrarowheight}{+.1cm}
\newdimen\digitwidth
\settowidth\digitwidth{9}
\def\divrule#1#2{
\noalign{\moveright#1\digitwidth
\vbox{\hrule width#2\digitwidth}}}






\DeclareMathOperator{\arccot}{arccot}
\DeclareMathOperator{\arcsec}{arcsec}
\DeclareMathOperator{\arccsc}{arccsc}

















%%This is to help with formatting on future title pages.
\newenvironment{sectionOutcomes}{}{}


\title{Formulas}

\begin{document}

\begin{abstract}

\end{abstract}
\maketitle


Functions are relations, which makes them packages.  They contain three sets.  They contain a set of real numbers called the \textbf{domain}.  They contain a seond set of real numbers called the \textbf{range}. Finally, they contain a third set of pairs of numbers.  The pairs are all constructed with a number from each of the domain and range.

The pairs are the reason we study functions.  The pairs are the connection between the domain and range.  They identify which numbers are associated together from both sets. They itemize the association between the domain and range.

Therefore, it comes as no surprise, that we have a toolbox full of tools to help us describes these pairs.


\section{Function Notation}

We have some shorthand notation to help us communicate individual pairs: $F(d) = r$.



\section{Graphs}

We have visual tools where $F(d) = r$ is encoded by plotting a dot at $(d, r)$ on the Cartesian plane.  Graphs are drawings, which makes them inherently inaccurate.  They are not exact tools.  They are global tools.  They give a big picture of the function pairs.  From this bird's eye view, we can see trands in the funciotn values, extremen funciton values, important function values. The global perspective quickly communicates over all characteristics and locates function features.


\section{Formulas}
We also want exactness!

The price we pay for exactness is that we can only investigate the function, one pair at a time.  Our tool for this type of investigation is called a \textbf{formula}.































\begin{sectionOutcomes}
After completing this section, students should 

\begin{itemize}
\item evaluate a function via a formula.
\item .
\item .
\item .
\end{itemize}
\end{sectionOutcomes}

\end{document}
