\documentclass{ximera}


\graphicspath{
  {./}
  {ximeraTutorial/}
  {basicPhilosophy/}
}

\newcommand{\mooculus}{\textsf{\textbf{MOOC}\textnormal{\textsf{ULUS}}}}

\usepackage{tkz-euclide}\usepackage{tikz}
\usepackage{tikz-cd}
\usetikzlibrary{arrows}
\tikzset{>=stealth,commutative diagrams/.cd,
  arrow style=tikz,diagrams={>=stealth}} %% cool arrow head
\tikzset{shorten <>/.style={ shorten >=#1, shorten <=#1 } } %% allows shorter vectors

\usetikzlibrary{backgrounds} %% for boxes around graphs
\usetikzlibrary{shapes,positioning}  %% Clouds and stars
\usetikzlibrary{matrix} %% for matrix
\usepgfplotslibrary{polar} %% for polar plots
\usepgfplotslibrary{fillbetween} %% to shade area between curves in TikZ
\usetkzobj{all}
\usepackage[makeroom]{cancel} %% for strike outs
%\usepackage{mathtools} %% for pretty underbrace % Breaks Ximera
%\usepackage{multicol}
\usepackage{pgffor} %% required for integral for loops



%% http://tex.stackexchange.com/questions/66490/drawing-a-tikz-arc-specifying-the-center
%% Draws beach ball
\tikzset{pics/carc/.style args={#1:#2:#3}{code={\draw[pic actions] (#1:#3) arc(#1:#2:#3);}}}



\usepackage{array}
\setlength{\extrarowheight}{+.1cm}
\newdimen\digitwidth
\settowidth\digitwidth{9}
\def\divrule#1#2{
\noalign{\moveright#1\digitwidth
\vbox{\hrule width#2\digitwidth}}}






\DeclareMathOperator{\arccot}{arccot}
\DeclareMathOperator{\arcsec}{arcsec}
\DeclareMathOperator{\arccsc}{arccsc}

















%%This is to help with formatting on future title pages.
\newenvironment{sectionOutcomes}{}{}


\title{Assemble}

\begin{document}

\begin{abstract}
putting together
\end{abstract}
\maketitle






Let $Out(x) = 2|x-3|+1$ with its natural or implied domain. \\

Let $In(t) = -(t-1)^2 + 5$ with its natural or implied domain. \\









Graph of $y = Out(x) =2|x-3|+1$

\begin{image}
\begin{tikzpicture}
  \begin{axis}[
            domain=-10:10, ymax=10, xmax=10, ymin=-10, xmin=-10,
            axis lines =center, xlabel=$x$, ylabel={$y=Out(x)$}, grid = major,
            ytick={-10,-8,-6,-4,-2,2,4,6,8,10},
            xtick={-10,-8,-6,-4,-2,2,4,6,8,10},
            yticklabels={$-10$,$-8$,$-6$,$-4$,$-2$,$2$,$4$,$6$,$8$,$10$}, 
            xticklabels={$-10$,$-8$,$-6$,$-4$,$-2$,$2$,$4$,$6$,$8$,$10$},
            ticklabel style={font=\scriptsize},
            every axis y label/.style={at=(current axis.above origin),anchor=south},
            every axis x label/.style={at=(current axis.right of origin),anchor=west},
            axis on top
          ]
          
          %\addplot [line width=2, penColor2, smooth,samples=100,domain=(-6:2)] {-2*x-3};
			\addplot [line width=2, penColor, smooth,samples=100,domain=(-1:7),<->] {2*abs(x-3)+1};

          %\addplot[color=penColor,fill=penColor2,only marks,mark=*] coordinates{(-6,9)};
          %\addplot[color=penColor,fill=penColor2,only marks,mark=*] coordinates{(2,-7)};



           

  \end{axis}
\end{tikzpicture}
\end{image}
The natural domain of $Out$ is $\mathbb{R}$. \\

The graph of $Out$ has a "V"-shape, opening up, with a corner at $(3,1)$. \\
This illustrates that $Out$ decreases on $(-\infty, 3]$ and increases on $[3, \infty)$.  $Out$ has a minimum value of $1$, which occurs at $3$. \\

If we think of tracing the domain from left to right on the graph, the corresponding values of $Out$ decrease from $\infty$ down to $1$, then reverse direction and increase from $1$ back to $\infty$. \\



Graph of $z = In(t) = -(t-1)^2 + 5$





\begin{image}
\begin{tikzpicture}
  \begin{axis}[
            domain=-10:10, ymax=10, xmax=10, ymin=-10, xmin=-10,
            axis lines =center, xlabel=$t$, ylabel={$z=In(t)$}, grid = major,
            ytick={-10,-8,-6,-4,-2,2,4,6,8,10},
            xtick={-10,-8,-6,-4,-2,2,4,6,8,10},
            yticklabels={$-10$,$-8$,$-6$,$-4$,$-2$,$2$,$4$,$6$,$8$,$10$}, 
            xticklabels={$-10$,$-8$,$-6$,$-4$,$-2$,$2$,$4$,$6$,$8$,$10$},
            ticklabel style={font=\scriptsize},
            every axis y label/.style={at=(current axis.above origin),anchor=south},
            every axis x label/.style={at=(current axis.right of origin),anchor=west},
            axis on top
          ]
          
          %\addplot [line width=2, penColor2, smooth,samples=100,domain=(-6:2)] {-2*x-3};
			\addplot [line width=2, penColor, smooth,samples=100,domain=(-2.5:4.5),<->] {-(x-1)^2 + 5)};

          %\addplot[color=penColor,fill=penColor2,only marks,mark=*] coordinates{(-6,9)};
          %\addplot[color=penColor,fill=penColor2,only marks,mark=*] coordinates{(2,-7)};



           

  \end{axis}
\end{tikzpicture}
\end{image}
The natural domain of $In$ is $\mathbb{R}$. \\


Since $In$ is a quadratic function, its graph has the shape of a parabola, opening down.  $In$ increases on $(-\infty, 1]$ and decreases on $[1, \infty)$. $In$ has a maximum value of $5$, which occurs at $1$. \\

If we think of tracing the domain from left to right on the graph, the values of $In$ increase from $-\infty$ up to $5$, then reverse direction and decrease from $5$ back to $-\infty$.


The output from the function $In$ will become the input to $Out$.  Therefore, in the composition, the new input to $Out$ will be the interval $(-\infty, 5]$, which is the output of $In$. This interval is traced twice. The values going into $Out$ will begin at $-\infty$, go up to $5$, then turn around and go back down to $-\infty$. The values of $Out$ will similarly repeat themselves in reverse.\\






\textbf{\textcolor{purple!85!blue}{Now to examine the composition:}} \\



\textbf{\textcolor{blue!75!black}{$\blacktriangleright$}}   First, the inside function: $In(t) = -(t-1)^2 + 5$ \\

The implied domain of $In(t)$ is the whole real line. As we move from left to right along the real line, the domain numbers increase from $-\infty$ to $\infty$.

The corresponding movement in the range has the function values increasing from $-\infty$ to $5$.  The maximum function value $5$ occurs at $1$ in the domain.  As we keep moving beyond $1$ in the domain, the corresponding movement in the range is for the function values to decrease from $1$ to $-\infty$.


This movement in the range of $In(t)$ becomes the movement in the domain of $Out(x)$. \\













\begin{image}
\begin{tikzpicture}
    \begin{axis}[name = sinax, domain=-10:10, ymax=10, xmax=10, ymin=-10, xmin=-10, width=3in, height=3in,
                  axis lines =center, xlabel=$x$, ylabel={$y=In(x)$}, grid = major,
                  ytick={-10,-8,-6,-4,-2,2,4,6,8,10},
                  xtick={-10,-8,-6,-4,-2,2,4,6,8,10},
                  yticklabels={$-10$,$-8$,$-6$,$-4$,$-2$,$2$,$4$,$6$,$8$,$10$}, 
                  xticklabels={$-10$,$-8$,$-6$,$-4$,$-2$,$2$,$4$,$6$,$8$,$10$},
                  ticklabel style={font=\scriptsize},
                  every axis y label/.style={at=(current axis.above origin),anchor=south},
                  every axis x label/.style={at=(current axis.right of origin),anchor=west},
                  axis on top]
        \addplot [line width=2, penColor, smooth,samples=100,domain=(-2.5:4.5),<->] {-(x-1)^2 + 5)};

        \addplot [line width=1, penColor2, smooth, domain=(-9.5:5)] ({-0.75},{x});
        \addplot [line width=1, penColor2, smooth, domain=(-9.5:5),<-] ({-0.25},{x});
        \addplot [line width=1, penColor2, smooth, domain=(0:3.14)] ({0.25*cos(deg(x))-0.5},{0.25*sin(deg(x))+5});

    \end{axis}
    \begin{axis}[at={(sinax.outer east)},anchor=outer west, domain=-10:10, ymax=10, xmax=10, ymin=-10, xmin=-10, 
                  width=3in, height=3in,
                  axis lines =center, xlabel=$t$, ylabel={$z=Out(t)$}, grid = major,
                  ytick={-10,-8,-6,-4,-2,2,4,6,8,10},
                  xtick={-10,-8,-6,-4,-2,2,4,6,8,10},
                  yticklabels={$-10$,$-8$,$-6$,$-4$,$-2$,$2$,$4$,$6$,$8$,$10$}, 
                  xticklabels={$-10$,$-8$,$-6$,$-4$,$-2$,$2$,$4$,$6$,$8$,$10$},
                  ticklabel style={font=\scriptsize},
                  every axis y label/.style={at=(current axis.above origin),anchor=south},
                  every axis x label/.style={at=(current axis.right of origin),anchor=west},
                  axis on top]
        \addplot [line width=2, penColor, smooth,samples=100,domain=(-1:7),<->] {2*abs(x-3)+1};

        \addplot [line width=1, penColor2, smooth, domain=(-9.5:5)] ({x},{-0.25});
        \addplot [line width=1, penColor2, smooth, domain=(-9.5:5),<-] ({x},{-0.75});
        \addplot [line width=1, penColor2, smooth, domain=(-1.57:1.57)] ({0.25*cos(deg(x))+5},{0.25*sin(deg(x))-0.5});

    \end{axis}


\end{tikzpicture}
\end{image}



















\textbf{\textcolor{blue!75!black}{$\blacktriangleright$}}  Movement inside the domain of $Out(x)$.\\



Inside the implied domain of $Out(x) =2|x-3|+1$, the domain numbers will increase from $-\infty$ to $5$.  Then they will decrease from $5$ back down to $-\infty$. \\

Tracing our graph of $Out(x)$, we will move from the far left toward the right until we reach $5$.  Then we will turn around and move toward the left.\\



\textbf{\textcolor{blue!75!black}{$\blacktriangleright$}}  Put those together. \\



We have the input to $Out$ moving from $-\infty$ up to $5$ and then back down to $-\infty$.  What are the corresponding function values for $Out$?





\begin{itemize}

\item As the domain numbers move through $(-\infty, 5]$, the values of $Out$ decrease from $\infty$ to $1$ and then increase from $1$ to $5$.  In the graph, we can see this reversal as a corner in the graph. The graph comes down to the right, hits the corner, then moves back up to the right.





\begin{image}
\begin{tikzpicture}
  \begin{axis}[
            domain=-10:10, ymax=10, xmax=10, ymin=-10, xmin=-10,
            axis lines =center, xlabel=$x$, ylabel=$y$, grid = major,
            ytick={-10,-8,-6,-4,-2,2,4,6,8,10},
            xtick={-10,-8,-6,-4,-2,2,4,6,8,10},
            yticklabels={$-10$,$-8$,$-6$,$-4$,$-2$,$2$,$4$,$6$,$8$,$10$}, 
            xticklabels={$-10$,$-8$,$-6$,$-4$,$-2$,$2$,$4$,$6$,$8$,$10$},
            ticklabel style={font=\scriptsize},
            every axis y label/.style={at=(current axis.above origin),anchor=south},
            every axis x label/.style={at=(current axis.right of origin),anchor=west},
            axis on top
          ]
          
          %\addplot [line width=2, penColor2, smooth,samples=100,domain=(-6:2)] {-2*x-3};
      \addplot [line width=2, penColor, smooth,samples=100,domain=(-1:5),<-] {2*abs(x-3)+1};
      

          %\addplot[color=penColor,fill=penColor2,only marks,mark=*] coordinates{(-6,9)};
          %\addplot[color=penColor,fill=penColor2,only marks,mark=*] coordinates{(2,-7)};



           

  \end{axis}
\end{tikzpicture}
\end{image}




So, we really should view $(-\infty, 5]$  as  $(-\infty, 3] \cup (3,5]$ and highlight the change in behavior.







\begin{image}
\begin{tikzpicture}
  \begin{axis}[
            domain=-10:10, ymax=10, xmax=10, ymin=-10, xmin=-10,
            axis lines =center, xlabel=$x$, ylabel=$y$, grid = major,
            ytick={-10,-8,-6,-4,-2,2,4,6,8,10},
            xtick={-10,-8,-6,-4,-2,2,4,6,8,10},
            yticklabels={$-10$,$-8$,$-6$,$-4$,$-2$,$2$,$4$,$6$,$8$,$10$}, 
            xticklabels={$-10$,$-8$,$-6$,$-4$,$-2$,$2$,$4$,$6$,$8$,$10$},
            ticklabel style={font=\scriptsize},
            every axis y label/.style={at=(current axis.above origin),anchor=south},
            every axis x label/.style={at=(current axis.right of origin),anchor=west},
            axis on top
          ]
          
          %\addplot [line width=2, penColor2, smooth,samples=100,domain=(-6:2)] {-2*x-3};
      \addplot [line width=2, penColor, smooth,samples=100,domain=(-1:3),<-] {2*abs(x-3)+1};
      \addplot [line width=2, penColor2, smooth,samples=100,domain=(3:5)] {2*abs(x-3)+1};

          %\addplot[color=penColor,fill=penColor2,only marks,mark=*] coordinates{(-6,9)};
          %\addplot[color=penColor,fill=penColor2,only marks,mark=*] coordinates{(2,-7)};



           

  \end{axis}
\end{tikzpicture}
\end{image}




	\begin{itemize}[label=$\star$]
		\item On $(-\infty, 3]$, $Out$ decreases from $\infty$ to $1$.

		\item On $[3,5]$, $Out$ increases from $1$ to $5$

	\end{itemize}



That was on our first pass through $(-\infty, 5]$. \\



The input to $Out$ continues beginning at $5$ and moving towards $-\infty$.  \\

As the overall domain to the whole compostion continues to cover $[5, \infty)$, The function $In$ reverses this interval and sends the reversed interval into $Out$. As the overall composition domain continues to move through $[5, \infty)$, the input to $Out$ traces backwards from $5$ to $-\infty$.  The graph keeps moving to the right, but it is tracing the left side in reverse, mirroring it.




\item Moving backwards through $(-\infty, 5])$, The values of $Out$ decrease from $5$ to $1$, arriving again at the corner that occurs at $3$.  So, we really should look at $(-\infty, 5]$  as  $(-\infty, 3] \cup [3,5]$.

	\begin{itemize}
		\item As the domain moves from $5$ to $3$, $Out$ decreases from $5$ to $1$.

		\item As the domain moves from $3$ to $-\infty$, $Out$ increases from $1$ to $\infty$.

	\end{itemize}

\end{itemize}

Since we reverse our travelling direction inside the domain, we hit the corner twice.  Plus, right at our reversal, we will create a hill in the graph made by our own retracing of the domain steps.






\begin{image}
\begin{tikzpicture}
  \begin{axis}[
            domain=-10:10, ymax=10, xmax=10, ymin=-10, xmin=-10,
            axis lines =center, xlabel=$m$, ylabel=$w$, grid = major,
            ytick={-10,-8,-6,-4,-2,2,4,6,8,10},
            xtick={-10,-8,-6,-4,-2,2,4,6,8,10},
            yticklabels={$-10$,$-8$,$-6$,$-4$,$-2$,$2$,$4$,$6$,$8$,$10$}, 
            xticklabels={$-10$,$-8$,$-6$,$-4$,$-2$,$2$,$4$,$6$,$8$,$10$},
            ticklabel style={font=\scriptsize},
            every axis y label/.style={at=(current axis.above origin),anchor=south},
            every axis x label/.style={at=(current axis.right of origin),anchor=west},
            axis on top
          ]
            \addplot [line width=2, penColor, smooth,samples=100,domain=(-1.5:-0.414),<-] {2*abs(-(x-1)^2 + 2)+1};
            \addplot [line width=2, penColor2, smooth,samples=100,domain=(-0.414:2.414)] {2*abs(-(x-1)^2 + 2)+1};
            \addplot [line width=2, penColor, smooth,samples=100,domain=(2.414:3.5),->] {2*abs(-(x-1)^2 + 2)+1};

  \end{axis}
\end{tikzpicture}
\end{image}














\textbf{\textcolor{blue!55!black}{All together now...}}   \\


Graph of $w = Out(In(m)) =2 \, |(-(m-1)^2 + 5)-3|+1 = 2 \, |-(m-1)^2 + 2| + 1$







\begin{image}
\begin{tikzpicture}
  \begin{axis}[
            domain=-10:10, ymax=10, xmax=10, ymin=-10, xmin=-10,
            axis lines =center, xlabel=$m$, ylabel=$w$, grid = major,
            ytick={-10,-8,-6,-4,-2,2,4,6,8,10},
            xtick={-10,-8,-6,-4,-2,2,4,6,8,10},
            yticklabels={$-10$,$-8$,$-6$,$-4$,$-2$,$2$,$4$,$6$,$8$,$10$}, 
            xticklabels={$-10$,$-8$,$-6$,$-4$,$-2$,$2$,$4$,$6$,$8$,$10$},
            ticklabel style={font=\scriptsize},
            every axis y label/.style={at=(current axis.above origin),anchor=south},
            every axis x label/.style={at=(current axis.right of origin),anchor=west},
            axis on top
          ]
          	\addplot [line width=2, penColor, smooth,samples=100,domain=(-1.5:3.5),<->] {2*abs(-(x-1)^2 + 2)+1};

   

  \end{axis}
\end{tikzpicture}
\end{image}






























\begin{example}  Composition



Let $Out(x) = -2|x-3|+2$ with its natural or implied domain. \\
Let $In(t) = 2 \sin(t)+2$ with its natural or implied domain. \\



Graph $w = (Out \circ In)(r) = Out(In(r))$


\begin{explanation}

Graph of $y = Out(x) = -2|x-3|+2$

\begin{image}
\begin{tikzpicture}
  \begin{axis}[
            domain=-10:10, ymax=10, xmax=10, ymin=-10, xmin=-10,
            axis lines =center, xlabel=$x$, ylabel={$y=Out(x)$}, grid = major,
            ytick={-10,-8,-6,-4,-2,2,4,6,8,10},
            xtick={-10,-8,-6,-4,-2,2,4,6,8,10},
            yticklabels={$-10$,$-8$,$-6$,$-4$,$-2$,$2$,$4$,$6$,$8$,$10$}, 
            xticklabels={$-10$,$-8$,$-6$,$-4$,$-2$,$2$,$4$,$6$,$8$,$10$},
            ticklabel style={font=\scriptsize},
            every axis y label/.style={at=(current axis.above origin),anchor=south},
            every axis x label/.style={at=(current axis.right of origin),anchor=west},
            axis on top
          ]
          
          %\addplot [line width=2, penColor2, smooth,samples=100,domain=(-6:2)] {-2*x-3};
      \addplot [line width=2, penColor, smooth,samples=100,domain=(-3:9),<->] {-2*abs(x-3)+2};

          %\addplot[color=penColor,fill=penColor2,only marks,mark=*] coordinates{(-6,9)};
          %\addplot[color=penColor,fill=penColor2,only marks,mark=*] coordinates{(2,-7)};



           

  \end{axis}
\end{tikzpicture}
\end{image}
The natural domain of $Out$ is $\mathbb{R}$. \\

If we think of moving through the domain from $-\infty$ to $\infty$, then the graph illustrates that $Out$ will increase on $(-\infty, 3]$ and decrease on $[3, \infty)$.  The function $Out$ has a maximum value of $2$, which occurs at $3$. \\


However, the output from $In$ is not $\mathbb{R}$, so we will not get the full output of $Out$. In addition, the output of $In$ will oscillate, which means the input values coming into $Out$ will osciallate.  This will make the output of $Out$ oscillate. \\






Graph of $z = In(t) = 2 \sin(t)+2$





\begin{image}
\begin{tikzpicture}
  \begin{axis}[
            domain=-10:10, ymax=10, xmax=10, ymin=-10, xmin=-10,
            axis lines =center, xlabel=$t$, ylabel=$z$, grid = major,
            ytick={-10,-8,-6,-4,-2,2,4,6,8,10},
            xtick={-10,-8,-6,-4,-2,2,4,6,8,10},
            yticklabels={$-10$,$-8$,$-6$,$-4$,$-2$,$2$,$4$,$6$,$8$,$10$}, 
            xticklabels={$-10$,$-8$,$-6$,$-4$,$-2$,$2$,$4$,$6$,$8$,$10$},
            ticklabel style={font=\scriptsize},
            every axis y label/.style={at=(current axis.above origin),anchor=south},
            every axis x label/.style={at=(current axis.right of origin),anchor=west},
            axis on top
          ]
          
          %\addplot [line width=2, penColor2, smooth,samples=100,domain=(-6:2)] {-2*x-3};
      \addplot [line width=2, penColor, smooth,samples=100,domain=(-9:9),<->] {2*sin(deg(x))+2};

          %\addplot[color=penColor,fill=penColor2,only marks,mark=*] coordinates{(-6,9)};
          %\addplot[color=penColor,fill=penColor2,only marks,mark=*] coordinates{(2,-7)};



           

  \end{axis}
\end{tikzpicture}
\end{image}




The input into $In(t)$ is the whole real line.  We naturally think of moving left to right, from $-\infty$ to $\infty$.  As we do this, the outputs from $In(t)$ oscillate between $0$ and $\answer{4}$.  Therefore, the inputs into $Out(x)$ oscillate back and forth along the interval $\left[ \answer{0}, \answer{4} \right]$.





The input into $Out(x)$ keep going back and forth across the interval $[0,4]$. 



\begin{image}
\begin{tikzpicture}
  \begin{axis}[
            domain=-10:10, ymax=10, xmax=10, ymin=-10, xmin=-10,
            axis lines =center, xlabel=$x$, ylabel=$y$, grid = major,
            ytick={-10,-8,-6,-4,-2,2,4,6,8,10},
            xtick={-10,-8,-6,-4,-2,2,4,6,8,10},
            yticklabels={$-10$,$-8$,$-6$,$-4$,$-2$,$2$,$4$,$6$,$8$,$10$}, 
            xticklabels={$-10$,$-8$,$-6$,$-4$,$-2$,$2$,$4$,$6$,$8$,$10$},
            ticklabel style={font=\scriptsize},
            every axis y label/.style={at=(current axis.above origin),anchor=south},
            every axis x label/.style={at=(current axis.right of origin),anchor=west},
            axis on top
          ]
          
          %\addplot [line width=2, penColor2, smooth,samples=100,domain=(-6:2)] {-2*x-3};
      \addplot [line width=2, penColor, smooth,samples=100,domain=(-3:9),<->] {-2*abs(x-3)+2};

      \addplot [line width=1, penColor2, smooth,samples=300,domain=(-9:9),<->] ({2*sin(deg(x))+2},{0.2*x});

          %\addplot[color=penColor,fill=penColor2,only marks,mark=*] coordinates{(-6,9)};
          %\addplot[color=penColor,fill=penColor2,only marks,mark=*] coordinates{(2,-7)};



           

  \end{axis}
\end{tikzpicture}
\end{image}

The only part of the graph of $Out$ that is used is on the interval $[0, 4]$.






\begin{image}
\begin{tikzpicture}
  \begin{axis}[
            domain=-10:10, ymax=10, xmax=10, ymin=-10, xmin=-10,
            axis lines =center, xlabel=$x$, ylabel=$y$, grid = major,
            ytick={-10,-8,-6,-4,-2,2,4,6,8,10},
            xtick={-10,-8,-6,-4,-2,2,4,6,8,10},
            yticklabels={$-10$,$-8$,$-6$,$-4$,$-2$,$2$,$4$,$6$,$8$,$10$}, 
            xticklabels={$-10$,$-8$,$-6$,$-4$,$-2$,$2$,$4$,$6$,$8$,$10$},
            ticklabel style={font=\scriptsize},
            every axis y label/.style={at=(current axis.above origin),anchor=south},
            every axis x label/.style={at=(current axis.right of origin),anchor=west},
            axis on top
          ]
          
          %\addplot [line width=2, penColor2, smooth,samples=100,domain=(-6:2)] {-2*x-3};
      \addplot [line width=2, penColor, smooth,samples=100,domain=(0:4),<->] {-2*abs(x-3)+2};

  \end{axis}
\end{tikzpicture}
\end{image}






Therefore, that part of the graph of $Out$ just keeps repeating. The smooth rounding of the sine curve will round out the corners of the absolute value graph.







\begin{image}
\begin{tikzpicture}
  \begin{axis}[
            domain=-10:10, ymax=10, xmax=10, ymin=-10, xmin=-10,
            axis lines =center, xlabel=$r$, ylabel=$w$, grid = major,
            ytick={-10,-8,-6,-4,-2,2,4,6,8,10},
            xtick={-10,-8,-6,-4,-2,2,4,6,8,10},
            yticklabels={$-10$,$-8$,$-6$,$-4$,$-2$,$2$,$4$,$6$,$8$,$10$}, 
            xticklabels={$-10$,$-8$,$-6$,$-4$,$-2$,$2$,$4$,$6$,$8$,$10$},
            ticklabel style={font=\scriptsize},
            every axis y label/.style={at=(current axis.above origin),anchor=south},
            every axis x label/.style={at=(current axis.right of origin),anchor=west},
            axis on top
          ]
          
          %\addplot [line width=2, penColor2, smooth,samples=100,domain=(-6:2)] {-2*x-3};
      \addplot [line width=2, penColor, smooth,samples=100,domain=(-9:9),<->] {-2*abs((2*sin(deg(x))+2)-3)+2};

      %\addplot [line width=1, penColor2, smooth,samples=300,domain=(-9:9),<->] ({2*sin(deg(x))+2},{0.2*x});

          %\addplot[color=penColor,fill=penColor2,only marks,mark=*] coordinates{(-6,9)};
          %\addplot[color=penColor,fill=penColor2,only marks,mark=*] coordinates{(2,-7)};



           

  \end{axis}
\end{tikzpicture}
\end{image}




\end{explanation}

\end{example}


Graphically, we have to keep our eyes on several things at once.  We watch the original input into $In$, then we watch the output of $In$ and picture that as the new input into $Out$, then we watch the output coming from $Out$.







Algebraically, we just replace.


















\section{Algebraically}







Let $Out(x) = 3x^3 + \sin(4x) - \frac{3}{2-x}$ with its natural or implied domain. \\
Let $In(t) = \frac{5t-7}{t^2-8}$ with its natural or implied domain. \\



Algebraically, composition is accomplished by replacing all occurences of the variable with the entire formula for the other function.




\[
(Out \circ In)(k) = Out(In(k)) = 3 {\frac{5k-7}{k^2-8}}^3 + \sin(4 \frac{5k-7}{k^2-8}) - \frac{3}{2-\frac{5t-7}{k^2-8}}
\]





Obviously, parentheses are vital here.





\[
(Out \circ In)(k) = Out(In(k)) = 3 \left( \frac{5k-7}{k^2-8} \right)^3 + \sin\left( 4 \left( \frac{5k-7}{k^2-8} \right) \right) - \frac{3}{2 - \left( \frac{5k-7}{k^2-8} \right)}
\]






We can always make two compositions from two functions.




\[
(In \circ Out)(u) = In(Out(u)) = \frac{5 \left( 3u^3 + \sin(4u) - \frac{3}{2-u} \right)-7}{\left( 3u^3 + \sin(4u) - \frac{3}{2-u} \right)^2-8}
\]





\begin{example} Composition 



Let $f(x) = 5x^2 - 4x + 2$ with its natural or implied domain. \\
Let $g(t) = \frac{3}{6-t}$ with its natural or implied domain. \\



\[
(f \circ g)(y) = f(g(y)) = 5 \left( \answer{\frac{3}{6-y}} \right)^2 - 4 \left( \answer{\frac{3}{6-y}} \right) + 2
\]





\[
(g \circ f)(w) = g(f(w)) = \frac{3}{6 - \left( \answer{5w^2 - 4w + 2} \right)}
\]



\end{example}










\begin{example} Composition 



Let $f(x) = 3x^2 - x + 5$ with its natural or implied domain. \\
Let $g(t) = \frac{4}{7-t}$ with its natural or implied domain. \\





There are two numbers where $f(x) = 7$.  One of them is $1$, the other is $\answer{\frac{-2}{3}}$.






\[
(g \circ f)(w) = g(f(w)) = \frac{4}{7 - (5w^2 - 4w + 2)}
\]



In the $g \circ f$ composition, $f$ is no longer allowed to equal $\answer{7}$.

Select all real numbers which cannot be in the domain of $g \circ f$.

\begin{selectAll}
\choice{$7$}
\choice{$0$}
\choice[correct]{$1$}
\choice{$\frac{2}{3}$}
\choice{$-1$}
\choice[correct]{$\frac{-2}{3}$}
\end{selectAll}


\end{example}




























\end{document}
