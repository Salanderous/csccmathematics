\documentclass{ximera}


\graphicspath{
  {./}
  {ximeraTutorial/}
  {basicPhilosophy/}
}

\newcommand{\mooculus}{\textsf{\textbf{MOOC}\textnormal{\textsf{ULUS}}}}

\usepackage{tkz-euclide}\usepackage{tikz}
\usepackage{tikz-cd}
\usetikzlibrary{arrows}
\tikzset{>=stealth,commutative diagrams/.cd,
  arrow style=tikz,diagrams={>=stealth}} %% cool arrow head
\tikzset{shorten <>/.style={ shorten >=#1, shorten <=#1 } } %% allows shorter vectors

\usetikzlibrary{backgrounds} %% for boxes around graphs
\usetikzlibrary{shapes,positioning}  %% Clouds and stars
\usetikzlibrary{matrix} %% for matrix
\usepgfplotslibrary{polar} %% for polar plots
\usepgfplotslibrary{fillbetween} %% to shade area between curves in TikZ
\usetkzobj{all}
\usepackage[makeroom]{cancel} %% for strike outs
%\usepackage{mathtools} %% for pretty underbrace % Breaks Ximera
%\usepackage{multicol}
\usepackage{pgffor} %% required for integral for loops



%% http://tex.stackexchange.com/questions/66490/drawing-a-tikz-arc-specifying-the-center
%% Draws beach ball
\tikzset{pics/carc/.style args={#1:#2:#3}{code={\draw[pic actions] (#1:#3) arc(#1:#2:#3);}}}



\usepackage{array}
\setlength{\extrarowheight}{+.1cm}
\newdimen\digitwidth
\settowidth\digitwidth{9}
\def\divrule#1#2{
\noalign{\moveright#1\digitwidth
\vbox{\hrule width#2\digitwidth}}}






\DeclareMathOperator{\arccot}{arccot}
\DeclareMathOperator{\arcsec}{arcsec}
\DeclareMathOperator{\arccsc}{arccsc}

















%%This is to help with formatting on future title pages.
\newenvironment{sectionOutcomes}{}{}


\title{Assemble}

\begin{document}

\begin{abstract}
putting together
\end{abstract}
\maketitle






Let $Out(x) = 2|x-3|+1$ with its implied domain. \\
Let $In(t) = -(t-1)^2 + 5$.









Graph of $y = Out(x) =2|x-3|+1$

\begin{image}
\begin{tikzpicture}
  \begin{axis}[
            domain=-10:10, ymax=10, xmax=10, ymin=-10, xmin=-10,
            axis lines =center, xlabel=$x$, ylabel=$y$, grid = major,
            ytick={-10,-8,-6,-4,-2,2,4,6,8,10},
            xtick={-10,-8,-6,-4,-2,2,4,6,8,10},
            yticklabels={$-10$,$-8$,$-6$,$-4$,$-2$,$2$,$4$,$6$,$8$,$10$}, 
            xticklabels={$-10$,$-8$,$-6$,$-4$,$-2$,$2$,$4$,$6$,$8$,$10$},
            ticklabel style={font=\scriptsize},
            every axis y label/.style={at=(current axis.above origin),anchor=south},
            every axis x label/.style={at=(current axis.right of origin),anchor=west},
            axis on top
          ]
          
          %\addplot [line width=2, penColor2, smooth,samples=100,domain=(-6:2)] {-2*x-3};
			\addplot [line width=2, penColor, smooth,samples=100,domain=(-1:7),<->] {2*abs(x-3)+1};

          %\addplot[color=penColor,fill=penColor2,only marks,mark=*] coordinates{(-6,9)};
          %\addplot[color=penColor,fill=penColor2,only marks,mark=*] coordinates{(2,-7)};



           

  \end{axis}
\end{tikzpicture}
\end{image}


Graph of $z = In(t) = -(t-1)^2 + 5$





\begin{image}
\begin{tikzpicture}
  \begin{axis}[
            domain=-10:10, ymax=10, xmax=10, ymin=-10, xmin=-10,
            axis lines =center, xlabel=$t$, ylabel=$z$, grid = major,
            ytick={-10,-8,-6,-4,-2,2,4,6,8,10},
            xtick={-10,-8,-6,-4,-2,2,4,6,8,10},
            yticklabels={$-10$,$-8$,$-6$,$-4$,$-2$,$2$,$4$,$6$,$8$,$10$}, 
            xticklabels={$-10$,$-8$,$-6$,$-4$,$-2$,$2$,$4$,$6$,$8$,$10$},
            ticklabel style={font=\scriptsize},
            every axis y label/.style={at=(current axis.above origin),anchor=south},
            every axis x label/.style={at=(current axis.right of origin),anchor=west},
            axis on top
          ]
          
          %\addplot [line width=2, penColor2, smooth,samples=100,domain=(-6:2)] {-2*x-3};
			\addplot [line width=2, penColor, smooth,samples=100,domain=(-2.5:4.5),<->] {-(x-1)^2 + 5)};

          %\addplot[color=penColor,fill=penColor2,only marks,mark=*] coordinates{(-6,9)};
          %\addplot[color=penColor,fill=penColor2,only marks,mark=*] coordinates{(2,-7)};



           

  \end{axis}
\end{tikzpicture}
\end{image}





Now to examine the composition. \\



$\blacktriangleright$  First, the inside function: $In(t) = -(t-1)^2 + 5$ \\

The implied domain of $In(t)$ is the whole real line. As we move from left to right along the real line, the domain numbers increase from $-\infty$ to $\infty$.

The corresponding movement in the range is for the funciton values to increase from $-\infty$ to $5$.  The function value $5$ occurs at $1$ in the domain.  As we keep moving beyond $1$ in the domain, the corresponding movement in the range is for the function values to decrease from $1$ to $-\infty$.


This movement in the range of $In(t)$ is the movement in the domain of $Out(x)$. \\


$\blacktriangleright$ Movement inside the domain of $Out(x)$.\\



Inside the implied domain of $Out(x) =2|x-3|+1$. the domain numbers will increase from $-\infty$ to $5$.  Then they will decrease from $5$ back down to $-\infty$. \\



$\blacktriangleright$ Put those together. \\

The whole composition, $(Out \circ In)$, never sees the inside speeding up, slowing down, turning around, running in circles, or any other swivelling and swirling.  It just sees the normal domain movement left ot right from $-\infty$ to $\infty$.  And, it sees whatever function values from $Out$ that the come out of the process.

We have the input to $Out$ moving from $-\infty$ up to $5$ and then back down to $-\infty$.  What are the corresponding function values for $Out$?





\begin{itemize}
\item As the domain numbers move through $(-\infty, 5]$, The values of $Out$ are decreasing from $\infty$ to $1$.  This is the corner that occurs at $3$.  So, we really should look at $(-\infty, 5]$  as  $(-\infty, 1] \cup [1,5]$

	\begin{itemize}[label=$\star$]
		\item $(-\infty, 1]$

		\item $[1,5]$

	\end{itemize}












\end{itemize}










































\end{document}
