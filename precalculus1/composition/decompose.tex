\documentclass{ximera}


\graphicspath{
  {./}
  {ximeraTutorial/}
  {basicPhilosophy/}
}

\newcommand{\mooculus}{\textsf{\textbf{MOOC}\textnormal{\textsf{ULUS}}}}

\usepackage{tkz-euclide}\usepackage{tikz}
\usepackage{tikz-cd}
\usetikzlibrary{arrows}
\tikzset{>=stealth,commutative diagrams/.cd,
  arrow style=tikz,diagrams={>=stealth}} %% cool arrow head
\tikzset{shorten <>/.style={ shorten >=#1, shorten <=#1 } } %% allows shorter vectors

\usetikzlibrary{backgrounds} %% for boxes around graphs
\usetikzlibrary{shapes,positioning}  %% Clouds and stars
\usetikzlibrary{matrix} %% for matrix
\usepgfplotslibrary{polar} %% for polar plots
\usepgfplotslibrary{fillbetween} %% to shade area between curves in TikZ
\usetkzobj{all}
\usepackage[makeroom]{cancel} %% for strike outs
%\usepackage{mathtools} %% for pretty underbrace % Breaks Ximera
%\usepackage{multicol}
\usepackage{pgffor} %% required for integral for loops



%% http://tex.stackexchange.com/questions/66490/drawing-a-tikz-arc-specifying-the-center
%% Draws beach ball
\tikzset{pics/carc/.style args={#1:#2:#3}{code={\draw[pic actions] (#1:#3) arc(#1:#2:#3);}}}



\usepackage{array}
\setlength{\extrarowheight}{+.1cm}
\newdimen\digitwidth
\settowidth\digitwidth{9}
\def\divrule#1#2{
\noalign{\moveright#1\digitwidth
\vbox{\hrule width#2\digitwidth}}}






\DeclareMathOperator{\arccot}{arccot}
\DeclareMathOperator{\arcsec}{arcsec}
\DeclareMathOperator{\arccsc}{arccsc}

















%%This is to help with formatting on future title pages.
\newenvironment{sectionOutcomes}{}{}


\title{Disassemble}

\begin{document}

\begin{abstract}
taking apart
\end{abstract}
\maketitle





\begin{example}


Let $H(d)$ and $M(p)$ be two functions and $H \circ M$ be their composition.


Suppose $M(p) = 3 p - 5$ \\

Suppose $(H \circ M)(k) = 5 k + 1$


Find $H(d)$.



\textbf{\textcolor{purple!50!blue!90!black}{SOLUTION}} \\




$(H \circ M)(k) = H(M(k)) = H(3 k - 5)$  \\



$H$ is going to do something to $3 k - 5$ and produce $5 k + 1$. \\




These are both linear functions, therefore $H(d)$ should be a linear function.


First $H$ should multiply $3 k - 5$ by $\frac{5}{3}$ to produce the $5$.


\[    \frac{5}{3} (3 k - 5) = 5k - \frac{25}{3}   \]



Now, we need $-\frac{25}{3}$ to become $1$.  $H$ should add $\frac{28}{3}$.

\[  H(d) =    \frac{5}{3} d +     \frac{28}{3} \]



$\blacktriangleright$ Check: 


\[    (H \circ M)(k) = H(M(k)) = H(3 k - 5) =    \frac{5}{3} (3 k - 5) +     \frac{28}{3}  = 5k - \frac{25}{3}  + \frac{28}{3}  = 5k + \frac{3}{3}  = 5k + 1\]

\end{example}













\begin{example}


Let $H(d)$ and $M(p)$ be two functions and $H \circ M$ be their composition.


Suppose $H(d) = 3 d - 5$ \\

Suppose $(H \circ M)(k) = 5 k + 1$


Find $M(p)$.



\textbf{\textcolor{purple!50!blue!90!black}{SOLUTION}} \\




$(H \circ M)(k) = H(M(k)) = 3 M(k) - 5 $  \\




These are both linear functions, therefore $M(k)$ should be a linear function.


$M$ is going to eb multiplied by $3$, therefore $M$ should start with $M(p) = \frac{5}{3}p$. That way, when it is multiplied by $3$, we will get $5$.




Now, we need $3$ times something to be $1$.  $M$ should add $\frac{1}{3}$.

\[  M(p) =    \frac{5}{3} p +     \frac{1}{3} \]



$\blacktriangleright$ Check: 


\[    (H \circ M)(k) = H(M(k)) = 3 \left( \frac{5}{3} k + \frac{1}{3} \right) - 5 =    5 k + 1    \]

\end{example}










\begin{example}


Let $f(x)$ and $g(w)$ be two functions and $f \circ g$ be their composition.


Suppose $g(w) = 3w + 1$ \\

Suppose $(f \circ g)(t) = \frac{2t}{t-1}$


Find $f(x)$.



\textbf{\textcolor{purple!50!blue!90!black}{SOLUTION}} \\




$(f \circ g)(t) =  f(g(t)) =  \frac{2t}{t-1}$ \\



The composition has two occurrences of the variable.  $g$ has one occurrence and when it is put into $f$ there are two occurrences.  $f$ must have two occurences of its variable - one in a numerator and one in a denominator.  $3t + 1$ will replace each in the composition.

So, let's start contrucing. \\


$\blacktriangleright$  Guess \#1 \\

\[      f(x) = \frac{x}{x}       \]


this gives us

\[   f(g(t)) =      \frac{3t + 1}{3t + 1}       \]



We need the coefficients of $t$ to be $2$ in the numerator and $1$ in the denominator.


\[   f(g(t)) =      \frac{\frac{2}{3}(3t + 1)}{\frac{1}{3}(3t + 1)}       \]

That gives


\[   f(g(t)) =      \frac{2t + \frac{2}{3}}{t + \frac{1}{3}}       \]



Now, we need constant terms of $0$ in the numerator and $-1$ in the denominator.  Subtract $\frac{2}{3}$ from the numerator. Subtract $\frac{4}{3}$ from the denominator.




\[   f(g(t)) =      \frac{\frac{2}{3}(3t + 1) - \frac{2}{3}}{\frac{1}{3}(3t + 1) - \frac{4}{3}}       \]

That tells us what $f$ needs to do.




\[   f(x) =      \frac{\frac{2}{3}x - \frac{2}{3}}{\frac{1}{3}x - \frac{4}{3}}       \]



\end{example}























\end{document}
