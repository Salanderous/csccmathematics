\documentclass{ximera}


\graphicspath{
  {./}
  {ximeraTutorial/}
  {basicPhilosophy/}
}

\newcommand{\mooculus}{\textsf{\textbf{MOOC}\textnormal{\textsf{ULUS}}}}

\usepackage{tkz-euclide}\usepackage{tikz}
\usepackage{tikz-cd}
\usetikzlibrary{arrows}
\tikzset{>=stealth,commutative diagrams/.cd,
  arrow style=tikz,diagrams={>=stealth}} %% cool arrow head
\tikzset{shorten <>/.style={ shorten >=#1, shorten <=#1 } } %% allows shorter vectors

\usetikzlibrary{backgrounds} %% for boxes around graphs
\usetikzlibrary{shapes,positioning}  %% Clouds and stars
\usetikzlibrary{matrix} %% for matrix
\usepgfplotslibrary{polar} %% for polar plots
\usepgfplotslibrary{fillbetween} %% to shade area between curves in TikZ
\usetkzobj{all}
\usepackage[makeroom]{cancel} %% for strike outs
%\usepackage{mathtools} %% for pretty underbrace % Breaks Ximera
%\usepackage{multicol}
\usepackage{pgffor} %% required for integral for loops



%% http://tex.stackexchange.com/questions/66490/drawing-a-tikz-arc-specifying-the-center
%% Draws beach ball
\tikzset{pics/carc/.style args={#1:#2:#3}{code={\draw[pic actions] (#1:#3) arc(#1:#2:#3);}}}



\usepackage{array}
\setlength{\extrarowheight}{+.1cm}
\newdimen\digitwidth
\settowidth\digitwidth{9}
\def\divrule#1#2{
\noalign{\moveright#1\digitwidth
\vbox{\hrule width#2\digitwidth}}}






\DeclareMathOperator{\arccot}{arccot}
\DeclareMathOperator{\arcsec}{arcsec}
\DeclareMathOperator{\arccsc}{arccsc}

















%%This is to help with formatting on future title pages.
\newenvironment{sectionOutcomes}{}{}


\title{Polynomials}

\begin{document}

\begin{abstract}
characteristics
\end{abstract}
\maketitle




\begin{definition}
\textbf{Polynomials} (over the real numbers) are functions that can be described as sums of power functions


\[    p(x) = a_n x^n + a_{n-1} x^{n-1} + \cdots + a_3 x^3 + a_2 x^2 + a_1 x^1 + a_0 x^0      \]

where the $a_k$ are real numbers and $a_n \ne 0$.


\begin{itemize}
\item Each of the summands are called a \textbf{term}.  
\item $n$ is called the \textbf{degree} of the polynomial.
\item $a_k$ are called \textbf{coefficients}.
\item $a_n$ is called the \textbf{leading coefficient}.
\item $a_n x^n$ is called the \textbf{leading term}.
\item $a_2 x^2$ is called the \textbf{quadratic term}.
\item $a_1 x^1$ is called the \textbf{linear term}, often written as $a_1 x$.
\item $a_0 x^0$ is called the \textbf{constant term}, often written as $a_0$.

\end{itemize}

\end{definition}





For analytical purposes, we would prefer a product

\[   p(x) = a (x-r_n)(x-r_{n-1})  \cdots (x-r_2)(x-r_1)  \]


\begin{itemize}

\item $a$ is called the \textbf{leading coefficient}.
\item $(x-r_k)$ are called \textbf{factors}.
\item $r_k$ are called the \textbf{zeros} or \textbf{roots}.

\end{itemize}






Polynomials are the nicest functions we have. \\

The graph of $y = f(x) = \frac{1}{100} x^3 + \frac{1}{25} x^2 - \frac{31}{100} x - \frac{7}{10}$

\begin{image}
\begin{tikzpicture} 
  \begin{axis}[
            domain=-10:10, ymax=10, xmax=10, ymin=-10, xmin=-10,
            axis lines =center, xlabel=$x$, ylabel=$y$,
            ytick={-10,-8,-6,-4,-2,2,4,6,8,10},
            xtick={-10,-8,-6,-4,-2,2,4,6,8,10},
            ticklabel style={font=\scriptsize},
            every axis y label/.style={at=(current axis.above origin),anchor=south},
            every axis x label/.style={at=(current axis.right of origin),anchor=west},
            axis on top
          ]
          
          \addplot [line width=2, penColor, smooth, domain=(-9:9),<->] {0.01*(x+7)*(x+2)*(x-5)};

           

  \end{axis}
\end{tikzpicture}
\end{image}



All of their characteristics and features are nice.

\begin{itemize}
\item Their domains include all real numbers. 
\item They are continuous everywhere.
\item No discontinuities or singularities.
\item No asymptotes on the graph.
\item Their graphs are smooth - no corners or endpoints
\end{itemize}


Polynomials smoothly alternate between increasing and decreasing, which switch at global and local maximums and minimums.

As you will see later, polynomials have a limit on the number of zeros and extrema they can have.  They cannot have more than the degree of the polynomial. So, what happens outside all of the wiggling?  This is known as the \textbf{endbehavior}.  All polynomials continue without bound after the zeros and extrema.  Either they tend to infinity or negative infinity.



\section{Factored Form}


$p(x) = a (x-r_n)(x-r_{n-1})  \cdots (x-r_2)(x-r_1)$ is the factored form of a polynomial.  In an effort to be clearer, we clean up the roots, because there could be repeated roots and we would like to point this out.  Therefore the standard factored form looks like




\[   p(x) = a (x-r_n)^{e_n} (x-r_{n-1})^{e_{n-1}}  \cdots (x-r_2)^{e_2} (x-r_1)^{e_1}  \]



In this version, the $r_k$ are distinct roots.  They are all different. \\
The $e_k$ are the exponents and tell us how many times a particular root repeats.  The exponents are sometimes referred to as the factor's \textbf{multiplicity} or the root's multiplicity.\\

As we noted with quadratics, this factorization into linear factors often requires complex numbers.  Therefore, our factorizations will include linear factors as well as irreducible quadratic factors.

Irreducible quadratics do not have real roots.  Therefore, for this discussion on roots, we can pretend that we have all rlinear factors.










\textbf{\large Running the Number Line}

If we think of the real numbers as lining up to form the number line, then we could imagine ourselves running over the number line from $-\infty$ to $\infty$, or from left to right.  We would hit every real number once.  We would run through each distinct root once.  As we ran from the left side to the right side of each root, the sign of the value of the polynomial with either change or stay the same.

At each root. $r_k$, the value of the polynomial is $0$.  But on either side of the root, the polynomial is either positive or negative. There are four possible combinations.


\[
\begin{array}{rcl}
\text{Left of Root}  & \text{Root}  & \text{Right of Root} \\
positive & 0 & positive \\
negative & 0 & negative \\
positive & 0 & negative \\
negative & 0 & positive
\end{array}
\]


Either the sign switches or it stays the same across the root.  What in the formula dictates this?



\[   f(x) = a (x-r_n)^{e_n} (x-r_{n-1})^{e_{n-1}}  \cdots (x-r_2)^{e_2} (x-r_1)^{e_1}  \]


If we imagine ourselves running through $r_k$, then $(x-r_k)^{e_k}$ is the only factor that could possibly change sign.  All of the other factors maintain their same sign when we run through $r_k$.


When $x<r_k$, then $x-r_k <0$.  \\

When $x>r_k$, then $x-r_k >0$. \\


If $e_k$ is even then it doesn't matter if $x-r_k <0$ or $x-r_k >0$, $(x-r_k)^{e_k}$ will be positve. 

If $e_k$ is odd then it does matter, because, unlike an even power, a negative number raise to an odd power will remain negative.





\begin{itemize}
\item If $e_k$ is odd, then $(x-r_k)^{e_k}$ will change its sign.
\item If $e_k$ is even, then $(x-r_k)^{e_k}$ will not change its sign.
\end{itemize}







\begin{itemize}
\item If $r_k$ is a root of odd multiplicty, then $(x-r_k)^{e_k}$ will change its sign.
\item If $e_k$ is a root of even multiplicty, then $(x-r_k)^{e_k}$ will not change its sign.
\end{itemize}




\textbf{\large Graphically}


Graphically, the values of the function are measured vertically. Zeroes are represented as dots positioned on the horizontal axis.  The graph either flows from one side of the horizontal axis to the other side as the function changes sign or the graph rebounds from the intercept and stays on the same side of the horizontal axis.  The multiplicity of the root tells us which.






\begin{example}

Change the exponent of the $x+2$ factor from odd to even.  The graph will crossover the $x$-axis for odd exponents and bounce back for even expoents.  If you change the sign of the leading coefficient, then the crossing and bouncing will jump to the other side of the $x$-axis.



\begin{center}
\desmos{igck8p6w15}{400}{300}
\end{center}

\end{example}






















\section{Extrema}

Unless we have a quadratic polynomial, the best we can do for the maximums and minimums, at this point, is estimate them.







\begin{example}

The graph of $y = f(x) = 0.1(x+2)^3(x-3)$.



\begin{center}
\desmos{igck8p6w15}{400}{300}
\end{center}



$f(x)$ has an approximate global minimum value of $\answer[tolerance=0.1]{1.75}$, which occurs approximately at $\answer[tolerance=0.1]{-6.6}$.


\end{example}




The extreme values of the function occur at numbers in the domain called \textbf{critical numbers}.  Thus, critical numbers play an important role in function analysis.  


It will take some analysis to get a full definition of critical numbers. We'll start with critical numbers pointing out horizontal tangent lines in the graph and improve from there.









\begin{definition}

\textbf{Critical numbers} are domain numbers, which correspond to points on the graph where the tangent line is horizontal.

\end{definition}




Critical numbers are domain numbers marking flat/horizontal places in the graph.  These include tops of hills and bottoms of valleys. Critical numbers include places where extrema value occur.


The next example shows that they also include places that are not marking maximums or minimums.














\section{Rate of Change}



Polynomials are nice.  They increase and decrease on intervals defined by critical numbers marking places where the graph is horizontal and possible maximums and minimums.









\begin{example}

The graph of $y = f(x) = 0.1(x+2)^3(x-3)$.



\begin{center}
\desmos{qyzq162w2e}{400}{300}
\end{center}



$f$ has two critical numbers: $-2$ and $1.75$. These correspond to the two places where the graph has a horizontal tangent line.

\begin{itemize}
\item $f$ decreases on $\left(-\infty, \answer{1.75}\right]$.
\item $f$ increases on $\left[\answer{1.75}, \infty\right)$.
\end{itemize}




\end{example}


\textbf{Note:} $-2$ is a critical number, because the graph is flat there.  However, the sign of the rate of change of $f$ does not switch there, since $f$ does not have an extreme value there.










\section{Zeros, Factors, and Intercepts}




$\blacktriangleright$ Polynomial functions are nice.  They allow us to see many connections that we will keep in mind as we investigate other types of functions.

\begin{quote}
Zeros, Factors, and Intercepts are different things but they all refer to the same idea.
\end{quote}




\begin{itemize}

\item If $z_0$ is a zero of the polynomial $p(x)$, means $p(z_0)=0$.


\item If $(x-z_0)$ is a factor of the polynomial $p(x)$, means $p(x) = q(x) \cdot (x-z_0)$ for some polynomial $q(x)$.


\item If $(z_0,0)$ is an intercept of the graph of the polynomial $p(x)$, means $(z_0,0)$ is both on the horizontal axis and on the graph of $p$.

\end{itemize}


However, the existence of any of these implies the other two






\begin{itemize}

\item If $z_0$ is a zero of the polynomial $p(x)$, then $(x-z_0)$ is a factor and $(z_0,0)$ is on the graph.


\item If $(x-z_0)$ is a factor of the polynomial $p(x)$, then $p(z_0)=0$ and $(z_0,0)$ is on the graph.


\item If $(z_0,0)$ is an intercept of the graph of the polynomial $p(x)$, then $p(z_0)=0$ and $(x-z_0)$ is a factor.

\end{itemize}













\end{document}
