\documentclass{ximera}


\graphicspath{
  {./}
  {ximeraTutorial/}
  {basicPhilosophy/}
}

\newcommand{\mooculus}{\textsf{\textbf{MOOC}\textnormal{\textsf{ULUS}}}}

\usepackage{tkz-euclide}\usepackage{tikz}
\usepackage{tikz-cd}
\usetikzlibrary{arrows}
\tikzset{>=stealth,commutative diagrams/.cd,
  arrow style=tikz,diagrams={>=stealth}} %% cool arrow head
\tikzset{shorten <>/.style={ shorten >=#1, shorten <=#1 } } %% allows shorter vectors

\usetikzlibrary{backgrounds} %% for boxes around graphs
\usetikzlibrary{shapes,positioning}  %% Clouds and stars
\usetikzlibrary{matrix} %% for matrix
\usepgfplotslibrary{polar} %% for polar plots
\usepgfplotslibrary{fillbetween} %% to shade area between curves in TikZ
\usetkzobj{all}
\usepackage[makeroom]{cancel} %% for strike outs
%\usepackage{mathtools} %% for pretty underbrace % Breaks Ximera
%\usepackage{multicol}
\usepackage{pgffor} %% required for integral for loops



%% http://tex.stackexchange.com/questions/66490/drawing-a-tikz-arc-specifying-the-center
%% Draws beach ball
\tikzset{pics/carc/.style args={#1:#2:#3}{code={\draw[pic actions] (#1:#3) arc(#1:#2:#3);}}}



\usepackage{array}
\setlength{\extrarowheight}{+.1cm}
\newdimen\digitwidth
\settowidth\digitwidth{9}
\def\divrule#1#2{
\noalign{\moveright#1\digitwidth
\vbox{\hrule width#2\digitwidth}}}






\DeclareMathOperator{\arccot}{arccot}
\DeclareMathOperator{\arcsec}{arcsec}
\DeclareMathOperator{\arccsc}{arccsc}

















%%This is to help with formatting on future title pages.
\newenvironment{sectionOutcomes}{}{}


\title{Polynomials}

\begin{document}

\begin{abstract}
characteristics
\end{abstract}
\maketitle


Polynomials are functions that can be described by sums power functions


\[    a_n x^n + a_{n-1} x^{n-1} + \cdots + a_3 x^3 + a_2 x^2 + a_1 x^1 + a_0 x^0      \]

where the $a_k$ are real numbers and $a_n \ne 0$.


\begin{itemize}

\item $n$ is called the \textbf{degree} of the polynomial.
\item $a_k$ are called \textbf{coefficients}.
\item $a_n$ is called the \textbf{leading coefficient}.
\item Each of the summands are called a \textbf{term}.  
\item $a_n x^n$ is called the \textbf{leading term}.
\item $a_2 x^2$ is called the \textbf{quadratic term}.
\item $a_1 x^1$ is called the \textbf{linear term}, often written as $a_1 x$.
\item $a_0 x^0$ is called the \textbf{constant term}, often written as $a_0$.

\end{itemize}







For analytical purposes, we would prefer a product

\[   f(x) = a (x-r_n)(x-r_{n-1})  \cdots (x-r_2)(x-r_1)  \]


\begin{itemize}

\item $a$ is called the \textbf{leading coefficient}.
\item $(x-r_k)$ are called \textbf{factors}.
\item $r_k$ are called the \textbf{zeros} or \textbf{roots}.

\end{itemize}






Polynomials are the nicest functions we have. \\

The graph of $y = f(x) = \frac{1}{100} x^3 + \frac{1}{25} x^2 - \frac{31}{100} x - \frac{7}{10}$

\begin{image}
\begin{tikzpicture} 
  \begin{axis}[
            domain=-10:10, ymax=10, xmax=10, ymin=-10, xmin=-10,
            axis lines =center, xlabel=$x$, ylabel=$y$,
            every axis y label/.style={at=(current axis.above origin),anchor=south},
            every axis x label/.style={at=(current axis.right of origin),anchor=west},
            axis on top
          ]
          
          \addplot [line width=2, penColor, smooth, domain=(-9:9),<->] {0.01*(x+7)*(x+2)*(x-5)};

           

  \end{axis}
\end{tikzpicture}
\end{image}



All of their characteristics and features are nice.

\begin{itemize}
\item Their domains include all real numbers. 
\item They are continuous everywhere.
\item No discontinuities or singularities.
\item No asymptotes on the graph.
\item Their graphs are smooth - no corners or endpoints
\end{itemize}


Polynomials smoothly alternative between increasing and decreasing and they switch at maximums and minimums.

As you will see later, polynomials have a limit on the number of zeros and extrema they can have.  They cannot have more than the degree of the polynomial. So, what happens outside all of the zeros?  This is known as the \textbf{endbehavior}.  All polynomials continue without bound after the zeros.  Either the tend to infinity or negative infinity.



\section{Factored Form}


$f(x) = a (x-r_n)(x-r_{n-1})  \cdots (x-r_2)(x-r_1)$ is the factored form of a polynomial.  In an effort to be clearer, we clean up the roots, because their could be repeated roots and we would like to point this out.  Therefore the standard factored form looks like




\[   f(x) = a (x-r_n)^{e_n} (x-r_{n-1})^{e_{n-1}}  \cdots (x-r_2)^{e_2} (x-r_1)^{e_1}  \]



In this version, the $r_k$ are distinct roots.  They are all different. \\
The $e_k$ are the exponents and tell us how many times a particular root repeats.  The exponents are sometimes referred to as the factor's \textbf{multiplicity}. \\



\textbf{\large Running the Number Line}

If we think of the real numbers as lining up to form the number line, then we could imagine ourselves running over the number line from $-\infty$ to $\infty$, or from left to right.  We would hit every real number once.  We would run through each distinct.  As we ran from the left side to the right side of each root, the sign of the value of the polynomial with either change or stay the same.

At each root. $r_k$, the value of the polynomial is $0$.  But on either side of hte root, the polynomial is either positive or negative.


\[
\begin{array}{rcl}
\text{Left of Root}  & \text{Root}  & \text{Right of Root} \\
positive & 0 & positive \\
negative & 0 & negative \\
positive & 0 & negative \\
negative & 0 & positive
\end{array}
\]


Either the sign switches or it stays the same.  What in the formula decides this?



\[   f(x) = a (x-r_n)^{e_n} (x-r_{n-1})^{e_{n-1}}  \cdots (x-r_2)^{e_2} (x-r_1)^{e_1}  \]


If we imagine ourselves running through $(x-r_k)^{e_k}$ is the only factor that could possibly change sign.  All of the other factors maintain their same sign when we run through $r_k$.


When $x<r_k$, then $x-r_k <0$.  When $x>r_k$, then $x-r_k >0$.

If $e_k$ is even then it doesn't matter if $x-r_k <0$ or $x-r_k >0$. $(x-r_k)^{e_k}$ will be positve. 

If $e_k$ is odd then it does matter, because, unlike an even power, a negative number raise to an odd power will remain negative.





\begin{itemize}
\item If $e_k$ is odd, then $(x-r_k)^{e_k}$ will change its sign.
\item If $e_k$ is even, then $(x-r_k)^{e_k}$ will not change its sign.
\end{itemize}



































\end{document}
