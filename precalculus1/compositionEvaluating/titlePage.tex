\documentclass{ximera}


\graphicspath{
  {./}
  {ximeraTutorial/}
  {basicPhilosophy/}
}

\newcommand{\mooculus}{\textsf{\textbf{MOOC}\textnormal{\textsf{ULUS}}}}

\usepackage{tkz-euclide}\usepackage{tikz}
\usepackage{tikz-cd}
\usetikzlibrary{arrows}
\tikzset{>=stealth,commutative diagrams/.cd,
  arrow style=tikz,diagrams={>=stealth}} %% cool arrow head
\tikzset{shorten <>/.style={ shorten >=#1, shorten <=#1 } } %% allows shorter vectors

\usetikzlibrary{backgrounds} %% for boxes around graphs
\usetikzlibrary{shapes,positioning}  %% Clouds and stars
\usetikzlibrary{matrix} %% for matrix
\usepgfplotslibrary{polar} %% for polar plots
\usepgfplotslibrary{fillbetween} %% to shade area between curves in TikZ
\usetkzobj{all}
\usepackage[makeroom]{cancel} %% for strike outs
%\usepackage{mathtools} %% for pretty underbrace % Breaks Ximera
%\usepackage{multicol}
\usepackage{pgffor} %% required for integral for loops



%% http://tex.stackexchange.com/questions/66490/drawing-a-tikz-arc-specifying-the-center
%% Draws beach ball
\tikzset{pics/carc/.style args={#1:#2:#3}{code={\draw[pic actions] (#1:#3) arc(#1:#2:#3);}}}



\usepackage{array}
\setlength{\extrarowheight}{+.1cm}
\newdimen\digitwidth
\settowidth\digitwidth{9}
\def\divrule#1#2{
\noalign{\moveright#1\digitwidth
\vbox{\hrule width#2\digitwidth}}}






\DeclareMathOperator{\arccot}{arccot}
\DeclareMathOperator{\arcsec}{arcsec}
\DeclareMathOperator{\arccsc}{arccsc}

















%%This is to help with formatting on future title pages.
\newenvironment{sectionOutcomes}{}{}


\title{Composition Pointwise}

\begin{document}

\begin{abstract}
%Stuff can go here later if we want!
\end{abstract}
\maketitle



\section{Objects}


\begin{quote}
Functions are packages containing three sets: a domain, a range, and a set of pairs.
\end{quote}

In this view, functions are objects.  They consist of pieces. They have features, characteristics, and attributes.  As objects, they can be examined, measured, and involved in operations.


We can also link these objects as a chain by gluing together the range of one function with the domain of another function.  In this way, we produce a new function c alled the \textbf{composition}.












\section{Machines}



We can also view functions as machines, so-called black-boxes.  They take input from the domain, process it, and return an output from the range.  For each input, the process produces exactly one output.



This output could easily be fed into another function, processed by this seocond function, and producing a final output. \\



Before chaining functions together to form new functions, we will simply follow individual numbers through the chaining process and see where they end up.


















\begin{sectionOutcomes}
After completing this section, students should 

\begin{itemize}
\item evaluate compositions pointwise graphically.

\end{itemize}
\end{sectionOutcomes}

\end{document}
