\documentclass{ximera}


\graphicspath{
  {./}
  {ximeraTutorial/}
  {basicPhilosophy/}
}

\newcommand{\mooculus}{\textsf{\textbf{MOOC}\textnormal{\textsf{ULUS}}}}

\usepackage{tkz-euclide}\usepackage{tikz}
\usepackage{tikz-cd}
\usetikzlibrary{arrows}
\tikzset{>=stealth,commutative diagrams/.cd,
  arrow style=tikz,diagrams={>=stealth}} %% cool arrow head
\tikzset{shorten <>/.style={ shorten >=#1, shorten <=#1 } } %% allows shorter vectors

\usetikzlibrary{backgrounds} %% for boxes around graphs
\usetikzlibrary{shapes,positioning}  %% Clouds and stars
\usetikzlibrary{matrix} %% for matrix
\usepgfplotslibrary{polar} %% for polar plots
\usepgfplotslibrary{fillbetween} %% to shade area between curves in TikZ
\usetkzobj{all}
\usepackage[makeroom]{cancel} %% for strike outs
%\usepackage{mathtools} %% for pretty underbrace % Breaks Ximera
%\usepackage{multicol}
\usepackage{pgffor} %% required for integral for loops



%% http://tex.stackexchange.com/questions/66490/drawing-a-tikz-arc-specifying-the-center
%% Draws beach ball
\tikzset{pics/carc/.style args={#1:#2:#3}{code={\draw[pic actions] (#1:#3) arc(#1:#2:#3);}}}



\usepackage{array}
\setlength{\extrarowheight}{+.1cm}
\newdimen\digitwidth
\settowidth\digitwidth{9}
\def\divrule#1#2{
\noalign{\moveright#1\digitwidth
\vbox{\hrule width#2\digitwidth}}}






\DeclareMathOperator{\arccot}{arccot}
\DeclareMathOperator{\arcsec}{arcsec}
\DeclareMathOperator{\arccsc}{arccsc}

















%%This is to help with formatting on future title pages.
\newenvironment{sectionOutcomes}{}{}


\title{Output Is Input}

\begin{document}

\begin{abstract}
evaluating
\end{abstract}
\maketitle











\section{Graphs}

Let $T$ be a function.  Its graph is shown here.




\begin{center}
\desmos{jukryzlv9o}{400}{300}
\end{center}




\begin{question}


Using the graph estimate the following values.


\begin{itemize}

\item $T(2) = \answer[tolerance=0.1]{-5.9}$ \\

\item $T(-6) = \answer[tolerance=0.1]{3.95}$ \\

\item $T(0) = \answer[tolerance=0.1]{3}$ \\

\end{itemize}


\end{question}



Let $f$ be a function.  Its graph is shown here. 



\begin{center}
\desmos{i02dlbl2ax}{400}{300}
\end{center}







All of the values of $T$ you determined above were real numbers. 


\begin{itemize}
\item $T(2)$ is a real number.  
\item $T(-6)$ is a real number. 
\item $T(0)$ is a real number.  
\end{itemize}


Therefore, we could evaluate $f$ at these numbers.


\begin{question}


Using the graph estimate the following values.


\begin{itemize}

\item $f(T(2)) = \answer[tolerance=0.1]{3}$ \\

\item $f(T(-6)) = \answer[tolerance=0.1]{4}$ \\

\item $f(T(0)) = \answer[tolerance=0.1]{2.75}$ \\

\end{itemize}


\end{question}










\begin{question}


Use the graphs above to approximate the following expressions.



\begin{enumerate}

\item $f(T(8)) = \answer[tolerance=0.1]{2.37}$ \\

\item $T(f(0)) = \answer[tolerance=0.1]{-1.089}$ \\

\item $f(f(-4)) = \answer[tolerance=0.1]{0.9}$ \\

\item $T(T(3)) = \answer[tolerance=0.1]{3.036}$ \\

\item $f(T(-2.5)) = \answer[tolerance=0.1]{0.874}$ \\

\item $T(f(-1))+1 = \answer[tolerance=0.1]{-6.058}$ \\

\item $f(f(-6)) -5 = \answer[tolerance=0.1]{-2.255}$ \\

\end{enumerate}



\end{question}

















\section{Formulas}


Define the functions $M$, $N$, and $P$ by the following formulas with their implied or natural domains.

$\blacktriangleright$  $M(t) = t^2 - 1$ \\

$\blacktriangleright$ $N(k) = \frac{k^2}{k + 1}$ \\

$\blacktriangleright$ $P(\theta) = 3 \, \cos(\theta) + 2$










\begin{question}


Use the formulas above to approximate the following expressions.



\begin{enumerate}

\item $N(M(1.7)) = \answer[tolerance=0.1]{1.236}$ \\

\item $M(P(0)) = \answer[tolerance=0.1]{24}$ \\

\item $N(N(8.3)) = \answer[tolerance=0.1]{6.526}$ \\

\item $P(P(-\pi)) = \answer[tolerance=0.1]{3.621}$ \\

\item $M(N(-2.5)) = \answer[tolerance=0.1]{16.361}$ \\

\item $N(M(-2.5)) = \answer[tolerance=0.1]{4.41}$ \\

\end{enumerate}



\end{question}




















\begin{center}
\textbf{\textcolor{green!50!black}{ooooo=-=-=-=-=-=-=-=-=-=-=-=-=ooOoo=-=-=-=-=-=-=-=-=-=-=-=-=ooooo}} \\

more examples can be found by following this link\\ \link[More Examples of Pointwise Composition]{https://ximera.osu.edu/csccmathematics/precalculus1/precalculus1/compositionEvaluating/examples/exampleList}

\end{center}





\end{document}
