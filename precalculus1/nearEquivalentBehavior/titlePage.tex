\documentclass{ximera}


\graphicspath{
  {./}
  {ximeraTutorial/}
  {basicPhilosophy/}
}

\newcommand{\mooculus}{\textsf{\textbf{MOOC}\textnormal{\textsf{ULUS}}}}

\usepackage{tkz-euclide}\usepackage{tikz}
\usepackage{tikz-cd}
\usetikzlibrary{arrows}
\tikzset{>=stealth,commutative diagrams/.cd,
  arrow style=tikz,diagrams={>=stealth}} %% cool arrow head
\tikzset{shorten <>/.style={ shorten >=#1, shorten <=#1 } } %% allows shorter vectors

\usetikzlibrary{backgrounds} %% for boxes around graphs
\usetikzlibrary{shapes,positioning}  %% Clouds and stars
\usetikzlibrary{matrix} %% for matrix
\usepgfplotslibrary{polar} %% for polar plots
\usepgfplotslibrary{fillbetween} %% to shade area between curves in TikZ
\usetkzobj{all}
\usepackage[makeroom]{cancel} %% for strike outs
%\usepackage{mathtools} %% for pretty underbrace % Breaks Ximera
%\usepackage{multicol}
\usepackage{pgffor} %% required for integral for loops



%% http://tex.stackexchange.com/questions/66490/drawing-a-tikz-arc-specifying-the-center
%% Draws beach ball
\tikzset{pics/carc/.style args={#1:#2:#3}{code={\draw[pic actions] (#1:#3) arc(#1:#2:#3);}}}



\usepackage{array}
\setlength{\extrarowheight}{+.1cm}
\newdimen\digitwidth
\settowidth\digitwidth{9}
\def\divrule#1#2{
\noalign{\moveright#1\digitwidth
\vbox{\hrule width#2\digitwidth}}}






\DeclareMathOperator{\arccot}{arccot}
\DeclareMathOperator{\arcsec}{arcsec}
\DeclareMathOperator{\arccsc}{arccsc}

















%%This is to help with formatting on future title pages.
\newenvironment{sectionOutcomes}{}{}


\title{Approximate Behavior}

\begin{document}

\begin{abstract}
%Stuff can go here later if we want!
\end{abstract}
\maketitle



What are we supposed to do with a function like this?

\[  C(x)   \frac{|sin(2x)+2|^x}{5(x^2+1)}       \]




\begin{image}
\begin{tikzpicture}
  \begin{axis}[
            domain=-10:10, ymax=10, xmax=10, ymin=-10, xmin=-10,
            axis lines =center, xlabel=$x$, ylabel={$y=C(x)$}, grid = major,
            ytick={-10,-8,-6,-4,-2,2,4,6,8,10},
            xtick={-10,-8,-6,-4,-2,2,4,6,8,10},
            yticklabels={$-10$,$-8$,$-6$,$-4$,$-2$,$2$,$4$,$6$,$8$,$10$}, 
            xticklabels={$-10$,$-8$,$-6$,$-4$,$-2$,$2$,$4$,$6$,$8$,$10$},
            ticklabel style={font=\scriptsize},
            every axis y label/.style={at=(current axis.above origin),anchor=south},
            every axis x label/.style={at=(current axis.right of origin),anchor=west},
            axis on top
          ]
          
          %\addplot [line width=2, penColor2, smooth,samples=100,domain=(-6:2)] {-2*x-3};
            \addplot [line width=2, penColor2, smooth,samples=300,domain=(-10:9.5)] {(abs(sin(deg(2*x))+2)^x)/(5*(x^2+1))};

          %\addplot[color=penColor,fill=penColor2,only marks,mark=*] coordinates{(-6,9)};
          %\addplot[color=penColor,fill=penColor2,only marks,mark=*] coordinates{(2,-7)};

          %\addplot[color=penColor2,fill=white,only marks,mark=*] coordinates{(2,-4.5)};
          %\addplot[color=penColor2,fill=white,only marks,mark=*] coordinates{(8,6)};


           

  \end{axis}
\end{tikzpicture}
\end{image}





The function is too complicated to consider as a whole, so we think of it in pieces.  The formula is also too difficult to manipulate with algebra.  

So, we use a replacement that does a pretty good job of approximating this function over a small interval.

And, our favorite functions are linear functions.


$C(x)$ appears to be linear-ish on the interval $(6.3, 7.3)$. Our plan is to create a linear function that does a pretty good job of approximating $C(x)$ on the interval. Graphically, that means a tangent line around $6.8$.  We need two data for this.  We need $C(6.8)$ and $C'(6.8)$.





Out linear approximation will be $Lapprox(x) = C'(6.8)(x-6.8) + C(6.8)$.

Of course, it will only be useful on $(6.3, 7.3)$, if that.









\begin{sectionOutcomes}
In this section, students will 

\begin{itemize}
\item create linear approximations.
\item accumulation via linear approximation.
\item .
\item .
\item .
\end{itemize}
\end{sectionOutcomes}

\end{document}
