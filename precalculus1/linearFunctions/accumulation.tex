\documentclass{ximera}


\graphicspath{
  {./}
  {ximeraTutorial/}
  {basicPhilosophy/}
}

\newcommand{\mooculus}{\textsf{\textbf{MOOC}\textnormal{\textsf{ULUS}}}}

\usepackage{tkz-euclide}\usepackage{tikz}
\usepackage{tikz-cd}
\usetikzlibrary{arrows}
\tikzset{>=stealth,commutative diagrams/.cd,
  arrow style=tikz,diagrams={>=stealth}} %% cool arrow head
\tikzset{shorten <>/.style={ shorten >=#1, shorten <=#1 } } %% allows shorter vectors

\usetikzlibrary{backgrounds} %% for boxes around graphs
\usetikzlibrary{shapes,positioning}  %% Clouds and stars
\usetikzlibrary{matrix} %% for matrix
\usepgfplotslibrary{polar} %% for polar plots
\usepgfplotslibrary{fillbetween} %% to shade area between curves in TikZ
\usetkzobj{all}
\usepackage[makeroom]{cancel} %% for strike outs
%\usepackage{mathtools} %% for pretty underbrace % Breaks Ximera
%\usepackage{multicol}
\usepackage{pgffor} %% required for integral for loops



%% http://tex.stackexchange.com/questions/66490/drawing-a-tikz-arc-specifying-the-center
%% Draws beach ball
\tikzset{pics/carc/.style args={#1:#2:#3}{code={\draw[pic actions] (#1:#3) arc(#1:#2:#3);}}}



\usepackage{array}
\setlength{\extrarowheight}{+.1cm}
\newdimen\digitwidth
\settowidth\digitwidth{9}
\def\divrule#1#2{
\noalign{\moveright#1\digitwidth
\vbox{\hrule width#2\digitwidth}}}






\DeclareMathOperator{\arccot}{arccot}
\DeclareMathOperator{\arcsec}{arcsec}
\DeclareMathOperator{\arccsc}{arccsc}

















%%This is to help with formatting on future title pages.
\newenvironment{sectionOutcomes}{}{}


\title{Proportional Reasoning}

\begin{document}

\begin{abstract}
similarity
\end{abstract}
\maketitle



When two quantities vary \textbf{proportionally}, then they are related by a linear function of the form $f(x) = m \, x$.  The ratio of corresponding amounts is a constant: $\frac{f(x)}{x} = m$.   

The graph is a line going through the origin with slope $m$.  This means that the ratio of change is also a constant - the same constant $\frac{\Delta f(x)}{\Delta x} = m$. 

The value of one quantity is always the same multiple of the other quantity. The change in the value of one quantity is always the same multiple as the change in the other quantity.




\begin{example}
In a simple circuit with a fixed resistor, the flow of current through the resistor, $I$, and the voltage drop across the resistor, $V$, are proportial.

\[   V = R \, I  \]

Here the rate of change is the resistance, $R$.  It is the multiplier.   When the current changes, that change is multiplied by $\answer{R}$ giving the corresponding change in voltage $V$.


\end{example}







We can also chain multiple proportions together.



\begin{example} Chaining Proportions

Water rushes through a turbine, which turns a generator to make electricity. The water posses potential energy, but not all of this energy is transmitted through the turbine.  The turbine has an 86\% efficiency. The energy coming out of the turbine is then processed through the generator at an efficiency of 92\%.

Suppose the generator produces $864 \, gigaJoules$ of energy in the form  of electricity.  What amount of energy did the water originally hold?


We have two proportions:

\begin{itemize}
\item $T = 0.86 W$
\item $G = 0.92 T$
\end{itemize} 

Combining these gives 
\[   G = 0.92 T = 0.92 \cdot 0.86 W   \]

\[   864 \, gigaJoules = 0.92 \cdot 0.86 W    \]


\[   1092 \, gigaJoules = W    \]



The water initially held $1092 \, gigaJoules$ of potential energy.

\end{example}


Written in function language, the two proportions would have looked like

\begin{itemize}
\item $T(W) = 0.86 W$
\item $G(T) = 0.92 T$
\end{itemize} 

And, the chaining would have looked like, $G(T(W)) = 0.92 (0.86 W))$.  In the function world, this is known as \textbf{composition}.






\section{Multiplicative vs Additive thinking}




If the quantities $Q_1$ and $Q_2$ are proportional then there is some constant, $m$, called the \textbf{constant of proportionality}, such that $Q_1 = m \, Q_2$.  To obtain $Q_1$ from $Q_2$, we multiply by $m$. When comparing $Q_1$ and $Q_2$, we examine their quotient, $\frac{Q_1}{Q_2}$, because it should be the same constant.  We call this \textit{mulitplicative thinking}.


In contrast are two sistes, Janet and Lisa. Janet is the older sister, by $3$ years. To get Janet's age from Lisa's age, you add $3$.  Their difference is always the same constant.






$\blacktriangleright$ Linear functions are usually neither mulitplicative nor additive thinking \\




Our template for a linear function formula looks like $L(t) = a \, t + b$. \\



\begin{itemize}

\item This relates $L$ and $t$ additively when $a = 1$.

\item When $b \ne 0$, then our function does not relate $L$ and $t$ proportionally.  If $b = 0$, then our formula looks like $L(t) = a \, t$ a - a proportion relating $L$ and $t$.


\end{itemize}


However, we have seen that we can always find an equivalent linear formula that is written as a product.  We can \textit{factor out} the zero.




The zero or root of $L(t) = a \, t + b$ is $\frac{-b}{a}$, which means $\answer{t - \frac{-b}{a}}$ is a factor.




\[
L(t) = a \, t + b = a \left( t -  \frac{-b}{a} \right) =  a \left( t + \frac{b}{a} \right)
\]



$L$ is proportional to $t -  \frac{-b}{a}$, the change from the zero.


For linear functions, the proportionality shows up not with the values of the variables, but in their changes.





\section{Rates of Change}




Let $L(x) = a \, x + b$ describe a linear function.   This equation describes the relationship between $L$ and $x$, the range and domain.


Let $x_1$ and $x_2$ be two domain numbers.  We use the Greek letter $\Delta$ to represent the change, $\Delta x = x_2 - x_1$.


If $x$ changes by $\Delta x$, then what is the corresponding change in $L$?


\begin{itemize}
\item $L(x_1) = a \, x_1 + b$
\item $L(x_2) = a \, x_2 + b$
\end{itemize}


\[
\Delta L = L(x_2) - L(x_1) = a \, x_2 + b - (a \, x_1 + b) = a \, x_2  - a \, x_1  = a \, (x_2  -  \, x_1) = a \, \Delta x
\]



\[
\Delta L =  a \, \Delta x
\]


\[
\frac{\Delta L}{\Delta x} = a
\]




Some special linear functions describe a proportional relationship between the domain and range.  This all linear functions describe a proportional relationship between the changes in domain and range values. Graphically, this proportion is seen as the slope, while algebraically, we call this proportion the rate of change.

























\end{document}
