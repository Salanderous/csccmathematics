\documentclass{ximera}


\graphicspath{
  {./}
  {ximeraTutorial/}
  {basicPhilosophy/}
}

\newcommand{\mooculus}{\textsf{\textbf{MOOC}\textnormal{\textsf{ULUS}}}}

\usepackage{tkz-euclide}\usepackage{tikz}
\usepackage{tikz-cd}
\usetikzlibrary{arrows}
\tikzset{>=stealth,commutative diagrams/.cd,
  arrow style=tikz,diagrams={>=stealth}} %% cool arrow head
\tikzset{shorten <>/.style={ shorten >=#1, shorten <=#1 } } %% allows shorter vectors

\usetikzlibrary{backgrounds} %% for boxes around graphs
\usetikzlibrary{shapes,positioning}  %% Clouds and stars
\usetikzlibrary{matrix} %% for matrix
\usepgfplotslibrary{polar} %% for polar plots
\usepgfplotslibrary{fillbetween} %% to shade area between curves in TikZ
\usetkzobj{all}
\usepackage[makeroom]{cancel} %% for strike outs
%\usepackage{mathtools} %% for pretty underbrace % Breaks Ximera
%\usepackage{multicol}
\usepackage{pgffor} %% required for integral for loops



%% http://tex.stackexchange.com/questions/66490/drawing-a-tikz-arc-specifying-the-center
%% Draws beach ball
\tikzset{pics/carc/.style args={#1:#2:#3}{code={\draw[pic actions] (#1:#3) arc(#1:#2:#3);}}}



\usepackage{array}
\setlength{\extrarowheight}{+.1cm}
\newdimen\digitwidth
\settowidth\digitwidth{9}
\def\divrule#1#2{
\noalign{\moveright#1\digitwidth
\vbox{\hrule width#2\digitwidth}}}






\DeclareMathOperator{\arccot}{arccot}
\DeclareMathOperator{\arcsec}{arcsec}
\DeclareMathOperator{\arccsc}{arccsc}

















%%This is to help with formatting on future title pages.
\newenvironment{sectionOutcomes}{}{}


\title{Accumulation}

\begin{document}

\begin{abstract}
adding rates
\end{abstract}
\maketitle





Suppose you are driving a car at a constant velocity (rate of change) of 20 miles per hour. 20 miles per hour tells us how the miles are changing compared to the change in hours.  Whenever the time changes by 1 hour, the distance travelled changes by 20 miles.

\[
\text{rate of change} = \text{velocity} = \text{20 miles per hour} = \frac{20 \, miles}{1 \, hour} = \frac{\Delta miles}{\Delta hours}
\]


The graph of velocity would be a horizontal line.



\begin{image}
\begin{tikzpicture}
  \begin{axis}[
            domain=0:10, ymax=100, xmax=10, ymin=0, xmin=0,
            axis lines =center, xlabel=$t$, ylabel=$v$, 
            ytick={0, 10, 20, 30, 40, 50, 60, 70, 80, 90},
            xtick={0, 2, 4, 6, 8, 10},
            ticklabel style={font=\scriptsize},
            every axis y label/.style={at=(current axis.above origin),anchor=south},
            every axis x label/.style={at=(current axis.right of origin),anchor=west},
            axis on top
          ]
          

			\addplot [draw=penColor, very thick, smooth, domain=(0:8),->] {20};
			\addplot[color=penColor,fill=penColor,only marks,mark=*] coordinates{(0,20)};



  \end{axis}
\end{tikzpicture}
\end{image}

In the velocity story, we know that at any time the car's distance changes by 20 miles every time the time changes by 1 hour.  The reverse of this rate of change story is an accumulation story.

Suppose we let time run for a while, how much total distance does the car travel?



\section{Accumulation}


To get the accumulated distance, we multiply the velocity by the time.

\[
Distance = Velocity \cdot Time = \frac{\Delta miles}{\Delta hours} \cdot \Delta hours
\]


We can get a geometric interpretation of this in our velocity graph.




In that graph, miles per hour are the vertical units and hours are the horizontal units.  We can represent $\frac{\Delta miles}{\Delta hours} \cdot \Delta hours$ as the area of the rectangle underneath the graph of velocity.





\begin{image}
\begin{tikzpicture}
  \begin{axis}[
            domain=0:10, ymax=100, xmax=10, ymin=-10, xmin=0,
            axis lines =center, xlabel=$t$, ylabel=$v$, 
            ytick={0, 10, 20, 30, 40, 50, 60, 70, 80, 90},
            xtick={0, 2, 4, 6, 8, 10},
            ticklabel style={font=\scriptsize},
            every axis y label/.style={at=(current axis.above origin),anchor=south},
            every axis x label/.style={at=(current axis.right of origin),anchor=west},
            axis on top
          ]
          

			\addplot [draw=penColor, very thick, smooth, domain=(0:8),->] {20};
			%\addplot[color=penColor,fill=penColor,only marks,mark=*] coordinates{(0,50)};



			\addplot [name path=A,domain=0:3,draw=none] {20};   
			\addplot [name path=B,domain=0:3,draw=none] {0};
			\addplot [fillp] fill between[of=A and B];

			\draw[penColor,thick] (0,10) -- (0,30);
			\draw[penColor,thick] (30,10) -- (30,30);

			\node at (axis cs:1.5,10) [penColor] {$area$};
			\node at (axis cs:3,-5) [penColor] {$t$};



  \end{axis}
\end{tikzpicture}
\end{image}



The area under the graph of $v(t) = 20$ over the interval $(0, t)$ is given by $20 \, t$, which is a linear function.






The graph of accumulated distance, $D(t) = 20 \, t$ is a line with slope $20$.




\begin{image}
\begin{tikzpicture}
  \begin{axis}[
            domain=0:10, ymax=100, xmax=10, ymin=-10, xmin=0,
            axis lines =center, xlabel=$t$, ylabel=$d$, 
            ytick={0, 10, 20, 30, 40, 50, 60, 70, 80, 90},
            xtick={0, 2, 4, 6, 8, 10},
            ticklabel style={font=\scriptsize},
            every axis y label/.style={at=(current axis.above origin),anchor=south},
            every axis x label/.style={at=(current axis.right of origin),anchor=west},
            axis on top
          ]
          

			\addplot [draw=penColor, very thick, smooth, domain=(0:4),->] {20*x};
			%\addplot[color=penColor,fill=penColor,only marks,mark=*] coordinates{(0,50)};









  \end{axis}
\end{tikzpicture}
\end{image}
There is a reverse relationship between rate of change and accumulation. \\


$\blacktriangleright$ The value of $v(t)$ gives the rate of change $D(t)$.


$\blacktriangleright$ The value of $D(t)$ gives the area under the graph of $v(t)$, between $0$ and $t$.






Of course, the constant function $v(t) = 50$ is also a linear function. Its rate of change is the constant function $0$.  We can view $v(t)$ as a function with a rate of change or we can view it as the rate of change of some function, $D(t)$. This function, $D(t)$, is given by the area under the graph of $v(t)$.



We can repeat this same story with $D(t)$.  That is, $D(t)$ is the rate of change of some function.  That function is the area under the graph of $D(t)$.


What is the area under $y = D(t) = 20 \, t$?









\begin{image}
\begin{tikzpicture}
  \begin{axis}[
            domain=0:10, ymax=100, xmax=10, ymin=-10, xmin=0,
            axis lines =center, xlabel=$t$, ylabel=$d$, 
            ytick={0, 10, 20, 30, 40, 50, 60, 70, 80, 90},
            xtick={0, 2, 4, 6, 8, 10},
            ticklabel style={font=\scriptsize},
            every axis y label/.style={at=(current axis.above origin),anchor=south},
            every axis x label/.style={at=(current axis.right of origin),anchor=west},
            axis on top
          ]
          

			\addplot [draw=penColor, very thick, smooth, domain=(0:4),->] {20*x};
			%\addplot[color=penColor,fill=penColor,only marks,mark=*] coordinates{(0,50)};




			\addplot [name path=A,domain=0:3,draw=none] {20*x};   
			\addplot [name path=B,domain=0:3,draw=none] {0};
			\addplot [fillp] fill between[of=A and B];

			\draw[penColor,thick] (0,10) -- (0,30);
			\draw[penColor,thick] (30,10) -- (30,70);

			\node at (axis cs:1.5,10) [penColor] {$area$};
			\node at (axis cs:3,-5) [penColor] {$t$};




  \end{axis}
\end{tikzpicture}
\end{image}



Geometrically, we can see the area under the graph forms a triangle.

\[
Area(t) = \frac{1}{2} \, Base \cdot Height = \frac{1}{2}  \, \answer{t} \cdot \left(\answer{20 \, t}\right) = 10 \, t^2
\]














\begin{image}
\begin{tikzpicture}
  \begin{axis}[
            domain=0:10, ymax=100, xmax=10, ymin=-10, xmin=0,
            axis lines =center, xlabel=$t$, ylabel=$A$, 
            ytick={0, 10, 20, 30, 40, 50, 60, 70, 80, 90},
            xtick={0, 2, 4, 6, 8, 10},
            ticklabel style={font=\scriptsize},
            every axis y label/.style={at=(current axis.above origin),anchor=south},
            every axis x label/.style={at=(current axis.right of origin),anchor=west},
            axis on top
          ]
          

			\addplot [draw=penColor, very thick, smooth, domain=(0:2),->] {20*x^2};
			%\addplot[color=penColor,fill=penColor,only marks,mark=*] coordinates{(0,50)};






  \end{axis}
\end{tikzpicture}
\end{image}













The right half of a parabola.











\begin{image}
\begin{tikzpicture}
  \begin{axis}[
            domain=-4:4, ymax=100, xmax=4, ymin=-10, xmin=-4,
            axis lines =center, xlabel=$t$, ylabel=$A$, 
            ytick={0, 10, 20, 30, 40, 50, 60, 70, 80, 90},
            xtick={-4, -2, 0, 2, 4},
            ticklabel style={font=\scriptsize},
            every axis y label/.style={at=(current axis.above origin),anchor=south},
            every axis x label/.style={at=(current axis.right of origin),anchor=west},
            axis on top
          ]
          

			\addplot [draw=penColor, very thick, smooth, domain=(0:2),->] {20*x^2};
			\addplot [draw=penColor, very thick, dashed, domain=(-2:0),<-] {20*x^2};
			%\addplot[color=penColor,fill=penColor,only marks,mark=*] coordinates{(0,50)};






  \end{axis}
\end{tikzpicture}
\end{image}












\section{Shifting}


Suppose the car travelling at 20 mph begins its journey 3 hours late.  When the car begins, our timer already reads 3 hours.



The graph of velocity would still be a horizontal line, well a ray.



\begin{image}
\begin{tikzpicture}
  \begin{axis}[
            domain=0:10, ymax=100, xmax=10, ymin=0, xmin=0,
            axis lines =center, xlabel=$t$, ylabel=$v$, 
            ytick={0, 10, 20, 30, 40, 50, 60, 70, 80, 90},
            xtick={0, 2, 4, 6, 8, 10},
            ticklabel style={font=\scriptsize},
            every axis y label/.style={at=(current axis.above origin),anchor=south},
            every axis x label/.style={at=(current axis.right of origin),anchor=west},
            axis on top
          ]
          

			\addplot [draw=penColor, very thick, smooth, domain=(3:8),->] {20};
			\addplot[color=penColor,fill=penColor,only marks,mark=*] coordinates{(3,20)};



  \end{axis}
\end{tikzpicture}
\end{image}




The area below the graph stills forms rectangles.  The left side is just at 3 now.









\begin{image}
\begin{tikzpicture}
  \begin{axis}[
            domain=0:10, ymax=100, xmax=10, ymin=-10, xmin=0,
            axis lines =center, xlabel=$t$, ylabel=$v$, 
            ytick={0, 10, 20, 30, 40, 50, 60, 70, 80, 90},
            xtick={0, 2, 4, 6, 8, 10},
            ticklabel style={font=\scriptsize},
            every axis y label/.style={at=(current axis.above origin),anchor=south},
            every axis x label/.style={at=(current axis.right of origin),anchor=west},
            axis on top
          ]
          

			\addplot [draw=penColor, very thick, smooth, domain=(3:8),->] {20};
			%\addplot[color=penColor,fill=penColor,only marks,mark=*] coordinates{(0,50)};



			\addplot [name path=A,domain=3:5,draw=none] {20};   
			\addplot [name path=B,domain=3:5,draw=none] {0};
			\addplot [fillp] fill between[of=A and B];

			\draw[penColor,thick] (30,10) -- (30,30);
			\draw[penColor,thick] (50,10) -- (50,30);

			\node at (axis cs:5,-5) [penColor] {$t$};



  \end{axis}
\end{tikzpicture}
\end{image}
The area is still $length \cdot height$.



\begin{itemize}
\item length = $t - 3$
\item height = $v(t) = 20$
	\end{itemize}




\[
Area = 20(t-3)
\]



The area under the velocity graph represents the accumulated distance travelled.






\[
D(t) = 20(t-3) \, \text{ for }  \, t \geq 3
\]








\begin{image}
\begin{tikzpicture}
  \begin{axis}[
            domain=0:10, ymax=100, xmax=10, ymin=-10, xmin=0,
            axis lines =center, xlabel=$t$, ylabel=$d$, 
            ytick={0, 10, 20, 30, 40, 50, 60, 70, 80, 90},
            xtick={0, 2, 4, 6, 8, 10},
            ticklabel style={font=\scriptsize},
            every axis y label/.style={at=(current axis.above origin),anchor=south},
            every axis x label/.style={at=(current axis.right of origin),anchor=west},
            axis on top
          ]
          

			\addplot [draw=penColor, very thick, smooth, domain=(3:7),->] {20*(x-3)};
			%\addplot[color=penColor,fill=penColor,only marks,mark=*] coordinates{(0,50)};









  \end{axis}
\end{tikzpicture}
\end{image}

The area under the graph of $D(t)$ now forms a triangle.














\begin{image}
\begin{tikzpicture}
  \begin{axis}[
            domain=0:10, ymax=100, xmax=10, ymin=-10, xmin=0,
            axis lines =center, xlabel=$t$, ylabel=$d$, 
            ytick={0, 10, 20, 30, 40, 50, 60, 70, 80, 90},
            xtick={0, 2, 4, 6, 8, 10},
            ticklabel style={font=\scriptsize},
            every axis y label/.style={at=(current axis.above origin),anchor=south},
            every axis x label/.style={at=(current axis.right of origin),anchor=west},
            axis on top
          ]
          

			\addplot [draw=penColor, very thick, smooth, domain=(3:7),->] {20*(x-3)};
			%\addplot[color=penColor,fill=penColor,only marks,mark=*] coordinates{(0,50)};




			\addplot [name path=A,domain=3:5,draw=none] {20*(x-3)};   
			\addplot [name path=B,domain=3:5,draw=none] {0};
			\addplot [fillp] fill between[of=A and B];


			\draw[penColor,thick] (50,10) -- (50,50);

			\node at (axis cs:5,-5) [penColor] {$t$};




  \end{axis}
\end{tikzpicture}
\end{image}





The area under the graph of $D(t)$ now forms a triangle, with base $t-3$ and height $D(t) = 20(t-3)$.



We now have a formula for the accumulated area under the graph of $D(t) = 20(t-3)$.


\[
A(t) = \frac{1}{2} (t-3) \cdot 20(t-3) = 20 (t-3)^2   \, \text{ for }  \, t \geq 3
\]








\begin{image}
\begin{tikzpicture}
  \begin{axis}[
            domain=0:10, ymax=100, xmax=10, ymin=-10, xmin=0,
            axis lines =center, xlabel=$t$, ylabel=$A$, 
            ytick={0, 10, 20, 30, 40, 50, 60, 70, 80, 90},
            xtick={0, 2, 4, 6, 8, 10},
            ticklabel style={font=\scriptsize},
            every axis y label/.style={at=(current axis.above origin),anchor=south},
            every axis x label/.style={at=(current axis.right of origin),anchor=west},
            axis on top
          ]
          

			\addplot [draw=penColor, very thick, smooth, domain=(3:5),->] {20*(x-3)^2};
			%\addplot[color=penColor,fill=penColor,only marks,mark=*] coordinates{(0,50)};






  \end{axis}
\end{tikzpicture}
\end{image}

Rate of change and accumulation are two sides of the same coin.  They tell the same story forwards and backwards.



There is a reverse relationship between rate of change and accumulation.


$\blacktriangleright$ The constant function $20$ gives the rate of change of the linear function $20(t-3)$.


$\blacktriangleright$ The linear function $20(t-3)$ is the accumulation function of the constant function $20$.




We have now seen that \\

$\blacktriangleright$ The quadratic function $20(t-3)^2$ is the accumulation function of the linear function $20(t-3)$. \\


$\blacktriangleright$ Claim: the linear function $20(t-3)$ gives rate of change of the quadratic function $20(t-3)^2$.





Let $A(t) = 20 (t-3)^2$


The rate of change of $A(t)$ over the interval $(3, t)$ is given by




\[
\frac{\Delta A}{\Delta t} = \frac{A(t) - A(3)}{t-3} = \frac{20 (t-3)^2}{t-3} = 20(t-3)
\]











Rate of change and accumulation are two directions of the same story.














\end{document}
