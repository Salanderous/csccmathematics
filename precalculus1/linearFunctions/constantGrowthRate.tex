\documentclass{ximera}


\graphicspath{
  {./}
  {ximeraTutorial/}
  {basicPhilosophy/}
}

\newcommand{\mooculus}{\textsf{\textbf{MOOC}\textnormal{\textsf{ULUS}}}}

\usepackage{tkz-euclide}\usepackage{tikz}
\usepackage{tikz-cd}
\usetikzlibrary{arrows}
\tikzset{>=stealth,commutative diagrams/.cd,
  arrow style=tikz,diagrams={>=stealth}} %% cool arrow head
\tikzset{shorten <>/.style={ shorten >=#1, shorten <=#1 } } %% allows shorter vectors

\usetikzlibrary{backgrounds} %% for boxes around graphs
\usetikzlibrary{shapes,positioning}  %% Clouds and stars
\usetikzlibrary{matrix} %% for matrix
\usepgfplotslibrary{polar} %% for polar plots
\usepgfplotslibrary{fillbetween} %% to shade area between curves in TikZ
\usetkzobj{all}
\usepackage[makeroom]{cancel} %% for strike outs
%\usepackage{mathtools} %% for pretty underbrace % Breaks Ximera
%\usepackage{multicol}
\usepackage{pgffor} %% required for integral for loops



%% http://tex.stackexchange.com/questions/66490/drawing-a-tikz-arc-specifying-the-center
%% Draws beach ball
\tikzset{pics/carc/.style args={#1:#2:#3}{code={\draw[pic actions] (#1:#3) arc(#1:#2:#3);}}}



\usepackage{array}
\setlength{\extrarowheight}{+.1cm}
\newdimen\digitwidth
\settowidth\digitwidth{9}
\def\divrule#1#2{
\noalign{\moveright#1\digitwidth
\vbox{\hrule width#2\digitwidth}}}






\DeclareMathOperator{\arccot}{arccot}
\DeclareMathOperator{\arcsec}{arcsec}
\DeclareMathOperator{\arccsc}{arccsc}

















%%This is to help with formatting on future title pages.
\newenvironment{sectionOutcomes}{}{}


\title{Growth Rate}

\begin{document}

\begin{abstract}
constant
\end{abstract}
\maketitle



Linear functions have a constant growth rate or rate-of-change and this has the same value as the slope of the function's graph, which is a line.


Let $f(x)$ be a linear function.  Then, $f(x)$ has a formulas of the form $f(x) = m(x-a) + f(a)$.  $m$ is called the rate-of-change of $f(x)$.

Suppose we have a Cartesian Plane displaying the graph of $f(x)$ with the vertical axis, $y$, representing $f(x)$.  Then we have $y = f(x)$, which gives us $y = m(x-a) + f(a)$, which is the equation for the plotted curve - a line.

\begin{itemize}
\item We have an algebraic object - the function $f(x)$. 
\item We have a geometric object - the line. 
\end{itemize}


This algebraic - geometric connection connects many algebraic characteristics with graphical features.








\section{Parallel}

\textbf{Parallel lines} are lines with the same slope.  They are graphs of linear functions with the same rate-of-change.


Let $f(x) = \frac{1}{2} x - 4$ and $g(x) = \frac{1}{2} x + 1$.




Below are the graphs of $y = f(x)$ and $y = g(x)$.


\begin{image}
\begin{tikzpicture}
     \begin{axis}[
            	domain=-10:10, ymax=10, xmax=10, ymin=-10, xmin=-10,
            	axis lines =center, xlabel=$x$, ylabel=$y$,
            	every axis y label/.style={at=(current axis.above origin),anchor=south},
            	every axis x label/.style={at=(current axis.right of origin),anchor=west},
            	axis on top,
          		]

        
        \addplot [draw=penColor, very thick, smooth, domain=(-8:8),<->] {0.5*x+1};
        \addplot [draw=penColor, very thick, smooth, domain=(-8:8),<->] {0.5*x-4};




    \end{axis}
\end{tikzpicture}
\end{image}



If two linear functions have the same rate-of-change, then they must differ by only a constant. 


$f(x) - g(x) = -5$ or $f(x) = g(x) - 5$. \\

Graphically, this means if the top line is shifted down $5$ units, then the lines will be the same line.










\section{Perpendicular}

\textbf{Prependicular lines} are lines that form a right angle.   Let's see what that looks like.





\begin{image}
\begin{tikzpicture}
     \begin{axis}[
            	domain=-10:10, ymax=10, xmax=10, ymin=-10, xmin=-10,
            	axis lines =center, xlabel=$x$, ylabel=$y$,
            	every axis y label/.style={at=(current axis.above origin),anchor=south},
            	every axis x label/.style={at=(current axis.right of origin),anchor=west},
            	axis on top,
          		]

        
        \addplot [draw=penColor, very thick, smooth, domain=(-8:8),<->] {x-1};
        \addplot [draw=penColor, very thick, smooth, domain=(-5:8),<->] {-x+5};
        \addplot[color=penColor,fill=penColor,only marks,mark=*] coordinates{(3,2)};
        \node at (axis cs:3.2,2) [anchor=west] {$(a, b)$};




    \end{axis}
\end{tikzpicture}
\end{image}



The equation of one line would be $y = m(x-a) + b$ and the other would have the equation $y = n(x-a) + b$, where $m$ and $n$ are their respective slopes.

Let's select another point on each line.








\begin{image}
\begin{tikzpicture}
     \begin{axis}[
            	domain=-10:10, ymax=10, xmax=10, ymin=-10, xmin=-10,
            	axis lines =center, xlabel=$x$, ylabel=$y$,
            	every axis y label/.style={at=(current axis.above origin),anchor=south},
            	every axis x label/.style={at=(current axis.right of origin),anchor=west},
            	axis on top,
          		]

        
        \addplot [draw=penColor, very thick, smooth, domain=(-8:8),<->] {x-1};
        \addplot [draw=penColor, very thick, smooth, domain=(-5:8),<->] {-x+5};
        \addplot[color=penColor,fill=penColor,only marks,mark=*] coordinates{(3,2)};
        \node at (axis cs:3.2,2) [anchor=west] {$(a, b)$};

        \addplot[color=penColor,fill=penColor,only marks,mark=*] coordinates{(4,3)};
        \node at (axis cs:0.5,5) [anchor=west] {$(a+1, m+b)$};

        \addplot[color=penColor,fill=penColor,only marks,mark=*] coordinates{(4,1)};
        \node at (axis cs:0.5,-2) [anchor=west] {$(a+1, n+b)$};




    \end{axis}
\end{tikzpicture}
\end{image}


Since we have a right angle, these the points are corners of a right triangle.  Therefore, the lengths have to satisfy the Pythagorean Theorem.  The lengths are

\begin{itemize}
\item $\sqrt{(a + 1 - a)^2 + (n + b) - b)^2} = \sqrt{(1 + n^2}$
\item $\sqrt{(a + 1 - a)^2 + (m + b) - b)^2} = \sqrt{(1 + m^2}$
\item $m-n$
\end{itemize}



The Pythagorean Theorem gives us


\begin{align*}
(\sqrt{1 + n^2})^2 + (\sqrt{1 + m^2})^2 & = (\sqrt{(m-n)})^2 \\
1 + n^2 + 1 + m^2 & = m^2 - 2 m n + n^2  \\
2 & = -2 m n \\
1 & = -m n
\frac{1}{-m} & = n
\end{align*}

The slopes are negative reciprocals of each other.



















\section{Intercepts}


ALmost every line intercepts both the horizontal and vertical axes.  These points are called the \textbf{intercepts} of the line.  

The horizontal intercept looks like $(a, 0)$, which means $f(a)=0$ and $a$ is the zero of the corresponding linear function $f(x) = m (x-a)$.

The vertical intercept looks like $(0, b)$, where $f(x) = m(x-x_0)+b$.





\textbf{Horizontal Lines} \\
Horizontal lines have equations of the form $y = y_0$, where the $y$ is the name of the vertical axis. You can view this equation as $y = 0 \cdot x + y_0$ to see that every point on this line is of the form $(x, y_0)$, making it horizontal.





\begin{example}

\begin{image}
\begin{tikzpicture}
     \begin{axis}[
            	domain=-10:10, ymax=10, xmax=10, ymin=-10, xmin=-10,
            	axis lines =center, xlabel=$x$, ylabel=$y$,
            	every axis y label/.style={at=(current axis.above origin),anchor=south},
            	every axis x label/.style={at=(current axis.right of origin),anchor=west},
            	axis on top,
          		]

        
        \addplot [draw=penColor, very thick, smooth, domain=(-8:8),<->] {3};






    \end{axis}
\end{tikzpicture}
\end{image}

The equation of this line is $y=3$.  Every point on this line is of the form $(x, 3)$. It is the graph of the constant function $f(x)=3$.


\end{example}













\textbf{Vertical Lines} \\
Vertical lines have equations of the form $x = x_0$, where the $x$ is the name of the horizontal axis. You can view this equation as $x = 0 \cdot y + x_0$ to see that every point on this line is of the form $(x_0,y)$, making it vertical.





\begin{example}

\begin{image}
\begin{tikzpicture}
     \begin{axis}[
            	domain=-10:10, ymax=10, xmax=10, ymin=-10, xmin=-10,
            	axis lines =center, xlabel=$x$, ylabel=$y$,
            	every axis y label/.style={at=(current axis.above origin),anchor=south},
            	every axis x label/.style={at=(current axis.right of origin),anchor=west},
            	axis on top,
          		]

        
        \addplot [draw=penColor, very thick, smooth, domain=(-8:8),<->] ({4},{x});






    \end{axis}
\end{tikzpicture}
\end{image}

The equation of this line is $x=4$.  Every point on this line is of the form $(4, y)$. This graph corresponds to no linear function, because the single domain number $4$ is paired with more than one codomain number.


\end{example}






























\section{Product Form}


We have $f(x) = m(x-a) + f(a)$, which is a sum and highlights the vertical intercept. However, with functions we are more interested in their \textbf{zeros}.  Zeros of a function are domain numbers where the function's value is $0$. These much easier to identify if we wil write the formula for the function as a product.


\begin{align*}
f(x) & = m(x-a) + f(a) \\
& = m \left(x - a + \frac{f(a)}{m}\right)  \\
& = m \left(x - \left(a - \frac{f(a)}{m}\right)\right) 
\end{align*}



$f$ has one zero.  It is $a - \frac{f(a)}{m}$.  It is visually encoded as the horizontal intercept, $\left(a - \frac{f(a)}{m}, 0\right)$.

































\end{document}

