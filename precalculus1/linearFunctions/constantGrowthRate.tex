\documentclass{ximera}


\graphicspath{
  {./}
  {ximeraTutorial/}
  {basicPhilosophy/}
}

\newcommand{\mooculus}{\textsf{\textbf{MOOC}\textnormal{\textsf{ULUS}}}}

\usepackage{tkz-euclide}\usepackage{tikz}
\usepackage{tikz-cd}
\usetikzlibrary{arrows}
\tikzset{>=stealth,commutative diagrams/.cd,
  arrow style=tikz,diagrams={>=stealth}} %% cool arrow head
\tikzset{shorten <>/.style={ shorten >=#1, shorten <=#1 } } %% allows shorter vectors

\usetikzlibrary{backgrounds} %% for boxes around graphs
\usetikzlibrary{shapes,positioning}  %% Clouds and stars
\usetikzlibrary{matrix} %% for matrix
\usepgfplotslibrary{polar} %% for polar plots
\usepgfplotslibrary{fillbetween} %% to shade area between curves in TikZ
\usetkzobj{all}
\usepackage[makeroom]{cancel} %% for strike outs
%\usepackage{mathtools} %% for pretty underbrace % Breaks Ximera
%\usepackage{multicol}
\usepackage{pgffor} %% required for integral for loops



%% http://tex.stackexchange.com/questions/66490/drawing-a-tikz-arc-specifying-the-center
%% Draws beach ball
\tikzset{pics/carc/.style args={#1:#2:#3}{code={\draw[pic actions] (#1:#3) arc(#1:#2:#3);}}}



\usepackage{array}
\setlength{\extrarowheight}{+.1cm}
\newdimen\digitwidth
\settowidth\digitwidth{9}
\def\divrule#1#2{
\noalign{\moveright#1\digitwidth
\vbox{\hrule width#2\digitwidth}}}






\DeclareMathOperator{\arccot}{arccot}
\DeclareMathOperator{\arcsec}{arcsec}
\DeclareMathOperator{\arccsc}{arccsc}

















%%This is to help with formatting on future title pages.
\newenvironment{sectionOutcomes}{}{}


\title{Lines}

\begin{document}

\begin{abstract}
slope
\end{abstract}
\maketitle



Linear functions have a constant growth rate or constant rate of change and this is represented graphically by the slope of the function's graph, which is a line.


\begin{definition} \textbf{\textcolor{green!50!black}{Slope}} \\


\textbf{Slope} is a measurement of the tilt of a line.


\[
slope = \frac{rise}{run} = \frac{\Delta vertical}{\Delta horizontal}
\]


If $(x_1, y_1)$ and $(x_2, y_2)$ are twopoints on a line, then the slope can be calculated as

\[
slope = \frac{rise}{run} = \frac{\Delta vertical}{\Delta horizontal} = \frac{y_2 - y_1}{x_2 - x_1}
\]



\end{definition}


$m$ is a common name for slope in equations.



$\blacktriangleright$ Slope tells us how a line is tilted.


\begin{itemize}
\item If the slope is positive, then the line slants uphill to the right.
\item If the slope is negative, then the line slants downhill to the right.
\item If the slope is zero, then the line is horizontal.
\end{itemize}


\begin{itemize}
\item Steeper lines have greater slopes.
\item Flatter lines have smaller slopes.
\end{itemize}

Suppose we have a line that contains the point $(a,b)$ and slope $m$.  Let $(x,y)$ represent any other point on the line.  Then,



\[
\frac{y-b}{x-a} = m
\]

This can be rewritten as $y - b = m(x - a)$. We call this the \textbf{point-slope} form of a line.  This can be rewritten as $y = m(x - a) + b$, which can be viewed as a formula for a function: $y(x) = m(x - a) + b$ \\


We have three views of constant rate of change.

\begin{itemize}
\item lines are graphs constructed of dots. They have a constant slope.7
\item linear equations are satisfied by the coordinates of each dot on a line.
\item linear functions are functions of constant rate of change.
\end{itemize}
















Let $f(x)$ be a linear function.  Then, $f(x)$ has a formula of the form $f(x) = m(x-a) + f(a)$.  $m$ is called the rate-of-change of $f(x)$.


Suppose we have a Cartesian Plane displaying the graph of $f(x)$ with the vertical axis, $y$, representing $f(x)$.  Then we have $y = f(x)$, which gives us $y = m(x-a) + f(a)$, which is a linear equation, which is the equation for the plotted curve - a line.

\begin{itemize}
\item We have an algebraic object - the equation, which describes the pairs or points. 
\item We have a geometric object - the line, which is a collection of points. 
\item We have an algebraic object - the function $f(x)$, which collects all of the pieces in a single package. 
\end{itemize}


This algebraic - geometric connection connects many algebraic characteristics with graphical features.








\section{Parallel}

\textbf{Parallel lines} are lines with the same slope.  They are graphs of linear functions with the same rate of change.


Let $f(x) = \frac{1}{2} x - 4$ and $g(x) = \frac{1}{2} x + 1$.




Below are the graphs of $y = f(x)$ and $y = g(x)$.


\begin{image}
\begin{tikzpicture}
     \begin{axis}[
            	domain=-10:10, ymax=10, xmax=10, ymin=-10, xmin=-10,
            	axis lines =center, xlabel=$x$, ylabel=$y$, grid = major,
                ytick={-10,-8,-6,-4,-2,2,4,6,8,10},
                xtick={-10,-8,-6,-4,-2,2,4,6,8,10},
                ticklabel style={font=\scriptsize},
            	every axis y label/.style={at=(current axis.above origin),anchor=south},
            	every axis x label/.style={at=(current axis.right of origin),anchor=west},
            	axis on top,
          		]

        
        \addplot [draw=penColor, very thick, smooth, domain=(-8:8),<->] {0.5*x+1};
        \addplot [draw=penColor, very thick, smooth, domain=(-8:8),<->] {0.5*x-4};




    \end{axis}
\end{tikzpicture}
\end{image}



If two linear functions have the same rate of change, then they must differ by only a constant. 


In the example above, $f(x) - g(x) = -5$ or $f(x) = g(x) - 5$. \\

Graphically, this means that either line can be shifted vertically by $5$ units to land on the other.










\section{Perpendicular}

\textbf{Prependicular lines} are lines that form a right angle.   


Below are the graphs of two lines, $L_1$ and $L_2$, whch intersect at the point $(a,b)$.


\begin{image}
\begin{tikzpicture}
     \begin{axis}[
            	domain=-10:10, ymax=10, xmax=10, ymin=-10, xmin=-10,
            	axis lines =center, xlabel=$x$, ylabel=$y$, 
                ytick={-10,-8,-6,-4,-2,2,4,6,8,10},
                xtick={-10,-8,-6,-4,-2,2,4,6,8,10},
                ticklabel style={font=\scriptsize},
            	every axis y label/.style={at=(current axis.above origin),anchor=south},
            	every axis x label/.style={at=(current axis.right of origin),anchor=west},
            	axis on top,
          		]

        

        \addplot [draw=black, thin, smooth, domain=(3:3.5)] {-x+6};
        \addplot [draw=black, thin, smooth, domain=(2.5:3)] {x};

        \addplot [draw=penColor, very thick, smooth, domain=(-8:8),<->] {x-1};
        \addplot [draw=penColor, very thick, smooth, domain=(-5:8),<->] {-x+5};
        \addplot[color=penColor,fill=penColor,only marks,mark=*] coordinates{(3,2)};
        \node at (axis cs:3.2,2) [anchor=west] {$(a, b)$};

        \node at (axis cs:-5,7) [anchor=west] {$L_1$};
        \node at (axis cs:-7,-5) [anchor=west] {$L_2$};




    \end{axis}
\end{tikzpicture}
\end{image}



The equation of line $L_1$ would be $y = m(x-a) + b$ and $L_2$ would have the equation $y = n(x-a) + b$, where $m$ and $n$ are their respective slopes.

Let's select another point on each line.

Each line has a point where $x = a + 1$.


$\blacktriangleright$  For $L_1$, the y-coordinate would be $y = \answer{m+b}$ \\

$\blacktriangleright$  For $L_2$, the y-coordinate would be $y = \answer{n+b}$






\begin{image}
\begin{tikzpicture}
     \begin{axis}[
            	domain=-10:10, ymax=10, xmax=10, ymin=-10, xmin=-10,
            	axis lines =center, xlabel=$x$, ylabel=$y$, 
                ytick={-10,-8,-6,-4,-2,2,4,6,8,10},
                xtick={-10,-8,-6,-4,-2,2,4,6,8,10},
                ticklabel style={font=\scriptsize},
            	every axis y label/.style={at=(current axis.above origin),anchor=south},
            	every axis x label/.style={at=(current axis.right of origin),anchor=west},
            	axis on top,
          		]

        
        \addplot [draw=penColor, very thick, smooth, domain=(-8:8),<->] {x-1};
        \addplot [draw=penColor, very thick, smooth, domain=(-5:8),<->] {-x+5};
        \addplot[color=penColor,fill=penColor,only marks,mark=*] coordinates{(3,2)};
        \node at (axis cs:2.8,2) [anchor=east] {$(a, b)$};

        \addplot[color=penColor,fill=penColor,only marks,mark=*] coordinates{(4,3)};
        \node at (axis cs:0.5,5) [anchor=west] {$(a+1, m+b)$};

        \addplot[color=penColor,fill=penColor,only marks,mark=*] coordinates{(4,1)};
        \node at (axis cs:0.5,-2) [anchor=west] {$(a+1, n+b)$};

        \node at (axis cs:-5,7) [anchor=west] {$L_1$};
        \node at (axis cs:-7,-5) [anchor=west] {$L_2$};




    \end{axis}
\end{tikzpicture}
\end{image}


Since we have a right angle, these the points are corners of a right triangle.  Therefore, the lengths have to satisfy the Pythagorean Theorem.  The lengths are

\begin{itemize}
\item lower side : $\sqrt{(a + 1 - a)^2 + (n + b) - b)^2} = \sqrt{\answer{1 + n^2}}$
\item higher side : $\sqrt{(a + 1 - a)^2 + (m + b) - b)^2} = \sqrt{\answer{1 + m^2}}$
\item hypotenuse : $\answer{m-n}$
\end{itemize}



The Pythagorean Theorem gives us


\begin{align*}
\left(\sqrt{1 + n^2}\right)^2 + \left(\sqrt{1 + m^2}\right)^2 & = \left(\sqrt{(m-n)}\right)^2 \\
1 + n^2 + 1 + m^2 & = m^2 - 2 m n + n^2  \\
2 & = -2 m n \\
1 & = -m n \\
\frac{1}{-m} & = n
\end{align*}

The slopes are negative reciprocals of each other. \\


\begin{itemize}
\item The slopes of perpendicular lines are negative reciprocals.
\item The product of the slopes of perpendicular lines equals $-1$.
\end{itemize}



















\section{Intercepts}


Almost every line intercepts both the horizontal and vertical axes.  These points are called the \textbf{intercepts} of the line.  

If we think of the line as the graph of a linbear function, $f$, then the horizontal intercept looks like $(a, 0)$, which means $f(a)=0$ and $a$ is the zero of $f$. IN this case, our formula looks like $f(x) = m (x-a)$.   $(x - a)$ is a factor of the formula.



The trio 

\begin{itemize}
\item horizontal intercepts
\item factors
\item zeros
\end{itemize}

all represent the same idea. \\



The vertical intercept looks like $(0, b)$, where $f(x) = m(x-0)+b = m \, x + b$.  This is not as valuable as the horizontal intercepts, which correspond to zeros and factors. \\










\textbf{Horizontal Lines} \\
Horizontal lines have equations of the form $y = y_0$, where the $y$ is the name of the vertical axis. You can view this equation as $y = 0 \cdot x + y_0$ to see that every point on this line is of the form $(x, y_0)$, making it horizontal.





\begin{example} Horizontal Line





The equation of this line is $w=3$.  

\begin{image}
\begin{tikzpicture}
     \begin{axis}[
            	domain=-10:10, ymax=10, xmax=10, ymin=-10, xmin=-10,
            	axis lines =center, xlabel=$t$, ylabel=$w$,
                ytick={-10,-8,-6,-4,-2,2,4,6,8,10},
                xtick={-10,-8,-6,-4,-2,2,4,6,8,10},
                ticklabel style={font=\scriptsize},
            	every axis y label/.style={at=(current axis.above origin),anchor=south},
            	every axis x label/.style={at=(current axis.right of origin),anchor=west},
            	axis on top,
          		]

        
        \addplot [draw=penColor, very thick, smooth, domain=(-8:8),<->] {3};






    \end{axis}
\end{tikzpicture}
\end{image}

Every point on this line is of the form $\left(t_0, \answer{3}\right)$. It is the graph of the constant function $f(t)=3$.


\end{example}
Horizontal lines are graphs of constant functions.  Constant functions are linear functions with a growth rate of $0$. \\












\textbf{Vertical Lines} \\


Vertical lines have equations of the form $x = x_0$, where the $x$ is the name of the horizontal axis. You can view this equation as $x = 0 \cdot y + x_0$ to see that every point on this line is of the form $(x_0,y)$, making it vertical.





\begin{example} Vertical Line


The equation of this line is $h=4$. 

\begin{image}
\begin{tikzpicture}
     \begin{axis}[
            	domain=-10:10, ymax=10, xmax=10, ymin=-10, xmin=-10,
            	axis lines =center, xlabel=$h$, ylabel=$g$, 
                ytick={-10,-8,-6,-4,-2,2,4,6,8,10},
                xtick={-10,-8,-6,-4,-2,2,4,6,8,10},
                ticklabel style={font=\scriptsize},
            	every axis y label/.style={at=(current axis.above origin),anchor=south},
            	every axis x label/.style={at=(current axis.right of origin),anchor=west},
            	axis on top,
          		]

        
        \addplot [draw=penColor, very thick, smooth, domain=(-8:8),<->] ({4},{x});






    \end{axis}
\end{tikzpicture}
\end{image}

Every point on this line is of the form $\left(\answer{4}, g_0\right)$. This graph corresponds to no linear function, because the single domain number $4$ is paired with more than one codomain number.


\end{example}

Every line is the graph of a linear function, except vertical lines. \\






























\section{Product Form}


We have many forms for writing linear equations and formulas for linear functions.



$\blacktriangleright$ Point-Slope Form

Given a point on a line, $(a, f(a))$, and the slope, $m$, we can write the formula as $f(x) = m(x-a) + f(a)$. This emphasizes the behavior around $a$.



$\blacktriangleright$ Slope-Intercept Form

If we choose our point to be the vertical intercept $(0, b)$, then the slope-intercept form looks like $f(x) = m \, x + f(0)$ or $f(x) = m \, x + b$.



These are sums.


However, our interest focuses on zeros of the function, which we would like to correspond to factors in our formulas.  As long as $m \ne 0$, we can take our sum and transform it into a product.




\begin{align*}
f(x) & = m(x-a) + f(a) \\
& = m \left(x - a + \frac{f(a)}{\answer{m}}\right)  \\
& = m \left(x - \left(a - \frac{f(a)}{m}\right)\right) 
\end{align*}



Since $f$ is linear, $f$ has one zero.  It is $a - \frac{f(a)}{m}$.  It is visually encoded as the horizontal intercept, $\left(a - \frac{f(a)}{m}, 0\right)$.


This zero will always exist as long as $m \ne 0$.



\begin{example}  Factored Form



Let $K(t) = 3t - 15$.


$5$ is the only root of $L$.  We can rewrite this formula in factored form knowing that $t-5$ has to be a factor.


$K(t) = 3(t - \answer{5})$.





\end{example}

Since $5$ is the only root of $K(t)$, we know that $\left(\answer{5}, \answer{0}\right)$ is the only $t$-intercept of the corresponding line.  In either form, the coefficient of $t$ is $\answer{3}$.  THis coefficent measures the rate of change of $K$ and the slope of the line.



























\end{document}

