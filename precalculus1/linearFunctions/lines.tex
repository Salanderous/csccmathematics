\documentclass{ximera}


\graphicspath{
  {./}
  {ximeraTutorial/}
  {basicPhilosophy/}
}

\newcommand{\mooculus}{\textsf{\textbf{MOOC}\textnormal{\textsf{ULUS}}}}

\usepackage{tkz-euclide}\usepackage{tikz}
\usepackage{tikz-cd}
\usetikzlibrary{arrows}
\tikzset{>=stealth,commutative diagrams/.cd,
  arrow style=tikz,diagrams={>=stealth}} %% cool arrow head
\tikzset{shorten <>/.style={ shorten >=#1, shorten <=#1 } } %% allows shorter vectors

\usetikzlibrary{backgrounds} %% for boxes around graphs
\usetikzlibrary{shapes,positioning}  %% Clouds and stars
\usetikzlibrary{matrix} %% for matrix
\usepgfplotslibrary{polar} %% for polar plots
\usepgfplotslibrary{fillbetween} %% to shade area between curves in TikZ
\usetkzobj{all}
\usepackage[makeroom]{cancel} %% for strike outs
%\usepackage{mathtools} %% for pretty underbrace % Breaks Ximera
%\usepackage{multicol}
\usepackage{pgffor} %% required for integral for loops



%% http://tex.stackexchange.com/questions/66490/drawing-a-tikz-arc-specifying-the-center
%% Draws beach ball
\tikzset{pics/carc/.style args={#1:#2:#3}{code={\draw[pic actions] (#1:#3) arc(#1:#2:#3);}}}



\usepackage{array}
\setlength{\extrarowheight}{+.1cm}
\newdimen\digitwidth
\settowidth\digitwidth{9}
\def\divrule#1#2{
\noalign{\moveright#1\digitwidth
\vbox{\hrule width#2\digitwidth}}}






\DeclareMathOperator{\arccot}{arccot}
\DeclareMathOperator{\arcsec}{arcsec}
\DeclareMathOperator{\arccsc}{arccsc}

















%%This is to help with formatting on future title pages.
\newenvironment{sectionOutcomes}{}{}


\title{Lines}

\begin{document}

\begin{abstract}
constant slope
\end{abstract}
\maketitle


Linear functions are those functions with a constant growth rate or rate-of-change.  If you select any two distinct numbers from the domain of a linear function and calculate the rate-of-change, you will get the same number, no matter which two domain numbers you select.

Suppose $L$ is a linear function and $a \ne b$ are two numbers in the domain of $L$. Then, 

\[    \frac{L(b) - L(a)}{b - a} = m   \]

for some constant $m$.


No matter which two numbers you select from the domain of $L$, the rate-of-change always turns out to be $m$.  Each linear function has its own constant rate-of-change.



\begin{example} \textit{Constant Rate-of-Change}


Suppose $f$ is a function, which contains the pairs $(3, 7)$, $(5, 17)$, and $(6, 23)$.


The rate-of-change from $3$ to $5$ is $\answer{5}$.

The rate-of-change from $5$ to $6$ is $\answer{6}$.

$f$ is a linear function.
\begin{multipleChoice}
\choice {True}
\choice [correct]{False}
\end{multipleChoice}


\end{example}



\section{A Formula}

Let $L$ be a linear function.  That means it has is own constant rate-of-chnage.  Let's call it $m$. \\

Let $(a, b)$ be one specific point in the domain of $L$.\\


Let $(x, y)$ represent any other pair in the funciton $L$. Then the rate-of-change from $(a,b)$ to $(x, y)$ has to equal $m$.


\[  \frac{y - b}{y-a} = m \]

However, since $(a, b)$ is a pair in $L$, we know that $b = L(a)$.  And, since $(x, y)$ is a pair in $L$, we know that $y = L(x)$.  Repacing these in the equation for constant slope gives


\[  \frac{L(x) - L(a)}{y-a} = m \]

Solving this for $L(x)$ gives

\[  L(x) = m (x-a) + L(a)     \]








\begin{example} \textit{A pair and a rate-of-change}


Suppose $W$ is a linear function with constant rate-of-change equal to $5$ and $(3, -1)$ is one pair in $W$.

Then the template: $L(x) = m (x-a) + L(a)$ tell us that a formula for $W$ is 


\[  W(x) = 5 (x-3) + (-1)     \]


\[  W(x) = 5 (x-3) - 1     \]

We could multiply this out and collect like terms and obtain the equivalent equation


\[  W(x) = 5x - 16   \]


Perhaps, we do not like $x$ as the variable for our formula.  Perhaps $v$ suits our situation better.

\[  W(v) = 5v - 16   \]

\end{example}









\begin{example} \textit{Two Pairs}


Suppose $g$ is a linear function with $(0, 6)$ and $(-2, -5)$ as two pairs in $W$.

Then the template: $L(x) = m (x-a) + L(a)$ will need a rate-of-change



\[  m = frac{-5 - 6}{-2 - 0} = \frac{-11}{-2} = \frac{11}{2}  \]

A formula for $g$ is


\[  g(t) = \frac{11}{2} (t-0) + 6     \]


\[  g(t) = \frac{11}{2} t + 6    \]



\end{example}










\section{A Graph}




A linear function has a line as its graph.  The line includes points for every pair in the function.  And, since it is a line, only two points are needed to draw the graph.  Any two points will do.





\begin{example} \textit{A Line}


Let $g(k) = 0.5k - 4$ be a linear function.  Let's select two random domain numbers: $-4$ and $6$.  The function values at these domain numbers are $g(-4) = -6$ and $g(6) = -1$.  Therefore, the points $(-4, -6)$ and $(6, -1)$ are on the graph, which is a line.  We'll plot the two points and draw a line through them.


Below is the graph of $y=g(k)$.


\begin{image}
\begin{tikzpicture}
     \begin{axis}[
            	domain=-10:10, ymax=10, xmax=10, ymin=-10, xmin=-10,
            	axis lines =center, xlabel=$k$, ylabel=$y$,
            	every axis y label/.style={at=(current axis.above origin),anchor=south},
            	every axis x label/.style={at=(current axis.right of origin),anchor=west},
            	axis on top,
          		]

        
        \addplot [draw=penColor, very thick, smooth, domain=(-8:8),<->] {0.5*x-4};

        \addplot[color=penColor,fill=penColor,only marks,mark=*] coordinates{(-4,-6)};
        \addplot[color=penColor,fill=penColor,only marks,mark=*] coordinates{(6,-1)};


    \end{axis}
\end{tikzpicture}
\end{image}



\end{example}










\begin{example} \textit{A Line}


Let $B(t)$ be a linear function.    Below is the graph of $y = B(t)$.


\begin{image}
\begin{tikzpicture}
     \begin{axis}[
            	domain=-10:10, ymax=10, xmax=10, ymin=-10, xmin=-10,
            	axis lines =center, xlabel=$t$, ylabel=$y$,
            	every axis y label/.style={at=(current axis.above origin),anchor=south},
            	every axis x label/.style={at=(current axis.right of origin),anchor=west},
            	axis on top,
          		]

        
        \addplot [draw=penColor, very thick, smooth, domain=(-3:8),<->] {(3/2)*x-5};

        %\addplot[color=penColor,fill=penColor,only marks,mark=*] coordinates{(-4,-6)};
        %\addplot[color=penColor,fill=penColor,only marks,mark=*] coordinates{(6,-1)};


    \end{axis}
\end{tikzpicture}
\end{image}



From the graph we can approximate the points $(0, -5)$ and $(3.3, 0)$.  These give a slope of

\[  slope = \frac{0 - (-5)}{3.3 - 0} \approx 1.5     \]


This would give the formula $B(t) = 1.5 (t - 0) - 5 = 1.5 t - 5$


\end{example}
























\end{document}
