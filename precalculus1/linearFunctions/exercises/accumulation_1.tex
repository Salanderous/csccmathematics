\documentclass{ximera}


\graphicspath{
  {./}
  {ximeraTutorial/}
  {basicPhilosophy/}
}

\newcommand{\mooculus}{\textsf{\textbf{MOOC}\textnormal{\textsf{ULUS}}}}

\usepackage{tkz-euclide}\usepackage{tikz}
\usepackage{tikz-cd}
\usetikzlibrary{arrows}
\tikzset{>=stealth,commutative diagrams/.cd,
  arrow style=tikz,diagrams={>=stealth}} %% cool arrow head
\tikzset{shorten <>/.style={ shorten >=#1, shorten <=#1 } } %% allows shorter vectors

\usetikzlibrary{backgrounds} %% for boxes around graphs
\usetikzlibrary{shapes,positioning}  %% Clouds and stars
\usetikzlibrary{matrix} %% for matrix
\usepgfplotslibrary{polar} %% for polar plots
\usepgfplotslibrary{fillbetween} %% to shade area between curves in TikZ
\usetkzobj{all}
\usepackage[makeroom]{cancel} %% for strike outs
%\usepackage{mathtools} %% for pretty underbrace % Breaks Ximera
%\usepackage{multicol}
\usepackage{pgffor} %% required for integral for loops



%% http://tex.stackexchange.com/questions/66490/drawing-a-tikz-arc-specifying-the-center
%% Draws beach ball
\tikzset{pics/carc/.style args={#1:#2:#3}{code={\draw[pic actions] (#1:#3) arc(#1:#2:#3);}}}



\usepackage{array}
\setlength{\extrarowheight}{+.1cm}
\newdimen\digitwidth
\settowidth\digitwidth{9}
\def\divrule#1#2{
\noalign{\moveright#1\digitwidth
\vbox{\hrule width#2\digitwidth}}}






\DeclareMathOperator{\arccot}{arccot}
\DeclareMathOperator{\arcsec}{arcsec}
\DeclareMathOperator{\arccsc}{arccsc}

















%%This is to help with formatting on future title pages.
\newenvironment{sectionOutcomes}{}{}


\outcome{outcome.}
\outcome{outcome.}
\outcome{outcome.}

\author{Lee Wayand}

\begin{document}





\begin{exercise} Area Functions \\




\begin{image}
\begin{tikzpicture}
  \begin{axis}[
            domain=-10:10, ymax=10, xmax=10, ymin=-10, xmin=-10,
            axis lines =center, xlabel=$t$, ylabel=$v$, 
            ytick={-10,-8,-6,-4,-2,2,4,6,8,10},
            xtick={-10,-8,-6,-4,-2,2,4,6,8,10},
            ticklabel style={font=\scriptsize},
            every axis y label/.style={at=(current axis.above origin),anchor=south},
            every axis x label/.style={at=(current axis.right of origin),anchor=west},
            axis on top
          ]
          

			\addplot [draw=penColor, very thick, smooth, domain=(0:8),->] {7};
			%\addplot[color=penColor,fill=penColor,only marks,mark=*] coordinates{(0,50)};



			\addplot [name path=A,domain=0:5,draw=none] {7};   
			\addplot [name path=B,domain=0:5,draw=none] {0};
			\addplot [fillp] fill between[of=A and B];

			\node at (axis cs:2,4) [penColor] {$area$};
			\node at (axis cs:5,-2) [penColor] {$T$};



  \end{axis}
\end{tikzpicture}
\end{image}



The area underneath the graphs of $v = 7$, above the $t$-axis, and between $t=0$ and $t=T$ is given by

\[
Area(T) = \answer{7T}
\]


\end{exercise}















\begin{exercise} Area Functions \\




\begin{image}
\begin{tikzpicture}
  \begin{axis}[
            domain=-10:10, ymax=10, xmax=10, ymin=-10, xmin=-10,
            axis lines =center, xlabel=$w$, ylabel=$y$, 
            ytick={-10,-8,-6,-4,-2,2,4,6,8,10},
            xtick={-10,-8,-6,-4,-2,2,4,6,8,10},
            ticklabel style={font=\scriptsize},
            every axis y label/.style={at=(current axis.above origin),anchor=south},
            every axis x label/.style={at=(current axis.right of origin),anchor=west},
            axis on top
          ]
          

			\addplot [draw=penColor, very thick, smooth, domain=(2:8),->] {6};
			\addplot[color=penColor,fill=penColor,only marks,mark=*] coordinates{(2,6)};



			\addplot [name path=A,domain=2:5,draw=none] {6};   
			\addplot [name path=B,domain=2:5,draw=none] {0};
			\addplot [fillp] fill between[of=A and B];

			%\node at (axis cs:4,3) [penColor] {$area$};
			\node at (axis cs:5,-2) [penColor] {$T$};



  \end{axis}
\end{tikzpicture}
\end{image}



The area underneath the graphs of $y = 6$, above the $w$-axis, and between $t=2$ and $t=T$ is given by

\[
Area(T) = \answer{6(T-2)}
\]




The graph of $a = Area(T)$ is a ray (a line with a restricted domain) with slope $\answer{6}$.




\end{exercise}















\end{document}