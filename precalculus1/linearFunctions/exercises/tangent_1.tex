\documentclass{ximera}


\graphicspath{
  {./}
  {ximeraTutorial/}
  {basicPhilosophy/}
}

\newcommand{\mooculus}{\textsf{\textbf{MOOC}\textnormal{\textsf{ULUS}}}}

\usepackage{tkz-euclide}\usepackage{tikz}
\usepackage{tikz-cd}
\usetikzlibrary{arrows}
\tikzset{>=stealth,commutative diagrams/.cd,
  arrow style=tikz,diagrams={>=stealth}} %% cool arrow head
\tikzset{shorten <>/.style={ shorten >=#1, shorten <=#1 } } %% allows shorter vectors

\usetikzlibrary{backgrounds} %% for boxes around graphs
\usetikzlibrary{shapes,positioning}  %% Clouds and stars
\usetikzlibrary{matrix} %% for matrix
\usepgfplotslibrary{polar} %% for polar plots
\usepgfplotslibrary{fillbetween} %% to shade area between curves in TikZ
\usetkzobj{all}
\usepackage[makeroom]{cancel} %% for strike outs
%\usepackage{mathtools} %% for pretty underbrace % Breaks Ximera
%\usepackage{multicol}
\usepackage{pgffor} %% required for integral for loops



%% http://tex.stackexchange.com/questions/66490/drawing-a-tikz-arc-specifying-the-center
%% Draws beach ball
\tikzset{pics/carc/.style args={#1:#2:#3}{code={\draw[pic actions] (#1:#3) arc(#1:#2:#3);}}}



\usepackage{array}
\setlength{\extrarowheight}{+.1cm}
\newdimen\digitwidth
\settowidth\digitwidth{9}
\def\divrule#1#2{
\noalign{\moveright#1\digitwidth
\vbox{\hrule width#2\digitwidth}}}






\DeclareMathOperator{\arccot}{arccot}
\DeclareMathOperator{\arcsec}{arcsec}
\DeclareMathOperator{\arccsc}{arccsc}

















%%This is to help with formatting on future title pages.
\newenvironment{sectionOutcomes}{}{}


\outcome{outcome.}
\outcome{outcome.}
\outcome{outcome.}

\author{Lee Wayand}

\begin{document}





Below is the graph of $m = M(w) = \frac{1}{10} (w+6) (w+1) (w-3)$ with is natural domain.  The tangent line at the point $\left( 2, -\frac{24}{10} \right)$ is also drawn.


\begin{image}
\begin{tikzpicture} 
  \begin{axis}[
            domain=-10:10, ymax=10, xmax=10, ymin=-10, xmin=-10,
            axis lines =center, xlabel=$w$, ylabel=$m$,
            ytick={-10,-8,-6,-4,-2,2,4,6,8,10},
            xtick={-10,-8,-6,-4,-2,2,4,6,8,10},
            ticklabel style={font=\scriptsize},
            every axis y label/.style={at=(current axis.above origin),anchor=south},
            every axis x label/.style={at=(current axis.right of origin),anchor=west},
            axis on top
          ]
          
          \addplot [line width=2, penColor, smooth, samples=200, domain=(-7:4.5),<->] {0.1*(x+6)*(x+1)*(x-3)};
          \addplot [line width=2, penColor2, smooth, samples=200, domain=(-4:8),<->] {1.3*(x-2)-2.4};


          \addplot[color=penColor,fill=penColor,only marks,mark=*] coordinates{(2,-2.4)};
          %\addplot[color=penColor,fill=penColor,only marks,mark=*] coordinates{(2,-2)};


           

  \end{axis}
\end{tikzpicture}
\end{image}





\begin{exercise} Equation

The tangent line above goes through the point $\left( 2, -\frac{24}{10} \right)$ with a slope of $\frac{13}{10}$. \\





Create an equation for this line:  $m = \answer{\frac{13}{10} w-\frac{50}{10}}$. \\

The $w$-intercept of the tangent line is $\left( \answer{\frac{50}{13}}, \answer{0} \right)$


\end{exercise}




\begin{exercise} Function

The line above is the graph of a linear function called $H$.  \\
 
Create a formula for this function:  $H(w) = \answer{\frac{13}{10} w-\frac{50}{10}}$. \\

The of $H$ is $\answer{\frac{50}{13}}$

\end{exercise}










\begin{exercise} Intersection

It appears from the graph that if we extend both graphs to the left, they will intersect at a second intersection point.  This second intersection point is on both graphs, which means that the two functions have the same value at the corresponding domain number.


\[
\frac{1}{10} (w+6) (w+1) (w-3) = \frac{13}{10} (w-2) -\frac{24}{10}
\]



\[
(w+6) (w+1) (w-3) = 13 (w-2) - 24
\]



\[
w^3 + \answer{4} w^2 - \answer{15} w - \answer{18} = 13 w - 50
\]


\[
w^3 + 4 w^2 - \answer{28} w + \answer{32} = 0
\]


We know that $2$ is a solution to this, because the tangent point is an intersection point where the two functions would be equal.  Therefore, $w-2$ must be a factor.  The other factor must be a quadratic, since together their product is a cubic.



\[
w^3 + 4 w^2 - \answer{28} w + \answer{32} = (w - 2) \cdot (A \, w^2 + B \, w + C)
\]


Let's multiply out the left side and equate coefficients.


\[
w^3 + 4 w^2 - \answer{28} w + \answer{32} = (w - 2) \cdot (A \, w^2 + B \, w + C)
\]


\[
w^3 + 4 w^2 - \answer{28} w + \answer{32} = A \, w^3 + (B - 2A) \, w^2 + (C - 2B) \, w - 2C
\]


From this, we can see that   $A = \answer{1}$ and $C = \answer{-16}$.

Substituting these value in gives us



\[
w^3 + 4 w^2 - \answer{28} w + \answer{32} = w^3 + (B - 2) \, w^2 + (-16 - 2B) \, w + 32
\]


Comparing the $w^2$-term, we see that $B = \answer{6}$.






\[
w^3 + 4 w^2 - \answer{28} w + \answer{32} = (w - 2) \cdot (w^2 + 6 \, w - 16)
\]


This quadratic factors.



\[
w^3 + 4 w^2 - \answer{28} w + \answer{32} = (w - 2) \left( w - \answer{2} \right) \left( w + \answer{8} \right)
\]



The solution that corresonds to our second intersection point is $\answer{-8}$. \\

The other intersection point is $\left( -8, -\frac{154}{10} \right)$.



\end{exercise}


\textbf{Note:}  The tangent point corresponded to a double root and a double factor - a factor with a multiplcity (exponent) of $2$.







\end{document}