\documentclass{ximera}


\graphicspath{
  {./}
  {ximeraTutorial/}
  {basicPhilosophy/}
}

\newcommand{\mooculus}{\textsf{\textbf{MOOC}\textnormal{\textsf{ULUS}}}}

\usepackage{tkz-euclide}\usepackage{tikz}
\usepackage{tikz-cd}
\usetikzlibrary{arrows}
\tikzset{>=stealth,commutative diagrams/.cd,
  arrow style=tikz,diagrams={>=stealth}} %% cool arrow head
\tikzset{shorten <>/.style={ shorten >=#1, shorten <=#1 } } %% allows shorter vectors

\usetikzlibrary{backgrounds} %% for boxes around graphs
\usetikzlibrary{shapes,positioning}  %% Clouds and stars
\usetikzlibrary{matrix} %% for matrix
\usepgfplotslibrary{polar} %% for polar plots
\usepgfplotslibrary{fillbetween} %% to shade area between curves in TikZ
\usetkzobj{all}
\usepackage[makeroom]{cancel} %% for strike outs
%\usepackage{mathtools} %% for pretty underbrace % Breaks Ximera
%\usepackage{multicol}
\usepackage{pgffor} %% required for integral for loops



%% http://tex.stackexchange.com/questions/66490/drawing-a-tikz-arc-specifying-the-center
%% Draws beach ball
\tikzset{pics/carc/.style args={#1:#2:#3}{code={\draw[pic actions] (#1:#3) arc(#1:#2:#3);}}}



\usepackage{array}
\setlength{\extrarowheight}{+.1cm}
\newdimen\digitwidth
\settowidth\digitwidth{9}
\def\divrule#1#2{
\noalign{\moveright#1\digitwidth
\vbox{\hrule width#2\digitwidth}}}






\DeclareMathOperator{\arccot}{arccot}
\DeclareMathOperator{\arcsec}{arcsec}
\DeclareMathOperator{\arccsc}{arccsc}

















%%This is to help with formatting on future title pages.
\newenvironment{sectionOutcomes}{}{}


\outcome{outcome.}
\outcome{outcome.}
\outcome{outcome.}

\author{Lee Wayand}

\begin{document}





Below is the graph of $m = M(w) = \frac{1}{10} (w+6) (w+1) (w-3)$ with is natural domain.  The tangent line at the point $\left( 2, -\frac{24}{10} \right)$ is also drawn.


\begin{image}
\begin{tikzpicture} 
  \begin{axis}[
            domain=-10:10, ymax=10, xmax=10, ymin=-10, xmin=-10,
            axis lines =center, xlabel=$w$, ylabel=$m$,
            ytick={-10,-8,-6,-4,-2,2,4,6,8,10},
            xtick={-10,-8,-6,-4,-2,2,4,6,8,10},
            ticklabel style={font=\scriptsize},
            every axis y label/.style={at=(current axis.above origin),anchor=south},
            every axis x label/.style={at=(current axis.right of origin),anchor=west},
            axis on top
          ]
          
          \addplot [line width=2, penColor, smooth, samples=200, domain=(-7:4.5),<->] {0.1*(x+6)*(x+1)*(x-3)};
          \addplot [line width=2, penColor2, smooth, samples=200, domain=(-4:8),<->] {1.3*(x-2)-2.4};


          \addplot[color=penColor,fill=penColor,only marks,mark=*] coordinates{(2,-2.4)};
          %\addplot[color=penColor,fill=penColor,only marks,mark=*] coordinates{(2,-2)};

           


           

  \end{axis}
\end{tikzpicture}
\end{image}










We could draw tangent lines at lots of points on the curve.




\begin{image}
\begin{tikzpicture} 
  \begin{axis}[
            domain=-10:10, ymax=10, xmax=10, ymin=-10, xmin=-10,
            axis lines =center, xlabel=$w$, ylabel=$m$,
            ytick={-10,-8,-6,-4,-2,2,4,6,8,10},
            xtick={-10,-8,-6,-4,-2,2,4,6,8,10},
            ticklabel style={font=\scriptsize},
            every axis y label/.style={at=(current axis.above origin),anchor=south},
            every axis x label/.style={at=(current axis.right of origin),anchor=west},
            axis on top
          ]
          
          \addplot [line width=2, penColor, smooth, samples=200, domain=(-7:4.5),<->] {0.1*(x+6)*(x+1)*(x-3)};
          %\addplot [line width=2, penColor2, smooth, samples=200, domain=(-4:8),<->] {1.3*(x-2)-2.4};


          \addplot[color=penColor,fill=penColor,only marks,mark=*] coordinates{(2,-2.4)};
          \addplot[color=penColor,fill=penColor,only marks,mark=*] coordinates{(-6,0) (-3,3.6) (0,-1.8) (4,5)};

          \node at (axis cs:-6.2,0.7) [anchor=east] {$A$};
          \node at (axis cs:-2.7,4) [anchor=south] {$B$};
          \node at (axis cs:0.5,-1.7) [anchor=south] {$C$};
          \node at (axis cs:2,-3) [anchor=west] {$D$};
          \node at (axis cs:4.2,5) [anchor=west] {$E$};


           

  \end{axis}
\end{tikzpicture}
\end{image}










\begin{exercise} Slopes


The tangent line at point A would have a \wordChoice{\choice{negative} \choice{zero} \choice[correct]{positive}} slope. \\

The tangent line at point B would have a \wordChoice{\choice[correct]{negative} \choice{zero} \choice{positive}} slope. \\

The tangent line at point C would have a \wordChoice{\choice[correct]{negative} \choice{zero} \choice{positive}} slope. \\

The tangent line at point D would have a \wordChoice{\choice{negative} \choice{zero} \choice[correct]{positive}} slope. \\

The tangent line at point E would have a \wordChoice{\choice{negative} \choice{zero} \choice[correct]{positive}} slope. \\




\end{exercise}
\






\end{document}