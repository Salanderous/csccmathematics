\documentclass{ximera}


\graphicspath{
  {./}
  {ximeraTutorial/}
  {basicPhilosophy/}
}

\newcommand{\mooculus}{\textsf{\textbf{MOOC}\textnormal{\textsf{ULUS}}}}

\usepackage{tkz-euclide}\usepackage{tikz}
\usepackage{tikz-cd}
\usetikzlibrary{arrows}
\tikzset{>=stealth,commutative diagrams/.cd,
  arrow style=tikz,diagrams={>=stealth}} %% cool arrow head
\tikzset{shorten <>/.style={ shorten >=#1, shorten <=#1 } } %% allows shorter vectors

\usetikzlibrary{backgrounds} %% for boxes around graphs
\usetikzlibrary{shapes,positioning}  %% Clouds and stars
\usetikzlibrary{matrix} %% for matrix
\usepgfplotslibrary{polar} %% for polar plots
\usepgfplotslibrary{fillbetween} %% to shade area between curves in TikZ
\usetkzobj{all}
\usepackage[makeroom]{cancel} %% for strike outs
%\usepackage{mathtools} %% for pretty underbrace % Breaks Ximera
%\usepackage{multicol}
\usepackage{pgffor} %% required for integral for loops



%% http://tex.stackexchange.com/questions/66490/drawing-a-tikz-arc-specifying-the-center
%% Draws beach ball
\tikzset{pics/carc/.style args={#1:#2:#3}{code={\draw[pic actions] (#1:#3) arc(#1:#2:#3);}}}



\usepackage{array}
\setlength{\extrarowheight}{+.1cm}
\newdimen\digitwidth
\settowidth\digitwidth{9}
\def\divrule#1#2{
\noalign{\moveright#1\digitwidth
\vbox{\hrule width#2\digitwidth}}}






\DeclareMathOperator{\arccot}{arccot}
\DeclareMathOperator{\arcsec}{arcsec}
\DeclareMathOperator{\arccsc}{arccsc}

















%%This is to help with formatting on future title pages.
\newenvironment{sectionOutcomes}{}{}


\title{Shifting Domain}

\begin{document}

\begin{abstract}
same characteristics
\end{abstract}
\maketitle








Below is the piecewise defined function, $T(v)$.  $v$ is representing the domain values from $(-4,-1] \cup [1,7)$.




\[
T(v) = 
\begin{cases}
  2v-1 & \text{ if }  -4 < v \leq -1 \\
  -v+3 & \text{ if } 1 \leq v < 7
\end{cases}
\]


On the interval $(-4, -1]$, the graph should be a line for $T(v) = 2v-1$. On the interval $[1, 7)$, the graph should be another line for $T(v) = -v+3$




Graph of $y = T(v)$.
\begin{image}
\begin{tikzpicture}
	\begin{axis}[
            domain=-10:10, ymax=10, xmax=10, ymin=-10, xmin=-10,
            axis lines =center, xlabel=$v$, ylabel=$y$,
            every axis y label/.style={at=(current axis.above origin),anchor=south},
            every axis x label/.style={at=(current axis.right of origin),anchor=west},
            axis on top
          ]
          
	\addplot [draw=penColor,very thick,smooth,domain=(-4:-1)] {2*x-1};
	\addplot [draw=penColor,very thick,smooth,domain=(1:7)] {-x+3};
	\addplot[color=penColor,only marks,mark=*] coordinates{(-1,-3)}; 
	\addplot[color=penColor,fill=white,only marks,mark=*] coordinates{(-4,-9)}; 
	\addplot[color=penColor,only marks,mark=*] coordinates{(1,2)}; 
	\addplot[color=penColor,fill=white,only marks,mark=*] coordinates{(7,-4)}; 


    \end{axis}
\end{tikzpicture}
\end{image}







The domain of $T$ has two maximal intervals:, $(-4,-1]$ and $[1,7)$.  These correspond to two line segments on the graph. The endpoints give us four important points on the graph: 

\begin{itemize}

\item $(-4, -9)$, which is an open point on the graph.
\item $(-1, -3)$, which is a closed point on the graph.
\item $(1, 2)$, which is a closed point on the graph.
\item $(7, -4)$, which is an open point on the graph.

\end{itemize}




\section{A New Function}

$B(k)$ is a new function, but its definition is based on $T(v)$.


$B(k) = T(k+3)$ with the implied domain.


\textbf{First Question:} What is the domain of $B$?

$k$ is representing the values of the domain of $B$ and $k+3$ is now representing the domain values of $T$.  Therefore $k+3$ must take on the values in $(-4,-1] \cup [1,7)$.

\[     k+3 \in      (-4,-1] \cup [1,7)     \]


$k$ must be in a simialr set, but shifted tothe left by $3$.


\[     k \in      (-7,-4] \cup [-2,4)     \]


The domain of $B$ is $(-7,-4] \cup [-2,4)$.   It was implied from the domain of $T$ and the definieiotn of $B$, which is using $T$.  The values in the domain of $B$ are real numbers, $k$, such that $k+3$ is in the domain of $T$.

The domain has shifted. 


But, the structure of the function hasn't changed.  The same formulas are in use.  They are just used by different domain numbers.



\begin{itemize}

\item When $k \in (-7,-4]$, then $k+3 \in (-4,-1]$.  $v=k+3$, so $v \in (-4,-1]$ and we are using the formula $2v-1$ with $v=k+3$.
All together, when $k \in (-7,-4]$, then $B(k) = 2(k+3)-1 = 2k+5$.

\item When $k \in [-2,4)$, then $k+3 \in [1,7)$.  $v=k+3$, so $v \in [1,7)$ and we are using the formula $-v+3$ with $v=k+3$.
All together, when $k \in [-2,4)$, then $B(k) = -(k+3)+3 = -k$.


\end{itemize}









\[
B(k) = 
\begin{cases}
  2k+5 & \text{ if }  -7 < k \leq -4 \\
  -k & \text{ if } -2 \leq k < 4
\end{cases}
\]





Graph of $z = B(k)$.

\begin{image}
\begin{tikzpicture}
	\begin{axis}[
            domain=-10:10, ymax=10, xmax=10, ymin=-10, xmin=-10,
            axis lines =center, xlabel=$k$, ylabel=$z$,
            every axis y label/.style={at=(current axis.above origin),anchor=south},
            every axis x label/.style={at=(current axis.right of origin),anchor=west},
            axis on top
          ]
          
	\addplot [draw=penColor,very thick,smooth,domain=(-7:-4)] {2*x+5};
	\addplot [draw=penColor,very thick,smooth,domain=(-2:4)] {-x};
	\addplot[color=penColor,only marks,mark=*] coordinates{(-4,-3)}; 
	\addplot[color=penColor,fill=white,only marks,mark=*] coordinates{(-7,-9)}; 
	\addplot[color=penColor,only marks,mark=*] coordinates{(-2,2)}; 
	\addplot[color=penColor,fill=white,only marks,mark=*] coordinates{(4,-4)}; 


    \end{axis}
\end{tikzpicture}
\end{image}






When we look at the graphs side-by-side, we can see that the graph has simply shifted left $3$.








\begin{image}
\begin{tikzpicture}
	\begin{axis}[name = leftgraph,
            domain=-10:10, ymax=10, xmax=10, ymin=-10, xmin=-10,
            axis lines =center, xlabel=$v$, ylabel={$y=T(v)$},
            every axis y label/.style={at=(current axis.above origin),anchor=south},
            every axis x label/.style={at=(current axis.right of origin),anchor=west},
            axis on top
          ]
          
	\addplot [draw=penColor,very thick,smooth,domain=(-4:-1)] {2*x-1};
	\addplot [draw=penColor,very thick,smooth,domain=(1:7)] {-x+3};
	\addplot[color=penColor,only marks,mark=*] coordinates{(-1,-3)}; 
	\addplot[color=penColor,fill=white,only marks,mark=*] coordinates{(-4,-9)}; 
	\addplot[color=penColor,only marks,mark=*] coordinates{(1,2)}; 
	\addplot[color=penColor,fill=white,only marks,mark=*] coordinates{(7,-4)}; 


    \end{axis}
	\begin{axis}[at={(leftgraph.outer east)},anchor=outer west, 
            domain=-10:10, ymax=10, xmax=10, ymin=-10, xmin=-10,
            axis lines =center, xlabel=$k$, ylabel={$z=B(k)$},
            every axis y label/.style={at=(current axis.above origin),anchor=south},
            every axis x label/.style={at=(current axis.right of origin),anchor=west},
            axis on top
          ]
          
	\addplot [draw=penColor,very thick,smooth,domain=(-7:-4)] {2*x+5};
	\addplot [draw=penColor,very thick,smooth,domain=(-2:4)] {-x};
	\addplot[color=penColor,only marks,mark=*] coordinates{(-4,-3)}; 
	\addplot[color=penColor,fill=white,only marks,mark=*] coordinates{(-7,-9)}; 
	\addplot[color=penColor,only marks,mark=*] coordinates{(-2,2)}; 
	\addplot[color=penColor,fill=white,only marks,mark=*] coordinates{(4,-4)}; 


    \end{axis}


\end{tikzpicture}
\end{image}




From the definition, $B(k) = T(k+3)$, we can see that $v=k+3$, or $k=v-3$.  All of the $k$-values are $3$ less than the $v$-values.  $B$ is $T$ shifted left $3$.






\begin{example} Shifting the Domain 






\[
V(h) = 
\begin{cases}
  -2h-3 & \text{ if } [-6, -2]   \\
  -(h+3)(h-3) & \text{ if } (-2, 4]  \\
  \frac{7h}{4} - 8 & \text{ if } (4,6)
\end{cases}
\]



A graph always helps our thinking. Here is theg raph of $y = V(h)$.








\begin{image}
\begin{tikzpicture} 
  \begin{axis}[
            domain=-10:10, ymax=10, xmax=10, ymin=-10, xmin=-10,
            axis lines =center, xlabel=$h$, ylabel=$y$,
            every axis y label/.style={at=(current axis.above origin),anchor=south},
            every axis x label/.style={at=(current axis.right of origin),anchor=west},
            axis on top
          ]
          
      \addplot [line width=2, penColor, smooth,samples=100,domain=(-6:-2)] {-2*x-3};
          \addplot [line width=2, penColor, smooth,samples=100,domain=(-2:4)] {-1*(x+3)*(x-3))};
          \addplot [line width=2, penColor, smooth,samples=100,domain=(4:6)] {1.75*x-8};




      \addplot[color=penColor,fill=penColor,only marks,mark=*] coordinates{(-6,9)};
      \addplot[color=penColor,fill=penColor,only marks,mark=*] coordinates{(-2,1)};

      \addplot[color=penColor,fill=white,only marks,mark=*] coordinates{(-2,5)};
      \addplot[color=penColor,fill=penColor,only marks,mark=*] coordinates{(4,-7)};

      \addplot[color=penColor,fill=white,only marks,mark=*] coordinates{(4,-1)};
      \addplot[color=penColor,fill=white,only marks,mark=*] coordinates{(6,2.5)};


           

  \end{axis}
\end{tikzpicture}
\end{image}





Let $f(x) = V(x-1)$ with the induced domain.

The domain of $V$ is $[-6, 6)$.  These values are represented by $x-1$ in this definition, because $x-1$ is the expression inside the parentheses for $V$ and that is where the expression lives for the domain values of $V$.

However, $x$ represents the domain values of $f$.   If $x - 1 \in [-6, 6)$, then $x \in [-5, 7)$.  This is the domain for $f$.  It is shifted up $1$ or to the right $1$.  The individual maximal intervals become $-5, -1]$, $(-1, 5]$, and $(5, 7)$.





\[
f(x) = V(x-1) = 
\begin{cases}
  -2(x-1)-3 & \text{ if } [-5, -1]   \\
  -((x-1)+3)((x-1)-3) & \text{ if } (-1, 5]  \\
  \frac{7(x-1)}{4} - 8 & \text{ if } (5, 7)
\end{cases}
\]





\[
f(x) = 
\begin{cases}
  -2x - 1 & \text{ if } [-5, -1]   \\
  -(x+2)(x-4) & \text{ if } (-1, 5]  \\
  \frac{7(x-1)}{4} - 8 & \text{ if } (5, 7)
\end{cases}
\]

















\begin{image}
\begin{tikzpicture} 
  \begin{axis}[
            domain=-10:10, ymax=10, xmax=10, ymin=-10, xmin=-10,
            axis lines =center, xlabel=$x$, ylabel={$z=f(x)$},
            every axis y label/.style={at=(current axis.above origin),anchor=south},
            every axis x label/.style={at=(current axis.right of origin),anchor=west},
            axis on top
          ]
          
          \addplot [line width=2, penColor, smooth,samples=100,domain=(-5:-1)] {-2*x-1};
          \addplot [line width=2, penColor, smooth,samples=100,domain=(-1:5)] {-1*(x+2)*(x-4))};
          \addplot [line width=2, penColor, smooth,samples=100,domain=(5:7)] {1.75*(x-1)-8};




      \addplot[color=penColor,fill=penColor,only marks,mark=*] coordinates{(-5,9)};
      \addplot[color=penColor,fill=penColor,only marks,mark=*] coordinates{(-1,1)};

      \addplot[color=penColor,fill=white,only marks,mark=*] coordinates{(-1,5)};
      \addplot[color=penColor,fill=penColor,only marks,mark=*] coordinates{(5,-7)};

      \addplot[color=penColor,fill=white,only marks,mark=*] coordinates{(5,-1)};
      \addplot[color=penColor,fill=white,only marks,mark=*] coordinates{(7,2.5)};


           

  \end{axis}
\end{tikzpicture}
\end{image}




The graph has the same three shapes. The endpoints are hollow or folled as they were in the graph of $y=V(h)$.



\end{example}






\begin{example} Shifting Domain


Let $P(t)$ be a function with domain $[-11, -7) \cup {-6} \cup [-4,3) \cup (4, 7] \cup {8} \cup [9, 10)$.

Let $g(w)$ be defined as a shifted domain version of $p(t)$ with domain $[-7, -3) \cup {-2} \cup [0,7) \cup (8, 11] \cup {12} \cup [13, 14)$.

Then  $g(w) = P\left(\answer{w+4}\right)$



\end{example}





When a new function is defined as a domain shift of an existing function, then there are two different domains.  It is natural to think "what was done to the old domain to get the new one?"

Suppose $f(t)$ is an existing function with domain $[-3, 5]$.

Define a new function $g(x)$ as $g(x) = f(x-8)$ with the induced domain.

"$t$" represents elements of the old domain of $f(t)$.  "$x$" represents elements of the new domain of $g(x)$.  It is natural to ask "what happens to $t$ to get $x$?"  But that is not how it is presented to us.  We are told that $t=x-8$.  But that is what happens to $x$ to get $t$.  That is backwards of how we naturally think about old and new.  We need to solve for $x$ to see what is done to $t$.

\[ t+8=x \]

To get the new dmoain number, $x$, add $8$ to the old domain numbers $t$.  This tells us that the graph of $g$ will be shifted to the right by $8$, compared to the position of the graph of $f$.


It is reversed from how it was originally defined, $g(x) = f(x-8)$, because the "inside" of $f$ is $x-8$ here.  That is not what is done to $t$.  When we solve $t=x-8$ for $t$, we reverse all of the arithmetic.




\begin{example} Shifting Domain


Let $D(y)$ be a function with its domain.

Let $K(r)$ be defined as $K(r) = D(r+7)$ with its induced domain.


Then the domain of $K$ is the domain of $D$, but shifted \wordChoice{\choice[correct] {Left} \choice {Right}} by 
\wordChoice{\choice{5} \choice{6} \choice[correct]{7}}


Then the graph of $K$ is the domain of $D$, but shifted \wordChoice{\choice[correct] {Left} \choice {Right}} by 
\wordChoice{\choice{5} \choice{6} \choice[correct]{7}}




\end{example}















\end{document}
