\documentclass{ximera}


\graphicspath{
  {./}
  {ximeraTutorial/}
  {basicPhilosophy/}
}

\newcommand{\mooculus}{\textsf{\textbf{MOOC}\textnormal{\textsf{ULUS}}}}

\usepackage{tkz-euclide}\usepackage{tikz}
\usepackage{tikz-cd}
\usetikzlibrary{arrows}
\tikzset{>=stealth,commutative diagrams/.cd,
  arrow style=tikz,diagrams={>=stealth}} %% cool arrow head
\tikzset{shorten <>/.style={ shorten >=#1, shorten <=#1 } } %% allows shorter vectors

\usetikzlibrary{backgrounds} %% for boxes around graphs
\usetikzlibrary{shapes,positioning}  %% Clouds and stars
\usetikzlibrary{matrix} %% for matrix
\usepgfplotslibrary{polar} %% for polar plots
\usepgfplotslibrary{fillbetween} %% to shade area between curves in TikZ
\usetkzobj{all}
\usepackage[makeroom]{cancel} %% for strike outs
%\usepackage{mathtools} %% for pretty underbrace % Breaks Ximera
%\usepackage{multicol}
\usepackage{pgffor} %% required for integral for loops



%% http://tex.stackexchange.com/questions/66490/drawing-a-tikz-arc-specifying-the-center
%% Draws beach ball
\tikzset{pics/carc/.style args={#1:#2:#3}{code={\draw[pic actions] (#1:#3) arc(#1:#2:#3);}}}



\usepackage{array}
\setlength{\extrarowheight}{+.1cm}
\newdimen\digitwidth
\settowidth\digitwidth{9}
\def\divrule#1#2{
\noalign{\moveright#1\digitwidth
\vbox{\hrule width#2\digitwidth}}}






\DeclareMathOperator{\arccot}{arccot}
\DeclareMathOperator{\arcsec}{arcsec}
\DeclareMathOperator{\arccsc}{arccsc}

















%%This is to help with formatting on future title pages.
\newenvironment{sectionOutcomes}{}{}


\title{Shifting}

\begin{document}

\begin{abstract}
%Stuff can go here later if we want to!
\end{abstract}
\maketitle



Our Library of Elementary Functions is the starting point.  The Elementary Functions are our basic building blocks.  We use them to construct other, more complex, functions.  ANd, many new functions can be created from these functions.


$\blacktriangleright$ \textbf{Arithmetic:} A new function can be created as the sum, difference, product, or quotient of two known functions. \\


$\blacktriangleright$ \textbf{Piecewise:} A new function can be created by gluing together pieces of other functions.  \\


$\blacktriangleright$ \textbf{Shifting:}  A function is essentially a collection of domain-range pairs.  


\begin{itemize}
\item We can shift the domain numbers in the pairs to create a new function.
\item We can shift the range numbers in the pairs to create a new function.
\end{itemize}




\subsection{Learning Outcomes}


\begin{sectionOutcomes}
In this section, students will 

\begin{itemize}
\item examine the effects of shifting a domain.
\item examine the effects of shifting a range.
\end{itemize}
\end{sectionOutcomes}











\begin{center}
\textbf{\textcolor{green!50!black}{ooooo=-=-=-=-=-=-=-=-=-=-=-=-=ooOoo=-=-=-=-=-=-=-=-=-=-=-=-=ooooo}} \\

more examples can be found by following this link\\ \link[More Examples of Shifting]{https://ximera.osu.edu/csccmathematics/precalculus1/precalculus1/shifting/examples/exampleList}

\end{center}







\end{document}
