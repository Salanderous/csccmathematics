\documentclass{ximera}


\graphicspath{
  {./}
  {ximeraTutorial/}
  {basicPhilosophy/}
}

\newcommand{\mooculus}{\textsf{\textbf{MOOC}\textnormal{\textsf{ULUS}}}}

\usepackage{tkz-euclide}\usepackage{tikz}
\usepackage{tikz-cd}
\usetikzlibrary{arrows}
\tikzset{>=stealth,commutative diagrams/.cd,
  arrow style=tikz,diagrams={>=stealth}} %% cool arrow head
\tikzset{shorten <>/.style={ shorten >=#1, shorten <=#1 } } %% allows shorter vectors

\usetikzlibrary{backgrounds} %% for boxes around graphs
\usetikzlibrary{shapes,positioning}  %% Clouds and stars
\usetikzlibrary{matrix} %% for matrix
\usepgfplotslibrary{polar} %% for polar plots
\usepgfplotslibrary{fillbetween} %% to shade area between curves in TikZ
\usetkzobj{all}
\usepackage[makeroom]{cancel} %% for strike outs
%\usepackage{mathtools} %% for pretty underbrace % Breaks Ximera
%\usepackage{multicol}
\usepackage{pgffor} %% required for integral for loops



%% http://tex.stackexchange.com/questions/66490/drawing-a-tikz-arc-specifying-the-center
%% Draws beach ball
\tikzset{pics/carc/.style args={#1:#2:#3}{code={\draw[pic actions] (#1:#3) arc(#1:#2:#3);}}}



\usepackage{array}
\setlength{\extrarowheight}{+.1cm}
\newdimen\digitwidth
\settowidth\digitwidth{9}
\def\divrule#1#2{
\noalign{\moveright#1\digitwidth
\vbox{\hrule width#2\digitwidth}}}






\DeclareMathOperator{\arccot}{arccot}
\DeclareMathOperator{\arcsec}{arcsec}
\DeclareMathOperator{\arccsc}{arccsc}

















%%This is to help with formatting on future title pages.
\newenvironment{sectionOutcomes}{}{}


\title{Shifting Range}

\begin{document}

\begin{abstract}
same characteristics
\end{abstract}
\maketitle


















Below is the piecewise defined function, $T(v)$.  $v$ is representing the domain values from $(-4,-1] \cup [1,7)$. $T(v)$ represents the range number paired with $v$.   Therefore, $T(v)$ represents numbers from $(-9, 2]$.




\[
T(v) = 
\begin{cases}
  2v-1 & \text{ if }  -4 < v \leq -1 \\
  -v+3 & \text{ if } 1 \leq v < 7
\end{cases}
\]


On the interval $(-4, -1]$, the graph should be a line for $T(v) = 2v-1$. On the interval $[1, 7)$, the graph should be another line for $T(v) = -v+3$




Graph of $y = T(v)$.
\begin{image}
\begin{tikzpicture}
	\begin{axis}[
            domain=-10:10, ymax=10, xmax=10, ymin=-10, xmin=-10,
            axis lines =center, xlabel=$v$, ylabel=$y$,
            every axis y label/.style={at=(current axis.above origin),anchor=south},
            every axis x label/.style={at=(current axis.right of origin),anchor=west},
            axis on top
          ]
          
	\addplot [draw=penColor,very thick,smooth,domain=(-4:-1)] {2*x-1};
	\addplot [draw=penColor,very thick,smooth,domain=(1:7)] {-x+3};
	\addplot[color=penColor,only marks,mark=*] coordinates{(-1,-3)}; 
	\addplot[color=penColor,fill=white,only marks,mark=*] coordinates{(-4,-9)}; 
	\addplot[color=penColor,only marks,mark=*] coordinates{(1,2)}; 
	\addplot[color=penColor,fill=white,only marks,mark=*] coordinates{(7,-4)}; 


    \end{axis}
\end{tikzpicture}
\end{image}







The domain of $T$ has two maximal intervals:, $(-4,-1]$ and $[1,7)$.  These correspond to two line segments on the graph. The endpoints give us four important points on the graph: 

\begin{itemize}

\item $(-4, -9)$, which is an open point on the graph.
\item $(-1, -3)$, which is a closed point on the graph.
\item $(1, 2)$, which is a closed point on the graph.
\item $(7, -4)$, which is an open point on the graph.

\end{itemize}


The function $T$ has no minimum values.  It has a global maximum of $2$, which occurs at $1$.  This is also a local maximum.  There is another local maximum value of $-3$, which occurs at $-1$.

The function $T$ is increasing on the interval $(-4,-1]$. \\
The function $T$ is decreasing on the interval $[-1, 7)$.


\section{A New Function}

$B(k)$ is a new function, but its definition is based on $T(v)$.


$B(k) = T(k)+4$ with the implied domain. \\


\textbf{First Question:} What is the domain of $B$?

$k$ represents the domain of $B$, which means it comes from $(-4,-1] \cup [1,7)$. $k$ is also representing the domain of $B$.  Therefore, the domain $B$ is also $(-4,-1] \cup [1,7)$.


The addition of $4$ is not inside the domain parentheses next to $B$.  It is on the outside of the domain parentheses. $T(k)$ represents range values or values of the function and these are increasing by $4$.





THe formula for $B$ is simnply the formula for $T$ with $4$ added.




\[
B(k) = 
\begin{cases}
  (2k-1)+4 = 2k+3 & \text{ if }  -4 < k \leq -1 \\
  (-k+3)+4 = -k+7 & \text{ if } 1 \leq k < 7
\end{cases}
\]








Graph of $z = B(k)$.
\begin{image}
\begin{tikzpicture}
	\begin{axis}[
            domain=-10:10, ymax=10, xmax=10, ymin=-10, xmin=-10,
            axis lines =center, xlabel=$k$, ylabel=$z$,
            every axis y label/.style={at=(current axis.above origin),anchor=south},
            every axis x label/.style={at=(current axis.right of origin),anchor=west},
            axis on top
          ]
          
	\addplot [draw=penColor,very thick,smooth,domain=(-4:-1)] {2*x-1+4};
	\addplot [draw=penColor,very thick,smooth,domain=(1:7)] {-x+3+4};
	\addplot[color=penColor,only marks,mark=*] coordinates{(-1,1)}; 
	\addplot[color=penColor,fill=white,only marks,mark=*] coordinates{(-4,-5)}; 
	\addplot[color=penColor,only marks,mark=*] coordinates{(1,6)}; 
	\addplot[color=penColor,fill=white,only marks,mark=*] coordinates{(7,0)}; 


    \end{axis}
\end{tikzpicture}
\end{image}


The shape of the graph has not changed.  It has just slide up the axes.

The 




$B(k)$ has a global maximum of $2+4$, which occurs at $1$.  This is also a local maximum.  There is another local maximum value of $-3+4$, which occurs at $-1$.

The function $B$ is increasing on the interval $(-4,-1]$, just like $T$. \\
The function $B$ is decreasing on the interval $[-1, 7)$, just like $T$.

























\begin{example} Shifting the Range 




Define $V(h)$ as

\[
V(h) = 
\begin{cases}
  -2h-3 & \text{ if } [-6, -2]   \\
  -(h+3)(h-3) & \text{ if } (-2, 4]  \\
  \frac{7h}{4} - 8 & \text{ if } (4,6)
\end{cases}
\]



A graph always helps our thinking. Here is theg raph of $y = V(h)$.








\begin{image}
\begin{tikzpicture} 
  \begin{axis}[
            domain=-10:10, ymax=10, xmax=10, ymin=-10, xmin=-10,
            axis lines =center, xlabel=$h$, ylabel=$y$,
            every axis y label/.style={at=(current axis.above origin),anchor=south},
            every axis x label/.style={at=(current axis.right of origin),anchor=west},
            axis on top
          ]
          
      \addplot [line width=2, penColor, smooth,samples=100,domain=(-6:-2)] {-2*x-3};
          \addplot [line width=2, penColor, smooth,samples=100,domain=(-2:4)] {-1*(x+3)*(x-3))};
          \addplot [line width=2, penColor, smooth,samples=100,domain=(4:6)] {1.75*x-8};




      \addplot[color=penColor,fill=penColor,only marks,mark=*] coordinates{(-6,9)};
      \addplot[color=penColor,fill=penColor,only marks,mark=*] coordinates{(-2,1)};

      \addplot[color=penColor,fill=white,only marks,mark=*] coordinates{(-2,5)};
      \addplot[color=penColor,fill=penColor,only marks,mark=*] coordinates{(4,-7)};

      \addplot[color=penColor,fill=white,only marks,mark=*] coordinates{(4,-1)};
      \addplot[color=penColor,fill=white,only marks,mark=*] coordinates{(6,2.5)};


           

  \end{axis}
\end{tikzpicture}
\end{image}



\begin{itemize}

\item The domain of $V$ is $[-6,6)$.
\item $V$ has a global maximum of $9$, which occurs at $0$.
\item $V$ has a global minimum of $-7$, which occurs at $4$.
\item $V$ has a jump discontinuity at $-2$.
\item $V$ has a jump discontinuity at $4$.
\item $V$ is decreasing on $[-6, -2]$.
\item $V$ is increasing on $[-2, 0]$.
\item $V$ is decreasing on $[0, 4]$.
\item $V$ is increasing on $[4, 6)$.


\end{itemize}








Let's define a new function based on $V$.\\





Let $f(x) = V(x)-2$ with the induced domain.

The domain of $V$ is $[-6, 6)$, same as $V$. The range values or function values $f$ are all $2$ less than the function values of $V$.







\[
f(x) = 
\begin{cases}
  -2x-5 & \text{ if } [-6, -2]   \\
  -(x+3)(x-3)-2 & \text{ if } (-2, 4]  \\
  \frac{7x}{4} - 10 & \text{ if } (4,6)
\end{cases}
\]














\begin{image}
\begin{tikzpicture} 
  \begin{axis}[
            domain=-10:10, ymax=10, xmax=10, ymin=-10, xmin=-10,
            axis lines =center, xlabel=$h$, ylabel=$y$,
            every axis y label/.style={at=(current axis.above origin),anchor=south},
            every axis x label/.style={at=(current axis.right of origin),anchor=west},
            axis on top
          ]
          
      \addplot [line width=2, penColor, smooth,samples=100,domain=(-6:-2)] {-2*x-5};
          \addplot [line width=2, penColor, smooth,samples=100,domain=(-2:4)] {-1*(x+3)*(x-3))-2};
          \addplot [line width=2, penColor, smooth,samples=100,domain=(4:6)] {1.75*x-10};




      \addplot[color=penColor,fill=penColor,only marks,mark=*] coordinates{(-6,7)};
      \addplot[color=penColor,fill=penColor,only marks,mark=*] coordinates{(-2,-1)};

      \addplot[color=penColor,fill=white,only marks,mark=*] coordinates{(-2,3)};
      \addplot[color=penColor,fill=penColor,only marks,mark=*] coordinates{(4,-9)};

      \addplot[color=penColor,fill=white,only marks,mark=*] coordinates{(4,-3)};
      \addplot[color=penColor,fill=white,only marks,mark=*] coordinates{(6,0.5)};


           

  \end{axis}
\end{tikzpicture}
\end{image}







\begin{itemize}

\item The domain of $f$ is $[-6,6)$.
\item $f$ has a global maximum of $7$, which occurs at $0$.
\item $f$ has a global minimum of $-9$, which occurs at $4$.
\item $f$ has a jump discontinuity at $-2$.
\item $f$ has a jump discontinuity at $4$.
\item $f$ is decreasing on $[-6, -2]$.
\item $f$ is increasing on $[-2, 0]$.
\item $f$ is decreasing on $[0, 4]$.
\item $f$ is increasing on $[4, 6)$.


\end{itemize}



The places in the domain where characteristics and features occur have not changed.  The maximums and minimums have all dropped by $2$.  The endpoints are all still solid or hollow.  Their vertical coordinates have all dropped by $2$.






\end{example}





































\end{document}
