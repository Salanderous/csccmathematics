\documentclass{ximera}


\graphicspath{
  {./}
  {ximeraTutorial/}
  {basicPhilosophy/}
}

\newcommand{\mooculus}{\textsf{\textbf{MOOC}\textnormal{\textsf{ULUS}}}}

\usepackage{tkz-euclide}\usepackage{tikz}
\usepackage{tikz-cd}
\usetikzlibrary{arrows}
\tikzset{>=stealth,commutative diagrams/.cd,
  arrow style=tikz,diagrams={>=stealth}} %% cool arrow head
\tikzset{shorten <>/.style={ shorten >=#1, shorten <=#1 } } %% allows shorter vectors

\usetikzlibrary{backgrounds} %% for boxes around graphs
\usetikzlibrary{shapes,positioning}  %% Clouds and stars
\usetikzlibrary{matrix} %% for matrix
\usepgfplotslibrary{polar} %% for polar plots
\usepgfplotslibrary{fillbetween} %% to shade area between curves in TikZ
\usetkzobj{all}
\usepackage[makeroom]{cancel} %% for strike outs
%\usepackage{mathtools} %% for pretty underbrace % Breaks Ximera
%\usepackage{multicol}
\usepackage{pgffor} %% required for integral for loops



%% http://tex.stackexchange.com/questions/66490/drawing-a-tikz-arc-specifying-the-center
%% Draws beach ball
\tikzset{pics/carc/.style args={#1:#2:#3}{code={\draw[pic actions] (#1:#3) arc(#1:#2:#3);}}}



\usepackage{array}
\setlength{\extrarowheight}{+.1cm}
\newdimen\digitwidth
\settowidth\digitwidth{9}
\def\divrule#1#2{
\noalign{\moveright#1\digitwidth
\vbox{\hrule width#2\digitwidth}}}






\DeclareMathOperator{\arccot}{arccot}
\DeclareMathOperator{\arcsec}{arcsec}
\DeclareMathOperator{\arccsc}{arccsc}

















%%This is to help with formatting on future title pages.
\newenvironment{sectionOutcomes}{}{}



\author{Lee Wayand}

\begin{document}
\begin{exercise}





Since sine and cosine are periodic, our analysis often focuses on the interval $[0, 2\pi)$ and then reasons the remaining information from the periodicity.  \\

\begin{itemize}
\item On $[0, 2\pi)$, the maximum of $\sin(\theta)$ occurs at $\answer{\frac{\pi}{2}}$. 
\item On $[0, 2\pi)$, the minimum of $\sin(\theta)$ occurs at $\answer{\frac{3\pi}{2}}$. 
\end{itemize}










\begin{question} Shifted Domain 

Define $f(t) = \sin\left(t + \frac{\pi}{3} \right)$. \\




Which interval would we focus on when analyzing $f(t)$? \\

\[
\left[      \answer{-\frac{\pi}{3}},   \answer{\frac{5\pi}{3}}     \right)
\]




Where does the maximum of $f(t)$ occur?

\[
t = \answer{\frac{\pi}{6}} 
\]





Where does the minimum of $f(t)$ occur?

\[
t = \answer{\frac{7\pi}{6}} 
\]



\end{question}


















\end{exercise}
\end{document}