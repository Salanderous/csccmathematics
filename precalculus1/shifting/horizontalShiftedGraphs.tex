\documentclass{ximera}


\graphicspath{
  {./}
  {ximeraTutorial/}
  {basicPhilosophy/}
}

\newcommand{\mooculus}{\textsf{\textbf{MOOC}\textnormal{\textsf{ULUS}}}}

\usepackage{tkz-euclide}\usepackage{tikz}
\usepackage{tikz-cd}
\usetikzlibrary{arrows}
\tikzset{>=stealth,commutative diagrams/.cd,
  arrow style=tikz,diagrams={>=stealth}} %% cool arrow head
\tikzset{shorten <>/.style={ shorten >=#1, shorten <=#1 } } %% allows shorter vectors

\usetikzlibrary{backgrounds} %% for boxes around graphs
\usetikzlibrary{shapes,positioning}  %% Clouds and stars
\usetikzlibrary{matrix} %% for matrix
\usepgfplotslibrary{polar} %% for polar plots
\usepgfplotslibrary{fillbetween} %% to shade area between curves in TikZ
\usetkzobj{all}
\usepackage[makeroom]{cancel} %% for strike outs
%\usepackage{mathtools} %% for pretty underbrace % Breaks Ximera
%\usepackage{multicol}
\usepackage{pgffor} %% required for integral for loops



%% http://tex.stackexchange.com/questions/66490/drawing-a-tikz-arc-specifying-the-center
%% Draws beach ball
\tikzset{pics/carc/.style args={#1:#2:#3}{code={\draw[pic actions] (#1:#3) arc(#1:#2:#3);}}}



\usepackage{array}
\setlength{\extrarowheight}{+.1cm}
\newdimen\digitwidth
\settowidth\digitwidth{9}
\def\divrule#1#2{
\noalign{\moveright#1\digitwidth
\vbox{\hrule width#2\digitwidth}}}






\DeclareMathOperator{\arccot}{arccot}
\DeclareMathOperator{\arcsec}{arcsec}
\DeclareMathOperator{\arccsc}{arccsc}

















%%This is to help with formatting on future title pages.
\newenvironment{sectionOutcomes}{}{}


\title{Left and Right}

\begin{document}

\begin{abstract}
sliding the graph
\end{abstract}
\maketitle






$\blacktriangleright$ Graphically, shifting the domain of a function appears to shift the graph horizontally - left or right.  The shape of the graph doesn't change.  The whole graph moves rigidly left or right.  All points move the same distance horizontally. \\


\begin{itemize}
\item Let $F(x)$ be a function with its domain.

\item Let $G(t)$ be a new function defined as $G(t) = F(t+d_0)$ with its induced domain.
\end{itemize}



We begin with the function $F$.  The domain values of $F$ are represented by $x$.  We then define a new function, $G$. The domain values of $G$ are represented by $t$. 

$F$ and $G$ are connected.  To evaluate $G$ at $t$, you evaluate $F$ at $t + d_0$, where $d_0$ is some constant.  Therefore, $t + d_0$ are $x$-values. We have $x = t + d_0$.  This tells us what to do to $t$ to get $x$.  However, that is not how our story is told.  Our story begins with $F$ and then $G$ is defined from $F$.  

We want to know how to get $t$ from $x$.  


\begin{itemize}
\item $x = t + d_0$

\item $x - d_0 = t$
\end{itemize}


To get corresponding values of $t$ from values of $x$, subtract $d_0$.   



$\blacktriangleright$  The graph of $G$ is obtained from the graph of $F$ by subtracting $d_0$, which appears to be reverse of the definition, $G(t) = F(t+d_0)$.  That is because the definition tells how to get the old domain for $F$, rather than getting the new domain for $G$.



\begin{example} Shifting Horizontally




Graph of $y = T(v)$.

\begin{image}
\begin{tikzpicture}
	\begin{axis}[
            domain=-10:10, ymax=10, xmax=10, ymin=-10, xmin=-10,
            axis lines =center, xlabel=$v$, ylabel=$y$,
            ytick={-10,-8,-6,-4,-2,2,4,6,8,10},
            xtick={-10,-8,-6,-4,-2,2,4,6,8,10},
            ticklabel style={font=\scriptsize},
            every axis y label/.style={at=(current axis.above origin),anchor=south},
            every axis x label/.style={at=(current axis.right of origin),anchor=west},
            axis on top
          ]
          
	\addplot [draw=penColor,very thick,smooth,domain=(-7:-4)] {-x-6};
	\addplot [draw=penColor,very thick,smooth,domain=(-2:1)] {-7};
	\addplot [draw=penColor,very thick,smooth,domain=(1:7)] {-x+7};
	
	\addplot[color=penColor,only marks,mark=*] coordinates{(-7,1)}; 
	\addplot[color=penColor,fill=white,only marks,mark=*] coordinates{(-4,-2)}; 
	\addplot[color=penColor,only marks,mark=*] coordinates{(-2,-7)}; 
	\addplot[color=penColor,only marks,mark=*] coordinates{(1,-7)}; 
	\addplot[color=penColor,only marks,mark=*] coordinates{(1,6)}; 
	\addplot[color=penColor,fill=white,only marks,mark=*] coordinates{(7,0)}; 


    \end{axis}
\end{tikzpicture}
\end{image}


Define a new function $W$ by $W(h)=T(h+3)$, with the implied domain.

Which is the grapgs below is the graph of $z=W(h)$?





\begin{image}
\begin{tikzpicture}
	\begin{axis}[name = leftgraph,
            domain=-10:10, ymax=10, xmax=10, ymin=-10, xmin=-10,
            axis lines =center, xlabel=$h$, ylabel=$z$,
            ytick={-10,-8,-6,-4,-2,2,4,6,8,10},
            xtick={-10,-8,-6,-4,-2,2,4,6,8,10},
            ticklabel style={font=\scriptsize},
            every axis y label/.style={at=(current axis.above origin),anchor=south},
            every axis x label/.style={at=(current axis.right of origin),anchor=west},
            axis on top
          ]
          
	\addplot [draw=penColor,very thick,smooth,domain=(-10:-7)] {-(x+3)-6};
	\addplot [draw=penColor,very thick,smooth,domain=(-5:-2)] {-7};
	\addplot [draw=penColor,very thick,smooth,domain=(-2:4)] {-(x+3)+7};
	
	\addplot[color=penColor,only marks,mark=*] coordinates{(-10,1)}; 
	\addplot[color=penColor,fill=white,only marks,mark=*] coordinates{(-7,-2)}; 
	\addplot[color=penColor,only marks,mark=*] coordinates{(-5,-7)}; 
	\addplot[color=penColor,only marks,mark=*] coordinates{(-2,-7)}; 
	\addplot[color=penColor,only marks,mark=*] coordinates{(-2,6)}; 
	\addplot[color=penColor,fill=white,only marks,mark=*] coordinates{(4,0)}; 


    \end{axis}
	\begin{axis}[at={(leftgraph.outer east)},anchor=outer west, 
            domain=-10:10, ymax=10, xmax=10, ymin=-10, xmin=-10,
            axis lines =center, xlabel=$h$, ylabel=$z$,
            ytick={-10,-8,-6,-4,-2,2,4,6,8,10},
            xtick={-10,-8,-6,-4,-2,2,4,6,8,10},
            ticklabel style={font=\scriptsize},
            every axis y label/.style={at=(current axis.above origin),anchor=south},
            every axis x label/.style={at=(current axis.right of origin),anchor=west},
            axis on top
          ]
          
	\addplot [draw=penColor,very thick,smooth,domain=(-4:-1)] {-(x-3)-6};
	\addplot [draw=penColor,very thick,smooth,domain=(1:4)] {-7};
	\addplot [draw=penColor,very thick,smooth,domain=(4:10)] {-(x-3)+7};
	
	\addplot[color=penColor,only marks,mark=*] coordinates{(-4,1)}; 
	\addplot[color=penColor,fill=white,only marks,mark=*] coordinates{(-1,-2)}; 
	\addplot[color=penColor,only marks,mark=*] coordinates{(1,-7)}; 
	\addplot[color=penColor,only marks,mark=*] coordinates{(4,-7)}; 
	\addplot[color=penColor,only marks,mark=*] coordinates{(4,6)}; 
	\addplot[color=penColor,fill=white,only marks,mark=*] coordinates{(10,0)}; 


    \end{axis}


\end{tikzpicture}
\end{image}





\begin{multipleChoice}
\choice[correct]{graph on the left}
\choice{graph on the right}
\end{multipleChoice}

To figure this out, we need to know how to getthe new variable $h$ from the old variable $v$. \\




\begin{itemize}
\item $h$ represents the domain of $W$. 
\item $v$ represents the domain of $T$.  
\item $v$ = $h+3$
\end{itemize}


These tell us that $v-3 = h$ and the graph of $W$ is the graph of $T$ shifted left $3$.



\end{example}


The shape of the graph didn't change. It just slide to the left. There are still three pieces. \\




\begin{itemize}
\item The domain still has a gap of length $2$ in it.  
\item The middle pieces is still horzontal.  
\item The solid and hollow endpoint dots are still in the same positions. 
\end{itemize}





Shifting doesn't change the shape of the graph or any of the relative measurements. It is a rigid movement.












\begin{example} Sine



Graph of $y = sin(\theta)$.

\begin{image}
\begin{tikzpicture} 
  \begin{axis}[
            domain=-10:10, ymax=1.5, xmax=10, ymin=-1.5, xmin=-10,
            xtick={-6.28, -3.14, 3.14, 6.28}, 
            xticklabels={$-2\pi$, $-\pi$, $\pi$, $2\pi$},
            axis lines =center,  xlabel={$\theta$}, ylabel=$y$,
            ticklabel style={font=\scriptsize},
            every axis y label/.style={at=(current axis.above origin),anchor=south},
            every axis x label/.style={at=(current axis.right of origin),anchor=west},
            axis on top
          ]
          
            \addplot [line width=2, penColor, smooth,samples=200,domain=(-9:9), <->] {sin(deg(x))};

           

  \end{axis}
\end{tikzpicture}
\end{image}



Let's create a new function called $T$, which is a shifted sine function.

\[  T(t) = sin(t + \tfrac{\pi}{6})  \]



\begin{image}
\begin{tikzpicture} 
  \begin{axis}[
            domain=-10:10, ymax=1.5, xmax=10, ymin=-1.5, xmin=-10,
            xtick={-6.28, -3.14, 3.14, 6.28}, 
            xticklabels={$-2\pi$, $-\pi$, $\pi$, $2\pi$},
            axis lines =center,  xlabel={$\theta$}, ylabel=$y$,
            ticklabel style={font=\scriptsize},
            every axis y label/.style={at=(current axis.above origin),anchor=south},
            every axis x label/.style={at=(current axis.right of origin),anchor=west},
            axis on top
          ]
          
            \addplot [line width=2, penColor2, smooth,samples=200,domain=(-9:9), <->] {sin(deg(x+0.523599))};

           

  \end{axis}
\end{tikzpicture}
\end{image}
This graph looks the same as the one above, except shifted to the left.


$\theta = t + \tfrac{\pi}{6}$   or $\theta - \tfrac{\pi}{6}= t $



If we label the horizontal axes $\theta$ and $t$, then we can better compare.

\end{example}









\begin{image}
\begin{tikzpicture} 
  \begin{axis}[
            domain=-10:10, ymax=1.5, xmax=10, ymin=-1.5, xmin=-10,
            xtick={-6.28, -3.14, 3.14, 6.28}, 
            xticklabels={$-2\pi$, $-\pi$, $\pi$, $2\pi$},
            axis lines =center,  xlabel={$\theta$/$t$}, ylabel=$y$,
            ticklabel style={font=\scriptsize},
            every axis y label/.style={at=(current axis.above origin),anchor=south},
            every axis x label/.style={at=(current axis.right of origin),anchor=west},
            axis on top
          ]
          
            \addplot [line width=2, penColor, smooth,samples=200,domain=(-9:9), <->] {sin(deg(x))};
            \addplot [line width=2, penColor2, smooth,samples=200,domain=(-9:9), <->] {sin(deg(x+0.523599))};

           

  \end{axis}
\end{tikzpicture}
\end{image}











\begin{example} Sine



Graph of $y = sin(\theta)$.

\begin{image}
\begin{tikzpicture} 
  \begin{axis}[
            domain=-10:10, ymax=1.5, xmax=10, ymin=-1.5, xmin=-10,
            xtick={-6.28, -3.14, 3.14, 6.28}, 
            xticklabels={$-2\pi$, $-\pi$, $\pi$, $2\pi$},
            axis lines =center,  xlabel={$\theta$}, ylabel=$y$,
            ticklabel style={font=\scriptsize},
            every axis y label/.style={at=(current axis.above origin),anchor=south},
            every axis x label/.style={at=(current axis.right of origin),anchor=west},
            axis on top
          ]
          
          	\addplot [line width=2, penColor, smooth,samples=200,domain=(-9:9), <->] {sin(deg(x))};

           

  \end{axis}
\end{tikzpicture}
\end{image}



$sin(\theta)$ is periodic with a period of $2\pi$.  If it is shifted by $2\pi$ or any integer multiple of $2\pi$, then the resulting function is again $sin(\theta)$.


\[    sin(\theta + 2k\pi) = sin(\theta)   \,   \text{ where }  \,  k \in \textbf{Z}       \]



\begin{itemize}
\item The zeros of $sin(\theta)$ are all integer multiples of $\pi$.
\item The maximum value is $1$ and it occurs at:  $\left\{     \frac{(4k+1)\pi}{2} \, | \, k \in \textbf{Z}     \right\}$
\item The minimum value is $-1$ and it occurs at:  $\left\{    \frac{(4k+3)\pi}{2} \, | \, k \in \textbf{Z}     \right\}$
\end{itemize}


If $sin(\theta)$ is shifted left by $\frac{\pi}{2}$, then we get $cos(\theta)$.


\[    sin\left(\theta + \frac{\pi}{2}\right) = cos(\theta)   \]


Graph of $y = cos(\theta)$.

\begin{image}
\begin{tikzpicture} 
  \begin{axis}[
            domain=-10:10, ymax=1.5, xmax=10, ymin=-1.5, xmin=-10,
            xtick={-6.28, -3.14, 3.14, 6.28}, 
            xticklabels={$-2\pi$, $-\pi$, $\pi$, $2\pi$},
            axis lines =center,  xlabel={$\theta$}, ylabel=$y$,
            ticklabel style={font=\scriptsize},
            every axis y label/.style={at=(current axis.above origin),anchor=south},
            every axis x label/.style={at=(current axis.right of origin),anchor=west},
            axis on top
          ]
          
          	\addplot [line width=2, penColor, smooth,samples=200,domain=(-9:9), <->] {cos(deg(x))};

           

  \end{axis}
\end{tikzpicture}
\end{image}



\end{example}




This agrees with the unit circle.  As you move along the unit circle, the right/vertical coordinate (sine) has the same value as the left/horizontal coordinate (cosine) back a quarter-circle.









\begin{example} Absolute Value



Graph of $y = |x|$.



\begin{image}
\begin{tikzpicture} 
  \begin{axis}[
            domain=-10:10, ymax=10, xmax=10, ymin=-10, xmin=-10,
            axis lines =center, xlabel=$r$, ylabel=$y$,
            ytick={-10,-8,-6,-4,-2,2,4,6,8,10},
            xtick={-10,-8,-6,-4,-2,2,4,6,8,10},
            ticklabel style={font=\scriptsize},
            every axis y label/.style={at=(current axis.above origin),anchor=south},
            every axis x label/.style={at=(current axis.right of origin),anchor=west},
            axis on top
          ]
          
          \addplot [line width=2, penColor, smooth, samples=200, domain=(-7:7),<->] {abs(x)};
        

  \end{axis}
\end{tikzpicture}
\end{image}





What does the graph of $z = |t-2|$ look like?

This is a shift from the basic absolute value graph.  All we really need to know is where is the corner.  

\begin{center}
The corner occurs when the inside of the absolute value equals $0$.
\end{center}



$t-2=0$ when $t=2$.  That is where the new corner sits.









\begin{image}
\begin{tikzpicture} 
  \begin{axis}[
            domain=-10:10, ymax=10, xmax=10, ymin=-10, xmin=-10,
            axis lines =center, xlabel=$t$, ylabel=$z$,
            ytick={-10,-8,-6,-4,-2,2,4,6,8,10},
            xtick={-10,-8,-6,-4,-2,2,4,6,8,10},
            ticklabel style={font=\scriptsize},
            every axis y label/.style={at=(current axis.above origin),anchor=south},
            every axis x label/.style={at=(current axis.right of origin),anchor=west},
            axis on top
          ]
          
          \addplot [line width=2, penColor, smooth, samples=200, domain=(-7:9),<->] {abs(x-2)};
        

  \end{axis}
\end{tikzpicture}
\end{image}








\end{example}


Much of graphing follows this example.  \\


There are important points for the function's graph. You identify the position of those points.  Then, the shifted graph follows the basic shape of the original graph.






For example, $y = log_k(t)$ including $y = ln(t)$ has a vertical asymptote when the inside of the logarithm equals $0$. The zero, and corresponding horizontal intercept, occurs when they inside equals $1$.  






\begin{example}

Here is the graph of $y = log_2(t+5)$.

\begin{itemize}
\item vertical asymptote: $t+5=0$, when $t=\answer{-5}$
\item horizontal intercept: $t+5=1$, when $t=\answer{-4}$
\end{itemize}


\begin{image}
\begin{tikzpicture} 
  \begin{axis}[
            domain=-10:10, ymax=10, xmax=10, ymin=-10, xmin=-10,
            axis lines =center, xlabel=$x$, ylabel=$y$,
            ytick={-10,-8,-6,-4,-2,2,4,6,8,10},
            xtick={-10,-8,-6,-4,-2,2,4,6,8,10},
            ticklabel style={font=\scriptsize},
            every axis y label/.style={at=(current axis.above origin),anchor=south},
            every axis x label/.style={at=(current axis.right of origin),anchor=west},
            axis on top
          ]
          
          \addplot [line width=2, penColor, smooth,samples=200,domain=(-4.97:9),<->] {ln(x+5)/ln(2)};
          \addplot [line width=1, gray, dashed,domain=(-9:9),<->] ({-5},{x});

          \addplot[color=penColor,only marks,mark=*] coordinates{(-4,0)}; 

           

  \end{axis}
\end{tikzpicture}
\end{image}


The graph looks the same as the basic logarithm graph, just slid left $5$.




\end{example}









\begin{example} Shifting Domains


Let $M$ be a function with domain $[-7,-4) \cup [-2,1] \cup [1,7)$. \\

Below is the graph of $y = M(d)$.




\begin{image}
\begin{tikzpicture}
  \begin{axis}[
            domain=-10:10, ymax=10, xmax=10, ymin=-10, xmin=-10,
            axis lines =center, xlabel=$d$, ylabel=$y$,
            ytick={-10,-8,-6,-4,-2,2,4,6,8,10},
            xtick={-10,-8,-6,-4,-2,2,4,6,8,10},
            ticklabel style={font=\scriptsize},
            every axis y label/.style={at=(current axis.above origin),anchor=south},
            every axis x label/.style={at=(current axis.right of origin),anchor=west},
            axis on top
          ]
          
  \addplot [draw=penColor,very thick,smooth,domain=(-7:-4)] {-x-6};
  \addplot [draw=penColor,very thick,smooth,domain=(-2:1)] {-7};
  \addplot [draw=penColor,very thick,smooth,domain=(1:7)] {-x+7};
  
  \addplot[color=penColor,only marks,mark=*] coordinates{(-7,1)}; 
  \addplot[color=penColor,fill=white,only marks,mark=*] coordinates{(-4,-2)}; 
  \addplot[color=penColor,only marks,mark=*] coordinates{(-2,-7)}; 
  \addplot[color=penColor,only marks,mark=*] coordinates{(1,-7)}; 
  \addplot[color=penColor,only marks,mark=*] coordinates{(1,6)}; 
  \addplot[color=penColor,fill=white,only marks,mark=*] coordinates{(7,0)}; 


    \end{axis}
\end{tikzpicture}
\end{image}
A new function is created by shifting $M$. \\





The function $H(k)$ is defined as $H(k) = M(k+3)$ with the implied domain.



\begin{question}

We can see that $M(-7) = 1$.  This tells us that $H\left(\answer{-10}\right) = \answer{1}$. \\

There is an open dot on the graph of $y= M(d)$  when $d = -4$.  There will be a corresponding open dot on the graph of $z = H(k)$ when $k = \answer{-7}$.

\end{question}


\begin{question}

There is a horizontal line segment in the graph of $y = M(d)$ from $(-2, -7)$ to $(1, -7)$.  There will be a horizontal line segment in the graph of $z = H(k)$ from $\left(\answer{-5}, -7\right)$ to $\left(\answer{-2}, -7\right)$.

\end{question}



\begin{question}

There is a closed dot on the graph of $y= M(d)$  when $d = 1$.  There will be a corresponding closed dot on the graph of $z = H(k)$ when $k = \answer{-2}$. \\


There is an open dot on the graph of $y= M(d)$  when $d = 7$.  There will be a corresponding open dot on the graph of $z = H(k)$ when $k = \answer{4}$.

\end{question}




\end{example}
























\end{document}
