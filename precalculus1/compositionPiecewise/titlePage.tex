\documentclass{ximera}


\graphicspath{
  {./}
  {ximeraTutorial/}
  {basicPhilosophy/}
}

\newcommand{\mooculus}{\textsf{\textbf{MOOC}\textnormal{\textsf{ULUS}}}}

\usepackage{tkz-euclide}\usepackage{tikz}
\usepackage{tikz-cd}
\usetikzlibrary{arrows}
\tikzset{>=stealth,commutative diagrams/.cd,
  arrow style=tikz,diagrams={>=stealth}} %% cool arrow head
\tikzset{shorten <>/.style={ shorten >=#1, shorten <=#1 } } %% allows shorter vectors

\usetikzlibrary{backgrounds} %% for boxes around graphs
\usetikzlibrary{shapes,positioning}  %% Clouds and stars
\usetikzlibrary{matrix} %% for matrix
\usepgfplotslibrary{polar} %% for polar plots
\usepgfplotslibrary{fillbetween} %% to shade area between curves in TikZ
\usetkzobj{all}
\usepackage[makeroom]{cancel} %% for strike outs
%\usepackage{mathtools} %% for pretty underbrace % Breaks Ximera
%\usepackage{multicol}
\usepackage{pgffor} %% required for integral for loops



%% http://tex.stackexchange.com/questions/66490/drawing-a-tikz-arc-specifying-the-center
%% Draws beach ball
\tikzset{pics/carc/.style args={#1:#2:#3}{code={\draw[pic actions] (#1:#3) arc(#1:#2:#3);}}}



\usepackage{array}
\setlength{\extrarowheight}{+.1cm}
\newdimen\digitwidth
\settowidth\digitwidth{9}
\def\divrule#1#2{
\noalign{\moveright#1\digitwidth
\vbox{\hrule width#2\digitwidth}}}






\DeclareMathOperator{\arccot}{arccot}
\DeclareMathOperator{\arcsec}{arcsec}
\DeclareMathOperator{\arccsc}{arccsc}

















%%This is to help with formatting on future title pages.
\newenvironment{sectionOutcomes}{}{}


\title{Piecewise Composition}

\begin{document}

\begin{abstract}
%Stuff can go here later if we want!
\end{abstract}
\maketitle




Functions, like numbers, are a system of things (functions) and rules (operations).  

\begin{itemize}
\item the sum of two functions is another function.
\item the difference of two functions is another function.
\item the product of two functions is another function.
\item the quotient of two functions is another function.
\end{itemize}

Of course, we always have to pay some attention to domains.



The usual attention to domains includes adopting the intersection of domains to be the new domain, and avoiding domain numbers which make the denominator equal to $0$.



In addition to these four familiar operations, functions have another operation: \textbf{composition}.

We have already peeked at this. We took a single output number of one function and used it as the input to anther function.  We can extend this idea to the whole function.


The \textbf{composition of F and G}, symbolized by $F \circ G$ or $F(G)$, is a new function.  Its domain comes from the domain of $G$. Its range comes from the range of $F$.



In this section, we will restrict our study to the composition of linear functions.














\subsection{Expectations}



\begin{sectionOutcomes}
In this section, students will 

\begin{itemize}
\item Identify implied domains.
\item Create formulas for compositions.
\item Create graphs for compositions.
\end{itemize}
\end{sectionOutcomes}

\end{document}
