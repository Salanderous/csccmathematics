\documentclass{ximera}


\graphicspath{
  {./}
  {ximeraTutorial/}
  {basicPhilosophy/}
}

\newcommand{\mooculus}{\textsf{\textbf{MOOC}\textnormal{\textsf{ULUS}}}}

\usepackage{tkz-euclide}\usepackage{tikz}
\usepackage{tikz-cd}
\usetikzlibrary{arrows}
\tikzset{>=stealth,commutative diagrams/.cd,
  arrow style=tikz,diagrams={>=stealth}} %% cool arrow head
\tikzset{shorten <>/.style={ shorten >=#1, shorten <=#1 } } %% allows shorter vectors

\usetikzlibrary{backgrounds} %% for boxes around graphs
\usetikzlibrary{shapes,positioning}  %% Clouds and stars
\usetikzlibrary{matrix} %% for matrix
\usepgfplotslibrary{polar} %% for polar plots
\usepgfplotslibrary{fillbetween} %% to shade area between curves in TikZ
\usetkzobj{all}
\usepackage[makeroom]{cancel} %% for strike outs
%\usepackage{mathtools} %% for pretty underbrace % Breaks Ximera
%\usepackage{multicol}
\usepackage{pgffor} %% required for integral for loops



%% http://tex.stackexchange.com/questions/66490/drawing-a-tikz-arc-specifying-the-center
%% Draws beach ball
\tikzset{pics/carc/.style args={#1:#2:#3}{code={\draw[pic actions] (#1:#3) arc(#1:#2:#3);}}}



\usepackage{array}
\setlength{\extrarowheight}{+.1cm}
\newdimen\digitwidth
\settowidth\digitwidth{9}
\def\divrule#1#2{
\noalign{\moveright#1\digitwidth
\vbox{\hrule width#2\digitwidth}}}






\DeclareMathOperator{\arccot}{arccot}
\DeclareMathOperator{\arcsec}{arcsec}
\DeclareMathOperator{\arccsc}{arccsc}

















%%This is to help with formatting on future title pages.
\newenvironment{sectionOutcomes}{}{}


\title{Piecewise Functions}

\begin{document}

\begin{abstract}
piecewise linear
\end{abstract}
\maketitle




Let's extend our composition of linear functions to piecewise linear funcitons.






\[
T(v) = 
\begin{cases}
  2v-1 & \text{ if }  -4 < v \leq -1 \\
  -v+3 & \text{ if } 1 \leq v < 7
\end{cases}
\]





Graph of $y = T(v)$.
\begin{image}
\begin{tikzpicture}
  \begin{axis}[
            domain=-10:10, ymax=10, xmax=10, ymin=-10, xmin=-10,
            axis lines =center, xlabel=$h$, ylabel={$y=T(v)$}, grid = major,
            ytick={-10,-8,-6,-4,-2,2,4,6,8,10},
          xtick={-10,-8,-6,-4,-2,2,4,6,8,10},
            every axis y label/.style={at=(current axis.above origin),anchor=south},
            every axis x label/.style={at=(current axis.right of origin),anchor=west},
            axis on top
          ]
          
	\addplot [draw=penColor,very thick,smooth,domain=(-4:-1)] {2*x-1};
	\addplot [draw=penColor,very thick,smooth,domain=(1:7)] {-x+3};
	\addplot[color=penColor,only marks,mark=*] coordinates{(-1,-3)}; 
	\addplot[color=penColor,fill=white,only marks,mark=*] coordinates{(-4,-9)}; 
	\addplot[color=penColor,only marks,mark=*] coordinates{(1,2)}; 
	\addplot[color=penColor,fill=white,only marks,mark=*] coordinates{(7,-4)}; 


    \end{axis}
\end{tikzpicture}
\end{image}





\[  \frac{7}{4}x-8  \, \text{ with domain } \, (2, 8)   \]


	


\begin{image}
\begin{tikzpicture}
  \begin{axis}[
            domain=-10:10, ymax=10, xmax=10, ymin=-10, xmin=-10,
            axis lines =center, xlabel=$x$, ylabel={$y=g(x)$}, grid = major,
            ytick={-10,-8,-6,-4,-2,2,4,6,8,10},
          	xtick={-10,-8,-6,-4,-2,2,4,6,8,10},
          	yticklabels={$-10$,$-8$,$-6$,$-4$,$-2$,$2$,$4$,$6$,$8$,$10$}, xticklabels={$-10$,$-8$,$-6$,$-4$,$-2$,$2$,$4$,$6$,$8$,$10$},
            ticklabel style={font=\scriptsize},
            every axis y label/.style={at=(current axis.above origin),anchor=south},
            every axis x label/.style={at=(current axis.right of origin),anchor=west},
            axis on top
          ]
          
      		%\addplot [line width=2, penColor2, smooth,samples=100,domain=(-6:2)] {-2*x-3};
          	\addplot [line width=2, penColor2, smooth,samples=100,domain=(2:8)] {1.75*x-8};




      		%\addplot[color=penColor,fill=penColor2,only marks,mark=*] coordinates{(-6,9)};
      		%\addplot[color=penColor,fill=penColor2,only marks,mark=*] coordinates{(2,-7)};


      		\addplot[color=penColor2,fill=white,only marks,mark=*] coordinates{(2,-4.5)};
      		\addplot[color=penColor2,fill=white,only marks,mark=*] coordinates{(8,6)};


           

  \end{axis}
\end{tikzpicture}
\end{image}








Let's create the composition $T \circ g = T(g)$.


This means the range of $g$ needs to be inside the domain of $T$. But as the chart below shows, there are numbers in
the range of $g$ that are not in the domain of $T$.





For instance, $0$ is in the range of $g$, but not in the domain of $T$. \\

We need to remove the numbers in the domain of $g$, whose funciton value is $0$.


For instance, $-\frac{32}{7}$ is in the domain of $g$, and $g\left(-\frac{32}{7}\right) = 0$. $-\frac{32}{7}$ has to be removed from the domain of $g$.




\begin{image}
	\begin{tikzpicture}
	\begin{axis}[
            domain=-10:10, ymax=2, xmax=10, ymin=-2, xmin=-10,
            %width=3in,
            clip=false,
            axis lines=center,
            %ticks=none,
            %unit vector ratio*=1 1 1,
            ymajorticks=false,
            xtick={-10,-8,-6,-4,-2,2,4,6,8,10},
            %xlabel=$x$, ylabel=$y$,
            %every axis y label/.style={at=(current axis.above origin),anchor=south},
            every axis x label/.style={at=(current axis.right of origin),anchor=west},
          ]      
       
          	%\addplot [line width=2, penColor2, smooth,samples=100,domain=(-7:9)] ({x},{1});
          	\addplot [line width=2, penColor2, smooth,samples=100,domain=(-4.5:6)] ({x},{0.5});
          	%\node at (axis cs:-7,1) [penColor2] {$[$};
          	%\node at (axis cs:9,1) [penColor2] {$]$};
          	\node at (axis cs:-4.5,0.5) [penColor2] {$($};
          	\node at (axis cs:6,0.5) [penColor2] {$)$};
          	\node at (axis cs:5,1) [penColor2] {Range of $g$};




          	\addplot [line width=2, penColor, smooth,samples=100,domain=(-4:-1)] ({x},{-0.5});
          	\addplot [line width=2, penColor, smooth,samples=100,domain=(1:7)] ({x},{-0.5});
          	\node at (axis cs:-4,-0.5) [penColor] {$($};
          	\node at (axis cs:-1,-0.5) [penColor] {$]$};
          	\node at (axis cs:1,-0.5) [penColor] {$[$};
          	\node at (axis cs:7,-0.5) [penColor] {$)$};
          	\node at (axis cs:5,-1) [penColor] {Domain of $T$};




    \end{axis}
	\end{tikzpicture}
	\end{image}




We need to map the domain of $T$ back onto the range of $g$ and then work our way back into the domain of $g$, in order to restrict the domain of $g$ to just the numbers that work in the composition.






\begin{image}
	\begin{tikzpicture}
	\begin{axis}[
            domain=-10:10, ymax=2, xmax=10, ymin=-2, xmin=-10,
            %width=3in,
            clip=false,
            axis lines=center,
            %ticks=none,
            %unit vector ratio*=1 1 1,
            ymajorticks=false,
            xtick={-10,-8,-6,-4,-2,2,4,6,8,10},
            %xlabel=$x$, ylabel=$y$,
            %every axis y label/.style={at=(current axis.above origin),anchor=south},
            every axis x label/.style={at=(current axis.right of origin),anchor=west},
          ]      
       
          	%\addplot [line width=2, penColor2, smooth,samples=100,domain=(-7:9)] ({x},{1});
          	\addplot [line width=2, penColor2, smooth,samples=100,domain=(-4.5:6)] ({x},{0.5});
          	%\node at (axis cs:-7,1) [penColor2] {$[$};
          	%\node at (axis cs:9,1) [penColor2] {$]$};
          	\node at (axis cs:-4.5,0.5) [penColor2] {$($};
          	\node at (axis cs:6,0.5) [penColor2] {$)$};
          	\node at (axis cs:5,1) [penColor2] {Range of $g$};




          	\addplot [line width=2, penColor, smooth,samples=100,domain=(-4:-1)] ({x},{-0.5});
          	\addplot [line width=2, penColor, smooth,samples=100,domain=(1:7)] ({x},{-0.5});
          	\node at (axis cs:-4,-0.5) [penColor] {$($};
          	\node at (axis cs:-1,-0.5) [penColor] {$]$};
          	\node at (axis cs:1,-0.5) [penColor] {$[$};
          	\node at (axis cs:7,-0.5) [penColor] {$)$};
          	\node at (axis cs:5,-1) [penColor] {Domain of $T$};

          	\addplot [line width=1, gray, dashed,samples=100,domain=(-1:1)] ({-4},{x});
          	\addplot [line width=1, gray, dashed,samples=100,domain=(-1:1)] ({-1},{x});
          	\addplot [line width=1, gray, dashed,samples=100,domain=(-1:1)] ({1},{x});
          	\addplot [line width=1, gray, dashed,samples=100,domain=(-1:1)] ({6},{x});




    \end{axis}
	\end{tikzpicture}
	\end{image}



We need to find the preimages of $-4$, $-1$, $1$, and $6$ in $g$.





\section{Domain of $T \circ g$}




\begin{itemize}
\item $g(x) = \frac{7}{4}x -8 = -4$
\item $\frac{7}{4}x = 4$
\item $x = \frac{16}{7} \approx 2.3$
\end{itemize}



\begin{itemize}
\item $g(x) = \frac{7}{4}x -8 = -1$
\item $\frac{7}{4}x = 7$
\item $x = 4$
\end{itemize}


\begin{itemize}
\item $g(x) = \frac{7}{4}x -8 = 1$
\item $\frac{7}{4}x = 9$
\item $x = \frac{36}{7} \approx 5.1$
\end{itemize}


\begin{itemize}
\item $g(x) = \frac{7}{4}x -8 = 6$
\item $\frac{7}{4}x = 14$
\item $x = 8$
\end{itemize}



The domain of $T \circ g$ is $\left(\frac{16}{7}, 4\right] \cup \left[\frac{36}{7}, 8\right)$ \\


\textbf{Now for the formulas} \\


We have a composition: $T \circ g$.  We know this is a linear function, because it is the composition of two linear functions. \\

Our composition has a formula, let's use $c$ for its variable. \\


On $\left(\frac{16}{7}, 4\right]$, we will have $T(v) = 2v - 1$.  \\



$(T \circ g)(c) = 2\left(\frac{7}{4}c - 8\right) - 1 = \frac{7}{2}c - 17$ \\





On $\left[\frac{36}{7}, 8\right)$,  we will have $T(v) = -v + 3$. \\

$(T \circ g)(c) = -\left(\frac{7}{4}c - 8\right) + 3 = -\frac{7}{4}c + 11$










\[
T(g(c)) = 
\begin{cases}
  \frac{7}{2}c - 17     &     \text{ on }  \left(\frac{16}{7}, 4\right] \\
  -\frac{7}{4}c + 11   &     \text{ on }  \left[\frac{36}{7}, 8\right)
\end{cases}
\]















Graph of $y = T(g(c))$.

\begin{image}
\begin{tikzpicture}
  \begin{axis}[
            domain=-10:10, ymax=10, xmax=10, ymin=-10, xmin=-10,
            axis lines =center, xlabel=$c$, ylabel=$y$, grid = major,
            ytick={-10,-8,-6,-4,-2,2,4,6,8,10},
          	xtick={-10,-8,-6,-4,-2,2,4,6,8,10},
            every axis y label/.style={at=(current axis.above origin),anchor=south},
            every axis x label/.style={at=(current axis.right of origin),anchor=west},
            axis on top
          ]
          
			\addplot [line width=2, black, smooth,samples=100,domain=(2.3:4)] {3.5*x-17};
			\addplot [line width=2, black, smooth,samples=100,domain=(5.14:8)] {-1.75*x+11};
			\addplot[color=black,only marks,mark=*] coordinates{(2.3,-9)}; 
			\addplot[color=black,fill=white,only marks,mark=*] coordinates{(4,-3)}; 
			\addplot[color=black,only marks,mark=*] coordinates{(5.1,2)}; 
			\addplot[color=black,fill=white,only marks,mark=*] coordinates{(8,-3)}; 


    \end{axis}
\end{tikzpicture}
\end{image}





















\end{document}
