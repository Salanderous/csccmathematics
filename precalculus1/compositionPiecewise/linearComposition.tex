\documentclass{ximera}


\graphicspath{
  {./}
  {ximeraTutorial/}
  {basicPhilosophy/}
}

\newcommand{\mooculus}{\textsf{\textbf{MOOC}\textnormal{\textsf{ULUS}}}}

\usepackage{tkz-euclide}\usepackage{tikz}
\usepackage{tikz-cd}
\usetikzlibrary{arrows}
\tikzset{>=stealth,commutative diagrams/.cd,
  arrow style=tikz,diagrams={>=stealth}} %% cool arrow head
\tikzset{shorten <>/.style={ shorten >=#1, shorten <=#1 } } %% allows shorter vectors

\usetikzlibrary{backgrounds} %% for boxes around graphs
\usetikzlibrary{shapes,positioning}  %% Clouds and stars
\usetikzlibrary{matrix} %% for matrix
\usepgfplotslibrary{polar} %% for polar plots
\usepgfplotslibrary{fillbetween} %% to shade area between curves in TikZ
\usetkzobj{all}
\usepackage[makeroom]{cancel} %% for strike outs
%\usepackage{mathtools} %% for pretty underbrace % Breaks Ximera
%\usepackage{multicol}
\usepackage{pgffor} %% required for integral for loops



%% http://tex.stackexchange.com/questions/66490/drawing-a-tikz-arc-specifying-the-center
%% Draws beach ball
\tikzset{pics/carc/.style args={#1:#2:#3}{code={\draw[pic actions] (#1:#3) arc(#1:#2:#3);}}}



\usepackage{array}
\setlength{\extrarowheight}{+.1cm}
\newdimen\digitwidth
\settowidth\digitwidth{9}
\def\divrule#1#2{
\noalign{\moveright#1\digitwidth
\vbox{\hrule width#2\digitwidth}}}






\DeclareMathOperator{\arccot}{arccot}
\DeclareMathOperator{\arcsec}{arcsec}
\DeclareMathOperator{\arccsc}{arccsc}

















%%This is to help with formatting on future title pages.
\newenvironment{sectionOutcomes}{}{}


\title{Linear Composition}

\begin{document}

\begin{abstract}
linear linear linear
\end{abstract}
\maketitle


The \textbf{composition of two functions}, $F$ and $G$, is a new operation on functions, symbolized by $F \circ G$, which produces a new function.


\begin{definition} \textbf{\textcolor{green!50!black}{Composition}} 
Given $a$ in the domain of $G$, such that $G(a)$ is in the domain of $F$, $(F \circ G)(a)$ is defined to be  $F(G(a))$.


If $Dom_G$ and $Dom_F$ represent the domains of $G$ and $F$ respectively, the induced domain of $F \circ G$ is 




\[  Dom_{F \circ G} = \{  g \in  Dom_G  \,   |   \,   G(g) \in Dom_F                 \}           \]

\end{definition}

The order matters.  Generally speaking, $F \circ G \ne G \circ F$


















\section{Linear Composition}

In this section, we will explore the composition of linear functions. \\


Let $A(x) = a_1 \cdot x + a_0$ be a linear function with $\mathbb{R}$ as its domain. \\
Let $B(w) = b_1 \cdot w + b_0$ be a linear function with $\mathbb{R}$ as its domain. \\


Then $A \circ B$ is a new function. What is its formula? \\

To keep everything separate, let's use $v$ as the variable for the formula of $A \circ B$.


If we write $A \circ B(v)$, then people will miss read this to say $B(v)$.  So, let's put parentheses around $A \circ B$: $(A \circ B)(v)$.







\begin{align*}
(A \circ B)(v) & = A(B(v)) \\
& = A(b_1 \cdot v + b_0)  \\
& = a_1 \cdot (b_1 \cdot v + b_0) + a_0  \\
& = a_1 \cdot b_1 \cdot v + a_1 \cdot b_0 + a_0    \\
& = (a_1 \cdot b_1) v + (a_1 \cdot b_0 + a_0)
\end{align*}






The composition of two linear functions is again a linear function.

The rate-of-change of the composition is the \wordChoice{\choice{sum} \choice{difference} \choice[correct]{product}\choice{quotient}} of the rates-of-change of the original linear functions.





\begin{example} Linear Composition

Let $G(y) = 4 y + 3$ be a linear function with $\mathbb{R}$ as its domain. \\
Let $H(t) = 3 t -5$ be a linear function with $\mathbb{R}$ as its domain. \\


Define two new functions, $U(x)$ and $V(w)$, as compositions by

$U(x) = (G \circ H)(x) = G(H(x)) = 4 \left(\answer{3 x - 5}\right) + 3 = 12 x -17$ \\
$V(w) = (H \circ G)(x) =H(G(w)) = 3 \left(\answer{4 w + 3}\right) - 5 = 12 w + 4$ \\


Two different functions. Both are linear functions. The constant rate-of-change is $12$ for both. \\

Their graphs would be parallel lines. \\


\end{example}








\begin{example} Linear Composition

Let $k(m) = 2 m + 1$ be a linear function with $\mathbb{R}$ as its domain. \\
Let $h(n)$ be a linear function with $\mathbb{R}$ as its domain. \\


Let $r(v) = (k \circ h)(v) = k(h(v)) = 3 v - 2$ 

$\blacktriangleright$ Determine a formula for $h(n)$.





\begin{explanation}


$k$ is linear and the composition, $r$, is linear. A good guess is that $h$ is linear.  

In that case the rates-of-change would mulitply and $h$ would start off looking like $h(n) = \frac{3}{2} n + n_0$, for some $n_0$.


So, far we have 

\[   r(v) = k(h(v)) = 2 \left(\frac{3}{2} v + n_0\right) + 1  = 3v + 2 n_0 + 1\]


We need $2 n_0 + 1 = -2$.  Therefore, $n_0 = \answer{-\frac{3}{2}}$.


We have $h(n) = \frac{3}{2} n - \frac{3}{2}$


\end{explanation}
\end{example}



















\begin{example} Linear Composition

Let $k(m) = 2 m + 1$ be a linear function with \textbf{$R$} as its domain. \\
Let $h(n)$ be a linear function with \textbf{$R$} as its domain. \\


Let $r(v) = (h \circ k)(v) = h(k(v)) = 3 v - 2$ 

$\blacktriangleright$  Determine a formula for $h(n)$.




\begin{explanation}


$k$ is linear and the composition, $r$, is linear. A good guess is that $h$ is linear.  

In that case the rates-of-change would mulitply and $h$ would start off looking like $h(n) = \frac{3}{2} n + n_0$, for some $n_0$.


So, far we have 

\[   r(v) = h(k(v)) = \frac{3}{2} \left( \answer{2 v + 1} \right) + n_0 =  3 v + \frac{3}{2} + n_0\]


We need $\frac{3}{2} + n_0 = -2$.  Therefore, $n_0 = \answer{-\frac{7}{2}}$.


We have $h(n) = \frac{3}{2} n - \frac{7}{2}$


\end{explanation}
\end{example}








\section{Restricted Domains}


Let's restrict the domain of our linear functions.









\[  T(v) = -v+3  \, \text{ with domain } \,  [1,7)   \]






Graph of $y = T(v)$.
\begin{image}
\begin{tikzpicture}
  \begin{axis}[
            domain=-10:10, ymax=10, xmax=10, ymin=-10, xmin=-10,
            axis lines =center, xlabel=$v$, ylabel={$y=T(v)$}, grid = major,
            ytick={-10,-8,-6,-4,-2,2,4,6,8,10},
            xtick={-10,-8,-6,-4,-2,2,4,6,8,10},
            ticklabel style={font=\scriptsize},
            every axis y label/.style={at=(current axis.above origin),anchor=south},
            every axis x label/.style={at=(current axis.right of origin),anchor=west},
            axis on top
          ]
          
	%\addplot [draw=penColor,very thick,smooth,domain=(-4:-1)] {2*x-1};
	\addplot [draw=penColor,very thick,smooth,domain=(1:7)] {-x+3};
	%\addplot[color=penColor,only marks,mark=*] coordinates{(-1,-3)}; 
	%\addplot[color=penColor,fill=white,only marks,mark=*] coordinates{(-4,-9)}; 
	\addplot[color=penColor,only marks,mark=*] coordinates{(1,2)}; 
	\addplot[color=penColor,fill=white,only marks,mark=*] coordinates{(7,-4)}; 


    \end{axis}
\end{tikzpicture}
\end{image}





\[      g(x) = \frac{7}{4}x-8 \, \text{ with domain } \, (2,8)   \]


	


\begin{image}
\begin{tikzpicture}
  \begin{axis}[
            domain=-10:10, ymax=10, xmax=10, ymin=-10, xmin=-10,
            axis lines =center, xlabel=$x$, ylabel={$y=g(x)$}, grid = major,
            ytick={-10,-8,-6,-4,-2,2,4,6,8,10},
          	xtick={-10,-8,-6,-4,-2,2,4,6,8,10},
          	yticklabels={$-10$,$-8$,$-6$,$-4$,$-2$,$2$,$4$,$6$,$8$,$10$}, xticklabels={$-10$,$-8$,$-6$,$-4$,$-2$,$2$,$4$,$6$,$8$,$10$},
            ticklabel style={font=\scriptsize},
            every axis y label/.style={at=(current axis.above origin),anchor=south},
            every axis x label/.style={at=(current axis.right of origin),anchor=west},
            axis on top
          ]
          
      		%\addplot [line width=2, penColor2, smooth,samples=100,domain=(-6:2)] {-2*x-3};
          	\addplot [line width=2, penColor2, smooth,samples=100,domain=(2:8)] {1.75*x-8};

      		%\addplot[color=penColor,fill=penColor2,only marks,mark=*] coordinates{(-6,9)};
      		%\addplot[color=penColor,fill=penColor2,only marks,mark=*] coordinates{(2,-7)};

      		\addplot[color=penColor2,fill=white,only marks,mark=*] coordinates{(2,-4.5)};
      		\addplot[color=penColor2,fill=white,only marks,mark=*] coordinates{(8,6)};


           

  \end{axis}
\end{tikzpicture}
\end{image}






Let's create the composition $T \circ g = T(g)$.


This means the range of $g$ needs to be inside the domain of $T$. But as the chart below shows, there are numbers in
the range of $g$ that are not in the domain of $T$.





For instance, $0$ is in the range of $g$, but not in the domain of $T$. \\

We need to remove the numbers in the domain of $g$, whose function value is $0$.


Again, $-\frac{32}{7}$ is in the domain of $g$, and $g\left(-\frac{32}{7}\right) = 0$. $-\frac{32}{7}$ has to be removed from the domain of $g$.









\begin{image}
	\begin{tikzpicture}
	\begin{axis}[
            domain=-10:10, ymax=2, xmax=10, ymin=-2, xmin=-10,
            %width=3in,
            clip=false,
            axis lines=center,
            %ticks=none,
            %unit vector ratio*=1 1 1,
            ymajorticks=false,
            xtick={-10,-8,-6,-4,-2,2,4,6,8,10},
            %xlabel=$x$, ylabel=$y$,
            %every axis y label/.style={at=(current axis.above origin),anchor=south},
            every axis x label/.style={at=(current axis.right of origin),anchor=west},
          ]      
       
          	%\addplot [line width=2, penColor2, smooth,samples=100,domain=(-7:9)] ({x},{1});
          	\addplot [line width=2, penColor2, smooth,samples=100,domain=(-4.5:6)] ({x},{0.5});
          	%\node at (axis cs:-7,1) [penColor2] {$[$};
          	%\node at (axis cs:9,1) [penColor2] {$]$};
          	\node at (axis cs:-4.5,0.5) [penColor2] {$($};
          	\node at (axis cs:6,0.5) [penColor2] {$)$};
          	\node at (axis cs:5,1) [penColor2] {Range of $g$};




          	%\addplot [line width=2, penColor, smooth,samples=100,domain=(-4:-1)] ({x},{-0.5});
          	\addplot [line width=2, penColor, smooth,samples=100,domain=(1:7)] ({x},{-0.5});
          	%\node at (axis cs:-4,-0.5) [penColor] {$($};
          	%\node at (axis cs:-1,-0.5) [penColor] {$]$};
          	\node at (axis cs:1,-0.5) [penColor] {$[$};
          	\node at (axis cs:7,-0.5) [penColor] {$)$};
          	\node at (axis cs:5,-1) [penColor] {Domain of $T$};




    \end{axis}
	\end{tikzpicture}
	\end{image}


What part of the range of $g$ can we use?








\begin{image}
	\begin{tikzpicture}
	\begin{axis}[
            domain=-10:10, ymax=2, xmax=10, ymin=-2, xmin=-10,
            %width=3in,
            clip=false,
            axis lines=center,
            %ticks=none,
            %unit vector ratio*=1 1 1,
            ymajorticks=false,
            xtick={-10,-8,-6,-4,-2,2,4,6,8,10},
            %xlabel=$x$, ylabel=$y$,
            %every axis y label/.style={at=(current axis.above origin),anchor=south},
            every axis x label/.style={at=(current axis.right of origin),anchor=west},
          ]      
       
          	%\addplot [line width=2, penColor2, smooth,samples=100,domain=(-7:9)] ({x},{1});
          	\addplot [line width=2, penColor2, smooth,samples=100,domain=(-4.5:6)] ({x},{0.5});
          	%\node at (axis cs:-7,1) [penColor2] {$[$};
          	%\node at (axis cs:9,1) [penColor2] {$]$};
          	\node at (axis cs:-4.5,0.5) [penColor2] {$($};
          	\node at (axis cs:6,0.5) [penColor2] {$)$};
          	\node at (axis cs:5,1) [penColor2] {Range of $g$};




          	%\addplot [line width=2, penColor, smooth,samples=100,domain=(-4:-1)] ({x},{-0.5});
          	\addplot [line width=2, penColor, smooth,samples=100,domain=(1:7)] ({x},{-0.5});
          	%\node at (axis cs:-4,-0.5) [penColor] {$($};
          	%\node at (axis cs:-1,-0.5) [penColor] {$]$};
          	\node at (axis cs:1,-0.5) [penColor] {$[$};
          	\node at (axis cs:7,-0.5) [penColor] {$)$};
          	\node at (axis cs:5,-1) [penColor] {Domain of $T$};




          	\addplot [line width=1, gray, dashed,smooth,samples=100,domain=(-1:1)] ({1},{x});
          	\addplot [line width=1, gray, dashed,smooth,samples=100,domain=(-1:1)] ({6},{x});




    \end{axis}
	\end{tikzpicture}
	\end{image}

It looks like we can use the interval $\left[ \answer{1}, \answer{6} \right)$.   $6$ is not included, so we'll have to keep this in mind as we investigate the domain of $g$.


The range of $g$ needs to be restricted to $[1, 6)$.  Therefore we need to restrict the domain of $g$, so that the range is just $[1, 6)$. We need to find the \textbf{preimages} of $6$ and $1$ in the domain of $g$.




\begin{itemize}
\item $g(x) = \frac{7}{4}x-8 = 6$
\item $\frac{7}{4}x = \answer{14}$
\item $x = \frac{4}{7} \cdot 14 = \answer{8}$
\end{itemize}


This restriction on $g$ excludes $8$.  The domain of $g$ is $(2,8)$. $8$ is an endpoint of the interval that is already excluded.


\begin{itemize}
\item $g(x) = \frac{7}{4}x-8 = 1$
\item $\frac{7}{4}x = 9$
\item $x = \frac{4}{7} \cdot 9 = \frac{36}{7} \approx 5.14$
\end{itemize}

$\frac{36}{7}$ is inside $(2,8)$.

We need to restrict the domain of $g$ to $\left[ \answer{\frac{36}{7}}, \answer{8} \right)$


With this restricted domain, the range of $g$ will be $[1, 6)$, which is the biggest piece of the domain of $T$ that we can get from values of $g$.

We now have a domain for our composition: $Dom_{T \circ g} = \left[\frac{36}{7}, 8\right)$



\subsection{Halfway}



Now we need to take our restricted domain of $T$, $[1, 6)$, and determine the restricted range.



\begin{itemize}
\item $T(1) = -1+3 = 2$ - included\\
\item $T(6) = -6+3 = -3$ - excluded \\
\end{itemize}







\textbf{What do we have?}


We have a composition: $T \circ g$.  We know this is a linear function, because it is the composition of two linear functions. \\

Our composition has a formula, let's use $c$ for its variable. \\

$(T \circ g)(c) = -\left(\frac{7}{4}c - 8\right) + 3 =  -\frac{7}{4}c + 11$ \\

It's domain is $\left[ \answer{\frac{36}{7}}, \answer{8} \right)$.


It's range is $\left[ \answer{-3}, \answer{2} \right)$.






Graph of $y = T(g(c))$.
\begin{image}
\begin{tikzpicture}
  \begin{axis}[
            domain=-10:10, ymax=10, xmax=10, ymin=-10, xmin=-10,
            axis lines =center, xlabel=$c$, ylabel={$y=T(g(c))$}, grid = major,
            ytick={-10,-8,-6,-4,-2,2,4,6,8,10},
            xtick={-10,-8,-6,-4,-2,2,4,6,8,10},
            ticklabel style={font=\scriptsize},
            every axis y label/.style={at=(current axis.above origin),anchor=south},
            every axis x label/.style={at=(current axis.right of origin),anchor=west},
            axis on top
          ]
          
	%\addplot [draw=penColor,very thick,smooth,domain=(-4:-1)] {2*x-1};
	\addplot [draw=penColor,very thick,smooth,domain=(5.14:8)] {-1.75*x+11};
	%\addplot[color=penColor,only marks,mark=*] coordinates{(-1,-3)}; 
	%\addplot[color=penColor,fill=white,only marks,mark=*] coordinates{(-4,-9)}; 
	\addplot[color=penColor,only marks,mark=*] coordinates{(5.14,2)}; 
	\addplot[color=penColor,fill=white,only marks,mark=*] coordinates{(8,-3)}; 


    \end{axis}
\end{tikzpicture}
\end{image}










\end{document}
