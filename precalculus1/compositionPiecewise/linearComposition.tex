\documentclass{ximera}


\graphicspath{
  {./}
  {ximeraTutorial/}
  {basicPhilosophy/}
}

\newcommand{\mooculus}{\textsf{\textbf{MOOC}\textnormal{\textsf{ULUS}}}}

\usepackage{tkz-euclide}\usepackage{tikz}
\usepackage{tikz-cd}
\usetikzlibrary{arrows}
\tikzset{>=stealth,commutative diagrams/.cd,
  arrow style=tikz,diagrams={>=stealth}} %% cool arrow head
\tikzset{shorten <>/.style={ shorten >=#1, shorten <=#1 } } %% allows shorter vectors

\usetikzlibrary{backgrounds} %% for boxes around graphs
\usetikzlibrary{shapes,positioning}  %% Clouds and stars
\usetikzlibrary{matrix} %% for matrix
\usepgfplotslibrary{polar} %% for polar plots
\usepgfplotslibrary{fillbetween} %% to shade area between curves in TikZ
\usetkzobj{all}
\usepackage[makeroom]{cancel} %% for strike outs
%\usepackage{mathtools} %% for pretty underbrace % Breaks Ximera
%\usepackage{multicol}
\usepackage{pgffor} %% required for integral for loops



%% http://tex.stackexchange.com/questions/66490/drawing-a-tikz-arc-specifying-the-center
%% Draws beach ball
\tikzset{pics/carc/.style args={#1:#2:#3}{code={\draw[pic actions] (#1:#3) arc(#1:#2:#3);}}}



\usepackage{array}
\setlength{\extrarowheight}{+.1cm}
\newdimen\digitwidth
\settowidth\digitwidth{9}
\def\divrule#1#2{
\noalign{\moveright#1\digitwidth
\vbox{\hrule width#2\digitwidth}}}






\DeclareMathOperator{\arccot}{arccot}
\DeclareMathOperator{\arcsec}{arcsec}
\DeclareMathOperator{\arccsc}{arccsc}

















%%This is to help with formatting on future title pages.
\newenvironment{sectionOutcomes}{}{}


\title{Linear Composition}

\begin{document}

\begin{abstract}
linear linear linear
\end{abstract}
\maketitle


The \textbf{composition of two functions}, $F$ and $G$, is a new operation on functions, symbolized by $F \circ G$, which produces a new function.


\begin{definition}
Given $a$ in the domain of $G$, with $G(a)$ in the domain of $F$, $(F \circ G)(a)$ is defined to be  $F(G(a))$.


If $Dom_G$ and $Dom_F$ represent the domains of $G$ and $F$ respectively, the induced domain of $F \circ G$ is 




\[  Dom_{F \circ G} = \{  g \in  Dom_G  \,   |   \,   G(g) \in Dom_F                 \}           \]

\end{definition}





\section{Linear Composition}




Let $A(x) = a_1 \cdot x + a_0$ be a linear function with \textbf{$R$} as its domain. \\
Let $B(w) = b_1 \cdot w + b_0$ be a linear function with \textbf{$R$} as its domain. \\


Then $A \circ B$ is a new function. What is its formula?

To keep everything separate, let's use $v$ as the variable for the formula of $A \circ B$.


If we write $A \circ B(v)$, then people will miss read this to say $B(v)$.  So, let's put parentheses around $A \circ B$: $(A \circ B)(v)$.







\begin{align*}
(A \circ B)(v) & = A(B(v)) \\
& = A(b_1 \cdot v + b_0)  \\
& = a_1 \cdot (b_1 \cdot v + b_0) + a_0  \\
& = a_1 \cdot b_1 \cdot v + a_1 \cdot b_0 + a_0    \\
& = (a_1 \cdot b_1 \cdot) v + (a_1 \cdot b_0 + a_0)
\end{align*}






The composition of two linear functions is again a linear function.

The rate-of-change of the composition is the product of the rates-of-change of the original linear functions.





\begin{example} Linear Composition

Let $G(y) = 4 x + 3$ be a linear function with \textbf{$R$} as its domain. \\
Let $H(t) = 3 w -5$ be a linear function with \textbf{$R$} as its domain. \\


Define two new functions, $U(x)$ and $V(w)$, as compositions by

$U(x) = G(H(x)) = 4 (3 x - 5) + 3 = 12 x -17$ \\
$V(w) = H(G(w)) = 3 (4 w + 3) - 5 = 12 w + 4$ \\


Two different function. Both are linear functions. The constant rate-of-change is $12$ for both. \\

Their graphs would be parallel lines.


\end{example}








\begin{example} Linear Composition

Let $k(m) = 2 m + 1$ be a linear function with \textbf{$R$} as its domain. \\
Let $h(n)$ be a linear function with \textbf{$R$} as its domain. \\


Let $r(v) = k(h(v)) = 3 v - 2$ 

Determine a formula for $h(n)$.




\textbf{\scriptsize{\textcolor{purple!50!blue!90!black}{SOLUTION}}}



\end{example}


















\end{document}
