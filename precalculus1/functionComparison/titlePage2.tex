\documentclass{ximera}


\graphicspath{
  {./}
  {ximeraTutorial/}
  {basicPhilosophy/}
}

\newcommand{\mooculus}{\textsf{\textbf{MOOC}\textnormal{\textsf{ULUS}}}}

\usepackage{tkz-euclide}\usepackage{tikz}
\usepackage{tikz-cd}
\usetikzlibrary{arrows}
\tikzset{>=stealth,commutative diagrams/.cd,
  arrow style=tikz,diagrams={>=stealth}} %% cool arrow head
\tikzset{shorten <>/.style={ shorten >=#1, shorten <=#1 } } %% allows shorter vectors

\usetikzlibrary{backgrounds} %% for boxes around graphs
\usetikzlibrary{shapes,positioning}  %% Clouds and stars
\usetikzlibrary{matrix} %% for matrix
\usepgfplotslibrary{polar} %% for polar plots
\usepgfplotslibrary{fillbetween} %% to shade area between curves in TikZ
\usetkzobj{all}
\usepackage[makeroom]{cancel} %% for strike outs
%\usepackage{mathtools} %% for pretty underbrace % Breaks Ximera
%\usepackage{multicol}
\usepackage{pgffor} %% required for integral for loops



%% http://tex.stackexchange.com/questions/66490/drawing-a-tikz-arc-specifying-the-center
%% Draws beach ball
\tikzset{pics/carc/.style args={#1:#2:#3}{code={\draw[pic actions] (#1:#3) arc(#1:#2:#3);}}}



\usepackage{array}
\setlength{\extrarowheight}{+.1cm}
\newdimen\digitwidth
\settowidth\digitwidth{9}
\def\divrule#1#2{
\noalign{\moveright#1\digitwidth
\vbox{\hrule width#2\digitwidth}}}






\DeclareMathOperator{\arccot}{arccot}
\DeclareMathOperator{\arcsec}{arcsec}
\DeclareMathOperator{\arccsc}{arccsc}

















%%This is to help with formatting on future title pages.
\newenvironment{sectionOutcomes}{}{}


\title{Comparing Functions}

\begin{document}

\begin{abstract}
%Stuff can go here later if we want!
\end{abstract}
\maketitle







Our functions eventually settle down into simple patterns.  

The patterns include

\begin{itemize}
\item approach a constant function
\item approach a linear function
\item continue an oscillating pattern
\item approach a some basic elementary function 
\end{itemize}


Mostly, we are interested in approaching a constant function.  These correspond to horizontal asymptotes on the graph.

Then we are interested in linear patterns (oblique asymptotes)

Then we are interesting in if the function becomes unbounded.

Mostly.


But, even more important than these end-behavior discoveries is \textbf{HOW} we decide.

Establishing these end-behaviors often requires a comparison of end-behaviors between functions.  


If functions are each tending to infinity, do they do this differently?


If functions are each tending to $0$, do they do this differently?















\subsection{Learning Outcomes}


\begin{sectionOutcomes}
In this section, students will 

\begin{itemize}
\item compare endbehaviors of functions.
\item establish an order of dominance.
\end{itemize}
\end{sectionOutcomes}
















\begin{center}
\textbf{\textcolor{green!50!black}{ooooo=-=-=-=-=-=-=-=-=-=-=-=-=ooOoo=-=-=-=-=-=-=-=-=-=-=-=-=ooooo}} \\

more examples can be found by following this link\\ \link[More Examples of Formulas]{https://ximera.osu.edu/csccmathematics/precalculus1/precalculus1/formulas/examples/exampleList}

\end{center}







\begin{center}
\textbf{\textcolor{green!50!black}{ooooo=-=-=-=-=-=-=-=-=-=-=-=-=ooOoo=-=-=-=-=-=-=-=-=-=-=-=-=ooooo}} \\

more examples can be found by following this link\\ \link[More Examples of Dominance]{https://ximera.osu.edu/csccmathematics/precalculus1/precalculus1/functionComparison/examples/exampleList}

\end{center}










\end{document}
