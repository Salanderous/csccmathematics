\documentclass{ximera}


\graphicspath{
  {./}
  {ximeraTutorial/}
  {basicPhilosophy/}
}

\newcommand{\mooculus}{\textsf{\textbf{MOOC}\textnormal{\textsf{ULUS}}}}

\usepackage{tkz-euclide}\usepackage{tikz}
\usepackage{tikz-cd}
\usetikzlibrary{arrows}
\tikzset{>=stealth,commutative diagrams/.cd,
  arrow style=tikz,diagrams={>=stealth}} %% cool arrow head
\tikzset{shorten <>/.style={ shorten >=#1, shorten <=#1 } } %% allows shorter vectors

\usetikzlibrary{backgrounds} %% for boxes around graphs
\usetikzlibrary{shapes,positioning}  %% Clouds and stars
\usetikzlibrary{matrix} %% for matrix
\usepgfplotslibrary{polar} %% for polar plots
\usepgfplotslibrary{fillbetween} %% to shade area between curves in TikZ
\usetkzobj{all}
\usepackage[makeroom]{cancel} %% for strike outs
%\usepackage{mathtools} %% for pretty underbrace % Breaks Ximera
%\usepackage{multicol}
\usepackage{pgffor} %% required for integral for loops



%% http://tex.stackexchange.com/questions/66490/drawing-a-tikz-arc-specifying-the-center
%% Draws beach ball
\tikzset{pics/carc/.style args={#1:#2:#3}{code={\draw[pic actions] (#1:#3) arc(#1:#2:#3);}}}



\usepackage{array}
\setlength{\extrarowheight}{+.1cm}
\newdimen\digitwidth
\settowidth\digitwidth{9}
\def\divrule#1#2{
\noalign{\moveright#1\digitwidth
\vbox{\hrule width#2\digitwidth}}}






\DeclareMathOperator{\arccot}{arccot}
\DeclareMathOperator{\arcsec}{arcsec}
\DeclareMathOperator{\arccsc}{arccsc}

















%%This is to help with formatting on future title pages.
\newenvironment{sectionOutcomes}{}{}


\title{Indeterminate}

\begin{document}

\begin{abstract}
equally fast
\end{abstract}
\maketitle





Functions can dominate other functions, or they can approach infinity similarly.  If the two functions approach infinity atthe same rate, then the limit of their quotient equals a number.  The graph of the quotient approaches as horizontal asymptote.






\begin{example}

Consider the function $B(t) = \frac{3e^t - 5}{e^t + 1}$. \\

As $t \to \infty$, the exponential functions will dominate the constant functions.  

\[   B(t) = \frac{3e^t - 5}{e^t + 1} \sim \frac{3 e^t}{e^t} = 3   \]




As $t \to -\infty$, we have a different story.  $\lim_{t \to -\infty} e^t = 0$  In this case, the constant terms domainate.


\[   B(t) = \frac{3e^t - 5}{e^t + 1} \sim \frac{-5}{1} = -5   \]


The graph has two different horizontal asymptotes.








\begin{image}
\begin{tikzpicture}
  \begin{axis}[
            domain=-10:10, ymax=10, xmax=10, ymin=-10, xmin=-10,
            axis lines =center, xlabel=$x$, ylabel={$y=g(x)$}, grid = major,
            ytick={-10,-8,-6,-4,-2,2,4,6,8,10},
            xtick={-10,-8,-6,-4,-2,2,4,6,8,10},
            yticklabels={$-10$,$-8$,$-6$,$-4$,$-2$,$2$,$4$,$6$,$8$,$10$}, xticklabels={$-10$,$-8$,$-6$,$-4$,$-2$,$2$,$4$,$6$,$8$,$10$},
            ticklabel style={font=\scriptsize},
            every axis y label/.style={at=(current axis.above origin),anchor=south},
            every axis x label/.style={at=(current axis.right of origin),anchor=west},
            axis on top
          ]
          
          %\addplot [line width=2, penColor2, smooth,samples=100,domain=(-6:2)] {-2*x-3};
            \addplot [line width=2, penColor2, smooth,samples=100,domain=(-8:8),<->] {(3*(e^x) - 5)/(e^x + 1))};

            \addplot [line width=1, gray, dashed,samples=200,domain=(-9:9),<->] {3};
            \addplot [line width=1, gray, dashed,samples=200,domain=(-9:9),<->] {-5};




           

  \end{axis}
\end{tikzpicture}
\end{image}





\end{example}








\section{Indeterminate Forms}


We have been examining functions that approach infinity at the same rate.  The problem is that a fraction of the form $\frac{very big}{very big}$ could equal anything.



\[  \frac{10000000000}{2000000000} = 50      \]

\[  \frac{1000000000}{2000000000} = \frac{1}{2}      \]

\[  \frac{1000000000}{20000000000} = 0.2      \]





The same thing happens with values near $0$


\[  \frac{0.000000001}{0.0000000002} = 50      \]

\[  \frac{0.000000001}{0.000000002} = \frac{1}{2}      \]

\[  \frac{0.0000000001}{0.0000000002} = 0.2      \]





Functions whose values seem like they are approaching $\frac{\infty}{\infty}$ or $\frac{0}{0}$ are called \textbf{indeterminate forms}, because you can't easily determine their values.




\begin{example} Indeterminate Forms

Both the functions $S(t) = sin(t)$ and $L(t) = t$ approach $0$ as $t \to 0$. \\

Do they approach $0$ at the same rate? \\

Let's investigate $R(t) = \frac{sin(t)}{t}$








\begin{image}
\begin{tikzpicture}
  \begin{axis}[
            domain=-1:1, ymax=2, xmax=1, ymin=0, xmin=-1,
            axis lines =center, xlabel=$t$, ylabel={$y=R(t)$}, grid = major,
            %ytick={-10,-8,-6,-4,-2,2,4,6,8,10},
            %xtick={-10,-8,-6,-4,-2,2,4,6,8,10},
            %yticklabels={$-10$,$-8$,$-6$,$-4$,$-2$,$2$,$4$,$6$,$8$,$10$}, 
            %xticklabels={$-10$,$-8$,$-6$,$-4$,$-2$,$2$,$4$,$6$,$8$,$10$},
            ticklabel style={font=\scriptsize},
            every axis y label/.style={at=(current axis.above origin),anchor=south},
            every axis x label/.style={at=(current axis.right of origin),anchor=west},
            axis on top
          ]
          

            \addplot [line width=2, penColor2, smooth,samples=100,domain=(-8:8),<->] {sin(deg(x))/x};





           

  \end{axis}
\end{tikzpicture}
\end{image}

$sin(t)$ and $t$ approach $0$ at exactly the same rate.  Their quotient approaches $1$.



\end{example}
















\begin{example} Indeterminate Forms

We have seen an even weirder indeterminate form.

Everyone knows that $1$ to any power equals $1$. Everyone knows that numbers greater than $1$ to any power get bigger.


What about something like $\left( 1 + \frac{1}{x} \right)^x$ ?

The base is getting closer to $1$, but they are always greater than $1$.  The power is growing larger.  

We are heading to something like $1^{\infty}$.

And we have seen this approaches the value $e$. \\

$e \approx 2.71828. $How far out in the domain do you need to go for the graph to be inside the interval $[2.7, 2.8]$?








\begin{center}
\desmos{ttk7altjct}{400}{300}
\end{center}









\end{example}












\end{document}
