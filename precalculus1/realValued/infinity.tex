\documentclass{ximera}


\graphicspath{
  {./}
  {ximeraTutorial/}
  {basicPhilosophy/}
}

\newcommand{\mooculus}{\textsf{\textbf{MOOC}\textnormal{\textsf{ULUS}}}}

\usepackage{tkz-euclide}\usepackage{tikz}
\usepackage{tikz-cd}
\usetikzlibrary{arrows}
\tikzset{>=stealth,commutative diagrams/.cd,
  arrow style=tikz,diagrams={>=stealth}} %% cool arrow head
\tikzset{shorten <>/.style={ shorten >=#1, shorten <=#1 } } %% allows shorter vectors

\usetikzlibrary{backgrounds} %% for boxes around graphs
\usetikzlibrary{shapes,positioning}  %% Clouds and stars
\usetikzlibrary{matrix} %% for matrix
\usepgfplotslibrary{polar} %% for polar plots
\usepgfplotslibrary{fillbetween} %% to shade area between curves in TikZ
\usetkzobj{all}
\usepackage[makeroom]{cancel} %% for strike outs
%\usepackage{mathtools} %% for pretty underbrace % Breaks Ximera
%\usepackage{multicol}
\usepackage{pgffor} %% required for integral for loops



%% http://tex.stackexchange.com/questions/66490/drawing-a-tikz-arc-specifying-the-center
%% Draws beach ball
\tikzset{pics/carc/.style args={#1:#2:#3}{code={\draw[pic actions] (#1:#3) arc(#1:#2:#3);}}}



\usepackage{array}
\setlength{\extrarowheight}{+.1cm}
\newdimen\digitwidth
\settowidth\digitwidth{9}
\def\divrule#1#2{
\noalign{\moveright#1\digitwidth
\vbox{\hrule width#2\digitwidth}}}






\DeclareMathOperator{\arccot}{arccot}
\DeclareMathOperator{\arcsec}{arcsec}
\DeclareMathOperator{\arccsc}{arccsc}

















%%This is to help with formatting on future title pages.
\newenvironment{sectionOutcomes}{}{}


\title{$\infty$}

\begin{document}

\begin{abstract}
not a number
\end{abstract}
\maketitle




\section{Infinity: $\infty$}


\begin{center}
Infinity is not a real number.
\end{center}




This course presents a study of real numbers and real-valued functions. That doesn't include $\infty$.




\textbf{\textcolor{red!90!darkgray}{$\blacktriangleright$ }} You CANNOT perform arithmetic with $\infty$.

Addition, subtraction, multiplication, and division are operations for real numbers.  Since $\infty$ is not a real number, it CANNOT be involved with any of these operations


\begin{example}


\begin{itemize}
\item $\infty + \infty \ne \infty$
\item $\infty - \infty \ne 0$
\item $\infty \cdot \infty \ne \infty$
\item $\infty \div \infty \ne 1$
\item $\infty^{\infty} \ne \infty$
\end{itemize}


\end{example}

Our operations are strictly for real numbers, only.









\textbf{\textcolor{red!90!darkgray}{$\blacktriangleright$ }} You CANNOT form fractions with $\infty$.



\begin{example}


\begin{itemize}
\item $\frac{\infty}{\infty} \ne \infty$
\item $\frac{\infty}{\infty} \ne 1$
\item $\frac{\infty}{\infty} \ne 0$
\item $\frac{1}{\infty} \ne 0$
\item $\frac{\infty}{1} \ne \infty$
\end{itemize}


\end{example}








\textbf{\textcolor{red!90!darkgray}{$\blacktriangleright$ }} You CANNOT evaluate functions at $\infty$.



\begin{example}


Let $f(x) = 3x^2 + 5x + 9$, then $f(\infty) \ne \infty$ \\

Let $g(x) = \frac{6x + 5}{3x - 1}$, then $g(\infty) \ne 2$ \\

Let $h(x) = \frac{4}{7x + 2}$, then $h(\infty) \ne 0$ \\

\end{example}











\textbf{\textcolor{red!90!darkgray}{$\blacktriangleright$ }} $\infty$ CANNOT be a function value.



\begin{example}



Let $k(x) = \frac{8}{3x + 6}$, then $k(-2) \ne \infty$ \\

\end{example}

$\infty$ is not a real number and cannot be treated like a real number in any way.








\section{Infinity: What is it?}


If $\infty$ is not a real number, then what is it?


The problem here is the question itself.  The question presupposes that $\infty$ is a mathematical object.  For us, it isn't.   

For us, $\infty$ is simply shorthand communication.  It is the shorthand symbol we use to let readers know that we have encountered an unbounded situation. Unbounded meaning there is no number greater than the values we are examining.


If you keep studying mathematics, especially logic, then $\infty$ might become an object, perhaps with its own operations.

However, for us, it is shorthand communication that describes the set of values we are analyzing.







\begin{example}


\begin{itemize}
\item The interval $(5, \infty)$ describes the set of all real numbers greater than $5$.  $\infty$ tells us that there is no number greater than all of the numbers in this set.



\item The interval $(-\infty, 4)$ describes the set of all real numbers less than $4$.  $-\infty$ tells us that there is no number less than all of the numbers in this set.
\end{itemize}




\end{example}
In interval notation, $\infty$ gets a parenthesis because it is not a real number and cannot be included in any set of numbers.
































\end{document}

