\documentclass{ximera}


\graphicspath{
  {./}
  {ximeraTutorial/}
  {basicPhilosophy/}
}

\newcommand{\mooculus}{\textsf{\textbf{MOOC}\textnormal{\textsf{ULUS}}}}

\usepackage{tkz-euclide}\usepackage{tikz}
\usepackage{tikz-cd}
\usetikzlibrary{arrows}
\tikzset{>=stealth,commutative diagrams/.cd,
  arrow style=tikz,diagrams={>=stealth}} %% cool arrow head
\tikzset{shorten <>/.style={ shorten >=#1, shorten <=#1 } } %% allows shorter vectors

\usetikzlibrary{backgrounds} %% for boxes around graphs
\usetikzlibrary{shapes,positioning}  %% Clouds and stars
\usetikzlibrary{matrix} %% for matrix
\usepgfplotslibrary{polar} %% for polar plots
\usepgfplotslibrary{fillbetween} %% to shade area between curves in TikZ
\usetkzobj{all}
\usepackage[makeroom]{cancel} %% for strike outs
%\usepackage{mathtools} %% for pretty underbrace % Breaks Ximera
%\usepackage{multicol}
\usepackage{pgffor} %% required for integral for loops



%% http://tex.stackexchange.com/questions/66490/drawing-a-tikz-arc-specifying-the-center
%% Draws beach ball
\tikzset{pics/carc/.style args={#1:#2:#3}{code={\draw[pic actions] (#1:#3) arc(#1:#2:#3);}}}



\usepackage{array}
\setlength{\extrarowheight}{+.1cm}
\newdimen\digitwidth
\settowidth\digitwidth{9}
\def\divrule#1#2{
\noalign{\moveright#1\digitwidth
\vbox{\hrule width#2\digitwidth}}}






\DeclareMathOperator{\arccot}{arccot}
\DeclareMathOperator{\arcsec}{arcsec}
\DeclareMathOperator{\arccsc}{arccsc}

















%%This is to help with formatting on future title pages.
\newenvironment{sectionOutcomes}{}{}


\title{Symbols}

\begin{document}

\begin{abstract}
vocabulary
\end{abstract}
\maketitle




Mathematics is a language.  It uses symbols and vocabulary and language and pronunciation. \\




\subsection{Sets}



We will be investigating functions that are defined on sets. Naturally, we have symbols for sets. \\



\textbf{\textcolor{blue!55!black}{Membership:}}  The symbol $\in$ means ``is a member of''.


\[
7 \in \{ 4, 5, 7, 8 \}
\]

\begin{center}

$7$ is a member of the set $\{ 4, 5, 7, 8 \}$.

\end{center}

People will also say ``is an element of''. \\





\textbf{\textcolor{blue!55!black}{Subset:}}  The symbol $\subset$ means ``is a proper subset of''.


\[
\{ 4, 5 \} \subset \{ 4, 5, 7, 8 \}
\]

\begin{center}

$\{ 4, 5 \}$ is a subset of the set $\{ 4, 5, 7, 8 \}$.

\end{center}


Every member of $\{ 4, 5 \}$ is also a member of $\{ 4, 5, 7, 8 \}$. \\


\textbf{Proper} means that the subset is not equal to the larger set. \\


The symbol $\subseteq$ means ``is a subset of''.  This allows the possibility of the subset being equal to the larger set.

\[
\{ 4, 5, 7, 8 \} \subseteq \{ 4, 5, 7, 8 \}
\]









\textbf{\textcolor{blue!55!black}{Union:}}  The symbol $\cup$ stands for ``union''.


The union of two sets is another set.  The union contains all of the members of two original sets.

\[
\{ 1, 2, 3 \} \cup \{ 4, 5, 7, 8 \} = \{ 1, 2, 3, 4, 5, 7, 8 \}
\]








\textbf{\textcolor{blue!55!black}{Intersection:}}  The symbol $\cap$ stands for ``intersection''.

The intersection of two sets is another set.  The intersection contains all of the members shared by the two original sets.

\[
\{ 1, 2, 3, 4, 5 \} \cap \{ 4, 5, 7, 8 \} = \{ 4, 5 \}
\]








\textbf{\textcolor{blue!55!black}{Empty Set:}}  The symbol $\emptyset$ stands for ``the empty set''.

The empty set is a set.  It just contains no members.














\subsection{Numbers}

We have some standard sets of numbers and they have special symbols. \\



\begin{itemize}

\item $\mathbb{N}$ : the natural numbers
\item $\mathbb{Z}$ : the integers
\item $\mathbb{Q}$ : the rational numbers
\item $\mathbb{R}$ : the real numbers
\item $\mathbb{C}$ : the complex numbers
\end{itemize}

















\subsection{Small}



As we move toward Calculus, our attention will focus on ``close''...a lot! \\


The word \textbf{instantaneous} will describe our measurements. \\


So, we will use our symbols, notation, language to help us talk about ``close''. \\


We have two symbols from the Greek language that we traditionally use to mean ``a small positive number''. \\



\begin{notation} \textbf{\textcolor{red!80!black}{Small}} 


\textbf{\textcolor{blue!55!black}{epsilon:}}  $\epsilon$ \\

\textbf{\textcolor{blue!55!black}{delta:}}  $\delta$ \\


\end{notation}

$\epsilon$ and $\delta$ are used to mean ``a very small positive number''. \\  


$\epsilon$ and $\delta$ are how we talk algebraically about ``close''. \\





\begin{warning}


$\epsilon$ and $\delta$ are not specific numbers with specific numeric values, like $\pi$. \\


$\epsilon$ and $\delta$ are used to represent very small positive numbers, but not specific small positive numbers. \\


So, $\epsilon$ and $\delta$ are used as constants - small positive constants.





\end{warning}








\begin{center}
\textbf{\textcolor{green!50!black}{ooooo=-=-=-=-=-=-=-=-=-=-=-=-=ooOoo=-=-=-=-=-=-=-=-=-=-=-=-=ooooo}} \\

more examples can be found by following this link\\ \link[More Examples of Real-Valued Functions]{https://ximera.osu.edu/csccmathematics/precalculus1/precalculus1/realValued/examples/exampleList}

\end{center}



\end{document}
