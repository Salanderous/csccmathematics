\documentclass{ximera}


\graphicspath{
  {./}
  {ximeraTutorial/}
  {basicPhilosophy/}
}

\newcommand{\mooculus}{\textsf{\textbf{MOOC}\textnormal{\textsf{ULUS}}}}

\usepackage{tkz-euclide}\usepackage{tikz}
\usepackage{tikz-cd}
\usetikzlibrary{arrows}
\tikzset{>=stealth,commutative diagrams/.cd,
  arrow style=tikz,diagrams={>=stealth}} %% cool arrow head
\tikzset{shorten <>/.style={ shorten >=#1, shorten <=#1 } } %% allows shorter vectors

\usetikzlibrary{backgrounds} %% for boxes around graphs
\usetikzlibrary{shapes,positioning}  %% Clouds and stars
\usetikzlibrary{matrix} %% for matrix
\usepgfplotslibrary{polar} %% for polar plots
\usepgfplotslibrary{fillbetween} %% to shade area between curves in TikZ
\usetkzobj{all}
\usepackage[makeroom]{cancel} %% for strike outs
%\usepackage{mathtools} %% for pretty underbrace % Breaks Ximera
%\usepackage{multicol}
\usepackage{pgffor} %% required for integral for loops



%% http://tex.stackexchange.com/questions/66490/drawing-a-tikz-arc-specifying-the-center
%% Draws beach ball
\tikzset{pics/carc/.style args={#1:#2:#3}{code={\draw[pic actions] (#1:#3) arc(#1:#2:#3);}}}



\usepackage{array}
\setlength{\extrarowheight}{+.1cm}
\newdimen\digitwidth
\settowidth\digitwidth{9}
\def\divrule#1#2{
\noalign{\moveright#1\digitwidth
\vbox{\hrule width#2\digitwidth}}}






\DeclareMathOperator{\arccot}{arccot}
\DeclareMathOperator{\arcsec}{arcsec}
\DeclareMathOperator{\arccsc}{arccsc}

















%%This is to help with formatting on future title pages.
\newenvironment{sectionOutcomes}{}{}


\title{Open Sets}

\begin{document}

\begin{abstract}
always room
\end{abstract}
\maketitle





Why do we call $(1, 5)$ an \textbf{open interval} on the real line or in the real numbers? \\



The reason is that each number in the interval feels like it is out in the open - that there is room around it.  It may be a very small surrounding space around it, but room none-the-less.  


We extend this idea that there is always room "around" each number to \textbf{open sets in $\mathbb{R}$}.






\section{Open Sets in $\mathbb{R}$}

In an open set in $\mathbb{R}$, each number has room on either side.  




\begin{definition}  \textbf{\textcolor{green!50!black}{Open Set in $\mathbb{R}$}}


$S \subset \mathbb{R}$ is an \textbf{open set in $\mathbb{R}$} provided 

\[
\text{For each } a \in S, \text{ there exists an } \epsilon > 0 \text{ such that }  (a - \epsilon, a + \epsilon) \subset S
\]

\end{definition}



\begin{idea}

The idea is that inside an open set, every number sits inside an open interval inside the set.  There is space between each number and the endpoint. There is a positive distance between the number and each endpoint.
\end{idea}

Forcing a positive distance between each number and the endpoints turns out to have far reaching consequences. \\



\begin{example}  \textit{Open Intervals in $\mathbb{R}$}


The open interval $(4,7)$ is an open set in $\mathbb{R}$.


\begin{image}
  \begin{tikzpicture}
  \begin{axis}[
            %xmin=-25,xmax=25,ymin=-25,ymax=25,
            %width=3in,
            clip=false,
            axis lines=center,
            %ticks=none,
            unit vector ratio*=1 1 1,
            ymajorticks=false,
            xtick={-5,4, 7},
            %xlabel=$x$, ylabel=$y$,
            %every axis y label/.style={at=(current axis.above origin),anchor=south},
            every axis x label/.style={at=(current axis.right of origin),anchor=west},
          ]      
       
            \addplot [line width=2, penColor2, smooth,samples=100,domain=(4:7)] ({x},{0});

            \addplot [line width=0.5, penColor, smooth,samples=100,domain=(-10:-9.9)] ({x},{0});
            \addplot [line width=0.5, penColor, smooth,samples=100,domain=(9.9:10)] ({x},{0});


            \node at (axis cs:4,0) [penColor2] {$($};
            \node at (axis cs:7,0) [penColor2] {$)$};

    \end{axis}
  \end{tikzpicture}
  \end{image}


\begin{explanation}

Let $c$ be any number in $(4,7)$.  Then the open subinterval $\left( c - \frac{4+c}{2}, c + \frac{7+c}{2} \right)$ will always be inside $(4,7)$.

Therefore, for each $c \in (4, 7)$, there exists an open interval inside $(4,7)$, which contains $c$.

Since this is true for all of the numbers in $(4,7)$, the interval is an open set in $\mathbb{R}$.

\end{explanation}
\end{example}









\begin{example}  



The interval $(4,7]$ is not an open set in $\mathbb{R}$.


\begin{image}
  \begin{tikzpicture}
  \begin{axis}[
            %xmin=-25,xmax=25,ymin=-25,ymax=25,
            %width=3in,
            clip=false,
            axis lines=center,
            %ticks=none,
            unit vector ratio*=1 1 1,
            ymajorticks=false,
            xtick={-5,4, 7},
            %xlabel=$x$, ylabel=$y$,
            %every axis y label/.style={at=(current axis.above origin),anchor=south},
            every axis x label/.style={at=(current axis.right of origin),anchor=west},
          ]      
       
            \addplot [line width=2, penColor2, smooth,samples=100,domain=(4:7)] ({x},{0});

            \addplot [line width=0.5, penColor, smooth,samples=100,domain=(-10:-9.9)] ({x},{0});
            \addplot [line width=0.5, penColor, smooth,samples=100,domain=(9.9:10)] ({x},{0});


            \node at (axis cs:4,0) [penColor2] {$($};
            \node at (axis cs:7,0) [penColor2] {$]$};

    \end{axis}
  \end{tikzpicture}
  \end{image}


\begin{explanation}

Consider the number $7$.  Any open interval around $7$ would look like $(7 - \epsilon, 7 + \delta)$, with both $\epsilon > 0$  and $\delta > 0$.  However, there is always a real number in the interval $(7, 7 + \delta)$, like $7 + \frac{\delta}{2}$, which lies outside the set.

Therefore, no matter which open interval you choose around $7$, it will always containing a number outside $(4, 7]$.

$(4, 7]$  is not an open set in $\mathbb{R}$.

\end{explanation}
\end{example}

$\blacktriangleright$ Open intervals cannot contain their endpoints. \\





\begin{example}  \textit{Unions of Open Intervals} \\


Any union of open intervals is itself an open set.



\begin{explanation}

Let $S$ be a union of open intervals.  \\
Let $c \in S$. \\
Then $c$ is in one of the open intervals. \\
Then there exists an $\epsilon > 0$, such that $(c - \epsilon, c + \epsilon)$ is a subset of this one open interval.  This interval is a subset of $S$.  Therefore, $(c - \epsilon, c + \epsilon) \subset S$. \\


\end{explanation}

\end{example}















\begin{example}  \textit{The Empty Set} \\


The \textbf{empty set}, $\emptyset$, is the set with no element.  The empty set is a subset of the real numbers.  It is the subset containing no real numbers. \\


$\emptyset$ is an open set. \\



\begin{explanation}


The reasoning is sort of reverse thinking.  If $\emptyset$ is not open, then it MUST contain a number, such that there is no tiny open interval around it.  There is no such number.  Therefore, $\emptyset$ is an open set.


\end{explanation}

\end{example}

$\blacktriangleright$ These types of statements are oftern said to be \textbf{vacuously true}.  They are true, because there is nothing that violates the definition - because there isn't anything at all. 










\end{document}
