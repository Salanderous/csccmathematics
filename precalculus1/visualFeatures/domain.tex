\documentclass{ximera}


\graphicspath{
  {./}
  {ximeraTutorial/}
  {basicPhilosophy/}
}

\newcommand{\mooculus}{\textsf{\textbf{MOOC}\textnormal{\textsf{ULUS}}}}

\usepackage{tkz-euclide}\usepackage{tikz}
\usepackage{tikz-cd}
\usetikzlibrary{arrows}
\tikzset{>=stealth,commutative diagrams/.cd,
  arrow style=tikz,diagrams={>=stealth}} %% cool arrow head
\tikzset{shorten <>/.style={ shorten >=#1, shorten <=#1 } } %% allows shorter vectors

\usetikzlibrary{backgrounds} %% for boxes around graphs
\usetikzlibrary{shapes,positioning}  %% Clouds and stars
\usetikzlibrary{matrix} %% for matrix
\usepgfplotslibrary{polar} %% for polar plots
\usepgfplotslibrary{fillbetween} %% to shade area between curves in TikZ
\usetkzobj{all}
\usepackage[makeroom]{cancel} %% for strike outs
%\usepackage{mathtools} %% for pretty underbrace % Breaks Ximera
%\usepackage{multicol}
\usepackage{pgffor} %% required for integral for loops



%% http://tex.stackexchange.com/questions/66490/drawing-a-tikz-arc-specifying-the-center
%% Draws beach ball
\tikzset{pics/carc/.style args={#1:#2:#3}{code={\draw[pic actions] (#1:#3) arc(#1:#2:#3);}}}



\usepackage{array}
\setlength{\extrarowheight}{+.1cm}
\newdimen\digitwidth
\settowidth\digitwidth{9}
\def\divrule#1#2{
\noalign{\moveright#1\digitwidth
\vbox{\hrule width#2\digitwidth}}}






\DeclareMathOperator{\arccot}{arccot}
\DeclareMathOperator{\arcsec}{arcsec}
\DeclareMathOperator{\arccsc}{arccsc}

















%%This is to help with formatting on future title pages.
\newenvironment{sectionOutcomes}{}{}


\title{Domain}

\begin{document}

\begin{abstract}
inputs
\end{abstract}
\maketitle


A function is a package of three sets.  One set is the domain. One set is the range. The third set is a set of pairs. Since a graph is visually representing the function, we should be able to determine these three sets from the graph.\\


The graph of a function is just a big collection of individual dots. Each dot is visually encoding a function pair. \\



\begin{itemize}
     \item the first/left coordinate of a point (the abscissa) is a domain number.
     \item The second/right coordinate of a point (the ordinate) is  the funciton value at this domain number.
\end{itemize}



The domain would be the collection of all of the first coordinates of all of the plotted points on the graph. \\






Here is the complete graph of the function $G(x)$. \\


\begin{image}
\begin{tikzpicture}
     \begin{axis}[
               domain=-10:10, ymax=10, xmax=10, ymin=-10, xmin=-10,
               axis lines =center, xlabel=$x$, ylabel=$y$,
                ytick={-10,-8,-6,-4,-2,2,4,6,8,10},
                xtick={-10,-8,-6,-4,-2,2,4,6,8,10},
                yticklabels={$-10$,$-8$,$-6$,$-4$,$-2$,$2$,$4$,$6$,$8$,$10$}, 
                xticklabels={$-10$,$-8$,$-6$,$-4$,$-2$,$2$,$4$,$6$,$8$,$10$},
                ticklabel style={font=\scriptsize},
               every axis y label/.style={at=(current axis.above origin),anchor=south},
               every axis x label/.style={at=(current axis.right of origin),anchor=west},
               axis on top,
                    ]

        
        \addplot [draw=penColor, very thick, smooth, domain=(-6:3), <->] {1/(x-3) + 2};
        \addplot [draw=penColor, very thick, smooth, domain=(3:8), ->] {1/(x-3) + 2};

        \addplot [line width=0.5, gray, dashed,samples=100,domain=(-9:9)] ({3},{x});
        \addplot [line width=0.5, gray, dashed,samples=100,domain=(-9:9)] ({x},{2});

        \addplot[color=penColor,only marks,mark=*] coordinates{(3.2,7)}; 
        %\addplot[color=penColor,only marks,mark=*] coordinates{(8,2.2)}; 


    \end{axis}



\end{tikzpicture}
\end{image}



The graph has two pieces and two asymptotes. $x=3$ is a vertical asymptote. $y=2$ is a horizontal asymptote.\\

The left part of the graph includes two arrows, which tell us that the first coordinates fill up the interval $(\-infty, 3)$.

The right piece of the graph includes the point $(3.2, 7)$ and then an arrow, which tell us that the first coordinates fill up the interval $[3.2, \infty)$. \\

The domain is

\[   (-\infty, 3) \cup [3.2, \infty]   \]






We can visualize the dots on the graph smashed vertically to the horizontal axis.






\begin{image}
\begin{tikzpicture}
     \begin{axis}[
               domain=-10:10, ymax=10, xmax=10, ymin=-10, xmin=-10,
               axis lines =center, xlabel=$x$, ylabel=$y$,
                ytick={-10,-8,-6,-4,-2,2,4,6,8,10},
                xtick={-10,-8,-6,-4,-2,2,4,6,8,10},
                yticklabels={$-10$,$-8$,$-6$,$-4$,$-2$,$2$,$4$,$6$,$8$,$10$}, 
                xticklabels={$-10$,$-8$,$-6$,$-4$,$-2$,$2$,$4$,$6$,$8$,$10$},
                ticklabel style={font=\scriptsize},
               every axis y label/.style={at=(current axis.above origin),anchor=south},
               every axis x label/.style={at=(current axis.right of origin),anchor=west},
               axis on top,
                    ]

        
        \addplot [draw=penColor2, very thick, smooth, domain=(-6:3), <->] {0};
        \addplot [draw=penColor, very thick, smooth, domain=(3:8), ->] {0};

        \addplot [line width=0.5, gray, dashed,samples=100,domain=(-9:9)] ({3},{x});
        \addplot [line width=0.5, gray, dashed,samples=100,domain=(-9:9)] ({x},{2});

        \addplot[color=penColor2,fill=white,only marks,mark=*] coordinates{(3.2,0)}; 
        %\addplot[color=penColor,only marks,mark=*] coordinates{(8,2.2)}; 


    \end{axis}



\end{tikzpicture}
\end{image}






\[   (-\infty, 3) \cup [3.2, \infty]   \]

















\begin{example}


Here is a complete graph of $K(v)$. \\


\begin{image}
\begin{tikzpicture}
     \begin{axis}[
                domain=-10:10, ymax=10, xmax=10, ymin=-10, xmin=-10,
                axis lines =center, xlabel=$v$, ylabel=${y=K(v)}$,
                ytick={-10,-8,-6,-4,-2,2,4,6,8,10},
                xtick={-10,-8,-6,-4,-2,2,4,6,8,10},
                yticklabels={$-10$,$-8$,$-6$,$-4$,$-2$,$2$,$4$,$6$,$8$,$10$}, 
                xticklabels={$-10$,$-8$,$-6$,$-4$,$-2$,$2$,$4$,$6$,$8$,$10$},
                ticklabel style={font=\scriptsize},
                every axis y label/.style={at=(current axis.above origin),anchor=south},
                every axis x label/.style={at=(current axis.right of origin),anchor=west},
                axis on top,
                ]

        
        \addplot [draw=penColor, very thick, smooth, domain=(-6:10), ->] {10*sin(deg(x))/((x+10)^1.2) + e^(-x-5) + 2};
        %\addplot [draw=penColor, very thick, smooth, domain=(3:8), <->] {1/(x-3) + 2};

        %\addplot [line width=1, gray, dashed,samples=100,domain=(-9:9)] ({3},{x});
        \addplot [line width=0.5, gray, dashed,samples=100,domain=(1:9)] ({x},{2});
        \addplot[color=penColor,only marks,mark=*] coordinates{(-6,5.25)}; 


    \end{axis}



\end{tikzpicture}
\end{image}







\begin{image}
\begin{tikzpicture}
     \begin{axis}[
                domain=-10:10, ymax=10, xmax=10, ymin=-10, xmin=-10,
                axis lines =center, xlabel=$v$, ylabel=${y=K(v)}$,
                ytick={-10,-8,-6,-4,-2,2,4,6,8,10},
                xtick={-10,-8,-6,-4,-2,2,4,6,8,10},
                yticklabels={$-10$,$-8$,$-6$,$-4$,$-2$,$2$,$4$,$6$,$8$,$10$}, 
                xticklabels={$-10$,$-8$,$-6$,$-4$,$-2$,$2$,$4$,$6$,$8$,$10$},
                ticklabel style={font=\scriptsize},
                every axis y label/.style={at=(current axis.above origin),anchor=south},
                every axis x label/.style={at=(current axis.right of origin),anchor=west},
                axis on top,
                ]

        
        \addplot [draw=penColor2, very thick, smooth, domain=(-6:10), ->] {0};
        %\addplot [draw=penColor, very thick, smooth, domain=(3:8), <->] {1/(x-3) + 2};

        %\addplot [line width=1, gray, dashed,samples=100,domain=(-9:9)] ({3},{x});
        \addplot [line width=0.5, gray, dashed,samples=100,domain=(1:9)] ({x},{2});
        \addplot[color=penColor2,only marks,mark=*] coordinates{(-6,0)}; 


    \end{axis}



\end{tikzpicture}
\end{image}




The domain of the function $K$ is $[-6, \infty)$. 

\end{example}















\begin{example}


Here is a complete graph of $p(x)$. \\


\begin{image}
\begin{tikzpicture}
     \begin{axis}[
                domain=-10:10, ymax=10, xmax=10, ymin=-10, xmin=-10,
                axis lines =center, xlabel=$x$, ylabel=${y=p(x)}$,
                ytick={-10,-8,-6,-4,-2,2,4,6,8,10},
                xtick={-10,-8,-6,-4,-2,2,4,6,8,10},
                yticklabels={$-10$,$-8$,$-6$,$-4$,$-2$,$2$,$4$,$6$,$8$,$10$}, 
                xticklabels={$-10$,$-8$,$-6$,$-4$,$-2$,$2$,$4$,$6$,$8$,$10$},
                ticklabel style={font=\scriptsize},
                every axis y label/.style={at=(current axis.above origin),anchor=south},
                every axis x label/.style={at=(current axis.right of origin),anchor=west},
                axis on top,
                ]

        
        \addplot [draw=penColor, very thick, smooth, domain=(-6:10)] {10*sin(deg(x))/((x+10)^1.2) + e^(-x-5) + 2};
        %\addplot [draw=penColor, very thick, smooth, domain=(3:8), <->] {1/(x-3) + 2};

        %\addplot [line width=1, gray, dashed,samples=100,domain=(-9:9)] ({3},{x});
        %\addplot [line width=0.5, gray, dashed,samples=100,domain=(1:9)] ({x},{2});
        \addplot[color=penColor,only marks,mark=*] coordinates{(-6,5.25)}; 
        \addplot[color=penColor,only marks,mark=*] coordinates{(10,1.8)}; 


    \end{axis}



\end{tikzpicture}
\end{image}








\begin{image}
\begin{tikzpicture}
     \begin{axis}[
                domain=-10:10, ymax=10, xmax=10, ymin=-10, xmin=-10,
                axis lines =center, xlabel=$x$, ylabel=${y=p(x)}$,
                ytick={-10,-8,-6,-4,-2,2,4,6,8,10},
                xtick={-10,-8,-6,-4,-2,2,4,6,8,10},
                yticklabels={$-10$,$-8$,$-6$,$-4$,$-2$,$2$,$4$,$6$,$8$,$10$}, 
                xticklabels={$-10$,$-8$,$-6$,$-4$,$-2$,$2$,$4$,$6$,$8$,$10$},
                ticklabel style={font=\scriptsize},
                every axis y label/.style={at=(current axis.above origin),anchor=south},
                every axis x label/.style={at=(current axis.right of origin),anchor=west},
                axis on top,
                ]

        
        \addplot [draw=penColor2, very thick, smooth, domain=(-6:10)] {0};
        %\addplot [draw=penColor, very thick, smooth, domain=(3:8), <->] {1/(x-3) + 2};

        %\addplot [line width=1, gray, dashed,samples=100,domain=(-9:9)] ({3},{x});
        %\addplot [line width=0.5, gray, dashed,samples=100,domain=(1:9)] ({x},{2});
        \addplot[color=penColor2,only marks,mark=*] coordinates{(-6,0)}; 
        \addplot[color=penColor2,only marks,mark=*] coordinates{(10,0)}; 


    \end{axis}



\end{tikzpicture}
\end{image}







The domain of the function $p$ is $[-6, 10]$. 


\end{example}












\begin{center}
\textbf{\textcolor{green!50!black}{ooooo=-=-=-=-=-=-=-=-=-=-=-=-=ooOoo=-=-=-=-=-=-=-=-=-=-=-=-=ooooo}} \\

more examples can be found by following this link\\ \link[More Examples of Visual Features]{https://ximera.osu.edu/csccmathematics/precalculus1/precalculus1/visualFeatures/examples/exampleList}

\end{center}











\end{document}
