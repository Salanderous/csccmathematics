\documentclass{ximera}


\graphicspath{
  {./}
  {ximeraTutorial/}
  {basicPhilosophy/}
}

\newcommand{\mooculus}{\textsf{\textbf{MOOC}\textnormal{\textsf{ULUS}}}}

\usepackage{tkz-euclide}\usepackage{tikz}
\usepackage{tikz-cd}
\usetikzlibrary{arrows}
\tikzset{>=stealth,commutative diagrams/.cd,
  arrow style=tikz,diagrams={>=stealth}} %% cool arrow head
\tikzset{shorten <>/.style={ shorten >=#1, shorten <=#1 } } %% allows shorter vectors

\usetikzlibrary{backgrounds} %% for boxes around graphs
\usetikzlibrary{shapes,positioning}  %% Clouds and stars
\usetikzlibrary{matrix} %% for matrix
\usepgfplotslibrary{polar} %% for polar plots
\usepgfplotslibrary{fillbetween} %% to shade area between curves in TikZ
\usetkzobj{all}
\usepackage[makeroom]{cancel} %% for strike outs
%\usepackage{mathtools} %% for pretty underbrace % Breaks Ximera
%\usepackage{multicol}
\usepackage{pgffor} %% required for integral for loops



%% http://tex.stackexchange.com/questions/66490/drawing-a-tikz-arc-specifying-the-center
%% Draws beach ball
\tikzset{pics/carc/.style args={#1:#2:#3}{code={\draw[pic actions] (#1:#3) arc(#1:#2:#3);}}}



\usepackage{array}
\setlength{\extrarowheight}{+.1cm}
\newdimen\digitwidth
\settowidth\digitwidth{9}
\def\divrule#1#2{
\noalign{\moveright#1\digitwidth
\vbox{\hrule width#2\digitwidth}}}






\DeclareMathOperator{\arccot}{arccot}
\DeclareMathOperator{\arcsec}{arcsec}
\DeclareMathOperator{\arccsc}{arccsc}

















%%This is to help with formatting on future title pages.
\newenvironment{sectionOutcomes}{}{}


\title{Subsets of Real Numbers}

\begin{document}

\begin{abstract}
intervals
\end{abstract}
\maketitle




We have whittled our investigations down to a study of real-valued functions.  These are functions whose domain and range are both subsets of the real numbers.  Basically, Precalculus 1 is a study of the real numbers.

This means we will need to communicate about sets of real numbers.  We have several ways of communicating our ideas.


\section{Building Blocks}

The subsets we are interested in are collections of two types of basic building blocks.

\begin{itemize}
\item Building Block: Individual Numbers

Our subsets might include one or more individual isolated numbers.  
	\begin{itemize}
	\item Our set might be just the numbers -3, 2, and 5.

	Curly braces around a comma-separated list denotes a set of individual numbers $\{ -3, 2, 5 \}$

	\item Graphically, individual dots on a numnber line represent a set of the corresponding individual numbers.

	\begin{image}
	\begin{tikzpicture}
		\begin{axis}[
            %xmin=-25,xmax=25,ymin=-25,ymax=25,
            %width=3in,
            clip=false,
            axis lines=center,
            %ticks=none,
            unit vector ratio*=1 1 1,
            ymajorticks=false,
            xtick={-3,2,5},
            %xlabel=$x$, ylabel=$y$,
            %every axis y label/.style={at=(current axis.above origin),anchor=south},
            every axis x label/.style={at=(current axis.right of origin),anchor=west},
          ]      
       
       		\addplot [line width=1, penColor, smooth,samples=100,domain=(-10:-9.9)] ({x},{0});
          	\addplot [line width=1, penColor, smooth,samples=100,domain=(9.9:10)] ({x},{0});
          	\addplot [color=penColor2,only marks,mark=*] coordinates{(-3,0)};
          	\addplot [color=penColor2,only marks,mark=*] coordinates{(2,0)};
          	\addplot [color=penColor2,only marks,mark=*] coordinates{(5,0)};

        \end{axis}
	\end{tikzpicture}
	\end{image}



	\end{itemize}



\item  Building Block: Finite Intervals

Our subsets might include pieces of the real number line.  These are called \textbf{intervals}.   Intervals  include \textit{ALL} of the real numbers between one specific number and another specific number. 

	\begin{itemize}
	\item All of the real numbers between 4 and 7 including both 4 and 7.  We use square brackets to indicate inclusion.  $[4, 7]$
	\begin{image}
	\begin{tikzpicture}
	\begin{axis}[
            %xmin=-25,xmax=25,ymin=-25,ymax=25,
            %width=3in,
            clip=false,
            axis lines=center,
            %ticks=none,
            unit vector ratio*=1 1 1,
            ymajorticks=false,
            xtick={-5,4, 7},
            %xlabel=$x$, ylabel=$y$,
            %every axis y label/.style={at=(current axis.above origin),anchor=south},
            every axis x label/.style={at=(current axis.right of origin),anchor=west},
          ]      
       
          	\addplot [line width=2, penColor2, smooth,samples=100,domain=(4:7)] ({x},{0});
          	\addplot [line width=1, penColor, smooth,samples=100,domain=(-10:-9.9)] ({x},{0});
          	\addplot [line width=1, penColor, smooth,samples=100,domain=(9.9:10)] ({x},{0});


          	\node at (axis cs:4,0) [penColor2] {$[$};
          	\node at (axis cs:7,0) [penColor2] {$]$};

    \end{axis}
	\end{tikzpicture}
	\end{image}





	\item All of the real numbers between 4 and 7 including 4 but not 7.  We use parenthese to indicate exclusion.  $[4, 7)$
	\begin{image}
	\begin{tikzpicture}
	\begin{axis}[
            %xmin=-25,xmax=25,ymin=-25,ymax=25,
            %width=3in,
            clip=false,
            axis lines=center,
            %ticks=none,
            unit vector ratio*=1 1 1,
            ymajorticks=false,
            xtick={-5,4, 7},
            %xlabel=$x$, ylabel=$y$,
            %every axis y label/.style={at=(current axis.above origin),anchor=south},
            every axis x label/.style={at=(current axis.right of origin),anchor=west},
          ]      
       
          	\addplot [line width=2, penColor2, smooth,samples=100,domain=(4:7)] ({x},{0});
          	\addplot [line width=1, penColor, smooth,samples=100,domain=(-10:-9.9)] ({x},{0});
          	\addplot [line width=1, penColor, smooth,samples=100,domain=(9.9:10)] ({x},{0});


          	\node at (axis cs:4,0) [penColor2] {$[$};
          	\node at (axis cs:7,0) [penColor2] {$)$};

    \end{axis}
	\end{tikzpicture}
	\end{image}




	\item All of the real numbers between 4 and 7 including 7 but not 4.   $(4, 7]$
	\begin{image}
	\begin{tikzpicture}
	\begin{axis}[
            %xmin=-25,xmax=25,ymin=-25,ymax=25,
            %width=3in,
            clip=false,
            axis lines=center,
            %ticks=none,
            unit vector ratio*=1 1 1,
            ymajorticks=false,
            xtick={-5,4, 7},
            %xlabel=$x$, ylabel=$y$,
            %every axis y label/.style={at=(current axis.above origin),anchor=south},
            every axis x label/.style={at=(current axis.right of origin),anchor=west},
          ]      
       
          	\addplot [line width=2, penColor2, smooth,samples=100,domain=(4:7)] ({x},{0});
          	\addplot [line width=1, penColor, smooth,samples=100,domain=(-10:-9.9)] ({x},{0});
          	\addplot [line width=1, penColor, smooth,samples=100,domain=(9.9:10)] ({x},{0});


          	\node at (axis cs:4,0) [penColor2] {$($};
          	\node at (axis cs:7,0) [penColor2] {$]$};

    \end{axis}
	\end{tikzpicture}
	\end{image}




	\item All of the real numbers between 4 and 7 including neither 4 nor 7.    $(4, 7)$
	\begin{image}
	\begin{tikzpicture}
	\begin{axis}[
            %xmin=-25,xmax=25,ymin=-25,ymax=25,
            %width=3in,
            clip=false,
            axis lines=center,
            %ticks=none,
            unit vector ratio*=1 1 1,
            ymajorticks=false,
            xtick={-5,4, 7},
            %xlabel=$x$, ylabel=$y$,
            %every axis y label/.style={at=(current axis.above origin),anchor=south},
            every axis x label/.style={at=(current axis.right of origin),anchor=west},
          ]      
       
          	\addplot [line width=2, penColor2, smooth,samples=100,domain=(4:7)] ({x},{0});
          	\addplot [line width=1, penColor, smooth,samples=100,domain=(-10:-9.9)] ({x},{0});
          	\addplot [line width=1, penColor, smooth,samples=100,domain=(9.9:10)] ({x},{0});


          	\node at (axis cs:4,0) [penColor2] {$($};
          	\node at (axis cs:7,0) [penColor2] {$)$};

    \end{axis}
	\end{tikzpicture}
	\end{image}

	\end{itemize}




\item  Building Block: Infinite Intervals


Our subsets might include half of the real number line.  These are called \textbf{infinite intervals}.   Infinite intervals include \textit{ALL} of the real numbers less than a specific real number, or \textit{ALL} of the real numbers greater than a specific real number. 

	\begin{itemize}
	\item All of the real numbers less than 4 and including 4.  When writing, we use negative infinity to indicate the interval extends without end, $(-\infty, 4]$.  We drawing, we use an arrow.
	\begin{image}
	\begin{tikzpicture}
	\begin{axis}[
            %xmin=-25,xmax=25,ymin=-25,ymax=25,
            %width=3in,
            clip=false,
            axis lines=center,
            %ticks=none,
            unit vector ratio*=1 1 1,
            ymajorticks=false,
            xtick={-5,4, 7},
            %xlabel=$x$, ylabel=$y$,
            %every axis y label/.style={at=(current axis.above origin),anchor=south},
            every axis x label/.style={at=(current axis.right of origin),anchor=west},
          ]      
       
          	\addplot [line width=2, penColor2, smooth,samples=100,domain=(-2:4),<-] ({x},{0});
          	\addplot [line width=1, penColor, smooth,samples=100,domain=(-10:-9.9)] ({x},{0});
          	\addplot [line width=1, penColor, smooth,samples=100,domain=(9.9:10)] ({x},{0});


          	\node at (axis cs:4,0) [penColor2] {$]$};

    \end{axis}
	\end{tikzpicture}
	\end{image}


\item All of the real numbers less than 4 but not including 4, $(-\infty, 4)$
	\begin{image}
	\begin{tikzpicture}
	\begin{axis}[
            %xmin=-25,xmax=25,ymin=-25,ymax=25,
            %width=3in,
            clip=false,
            axis lines=center,
            %ticks=none,
            unit vector ratio*=1 1 1,
            ymajorticks=false,
            xtick={-5,4, 7},
            %xlabel=$x$, ylabel=$y$,
            %every axis y label/.style={at=(current axis.above origin),anchor=south},
            every axis x label/.style={at=(current axis.right of origin),anchor=west},
          ]      
       
          	\addplot [line width=2, penColor2, smooth,samples=100,domain=(-2:4),<-] ({x},{0});
          	\addplot [line width=1, penColor, smooth,samples=100,domain=(-10:-9.9)] ({x},{0});
          	\addplot [line width=1, penColor, smooth,samples=100,domain=(9.9:10)] ({x},{0});


          	\node at (axis cs:4,0) [penColor2] {$)$};

    \end{axis}
	\end{tikzpicture}
	\end{image}


\item All of the real numbers less than 4 and including 4.  When writing, we use positive infinity to indicate the interval extends without end, $[4, \infty)$.
	\begin{image}
	\begin{tikzpicture}
	\begin{axis}[
            %xmin=-25,xmax=25,ymin=-25,ymax=25,
            %width=3in,
            clip=false,
            axis lines=center,
            %ticks=none,
            unit vector ratio*=1 1 1,
            ymajorticks=false,
            xtick={-5,4, 7},
            %xlabel=$x$, ylabel=$y$,
            %every axis y label/.style={at=(current axis.above origin),anchor=south},
            every axis x label/.style={at=(current axis.right of origin),anchor=west},
          ]      
       
          	\addplot [line width=2, penColor2, smooth,samples=100,domain=(4:7),->] ({x},{0});
          	\addplot [line width=1, penColor, smooth,samples=100,domain=(-10:-9.9)] ({x},{0});
          	\addplot [line width=1, penColor, smooth,samples=100,domain=(9.9:10)] ({x},{0});


          	\node at (axis cs:4,0) [penColor2] {$[$};

    \end{axis}
	\end{tikzpicture}
	\end{image}


\item All of the real numbers less than 4 but not including 4, $(4, \infty)$.
	\begin{image}
	\begin{tikzpicture}
	\begin{axis}[
            %xmin=-25,xmax=25,ymin=-25,ymax=25,
            %width=3in,
            clip=false,
            axis lines=center,
            %ticks=none,
            unit vector ratio*=1 1 1,
            ymajorticks=false,
            xtick={-5,4, 7},
            %xlabel=$x$, ylabel=$y$,
            %every axis y label/.style={at=(current axis.above origin),anchor=south},
            every axis x label/.style={at=(current axis.right of origin),anchor=west},
          ]      
       
          	\addplot [line width=2, penColor2, smooth,samples=100,domain=(4:7),->] ({x},{0});
          	\addplot [line width=1, penColor, smooth,samples=100,domain=(-10:-9.9)] ({x},{0});
          	\addplot [line width=1, penColor, smooth,samples=100,domain=(9.9:10)] ({x},{0});


          	\node at (axis cs:4,0) [penColor2] {$($};

    \end{axis}
	\end{tikzpicture}
	\end{image}

	\end{itemize}

\end{itemize}






\section{Interval Notation}

In the examples above, we used \textbf{interval notation} to describe the intervals in writing. Interval notation adheres to a few rules.

\begin{itemize}
\item Write the two numbers in the order they would occur on the number line, separated by a comma.  These are called \textbf{endpoints}.
\item use a square bracket to indicate the endpoint is included in the interval.
\item use a parenthesis to indicate the endpoint is excluded from the interval.
\item use $-\infty$ to indicate the interval extends without end to the left. $-\infty$ always appears on the left side. $-\infty$ always has a parenthesis around it, because $-\infty$ is not a real number.
\item use $\infty$ to indicate the interval extends without end to the right. $\infty$ always appears on the right side. $-\infty$ always has a parenthesis around it, because $-\infty$ is not a real number.
\end{itemize}


\begin{problem}
Which is valid interval notation?
	\begin{multipleChoice}
	\choice {$[3, -2)$}
	\choice [correct]{$[-2, 3)$}
	\end{multipleChoice}
\end{problem}



\begin{problem}
Which is valid interval notation?
	\begin{multipleChoice}
	\choice {$[1, \infty]$}
	\choice [correct]{$[1, \infty)$}
	\end{multipleChoice}
\end{problem}









\section{Number Line Graphs}

First, we have graphical ways of communicating subsets of real numbers.

Our sets are collections of two types

\begin{image}
\begin{tikzpicture}
	\begin{axis}[
            %xmin=-25,xmax=25,ymin=-25,ymax=25,
            %width=3in,
            clip=false,
            axis lines=center,
            %ticks=none,
            unit vector ratio*=1 1 1,
            ymajorticks=false,
            xtick={-5,0,5,10,15},
            %xlabel=$x$, ylabel=$y$,
            %every axis y label/.style={at=(current axis.above origin),anchor=south},
            every axis x label/.style={at=(current axis.right of origin),anchor=west},
          ]      
       

          \addplot [line width=2, penColor2, smooth,samples=100,domain=(-5:2),<-] ({x},{0});
          \addplot [line width=2, penColor2, smooth,samples=100,domain=(5:9)] ({x},{0});
          \addplot [line width=2, penColor2, smooth,samples=100,domain=(10:14),->] ({x},{0});
          %\addplot [color=penColor2,fill=white,only marks,mark=*] coordinates{(4,0)};
          \addplot [color=penColor2,only marks,mark=*] coordinates{(4,0)};

          \node at (axis cs:2,0) [penColor2] {$]$};
          \node at (axis cs:5,0) [penColor2] {$($};
          \node at (axis cs:9,0) [penColor2] {$]$};
          \node at (axis cs:10,0) [penColor2] {$($};


           %\addplot [ultra thick, penColor5, smooth,domain=0:60] {1/2*(x-31)+26};  
           %\addplot[only marks, mark=*] coordinates {(31, 26)};
          %\addplot [ultra thick, penColor4, smooth,domain=20:43] {-2*(x-31)+26};  
            \end{axis}
\end{tikzpicture}
\end{image}






\begin{image}
\begin{tikzpicture}
	\begin{axis}[
            %xmin=-25,xmax=25,ymin=-25,ymax=25,
            %width=3in,
            clip=false,
            axis lines=center,
            %ticks=none,
            unit vector ratio*=1 1 1,
            ymajorticks=false,
            xtick={-5,0,5,10,15},
            %xlabel=$x$, ylabel=$y$,
            %every axis y label/.style={at=(current axis.above origin),anchor=south},
            every axis x label/.style={at=(current axis.right of origin),anchor=west},
          ]      
       

          \addplot [line width=2, penColor2, smooth,samples=100,domain=(-5:2),<-] ({x},{0});
          \addplot [line width=2, penColor2, smooth,samples=100,domain=(5:9)] ({x},{0});
          \addplot [line width=2, penColor2, smooth,samples=100,domain=(10:14),->] ({x},{0});

          \node at (axis cs:2,0) [penColor2] {$]$};
          \node at (axis cs:5,0) [penColor2] {$($};
          \node at (axis cs:9,0) [penColor2] {$]$};
          \node at (axis cs:10,0) [penColor2] {$($};


           %\addplot [ultra thick, penColor5, smooth,domain=0:60] {1/2*(x-31)+26};  
           %\addplot[only marks, mark=*] coordinates {(31, 26)};
          %\addplot [ultra thick, penColor4, smooth,domain=20:43] {-2*(x-31)+26};  
            \end{axis}
\end{tikzpicture}
\end{image}






\section{Set Builder Notation}

Set builder notation is an official way to use inequalities when describing sets of real numbers.


$\{ r \in R\, | \, r \leq 7 \}$





\section{Interval Notation}

$[-\infty, -3] \cup (4,8) \cup \{9\}$







\end{document}
