\documentclass{ximera}


\graphicspath{
  {./}
  {ximeraTutorial/}
  {basicPhilosophy/}
}

\newcommand{\mooculus}{\textsf{\textbf{MOOC}\textnormal{\textsf{ULUS}}}}

\usepackage{tkz-euclide}\usepackage{tikz}
\usepackage{tikz-cd}
\usetikzlibrary{arrows}
\tikzset{>=stealth,commutative diagrams/.cd,
  arrow style=tikz,diagrams={>=stealth}} %% cool arrow head
\tikzset{shorten <>/.style={ shorten >=#1, shorten <=#1 } } %% allows shorter vectors

\usetikzlibrary{backgrounds} %% for boxes around graphs
\usetikzlibrary{shapes,positioning}  %% Clouds and stars
\usetikzlibrary{matrix} %% for matrix
\usepgfplotslibrary{polar} %% for polar plots
\usepgfplotslibrary{fillbetween} %% to shade area between curves in TikZ
\usetkzobj{all}
\usepackage[makeroom]{cancel} %% for strike outs
%\usepackage{mathtools} %% for pretty underbrace % Breaks Ximera
%\usepackage{multicol}
\usepackage{pgffor} %% required for integral for loops



%% http://tex.stackexchange.com/questions/66490/drawing-a-tikz-arc-specifying-the-center
%% Draws beach ball
\tikzset{pics/carc/.style args={#1:#2:#3}{code={\draw[pic actions] (#1:#3) arc(#1:#2:#3);}}}



\usepackage{array}
\setlength{\extrarowheight}{+.1cm}
\newdimen\digitwidth
\settowidth\digitwidth{9}
\def\divrule#1#2{
\noalign{\moveright#1\digitwidth
\vbox{\hrule width#2\digitwidth}}}






\DeclareMathOperator{\arccot}{arccot}
\DeclareMathOperator{\arcsec}{arcsec}
\DeclareMathOperator{\arccsc}{arccsc}

















%%This is to help with formatting on future title pages.
\newenvironment{sectionOutcomes}{}{}


\title{Subsets of Real Numbers}

\begin{document}

\begin{abstract}
intervals
\end{abstract}
\maketitle




We have whittled our investigations down to a study of real-valued functions.  These are functions whose domain and range are both subsets of the real numbers.  

This means we will need to communicate about sets of real numbers.  We have several.



\section{Number Line Graphs}


\begin{image}
\begin{tikzpicture}
	\begin{axis}[
            %xmin=-25,xmax=25,ymin=-25,ymax=25,
            %width=3in,
            clip=false,
            axis lines=center,
            %ticks=none,
            unit vector ratio*=1 1 1,
            ymajorticks=false,
            xtick={-5,0,5,10,15},
            %xlabel=$x$, ylabel=$y$,
            %every axis y label/.style={at=(current axis.above origin),anchor=south},
            every axis x label/.style={at=(current axis.right of origin),anchor=west},
          ]      
       

          \addplot [line width=2, penColor2, smooth,samples=100,domain=(-5:2),<-] ({x},{0});
          \addplot [line width=2, penColor2, smooth,samples=100,domain=(5:9)] ({x},{0});
          \addplot [line width=2, penColor2, smooth,samples=100,domain=(10:14),->] ({x},{0});

          \node at (axis cs:2,0) [penColor2] {$]$};
          \node at (axis cs:5,0) [penColor2] {$($};
          \node at (axis cs:9,0) [penColor2] {$]$};
          \node at (axis cs:10,0) [penColor2] {$($};


           %\addplot [ultra thick, penColor5, smooth,domain=0:60] {1/2*(x-31)+26};  
           %\addplot[only marks, mark=*] coordinates {(31, 26)};
          %\addplot [ultra thick, penColor4, smooth,domain=20:43] {-2*(x-31)+26};  
            \end{axis}
\end{tikzpicture}
\end{image}






\begin{image}
\begin{tikzpicture}
	\begin{axis}[
            %xmin=-25,xmax=25,ymin=-25,ymax=25,
            %width=3in,
            clip=false,
            axis lines=center,
            %ticks=none,
            unit vector ratio*=1 1 1,
            ymajorticks=false,
            xtick={-5,0,5,10,15},
            %xlabel=$x$, ylabel=$y$,
            %every axis y label/.style={at=(current axis.above origin),anchor=south},
            every axis x label/.style={at=(current axis.right of origin),anchor=west},
          ]      
       

          \addplot [line width=2, penColor2, smooth,samples=100,domain=(-5:2),<-] ({x},{0});
          \addplot [line width=2, penColor2, smooth,samples=100,domain=(5:9)] ({x},{0});
          \addplot [line width=2, penColor2, smooth,samples=100,domain=(10:14),->] ({x},{0});

          \node at (axis cs:2,0) [penColor2] {$]$};
          \node at (axis cs:5,0) [penColor2] {$($};
          \node at (axis cs:9,0) [penColor2] {$]$};
          \node at (axis cs:10,0) [penColor2] {$($};


           %\addplot [ultra thick, penColor5, smooth,domain=0:60] {1/2*(x-31)+26};  
           %\addplot[only marks, mark=*] coordinates {(31, 26)};
          %\addplot [ultra thick, penColor4, smooth,domain=20:43] {-2*(x-31)+26};  
            \end{axis}
\end{tikzpicture}
\end{image}






\section{Set Builder Notation}

Set builder notation is an official way to use inequalities when describing sets of real numbers.


$\{ r \in R\, | \, r \leq 7 \}$





\section{Interval Notation}

$[-\infty, -3] \cup (4,8) \cup \{9\}$







\end{document}
