\documentclass{ximera}


\graphicspath{
  {./}
  {ximeraTutorial/}
  {basicPhilosophy/}
}

\newcommand{\mooculus}{\textsf{\textbf{MOOC}\textnormal{\textsf{ULUS}}}}

\usepackage{tkz-euclide}\usepackage{tikz}
\usepackage{tikz-cd}
\usetikzlibrary{arrows}
\tikzset{>=stealth,commutative diagrams/.cd,
  arrow style=tikz,diagrams={>=stealth}} %% cool arrow head
\tikzset{shorten <>/.style={ shorten >=#1, shorten <=#1 } } %% allows shorter vectors

\usetikzlibrary{backgrounds} %% for boxes around graphs
\usetikzlibrary{shapes,positioning}  %% Clouds and stars
\usetikzlibrary{matrix} %% for matrix
\usepgfplotslibrary{polar} %% for polar plots
\usepgfplotslibrary{fillbetween} %% to shade area between curves in TikZ
\usetkzobj{all}
\usepackage[makeroom]{cancel} %% for strike outs
%\usepackage{mathtools} %% for pretty underbrace % Breaks Ximera
%\usepackage{multicol}
\usepackage{pgffor} %% required for integral for loops



%% http://tex.stackexchange.com/questions/66490/drawing-a-tikz-arc-specifying-the-center
%% Draws beach ball
\tikzset{pics/carc/.style args={#1:#2:#3}{code={\draw[pic actions] (#1:#3) arc(#1:#2:#3);}}}



\usepackage{array}
\setlength{\extrarowheight}{+.1cm}
\newdimen\digitwidth
\settowidth\digitwidth{9}
\def\divrule#1#2{
\noalign{\moveright#1\digitwidth
\vbox{\hrule width#2\digitwidth}}}






\DeclareMathOperator{\arccot}{arccot}
\DeclareMathOperator{\arcsec}{arcsec}
\DeclareMathOperator{\arccsc}{arccsc}

















%%This is to help with formatting on future title pages.
\newenvironment{sectionOutcomes}{}{}


\title{Function Properties}

\begin{document}

\begin{abstract}
characteristics, features
\end{abstract}
\maketitle





Numbers have properties and we use these properties to define sets of numbers. \\



\begin{itemize}
\item Evens:  $\{ \cdots, -6, -4, -2, 0, 2, 4, 6, \cdots \} = \{ 2n   \, | \, n \in \mathbb{Z} \}$
\item Squares:  $\{ 1, 4, 9, 16, \cdots \} = \{ n^2   \, | \, n \in \mathbb{N} \}$
\item Primes:  $\{ 2, 3, 5, 7, 11,  \cdots \}$
\end{itemize}

Once we know that a number belongs to one of these sets, i.e. has the property, then we automatically know some information about the number. \\


\textbf{\textcolor{red!90!darkgray}{$\blacktriangleright$ Same with functions.}}  \\



A major goal of this course is to develop a list of function properties and define sets of functions via these properties.  Once we know a function belongs to a set, then we get a lot of free information about the function. \\









\section{Function Characteristics or Properties}


Let's start off with two basic properties: \textbf{Onto} and \textbf{One-to-One}.







\begin{definition} \textbf{\textcolor{green!50!black}{Onto}} \\

If the range equals the codomain, then the function is said to be \textbf{onto}.

\end{definition}









\begin{example} \textit{SuperBowlWinner}


\textit{SuperBowlWinner} is not an onto function.


\begin{itemize}
\item The Cleveland Browns are in teh codomain, but are not in the range of \textit{SuperBowlWinner}.  
\end{itemize}

\end{example}







Some functions are onto. Some are not. \\




The one and only rule for a function is that each domain item is in exactly one pair.  Range items can be in multiple pairs or no pairs. \\

However, if the range also follows this rule, then the function is called a \textbf{one-to-one} function.



\begin{definition} \textbf{\textcolor{green!50!black}{One-to-One}} \\

A \textbf{one-to-one} function is a function in which each range number is in exactly one pair.

\end{definition}




Some functions are one-to-one. Some are not. \\














\begin{example} \textit{SuperBowlWinner}


\textit{SuperBowlWinner} is not a one-to-one function.


\begin{itemize}
\item The Pittsburgh Steelers is a range team that is in multiple pairs. 
\end{itemize}

\end{example}












\begin{example} \textit{StateCapitals}


\textit{StateCapitals} pairs U.S. states with a city that has serve as the state's capital. Its domain is U.S. states and the codomain is all U.S. cities.\\



This function is not well-defined, because a state may have had different capital cities depending on the year. \\


\textit{StateCapitals2020} pairs U.S. states with their capital city on January 1, 2020. Its domain is U.S. states and the codomain is all U.S. cities.\\


This is a well-defined function.  It is a one-to-one function, because on January 1, 2020 no city was the capital of two states.  It is not onto, since not every U.S. city is a state capital.


\end{example}












$\blacktriangleright$ \textbf{Example:} A function that is both onto and one-to-one.  \\

\begin{center}
(arrows go from the domain to the codomain.)
\begin{image}
\includegraphics{pics/ontoandonetoone.png}
\end{image}
\end{center}






$\blacktriangleright$ \textbf{Example:} A function that is onto but not one-to-one.  \\

\begin{center}
(arrows go from the domain to the codomain.)
\begin{image}
\includegraphics{pics/onto.png}
\end{image}
\end{center}







$\blacktriangleright$ \textbf{Example:} A function that is one-to-one but not onto.  \\

\begin{center}
(arrows go from the domain to the codomain.)
\begin{image}
\includegraphics{pics/onetoone.png}
\end{image}
\end{center}






$\blacktriangleright$ \textbf{Example:} A function that is neither one-to-one nor onto.  \\

\begin{center}
(arrows go from the domain to the codomain.)
\begin{image}
\includegraphics{pics/neither.png}
\end{image}
\end{center}




These examples illustrate that onto and one-to-one are independent properties.  A function can have either property with or without the other.
















\begin{center}
\textbf{\textcolor{green!50!black}{ooooo=-=-=-=-=-=-=-=-=-=-=-=-=ooOoo=-=-=-=-=-=-=-=-=-=-=-=-=ooooo}} \\

More examples can be found by following this link\\ \link[More Examples of Functions]{https://ximera.osu.edu/csccmathematics/precalculus1/precalculus1/functions/examples/exampleList}

\end{center}





\end{document}
