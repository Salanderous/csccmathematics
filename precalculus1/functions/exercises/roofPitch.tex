\documentclass{ximera}


\graphicspath{
  {./}
  {ximeraTutorial/}
  {basicPhilosophy/}
}

\newcommand{\mooculus}{\textsf{\textbf{MOOC}\textnormal{\textsf{ULUS}}}}

\usepackage{tkz-euclide}\usepackage{tikz}
\usepackage{tikz-cd}
\usetikzlibrary{arrows}
\tikzset{>=stealth,commutative diagrams/.cd,
  arrow style=tikz,diagrams={>=stealth}} %% cool arrow head
\tikzset{shorten <>/.style={ shorten >=#1, shorten <=#1 } } %% allows shorter vectors

\usetikzlibrary{backgrounds} %% for boxes around graphs
\usetikzlibrary{shapes,positioning}  %% Clouds and stars
\usetikzlibrary{matrix} %% for matrix
\usepgfplotslibrary{polar} %% for polar plots
\usepgfplotslibrary{fillbetween} %% to shade area between curves in TikZ
\usetkzobj{all}
\usepackage[makeroom]{cancel} %% for strike outs
%\usepackage{mathtools} %% for pretty underbrace % Breaks Ximera
%\usepackage{multicol}
\usepackage{pgffor} %% required for integral for loops



%% http://tex.stackexchange.com/questions/66490/drawing-a-tikz-arc-specifying-the-center
%% Draws beach ball
\tikzset{pics/carc/.style args={#1:#2:#3}{code={\draw[pic actions] (#1:#3) arc(#1:#2:#3);}}}



\usepackage{array}
\setlength{\extrarowheight}{+.1cm}
\newdimen\digitwidth
\settowidth\digitwidth{9}
\def\divrule#1#2{
\noalign{\moveright#1\digitwidth
\vbox{\hrule width#2\digitwidth}}}






\DeclareMathOperator{\arccot}{arccot}
\DeclareMathOperator{\arcsec}{arcsec}
\DeclareMathOperator{\arccsc}{arccsc}

















%%This is to help with formatting on future title pages.
\newenvironment{sectionOutcomes}{}{}



\author{Alan Yang}

\begin{document}




The pitch of a roof is its vertical 'rise' over its horizontal 'span'. However, most often a ratio of "pitch" (also fraction) is slang used for the (more useful) 'slope' (of rise over 'run') of just one side (half the span) of a dual pitched roof. This is the 'slope' of geometry, stairways and other construction disciplines, or the trigonometric arc tangent function of its decimal fraction. In the imperial measurement systems, "pitch" is usually expressed with the rise first and run second. - Wikipedia


\begin{example} Pitch


Expressed as a fraction the pitch of a roof would look like

\[
\frac{rise}{span} = \frac{5 ft}{6 ft} = \frac{5}{6}
\]

If the units are the same, then they can be dropped, otherwise, the pitch requires the units.
\end{example}






\begin{example} More Common Pitch


Expressed as a fraction the pitch of a roof would look like

\[
\frac{rise}{run} = \frac{rise}{half-span} = \frac{5 ft}{3 ft} = \frac{5}{3}
\]

If the units are the same, then they can be dropped, otherwise, the pitch requires the units.
\end{example}






\begin{image}
\includegraphics{pics/roofPitch.png}
\end{image}


We'll use the more common pitch as $\frac{rise}{run}$, e.g. $run = \frac{1}{2} span$. And, our measurements will always be in feet.  Therefore, we will only need numbers in our domain and codomain.



\subsection{Pitch}
We can create the \textbf{Pitch} relation between the rise and run (half-span).  The domain and range are both length measurements.  An ordered pair $(rise, run)$ would represent a pitch.








\begin{question}

Suppose a roof with a span of $20 ft$ has a pitch of $\frac{2}{5}$.  

The \textit{run} of the roof is $\answer{10} ft$.

The \textit{rise} of the roof is $\answer{4} ft$.

In the \textit{Pitch} relation, this pitch would be represented as $\left( 2, \answer{5} \right)$.

\end{question}











\begin{question}

Suppose a roof with a rise of $8 ft$ has a pitch of $\frac{2}{3}$.  

The \textit{run} of the roof is $\answer{12} ft$.

The \textit{span} of the roof is $\answer{24} ft$.

In the \textit{Pitch} relation, this pitch would be represented as $\left( 2, \answer{5} \right)$.

\end{question}



\begin{exercise}  Function?

In the \textit{Pitch} relation, can $2$ appear as the domain number in more than one pair?
\begin{multipleChoice}
\choice [correct]{Yes}
\choice {No}
\end{multipleChoice}





Is the \textit{Pitch} relation a function?
\begin{multipleChoice}
\choice {Yes}
\choice [correct]{No}
\end{multipleChoice}

\end{exercise}




Let's define a new pitch relation called \textit{HalfPitch}.  All pitches in this relation equal $\frac{1}{2}$.  (The rise and the run are still always measured in feet.)




\begin{question}

Suppose a roof has a rise of $8 ft$.  

Then the run is $\answer{16} ft$ and the span is $\answer{32} ft$.

\[
\frac{1}{2} = \frac{8}{\answer{16}}
\]


In the \textit{HalfPitch} relation, this pitch would be represented as $\left( 8, \answer{16} \right)$.

\end{question}








\begin{question}

Suppose a roof has a span of $8 ft$.  

Then the run is $\answer{4} ft$ and the rise is $\answer{2} ft$.

\[
\frac{1}{2} = \frac{\answer{2}}{4}
\]


In the \textit{HalfPitch} relation, this pitch would be represented as $\left( 8, \answer{16} \right)$.

\end{question}







\begin{exercise}  Function?

In the \textit{HalfPitch} relation, can $2$ appear as the domain number in more than one pair?
\begin{multipleChoice}
\choice {Yes}
\choice [correct]{No}
\end{multipleChoice}





Is the \textit{HalfPitch} relation a function?
\begin{multipleChoice}
\choice [correct]{Yes}
\choice {No}
\end{multipleChoice}

\end{exercise}






\end{document}