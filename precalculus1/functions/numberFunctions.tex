\documentclass{ximera}


\graphicspath{
  {./}
  {ximeraTutorial/}
  {basicPhilosophy/}
}

\newcommand{\mooculus}{\textsf{\textbf{MOOC}\textnormal{\textsf{ULUS}}}}

\usepackage{tkz-euclide}\usepackage{tikz}
\usepackage{tikz-cd}
\usetikzlibrary{arrows}
\tikzset{>=stealth,commutative diagrams/.cd,
  arrow style=tikz,diagrams={>=stealth}} %% cool arrow head
\tikzset{shorten <>/.style={ shorten >=#1, shorten <=#1 } } %% allows shorter vectors

\usetikzlibrary{backgrounds} %% for boxes around graphs
\usetikzlibrary{shapes,positioning}  %% Clouds and stars
\usetikzlibrary{matrix} %% for matrix
\usepgfplotslibrary{polar} %% for polar plots
\usepgfplotslibrary{fillbetween} %% to shade area between curves in TikZ
\usetkzobj{all}
\usepackage[makeroom]{cancel} %% for strike outs
%\usepackage{mathtools} %% for pretty underbrace % Breaks Ximera
%\usepackage{multicol}
\usepackage{pgffor} %% required for integral for loops



%% http://tex.stackexchange.com/questions/66490/drawing-a-tikz-arc-specifying-the-center
%% Draws beach ball
\tikzset{pics/carc/.style args={#1:#2:#3}{code={\draw[pic actions] (#1:#3) arc(#1:#2:#3);}}}



\usepackage{array}
\setlength{\extrarowheight}{+.1cm}
\newdimen\digitwidth
\settowidth\digitwidth{9}
\def\divrule#1#2{
\noalign{\moveright#1\digitwidth
\vbox{\hrule width#2\digitwidth}}}






\DeclareMathOperator{\arccot}{arccot}
\DeclareMathOperator{\arcsec}{arcsec}
\DeclareMathOperator{\arccsc}{arccsc}

















%%This is to help with formatting on future title pages.
\newenvironment{sectionOutcomes}{}{}


\title{Number Functions}

\begin{document}

\begin{abstract}
measurements
\end{abstract}
\maketitle





Most of the comparisons that interest us in mathematics involve measurements and we measure eveything...\\


Count, Length, Weight, Volume, Odor, Density, Brightness, Strength, Pressure, Heat, Temperature, Loudness, Change, Speed, Direction, Angle, Moisture, Voltage, Current, Tone, Notes, Satisfaction, Likelihood, Distance, Absorption, Reflection, Position, Heat, Magnetism, Sweetness, Sour, Focus, Flexibility, Pollution, Time, Smoothness, Humor, Stress, Area, Rates, Shape, Location, Orientation, Health... \\



...it just keeps going and going and going. \\


We compare all of these to each other:



\begin{itemize}
\item Odor vs. Direction
\item Pollution vs. Location
\item Flexibility vs. Density
\item Heat vs. Pressure
\item Voltage vs. Shape 
\end{itemize}







\section{Number Functions}

We make functions connecting just about everything.  In particular, we could make functions where both the domain and codomain are sets of measurements. These are the types of functions we study in Calculus.  Therefore, we are making yet another effort to focus our attention.  And, since, measurements are real numbers accompanied by a unit, we will concentrate on functions that connect sets of real numbers with sets of real numbers. \\

Technically, our functions will connect sets of measurements with sets of measurements. However, we usually hold the measurement units off to the side, work with the numbers, and then bring back the units when we interpret our results.  Once we arrive at any conclusions, then we will interpret our findings within the context of the situation under investigation and the measurements involved.




\begin{question} \textit{Double} \\
The \textit{Double} function pairs a real number with its double.

domain = all real numbers  \\ 
codomain = all real numbers


\begin{itemize}
\item \textit{Double}($3$) = $6$.
\item \textit{Double}($-4$) = $-8$.
\item \textit{Double}($\pi$) = $2 \pi$.

\item $\textit{Double}(7) = \answer{14}$.
\end{itemize}

\end{question} 



\begin{question} \textit{Half} \\
The \textit{Half} function pairs a real number with its half.

domain = positive real numbers  \\ 
codomain = positive real numbers


Solve \textit{Half}(d) = $8$ \\

The solution is $d = \answer{16}$.

\end{question} 




\begin{question} \textit{Successor} \\
The \textit{Successor} function pairs an integer with one more than the integer.

domain = all integers  \\ 
codomain = all integers


\begin{itemize}
\item \textit{Successor}($3$) = $4$.
\item \textit{Successor}($-4$) = $-3$.
\item \textit{Successor}($0$) = $1$.
\item \textit{Successor}($\pi$) = $undefined$.

\item $\textit{Successor}(-7) = \answer{-8}$.
\end{itemize}


Solve \textit{Successor}(z) = $-1$ \\

The solution is $d = \answer{-2}$.

\end{question} 







\begin{question} \textit{Successor} \\
Is every integer in the range of \textit{Successor}?


\begin{multipleChoice}
	\choice[correct]{Yes}
	\choice{No}
\end{multipleChoice}
\begin{feedback}
Every integer has a previous integer.  
\end{feedback}


Is \textit{Successor} an onto function?
\begin{multipleChoice}
	\choice[correct]{Yes}
	\choice{No}
\end{multipleChoice}


\end{question} 











If our function domains and ranges will be sets of real numbers, then it seems we should know about sets of real numbers.

Therefore, we need a way to communicate about sets of real numbers.
















\end{document}
