\documentclass{ximera}


\graphicspath{
  {./}
  {ximeraTutorial/}
  {basicPhilosophy/}
}

\newcommand{\mooculus}{\textsf{\textbf{MOOC}\textnormal{\textsf{ULUS}}}}

\usepackage{tkz-euclide}\usepackage{tikz}
\usepackage{tikz-cd}
\usetikzlibrary{arrows}
\tikzset{>=stealth,commutative diagrams/.cd,
  arrow style=tikz,diagrams={>=stealth}} %% cool arrow head
\tikzset{shorten <>/.style={ shorten >=#1, shorten <=#1 } } %% allows shorter vectors

\usetikzlibrary{backgrounds} %% for boxes around graphs
\usetikzlibrary{shapes,positioning}  %% Clouds and stars
\usetikzlibrary{matrix} %% for matrix
\usepgfplotslibrary{polar} %% for polar plots
\usepgfplotslibrary{fillbetween} %% to shade area between curves in TikZ
\usetkzobj{all}
\usepackage[makeroom]{cancel} %% for strike outs
%\usepackage{mathtools} %% for pretty underbrace % Breaks Ximera
%\usepackage{multicol}
\usepackage{pgffor} %% required for integral for loops



%% http://tex.stackexchange.com/questions/66490/drawing-a-tikz-arc-specifying-the-center
%% Draws beach ball
\tikzset{pics/carc/.style args={#1:#2:#3}{code={\draw[pic actions] (#1:#3) arc(#1:#2:#3);}}}



\usepackage{array}
\setlength{\extrarowheight}{+.1cm}
\newdimen\digitwidth
\settowidth\digitwidth{9}
\def\divrule#1#2{
\noalign{\moveright#1\digitwidth
\vbox{\hrule width#2\digitwidth}}}






\DeclareMathOperator{\arccot}{arccot}
\DeclareMathOperator{\arcsec}{arcsec}
\DeclareMathOperator{\arccsc}{arccsc}

















%%This is to help with formatting on future title pages.
\newenvironment{sectionOutcomes}{}{}


\title{Real-Valued Functions}

\begin{document}

\begin{abstract}
connecting reals
\end{abstract}
\maketitle




\section{Real-Valued Functions}

For the most part, our attention in this course is on real-valued functions.




\begin{definition} real-valued \\

A real-valued function is one whose domain and codomain are both sets of real numbers.

\end{definition}
The values of a real-valued function are real numbers.







\begin{example} Squaring Function \\

Let the function $SQ$ be defined as follows.


\begin{itemize}
\item Domain of $SQ$ is $(-3, 5]$.
\item Codomain of $SQ$ is $[-30, 30)$.
\item $SQ$ pairs a domain number with its square.
\end{itemize}


First, this function is well-defined since the square of any number in $(-3, 5]$ will be in $[-30, 30)$ and each domain number has exactly one square.


\textbf{Shorthand Notation: } $SQ: (-3, 5] \rightarrow [-30, 30)$.

\begin{question}
Evaluate the following:

\begin{itemize}
	\item $SQ(-2) = \answer{4}$
	\item $SQ(0) = \answer{0}$
	\item $SQ(1.1) = \answer{1.21}$
	\item $SQ\left(\frac{7}{5}\right) = \answer{\frac{49}{25}}$
\end{itemize}

\end{question}

\end{example}











\begin{example} Collatz \\

Let the function $C$ be defined as follows.


\begin{itemize}
\item Domain of $C$ is the positive integers: $\mathbb{N}$.
\item Codomain of $C$ is the positive integers: $\mathbb{N}$.
\item $C$ pairs a domain number one of two numbers.
	\begin{itemize}
			\item If the domain number is even, then $C$ pairs it with half the domain number.
			\item If the domain number is odd, then $C$ pairs it with one more than three times the domain number.
	\end{itemize}
\end{itemize}


First, this function is well-defined the calculations can only produce one result.


\textbf{Shorthand Notation: } $C: \mathbb{N} \rightarrow \mathbb{N}$.

\begin{question}
Evaluate the following:

\begin{itemize}
	\item $C(8) = \answer{4}$
	\item $C(7) = \answer{22}$
	\item $C(1) = \answer{4}$
	\item $C(28) = \answer{14}$
\end{itemize}

\end{question}

\end{example}









\begin{example} Identity \\

Let the function $Id$ be defined as follows.


\begin{itemize}
\item Domain of $Id$ is all real numbers: $\mathbb{R}$.
\item Codomain of $Id$ is all real numbers: $\mathbb{R}$.
\item $Id$ pairs a domain number with itself.
\end{itemize}


First, this function is well-defined.


\textbf{Shorthand Notation: } $Id: \mathbb{R} \rightarrow \mathbb{R}$.

\begin{question}
Evaluate the following:

\begin{itemize}
	\item $Id(\pi) = \answer{\pi}$
	\item $Id(\sqrt{5}) = \answer{\sqrt{5}}$
	\item $Id\left(\frac{13}{27}\right) = \answer{\frac{13}{27}}$
	\item $Id(0) = \answer{0}$
\end{itemize}

\end{question}

\end{example}













\begin{example} Remainder \\

Let the function $Remainder$ be defined as follows.


\begin{itemize}
\item Domain of $Remainder$ is all natural numbers: $\mathbb{N}$.
\item Codomain of $Remainder$ is $\{ 0, 1, 2, 3, 4 \}$.
\item $Remainder$ pairs a natural number with the remainder when divided by $5$.
\end{itemize}


First, this function is well-defined. It can only have one remainder.


\textbf{Shorthand Notation: } $Remainder: \mathbb{N} \rightarrow \{ 0, 1, 2, 3, 4 \}$.

\begin{question}
Evaluate the following:

\begin{itemize}
	\item $Remainder(101) = \answer{1}$
	\item $Remainder(3) = \answer{3}$
	\item $Remainder(1652435) = \answer{0}$
	\item $Remainder(9) = \answer{4}$
\end{itemize}

\end{question}



\end{example}










\begin{example} Linear \\

Let the function $L$ be defined as follows.


\begin{itemize}
\item Domain of $L$ is $[-3, 4)$.
\item Codomain of $L$ is $\mathbb{R}$.
\item $L$ pairs a number with twice the number.
\end{itemize}


First, this function is well-defined. 


\textbf{Shorthand Notation: } $L: [-3, 4) \rightarrow \mathbb{R}$.

\begin{question}

The codomain of $L$ is $\mathbb{R}$ but the range is not.  

The range is \wordChoice{\choice[correct]{[}\choice{(}} \wordChoice{\choice{-8}\choice[correct]{-6},\choice{6}\choice{8}} , \wordChoice{\choice{-8}\choice{-6},\choice{6}\choice[correct]{8}} \wordChoice{\choice{]}\choice[correct]{)}}.

\end{question}



\end{example}
...more communication.

The range is also called the \textbf{image} of the function.   Sometimes the range partner of a domain number is called the \textbf{image} of the domain number.

$f(a)$ is called the "value of the function at a" or "the image of a".


$f(a)$ is pronounced "f of a".















\end{document}
