\documentclass{ximera}


\graphicspath{
  {./}
  {ximeraTutorial/}
  {basicPhilosophy/}
}

\newcommand{\mooculus}{\textsf{\textbf{MOOC}\textnormal{\textsf{ULUS}}}}

\usepackage{tkz-euclide}\usepackage{tikz}
\usepackage{tikz-cd}
\usetikzlibrary{arrows}
\tikzset{>=stealth,commutative diagrams/.cd,
  arrow style=tikz,diagrams={>=stealth}} %% cool arrow head
\tikzset{shorten <>/.style={ shorten >=#1, shorten <=#1 } } %% allows shorter vectors

\usetikzlibrary{backgrounds} %% for boxes around graphs
\usetikzlibrary{shapes,positioning}  %% Clouds and stars
\usetikzlibrary{matrix} %% for matrix
\usepgfplotslibrary{polar} %% for polar plots
\usepgfplotslibrary{fillbetween} %% to shade area between curves in TikZ
\usetkzobj{all}
\usepackage[makeroom]{cancel} %% for strike outs
%\usepackage{mathtools} %% for pretty underbrace % Breaks Ximera
%\usepackage{multicol}
\usepackage{pgffor} %% required for integral for loops



%% http://tex.stackexchange.com/questions/66490/drawing-a-tikz-arc-specifying-the-center
%% Draws beach ball
\tikzset{pics/carc/.style args={#1:#2:#3}{code={\draw[pic actions] (#1:#3) arc(#1:#2:#3);}}}



\usepackage{array}
\setlength{\extrarowheight}{+.1cm}
\newdimen\digitwidth
\settowidth\digitwidth{9}
\def\divrule#1#2{
\noalign{\moveright#1\digitwidth
\vbox{\hrule width#2\digitwidth}}}






\DeclareMathOperator{\arccot}{arccot}
\DeclareMathOperator{\arcsec}{arcsec}
\DeclareMathOperator{\arccsc}{arccsc}

















%%This is to help with formatting on future title pages.
\newenvironment{sectionOutcomes}{}{}


\title{Functions}

\begin{document}

\begin{abstract}
%Stuff can go here later if we want!
\end{abstract}
\maketitle





\section{Anatomy of a Function}


Functions are special relations. While a relation is just two sets of items and some pairings between the two sets, functions satisfy one rule: \\


\begin{condition} \textbf{\textcolor{purple!85!blue}{THE Rule}}  \\

\begin{itemize}
\item Functions are those relations where each domain item is paired with \underline{exactly} one codomain item.
\end{itemize}
\end{condition}


\begin{center}
\textbf{OR}
\end{center}


\begin{condition} \textbf{\textcolor{purple!85!blue}{THE Rule}} \\

\begin{itemize}
\item Functions are those relations where each domain item occurs in \underline{exactly} one pair.
\end{itemize}
\end{condition}




This simple rule makes a world of difference. But, we can do a little better for this course. \\




This course (and Calculus) is a study of measurements and how they compare to each other, which means it is a study of the real numbers.  Therefore, the domains and ranges of our functions will be subsets of the real numbers.  We say that our functions \textbf{map} real numbers to real numbers

\[
\text{ f : } \mathbb{R} \mapsto \mathbb{R}
\]





With the addition of this one rule and the restriction of our domains and codomains to be sets of real numbers, we are ready to begin our journey towards Calculus.








\subsection{Expectations}


\begin{sectionOutcomes}
In this section, students will 

\begin{itemize}
\item use function notation.
\item evaluate functions.
\item solve equations involving functions.
\item analyze functions.
\item focus on numeric functions.
\end{itemize}
\end{sectionOutcomes}







\begin{center}
\textbf{\textcolor{green!50!black}{ooooo=-=-=-=-=-=-=-=-=-=-=-=-=ooOoo=-=-=-=-=-=-=-=-=-=-=-=-=ooooo}} \\

more examples can be found by following this link\\ \link[More Examples of Functions]{https://ximera.osu.edu/csccmathematics/precalculus1/precalculus1/functions/examples/exampleList}

\end{center}







\end{document}
