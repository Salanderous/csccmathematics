\documentclass{ximera}


\graphicspath{
  {./}
  {ximeraTutorial/}
  {basicPhilosophy/}
}

\newcommand{\mooculus}{\textsf{\textbf{MOOC}\textnormal{\textsf{ULUS}}}}

\usepackage{tkz-euclide}\usepackage{tikz}
\usepackage{tikz-cd}
\usetikzlibrary{arrows}
\tikzset{>=stealth,commutative diagrams/.cd,
  arrow style=tikz,diagrams={>=stealth}} %% cool arrow head
\tikzset{shorten <>/.style={ shorten >=#1, shorten <=#1 } } %% allows shorter vectors

\usetikzlibrary{backgrounds} %% for boxes around graphs
\usetikzlibrary{shapes,positioning}  %% Clouds and stars
\usetikzlibrary{matrix} %% for matrix
\usepgfplotslibrary{polar} %% for polar plots
\usepgfplotslibrary{fillbetween} %% to shade area between curves in TikZ
\usetkzobj{all}
\usepackage[makeroom]{cancel} %% for strike outs
%\usepackage{mathtools} %% for pretty underbrace % Breaks Ximera
%\usepackage{multicol}
\usepackage{pgffor} %% required for integral for loops



%% http://tex.stackexchange.com/questions/66490/drawing-a-tikz-arc-specifying-the-center
%% Draws beach ball
\tikzset{pics/carc/.style args={#1:#2:#3}{code={\draw[pic actions] (#1:#3) arc(#1:#2:#3);}}}



\usepackage{array}
\setlength{\extrarowheight}{+.1cm}
\newdimen\digitwidth
\settowidth\digitwidth{9}
\def\divrule#1#2{
\noalign{\moveright#1\digitwidth
\vbox{\hrule width#2\digitwidth}}}






\DeclareMathOperator{\arccot}{arccot}
\DeclareMathOperator{\arcsec}{arcsec}
\DeclareMathOperator{\arccsc}{arccsc}

















%%This is to help with formatting on future title pages.
\newenvironment{sectionOutcomes}{}{}


\title{Closeness}

\begin{document}

\begin{abstract}
distance
\end{abstract}
\maketitle










Given two real numbers, $a$ and $b$, you can measure the distance between them, symbolized by $|b-a|$.  We use this measurement to decide if things are close.  The smaller $|b-a|$, the closer $a$ and $b$ are. \\



Or, we could think the other way.  Given a real number $a$ and a distance $\delta$, we could describe all of the real numbers closer to $a$ than $\delta$.  They would be the numbers inside the interval $(a-\delta, a+\delta)$, which we could describe with absolute value.

\[      (a-\delta, a+\delta) = \{ r \in \mathbb{R} \, | \, |r - a| < \delta        \}       \]


\textbf{Note:}  $\delta$ is a very popular symbol to represent small distances.  So, is $\epsilon$.



And, since $(a-\delta, a+\delta)$ is an open interval, we know there is a number in here.  In fact we know there is a number in the left half, $(a-\delta, a)$ and we know there is a number in the right half, $(a, a+\delta)$.


And, it doesn't matter how small $\delta$ is.  There will always be a number in the left half and in the right half, because we have an open interval.


In an open interval, we can always get closer.















\section{Image}


We could picture $(a-\delta, a+\delta)$ inside the domain of a function, $f$, and then investigate the image of this interval under $f$.  If $I$ represents an interval, then $f(I)$ represents its image under $f$.


\[       f(I) = \{   f(a)  \, | \, a \in I            \}             \]




\begin{example} Image

Let $f(x) = 2x + 5$ with its natural domain.

Let  $I = (4-\delta, 4+\delta)$, where $\delta > 0$ is a very small real number.

Then $I$ is an interval in the domain of $f$.


Its image is $f(I) = \{   f(a)  \, | \, a \in I \}  = \{   2a+5  \, | \, a \in (4-\delta, 4+\delta) \}$  \\



$f$ will stretch the interval by a factor of $2$ and shift it by $5$.


$(2(4-\delta)+5, 2(4+\delta)+5)$


$(13-2\delta, 13+2\delta)$






\end{example}













What if we went the other way and started with a target in the range? \\

A target maybe too difficult to hit EXACTLY.  Instead, let's just try to land inside the target interval. \\



What if, in the previous example, we wanted $f(I) \subset (6-\epsilon, 6+\epsilon)$, where $\epsilon > 0$ is some small distance?  \\


What are the possibilities for $I$?  \\







To land near $6$, $f(x) = 2x + 5 = 6$, $x$ would need to be near $\frac{1}{2}$.  \\

$I$ would need to look like $\left( \frac{1}{2} - \delta, \frac{1}{2}+ \delta \right)$  But this isn't going to work if $\delta$ is too big compared to $\epsilon$.  We need to describe $\delta$ in terms of $\epsilon$ to make sure.  And, we are in the business of making sure. \\



$\delta$ will be multiplied by $2$ and we need this to be smaller than $\epsilon$.  Therefore, we need $\delta < \frac{\epsilon}{2}$ for the image to land inside our target interval.  Let's pick $\delta < \frac{\epsilon}{3}$.  That makes sure. \\



$I = \left( \frac{1}{2} - \frac{\epsilon}{3}, \frac{1}{2} + \frac{\epsilon}{3} \right)$




When working backwards through the function, like this, we use the word \textbf{preimage}.



























\section{Preimage}




The image of a domain interval is the set of function values that occur at those domain numbers.  The \textbf{preimage} is the reverse.  

Given an interval, $I$, in the range (an interval of function values), the preimage is the set of od domain numbers where the function value is a mamber of $I$.

Unfortunately for communication purposes, we reuse some notation for the preimage.



\begin{definition} \textbf{\textcolor{green!50!black}{Preimage}}


Let $S \subset Ran_f$ be a subset of the range of $f$.  Then the \textbf{preimage of $S$} is given by

\[       f^{-1}(S) = \{   a \in Dom_f  \, | \, f(a) \in S  \}             \]



\textbf{Note:}  The $-1$ exponent means reverse here. The reverse of $f$.  We also use a $-1$ exponent to mean reciprocal (which is the reverse of multipication).  When reading mathematics, we need to take into account the context when interpreting the symbols and notation.


\end{definition}












\begin{example} Preimage

Let $G(t) = -4t - 3$ with its natural domain.  The natural range is $\mathbb{R}$.

Let  $I = (10-\epsilon, 10+\epsilon)$, where $\epsilon > 0$ is a small positive real number.  

Then $I$ is a small \textbf{neighborhood} of $10$ in the range of $G$.


What is the preimage of $I$?

\begin{explanation}





$G$ will stretch a domain interval by a factor of $\answer{4}$.  It will also reflect it and shift it $3$.  


First, we need a domain interval whose image will be around $10$.

$-4t-3 = \answer{10}$

$-4t = 13$

$t = -\frac{13}{4}$



The preimage domain interval looks like $\left( -\frac{13}{4} - \delta, -\frac{13}{4} + \delta \right)$.  Rather than just stating that $\delta$ is small, we need to be specific about its smallness compared to $\epsilon$, by specify $\delta$ in terms of $\epsilon$.

$G$ will stretch $\delta$ by a factor of $4$, therefore,  we need $\delta = \frac{\epsilon}{4}$. 




\[   G^{-1}((10-\epsilon, 10+\epsilon)) = \left( -\frac{13}{4} - \frac{\epsilon}{4}, -\frac{13}{4} + \frac{\epsilon}{4} \right)   \]


\end{explanation}
\end{example}











\begin{example} Preimage

Let $H(k) = 2k - 1$ with its natural domain.  The natural range is $\mathbb{R}$.

Let  $I = (-5-\epsilon, -5+\epsilon)$, where $\epsilon > 0$ is a small positive real number.  

Then $I$ is a small \textbf{neighborhood} of $-5$ in the range of $H$.


Identify an interval, $D$ in the domain of $H$, such that $H(D) \subset I$.

\begin{explanation}





$H$ will stretch a domain interval by a factor of $\answer{2}$.  It will also shift it $1$.  


First, we need a domain interval whose image will be around $-5$.

$2k - 1 = \answer{-5}$

$2k = -4$

$k = -2$



The preimage domain interval looks like $\left( -2 - \delta, -2 + \delta \right)$.  Rather than just stating that $\delta$ is small, we need to be specific about its smallness compared to $\epsilon$, by specify $\delta$ in terms of $\epsilon$.

$H$ will stretch $\delta$ by a factor of $\answer{2}$, therefore,  we need $\delta \leq \frac{\epsilon}{2}$.   

Let's pick $\delta = \frac{\epsilon}{5}$. 




\[   H \left( \left( -2 - \frac{\epsilon}{5}, -2 + \frac{\epsilon}{5} \right) \right)  = \left( 2\left(-2 - \frac{\epsilon}{5}\right) - 1, 2\left(-2 + \frac{\epsilon}{5}\right) - 1  \right)\]


\[    = \left( -5 - \frac{2 \epsilon}{5} , -5 + \frac{2 \epsilon}{5} \right) \subset \left( -5 - \frac{2 \epsilon}{2} , -5 + \frac{2 \epsilon}{2} \right)  = (-5-\epsilon, -5+\epsilon) \]



\end{explanation}
\end{example}






















\end{document}
