\documentclass{ximera}


\graphicspath{
  {./}
  {ximeraTutorial/}
  {basicPhilosophy/}
}

\newcommand{\mooculus}{\textsf{\textbf{MOOC}\textnormal{\textsf{ULUS}}}}

\usepackage{tkz-euclide}\usepackage{tikz}
\usepackage{tikz-cd}
\usetikzlibrary{arrows}
\tikzset{>=stealth,commutative diagrams/.cd,
  arrow style=tikz,diagrams={>=stealth}} %% cool arrow head
\tikzset{shorten <>/.style={ shorten >=#1, shorten <=#1 } } %% allows shorter vectors

\usetikzlibrary{backgrounds} %% for boxes around graphs
\usetikzlibrary{shapes,positioning}  %% Clouds and stars
\usetikzlibrary{matrix} %% for matrix
\usepgfplotslibrary{polar} %% for polar plots
\usepgfplotslibrary{fillbetween} %% to shade area between curves in TikZ
\usetkzobj{all}
\usepackage[makeroom]{cancel} %% for strike outs
%\usepackage{mathtools} %% for pretty underbrace % Breaks Ximera
%\usepackage{multicol}
\usepackage{pgffor} %% required for integral for loops



%% http://tex.stackexchange.com/questions/66490/drawing-a-tikz-arc-specifying-the-center
%% Draws beach ball
\tikzset{pics/carc/.style args={#1:#2:#3}{code={\draw[pic actions] (#1:#3) arc(#1:#2:#3);}}}



\usepackage{array}
\setlength{\extrarowheight}{+.1cm}
\newdimen\digitwidth
\settowidth\digitwidth{9}
\def\divrule#1#2{
\noalign{\moveright#1\digitwidth
\vbox{\hrule width#2\digitwidth}}}






\DeclareMathOperator{\arccot}{arccot}
\DeclareMathOperator{\arcsec}{arcsec}
\DeclareMathOperator{\arccsc}{arccsc}

















%%This is to help with formatting on future title pages.
\newenvironment{sectionOutcomes}{}{}


\title{Function Values}

\begin{document}

\begin{abstract}
expected values
\end{abstract}
\maketitle




Functions are packages.  The contain three sets: domain, range, and a set of pairs.  Each pair in the set of pairs partners a domain number with a range number.  This range partner is called the \textbf{value of the function \textcolor{purple!85!blue}{at} the domain number}.


We have several tools to help us with the pairs. Our favorites are formulas and graphs. \\


$\blacktriangleright$ \textbf{A formula} gives us step-by-step instructions on how to calculate the range partner for a given domain number.   Formulas are exact tools. They allow us to calculate function values exactly. However, they are local tools.  We can use formulas to get function values, but one at a time.


$\blacktriangleright$ \textbf{A graph} is a visual tool that shows us a picture of all of the pairs.  It is a global tool. However, a graph is inherently inaccurate.  We can approximate, but we cannot get exact values.  


We use formulas and graphs for different purposes. They provide different views of a function.  On the other hand, they both essentially providing the same information, function values. And, that is what we want from them.


Our main activity in Precalculus and Calculus is examining function values.  We want to know the function values.  We want to compare some function values to other function values and compare those values to their corresponding domain numbers.  We have already seen that this examination leads to expectations.  Sometimes these expectations materialize and we are satisfied.  

Sometimes these expectations are surprisingly wrong.  The function value is not what we were expecting. All scientists like such surprises. Such events are ``interesting''. 

\begin{itemize}
\item What function value was I expecting?
\item Why was I expecting a particular value?
\item What went wrong?
\end{itemize} 


Of course, we would like to categorize these unfulfilled expectations and find language to describe them. We want to communicate exactly about expectations and their comparisons to actuality.




\subsection{Learning Outcomes}


\begin{sectionOutcomes}
In this section, students will 

\begin{itemize}
\item contemplate closeness.
\end{itemize}
\end{sectionOutcomes}

\end{document}
