\documentclass{ximera}


\graphicspath{
  {./}
  {ximeraTutorial/}
  {basicPhilosophy/}
}

\newcommand{\mooculus}{\textsf{\textbf{MOOC}\textnormal{\textsf{ULUS}}}}

\usepackage{tkz-euclide}\usepackage{tikz}
\usepackage{tikz-cd}
\usetikzlibrary{arrows}
\tikzset{>=stealth,commutative diagrams/.cd,
  arrow style=tikz,diagrams={>=stealth}} %% cool arrow head
\tikzset{shorten <>/.style={ shorten >=#1, shorten <=#1 } } %% allows shorter vectors

\usetikzlibrary{backgrounds} %% for boxes around graphs
\usetikzlibrary{shapes,positioning}  %% Clouds and stars
\usetikzlibrary{matrix} %% for matrix
\usepgfplotslibrary{polar} %% for polar plots
\usepgfplotslibrary{fillbetween} %% to shade area between curves in TikZ
\usetkzobj{all}
\usepackage[makeroom]{cancel} %% for strike outs
%\usepackage{mathtools} %% for pretty underbrace % Breaks Ximera
%\usepackage{multicol}
\usepackage{pgffor} %% required for integral for loops



%% http://tex.stackexchange.com/questions/66490/drawing-a-tikz-arc-specifying-the-center
%% Draws beach ball
\tikzset{pics/carc/.style args={#1:#2:#3}{code={\draw[pic actions] (#1:#3) arc(#1:#2:#3);}}}



\usepackage{array}
\setlength{\extrarowheight}{+.1cm}
\newdimen\digitwidth
\settowidth\digitwidth{9}
\def\divrule#1#2{
\noalign{\moveright#1\digitwidth
\vbox{\hrule width#2\digitwidth}}}






\DeclareMathOperator{\arccot}{arccot}
\DeclareMathOperator{\arcsec}{arcsec}
\DeclareMathOperator{\arccsc}{arccsc}

















%%This is to help with formatting on future title pages.
\newenvironment{sectionOutcomes}{}{}


\title{Describe Everything}

\begin{document}

\begin{abstract}
features
\end{abstract}
\maketitle





\begin{example}  Complete Anlaysis


Completely analyze   

\[   B(f) = \frac{(f+3)(f-1)}{(f+5)(f-2)^2}        \]


\textbf{\textcolor{purple!50!blue!90!black}{explanation}}



The domain is all real numbers except $-5$ and $2$, which make the denominator equal to $0$. These are singularities of $B$.  Otherwise, $B$ is continuous.  There are no discontinuities.

Domain = $(-\infty, -5) \cup (-5, 2) \cup (2, \infty)$

The graph will help us with the domain, so we'll wait on that.

$B$ has zeros at $-3$ and $1$.  These make the numerator equal to $0$. These are represented on the graph with $(-5,0)$ and $(2,0)$

The graph will have vertical asymptotes $f=-5$ and $f=-2$.  $B$ will change signs over $f=-5$, since the multiplicity is odd.  $B$ will not change signs over $f=2$, since the multiplicity is even.

$B$ is a rational function and the derree of the denominator is greater than the degree of the numerator. Therefore, 


\[ \lim_{f \to -\infty} B(f) = 0   \, \text{ and }  \,  \lim_{f \to \infty} B(f) = 0    \]









\begin{image}
\begin{tikzpicture}
  \begin{axis}[
            domain=-10:10, ymax=10, xmax=10, ymin=-10, xmin=-10,
            axis lines =center, xlabel=$f$, ylabel={$y=B(f)$}, grid = major, grid style={dashed},
            ytick={-10,-8,-6,-4,-2,2,4,6,8,10},
            xtick={-10,-8,-6,-4,-2,2,4,6,8,10},
            yticklabels={$-10$,$-8$,$-6$,$-4$,$-2$,$2$,$4$,$6$,$8$,$10$}, 
            xticklabels={$-10$,$-8$,$-6$,$-4$,$-2$,$2$,$4$,$6$,$8$,$10$},
            ticklabel style={font=\scriptsize},
            every axis y label/.style={at=(current axis.above origin),anchor=south},
            every axis x label/.style={at=(current axis.right of origin),anchor=west},
            axis on top
          ]
          

          \addplot [line width=1, gray, dashed, domain=(-9.5:9.5),<->] ({-5},{x});
          \addplot [line width=1, gray, dashed, domain=(-9.5:9.5),<->] ({2},{x});
          \addplot [line width=1, gray, dashed, domain=(-10:10)] {0};



          \addplot [line width=2, penColor, smooth,samples=200,domain=(-9:-5.03),<->] {((x+3)*(x-1))/((x+5)*(x-2)^2)};
          \addplot [line width=2, penColor, smooth,samples=200,domain=(-4.97:1.75),<->] {((x+3)*(x-1))/((x+5)*(x-2)^2)};
          \addplot [line width=2, penColor, smooth,samples=200,domain=(2.3:9),<->] {((x+3)*(x-1))/((x+5)*(x-2)^2)};

          \addplot[color=penColor,fill=penColor,only marks,mark=*] coordinates{(-3,0)};
          \addplot[color=penColor,fill=penColor,only marks,mark=*] coordinates{(1,0)};


           

  \end{axis}
\end{tikzpicture}
\end{image}


Both zeros have odd multipicity, therefore the graph must go below the $f$-axis between them.  There must be a local minimum. From the graph we can estimate that the critical number is approximately $0.5$. $B(0.5) = \frac{14}{99} \approx -0.1414$ is a local minimum. With some Calculus tools, we may be able to get an exact value.


$B$ has no global maximum or minimum.

$B$ has no local maximum.


\begin{itemize}
\item $B$ decreases on $(-\infty, -5)$.
\item $B$ decreases on $(-5, 0.5)$.
\item $B$ increases on $(0.5 2)$.
\item $B$ decreases on $(2, \infty)$.
\end{itemize}



The graph makes is evident thatthe range is all real numbers.


\end{example}








With some grap[hing tools, we can get a better approximatiopn of the critical number.




\begin{center}
\desmos{sumsbfeafr}{400}{300}
\end{center}


















\end{document}
