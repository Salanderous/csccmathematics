\documentclass{ximera}


\graphicspath{
  {./}
  {ximeraTutorial/}
  {basicPhilosophy/}
}

\newcommand{\mooculus}{\textsf{\textbf{MOOC}\textnormal{\textsf{ULUS}}}}

\usepackage{tkz-euclide}\usepackage{tikz}
\usepackage{tikz-cd}
\usetikzlibrary{arrows}
\tikzset{>=stealth,commutative diagrams/.cd,
  arrow style=tikz,diagrams={>=stealth}} %% cool arrow head
\tikzset{shorten <>/.style={ shorten >=#1, shorten <=#1 } } %% allows shorter vectors

\usetikzlibrary{backgrounds} %% for boxes around graphs
\usetikzlibrary{shapes,positioning}  %% Clouds and stars
\usetikzlibrary{matrix} %% for matrix
\usepgfplotslibrary{polar} %% for polar plots
\usepgfplotslibrary{fillbetween} %% to shade area between curves in TikZ
\usetkzobj{all}
\usepackage[makeroom]{cancel} %% for strike outs
%\usepackage{mathtools} %% for pretty underbrace % Breaks Ximera
%\usepackage{multicol}
\usepackage{pgffor} %% required for integral for loops



%% http://tex.stackexchange.com/questions/66490/drawing-a-tikz-arc-specifying-the-center
%% Draws beach ball
\tikzset{pics/carc/.style args={#1:#2:#3}{code={\draw[pic actions] (#1:#3) arc(#1:#2:#3);}}}



\usepackage{array}
\setlength{\extrarowheight}{+.1cm}
\newdimen\digitwidth
\settowidth\digitwidth{9}
\def\divrule#1#2{
\noalign{\moveright#1\digitwidth
\vbox{\hrule width#2\digitwidth}}}






\DeclareMathOperator{\arccot}{arccot}
\DeclareMathOperator{\arcsec}{arcsec}
\DeclareMathOperator{\arccsc}{arccsc}

















%%This is to help with formatting on future title pages.
\newenvironment{sectionOutcomes}{}{}


\title{Function Analysis}

\begin{document}

\begin{abstract}
%Stuff can go here later if we want!
\end{abstract}
\maketitle




\textbf{\textcolor{blue!55!black}{$\vartriangleright$}}  Analyzing a function means telling its story, describing all of its features and characteristics.  \\


\textbf{\textcolor{blue!55!black}{$\vartriangleright$}} Analyzing a function means telling your story, describing how you arrived at your conclusions. \\ 






A complete analysis includes exact information from algebra and global information from graphs.  The algebraic and graphical information should agree, which means each can help the other as you think through the analysis.\\




A complete analysis includes your reasoning and explanations of your thinking.\\


A complete analysis is communication.  The purpose is for the reader to understand your thoughts. \\





A complete analysis includes algebraic and functional reasoning about 


\begin{itemize}
\item \textbf{\textcolor{blue!55!black}{the domain and range,}} 
\item \textbf{\textcolor{blue!55!black}{zeros,}} 
\item \textbf{\textcolor{blue!55!black}{intervals of continuity,}} 
\item \textbf{\textcolor{blue!55!black}{discontinuities and singularities with limiting behavior,}} 
\item \textbf{\textcolor{blue!55!black}{end-behavior,}} 
\item \textbf{\textcolor{blue!55!black}{critical numbers,}} 
\item \textbf{\textcolor{blue!55!black}{intervals where the function is increasing and decreasing,}} 
\item \textbf{\textcolor{blue!55!black}{global maximum and minimums, and}} 
\item \textbf{\textcolor{blue!55!black}{local maximums and minimums.}} 
\end{itemize}




A complete analysis also includes a nice graph.  Not necessarily an exact graph, but rather a graph that effectively communicates the function's story. It should include intercepts, dashed asymptotes, closed and hollow dots, and arrows to help the reader understand the behavior of the function.  

\textbf{\textcolor{red!80!black}{$\blacktriangleright$}}  Remember, you are communicating to other people. Your analysis must be organized, legible, logical, and help the reader understand your thoughts.

The idea is not to be correct.  The idea is to explain to the reader how you know you are correct.

















\subsection{Learning Outcomes}





\begin{sectionOutcomes}
In this section, students will 

\begin{itemize}
\item completely analyze functions.
\item produce nice graphs.
\end{itemize}
\end{sectionOutcomes}
















\begin{center}
\textbf{\textcolor{green!50!black}{ooooo=-=-=-=-=-=-=-=-=-=-=-=-=ooOoo=-=-=-=-=-=-=-=-=-=-=-=-=ooooo}} \\

more examples can be found by following this link\\ \link[More Examples of Function Analysis]{https://ximera.osu.edu/csccmathematics/precalculus1/precalculus1/functionAnalysis/examples/exampleList}

\end{center}



\end{document}
