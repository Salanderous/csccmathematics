\documentclass{ximera}


\graphicspath{
  {./}
  {ximeraTutorial/}
  {basicPhilosophy/}
}

\newcommand{\mooculus}{\textsf{\textbf{MOOC}\textnormal{\textsf{ULUS}}}}

\usepackage{tkz-euclide}\usepackage{tikz}
\usepackage{tikz-cd}
\usetikzlibrary{arrows}
\tikzset{>=stealth,commutative diagrams/.cd,
  arrow style=tikz,diagrams={>=stealth}} %% cool arrow head
\tikzset{shorten <>/.style={ shorten >=#1, shorten <=#1 } } %% allows shorter vectors

\usetikzlibrary{backgrounds} %% for boxes around graphs
\usetikzlibrary{shapes,positioning}  %% Clouds and stars
\usetikzlibrary{matrix} %% for matrix
\usepgfplotslibrary{polar} %% for polar plots
\usepgfplotslibrary{fillbetween} %% to shade area between curves in TikZ
\usetkzobj{all}
\usepackage[makeroom]{cancel} %% for strike outs
%\usepackage{mathtools} %% for pretty underbrace % Breaks Ximera
%\usepackage{multicol}
\usepackage{pgffor} %% required for integral for loops



%% http://tex.stackexchange.com/questions/66490/drawing-a-tikz-arc-specifying-the-center
%% Draws beach ball
\tikzset{pics/carc/.style args={#1:#2:#3}{code={\draw[pic actions] (#1:#3) arc(#1:#2:#3);}}}



\usepackage{array}
\setlength{\extrarowheight}{+.1cm}
\newdimen\digitwidth
\settowidth\digitwidth{9}
\def\divrule#1#2{
\noalign{\moveright#1\digitwidth
\vbox{\hrule width#2\digitwidth}}}






\DeclareMathOperator{\arccot}{arccot}
\DeclareMathOperator{\arcsec}{arcsec}
\DeclareMathOperator{\arccsc}{arccsc}

















%%This is to help with formatting on future title pages.
\newenvironment{sectionOutcomes}{}{}


\title{Convergence Intervals}

\begin{document}

\begin{abstract}
center
\end{abstract}
\maketitle






Let's extend our bridge between infinite series and closed form formulas for Geometric series.





\section{Geometric Template}



Currently, we have an equivalence between a geometric series and a rational function.




\[   \sum_{n=0}^{\infty} a (f(x))^n =  \frac{a}{1-f(x)}   \, \text{ where } \, f(0) = 0       \]


This equivalence is valid while $|f(x)| < 1$.  This gives us a domain known as the convergence interval of the series.  The domain is centered around $0$. \\


Another way to view this is that the convergence interval is centered around the zero of $f(x)$, which has been $0$ up to this point.  What if a different domain number is a zero of $f$?


Suppose some number, $c$ is the zero of $f$.  Our interval of convergence would be centered at $c$ instead.














\begin{example} Geometric Series



Create a series equivalent to $g(x)=\frac{3}{7 - 2x}$ centered at $5$.



\begin{explanation}


First a quick analysis of $g(x)=\frac{3}{7 - 2x}$.  We can see that $\frac{7}{2}$ is a singularity. The domain of our infinite series representation of $g(x)$ cannot contain this singularity.  The domain will be centered at $5$ and come right up to $\frac{7}{2}$.  That is an interval radius of $\frac{3}{2}$. Therefore, the interval of convergence should be $\left( \frac{7}{2}, \frac{13}{2} \right)$.

Now, let's create the series representation.

We must rewrite $\frac{3}{7 - 2x}$ in the form $\frac{a}{1-something}$ where $something(5) = 0$



\[      \frac{3}{7 - 2x}  = 3   \cdot \frac{1}{7 - 2x} =    3   \cdot \frac{1}{7 - 2(x-5) - 10} =   3   \cdot \frac{1}{-3 - 2(x-5)}   =   \frac{3}{-3}   \cdot \frac{1}{1 + \frac{2(x-5)}{3}}  = \frac{-1}{1 + \frac{2(x-5)}{3}}  \]



Almost.  We have $1 + something$ in the denominator. It must be $1 - something$ to match our template.


\[    g(x) =   \frac{-1}{1 - \frac{-2(x-5)}{3}}       \]

\[  g(x) =    -1 \cdot \frac{1}{1 - \frac{-2(x-5)}{3}}     =   -   \sum_{n=0}^{\infty}   \left( \frac{-2(x-5)}{3} \right)^n     =   -  \sum_{n=0}^{\infty}   \left( \frac{-2}{3} \right)^n  (x-5)^n\]


\[  g(x)     =   -  \sum_{n=0}^{\infty}   (-1)^n \left( \frac{2}{3} \right)^n  (x-5)^n    \]

\[  g(x)     =    \sum_{n=0}^{\infty}   (-1)^{n+1} \left( \frac{2}{3} \right)^n  (x-5)^n    \]

The first few few terms look like


\[   -1 + \frac{2}{3} (x-5) - \left( \frac{2}{3} \right)^2 (x-5)^2 +  \left( \frac{2}{3} \right)^3 (x-5)^3 - \left( \frac{2}{3} \right)^4 (x-5)^4    + \cdots   \]


This is an example of an \textbf{alternating series}, because the signs of the terms alternate. \\

Now, for the radius.  We know the interval of convergence for a basic Geometric series is $(-1, 1)$.


Therefore, solve



\begin{align*}
\frac{2(x-5)}{3} & = 1   \\
2(x-5) & = 3    \\
x-5 & = \frac{3}{2}    \\
x & = \frac{3}{2}  + 5 = \frac{13}{2}
\end{align*}




\begin{align*}
\frac{2(x-5)}{3} & = -1   \\
2(x-5) & = -3    \\
x-5 & = -\frac{3}{2}    \\
x & = -\frac{3}{2}  + 5 = \frac{7}{2}
\end{align*}




The interval of convergence is $\left( \frac{7}{2}, \frac{13}{2}\right)$, as predicted.



\end{explanation}

\end{example}


We have two different formulas for the same function.






















\section{Polynomial Approximation}


We have two different formulas for the same function, at least on $\left( \frac{7}{2}, \frac{13}{2}\right)$.


Let's compare their graphs.

Of course, we cannot type in an infinite sum.  We'll approximate the infinite series with just the first few terms.




\begin{center}
\desmos{ltmxrqjnsz}{400}{300}
\end{center}


Just a fourth degree polynomial is doing a pretty good job approximating $g(x)$ on a subinterval of our domain.  More terms may do a better job.



















































\end{document}
