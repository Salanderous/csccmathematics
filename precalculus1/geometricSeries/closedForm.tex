\documentclass{ximera}


\graphicspath{
  {./}
  {ximeraTutorial/}
  {basicPhilosophy/}
}

\newcommand{\mooculus}{\textsf{\textbf{MOOC}\textnormal{\textsf{ULUS}}}}

\usepackage{tkz-euclide}\usepackage{tikz}
\usepackage{tikz-cd}
\usetikzlibrary{arrows}
\tikzset{>=stealth,commutative diagrams/.cd,
  arrow style=tikz,diagrams={>=stealth}} %% cool arrow head
\tikzset{shorten <>/.style={ shorten >=#1, shorten <=#1 } } %% allows shorter vectors

\usetikzlibrary{backgrounds} %% for boxes around graphs
\usetikzlibrary{shapes,positioning}  %% Clouds and stars
\usetikzlibrary{matrix} %% for matrix
\usepgfplotslibrary{polar} %% for polar plots
\usepgfplotslibrary{fillbetween} %% to shade area between curves in TikZ
\usetkzobj{all}
\usepackage[makeroom]{cancel} %% for strike outs
%\usepackage{mathtools} %% for pretty underbrace % Breaks Ximera
%\usepackage{multicol}
\usepackage{pgffor} %% required for integral for loops



%% http://tex.stackexchange.com/questions/66490/drawing-a-tikz-arc-specifying-the-center
%% Draws beach ball
\tikzset{pics/carc/.style args={#1:#2:#3}{code={\draw[pic actions] (#1:#3) arc(#1:#2:#3);}}}



\usepackage{array}
\setlength{\extrarowheight}{+.1cm}
\newdimen\digitwidth
\settowidth\digitwidth{9}
\def\divrule#1#2{
\noalign{\moveright#1\digitwidth
\vbox{\hrule width#2\digitwidth}}}






\DeclareMathOperator{\arccot}{arccot}
\DeclareMathOperator{\arcsec}{arcsec}
\DeclareMathOperator{\arccsc}{arccsc}

















%%This is to help with formatting on future title pages.
\newenvironment{sectionOutcomes}{}{}


\title{Closed Form}

\begin{document}

\begin{abstract}
center
\end{abstract}
\maketitle






Let's walk the bridge backwards



\section{Closed Form}



\begin{example} Gemetric Series


Given 
\[    K(t) =   \sum_{n=0}^{\infty}  \left( \frac{t}{2} \right)^2\]

find the closed form formula and the interval of convergence.




\textbf{\textcolor{purple!50!blue!90!black}{explanation}}


This is a Geometric series.  The interval of convergence is $(-2, 2)$, since this is when the inside of the general term is $-1$ and $1$.


The closed form is

\[  \frac{1}{1 - \frac{t}{2}}  =     \frac{2}{2-t}           \]






\end{example}















\begin{example} Gemetric Series


Given 
\[    K(t) =   \sum_{n=2}^{\infty}  \left( \frac{t}{2} \right)^2    \]

find the closed form formula and the interval of convergence.




\textbf{\textcolor{purple!50!blue!90!black}{explanation}}


This is the same Geometric series, except missing the first two terms.  The interval of convergence is $(-2, 2)$, since this is when the inside of the general term is $-1$ and $1$.


The closed form is the same, except subtract off the first two terms.

\[  \frac{1}{1 - \frac{t}{2}}  - 1 - \frac{t}{2}    =     \frac{2}{2-t}  - \frac{2-t}{2-t} - \frac{t}{2}     \]


need a common denominator


\[=     \frac{4}{2(2-t)}  - \frac{2(2-t)}{2(2-t)} - \frac{t(2-t)}{2(2-t)} = \frac{4-2(2-t)-t(2-t)}{2(2-t)}  = \frac{t^2}{2(2-t)}   \]





\[    K(t) =   \frac{t^2}{2(2-t)}    \] on $(-2, 2)$.




\end{example}




On the other hand, we could have factored out $\left( \frac{t}{2} \right)$ in order to get the index back to $0$.


\[    K(t) =   \sum_{n=2}^{\infty}  \left( \frac{t}{2} \right)^n    =  \left( \frac{t}{2} \right)^2 \sum_{n=0}^{\infty}  \left( \frac{t}{2} \right)^n   \]


Then the closed form would be


\[    \left( \frac{t}{2} \right)^2  \cdot \frac{1}{1 - \frac{t}{2}}            \]



\[     \frac{t^2}{2(2 - t)}            \]



Same formula.


















\end{document}
