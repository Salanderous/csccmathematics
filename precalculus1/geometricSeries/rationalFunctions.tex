\documentclass{ximera}


\graphicspath{
  {./}
  {ximeraTutorial/}
  {basicPhilosophy/}
}

\newcommand{\mooculus}{\textsf{\textbf{MOOC}\textnormal{\textsf{ULUS}}}}

\usepackage{tkz-euclide}\usepackage{tikz}
\usepackage{tikz-cd}
\usetikzlibrary{arrows}
\tikzset{>=stealth,commutative diagrams/.cd,
  arrow style=tikz,diagrams={>=stealth}} %% cool arrow head
\tikzset{shorten <>/.style={ shorten >=#1, shorten <=#1 } } %% allows shorter vectors

\usetikzlibrary{backgrounds} %% for boxes around graphs
\usetikzlibrary{shapes,positioning}  %% Clouds and stars
\usetikzlibrary{matrix} %% for matrix
\usepgfplotslibrary{polar} %% for polar plots
\usepgfplotslibrary{fillbetween} %% to shade area between curves in TikZ
\usetkzobj{all}
\usepackage[makeroom]{cancel} %% for strike outs
%\usepackage{mathtools} %% for pretty underbrace % Breaks Ximera
%\usepackage{multicol}
\usepackage{pgffor} %% required for integral for loops



%% http://tex.stackexchange.com/questions/66490/drawing-a-tikz-arc-specifying-the-center
%% Draws beach ball
\tikzset{pics/carc/.style args={#1:#2:#3}{code={\draw[pic actions] (#1:#3) arc(#1:#2:#3);}}}



\usepackage{array}
\setlength{\extrarowheight}{+.1cm}
\newdimen\digitwidth
\settowidth\digitwidth{9}
\def\divrule#1#2{
\noalign{\moveright#1\digitwidth
\vbox{\hrule width#2\digitwidth}}}






\DeclareMathOperator{\arccot}{arccot}
\DeclareMathOperator{\arcsec}{arcsec}
\DeclareMathOperator{\arccsc}{arccsc}

















%%This is to help with formatting on future title pages.
\newenvironment{sectionOutcomes}{}{}


\title{Rational Functions}

\begin{document}

\begin{abstract}
series
\end{abstract}
\maketitle




Can we get a series for any rational function?



\[   R(x) = \frac{5-x}{(x+3)(x-2)}      \]

Can we get a series for this centered at $1$?.












\begin{image}
\begin{tikzpicture}
  \begin{axis}[
            domain=-10:10, ymax=10, xmax=10, ymin=-10, xmin=-10,
            axis lines =center, xlabel=$x$, ylabel={$y=R(x)$}, grid = major, grid style={dashed},
            ytick={-10,-8,-6,-4,-2,2,4,6,8,10},
            xtick={-10,-8,-6,-4,-2,2,4,6,8,10},
            yticklabels={$-10$,$-8$,$-6$,$-4$,$-2$,$2$,$4$,$6$,$8$,$10$}, 
            xticklabels={$-10$,$-8$,$-6$,$-4$,$-2$,$2$,$4$,$6$,$8$,$10$},
            ticklabel style={font=\scriptsize},
            every axis y label/.style={at=(current axis.above origin),anchor=south},
            every axis x label/.style={at=(current axis.right of origin),anchor=west},
            axis on top
          ]


			\addplot [line width=1, gray, dashed, domain=(-9.5:9.5),<->] ({-3},{x});
			\addplot [line width=1, gray, dashed, domain=(-9.5:9.5),<->] ({2},{x});
          

            \addplot [line width=2, penColor, smooth,samples=300,domain=(-9:-3.2),<->] {(5-x)/((x+3)*(x-2))};
            \addplot [line width=2, penColor, smooth,samples=300,domain=(-2.8:1.9),<->] {(5-x)/((x+3)*(x-2))};
            \addplot [line width=2, penColor, smooth,samples=300,domain=(2.1:9),<->] {(5-x)/((x+3)*(x-2))};

           

  \end{axis}
\end{tikzpicture}
\end{image}


If we can get a series centered at $1$, then the interval of convergence could only go out to $2$.  That's a radius of $1$.  The interval of convergence would be $(0,2)$.





Our first hurdle is that we know how to get series for expressions that look like $\frac{1}{1-something}$.  Therefore, we need to split the formula for $R(x)$ to look like

\[    \frac{A}{x+3} + \frac{B}{x-2} + stuff          \]



If we separate out the parts like $\frac{A}{x+3}$ and $\frac{B}{x-2}$, then no singularities should be left.  That suggests that the remaining stuff is a polynomial.


We can work backwards. We'll start by just combining the two fraction templates and see how close we get.





\[  \frac{A}{x+3} + \frac{B}{x-2}  = \frac{A(x-2) + B(x+3)}{(x+3)(x-2)}  = \frac{(A+B)x + (3B-2A)}{(x+3)(x-2)}       \]


This will work provide

\begin{itemize}
\item $A+B = -1$
\item $3B-2A = 5$
\end{itemize}

The first equation tells us that $A = -1 - B$.  Substituting that into the second equation gives us 


\begin{align*}
3B-2A     & = 5     \\
3B-2(-1 -B)     & = 5     \\
2 + 5B     & = 5     \\
5B     & = 3     \\
B & = \frac{3}{5}
\end{align*}


$A = -1 - \frac{3}{5} = \frac{-8}{5}$



\[ \frac{5-x}{(x+3)(x-2)}     =  \frac{-8}{5} \cdot \frac{1}{x+3} + \frac{3}{5} \cdot \frac{1}{x-2}       \]



Now, we need to rewrite the fractions in the form $\frac{1}{1 - something}$ where   $something(1) = 0$.






\[    \frac{1}{x+3}   = \frac{1}{3+x}  = \frac{1}{3+(x-1)+1}     = \frac{1}{4+(x-1)}     = \frac{1}{4-(-(x-1))}      =   \frac{1}{4} \cdot \frac{1}{1-\left( \frac{-(x-1)}{4} \right) }   \]



\[    \frac{1}{x-2} =  \frac{1}{-2 + (x-1)+1}    =  \frac{1}{-1 + (x-1)}   =  \frac{-1}{1 - (x-1)}  \]



We can substitute these into the formula.






\[ \frac{5-x}{(x+3)(x-2)}     =  \frac{-8}{5} \cdot \frac{1}{4} \cdot \frac{1}{1+\left( \frac{x-1}{4} \right) } + \frac{3}{5} \cdot \frac{-1}{1 - (x-1)}      \]



And, our series



\[ \frac{5-x}{(x+3)(x-2)}     =  \frac{-8}{5} \cdot \frac{1}{4} \cdot     \sum_{n=0}^{\infty} \left( \frac{-(x-1)}{4} \right)^n                 + \frac{3}{5} \cdot (-1) \cdot \sum_{n=0}^{\infty} (x-1)^n     \]






\[ \frac{5-x}{(x+3)(x-2)}     =  \frac{-2}{5}     \sum_{n=0}^{\infty} \left( \frac{-(x-1)}{4} \right)^n                 + \frac{-3}{5} \sum_{n=0}^{\infty} (x-1)^n     \]












\[ \frac{5-x}{(x+3)(x-2)}     =      \sum_{n=0}^{\infty} \frac{-2}{5}  (-1)^n \left( \frac{(x-1)}{4} \right)^n                 +  \sum_{n=0}^{\infty} \frac{-3}{5} (x-1)^n     \]







\[ \frac{5-x}{(x+3)(x-2)}     =      \sum_{n=0}^{\infty} \left( \frac{-2}{5}  (-1)^n \left( \frac{(x-1)}{4} \right)^n                 +  \frac{-3}{5} (x-1)^n  \right)   \]










Not a Geometric series, but we do have a series. Instead, a sum of two geometric series.





\begin{center}
\desmos{jsohctdybf}{400}{300}
\end{center}



\end{document}
