\documentclass{ximera}


\graphicspath{
  {./}
  {ximeraTutorial/}
  {basicPhilosophy/}
}

\newcommand{\mooculus}{\textsf{\textbf{MOOC}\textnormal{\textsf{ULUS}}}}

\usepackage{tkz-euclide}\usepackage{tikz}
\usepackage{tikz-cd}
\usetikzlibrary{arrows}
\tikzset{>=stealth,commutative diagrams/.cd,
  arrow style=tikz,diagrams={>=stealth}} %% cool arrow head
\tikzset{shorten <>/.style={ shorten >=#1, shorten <=#1 } } %% allows shorter vectors

\usetikzlibrary{backgrounds} %% for boxes around graphs
\usetikzlibrary{shapes,positioning}  %% Clouds and stars
\usetikzlibrary{matrix} %% for matrix
\usepgfplotslibrary{polar} %% for polar plots
\usepgfplotslibrary{fillbetween} %% to shade area between curves in TikZ
\usetkzobj{all}
\usepackage[makeroom]{cancel} %% for strike outs
%\usepackage{mathtools} %% for pretty underbrace % Breaks Ximera
%\usepackage{multicol}
\usepackage{pgffor} %% required for integral for loops



%% http://tex.stackexchange.com/questions/66490/drawing-a-tikz-arc-specifying-the-center
%% Draws beach ball
\tikzset{pics/carc/.style args={#1:#2:#3}{code={\draw[pic actions] (#1:#3) arc(#1:#2:#3);}}}



\usepackage{array}
\setlength{\extrarowheight}{+.1cm}
\newdimen\digitwidth
\settowidth\digitwidth{9}
\def\divrule#1#2{
\noalign{\moveright#1\digitwidth
\vbox{\hrule width#2\digitwidth}}}






\DeclareMathOperator{\arccot}{arccot}
\DeclareMathOperator{\arcsec}{arcsec}
\DeclareMathOperator{\arccsc}{arccsc}

















%%This is to help with formatting on future title pages.
\newenvironment{sectionOutcomes}{}{}


\title{The Series}

\begin{document}

\begin{abstract}
formula and domain
\end{abstract}
\maketitle




We have seen that $\sum_{n=0}^{\infty} a x^n = \frac{a}{1-x}$ for $x \in (-1, 1)$ and $a$ a real number.


This is a new kind of function formula called an infinite series.  This type of infinite series is called a \textbf{geometric series} and $(-1, 1)$ is its interval of convergence.  From a function point-of-view $(-1, 1)$  is the domain of this function.


On the other hand, this same function can be represented by the formula $\frac{a}{1-x}$.  The implied domain of $\frac{a}{1-x}$ is $(-\infty, 1) \cup (1, \infty)$, but the domain must be restricted to be equivalent to $\sum_{n=0}^{\infty} a x^n$.




$\blacktriangleright$ A \textbf{closed form} formula is one like  $\frac{a}{1-x}$.  It involves a finite number of our basic numeric operations.  This is different than our new types of formulas.  Infinite series list an infinite number of basic operations.  We prefer closed form formulas, unfortunately they just don't represent enough functions. Functions describe via infinite series widen our library of functions, considerably.

But, for right now, we are just investigating the bridge between closed form formulas and infinite series.






\section{Geometric Template}

The identity function, $Id(x) = x$, is just a function such that $I(0) = 0$.  We could describe $\frac{a}{1-x}$ for $x \in (-1, 1)$ as $\frac{a}{1-Id(x)}$ for $Id(x) \in (-1, 1)$ - a composition.

Or, we could expand this composition idea and think of other functions, $f$, such that $f(0) = 0$.



\[    \frac{a}{1-f(x)}   \, \text{ where } \, f(0) = 0       \]


This would be a valid function as long as $f(x) \ne 1$. 




We could then move this composition over to our Geometric series.




\[   \sum_{n=0}^{\infty} a (f(x))^n =  \frac{a}{1-f(x)}   \, \text{ where } \, f(0) = 0       \]


This would be valid up until $f(b)=1$ for some $b$.  Since $0$ is the center of the interval of convergence, we could calculate a radius of convergence and establish and interval of convergence - or a common domain where the series and the closed form formula represent the same function.












\begin{example} Geometric Series



Create a series equivalent to $g(x)=\frac{3}{7 - 2x}$ and find the interval of convergence.



\textbf{\textcolor{purple!50!blue!90!black}{explanation}}






We must rewrite $frac{3}{7 - 2x}$ in the form $\frac{a}{1-something} = a \cdot \frac{1}{1-something}$ such that $something(0) = 0$




\[     \frac{3}{7 - 2x}  =   \frac{3}{7} \cdot \frac{1}{1 - \frac{2x}{7}}   \]



\[  \frac{3}{7} \cdot \frac{1}{1 - \frac{2x}{7}} =  \frac{3}{7} \sum_{n=0}^{\infty} \left(\frac{2x}{7}\right)^n       \]





Now, we need the interval of convergence.  The interval is centered at $0$ and cannot go beyond the singularity of $g(x)$, which is $\frac{7}{2}$.  Since $0$ is the center, we have $\left(-\frac{7}{2}, \frac{7}{2} \right)$








\end{example}












































\end{document}
