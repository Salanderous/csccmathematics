\documentclass{ximera}


\graphicspath{
  {./}
  {ximeraTutorial/}
  {basicPhilosophy/}
}

\newcommand{\mooculus}{\textsf{\textbf{MOOC}\textnormal{\textsf{ULUS}}}}

\usepackage{tkz-euclide}\usepackage{tikz}
\usepackage{tikz-cd}
\usetikzlibrary{arrows}
\tikzset{>=stealth,commutative diagrams/.cd,
  arrow style=tikz,diagrams={>=stealth}} %% cool arrow head
\tikzset{shorten <>/.style={ shorten >=#1, shorten <=#1 } } %% allows shorter vectors

\usetikzlibrary{backgrounds} %% for boxes around graphs
\usetikzlibrary{shapes,positioning}  %% Clouds and stars
\usetikzlibrary{matrix} %% for matrix
\usepgfplotslibrary{polar} %% for polar plots
\usepgfplotslibrary{fillbetween} %% to shade area between curves in TikZ
\usetkzobj{all}
\usepackage[makeroom]{cancel} %% for strike outs
%\usepackage{mathtools} %% for pretty underbrace % Breaks Ximera
%\usepackage{multicol}
\usepackage{pgffor} %% required for integral for loops



%% http://tex.stackexchange.com/questions/66490/drawing-a-tikz-arc-specifying-the-center
%% Draws beach ball
\tikzset{pics/carc/.style args={#1:#2:#3}{code={\draw[pic actions] (#1:#3) arc(#1:#2:#3);}}}



\usepackage{array}
\setlength{\extrarowheight}{+.1cm}
\newdimen\digitwidth
\settowidth\digitwidth{9}
\def\divrule#1#2{
\noalign{\moveright#1\digitwidth
\vbox{\hrule width#2\digitwidth}}}






\DeclareMathOperator{\arccot}{arccot}
\DeclareMathOperator{\arcsec}{arcsec}
\DeclareMathOperator{\arccsc}{arccsc}

















%%This is to help with formatting on future title pages.
\newenvironment{sectionOutcomes}{}{}


\title{The Series}

\begin{document}

\begin{abstract}
formula and domain
\end{abstract}
\maketitle




We have seen that $\sum\limits_{n=0}^{\infty} a \, x^n = \frac{a}{1-x}$ for $x \in (-1, 1)$ and $a \in \mathbb{R}$. \\




This is a new kind of function formula called an \textbf{\textcolor{purple!85!blue}{infinite series}}.  This particular infinite series above is called a \textbf{\textcolor{purple!85!blue}{geometric series}} and $(-1, 1)$ is its \textbf{\textcolor{purple!85!blue}{interval of convergence}}.  From a function point-of-view, $(-1, 1)$  is the domain of this function.


On the other hand, this same function can be represented by the formula $\frac{a}{1-x}$.  The implied domain of $\frac{a}{1-x}$ is $(-\infty, 1) \cup (1, \infty)$. \\ 


If we restrict the domain then we have equivalent representations of the same function 


\[
\sum\limits_{n=0}^{\infty} a \, x^n   =    \frac{a}{1-x}    \, \text{ on } (-1, 1)
\]



$\blacktriangleright$ $\frac{a}{1-x}$ is called a \textbf{closed form} formula.  It involves a finite number of our basic numeric operations.  This is different than our new type of formula.  Infinite series list an infinite number of basic operations.  We prefer closed form formulas, unfortunately they just don't represent enough functions. Functions described via infinite series widen our library of functions, considerably.

But, for right now, we are just investigating the bridge between closed form formulas and infinite series using geometric series to illustrate.





\begin{example}


Let $f(x) = \sum\limits_{n=0}^{\infty} 4 \, x^n$. \\ 



Then, 


\[ f\left( \frac{1}{2} \right) = \sum\limits_{n=0}^{\infty} 4 \, \left( \frac {1}{2} \right)^n = \frac{4}{1-\frac{1}{2}} = 8
\]

\end{example}






\begin{example}


Let $G(x) = \sum\limits_{n=3}^{\infty} 4 \, x^n$. \\ 



Then, 
\[ G\left( \frac{1}{2} \right) = \sum\limits_{n=3}^{\infty} 4 \, \left( \frac {1}{2} \right)^n \]


\[ = \sum\limits_{n=0}^{\infty} 4 \, \left( \frac {1}{2} \right)^n  - 4 - 4 \left( \frac{1}{2} \right)  - 4 \left( \frac{1}{2} \right)^2    \]



\[
= \frac{4}{1-\frac{1}{2}} - 4 - 2 - 1 = 1
\]


\end{example}








\begin{example}


Let $G(x) = \sum\limits_{n=3}^{\infty} 4 \, x^n$. \\ 



Then, 
\[ G\left( \frac{1}{2} \right) = \sum\limits_{n=3}^{\infty} 4 \, \left( \frac {1}{2} \right)^n \]


\[ = \left( \frac{1}{2} \right)^3 \sum\limits_{n=0}^{\infty} 4 \, \left( \frac {1}{2} \right)^n \]

\[ =  \frac{1}{8} \cdot  \frac{4}{1-\frac{1}{2}} = 1 \]

\end{example}












\section{Geometric Template}


We have $\sum\limits_{n=0}^{\infty} a \, x^n = \frac{a}{1-x}$ for $x \in (-1, 1)$ \\

Of course, $x$ is actually a formula for a function, the identity function, $Id(x) = x$. \\


We could describe $\frac{a}{1-x}$ for $x \in (-1, 1)$ as $\frac{a}{1-Id(x)}$ for $Id(x) \in (-1, 1)$.



 A composition! \\




Or, we could expand this composition idea and think of other functions, $f$, with $f(0) = 0$, like $Id(x)$.



\[    \frac{a}{1-f(x)}   \, \text{ where } \, f(0) = 0       \]


This would be a valid function as long as $f(x) \ne 1$. 




We could then move this composition over to our Geometric series.




\[   \sum_{n=0}^{\infty} a (f(x))^n =  \frac{a}{1-f(x)}   \, \text{ as long as } \, f(0) = 0       \]


This would be valid up until $f(b)=-1$ for some $b$ or $f(b)=1$ for some $b$.  Since $0$ is the center of the interval of convergence, we could calculate a radius of convergence and establish an interval of convergence - or a common domain where the series and the closed form formula represent the same function.












\begin{example} Geometric Series as Rational Functions



Create an infinite series equivalent to $g(x)=\frac{3}{7 - 2x}$ and find the interval of convergence.



\begin{explanation}


We must rewrite $\frac{3}{7 - 2x}$ in the form $\frac{a}{1-something} = a \cdot \frac{1}{1-something}$ such that $something(0) = 0$




\[     \frac{3}{7 - 2x}  =   \frac{3}{7} \cdot \frac{1}{1 - \frac{2x}{7}}   \]



\[  \frac{3}{7} \cdot \frac{1}{1 - \frac{2x}{7}} =  \frac{3}{7} \sum\limits_{n=0}^{\infty} \left(\frac{2x}{7}\right)^n       \]





Now, we need the interval of convergence, a.k.a. the domain.  


$\blacktriangleright$ One view:  This geometric series converges provided   $\left| \frac{2x}{7} \right| < 1$.  The would be $\left(-\frac{7}{2}, \frac{7}{2} \right)$.



$\blacktriangleright$ Another view: The interval is centered at $0$ and cannot go beyond the singularity of $g(x)$, which is $\frac{7}{2}$.  Since $0$ is the center, we have $\left(-\frac{7}{2}, \frac{7}{2} \right)$


\end{explanation}





\end{example}






















\begin{example} Geometric Series



Create a series equivalent to $h(x) = \frac{2}{5 + 3x}$ and find the interval of convergence.



\begin{explanation}


We must rewrite $\frac{2}{5 + 3x}$ in the form $\frac{a}{1-something} = a \cdot \frac{1}{1-something}$ such that $something(0) = 0$




\[     \frac{2}{5 + 3x}  =   \frac{2}{5} \cdot \frac{1}{1 + \frac{3x}{5}}  =   \frac{2}{5} \cdot \frac{1}{1 - \frac{-3x}{5}}  \]



\[  \frac{2}{5} \cdot \frac{1}{1 - \frac{-3x}{5}}=  \frac{2}{5} \sum\limits_{n=0}^{\infty} \left(\frac{-3x}{5}\right)^n       \]





Now, we need the interval of convergence, a.k.a. the domain.  


$\blacktriangleright$ One view:  This geometric series converges provided   $\left| \frac{-3x}{5} \right| < 1$.  The would be $\left(-\frac{5}{3}, \frac{5}{3} \right)$.



$\blacktriangleright$ Another view: The interval is centered at $0$ and cannot go beyond the singularity of $h(x)$, which is $\frac{-5}{3}$.  Since $0$ is the center, we have $\left(-\frac{5}{3}, \frac{5}{3} \right)$


\end{explanation}


Sometimes people will separate out the $-1$ inside the series.

\[  \frac{2}{5} \sum\limits_{n=0}^{\infty} \left(\frac{-3x}{5}\right)^n    =   \frac{2}{5} \sum\limits_{n=0}^{\infty} (-1)^n \left(\frac{3x}{5}\right)^n   \]


Sometimes that is easier to read.


\end{example}






















\begin{center}
\textbf{\textcolor{green!50!black}{ooooo=-=-=-=-=-=-=-=-=-=-=-=-=ooOoo=-=-=-=-=-=-=-=-=-=-=-=-=ooooo}} \\

more examples can be found by following this link\\ \link[More Examples of Geometric Series]{https://ximera.osu.edu/csccmathematics/precalculus1/precalculus1/geometricSeries/examples/exampleList}

\end{center}












\end{document}
