\documentclass{ximera}


\graphicspath{
  {./}
  {ximeraTutorial/}
  {basicPhilosophy/}
}

\newcommand{\mooculus}{\textsf{\textbf{MOOC}\textnormal{\textsf{ULUS}}}}

\usepackage{tkz-euclide}\usepackage{tikz}
\usepackage{tikz-cd}
\usetikzlibrary{arrows}
\tikzset{>=stealth,commutative diagrams/.cd,
  arrow style=tikz,diagrams={>=stealth}} %% cool arrow head
\tikzset{shorten <>/.style={ shorten >=#1, shorten <=#1 } } %% allows shorter vectors

\usetikzlibrary{backgrounds} %% for boxes around graphs
\usetikzlibrary{shapes,positioning}  %% Clouds and stars
\usetikzlibrary{matrix} %% for matrix
\usepgfplotslibrary{polar} %% for polar plots
\usepgfplotslibrary{fillbetween} %% to shade area between curves in TikZ
\usetkzobj{all}
\usepackage[makeroom]{cancel} %% for strike outs
%\usepackage{mathtools} %% for pretty underbrace % Breaks Ximera
%\usepackage{multicol}
\usepackage{pgffor} %% required for integral for loops



%% http://tex.stackexchange.com/questions/66490/drawing-a-tikz-arc-specifying-the-center
%% Draws beach ball
\tikzset{pics/carc/.style args={#1:#2:#3}{code={\draw[pic actions] (#1:#3) arc(#1:#2:#3);}}}



\usepackage{array}
\setlength{\extrarowheight}{+.1cm}
\newdimen\digitwidth
\settowidth\digitwidth{9}
\def\divrule#1#2{
\noalign{\moveright#1\digitwidth
\vbox{\hrule width#2\digitwidth}}}






\DeclareMathOperator{\arccot}{arccot}
\DeclareMathOperator{\arcsec}{arcsec}
\DeclareMathOperator{\arccsc}{arccsc}

















%%This is to help with formatting on future title pages.
\newenvironment{sectionOutcomes}{}{}


\title{Space}

\begin{document}

\begin{abstract}
surrounding values
\end{abstract}
\maketitle




While examining a function and its values, we see patterns and trends and naturally extend these in our minds to arrive at expected values. We anticipate what might happen.  It doesn't always happen, which catches our eye.

Currently, we are examining when a function suddenly takes on a value which is a drastic change from all of the function values around it.  Or, when a function just drops a value at a single domain number.

We have expectations based on the surrounding function values and then these expectations are not met.

In order to arrive at these expectations, we need some function values for domain numbers ``around'' the single domain number under examination. \\


$\blacktriangleright$ We need some space around our domain number. \\




We need space to gather up function values from which we are drawing our expectations.  We need this space to provide a cushion around the domain number where we can see what the function is doing.


We need this space to be there no matter how close our examination is.  


Our close space must work for all levels of inspection.















\subsection{Learning Outcomes}


\begin{sectionOutcomes}
In this section, students will 

\begin{itemize}
\item contemplate space resulting from open intervals.
\end{itemize}
\end{sectionOutcomes}

\end{document}
