\documentclass{ximera}


\graphicspath{
  {./}
  {ximeraTutorial/}
  {basicPhilosophy/}
}

\newcommand{\mooculus}{\textsf{\textbf{MOOC}\textnormal{\textsf{ULUS}}}}

\usepackage{tkz-euclide}\usepackage{tikz}
\usepackage{tikz-cd}
\usetikzlibrary{arrows}
\tikzset{>=stealth,commutative diagrams/.cd,
  arrow style=tikz,diagrams={>=stealth}} %% cool arrow head
\tikzset{shorten <>/.style={ shorten >=#1, shorten <=#1 } } %% allows shorter vectors

\usetikzlibrary{backgrounds} %% for boxes around graphs
\usetikzlibrary{shapes,positioning}  %% Clouds and stars
\usetikzlibrary{matrix} %% for matrix
\usepgfplotslibrary{polar} %% for polar plots
\usepgfplotslibrary{fillbetween} %% to shade area between curves in TikZ
\usetkzobj{all}
\usepackage[makeroom]{cancel} %% for strike outs
%\usepackage{mathtools} %% for pretty underbrace % Breaks Ximera
%\usepackage{multicol}
\usepackage{pgffor} %% required for integral for loops



%% http://tex.stackexchange.com/questions/66490/drawing-a-tikz-arc-specifying-the-center
%% Draws beach ball
\tikzset{pics/carc/.style args={#1:#2:#3}{code={\draw[pic actions] (#1:#3) arc(#1:#2:#3);}}}



\usepackage{array}
\setlength{\extrarowheight}{+.1cm}
\newdimen\digitwidth
\settowidth\digitwidth{9}
\def\divrule#1#2{
\noalign{\moveright#1\digitwidth
\vbox{\hrule width#2\digitwidth}}}






\DeclareMathOperator{\arccot}{arccot}
\DeclareMathOperator{\arcsec}{arcsec}
\DeclareMathOperator{\arccsc}{arccsc}

















%%This is to help with formatting on future title pages.
\newenvironment{sectionOutcomes}{}{}


\title{Parameterize}

\begin{document}

\begin{abstract}
%Stuff can go here later if we want!
\end{abstract}
\maketitle





Functions are packages of three sets: a domain, a range, and a set of pairs, together with one rule - every domain item is in exactly one pair.

\begin{quote}
Representing functions is a different story.
\end{quote}



Functions vary so much that one method of representing them isn't enough.  We have listing tools.  We have algebraic tools for exactness.  We have graphical tools for a global perspective. Graphs of functions span the spectrum from smooth and contunuous to extremely chopped up.

The graphs of functions belong to a larger category called curves.

We have introduced algebraic curves, which are a picture of solutions to a polynomial equation.

As we work with Cartesian equations, we are learning that they are difficult to handle.

In this section, we introduce a different way of describing curves and functions called \textbf{parameterization}



$\blacktriangleright$ \textbf{Parameterization} \\


We are used to an independent and dependent variable.  We often call the dependent variable the function and call the independent variable the variable.

But this doesn't work for curves (and implicitly defined functions) as the vertical line test has shown us.  Curves often work better when both coordinates are functions of a thrid variable, often called a parameter.

One value is independently selected, like a variable, and it is used to calculate both coordinates.

























\begin{sectionOutcomes}
In this section, students will 

\begin{itemize}
\item analyze parameterizations.
\end{itemize}
\end{sectionOutcomes}

\end{document}
