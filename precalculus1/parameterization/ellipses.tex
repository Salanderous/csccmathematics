\documentclass{ximera}


\graphicspath{
  {./}
  {ximeraTutorial/}
  {basicPhilosophy/}
}

\newcommand{\mooculus}{\textsf{\textbf{MOOC}\textnormal{\textsf{ULUS}}}}

\usepackage{tkz-euclide}\usepackage{tikz}
\usepackage{tikz-cd}
\usetikzlibrary{arrows}
\tikzset{>=stealth,commutative diagrams/.cd,
  arrow style=tikz,diagrams={>=stealth}} %% cool arrow head
\tikzset{shorten <>/.style={ shorten >=#1, shorten <=#1 } } %% allows shorter vectors

\usetikzlibrary{backgrounds} %% for boxes around graphs
\usetikzlibrary{shapes,positioning}  %% Clouds and stars
\usetikzlibrary{matrix} %% for matrix
\usepgfplotslibrary{polar} %% for polar plots
\usepgfplotslibrary{fillbetween} %% to shade area between curves in TikZ
\usetkzobj{all}
\usepackage[makeroom]{cancel} %% for strike outs
%\usepackage{mathtools} %% for pretty underbrace % Breaks Ximera
%\usepackage{multicol}
\usepackage{pgffor} %% required for integral for loops



%% http://tex.stackexchange.com/questions/66490/drawing-a-tikz-arc-specifying-the-center
%% Draws beach ball
\tikzset{pics/carc/.style args={#1:#2:#3}{code={\draw[pic actions] (#1:#3) arc(#1:#2:#3);}}}



\usepackage{array}
\setlength{\extrarowheight}{+.1cm}
\newdimen\digitwidth
\settowidth\digitwidth{9}
\def\divrule#1#2{
\noalign{\moveright#1\digitwidth
\vbox{\hrule width#2\digitwidth}}}






\DeclareMathOperator{\arccot}{arccot}
\DeclareMathOperator{\arcsec}{arcsec}
\DeclareMathOperator{\arccsc}{arccsc}

















%%This is to help with formatting on future title pages.
\newenvironment{sectionOutcomes}{}{}


\title{Along Ellipses}

\begin{document}

\begin{abstract}
movement
\end{abstract}
\maketitle



Ellipses, like circles, are easy to parameterize.  So, they show up in lots of examples.



A circle of radius $r$, centered at $(x_c, y_c)$ can be described by $(x-a)^2+(y-b)^2=r^2$.  This can be written as 



\[          \left(\frac{x-x_c}{r}\right)^2+\left(\frac{y-y_c}{r}\right)^2 = 1 \]

Ellipses are sort of circles with two radii - one for each direction. Let $a$ be the radius in the $x$-direction and $b$ be the radius in the $y$-direction.  Then the equation for an ellipses looks like 



\[          \left(\frac{x-x_c}{a}\right)^2+\left(\frac{y-y_c}{b}\right)^2 = 1 \]







The parameterization for the circle was

\begin{itemize}
\item $x(t) = r cos(t) + x_c$
\item $y(t) = r sin(t) + y_c$
\end{itemize}









The parameterization for the ellipse just gives each formula its own radius.

\begin{itemize}
\item $x(t) = a cos(t) + x_c$
\item $y(t) = b sin(t) + y_c$
\end{itemize}







\begin{center}
\desmos{crezl9grlt}{400}{300}
\end{center}




The dot keeps going around the ellipse, because $t$ runs through the whole real line.


What is we made the angle $t$ run back and forth from $0$ to $\pi$?



$t(\theta) = \frac{\pi}{2} cos(\theta) + \frac{\pi}{2}$





\begin{image}
\begin{tikzpicture}
  \begin{axis}[
            domain=-10:10, ymax=4, xmax=10, ymin=-1, xmin=-10,
            axis lines =center, xlabel=$\theta$, ylabel=$t$, grid = major, grid style={dashed},
            ytick={-1,1,2,3,4},
            xtick={-10,-8,-6,-4,-2,2,4,6,8,10},
            yticklabels={$-1$,$1$,$2$,$3$,$4$}, 
            xticklabels={$-10$,$-8$,$-6$,$-4$,$-2$,$2$,$4$,$6$,$8$,$10$},
            ticklabel style={font=\scriptsize},
            every axis y label/.style={at=(current axis.above origin),anchor=south},
            every axis x label/.style={at=(current axis.right of origin),anchor=west},
            axis on top
          ]
          
          %\addplot [line width=2, penColor2, smooth,samples=100,domain=(-6:2)] {-2*x-3};
            \addplot [line width=2, penColor2, smooth,samples=100,domain=(-10:10)] {1.57*cos(deg(x))+1.57};




           

  \end{axis}
\end{tikzpicture}
\end{image}



We'll feed this into the paraemterization for the ellipse.





\begin{itemize}
\item $x(\theta) = a cos\left(\frac{\pi}{2} cos(\theta) + \frac{\pi}{2}\right) + x_c$
\item $y(\theta) = b sin\left(\frac{\pi}{2} cos(\theta) + \frac{\pi}{2}\right) + y_c$
\end{itemize}







\begin{center}
\desmos{gqaw1oljjq}{400}{300}
\end{center}










\end{document}
