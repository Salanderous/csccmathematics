\documentclass{ximera}


\graphicspath{
  {./}
  {ximeraTutorial/}
  {basicPhilosophy/}
}

\newcommand{\mooculus}{\textsf{\textbf{MOOC}\textnormal{\textsf{ULUS}}}}

\usepackage{tkz-euclide}\usepackage{tikz}
\usepackage{tikz-cd}
\usetikzlibrary{arrows}
\tikzset{>=stealth,commutative diagrams/.cd,
  arrow style=tikz,diagrams={>=stealth}} %% cool arrow head
\tikzset{shorten <>/.style={ shorten >=#1, shorten <=#1 } } %% allows shorter vectors

\usetikzlibrary{backgrounds} %% for boxes around graphs
\usetikzlibrary{shapes,positioning}  %% Clouds and stars
\usetikzlibrary{matrix} %% for matrix
\usepgfplotslibrary{polar} %% for polar plots
\usepgfplotslibrary{fillbetween} %% to shade area between curves in TikZ
\usetkzobj{all}
\usepackage[makeroom]{cancel} %% for strike outs
%\usepackage{mathtools} %% for pretty underbrace % Breaks Ximera
%\usepackage{multicol}
\usepackage{pgffor} %% required for integral for loops



%% http://tex.stackexchange.com/questions/66490/drawing-a-tikz-arc-specifying-the-center
%% Draws beach ball
\tikzset{pics/carc/.style args={#1:#2:#3}{code={\draw[pic actions] (#1:#3) arc(#1:#2:#3);}}}



\usepackage{array}
\setlength{\extrarowheight}{+.1cm}
\newdimen\digitwidth
\settowidth\digitwidth{9}
\def\divrule#1#2{
\noalign{\moveright#1\digitwidth
\vbox{\hrule width#2\digitwidth}}}






\DeclareMathOperator{\arccot}{arccot}
\DeclareMathOperator{\arcsec}{arcsec}
\DeclareMathOperator{\arccsc}{arccsc}

















%%This is to help with formatting on future title pages.
\newenvironment{sectionOutcomes}{}{}



\author{Lee Wayand}

\begin{document}
\begin{example}








\begin{image}
\begin{tikzpicture} 
  \begin{axis}[
            domain=-1.5:1.5, ymax=1.5, xmax=1.5, ymin=-1.5, xmin=-1.5,
            axis lines =center, unit vector ratio*=1 1 1, xlabel=$x$, ylabel=$y$,
            ticklabel style={font=\scriptsize},
            every axis y label/.style={at=(current axis.above origin),anchor=south},
            every axis x label/.style={at=(current axis.right of origin),anchor=west},
            axis on top
          ]
          
          	\addplot [line width=2, penColor, smooth,samples=100,domain=(0:6.3)] ({cos(deg(x))},{sin(deg(x)});


        	\addplot [color=penColor2,only marks,mark=*] coordinates{(0.5,0.866)};


        	%\draw[decoration={brace,raise=.2cm,mirror},decorate,thin] (axis cs:0.75,0)--(axis cs:0.75,0.866);
        	%\draw[decoration={brace,raise=.2cm},decorate,thin] (axis cs:0,0.95)--(axis cs:0.75,0.95);
        	%\node[anchor=east] at (axis cs:1.85,3) {$d$};
        	%\node[anchor=east] at (axis cs:4.6,1) {$f(d)$};
                     
        	\node at (axis cs:0.75,1.2) [penColor2] {$(\cos(\theta),\sin(\theta))$};


          \addplot [line width=1, penColor2, smooth,samples=100,domain=(0:0.5)] ({x},{1.7320*x});
          \node at (axis cs:0.25,0.15) [penColor2] {$\theta$};


          %\addplot [color=penColor,only marks,mark=*] coordinates{(-4,-1) (0,1) (1,-6.5) (7,-3.5)};

          %\addplot [line width=1, penColor2, smooth,samples=100,domain=(-4:0)] ({x},{0});
          %\addplot [line width=1, penColor2, smooth,samples=100,domain=(1:7)] ({x},{0});
           

  \end{axis}
\end{tikzpicture}
\end{image}




\begin{question}


As you travel on the unit circle counterclockwise from $0$ radians to $\frac{\pi}{2}$ radians, you move up and to the left. \\


On the interval $\left (0, \frac{\pi}{2} \right)$, $\sin(\theta)$ is \wordChoice{\choice[correct]{increasing} \choice{decreasing}}. \\

\end{question}







\begin{question}


As you travel on the unit circle counterclockwise from $\frac{\pi}{2}$ radians to $\pi$ radians, you move down and to the left. \\


On the interval $\left (0\frac{\pi}{2}, \pi \right)$, $\sin(\theta)$ is \wordChoice{\choice{increasing} \choice[correct]{decreasing}}. \\

\end{question}







\begin{question}


As you travel on the unit circle counterclockwise from $\pi$ radians to $\frac{3\pi}{2}$ radians, you move down and to the right. \\


On the interval $\left (\pi, \frac{3\pi}{2} \right)$, $\sin(\theta)$ is \wordChoice{\choice{increasing} \choice[correct]{decreasing}}. \\

\end{question}








\begin{question}


As you travel on the unit circle counterclockwise from $\frac{3\pi}{2}$ radians to $2\pi$ radians, you move up and to the right. \\


On the interval $\left (\frac{3\pi}{2}, 2\pi \right)$, $\sin(\theta)$ is \wordChoice{\choice[correct]{increasing} \choice{decreasing}}. \\

\end{question}




















\end{example}
\end{document}