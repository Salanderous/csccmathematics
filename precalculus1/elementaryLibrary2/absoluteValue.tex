\documentclass{ximera}


\graphicspath{
  {./}
  {ximeraTutorial/}
  {basicPhilosophy/}
}

\newcommand{\mooculus}{\textsf{\textbf{MOOC}\textnormal{\textsf{ULUS}}}}

\usepackage{tkz-euclide}\usepackage{tikz}
\usepackage{tikz-cd}
\usetikzlibrary{arrows}
\tikzset{>=stealth,commutative diagrams/.cd,
  arrow style=tikz,diagrams={>=stealth}} %% cool arrow head
\tikzset{shorten <>/.style={ shorten >=#1, shorten <=#1 } } %% allows shorter vectors

\usetikzlibrary{backgrounds} %% for boxes around graphs
\usetikzlibrary{shapes,positioning}  %% Clouds and stars
\usetikzlibrary{matrix} %% for matrix
\usepgfplotslibrary{polar} %% for polar plots
\usepgfplotslibrary{fillbetween} %% to shade area between curves in TikZ
\usetkzobj{all}
\usepackage[makeroom]{cancel} %% for strike outs
%\usepackage{mathtools} %% for pretty underbrace % Breaks Ximera
%\usepackage{multicol}
\usepackage{pgffor} %% required for integral for loops



%% http://tex.stackexchange.com/questions/66490/drawing-a-tikz-arc-specifying-the-center
%% Draws beach ball
\tikzset{pics/carc/.style args={#1:#2:#3}{code={\draw[pic actions] (#1:#3) arc(#1:#2:#3);}}}



\usepackage{array}
\setlength{\extrarowheight}{+.1cm}
\newdimen\digitwidth
\settowidth\digitwidth{9}
\def\divrule#1#2{
\noalign{\moveright#1\digitwidth
\vbox{\hrule width#2\digitwidth}}}






\DeclareMathOperator{\arccot}{arccot}
\DeclareMathOperator{\arcsec}{arcsec}
\DeclareMathOperator{\arccsc}{arccsc}

















%%This is to help with formatting on future title pages.
\newenvironment{sectionOutcomes}{}{}


\title{Absolute Value}

\begin{document}

\begin{abstract}
distance
\end{abstract}
\maketitle



The absolute value function, $|x|$, gives the distance on the number line between a number, $x$, and $0$.  Since distance cannot be negative, it appears that the function just ``makes numbers positive''.


To do this, the absolute value function just returns nonnegative numbers unharmed, and makes negative numbers turn positive.  Algebraically, this is accomplished through a piecewise defined function.






\[
|x| = 
\begin{cases}
  -x &\text{if $x<0$,}\\
  x & \text{if $x\ge 0$}.
\end{cases}
\]


Technically speaking, $|-3| = -(-3)$ and then $-(-3) = 3$.  The absolute value function negates negative numbers.


\textbf{Note:} Just because you see a negaitve sign doesn't mean you have a negative number.



\begin{example}
\begin{itemize}
\item $|4| = 4$
\item $|0| = 0$
\item $|-\pi| = -(-\pi) = \pi$
\item $|cos(\pi)| = |-1| = -(-1) = 1$
\item $|sin(\tfrac{\pi}{4})| = \tfrac{1}{\sqrt{2}}$
\end{itemize}
\end{example}





\begin{example}
\begin{itemize}
\item $|-\sqrt{5}| = \answer{\sqrt{5}}$
\item $|4-4| = \answer{0}$
\item $\left|\frac{-4}{-3}\right| = \answer{\frac{4}{3}}$
\item $|cos(\tfrac{\pi}{2})| = \answer{0}$
\item $|sin(\tfrac{3\pi}{2})| = \answer{1}$
\item $|tan(\tfrac{3\pi}{4})| = \answer{1}$
\end{itemize}
\end{example}
There is no arithmetic operation called ``make positive''.  If a number is negative, then you make it positive by negating it. 


Negating is arithmetic.
``Make positive'' is not arithmetic.




Graph of $y = A(t) = |t|$.

\begin{image}
\begin{tikzpicture}
  \begin{axis}[
            domain=-10:10, ymax=10, xmax=10, ymin=-10, xmin=-10,
            axis lines =center, xlabel=$t$, ylabel=$y$, grid = major,
            ytick={-10,-8,-6,-4,-2,2,4,6,8,10},
            xtick={-10,-8,-6,-4,-2,2,4,6,8,10},
            ticklabel style={font=\scriptsize},
            every axis y label/.style={at=(current axis.above origin),anchor=south},
            every axis x label/.style={at=(current axis.right of origin),anchor=west},
            axis on top
          ]
          

          \addplot [line width=2, penColor, smooth, domain=(-9:9), <->] {abs(x)};
   


           

  \end{axis}
\end{tikzpicture}
\end{image}







\begin{example}

Solve $|m| = 18$.


Either $m = 18$  or $m = -18$

\end{example}




























\begin{center}
\textbf{\textcolor{green!50!black}{ooooo=-=-=-=-=-=-=-=-=-=-=-=-=ooOoo=-=-=-=-=-=-=-=-=-=-=-=-=ooooo}} \\

more examples can be found by following this link\\ \link[More Examples of Elementary Functions]{https://ximera.osu.edu/csccmathematics/precalculus1/precalculus1/elementaryLibrary2/examples/exampleList}

\end{center}







\end{document}
