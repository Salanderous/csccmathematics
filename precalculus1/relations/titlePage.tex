\documentclass{ximera}


\graphicspath{
  {./}
  {ximeraTutorial/}
  {basicPhilosophy/}
}

\newcommand{\mooculus}{\textsf{\textbf{MOOC}\textnormal{\textsf{ULUS}}}}

\usepackage{tkz-euclide}\usepackage{tikz}
\usepackage{tikz-cd}
\usetikzlibrary{arrows}
\tikzset{>=stealth,commutative diagrams/.cd,
  arrow style=tikz,diagrams={>=stealth}} %% cool arrow head
\tikzset{shorten <>/.style={ shorten >=#1, shorten <=#1 } } %% allows shorter vectors

\usetikzlibrary{backgrounds} %% for boxes around graphs
\usetikzlibrary{shapes,positioning}  %% Clouds and stars
\usetikzlibrary{matrix} %% for matrix
\usepgfplotslibrary{polar} %% for polar plots
\usepgfplotslibrary{fillbetween} %% to shade area between curves in TikZ
\usetkzobj{all}
\usepackage[makeroom]{cancel} %% for strike outs
%\usepackage{mathtools} %% for pretty underbrace % Breaks Ximera
%\usepackage{multicol}
\usepackage{pgffor} %% required for integral for loops



%% http://tex.stackexchange.com/questions/66490/drawing-a-tikz-arc-specifying-the-center
%% Draws beach ball
\tikzset{pics/carc/.style args={#1:#2:#3}{code={\draw[pic actions] (#1:#3) arc(#1:#2:#3);}}}



\usepackage{array}
\setlength{\extrarowheight}{+.1cm}
\newdimen\digitwidth
\settowidth\digitwidth{9}
\def\divrule#1#2{
\noalign{\moveright#1\digitwidth
\vbox{\hrule width#2\digitwidth}}}






\DeclareMathOperator{\arccot}{arccot}
\DeclareMathOperator{\arcsec}{arcsec}
\DeclareMathOperator{\arccsc}{arccsc}

















%%This is to help with formatting on future title pages.
\newenvironment{sectionOutcomes}{}{}


\title{Relations}

\begin{document}

\begin{abstract}

\end{abstract}
\maketitle



\section{Anatomy of a Question}


What is the underlying assumption of a question?


\begin{observation} Connecting Information

\begin{itemize} 
\item What ingredients are needed to make chili?
\item Who was in House of Wax (1953)?
\item What is the speed limit on Route 3?
\item What is Mary's social security number?
\item What ice cream flavors does Kevin like?
\item What is $32^\circ F$ in Celsius?
\item Could you provide a list of good mystery books and their authors?
\item Did you pass Chemistry?
\item How high is Mt. Everest?
\item Who are Timothy's neices?
\item What are the Post Office's hours?
\end{itemize}

$\blacktriangleright$We understand our world through connections between things. \\


We might view this as characteristics or attributes of something. 
\begin{itemize}
\item Mt Everest is $29,029 ft$ tall.
\end{itemize}


We might experience this as an if-then statement.
\begin{itemize}
\item If you are driving on Route 3, then do not go over 35 mph.
\end{itemize}



We might experience this as cause and effect.
\begin{itemize}
\item The Post Office is closed, because it is 9:00pm.
\end{itemize}

\end{observation}


All of our questions assume the same structure.  \\

Questions assume there are two sets or collections of things and that the items in each set are connected or associated with one another.



\begin{example} Movies and Actors \\
There is a set or collection of movies and a set or collection of actors.  A movie is associated with an actor if the actor appeared in the movie.
\end{example}

\begin{example} SSN \\
There is a set of 9-digit numbers and a collection of U.S. citizens.  A number is connected to a person if the person was issued the number as their social secutiry number.
\end{example}

\begin{example} Hours of Operation \\
There is a collection of Post Offices and a collection of times.  A time is connected to a Post Office if the Post Office is open for buiness at that time.
\end{example}




Questions express our curiosities into the structure of how collections of things are connected. We call this linking structure of bonds a \textbf{relation}. \\


This section explores relations and how we communicate about them. \\










\subsection{Expectations}

\begin{sectionOutcomes}
In this section, students will 

\begin{itemize}
\item view a relation as a package.
\item encode relation pairings as written ordered pairs.
\item decipher written ordered pairs into relation pairings.
\item trace connections back-and-forth between sets.
\end{itemize}
\end{sectionOutcomes}

\end{document}
