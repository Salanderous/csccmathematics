\documentclass{ximera}


\graphicspath{
  {./}
  {ximeraTutorial/}
  {basicPhilosophy/}
}

\newcommand{\mooculus}{\textsf{\textbf{MOOC}\textnormal{\textsf{ULUS}}}}

\usepackage{tkz-euclide}\usepackage{tikz}
\usepackage{tikz-cd}
\usetikzlibrary{arrows}
\tikzset{>=stealth,commutative diagrams/.cd,
  arrow style=tikz,diagrams={>=stealth}} %% cool arrow head
\tikzset{shorten <>/.style={ shorten >=#1, shorten <=#1 } } %% allows shorter vectors

\usetikzlibrary{backgrounds} %% for boxes around graphs
\usetikzlibrary{shapes,positioning}  %% Clouds and stars
\usetikzlibrary{matrix} %% for matrix
\usepgfplotslibrary{polar} %% for polar plots
\usepgfplotslibrary{fillbetween} %% to shade area between curves in TikZ
\usetkzobj{all}
\usepackage[makeroom]{cancel} %% for strike outs
%\usepackage{mathtools} %% for pretty underbrace % Breaks Ximera
%\usepackage{multicol}
\usepackage{pgffor} %% required for integral for loops



%% http://tex.stackexchange.com/questions/66490/drawing-a-tikz-arc-specifying-the-center
%% Draws beach ball
\tikzset{pics/carc/.style args={#1:#2:#3}{code={\draw[pic actions] (#1:#3) arc(#1:#2:#3);}}}



\usepackage{array}
\setlength{\extrarowheight}{+.1cm}
\newdimen\digitwidth
\settowidth\digitwidth{9}
\def\divrule#1#2{
\noalign{\moveright#1\digitwidth
\vbox{\hrule width#2\digitwidth}}}






\DeclareMathOperator{\arccot}{arccot}
\DeclareMathOperator{\arcsec}{arcsec}
\DeclareMathOperator{\arccsc}{arccsc}

















%%This is to help with formatting on future title pages.
\newenvironment{sectionOutcomes}{}{}


\title{Relations}

\begin{document}

\begin{abstract}
Information Packages
\end{abstract}
\maketitle


We view the world through relationships. That is, some information is connected to other information for some reason. These associations are how we navigate through our lives.

\begin{itemize}
\item Dishes have ingredients.
\item Movies have actors.
\item Roads have speed limits.
\item People have social security numbers.
\item Kindergarteners like ice cream flavors.
\item Measurements are quoted in different units.
\item Authors write books.
\item Students get course grades.
\item Mountains have heights.
\item Family's have trees.
\item Businesses have open hours.
\end{itemize}


All of these relationships are different and yet the same.  Some relationships can be viewed as objects possessing characteristics. Some relationships can be viewed if-then statements.  Some relationships can be viewed as cause and effect. But, they can all be viewed as two collections with associated elements.
\\






\begin{definition} \textbf{relation} 
A relation is a package containing three sets or collections. One set is called the \textbf{domain}. One set is called the \textbf{codomain}.  Finally, there is a set of pairings.  Each pairing pairs a member of the domain with a memmber of the codomain.
\end{definition}

Relations are just about the vaguest, thinnest structure any two sets could have. There are two sets and then some of their members are associated to one another. In fact, this structure is so thin that we cannot do much with it.  But, we don't want to.  We are just laying the groundwork for more complex structures.  Right now, we just want to invent some language and notation, so that we can talk about these types of structures. We have a start already. The two sets in a relation have the names domain and codomain.
\\

Because a relation might encountered as an if-then statement or a cause-and-effect association, we have a natural direction feeling for the information.  The "if" part comes before the "then" part.  The "cause" comes before the "effect".  We would like this feeling reflected in our relation structure.  Therefore, relations come prepackaged with a feeling the information is connected from the domain to the codomain.
\begin{itemize}
\item If you pick items from the domain, then you get items from the codomain.
\item Items from the domain cause items from the codomain.
\item Items in the codomain occur because of items in the domain.
\item Elements of the codomain happen at items of the domain.
\end{itemize}













\end{document}
