\documentclass{ximera}


\graphicspath{
  {./}
  {ximeraTutorial/}
  {basicPhilosophy/}
}

\newcommand{\mooculus}{\textsf{\textbf{MOOC}\textnormal{\textsf{ULUS}}}}

\usepackage{tkz-euclide}\usepackage{tikz}
\usepackage{tikz-cd}
\usetikzlibrary{arrows}
\tikzset{>=stealth,commutative diagrams/.cd,
  arrow style=tikz,diagrams={>=stealth}} %% cool arrow head
\tikzset{shorten <>/.style={ shorten >=#1, shorten <=#1 } } %% allows shorter vectors

\usetikzlibrary{backgrounds} %% for boxes around graphs
\usetikzlibrary{shapes,positioning}  %% Clouds and stars
\usetikzlibrary{matrix} %% for matrix
\usepgfplotslibrary{polar} %% for polar plots
\usepgfplotslibrary{fillbetween} %% to shade area between curves in TikZ
\usetkzobj{all}
\usepackage[makeroom]{cancel} %% for strike outs
%\usepackage{mathtools} %% for pretty underbrace % Breaks Ximera
%\usepackage{multicol}
\usepackage{pgffor} %% required for integral for loops



%% http://tex.stackexchange.com/questions/66490/drawing-a-tikz-arc-specifying-the-center
%% Draws beach ball
\tikzset{pics/carc/.style args={#1:#2:#3}{code={\draw[pic actions] (#1:#3) arc(#1:#2:#3);}}}



\usepackage{array}
\setlength{\extrarowheight}{+.1cm}
\newdimen\digitwidth
\settowidth\digitwidth{9}
\def\divrule#1#2{
\noalign{\moveright#1\digitwidth
\vbox{\hrule width#2\digitwidth}}}






\DeclareMathOperator{\arccot}{arccot}
\DeclareMathOperator{\arcsec}{arcsec}
\DeclareMathOperator{\arccsc}{arccsc}

















%%This is to help with formatting on future title pages.
\newenvironment{sectionOutcomes}{}{}


\title{A Package}

\begin{document}

\begin{abstract}
information packages
\end{abstract}
\maketitle


We view the world through relationships. That is, some information is connected to other information for some reason. These associations are how we navigate through our lives.

\begin{itemize}
\item Dishes have ingredients.
\item Movies have actors.
\item Roads have speed limits.
\item People have social security numbers.
\item Kindergarteners like ice cream flavors.
\item Measurements are quoted in different units.
\item Authors write books.
\item Students get course grades.
\item Mountains have heights.
\item Family's have trees.
\item Businesses have open hours.
\end{itemize}


All of these relationships are different and yet the same.  Some relationships can be viewed as objects possessing characteristics. Some relationships can be viewed as if-then statements.  Some relationships can be viewed as cause and effect. But, they can all be viewed as two collections with associated elements.







\begin{definition} \textbf{relation} \\
A relation is a package containing three sets or collections. One set is called the \textbf{domain}. One set is called the \textbf{codomain}.  Finally, there is a set of pairings.  Each pairing pairs a member of the domain with a memmber of the codomain.
\end{definition}


Relations are just about the vaguest, thinnest structure any two sets could possibly have. There are two sets and then some of their members are associated to one another. In fact, this structure is so thin that we cannot do much with it.  Fortunately, we don't want to do much with relations.  We are just laying the groundwork for more complex structures.  Right now, we just want to invent some language and notation, so that we can talk about these types of structures. We have a start already. The two sets in a relation have the names \textit{domain} and \textit{codomain}.


Because a relation might be encountered as an if-then statement or a cause-and-effect association, we have a natural directional feeling for the information.  The "if" part comes before the "then" part.  The "cause" comes before the "effect".  We would like this feeling reflected in our relation structure.  Therefore, relations come prepackaged with a feeling that the information is connected from the domain to the codomain.

\begin{itemize}
\item If you pick items from the domain, then you get items from the codomain.
\item Items from the domain cause items from the codomain.
\item Items in the codomain occur because of items in the domain.
\item Elements of the codomain happen at items of the domain.
\end{itemize}



\section{List Representations}

Most of our communication is going to be written in this course, so we need some agreements on how we will represent relations.  We already have several ways of representing sets.  The easiest is to just list the members inside curly braces.

\begin{center} 
domain = \{ Casablanca, House of Wax,  The Godfather, Lawrence of Arabia, Toy Story, The Fly \} 
\end{center}

\begin{center} 
codomain = \{ Marlon Brando, Vincent Price, Humphrey Bogart, Peter O'Toole, Harrison Ford, Al Pacino \} 
\end{center}

Now we need a way to indicate the pairings.  The traditonal way is to write them as ordered pairs.  The left (or first) item coming from the domain and the right (or second) item from the codomain.  We can list these ordered pairs in a set of ordered pairs. 

\begin{center} 
pairs = \{ (The Godfather, Marlon Brando), (House of Wax, Vincent Price), (Casablanca, Humphrey Bogart), (Lawrence of Arabia, Peter O'Toole), (The Fly, Vincent Price), (The Godfather, Al Pacino) \} 
\end{center}


These three sets would make up a relation. 

Of course, we will have many relations, which could get confusing.  Let's help ourselves out with names.  The name of this relation will be \textit{AppearedIn}.


\begin{example} \textit{AppearedIn} \\
\begin{itemize}
\item domain = \{ Casablanca, House of Wax,  The Godfather, Lawrence of Arabia, Toy Story, The Fly \}  
\item codomain = \{ Marlon Brando, Vincent Price, Humphrey Bogart, Peter O'Toole, Harrison Ford, Al Pacino  \} 
\item pairs = \{ (The Godfather, Marlon Brando), (House of Wax, Vincent Price), (Casablanca, Humphrey Bogart), (Lawrence of Arabia, Peter O'Toole), (The Fly, Vincent Price), (The Godfather, Al Pacino) \} 
\end{itemize}
\end{example}


A relation is a package.  It is a package of three sets. The domain and codomain are sets of information.  The third set is a set of pairs.  Each pair connects a member of the domain with a memeber of the codomain. 


\begin{template} Ordered Pairs \\
The template for writing an ordered pair looks like  

\[
\large{( domain \, item, codomain \, item )}
\]
\end{template}

There is a domain item written on the left and a codmain item written to the right.  They are separated with a comma.  All of htat is wrapped in parentheses.


\begin{warning} Did You Notice? \\
\begin{itemize}
\item Nobody said every member of the domain had to actually appear in a pair.  Toy Story is in the domain and in no pair of \textit{AppearedIn}.
\item Nobody said every member of the codomain had to actually appear in a pair.  Harrison Ford is in the codomain and in no pair of \textit{AppearedIn}.
\item Nobody said domain members could not appear in multiple pairs.  The Godfather appears in two pairs of \textit{AppearedIn}.
\item Nobody said codomain members could not appear in multiple pairs.  Vincent Price appears in two pairs of \textit{AppearedIn}.
\end{itemize}
\end{warning}




\section{Table Representations}



Lists are good representations of relations when the sets are not very big. The parentheses become difficult to browse through when there are a lot of them. Another representation is via a table. A table visually organizes the pairs much better.

\begin{example} \textit{KindergartenIceCream} \\
The \textit{KindergartenIceCream} relation pairs kindergarteners with their favorite ice cream flavors.

domain = \{ Kevin, Shay, Linda, Charmain, Charlie \}  \\
codomain = \{ Vanilla, Chocolate, Strawberry, Peach, Mango, Cherry \} 

\[
\begin{array}{l|l}
    domain      & codomain      \\ \hline
    Kevin   &  Chocolate \\
    Shay   & Strawberry \\
    Linda  &  Chocolate \\
    Linda  &  Peach \\
    Charmain &  Vanilla \\ 
\end{array}
\]


Each line of the table shows a pairing. From this table, we can tell that the relation \textit{KindergartenIceCream} pairs Kevin with Chocolate.  (Kevin, Chocolate) is a pair in \textit{KindergartenIceCream}.

\textbf{Note} Charlie is not in a pair.  Charlie doesn't like ice cream. Linda is in two pairs.  She has two favorite flavors.  Nobody has Cherry as their favorite flavor.

\end{example} 





\section{Questions}


We have created some mathematical structure for questions. From this structure, we can see that there are basically two kinds of questions.

\begin{itemize}
\item [Type 1] You know the domain item and want the corresponding codomain items.
\item [Type 2] You know the codomain item and want the corresponding domain items.
\end{itemize}



\begin{example} \textit{KindergartenIceCream} \\
The \textit{KindergartenIceCream} relation pairs kindergarteners with their favorite ice cream flavors.

domain = \{ Kevin, Shay, Linda, Charmain, Charlie \}  \\
codomain = \{ Vanilla, Chocolate, Strawberry, Peach, Mango, Cherry \} 

\[
\begin{array}{l|l}
    domain      & codomain      \\ \hline
    Kevin   &  Chocolate \\
    Shay   & Strawberry \\
    Linda  &  Chocolate \\
    Linda  &  Peach \\
    Charmain &  Vanilla \\ 
\end{array}
\]


Question Type 1 : What flavors does Linda like? \\
We are looking for pairs of the form (Linda, ???).

Question Type 2 : Who likes Chocolate? \\
We are looking for pairs of the form (???, Chocolate).

\end{example} 



\section{Too Much}

Our examples, so far, have been small.  But what are we going to do when we want to examine the relation between something like movies and actors?  


\begin{example} \textit{ActorsInMovies} \\
The \textit{ActorsInMovies} relation pairs actors with movies they were in.

domain = All actors \\
codomain = All movies

The table would begin like this.

\[
\begin{array}{l|l}
    domain      & codomain      \\ \hline
    \text{Vincent Price}   &  \text{House of Wax} \\
    \text{Humphrey Bogart}   & \text{Casablanca} \\
    \text{Peter O'Toole}  &  \text{Lawrence of Arabia} \\
    \text{Vincent Price}  &  \text{The Fly} \\
    \text{Al Pacino} &  \text{The Godfather} \\ 
\end{array}
\]

\end{example} 


This table would have many rows. The Internet Movie Database lists over a million movies.  There would be more than 100 rows just for Vincent Price. There is no way we could visually sift through such a table - just to talk about Vincent Price movies.

We need to narrow our investigation here, quickly. Otherwise, we will be buried in a mountain of data.

Our plan is to investigate only certain types of relations.








\section{Focusing Our Investigation}


Our examples, so far, have been small.  But what are we going to do when we want to examine the relation between movies and actors?  That list or table is going to be too big to look at with our eyes.  How would we look through a table with millions of rows and find the ones holding Vincent Price?  The topic of all relations is just too big of an investigation. Let's focus in on a particular type of relation.

Most of our questions really identify a single hypothesis (antecedent) and then want to know about the resulting conclusion (consequent).

This would translate into a domain member connected to a single codomain member.

Let's focus our investigation to these types of relations.  These types of relations are called \textbf{functions}.

















\end{document}
