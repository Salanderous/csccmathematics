\documentclass{ximera}


\graphicspath{
  {./}
  {ximeraTutorial/}
  {basicPhilosophy/}
}

\newcommand{\mooculus}{\textsf{\textbf{MOOC}\textnormal{\textsf{ULUS}}}}

\usepackage{tkz-euclide}\usepackage{tikz}
\usepackage{tikz-cd}
\usetikzlibrary{arrows}
\tikzset{>=stealth,commutative diagrams/.cd,
  arrow style=tikz,diagrams={>=stealth}} %% cool arrow head
\tikzset{shorten <>/.style={ shorten >=#1, shorten <=#1 } } %% allows shorter vectors

\usetikzlibrary{backgrounds} %% for boxes around graphs
\usetikzlibrary{shapes,positioning}  %% Clouds and stars
\usetikzlibrary{matrix} %% for matrix
\usepgfplotslibrary{polar} %% for polar plots
\usepgfplotslibrary{fillbetween} %% to shade area between curves in TikZ
\usetkzobj{all}
\usepackage[makeroom]{cancel} %% for strike outs
%\usepackage{mathtools} %% for pretty underbrace % Breaks Ximera
%\usepackage{multicol}
\usepackage{pgffor} %% required for integral for loops



%% http://tex.stackexchange.com/questions/66490/drawing-a-tikz-arc-specifying-the-center
%% Draws beach ball
\tikzset{pics/carc/.style args={#1:#2:#3}{code={\draw[pic actions] (#1:#3) arc(#1:#2:#3);}}}



\usepackage{array}
\setlength{\extrarowheight}{+.1cm}
\newdimen\digitwidth
\settowidth\digitwidth{9}
\def\divrule#1#2{
\noalign{\moveright#1\digitwidth
\vbox{\hrule width#2\digitwidth}}}






\DeclareMathOperator{\arccot}{arccot}
\DeclareMathOperator{\arcsec}{arcsec}
\DeclareMathOperator{\arccsc}{arccsc}

















%%This is to help with formatting on future title pages.
\newenvironment{sectionOutcomes}{}{}


\title{A Package}

\begin{document}

\begin{abstract}
information packages
\end{abstract}
\maketitle


We view the world through relationships. That is, some information is connected or related to other information for some reason. These associations are how we navigate through our lives.

\begin{itemize}
\item Dishes include ingredients.
\item Actors appear in movies.
\item Roads post speed limits.
\item People are issued social security numbers.
\item Kindergarteners like ice cream flavors.
\item Measurements are quoted in different units.
\item Authors write books.
\item Students earn course grades.
\item Mountains have heights.
\item Families are traced through trees.
\item Businesses are open during hours.
\end{itemize}


All of these relationships are different and yet the same.  Some relationships can be viewed as objects possessing characteristics. Some relationships can be viewed as if-then statements.  Some relationships can be viewed as cause and effect. But, they can all be viewed as two collections with associated elements.







\begin{definition} \textbf{\textcolor{green!50!black}{Relation}} \\
A \textbf{relation} is a package containing three sets or collections. One set is called the \textbf{domain}. One set is called the \textbf{codomain}.  Finally, there is a third set of pairings.  Each pairing associates a member of the domain with a member of the codomain.
\end{definition}


\begin{explanation} \textbf{Video: Relations}

[Click on the arrow to the right to expand for the video.]
\begin{expandable} 

\begin{center}
\youtube{N2gCtwa7jUw}
\end{center}

\end{expandable}
\end{explanation}





Relations are just about the vaguest, thinnest structure any two sets could possibly share. \\


There are two sets and then some of their members are associated to one another. In fact, this structure is so thin that we cannot do much with it.  Fortunately, we don't want to do much with relations.  We are just laying the groundwork for richer structures.  Right now, we just want to invent some language and notation, so that we can talk about these types of structures. We have a start already. The two sets of things in a relation have the names \textbf{\textcolor{purple!85!blue}{domain}} and \textbf{\textcolor{purple!85!blue}{codomain}}.




Because a relation might be encountered as an if-then statement or a cause-and-effect association, we have a natural directional feeling for the information.  The "if" part comes before the "then" part.  The "cause" comes before the "effect".  We would like this feeling reflected in our relation structure.  Therefore, relations come prepackaged with a feeling that the information is connected from the domain to the codomain.



\begin{idea}
If you are holding the ticket numered 23675, then you win the stuffed teddy bear. \\

\begin{itemize}
    \item Domain: ticket numbers
    \item Codomain: prizes
\end{itemize}

\end{idea}



\begin{idea}
Due to the artic vortex, tomorrow's high temperature will be $30^{\circ}$. \\

\begin{itemize}
    \item Domain: weather patterns
    \item Codomain: temperatures
\end{itemize}

\end{idea}







\textbf{\textcolor{red!90!darkgray}{A Feeling of Direction}}
\begin{itemize}
\item You pick items from the domain and then you get items in the codomain.
\item Items from the domain cause items in the codomain.
\item Items in the codomain occur because of items in the domain.
\item Elements of the codomain happen at places of the domain.
\end{itemize}



\subsection{List Representations}

Most of our communication is going to be written in this course, so we need some agreements on how we will represent relations in writing.  We already have several ways of representing sets.  The easiest is to just list the members inside curly braces and separate them with commas.

\begin{center} 
\textbf{\textcolor{blue!75!black}{ domain = \{ Casablanca, House of Wax,  The Godfather, Lawrence of Arabia, Toy Story, The Fly \} }}
\end{center}

\begin{center} 
\textbf{\textcolor{blue!75!black}{ codomain = \{ Marlon Brando, Vincent Price, Humphrey Bogart, Peter O'Toole, Harrison Ford, Al Pacino \} }}
\end{center}

Now we need a way to present the pairings.  The traditonal way is to write them as ordered pairs: the left (or first) item coming from the domain and the right (or second) item from the codomain.  We can list these ordered pairs in a set of ordered pairs. 

\begin{center} 
\textbf{\textcolor{blue!75!black}{ pairs = \{ (The Godfather, Marlon Brando), (House of Wax, Vincent Price), (Casablanca, Humphrey Bogart), (Lawrence of Arabia, Peter O'Toole), (The Fly, Vincent Price), (The Godfather, Al Pacino) \}  }}
\end{center}


These three sets would make up a relation. 



\begin{question}

Which item(s) from this codomain is(are) paired with 'The Fly', which is in the domain?
\begin{selectAll}
	\choice{Marlon Brando}
	\choice[correct]{Vincent Price}
	\choice{Humphrey Bogart}
	\choice{Peter O'Toole}
	\choice{Harrison Ford}
	\choice{Al Pacino}
\end{selectAll}

\end{question}



\begin{question}

Which item(s) from this domain is(are) paired with 'Al Pacino', which is in the codomain?
\begin{selectAll}
	\choice{Casablanca}
	\choice{House of Wax}
	\choice[correct]{The Godfather}
	\choice{Lawrence of Arabia}
	\choice{Toy Story}
	\choice{The Fly}
\end{selectAll}

\end{question}






Of course, we will have many relations, which could get confusing.  Let's help ourselves out by naming our relations.  The name of the relation above will be \textit{Starring}.


\begin{example} The \textit{Starring} Relation\\
\begin{itemize}
\item domain = \{ Casablanca, House of Wax,  The Godfather, Lawrence of Arabia, Toy Story, The Fly \}  
\item codomain = \{ Marlon Brando, Vincent Price, Humphrey Bogart, Peter O'Toole, Harrison Ford, Al Pacino  \} 
\item pairs = \{ (The Godfather, Marlon Brando), (House of Wax, Vincent Price), (Casablanca, Humphrey Bogart), (Lawrence of Arabia, Peter O'Toole), (The Fly, Vincent Price), (The Godfather, Al Pacino) \} 
\end{itemize}
\end{example}


A relation is a package.  \\

A relation is a package of three sets. The domain and codomain are sets of information.  The third set is a set of pairs.  Each pair connects a member of the domain with a memeber of the codomain. 


\begin{template} Ordered Pairs \\
The template for writing an ordered pair looks like  

\[
\large{( domain \, item, codomain \, item )}
\]
\end{template}

There is a domain item written on the left and a codmain item written to the right.  They are separated with a comma.  All of that is wrapped in parentheses.


\begin{warning} Did You Notice? \\
\begin{itemize}
\item Nobody said every member of the domain had to actually appear in a pair.  Toy Story is in the domain but in no pair of \textit{Starring}.
\item Nobody said every member of the codomain had to actually appear in a pair.  Harrison Ford is in the codomain but in no pair of \textit{Starring}.
\item Nobody said domain members could not appear in multiple pairs.  The Godfather appears in two pairs of \textit{Starring}.
\item Nobody said codomain members could not appear in multiple pairs.  Vincent Price appears in two pairs of \textit{Starring}.
\end{itemize}
\end{warning}
















\subsection{Table Representations}



Lists are good representations of relations when the sets are not very big. The parentheses become difficult to browse through when there are a lot of them. Another representation for a relation comes in the form of a table. A table visually organizes the pairs much better.

\begin{example} The \textit{KindergartenIceCream} Relation\\
The \textit{KindergartenIceCream} relation pairs kindergarteners with their favorite ice cream flavors.

domain = \{ Kevin, Shay, Linda, Charmain, Charlie \}  \\
codomain = \{ Vanilla, Chocolate, Strawberry, Peach, Mango, Cherry \} 

\[
\begin{array}{l|l}
    \text{domain item}      & \text{codomain item}      \\ \hline
    Kevin   &  Chocolate \\
    Shay   & Strawberry \\
    Linda  &  Chocolate \\
    Linda  &  Peach \\
    Charmain &  Vanilla \\ 
\end{array}
\]


Each line of the table shows a pairing. From this table, we can tell that the relation \textit{KindergartenIceCream} pairs Kevin with Chocolate.  (Kevin, Chocolate) is a pair in \textit{KindergartenIceCream}.

We can see that Charlie is not in a pair.  Charlie does not have a favorite flavor of ice cream. Linda is in two pairs.  She has two favorite flavors.  Nobody has Cherry as their favorite flavor.

\end{example} 





\section{Questions}


We have created some mathematical structure for questions. From this structure, we can see that there are basically two kinds of questions.

\begin{itemize}
\item \textbf{[Type 1]} You know the domain item and want the corresponding codomain partners.
\item \textbf{[Type 2]} You know the codomain item and want the corresponding domain partners.
\end{itemize}



\begin{example} \textit{KindergartenIceCream} \\
The \textit{KindergartenIceCream} relation pairs kindergarteners with their favorite ice cream flavors.

domain = \{ Kevin, Shay, Linda, Charmain, Charlie \}  \\
codomain = \{ Vanilla, Chocolate, Strawberry, Peach, Mango, Cherry \} 

\[
\begin{array}{l|l}
    \text{domain item}      & \text{codomain item}      \\ \hline
    Kevin   &  Chocolate \\
    Shay   & Strawberry \\
    Linda  &  Chocolate \\
    Linda  &  Peach \\
    Charmain &  Vanilla \\ 
\end{array}
\]


\begin{question}
\textbf{[Type 1]:} What are Linda's favorite flavors? 

\begin{selectAll}
\choice{Vanilla}
\choice[correct]{Chocolate}
\choice{Strawberry}
\choice[correct]{Peach}
\choice{Mango}
\choice{Cherry}
\end{selectAll}
\end{question}

We are looking for pairs of the form (Linda, ???) inside the \textit{KindergartenIceCream} relation.


\begin{question}
\textbf{[Type 2]:} Whose favorite flavor is Chocolate? 

\begin{selectAll}
\choice[correct]{Kevin}
\choice{Shay}
\choice[correct]{Linda}
\choice{Charmain}
\choice{Charlie}

\end{selectAll}
\end{question}

We are looking for pairs of the form (???, Chocolate) inside the \textit{KindergartenIceCream} relation.



\end{example} 














\section{Too Much}

Our examples, so far, have been small.  What are we going to do when we want to examine a relation between something like movies and actors?  


\begin{example} The \textit{ActorsInMovies} Relation\\
The \textit{ActorsInMovies} relation pairs actors with movies they were in.

domain = All actors \\
codomain = All movies

The table would begin like this.

\[
\begin{array}{l|l}
    domain      & codomain      \\ \hline
    \text{Vincent Price}   &  \text{House of Wax} \\
    \text{Humphrey Bogart}   & \text{Casablanca} \\
    \text{Peter O'Toole}  &  \text{Lawrence of Arabia} \\
    \text{Vincent Price}  &  \text{The Fly} \\
    \text{Al Pacino} &  \text{The Godfather} \\ 
    \text{...} &  \text{...} \\ 
\end{array}
\]

\end{example} 


This table would have many rows. The Internet Movie Database lists over a million movies.  There would be more than 100 rows just for Vincent Price. There is no way we could visually sift through such a table for a question about Vincent Price movies.

We need to narrow the scope of our investigation here, quickly. Otherwise, we will be buried in a mountain of data.

Our plan is to investigate only certain types of relations.








\section{Narrowing Our Investigation}


Our examples, so far, have been small.  What are we going to do when we want to examine a relation between atoms and molecules?  That list or table is going to be too big to look at with our eyes.  How would we look through a table with billions of rows and find the ones holding carbon?  The topic of all relations is just too big of an investigation. Let's focus in on a particular type of relation.

Most of our questions really identify a single hypothesis (antecedent) and then expect a single associated conclusion (consequent).

This would translate into each domain member always connected to a single codomain member.

Let's focus our investigation to these types of relations.  These types of relations are called \textbf{\textcolor{purple!85!blue}{functions}}.

















\end{document}
