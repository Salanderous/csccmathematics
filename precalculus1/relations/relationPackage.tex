\documentclass{ximera}


\graphicspath{
  {./}
  {ximeraTutorial/}
  {basicPhilosophy/}
}

\newcommand{\mooculus}{\textsf{\textbf{MOOC}\textnormal{\textsf{ULUS}}}}

\usepackage{tkz-euclide}\usepackage{tikz}
\usepackage{tikz-cd}
\usetikzlibrary{arrows}
\tikzset{>=stealth,commutative diagrams/.cd,
  arrow style=tikz,diagrams={>=stealth}} %% cool arrow head
\tikzset{shorten <>/.style={ shorten >=#1, shorten <=#1 } } %% allows shorter vectors

\usetikzlibrary{backgrounds} %% for boxes around graphs
\usetikzlibrary{shapes,positioning}  %% Clouds and stars
\usetikzlibrary{matrix} %% for matrix
\usepgfplotslibrary{polar} %% for polar plots
\usepgfplotslibrary{fillbetween} %% to shade area between curves in TikZ
\usetkzobj{all}
\usepackage[makeroom]{cancel} %% for strike outs
%\usepackage{mathtools} %% for pretty underbrace % Breaks Ximera
%\usepackage{multicol}
\usepackage{pgffor} %% required for integral for loops



%% http://tex.stackexchange.com/questions/66490/drawing-a-tikz-arc-specifying-the-center
%% Draws beach ball
\tikzset{pics/carc/.style args={#1:#2:#3}{code={\draw[pic actions] (#1:#3) arc(#1:#2:#3);}}}



\usepackage{array}
\setlength{\extrarowheight}{+.1cm}
\newdimen\digitwidth
\settowidth\digitwidth{9}
\def\divrule#1#2{
\noalign{\moveright#1\digitwidth
\vbox{\hrule width#2\digitwidth}}}






\DeclareMathOperator{\arccot}{arccot}
\DeclareMathOperator{\arcsec}{arcsec}
\DeclareMathOperator{\arccsc}{arccsc}

















%%This is to help with formatting on future title pages.
\newenvironment{sectionOutcomes}{}{}


\title{A Package}

\begin{document}

\begin{abstract}
Information Packages
\end{abstract}
\maketitle


We view the world through relationships. That is, some information is connected to other information for some reason. These associations are how we navigate through our lives.

\begin{itemize}
\item Dishes have ingredients.
\item Movies have actors.
\item Roads have speed limits.
\item People have social security numbers.
\item Kindergarteners like ice cream flavors.
\item Measurements are quoted in different units.
\item Authors write books.
\item Students get course grades.
\item Mountains have heights.
\item Family's have trees.
\item Businesses have open hours.
\end{itemize}


All of these relationships are different and yet the same.  Some relationships can be viewed as objects possessing characteristics. Some relationships can be viewed if-then statements.  Some relationships can be viewed as cause and effect. But, they can all be viewed as two collections with associated elements.







\begin{definition} \textbf{relation} 
A relation is a package containing three sets or collections. One set is called the \textbf{domain}. One set is called the \textbf{codomain}.  Finally, there is a set of pairings.  Each pairing pairs a member of the domain with a memmber of the codomain.
\end{definition}


Relations are just about the vaguest, thinnest structure any two sets could have. There are two sets and then some of their members are associated to one another. In fact, this structure is so thin that we cannot do much with it.  But, we don't want to.  We are just laying the groundwork for more complex structures.  Right now, we just want to invent some language and notation, so that we can talk about these types of structures. We have a start already. The two sets in a relation have the names domain and codomain.


Because a relation might encountered as an if-then statement or a cause-and-effect association, we have a natural direction feeling for the information.  The "if" part comes before the "then" part.  The "cause" comes before the "effect".  We would like this feeling reflected in our relation structure.  Therefore, relations come prepackaged with a feeling that the information is connected from the domain to the codomain.

\begin{itemize}
\item If you pick items from the domain, then you get items from the codomain.
\item Items from the domain cause items from the codomain.
\item Items in the codomain occur because of items in the domain.
\item Elements of the codomain happen at items of the domain.
\end{itemize}



Most of our communication is going to be written in this course, so we need some agreements on how we will represent relations.  We already have several ways of representing sets.  The easiest is to just list the memebers inside curly braces.

\begin{center} 
domain = \{ Casablanca, House of Wax,  The Godfather, Lawrence of Arabia, Toy Story, The Fly \} 
\end{center}

\begin{center} 
codomain = \{ Marlon Brando, Vincent Price, Humphrey Bogart, Peter O'Toole, Harrison Ford, Al Pacino \} 
\end{center}

Now we need a way to indicate the pairings.  The traditonal way is to written the as ordered pairs.  The left item coming from the domain and the right item the paired codomain item.  We can list these ordered pairs in a set of ordered pairs.

\begin{center} 
pairs = \{ (The Godfather, Marlon Brando), (House of Wax, Vincent Price), (Casablanca, Humphrey Bogart), (Lawrence of Arabia, Peter O'Toole), (The Fly, Vincent Price), (The Godfather, Al Pacino) \} 
\end{center}


These three sets would make up a relation. 

Of course, we will have many relations, which could get confusing.  Let's help ourselves out with names.  The name of this relation will be \textit{AppearedIn}.


\begin{example} \textit{AppearedIn}
\begin{itemize}
\item domain = \{ Casablanca, House of Wax,  The Godfather, Lawrence of Arabia, Toy Story, The Fly \}  
\item codomain = \{ Marlon Brando, Vincent Price, Humphrey Bogart, Peter O'Toole, Harrison Ford, Al Pacino  \} 
\item pairs = \{ (The Godfather, Marlon Brando), (House of Wax, Vincent Price), (Casablanca, Humphrey Bogart), (Lawrence of Arabia, Peter O'Toole), (The Fly, Vincent Price), (The Godfather, Al Pacino) \} 
\end{itemize}
\end{example}


A relation is a package.  It is a package of three sets. The domain and codomain are sets of information.  The third set is a set of pairs.  Each pair connects a member of the domain with a memeber of the codomain. 

\begin{observation}
\begin{itemize}
\item Nobody said every member of the domain had to actually appear in a pair.  Toy Story is in no pair of \textit{AppearedIn}.
\item Nobody said every member of the codomain had to actually appear in a pair.  Harrison Ford is in no pair of \textit{AppearedIn}.
\item Nobody said domain members could not appear in multiple pairs.  The Godfather appears in two pairs of \textit{AppearedIn}.
\item Nobody said codomain members could not appear in multiple pairs.  Vincent Price appears in two pairs of \textit{AppearedIn}.
\end{itemize}
\end{observation}



\large{Table Representations} \\

Lists are good representations of relations when the sets are not very big. The parentheses become difficulty to browse when there is a lot of them. Another representation is via a table. A table visually organizes the pairs much better.

\begin{example} \textit{KindergartenIceCream}
The \textit{KindergartenIceCream} relation pairs kindergarteners with the favorite ice cream flavors.

domain = \{ Kevin, Shay, Linda, Charmain, Charlie \}  
codomain = \{ Vanilla, Chocolate, Strawberry, Peach, Mango, Cherry \} 

\[
\begin{array}{l|l}
    domain      & codomain      \\ \hline
    Kevin   &  Chocolate \\
    Shay   & Strawberry \\
    Linda  &  Chocolate \\
    Linda  &  Peach \\
    Charmain &  Vanilla \\ 
\end{array}
\]


Each line of the table shows a pairing. From this table, we can tell that the relation \textit{KindergartenIceCream} pairs Kevin with Chocolate.  (Kevin, CHocolate) is a pair in \textit{KindergartenIceCream}.

\textbf{Notice} that Charlie is not in a pair.  Charlie doesn't like ice cream. Linda is in two pairs.  SHe has two favorite flavors.  Nobody has Cherry as their favorite flavor.




\end{example} 




























\end{document}
