\documentclass{ximera}


\graphicspath{
  {./}
  {ximeraTutorial/}
  {basicPhilosophy/}
}

\newcommand{\mooculus}{\textsf{\textbf{MOOC}\textnormal{\textsf{ULUS}}}}

\usepackage{tkz-euclide}\usepackage{tikz}
\usepackage{tikz-cd}
\usetikzlibrary{arrows}
\tikzset{>=stealth,commutative diagrams/.cd,
  arrow style=tikz,diagrams={>=stealth}} %% cool arrow head
\tikzset{shorten <>/.style={ shorten >=#1, shorten <=#1 } } %% allows shorter vectors

\usetikzlibrary{backgrounds} %% for boxes around graphs
\usetikzlibrary{shapes,positioning}  %% Clouds and stars
\usetikzlibrary{matrix} %% for matrix
\usepgfplotslibrary{polar} %% for polar plots
\usepgfplotslibrary{fillbetween} %% to shade area between curves in TikZ
\usetkzobj{all}
\usepackage[makeroom]{cancel} %% for strike outs
%\usepackage{mathtools} %% for pretty underbrace % Breaks Ximera
%\usepackage{multicol}
\usepackage{pgffor} %% required for integral for loops



%% http://tex.stackexchange.com/questions/66490/drawing-a-tikz-arc-specifying-the-center
%% Draws beach ball
\tikzset{pics/carc/.style args={#1:#2:#3}{code={\draw[pic actions] (#1:#3) arc(#1:#2:#3);}}}



\usepackage{array}
\setlength{\extrarowheight}{+.1cm}
\newdimen\digitwidth
\settowidth\digitwidth{9}
\def\divrule#1#2{
\noalign{\moveright#1\digitwidth
\vbox{\hrule width#2\digitwidth}}}






\DeclareMathOperator{\arccot}{arccot}
\DeclareMathOperator{\arcsec}{arcsec}
\DeclareMathOperator{\arccsc}{arccsc}

















%%This is to help with formatting on future title pages.
\newenvironment{sectionOutcomes}{}{}


\title{Function of Change}

\begin{document}

\begin{abstract}
rate as a relationship
\end{abstract}
\maketitle






Consider the function $Q(x) = \frac{(x+7)(x-5)}{5}$








\begin{image}
\begin{tikzpicture}
  \begin{axis}[
            domain=-10:10, ymax=10, xmax=10, ymin=-10, xmin=-10,
            axis lines =center, xlabel=$x$, ylabel={$y=g(x)$}, grid = major,
            ytick={-10,-8,-6,-4,-2,2,4,6,8,10},
            xtick={-10,-8,-6,-4,-2,2,4,6,8,10},
            yticklabels={$-10$,$-8$,$-6$,$-4$,$-2$,$2$,$4$,$6$,$8$,$10$}, 
            xticklabels={$-10$,$-8$,$-6$,$-4$,$-2$,$2$,$4$,$6$,$8$,$10$},
            ticklabel style={font=\scriptsize},
            every axis y label/.style={at=(current axis.above origin),anchor=south},
            every axis x label/.style={at=(current axis.right of origin),anchor=west},
            axis on top
          ]
          
          %\addplot [line width=2, penColor2, smooth,samples=100,domain=(-6:2)] {-2*x-3};
            \addplot [line width=2, penColor, smooth,samples=100,domain=(-9:8),<->] {.2*(x+7)*(x-5)};

          %\addplot[color=penColor,fill=penColor2,only marks,mark=*] coordinates{(-6,9)};
          %\addplot[color=penColor,fill=penColor2,only marks,mark=*] coordinates{(2,-7)};



           

  \end{axis}
\end{tikzpicture}
\end{image}




The rate-of-change over the interval $[-3, 4]$ is 
\[  \frac{Q(4)-Q(-3)}{4-(-3)} = \frac{4.2}{7} = 0.6      \]


This is the slope of the secant line through the points $(-3, Q(-3))$ and $(4, Q(4))$.











\begin{image}
\begin{tikzpicture}
  \begin{axis}[
            domain=-10:10, ymax=10, xmax=10, ymin=-10, xmin=-10,
            axis lines =center, xlabel=$x$, ylabel={$y=g(x)$}, grid = major,
            ytick={-10,-8,-6,-4,-2,2,4,6,8,10},
            xtick={-10,-8,-6,-4,-2,2,4,6,8,10},
            yticklabels={$-10$,$-8$,$-6$,$-4$,$-2$,$2$,$4$,$6$,$8$,$10$}, 
            xticklabels={$-10$,$-8$,$-6$,$-4$,$-2$,$2$,$4$,$6$,$8$,$10$},
            ticklabel style={font=\scriptsize},
            every axis y label/.style={at=(current axis.above origin),anchor=south},
            every axis x label/.style={at=(current axis.right of origin),anchor=west},
            axis on top
          ]
          
            \addplot [line width=2, penColor2, smooth,samples=100,domain=(-6:8),<->] {0.6*(x+3)-6.4};
            \addplot [line width=2, penColor, smooth,samples=100,domain=(-9:8),<->] {.2*(x+7)*(x-5)};

          %\addplot[color=penColor,fill=penColor2,only marks,mark=*] coordinates{(-6,9)};
          %\addplot[color=penColor,fill=penColor2,only marks,mark=*] coordinates{(2,-7)};



           

  \end{axis}
\end{tikzpicture}
\end{image}




This rate-of-change is the constant rate-of-change that is needed in order to get from one endpoint to the other.  Of course, we can see from the graph that the rate-of-change does indeed change over the interval. 

This interval would not be a good interval to use if you were interested in the rate-of-change near $-3$.

$[-3, -2.5]$ would probably give a better value. 






The rate-of-change over the interval $[-3, -2.5]$ is 
\[  \frac{Q(-2.5)-Q(-3)}{-2.5-(-3)} = \frac{-0.35}{0.5} = -0.7      \]


This is the slope of the secant line through the points $(-3, Q(-3))$ and $(-2.5, Q(-2.5))$.











\begin{image}
\begin{tikzpicture}
  \begin{axis}[
            domain=-10:10, ymax=10, xmax=10, ymin=-10, xmin=-10,
            axis lines =center, xlabel=$x$, ylabel={$y=g(x)$}, grid = major,
            ytick={-10,-8,-6,-4,-2,2,4,6,8,10},
            xtick={-10,-8,-6,-4,-2,2,4,6,8,10},
            yticklabels={$-10$,$-8$,$-6$,$-4$,$-2$,$2$,$4$,$6$,$8$,$10$}, 
            xticklabels={$-10$,$-8$,$-6$,$-4$,$-2$,$2$,$4$,$6$,$8$,$10$},
            ticklabel style={font=\scriptsize},
            every axis y label/.style={at=(current axis.above origin),anchor=south},
            every axis x label/.style={at=(current axis.right of origin),anchor=west},
            axis on top
          ]
          
            \addplot [line width=2, penColor2, smooth,samples=100,domain=(-6:8),<->] {-0.7*(x+3)-6.4};
            \addplot [line width=2, penColor, smooth,samples=100,domain=(-9:8),<->] {.2*(x+7)*(x-5)};

          %\addplot[color=penColor,fill=penColor2,only marks,mark=*] coordinates{(-6,9)};
          %\addplot[color=penColor,fill=penColor2,only marks,mark=*] coordinates{(2,-7)};



           

  \end{axis}
\end{tikzpicture}
\end{image}




That is looking more like a tangent line.

If we make this intervla even smaller, we should get an even better approximation for the slope of the tangent line.




The rate-of-change over the interval $[-3, -3+h]$ is 
\[  \frac{Q(-3+h)-Q(-3)}{(-3+h)-(-3)} = \frac{Q(-3+h)-Q(-3)}{h} =  \frac{\frac{(-3+h+7)(-3+h-5)}{5}+64}{h}   \]


This is the slope of the secant line through the points $(-3, Q(-3))$ and $(-3+h, Q(-3+h))$, which is very near the tangent line at $(-3, Q(-3))$.



By changing the value of $h$, we can get a good estimate of the slope of the tangent line.









\begin{center}
\desmos{di7qdhdfnn}{400}{300}
\end{center}




The slope of the line tangent to the graph at $(-3, -6.4)$ looks to be about $-0.8$.

Let's keep track of these tangent line slopes.

We'll plot the point $(-3,-0.8)$.  This will be a visual encoding of the tangent slope for the domain number $-3$.





\begin{image}
\begin{tikzpicture}
  \begin{axis}[
            domain=-10:10, ymax=10, xmax=10, ymin=-10, xmin=-10,
            axis lines =center, xlabel=$x$, ylabel={$y=g(x)$}, grid = major,
            ytick={-10,-8,-6,-4,-2,2,4,6,8,10},
            xtick={-10,-8,-6,-4,-2,2,4,6,8,10},
            yticklabels={$-10$,$-8$,$-6$,$-4$,$-2$,$2$,$4$,$6$,$8$,$10$}, 
            xticklabels={$-10$,$-8$,$-6$,$-4$,$-2$,$2$,$4$,$6$,$8$,$10$},
            ticklabel style={font=\scriptsize},
            every axis y label/.style={at=(current axis.above origin),anchor=south},
            every axis x label/.style={at=(current axis.right of origin),anchor=west},
            axis on top
          ]
          
            %\addplot [line width=2, penColor2, smooth,samples=100,domain=(-6:8),<->] {-0.7*(x+3)-6.4};
            \addplot [line width=2, penColor, smooth,samples=100,domain=(-9:8),<->] {.2*(x+7)*(x-5)};

          	\addplot[color=penColor2,fill=penColor2,only marks,mark=*] coordinates{(-3,-0.8)};




           

  \end{axis}
\end{tikzpicture}
\end{image}


We could do this for a bunch of domain numbers.

\begin{enumerate}
\item select a domain number
\item select a very tiny interval beginning at that domain number
\item calculate the rate-of-change over the tiny interval
\item plot a point on the graph.  First coordinate is the domain number.  Second coordinate is the rate-of-change.
\end{enumerate}







\begin{image}
\begin{tikzpicture}
  \begin{axis}[
            domain=-10:10, ymax=10, xmax=10, ymin=-10, xmin=-10,
            axis lines =center, xlabel=$x$, ylabel={$y=g(x)$}, grid = major,
            ytick={-10,-8,-6,-4,-2,2,4,6,8,10},
            xtick={-10,-8,-6,-4,-2,2,4,6,8,10},
            yticklabels={$-10$,$-8$,$-6$,$-4$,$-2$,$2$,$4$,$6$,$8$,$10$}, 
            xticklabels={$-10$,$-8$,$-6$,$-4$,$-2$,$2$,$4$,$6$,$8$,$10$},
            ticklabel style={font=\scriptsize},
            every axis y label/.style={at=(current axis.above origin),anchor=south},
            every axis x label/.style={at=(current axis.right of origin),anchor=west},
            axis on top
          ]
          
            %\addplot [line width=2, penColor2, smooth,samples=100,domain=(-6:8),<->] {-0.7*(x+3)-6.4};
            \addplot [line width=2, penColor, smooth,samples=100,domain=(-9:8),<->] {.2*(x+7)*(x-5)};

          	\addplot[color=penColor2,fill=penColor2,only marks,mark=*] coordinates{(-5,-1.6)};
          	\addplot[color=penColor2,fill=penColor2,only marks,mark=*] coordinates{(-3,-0.8)};
          	\addplot[color=penColor2,fill=penColor2,only marks,mark=*] coordinates{(-1,0)};
          	\addplot[color=penColor2,fill=penColor2,only marks,mark=*] coordinates{(1,0.8)};
          	\addplot[color=penColor2,fill=penColor2,only marks,mark=*] coordinates{(3,1.6)};
          	\addplot[color=penColor2,fill=penColor2,only marks,mark=*] coordinates{(5,2.4)};




           

  \end{axis}
\end{tikzpicture}
\end{image}






If we do this for every domain number, then we will have dreated a number function.  Every domain number will be paired with the slope of the tangent line at the corresponding point.

We could graph this slope function.







\begin{center}
\desmos{4fe0bvurqv}{400}{300}
\end{center}




This function whose values are the rates-of-change of another function is called the \textbf{derivative} of the other function.




$\blacktriangleright$ Let $f$ be a function.  Then the derivative of $f$ is $f'$, the derivative of $f$.



\begin{definition} The Derivative

Let $f$ be a function.

Then $f'$ represents the \textbf{derivative of $f$}.


$f'(a) = $ the slope of the tangent line at $(a, f(a))$, on the graph of $f$.




\end{definition}





$\blacktriangleright$ Take another look at the graphs above.  The derivative has a zero at $-1$.  Speculate on why $f'$ has a  zero there.





\begin{example} The Derivative





Graph of $y = f(x) = e^{\tfrac{x}{5}}$







\begin{image}
\begin{tikzpicture}
  \begin{axis}[
            domain=-10:10, ymax=10, xmax=10, ymin=-10, xmin=-10,
            axis lines =center, xlabel=$x$, ylabel={$y=f(x)$}, grid = major,
            ytick={-10,-8,-6,-4,-2,2,4,6,8,10},
            xtick={-10,-8,-6,-4,-2,2,4,6,8,10},
            yticklabels={$-10$,$-8$,$-6$,$-4$,$-2$,$2$,$4$,$6$,$8$,$10$}, 
            xticklabels={$-10$,$-8$,$-6$,$-4$,$-2$,$2$,$4$,$6$,$8$,$10$},
            ticklabel style={font=\scriptsize},
            every axis y label/.style={at=(current axis.above origin),anchor=south},
            every axis x label/.style={at=(current axis.right of origin),anchor=west},
            axis on top
          ]
          
            %\addplot [line width=2, penColor2, smooth,samples=100,domain=(-6:8),<->] {-0.7*(x+3)-6.4};
            \addplot [line width=2, penColor, smooth,samples=100,domain=(-9:9),<->] {e^(0.2*x)};


           

  \end{axis}
\end{tikzpicture}
\end{image}




The value of $f'(4)$ is
\begin{multipleChoice}
\choice[correct]{positive}
\choice{negative}
\end{multipleChoice}





\end{example}










\begin{example}



Here is a graph of $y = g(x) = sin(x)$ and $g'(x)$.






\begin{center}
\desmos{yp5zobfigx}{400}{300}
\end{center}



The derivative of $sin(x)$ is
\begin{multipleChoice}
\choice[correct]{$cos(x)$}
\choice{$sin(x)$}
\choice{$-cos(x)$}
\choice{$-sin(x)$}
\end{multipleChoice}


\end{example}







































\end{document}
