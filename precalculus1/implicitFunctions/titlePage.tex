\documentclass{ximera}


\graphicspath{
  {./}
  {ximeraTutorial/}
  {basicPhilosophy/}
}

\newcommand{\mooculus}{\textsf{\textbf{MOOC}\textnormal{\textsf{ULUS}}}}

\usepackage{tkz-euclide}\usepackage{tikz}
\usepackage{tikz-cd}
\usetikzlibrary{arrows}
\tikzset{>=stealth,commutative diagrams/.cd,
  arrow style=tikz,diagrams={>=stealth}} %% cool arrow head
\tikzset{shorten <>/.style={ shorten >=#1, shorten <=#1 } } %% allows shorter vectors

\usetikzlibrary{backgrounds} %% for boxes around graphs
\usetikzlibrary{shapes,positioning}  %% Clouds and stars
\usetikzlibrary{matrix} %% for matrix
\usepgfplotslibrary{polar} %% for polar plots
\usepgfplotslibrary{fillbetween} %% to shade area between curves in TikZ
\usetkzobj{all}
\usepackage[makeroom]{cancel} %% for strike outs
%\usepackage{mathtools} %% for pretty underbrace % Breaks Ximera
%\usepackage{multicol}
\usepackage{pgffor} %% required for integral for loops



%% http://tex.stackexchange.com/questions/66490/drawing-a-tikz-arc-specifying-the-center
%% Draws beach ball
\tikzset{pics/carc/.style args={#1:#2:#3}{code={\draw[pic actions] (#1:#3) arc(#1:#2:#3);}}}



\usepackage{array}
\setlength{\extrarowheight}{+.1cm}
\newdimen\digitwidth
\settowidth\digitwidth{9}
\def\divrule#1#2{
\noalign{\moveright#1\digitwidth
\vbox{\hrule width#2\digitwidth}}}






\DeclareMathOperator{\arccot}{arccot}
\DeclareMathOperator{\arcsec}{arcsec}
\DeclareMathOperator{\arccsc}{arccsc}

















%%This is to help with formatting on future title pages.
\newenvironment{sectionOutcomes}{}{}


\title{Implicit Functions}

\begin{document}

\begin{abstract}
%Stuff can go here later if we want!
\end{abstract}
\maketitle








We have been investigating functions.  More specifically, we have been investigating functions of one variable - all of our formulas have one variable.

We have investigated the graphs of these functions.  The graph of a function is a collection of points in the Cartesian plane.  If $f$ is the function, then each point belonging to the graph is of the form $(d, f(d))$, where $d$ is a member of the domain of $f$.

Our functions have domains which are subsets of the real numbers and they take on real number values.  Our notation for this looks like 

\[  f : \mathbb{R} \mapsto \mathbb{R}      \]



\begin{center}
People say, "$f$ maps $\mathbb{R}$ to $\mathbb{R}$."
\end{center}




We can view all of this in a different light. \\




Rather than thinking that the function pairs real numbers with real number, let's thinking of a function that takes real numbers to points in the Cartesian plane.


We might think of the function $F$: $F(d) = (d, f(d))$.  It maps a single real number to an ordered paired of real numbers.



\[  F : \mathbb{R} \mapsto \mathbb{R^2}      \]



When $f$ is a continuous function, then the image of $F$ is called a \textbf{curve}.










\begin{sectionOutcomes}
In this section, students will 

\begin{itemize}
\item compare curves and functions.
\item graph.
\item evaluate.
\item .
\item .
\end{itemize}
\end{sectionOutcomes}

\end{document}
