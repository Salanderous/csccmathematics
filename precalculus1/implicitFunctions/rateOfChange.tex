\documentclass{ximera}


\graphicspath{
  {./}
  {ximeraTutorial/}
  {basicPhilosophy/}
}

\newcommand{\mooculus}{\textsf{\textbf{MOOC}\textnormal{\textsf{ULUS}}}}

\usepackage{tkz-euclide}\usepackage{tikz}
\usepackage{tikz-cd}
\usetikzlibrary{arrows}
\tikzset{>=stealth,commutative diagrams/.cd,
  arrow style=tikz,diagrams={>=stealth}} %% cool arrow head
\tikzset{shorten <>/.style={ shorten >=#1, shorten <=#1 } } %% allows shorter vectors

\usetikzlibrary{backgrounds} %% for boxes around graphs
\usetikzlibrary{shapes,positioning}  %% Clouds and stars
\usetikzlibrary{matrix} %% for matrix
\usepgfplotslibrary{polar} %% for polar plots
\usepgfplotslibrary{fillbetween} %% to shade area between curves in TikZ
\usetkzobj{all}
\usepackage[makeroom]{cancel} %% for strike outs
%\usepackage{mathtools} %% for pretty underbrace % Breaks Ximera
%\usepackage{multicol}
\usepackage{pgffor} %% required for integral for loops



%% http://tex.stackexchange.com/questions/66490/drawing-a-tikz-arc-specifying-the-center
%% Draws beach ball
\tikzset{pics/carc/.style args={#1:#2:#3}{code={\draw[pic actions] (#1:#3) arc(#1:#2:#3);}}}



\usepackage{array}
\setlength{\extrarowheight}{+.1cm}
\newdimen\digitwidth
\settowidth\digitwidth{9}
\def\divrule#1#2{
\noalign{\moveright#1\digitwidth
\vbox{\hrule width#2\digitwidth}}}






\DeclareMathOperator{\arccot}{arccot}
\DeclareMathOperator{\arcsec}{arcsec}
\DeclareMathOperator{\arccsc}{arccsc}

















%%This is to help with formatting on future title pages.
\newenvironment{sectionOutcomes}{}{}


\title{Rate of Change}

\begin{document}

\begin{abstract}
tangents
\end{abstract}
\maketitle



















Define $y(x)$ via $(x+4)^2 + (y-2)^2 = 5^2$ with $x \in [-4, 1]$  and $y \in [-3, 2]$.  \\



We do not have a formula for $y(x)$.  We are using an equation to describe the relationship between $y$ and $x$.  We say that $y(x)$ is defined \textbf{implicitly}.


Here is a graph of $y(x)$.








\begin{image}
\begin{tikzpicture}
  \begin{axis}[
            domain=-10:10, ymax=10, xmax=10, ymin=-10, xmin=-10, unit vector ratio*=1 1 1,
            axis lines =center, xlabel=$x$, ylabel=$y$, grid = major, grid style={dashed},
            ytick={-10,-8,-6,-4,-2,2,4,6,8,10},
            xtick={-10,-8,-6,-4,-2,2,4,6,8,10},
            yticklabels={$-10$,$-8$,$-6$,$-4$,$-2$,$2$,$4$,$6$,$8$,$10$}, 
            xticklabels={$-10$,$-8$,$-6$,$-4$,$-2$,$2$,$4$,$6$,$8$,$10$},
            ticklabel style={font=\scriptsize},
            every axis y label/.style={at=(current axis.above origin),anchor=south},
            every axis x label/.style={at=(current axis.right of origin),anchor=west},
            axis on top
          ]
          
            
      %\addplot [line width=2, penColor, smooth,samples=200,domain=(-1:9)] {5};

		\addplot [line width=2, penColor, smooth, samples=200, domain=(-1.57:0)] ({5*cos(deg(x)) - 4},{5*sin(deg(x)) + 2});
		\addplot[color=penColor,fill=penColor,only marks,mark=*] coordinates{(-4,-3)};
		\addplot[color=penColor,fill=penColor,only marks,mark=*] coordinates{(1,2)};






  \end{axis}
\end{tikzpicture}
\end{image}


With these restrictions, we have a function.

From the equation, we can see that the point $(0, -1)$ is on the graph, which means that $y(0) = -1$.  

There must also be a value for $y'(0)$.  With derivative, we could get an exact measurement for this.  As it is, we will need to approximate graphically.  $y'(0)$ is the slope of the tangent limne to the curve at $(0, -1)$.












\begin{image}
\begin{tikzpicture}
  \begin{axis}[
            domain=-10:10, ymax=10, xmax=10, ymin=-10, xmin=-10, unit vector ratio*=1 1 1,
            axis lines =center, xlabel=$x$, ylabel=$y$, grid = major, grid style={dashed},
            ytick={-10,-8,-6,-4,-2,2,4,6,8,10},
            xtick={-10,-8,-6,-4,-2,2,4,6,8,10},
            yticklabels={$-10$,$-8$,$-6$,$-4$,$-2$,$2$,$4$,$6$,$8$,$10$}, 
            xticklabels={$-10$,$-8$,$-6$,$-4$,$-2$,$2$,$4$,$6$,$8$,$10$},
            ticklabel style={font=\scriptsize},
            every axis y label/.style={at=(current axis.above origin),anchor=south},
            every axis x label/.style={at=(current axis.right of origin),anchor=west},
            axis on top
          ]
          
            
      %\addplot [line width=2, penColor, smooth,samples=200,domain=(-1:9)] {5};

		\addplot [line width=2, penColor, smooth, samples=200, domain=(-1.57:0)] ({5*cos(deg(x)) - 4},{5*sin(deg(x)) + 2});
		\addplot[color=penColor,fill=penColor,only marks,mark=*] coordinates{(-4,-3)};
		\addplot[color=penColor,fill=penColor,only marks,mark=*] coordinates{(1,2)};

		\addplot[color=penColor2,fill=penColor2,only marks,mark=*] coordinates{(0,-1)};
		\addplot [line width=2, penColor2, smooth,samples=200,domain=(-6:6),<->] {1.33*x - 1};






  \end{axis}
\end{tikzpicture}
\end{image}






Here is the graph will some auxillary lines with slope $1$ and $2$ for comparison.








\begin{image}
\begin{tikzpicture}
  \begin{axis}[
            domain=-10:10, ymax=10, xmax=10, ymin=-10, xmin=-10, unit vector ratio*=1 1 1,
            axis lines =center, xlabel=$x$, ylabel=$y$, grid = major, grid style={dashed},
            ytick={-10,-8,-6,-4,-2,2,4,6,8,10},
            xtick={-10,-8,-6,-4,-2,2,4,6,8,10},
            yticklabels={$-10$,$-8$,$-6$,$-4$,$-2$,$2$,$4$,$6$,$8$,$10$}, 
            xticklabels={$-10$,$-8$,$-6$,$-4$,$-2$,$2$,$4$,$6$,$8$,$10$},
            ticklabel style={font=\scriptsize},
            every axis y label/.style={at=(current axis.above origin),anchor=south},
            every axis x label/.style={at=(current axis.right of origin),anchor=west},
            axis on top
          ]
          
		            
		\addplot [line width=1, gray, dashed,samples=200,domain=(-4:5),<->] {2*x - 1};
		\addplot [line width=1, gray, dashed,samples=200,domain=(-6:6),<->] {x - 1};

		\addplot [line width=2, penColor, smooth, samples=200, domain=(-1.57:0)] ({5*cos(deg(x)) - 4},{5*sin(deg(x)) + 2});
		\addplot[color=penColor,fill=penColor,only marks,mark=*] coordinates{(-4,-3)};
		\addplot[color=penColor,fill=penColor,only marks,mark=*] coordinates{(1,2)};

		\addplot[color=penColor2,fill=penColor2,only marks,mark=*] coordinates{(0,-1)};
		\addplot [line width=2, penColor2, smooth,samples=200,domain=(-6:6),<->] {1.33*x - 1};






  \end{axis}
\end{tikzpicture}
\end{image}


Let's gather a couple of points off of the tangent line:  $(0, -1)$ and $(4, 4.2)$. This gives us an approximate slope of 



\[
\frac{4.2 - (-1)}{4 - 1} = \frac{5.2}{3} \approx 1.7.
\]



The approximate equation of the tangent line at $(0,1)$ is $y - (-1) = 1.7(x - 0)$ or $y = 1.7x - 1$.



From the graph, we can also see that the zero of $y(x) = 1.7x - 1$, which is $\answer{\frac{1}{1.7}} \approx 0.5882$, is a very good approximation for the zero of our original function.  That zero is given by



\[
(x+4)^2 + (0-2)^2 = 5^2
\]


\[
(x+4)^2  = 25 - 4 = 21
\]



\[
x+4  = \sqrt{21}
\]

\[
x = \sqrt{21} - 4 \approx 0.58257
\]





This might be a method of approximating zeros or functions when the formula or equation is too difficult to work with.









\end{document}
