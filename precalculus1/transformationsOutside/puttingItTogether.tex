\documentclass{ximera}


\graphicspath{
  {./}
  {ximeraTutorial/}
  {basicPhilosophy/}
}

\newcommand{\mooculus}{\textsf{\textbf{MOOC}\textnormal{\textsf{ULUS}}}}

\usepackage{tkz-euclide}\usepackage{tikz}
\usepackage{tikz-cd}
\usetikzlibrary{arrows}
\tikzset{>=stealth,commutative diagrams/.cd,
  arrow style=tikz,diagrams={>=stealth}} %% cool arrow head
\tikzset{shorten <>/.style={ shorten >=#1, shorten <=#1 } } %% allows shorter vectors

\usetikzlibrary{backgrounds} %% for boxes around graphs
\usetikzlibrary{shapes,positioning}  %% Clouds and stars
\usetikzlibrary{matrix} %% for matrix
\usepgfplotslibrary{polar} %% for polar plots
\usepgfplotslibrary{fillbetween} %% to shade area between curves in TikZ
\usetkzobj{all}
\usepackage[makeroom]{cancel} %% for strike outs
%\usepackage{mathtools} %% for pretty underbrace % Breaks Ximera
%\usepackage{multicol}
\usepackage{pgffor} %% required for integral for loops



%% http://tex.stackexchange.com/questions/66490/drawing-a-tikz-arc-specifying-the-center
%% Draws beach ball
\tikzset{pics/carc/.style args={#1:#2:#3}{code={\draw[pic actions] (#1:#3) arc(#1:#2:#3);}}}



\usepackage{array}
\setlength{\extrarowheight}{+.1cm}
\newdimen\digitwidth
\settowidth\digitwidth{9}
\def\divrule#1#2{
\noalign{\moveright#1\digitwidth
\vbox{\hrule width#2\digitwidth}}}






\DeclareMathOperator{\arccot}{arccot}
\DeclareMathOperator{\arcsec}{arcsec}
\DeclareMathOperator{\arccsc}{arccsc}

















%%This is to help with formatting on future title pages.
\newenvironment{sectionOutcomes}{}{}


\title{All Together}

\begin{document}

\begin{abstract}
range
\end{abstract}
\maketitle








Let $L_i(t) = -2t + 1$. \\
Let $L_o(t) = \frac{1}{2}t + 2$.


Let $K(x)$ be a piecewise defined function defined by 


\[
K(x) = 
\begin{cases}
  2|x+6| - 4         &    \,     \text{ on } \,   [-8,-3)    \\
  -6               &    \,     \text{ on } \,    [-3,1)      \\
  -\frac{3}{2}(x-4)(x-6)    &   \,     \text{ on } \,    (4,8]
\end{cases}
\]







\begin{image}
\begin{tikzpicture}
  \begin{axis}[
            domain=-10:10, ymax=10, xmax=10, ymin=-10, xmin=-10,
            axis lines =center, xlabel=$x$, ylabel={$y=K(x)$}, grid = major,
            ytick={-10,-8,-6,-4,-2,2,4,6,8,10},
            xtick={-10,-8,-6,-4,-2,2,4,6,8,10},
            yticklabels={$-10$,$-8$,$-6$,$-4$,$-2$,$2$,$4$,$6$,$8$,$10$}, 
            xticklabels={$-10$,$-8$,$-6$,$-4$,$-2$,$2$,$4$,$6$,$8$,$10$},
            ticklabel style={font=\scriptsize},
            every axis y label/.style={at=(current axis.above origin),anchor=south},
            every axis x label/.style={at=(current axis.right of origin),anchor=west},
            axis on top
          ]
          
          %\addplot [line width=2, penColor2, smooth,samples=100,domain=(-6:2)] {-2*x-3};
            \addplot [line width=2, penColor, smooth,samples=100,domain=(-8:-3)] {2*abs(x+6)-4};
            \addplot[color=penColor,fill=penColor,only marks,mark=*] coordinates{(-8,0)};
          	\addplot[color=penColor,fill=white,only marks,mark=*] coordinates{(-3,2)};

            \addplot [line width=2, penColor, smooth,samples=100,domain=(-3:1)] {-6};
            \addplot[color=penColor,fill=penColor,only marks,mark=*] coordinates{(-3,-6)};
            \addplot[color=penColor,fill=white,only marks,mark=*] coordinates{(1,-6)};

            \addplot [line width=2, penColor, smooth,samples=100,domain=(4:8)] {-1.5*(x-4)*(x-8)};
            \addplot[color=penColor,fill=white,only marks,mark=*] coordinates{(4,0)};
            \addplot[color=penColor,fill=penColor,only marks,mark=*] coordinates{(8,0)};



  \end{axis}
\end{tikzpicture}
\end{image}



This time, we will compose three functions:  $T = L_o \circ K \circ L_i$.


What will be domain (horizontal) effects and range (vertical) effects?  The values $L_i$ will become the inputs to $K$. Therefore, $L_i$ affects the domain of $K$ and moves the graph horizontally.  $L_o$ will take the range values from $K$ and move the graph vertically.



$\blacktriangleright$ Domain - horizontal : $L_i(t) = -2t + 1$


\begin{itemize}
\item The leading coefficient for $L_i$ is negative.  The graph is going to be reflected horizontally. \\
The parabola will be on the left and the ``Vee'' on the right.
\item The leading coefficent is $-2$, which speeds up the input into $K$, which compresses the graph horizontally.
\item Finally, $L_i$ is adding $1$, which is going into $K$, which shifts the graph to the left.

\end{itemize}





$\blacktriangleright$ Range - vertical : $L_o = \frac{1}{2}t + 2$


\begin{itemize}
\item The leading coefficient for $L_o$ is $\frac{1}{2}$.  The graph is going to be compressed vertically. \\
\item Finally, the graph will be shifted up $2$.

\end{itemize}


Graph of $ y = T(m) = (L_o \circ K \circ L_i)(m)$.







\begin{image}
\begin{tikzpicture}
  \begin{axis}[
            domain=-10:10, ymax=10, xmax=10, ymin=-11, xmin=-10,
            axis lines =center, xlabel=$m$, ylabel=$y$, grid = major,
            ytick={-10,-8,-6,-4,-2,2,4,6,8,10},
            xtick={-10,-8,-6,-4,-2,2,4,6,8,10},
            yticklabels={$-10$,$-8$,$-6$,$-4$,$-2$,$2$,$4$,$6$,$8$,$10$}, 
            xticklabels={$-10$,$-8$,$-6$,$-4$,$-2$,$2$,$4$,$6$,$8$,$10$},
            ticklabel style={font=\scriptsize},
            every axis y label/.style={at=(current axis.above origin),anchor=south},
            every axis x label/.style={at=(current axis.right of origin),anchor=west},
            axis on top
          ]
          
          %\addplot [line width=2, penColor2, smooth,samples=100,domain=(-6:2)] {-2*x-3};
            \addplot [line width=2, penColor, smooth,samples=100,domain=(2:4.5)] {abs(-2*x+7)};
            \addplot[color=penColor,fill=white,only marks,mark=*] coordinates{(2,3)};
          	\addplot[color=penColor,fill=penColor,only marks,mark=*] coordinates{(4.5,2)};



            \addplot [line width=2, penColor, smooth,samples=100,domain=(0:2)] {-1};
            \addplot[color=penColor,fill=white,only marks,mark=*] coordinates{(0,-1)};
            \addplot[color=penColor,fill=penColor,only marks,mark=*] coordinates{(2,-1)};



            \addplot [line width=2, penColor, smooth,samples=100,domain=(-3.5:-1.5)] {-0.75*(2*x+3)*(2*x+7)+2};
            \addplot[color=penColor,fill=white,only marks,mark=*] coordinates{(-1.5,2)};
            \addplot[color=penColor,fill=penColor,only marks,mark=*] coordinates{(-3.5,2)};



  \end{axis}
\end{tikzpicture}
\end{image}





\textbf{Note:} All of the endpoints remained the same type.  

\begin{itemize}
\item The short arm of the ``Vee'' is a solid dot on both graphs.
\item The long arm of the ``Vee'' is a hollow dot on both graphs.
\item The end of the horizontal line segment nearest the ``Vee'' is a solid dot on both graphs.
\item The end of the horizontal line segment nearest the parabola is a hollow dot on both graphs.
\item The inside endpoint on the parabola is hollow on both graphs.
\item The outside endpoint on the parabola is solid on both graphs.
\end{itemize}










All horizontal measurements are half what they were, so that when $L_i$ multiplies them by $2$, they get back to their original size for input into $K$.

All height measurements are half what the were to begin with.



We can trace the solid left enpoint of the line segment: $(-3, K(-3)) = (-3, -6)$ on the graph of $K$ through the transformations:





\begin{align*}
\text{Start :} & \text{  } & (-3, K(-3))  \\
\text{Start :} & \text{  } & (-3, -6) \\
\text{Horz Shift Left $1$ :} & \text{  } & (-4,-6)   \\
\text{Horz Reflection :} & \text{  } & (4, -6)   \\
\text{Horz Compress Factor $\tfrac{1}{2}$ :}  & \text{  } & (2, -6)   \\
\text{Vert Compress Factor $\tfrac{1}{2}$:} & \text{  } & (2, -3)   \\
\text{Vert Shift Up $2$ :} & \text{  } & (2, -1)   \\
\text{Composition :} & \text{  } & (2, -1)   
\end{align*}


The solid endpoint of the horizontal line segment will be $(2, -1)$.












\begin{procedure} 



Trace the hollow endpoint of the parabola segment: $(4, 0)$





\begin{align*}
\text{Start :} & \text{  } & (4, K(4))  \\
\text{Start :} & \text{  } & (4, 0) \\
\text{Horz Shift Left $1$ :} & \text{  } & \left( \answer{3}, \answer{0} \right)   \\
\text{Horz Reflection :} & \text{  } & \left( \answer{-3}, \answer{0} \right)   \\
\text{Horz Compress Factor $\tfrac{1}{2}$ :}  & \text{  } & \left( \answer{-\frac{3}{2}}, \answer{0} \right)   \\
\text{Vert Compress Factor $\tfrac{1}{2}$:} & \text{  } & \left( \answer{-\frac{3}{2}}, \answer{0} \right)   \\
\text{Vert Shift Up $2$ :} & \text{  } & \left( \answer{-\frac{3}{2}}, \answer{2} \right)   \\
\text{Composition :} & \text{  } & \left( -\frac{3}{2}, 2 \right)   
\end{align*}


The hollow endpoint of the parabola segment will be $\left( -\frac{3}{2}, 2 \right)$.

\end{procedure}
















\begin{question} 

We know the horizontal line segment remains a horizontal line segment. Therefore the vertical coordinate of the hollow endpoint of the horizontal line segment will become $\answer{-1}$.

\end{question}




  








\begin{procedure} 



Trace the corner of the ``Vee'': $(-6, -4)$





\begin{align*}
\text{Start :} & \text{  } & (-6, K(-6))  \\
\text{Start :} & \text{  } & (-6, -4) \\
\text{Horz Shift Left $1$ :} & \text{  } & \left( \answer{-7}, \answer{-4} \right)   \\
\text{Horz Reflection :} & \text{  } & \left( \answer{7}, \answer{-4} \right)   \\
\text{Horz Compress Factor $\tfrac{1}{2}$ :}  & \text{  } & \left( \answer{\frac{7}{2}}, \answer{-4} \right)   \\
\text{Vert Compress Factor $\tfrac{1}{2}$:} & \text{  } & \left( \answer{\frac{7}{2}}, \answer{-2} \right)   \\
\text{Vert Shift Up $2$ :} & \text{  } & \left( \answer{\frac{7}{2}}, \answer{0} \right)   \\
\text{Composition :} & \text{  } & \left( \frac{7}{2}, 0 \right)   
\end{align*}


The hollow endpoint of the parabola segment will be $\left( \frac{7}{2}, 0 \right)$.

\end{procedure}













\begin{center}
\textbf{\textcolor{green!50!black}{ooooo=-=-=-=-=-=-=-=-=-=-=-=-=ooOoo=-=-=-=-=-=-=-=-=-=-=-=-=ooooo}} \\

more examples can be found by following this link\\ \link[More Examples of Transforming the Inside]{https://ximera.osu.edu/csccmathematics/precalculus1/precalculus1/transformationsOutside/examples/exampleList}

\end{center}


\end{document}
