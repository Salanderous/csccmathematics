\documentclass{ximera}


\graphicspath{
  {./}
  {ximeraTutorial/}
  {basicPhilosophy/}
}

\newcommand{\mooculus}{\textsf{\textbf{MOOC}\textnormal{\textsf{ULUS}}}}

\usepackage{tkz-euclide}\usepackage{tikz}
\usepackage{tikz-cd}
\usetikzlibrary{arrows}
\tikzset{>=stealth,commutative diagrams/.cd,
  arrow style=tikz,diagrams={>=stealth}} %% cool arrow head
\tikzset{shorten <>/.style={ shorten >=#1, shorten <=#1 } } %% allows shorter vectors

\usetikzlibrary{backgrounds} %% for boxes around graphs
\usetikzlibrary{shapes,positioning}  %% Clouds and stars
\usetikzlibrary{matrix} %% for matrix
\usepgfplotslibrary{polar} %% for polar plots
\usepgfplotslibrary{fillbetween} %% to shade area between curves in TikZ
\usetkzobj{all}
\usepackage[makeroom]{cancel} %% for strike outs
%\usepackage{mathtools} %% for pretty underbrace % Breaks Ximera
%\usepackage{multicol}
\usepackage{pgffor} %% required for integral for loops



%% http://tex.stackexchange.com/questions/66490/drawing-a-tikz-arc-specifying-the-center
%% Draws beach ball
\tikzset{pics/carc/.style args={#1:#2:#3}{code={\draw[pic actions] (#1:#3) arc(#1:#2:#3);}}}



\usepackage{array}
\setlength{\extrarowheight}{+.1cm}
\newdimen\digitwidth
\settowidth\digitwidth{9}
\def\divrule#1#2{
\noalign{\moveright#1\digitwidth
\vbox{\hrule width#2\digitwidth}}}






\DeclareMathOperator{\arccot}{arccot}
\DeclareMathOperator{\arcsec}{arcsec}
\DeclareMathOperator{\arccsc}{arccsc}

















%%This is to help with formatting on future title pages.
\newenvironment{sectionOutcomes}{}{}


\title{Outside}

\begin{document}

\begin{abstract}
range
\end{abstract}
\maketitle







We have already seen a couple of versions of composition. \\

$\blacktriangleright$ \textbf{Pointwise} composition was seen via indiviudal numbers: $(F \circ G)(a) = F(G(a))$.

A number, $a$, in the domain of $G$ was connected to its range partner, $G(a)$.  $G$ was evaluated at $a$ and the function value was then viewed as a member of the domain of $F$.  As a member of the domain of $F$, $F$ can be evaluated at $G(a)$ to get $F(G(a))$.




$\blacktriangleright$ \textbf{Linear} composition between two linear functions produced a whole new function.  Instead of thinking of domain numbers individually, this composition was viewed as an operation on linear functions.

\[    (L_o \circ L_i)(x) = L_o(L_i(x))  \]

There is an outside function, $L_o(x)$, and an inside function $L_i(x)$.  The composition operation, $\circ$, is applied and a new linear function is created.  Our symbol for this number function is $(L_o \circ L_i)$ or $L_o \circ L_i$.  The parentheses are used to clear up communication.

In our investigations, we have discovered that $(L_o \circ L_i)$ "is" $L_i$, just shifted, stretched, and reflected vertically.


We would like to extend this idea of a function operation beyond linear functions.





\section{Composition}










\begin{definition} Composition


Given two functions, $Outside$ and $Inside$, the composition, $Outside \circ Inside$ is defined by

\[      (Outside \circ Inside)(a) = Outside(Inside(a))        \]

For $ a \in \{  x \in Dom_{Inside} \, | \,    Inside(x) \in Dom_{Outside}  \}$

\end{definition}



We would like to focus on the outside function as a linear function.




Let $f(x)$ be any function. \\
Let $L(y)$ be any linear function. \\


Form the composition $L(f(z))$. \\

$\blacktriangleright$ How does $L$ affect $f$?




The main issue here is the range of $f$ intersecting the domain of $L$.  However, since the domain of a linear function is all real numbers, there shouldn't be a problem.\\






$\star$ \textbf{\textcolor{purple}{Outside = Linear}}


In this section, our $Outside$ function will always be a linear function.



\[      (L \circ Inside)(z) = L(Inside(z))        \]


where $L(x) = a \, x + b$, with $a$ and $b$ real numbers and $a \ne 0$.



Let's consider a quadratic function: $Q(h) = (h+1)(h-4)$ as the $inside$ function.











\begin{image}
\begin{tikzpicture}
  \begin{axis}[
            domain=-10:10, ymax=10, xmax=10, ymin=-10, xmin=-10,
            axis lines =center, xlabel=$h$, ylabel={$y=Q(h)$}, grid = major, grid style={dashed},
            ytick={-10,-8,-6,-4,-2,2,4,6,8,10},
            xtick={-10,-8,-6,-4,-2,2,4,6,8,10},
            yticklabels={$-10$,$-8$,$-6$,$-4$,$-2$,$2$,$4$,$6$,$8$,$10$}, 
            xticklabels={$-10$,$-8$,$-6$,$-4$,$-2$,$2$,$4$,$6$,$8$,$10$},
            ticklabel style={font=\scriptsize},
            every axis y label/.style={at=(current axis.above origin),anchor=south},
            every axis x label/.style={at=(current axis.right of origin),anchor=west},
            axis on top
          ]
          
            
      		\addplot [line width=2, penColor, smooth,samples=200,domain=(-2.5:5.5),<->] {(x+1)*(x-4)};








  \end{axis}
\end{tikzpicture}
\end{image}



From left to right, the range values, or function values, for $Q$ begin very big and positive.  These values decrease to $0$ and continue negative until they reach a value of $-6.25$. Then, increase to $0$ again and continue to very big and positive values.

Now we will transform these function values with a linear function.


Let $L(t) = -\frac{1}{4} t + 3$ with domain $\mathbb{R}$.


\begin{itemize}
\item $L$ will take a function value from $Q$ and compress it by a factor of $\frac{1}{4}$. Our parabola will be squished vertically a bit. 
\item Then $L$ will negate the values.  This will vertically reflect the parabola over the horizontal axis.
\item Then $L$ will add $3$ to all of the values.  This will shift the parabola up by $3$.
\end{itemize}












\begin{image}
\begin{tikzpicture}
  \begin{axis}[
            domain=-10:10, ymax=10, xmax=10, ymin=-10, xmin=-10,
            axis lines =center, xlabel=$h$, ylabel={$y=(L \circ Q)(h)$}, grid = major, grid style={dashed},
            ytick={-10,-8,-6,-4,-2,2,4,6,8,10},
            xtick={-10,-8,-6,-4,-2,2,4,6,8,10},
            yticklabels={$-10$,$-8$,$-6$,$-4$,$-2$,$2$,$4$,$6$,$8$,$10$}, 
            xticklabels={$-10$,$-8$,$-6$,$-4$,$-2$,$2$,$4$,$6$,$8$,$10$},
            ticklabel style={font=\scriptsize},
            every axis y label/.style={at=(current axis.above origin),anchor=south},
            every axis x label/.style={at=(current axis.right of origin),anchor=west},
            axis on top
          ]
          
            
      		\addplot [line width=2, penColor, smooth,samples=200,domain=(-6:9),<->] {-0.25*(x+1)*(x-4) + 3};








  \end{axis}
\end{tikzpicture}
\end{image}

The vertical measurements have all been processed linearly, which means the shape doesn't change.  It is still a parabola and all of its features are relatively in the same place.



The minimum of $Q$ is $-6.25$ and this occurred at $1.5$. We are flipping vertically, so the maximum of the composition is still at $1.5$. There were no horizontal transformations.  The maximum is $L(-6.25) = 4.5625$.




The zeros of $L \circ Q$ occur when $L(t) = 0$, which is at $t = 12$.  Therefore, the zeros of $L \circ Q$ occur when $Q(h) = 12$.





$Q(h) = (h+1)(h-4)$



\begin{align*}
Q(h) = (h+1)(h-4)     &  = 12  \\
h^2 - 3h - 4      & = 12   \\
h^2 - 3h - 16 = 0  \\
\end{align*}

This does not factor easily.  We'll use the quadratic formula.


\[  h = \frac{-(-3) \pm \sqrt{(-3)^2 - 4 \cdot 1 \cdot (-16)}}{2 \cdot 1}  = \frac{3 \pm \sqrt{73}}{2}     \]


We have two zeros: $\frac{3 + \sqrt{73}}{2}  \approx 5.77$ and $\frac{3 - \sqrt{73}}{2} \approx -2.77$, which agrees with our graph.



\begin{claim} Zeros


\[   Q(h) = (h+1)(h-4)   \]


\[   L(Q(z)) = -\frac{1}{4} (z+1)(z-4) + 3   \]


Let's verify that $\frac{3 + \sqrt{73}}{2}$ is a zero of $L(Q(z))$.



\[   L \left( Q \left( \frac{3 + \sqrt{73}}{2} \right) \right) = -\frac{1}{4} \left( \frac{3 + \sqrt{73}}{2}+1 \right) \left( \frac{3 + \sqrt{73}}{2}-4 \right) + 3    \]



\[  = -\frac{1}{4} \left( \frac{9 + 6 \sqrt{73} + 73}{4} - 3 \cdot \frac{3 + \sqrt{73}}{2} - 4 \right) + 3    \]


\[  = -\frac{1}{4} \left( \frac{82 + 6 \sqrt{73}}{4} - \frac{9 + 3 \sqrt{73}}{2} - 4 \right) + 3    \]


\[  = -\frac{1}{4} \left( \frac{82 + 6 \sqrt{73}}{4} - \frac{18 + 6 \sqrt{73}}{4} - 4 \right) + 3    \]


\[  = -\frac{1}{4} \left( \frac{48}{4} \right) + 3    \]


\[  = -3 + 3   = 0 \]


\end{claim}














\begin{claim} Zeros


\[   Q(h) = (h+1)(h-4)   \]


\[   L(Q(z)) = -\frac{1}{4} (z+1)(z-4) + 3   \]


Let's verify that $\frac{3 + \sqrt{73}}{2}$ is a zero of $L(Q(z))$.



\[   L \left( Q \left( \frac{3 - \sqrt{73}}{2} \right) \right) = -\frac{1}{4} \left( \frac{3 - \sqrt{73}}{2}+1 \right) \left( \frac{3 + \sqrt{73}}{2}-4 \right) + 3    \]



\[  = -\frac{1}{4} \left( \frac{\answer{9 - 6 \sqrt{73} + 73}}{4} - 3 \cdot \frac{3 - \sqrt{73}}{2} - 4 \right) + 3    \]


\[  = -\frac{1}{4} \left( \frac{82 - 6 \sqrt{73}}{4} - \frac{9 - 3 \sqrt{73}}{2} - 4 \right) + 3    \]


\[  = -\frac{1}{4} \left( \frac{82 - 6 \sqrt{73}}{4} - \frac{18 - \answer{6 \sqrt{73}}}{4} - 4 \right) + 3    \]


\[  = -\frac{1}{4} \left( \frac{48}{4} \right) + 3    \]


\[  = -3 + 3   = 0 \]


\end{claim}

















\end{document}
