\documentclass{ximera}


\graphicspath{
  {./}
  {ximeraTutorial/}
  {basicPhilosophy/}
}

\newcommand{\mooculus}{\textsf{\textbf{MOOC}\textnormal{\textsf{ULUS}}}}

\usepackage{tkz-euclide}\usepackage{tikz}
\usepackage{tikz-cd}
\usetikzlibrary{arrows}
\tikzset{>=stealth,commutative diagrams/.cd,
  arrow style=tikz,diagrams={>=stealth}} %% cool arrow head
\tikzset{shorten <>/.style={ shorten >=#1, shorten <=#1 } } %% allows shorter vectors

\usetikzlibrary{backgrounds} %% for boxes around graphs
\usetikzlibrary{shapes,positioning}  %% Clouds and stars
\usetikzlibrary{matrix} %% for matrix
\usepgfplotslibrary{polar} %% for polar plots
\usepgfplotslibrary{fillbetween} %% to shade area between curves in TikZ
\usetkzobj{all}
\usepackage[makeroom]{cancel} %% for strike outs
%\usepackage{mathtools} %% for pretty underbrace % Breaks Ximera
%\usepackage{multicol}
\usepackage{pgffor} %% required for integral for loops



%% http://tex.stackexchange.com/questions/66490/drawing-a-tikz-arc-specifying-the-center
%% Draws beach ball
\tikzset{pics/carc/.style args={#1:#2:#3}{code={\draw[pic actions] (#1:#3) arc(#1:#2:#3);}}}



\usepackage{array}
\setlength{\extrarowheight}{+.1cm}
\newdimen\digitwidth
\settowidth\digitwidth{9}
\def\divrule#1#2{
\noalign{\moveright#1\digitwidth
\vbox{\hrule width#2\digitwidth}}}






\DeclareMathOperator{\arccot}{arccot}
\DeclareMathOperator{\arcsec}{arcsec}
\DeclareMathOperator{\arccsc}{arccsc}

















%%This is to help with formatting on future title pages.
\newenvironment{sectionOutcomes}{}{}


\title{Graph Movement}

\begin{document}

\begin{abstract}
vertical
\end{abstract}
\maketitle













Let $L(t) = -(t+3) - 2$.


Let $K(x)$ be a piecewise defined function defined by 


\[
K(x) = 
\begin{cases}
  2|x+6| - 4         &    \,     \text{ on } \,   [-8,-3)    \\
  -6               &    \,     \text{ on } \,    [-3,1)      \\
  -\frac{3}{2}(x-4)(x-8)    &   \,     \text{ on } \,    (4,8]
\end{cases}
\]



Let $C(z) = (L \circ K)(z) = L(K(z))$ with the implied domain.



\begin{image}
\begin{tikzpicture}
  \begin{axis}[
            domain=-10:10, ymax=10, xmax=10, ymin=-10, xmin=-10,
            axis lines =center, xlabel=$x$, ylabel={$y=K(x)$}, grid = major,
            ytick={-10,-8,-6,-4,-2,2,4,6,8,10},
            xtick={-10,-8,-6,-4,-2,2,4,6,8,10},
            yticklabels={$-10$,$-8$,$-6$,$-4$,$-2$,$2$,$4$,$6$,$8$,$10$}, 
            xticklabels={$-10$,$-8$,$-6$,$-4$,$-2$,$2$,$4$,$6$,$8$,$10$},
            ticklabel style={font=\scriptsize},
            every axis y label/.style={at=(current axis.above origin),anchor=south},
            every axis x label/.style={at=(current axis.right of origin),anchor=west},
            axis on top
          ]
          
          %\addplot [line width=2, penColor2, smooth,samples=100,domain=(-6:2)] {-2*x-3};
            \addplot [line width=2, penColor, smooth,samples=100,domain=(-8:-3)] {2*abs(x+6)-4};
            \addplot[color=penColor,fill=penColor,only marks,mark=*] coordinates{(-8,0)};
          	\addplot[color=penColor,fill=white,only marks,mark=*] coordinates{(-3,2)};

            \addplot [line width=2, penColor, smooth,samples=100,domain=(-3:1)] {-6};
            \addplot[color=penColor,fill=penColor,only marks,mark=*] coordinates{(-3,-6)};
            \addplot[color=penColor,fill=white,only marks,mark=*] coordinates{(1,-6)};

            \addplot [line width=2, penColor, smooth,samples=100,domain=(4:8)] {-1.5*(x-4)*(x-8)};
            \addplot[color=penColor,fill=white,only marks,mark=*] coordinates{(4,0)};
            \addplot[color=penColor,fill=penColor,only marks,mark=*] coordinates{(8,0)};



  \end{axis}
\end{tikzpicture}
\end{image}




What is $L$ going to do to this graph?

$L$ is a linear function, which means it is not going to change the shapes.  There will be a "Vee".  There will be a horizontal line segment.  There will be a parabola.  $L$ is the outside function, which means it cannot affect horizontal features.  The "Vee" will be on the left. The parabola will be on the right.  The horizontal line segment will be in the middle.

$L$ is going to affect the vertical measurements. \\


\begin{itemize}
\item The leading coefficient for $L$ is negative.  The graph is going to be reflected vertically. \\
The "Vee" will open down.  The parabola will open up.
\item This reflecting will happen after the graph is move $3$ up.
\item Finally, the graph will be shifted down $2$.

\end{itemize}






Graph of $ y = (L \circ K)(m)$.







\begin{image}
\begin{tikzpicture}
  \begin{axis}[
            domain=-10:10, ymax=10, xmax=10, ymin=-11, xmin=-10,
            axis lines =center, xlabel=$m$, ylabel=$y$, grid = major,
            ytick={-10,-8,-6,-4,-2,2,4,6,8,10},
            xtick={-10,-8,-6,-4,-2,2,4,6,8,10},
            yticklabels={$-10$,$-8$,$-6$,$-4$,$-2$,$2$,$4$,$6$,$8$,$10$}, 
            xticklabels={$-10$,$-8$,$-6$,$-4$,$-2$,$2$,$4$,$6$,$8$,$10$},
            ticklabel style={font=\scriptsize},
            every axis y label/.style={at=(current axis.above origin),anchor=south},
            every axis x label/.style={at=(current axis.right of origin),anchor=west},
            axis on top
          ]
          
          %\addplot [line width=2, penColor2, smooth,samples=100,domain=(-6:2)] {-2*x-3};
            \addplot [line width=2, penColor, smooth,samples=100,domain=(-8:-3)] {-(2*abs(x+6)-4+3)-2};
            \addplot[color=penColor,fill=penColor,only marks,mark=*] coordinates{(-8,-5)};
          	\addplot[color=penColor,fill=white,only marks,mark=*] coordinates{(-3,-7)};

            \addplot [line width=2, penColor, smooth,samples=100,domain=(-3:1)] {-(-6+3)-2};
            \addplot[color=penColor,fill=penColor,only marks,mark=*] coordinates{(-3,1)};
            \addplot[color=penColor,fill=white,only marks,mark=*] coordinates{(1,1)};

            \addplot [line width=2, penColor, smooth,samples=100,domain=(4:8)] {-(-1.5*(x-4)*(x-8)+3)-2};
            \addplot[color=penColor,fill=white,only marks,mark=*] coordinates{(4,-5)};
            \addplot[color=penColor,fill=penColor,only marks,mark=*] coordinates{(8,-5)};



  \end{axis}
\end{tikzpicture}
\end{image}





\textbf{Note:} All of the endpoints are the same type.  

\begin{itemize}
\item The short arm of the "Vee" is a solid dot on both graphs.
\item The long arm of the "Vee" is a hollow dot on both graphs.
\item The end of the horizontal line segment nearest the "Vee" is a solid dot on both graphs.
\item The end of the horizontal line segment nearest the parabola is a hollow dot on both graphs.
\item The inside endpoint on the parabola is hollow on both graphs.
\item The outside endpoint on the parabola is solid on both graphs.
\end{itemize}



We can also think of $(L \circ K)(m)$ algebraically:


\[
(L \circ K)(m) = -(K(m)+3) - 2
\]


We can think of an existing value of $K(m)$.  $3$ is added to this value. The result is negated and then $2$ is subtracted. \\

Graphically, the coordinates of a point begin as $(m, K(m))$.  Then they go through the same steps: 


\begin{itemize}
\item $(m, K(m))$
\item $(m, K(m) + 3)$
\item $(m, -(K(m) + 3))$
\item $(m, -(K(m) + 3) - 2)$
\end{itemize}


For example, the endpoint $(-3, -6)$ goes through these transformational steps.

\begin{itemize}
\item $(-3, -6)$
\item $\left( \answer{-3},  \answer{-3} \right)$
\item $\left( \answer{-3},  \answer{3} \right)$
\item $\left( \answer{-3},  \answer{1} \right)$
\end{itemize}

The solid endpoint of the horziontal line segment becomes $(-3, 1)$.



For example, the corner point $(-6, -4)$ goes through these transformational steps.

\begin{itemize}
\item $(-6, -4)$
\item $\left( \answer{-6},  \answer{-1} \right)$
\item $\left( \answer{-6},  \answer{1} \right)$
\item $\left( \answer{-6},  \answer{-1} \right)$
\end{itemize}

The solid endpoint of the horziontal line segment becomes $(-6, -1)$.







\end{document}
