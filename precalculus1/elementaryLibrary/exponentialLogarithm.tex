\documentclass{ximera}


\graphicspath{
  {./}
  {ximeraTutorial/}
  {basicPhilosophy/}
}

\newcommand{\mooculus}{\textsf{\textbf{MOOC}\textnormal{\textsf{ULUS}}}}

\usepackage{tkz-euclide}\usepackage{tikz}
\usepackage{tikz-cd}
\usetikzlibrary{arrows}
\tikzset{>=stealth,commutative diagrams/.cd,
  arrow style=tikz,diagrams={>=stealth}} %% cool arrow head
\tikzset{shorten <>/.style={ shorten >=#1, shorten <=#1 } } %% allows shorter vectors

\usetikzlibrary{backgrounds} %% for boxes around graphs
\usetikzlibrary{shapes,positioning}  %% Clouds and stars
\usetikzlibrary{matrix} %% for matrix
\usepgfplotslibrary{polar} %% for polar plots
\usepgfplotslibrary{fillbetween} %% to shade area between curves in TikZ
\usetkzobj{all}
\usepackage[makeroom]{cancel} %% for strike outs
%\usepackage{mathtools} %% for pretty underbrace % Breaks Ximera
%\usepackage{multicol}
\usepackage{pgffor} %% required for integral for loops



%% http://tex.stackexchange.com/questions/66490/drawing-a-tikz-arc-specifying-the-center
%% Draws beach ball
\tikzset{pics/carc/.style args={#1:#2:#3}{code={\draw[pic actions] (#1:#3) arc(#1:#2:#3);}}}



\usepackage{array}
\setlength{\extrarowheight}{+.1cm}
\newdimen\digitwidth
\settowidth\digitwidth{9}
\def\divrule#1#2{
\noalign{\moveright#1\digitwidth
\vbox{\hrule width#2\digitwidth}}}






\DeclareMathOperator{\arccot}{arccot}
\DeclareMathOperator{\arcsec}{arcsec}
\DeclareMathOperator{\arccsc}{arccsc}

















%%This is to help with formatting on future title pages.
\newenvironment{sectionOutcomes}{}{}


\title{Exponential Logarithmic}

\begin{document}

\begin{abstract}
percentage growth
\end{abstract}
\maketitle




The defining characteristic of linear functions is that the have a constant growth rate.


\[   \frac{f(b)-f(a)}{b-a} = m       \]

This led to the equation or formula for linear functions:  $f(x) = m(x-a) + f(a)$





Exponential functions are similar, but it is their percentage growth rate that is constant.  

The growth of $f(x)$ over the interval $[a, b]$ is $f(b)-f(a)$. To get a percentage, we compare this back to $f(a)$: 

\[      \frac{f(b)-f(a)}{f(a)}    \]

And, then to get the percetnage growth rate we average over the interval



\[      \frac{\frac{f(b)-f(a)}{f(a)}}{b-a}    \]



FOr an exponential function, this is constant


\[      \frac{\frac{f(b)-f(a)}{f(a)}}{b-a}  = r  \]





Let's build such a function up from a given value of $f(0)$. Moving from $0$ to $1$ gives

\[      \frac{\frac{f(1)-f(0)}{f(0)}}{1-0}  = r  \]


\[      f(1)-f(0) = r f(0)  \]


\[      f(1) = r f(0) + f(0)  \]

\[      f(1) =  f(0) (r + 1)  \]



Moving from $1$ to $2$ gives

\[      f(2) =  f(0) (r + 1)^2  \]


Moving from $2$ to $3$ gives

\[      f(3) =  f(0) (r + 1)^3  \]


In general, exponential functions look like


\[      f(x) = f(0) a^x   \]



\section{Exponential Functions}


\begin{definition}

Exponential functions are those functions that exhibit a constant percentage rate of change.  Their formulas look like


\[      f(x) = k a^x   \]

where $k$ is a nonzero real number, and $a$ is a positive real number.


\end{definition}


There are two types of exponential functions that correspond to $0<a<1$ and $1<a$.






\begin{itemize}
\item If $0<a<1$, then greater positive exponents make the value smaller.   A negative exponent essentially gives us the reciprocal of $a$, which would be greater than $1$.  Therefore, greater negative exponents result in greater postive function values.
\item If $1<a$, then greater positive exponents make the value greater.   A negative exponent essentially gives us the reciprocal of $a$, which would be less than $1$.  Therefore, greater negative exponents result in smaller postive function values.
\end{itemize}



The graphs of $y = Y(x) = 3 \left(\frac{1}{2}\right)^x$ and $z = W(t) = 3 \cdot 2^t$




\begin{image}
\begin{tikzpicture}
  \begin{axis}[name = leftgraph, 
            domain=-10:10, ymax=10, xmax=10, ymin=-10, xmin=-10,
            axis lines =center, xlabel=$x$, ylabel=$y$,
            every axis y label/.style={at=(current axis.above origin),anchor=south},
            every axis x label/.style={at=(current axis.right of origin),anchor=west},
            axis on top
          ]
          
          \addplot [line width=2, penColor, smooth, samples=200, domain=(-1.5:9),<->] {03*(0.5^x)};
   

  \end{axis}
  \begin{axis}[at={(leftgraph.outer east)},anchor=outer west, 
            domain=-10:10, ymax=10, xmax=10, ymin=-10, xmin=-10,
            axis lines =center, xlabel=$t$, ylabel=$z$,
            every axis y label/.style={at=(current axis.above origin),anchor=south},
            every axis x label/.style={at=(current axis.right of origin),anchor=west},
            axis on top
          ]
          
          \addplot [line width=2, penColor, smooth, samples=200, domain=(-9:2.1),<->] {2*(2^x)};


  \end{axis}



\end{tikzpicture}
\end{image}






There is no vertical asymptote.  The domain of exponential functions is all real numbers.  $y=0$ is a horizontal asymptote on both graphs. Exponential functions can never be negative.


Since these formulas are centered around the exponent, they follow the exponent rules:



\begin{itemize}
\item $a^n \cdot a^m = a^{n+m}$

\item $\frac{a^n}{a^m} = a^{n-m}$

\item $(a^n)^m = a^{n \cdot m}$

\item $a^n \cdot b^n = (a \cdot b)^n$

\item $\frac{a^n}{b^n} = \left(\frac{a}{b}\right)^n$


\end{itemize}




\begin{example}

Let $T(f) = 4 3^f$.  Evaluate the following.

\begin{itemize}
\item $T(0) = \answer{4}$ 
\item $T(1) = \answer{12}$
\item $T(-1) = \answer{\frac{4}{3}}$
\end{itemize}
\end{example}








\section{Backwards}

What if we would like to go in reverse?  What if we have a function value for an exponential function and we would like to know which domain numbers are associated with it.  We would like to solve


\[    k \cdot a^t  =  t_0     \]


How would we solve for $t$?






\begin{example}

Let $T(f) = 4 3^f$.  If the function value "worked well", thenm we could probably guess.


Solve $T(f) = 36$

$4 \cdot 3^f = 36$

$3^f = 9$

$f = 2$

\end{example}









Most function values are not going to be so obvious. Solve $ 3^f = 17$.


We may not be able to quickly think up this number or even an approximaiton for it.  However, we can still talk about.

\begin{center}
We are looking for the number that you raise $3$ to, to get $17$.
\end{center}


That is a specific number. We can see from the graph that there is only one such number and we could visually approximate is around $2.5$.


\begin{example}
The following are all descriptions that identify unique real numbers.

\begin{itemize}
\item The number that you raise $5$ to, to get $97$.
\item The number that you raise $\frac{3}{4}$ to, to get $6$.
\item The number that you raise $7$ to, to get $\frac{1}{2}$.
\item The number that you raise $101$ to, to get $34$.
\item The number that you raise $10$ to, to get $1,000$.
\end{itemize}

\end{example}



As with all mathematical phrases, we have shorthand notation for these desciptions.








\begin{definition}

Let $a$ and $b$ be positive real numbers.  The number  you raise $a$ to, to get $b$ is called the \textbf{logarithm base a of b}.

The symbol for the logarithm base a of b is $log_a(b)$.



\[     a^{log_a(b)} = b          \]




\end{definition}





\begin{example}
The following are all descriptions that identify unique real numbers.

\begin{itemize}
\item The number that you raise $5$ to, to get $97$ is $log_5(97)$.
\item The number that you raise $\frac{3}{4}$ to, to get $6$ is $log_{\tfrac{3}{4}}(6)$.
\item The number that you raise $7$ to, to get $\frac{1}{2}$ is $log_7\left(\frac{1}{2}\right)$.
\item The number that you raise $101$ to, to get $34$ is $log_{101}(34)$.
\item The number that you raise $10$ to, to get $1,000$ is $log_{10}(1000)$.
\end{itemize}

\end{example}



\begin{example}
Evaluate the following expressions

\begin{itemize}
\item  $3^{log_3{56}} = \answer{56}$
\item  $13^{log_{13}{21}} = \answer{21}$
\item  $\pi^{log_{\pi}{82}} = \answer{82}$
\item  $4^{log_4{\sqrt{7}}} = \answer{\sqrt{7}}$
\item  $85^{log_{85}{2}} = \answer{2}$

\end{itemize}

\end{example}












Logarithms are exponents.  They can be positive or negative.



$9^{-2} = \frac{1}{81}$, therefore  $log_{9}\left(\frac{1}{81}\right) = -2$



We also now that raising a positve number to any exponent cannot produce a negative number or $0$.  Therefore, the number \textit{inside} the logarithm must be positive



It sounds like we have a new category of functions.




\begin{definition} \textit{Logarithmic Functions}

Give a positive \textbf{base}, $a$, a \textbf{Logarithmic Function} is a function that can be represented by formulas of the form

\[     L(x) =    log_a(x)            \]

The domain is positive real numbers and the range is all real numbers.

\end{definition}








\begin{example}

Here is the graph of $y = L(x) = log_2(x)$.

\begin{image}
\begin{tikzpicture} 
  \begin{axis}[
            domain=-10:10, ymax=10, xmax=10, ymin=-10, xmin=-10,
            axis lines =center, xlabel=$x$, ylabel=$y$,
            every axis y label/.style={at=(current axis.above origin),anchor=south},
            every axis x label/.style={at=(current axis.right of origin),anchor=west},
            axis on top
          ]
          
          \addplot [line width=2, penColor, smooth,samples=200,domain=(0:9),<->] {ln(x)/ln(2)};
          \addplot [line width=1, gray, dashed,domain=(-9:9),<->] ({0},{x});

           

  \end{axis}
\end{tikzpicture}
\end{image}


The intercept is $(1,0)$, because $log_2(1) = 0$, because $2^0 = 1$.

On the interval $(0,1)$, we are looking at $log_2(x)$ for $0<x<1$.  Remember, $log_2(x)$ is the number you raise $2$ to,m to get $x$, but here $0<x<1$.  Therefore, $2$ needs a negative exponent.  And, the smaller (closer to $0$) you want $x$, the bigger negative exponent.





\end{example}



If we switch the base from something greater than $1$, to something less than $1$, then all of the exponents flip.  The graph flips.






\begin{example}

Here is the graph of $y = L(x) = log_{\tfrac{1}{2}}(x)$.

\begin{image}
\begin{tikzpicture} 
  \begin{axis}[
            domain=-10:10, ymax=10, xmax=10, ymin=-10, xmin=-10,
            axis lines =center, xlabel=$x$, ylabel=$y$,
            every axis y label/.style={at=(current axis.above origin),anchor=south},
            every axis x label/.style={at=(current axis.right of origin),anchor=west},
            axis on top
          ]
          
          \addplot [line width=2, penColor, smooth,samples=200,domain=(0:9),<->] {ln(x)/ln(0.5)};
          \addplot [line width=1, gray, dashed,domain=(-9:9),<->] ({0},{x});

           

  \end{axis}
\end{tikzpicture}
\end{image}


The intercept is $(1,0)$, because $log_{\tfrac{1}{2}}(1) = 0$, because $\left(\frac{1}{2}\right)^0 = 1$.

On the interval $(0,1)$, we are looking at $log_{\tfrac{1}{2}}(x)$ for $0<x<1$.  Now we just need large positive exponents of $\frac{1}{2}$ to get small numbers.  On the other hand, to get large positive numbers we need to rais $\frac{1}{2}$ to negaitve powers.





\end{example}







\end{document}
