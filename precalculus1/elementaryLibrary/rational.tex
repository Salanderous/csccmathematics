\documentclass{ximera}


\graphicspath{
  {./}
  {ximeraTutorial/}
  {basicPhilosophy/}
}

\newcommand{\mooculus}{\textsf{\textbf{MOOC}\textnormal{\textsf{ULUS}}}}

\usepackage{tkz-euclide}\usepackage{tikz}
\usepackage{tikz-cd}
\usetikzlibrary{arrows}
\tikzset{>=stealth,commutative diagrams/.cd,
  arrow style=tikz,diagrams={>=stealth}} %% cool arrow head
\tikzset{shorten <>/.style={ shorten >=#1, shorten <=#1 } } %% allows shorter vectors

\usetikzlibrary{backgrounds} %% for boxes around graphs
\usetikzlibrary{shapes,positioning}  %% Clouds and stars
\usetikzlibrary{matrix} %% for matrix
\usepgfplotslibrary{polar} %% for polar plots
\usepgfplotslibrary{fillbetween} %% to shade area between curves in TikZ
\usetkzobj{all}
\usepackage[makeroom]{cancel} %% for strike outs
%\usepackage{mathtools} %% for pretty underbrace % Breaks Ximera
%\usepackage{multicol}
\usepackage{pgffor} %% required for integral for loops



%% http://tex.stackexchange.com/questions/66490/drawing-a-tikz-arc-specifying-the-center
%% Draws beach ball
\tikzset{pics/carc/.style args={#1:#2:#3}{code={\draw[pic actions] (#1:#3) arc(#1:#2:#3);}}}



\usepackage{array}
\setlength{\extrarowheight}{+.1cm}
\newdimen\digitwidth
\settowidth\digitwidth{9}
\def\divrule#1#2{
\noalign{\moveright#1\digitwidth
\vbox{\hrule width#2\digitwidth}}}






\DeclareMathOperator{\arccot}{arccot}
\DeclareMathOperator{\arcsec}{arcsec}
\DeclareMathOperator{\arccsc}{arccsc}

















%%This is to help with formatting on future title pages.
\newenvironment{sectionOutcomes}{}{}


\title{Rational}

\begin{document}

\begin{abstract}
fractions of polynomials
\end{abstract}
\maketitle




Rational functions are fractions of polynomials


\begin{definition}\textbf{\textcolor{green!50!black}{Rational Functions}}

A rational function is any function that can be represented as a quotient of two polynomials 

\[   \frac{ a_n x^n + a_{n-1} x^{n-1} + \cdots + a_3 x^3 + a_2 x^2 + a_1 x + a_0  } { b_m x^m + b_{m-1} x^{m-1} + \cdots + b_3 x^3 + b_2 x^2 + b_1 x + b_0 }   \]



where the $a_k$ and $b_k$ are real numbers and $a_n \ne 0$ and $b_m \ne 0$.





\end{definition}




Again, we prefer polynomials in factored form.  Therefore, usually, our first step is to transform the formula for the rational function to look like



\[   \frac{ (x-r_n)(x-r_{n-1})  \cdots (x-r_2)(x-r_1)  } { (x-s_m)(x-s_{m-1})  \cdots (x-s_2)(x-s_1) }   \]




And, again, we will be able to obtain a product of exclusively linear factors with the addition of complex numbers.  With real numbers we can get these products to consist of a mixture of linear and irreducible quadratics.










\begin{example}


The graph of $y = H(w) = \frac{(w-1)}{(w+3) (w-4)} $


\begin{image}
\begin{tikzpicture} 
  \begin{axis}[
            domain=-10:10, ymax=10, xmax=10, ymin=-10, xmin=-10,
            axis lines =center, xlabel=$w$, ylabel=$y$,
            ytick={-10,-8,-6,-4,-2,2,4,6,8,10},
            xtick={-10,-8,-6,-4,-2,2,4,6,8,10},
            ticklabel style={font=\scriptsize},
            every axis y label/.style={at=(current axis.above origin),anchor=south},
            every axis x label/.style={at=(current axis.right of origin),anchor=west},
            axis on top
          ]

          \addplot [line width=1, gray, dashed, domain=(-9.5:9.5),<->] ({-3},{x});
          \addplot [line width=1, gray, dashed, domain=(-9.5:9.5),<->] ({4},{x});

          \addplot [line width=2, penColor, smooth, domain=(-9:-3.1),<->] {(x-1)/((x+3)*(x-4))};
          \addplot [line width=2, penColor, smooth, domain=(-2.9:3.9),<->] {(x-1)/((x+3)*(x-4))};
          \addplot [line width=2, penColor, smooth, domain=(4.1:9),<->] {(x-1)/((x+3)*(x-4))};


  \end{axis}
\end{tikzpicture}
\end{image}

\end{example}








\begin{example}


The graph of $y = v(r) = \frac{(r-1)}{(r+3) (r-4)^2} $


\begin{image}
\begin{tikzpicture} 
  \begin{axis}[
            domain=-10:10, ymax=10, xmax=10, ymin=-10, xmin=-10,
            axis lines =center, xlabel=$r$, ylabel=$y$,
            ytick={-10,-8,-6,-4,-2,2,4,6,8,10},
            xtick={-10,-8,-6,-4,-2,2,4,6,8,10},
            ticklabel style={font=\scriptsize},
            every axis y label/.style={at=(current axis.above origin),anchor=south},
            every axis x label/.style={at=(current axis.right of origin),anchor=west},
            axis on top
          ]

          \addplot [line width=1, gray, dashed, domain=(-9.5:9.5),<->] ({-3},{x});
          \addplot [line width=1, gray, dashed, domain=(-9.5:9.5),<->] ({4},{x});

          \addplot [line width=2, penColor, smooth, samples=200, domain=(-9:-3.02),<->] {(x-1)/((x+3)*(x-4)^2)};
          \addplot [line width=2, penColor, smooth, samples=200, domain=(-2.98:3.77),<->] {(x-1)/((x+3)*(x-4)^2)};
          \addplot [line width=2, penColor, smooth, samples=200, domain=(4.25:9),<->] {(x-1)/((x+3)*(x-4)^2)};


  \end{axis}
\end{tikzpicture}
\end{image}

\end{example}








While polynomials only combined factors with positive integer powers, rational functions include factors with negative integer powers.


We can pull $H(w)$ apart to see these power-like functions.


\[    H(w) = \frac{(w-1)}{(w+3) (w-4)}  = \frac{4}{7} \cdot \frac{1}{w+3}  + \frac{3}{7} \cdot \frac{1}{w-4}    \]


Not strictly speaking power functions, but more like shifted power functions.









\subsection{Continuity}


Rational functions are continuous everywhere on their domain.  They have singularities at zeros of their denominators.  

Graphs of rational functions might have asymptotes at these singularities, where the function grows (positively or negatively) without bound.  As we can see in the examples above, the sign of the function can switch across a singularity, or not.  This has to do with the power of the corresponding factor, which we will investigate later.













\begin{example}


The graph of $y = B(c) = \frac{(c-1)(c-5)}{5(c+3)} $


\begin{image}
\begin{tikzpicture} 
  \begin{axis}[
            domain=-20:20, ymax=20, xmax=20, ymin=-20, xmin=-20,
            axis lines =center, xlabel=$c$, ylabel=$y$,
            ytick={-20,-16,-12,-8,-4,4,8,12,16,20},
            xtick={-20,-16,-12,-8,-4,4,8,12,16,20},
            ticklabel style={font=\scriptsize},
            every axis y label/.style={at=(current axis.above origin),anchor=south},
            every axis x label/.style={at=(current axis.right of origin),anchor=west},
            axis on top
          ]


          \addplot [line width=1, gray, dashed, domain=(-19.5:19.5),<->] ({-3},{x});
          \addplot [line width=1, gray, dashed, domain=(-19.5:19.5),<->] {0.2*x-1.8};


          \addplot [line width=2, penColor, smooth, samples=200, domain=(-19:-3.4),<->] {((x-1)*(x-5))/(5*(x+3))};
          \addplot [line width=2, penColor, smooth, samples=200, domain=(-2.9:19),<->] {((x-1)*(x-5))/(5*(x+3))};



  \end{axis}
\end{tikzpicture}
\end{image}

\end{example}




In addition to horizontal and vertical asymptotes, rational functions can have \textbf{oblique} asymptotes, which describe the endbehavior.












The graph of a rational function need not have an asymptote at a zero of the denominator.


\begin{example}


The graph of $y = h(t) = \frac{(t-1)(t-5)}{(t-5)(t+3)} $ \\

$5$ is a zero of the polynomial in the denominator.  However, $t=5$ is not an asymptote in the graph.


\begin{image}
\begin{tikzpicture} 
  \begin{axis}[
            domain=-20:20, ymax=20, xmax=20, ymin=-20, xmin=-20,
            axis lines =center, xlabel=$t$, ylabel=$y$,
            ytick={-20,-16,-12,-8,-4,4,8,12,16,20},
            xtick={-20,-16,-12,-8,-4,4,8,12,16,20},
            ticklabel style={font=\scriptsize},
            every axis y label/.style={at=(current axis.above origin),anchor=south},
            every axis x label/.style={at=(current axis.right of origin),anchor=west},
            axis on top
          ]
 
          \addplot [line width=1, gray, dashed, domain=(-19.5:19.5),<->] ({-3},{x});
          \addplot [line width=1, gray, dashed, domain=(-19.5:19.5),<->] ({x},{1});

          \addplot [line width=2, penColor, smooth, samples=200, domain=(-19:-3.4),<->] {(x-1)/(x+3)};
          \addplot [line width=2, penColor, smooth, samples=200, domain=(-2.9:19),<->] {(x-1)/(x+3)};


          \addplot[color=penColor,fill=white,only marks,mark=*] coordinates{(5,0.5)}; 
                 

  \end{axis}
\end{tikzpicture}
\end{image}

\end{example}


This is because $5$ is also a zero of the polynomial in the numerator.


\[ y = h(t) = \frac{(t-1)(t-5)}{(t-5)(t+3)} = \frac{(t-1)}{(t+3)}  \, \text{ for } \,  t \ne 5  \]




\end{document}
