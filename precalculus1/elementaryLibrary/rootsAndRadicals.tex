\documentclass{ximera}


\graphicspath{
  {./}
  {ximeraTutorial/}
  {basicPhilosophy/}
}

\newcommand{\mooculus}{\textsf{\textbf{MOOC}\textnormal{\textsf{ULUS}}}}

\usepackage{tkz-euclide}\usepackage{tikz}
\usepackage{tikz-cd}
\usetikzlibrary{arrows}
\tikzset{>=stealth,commutative diagrams/.cd,
  arrow style=tikz,diagrams={>=stealth}} %% cool arrow head
\tikzset{shorten <>/.style={ shorten >=#1, shorten <=#1 } } %% allows shorter vectors

\usetikzlibrary{backgrounds} %% for boxes around graphs
\usetikzlibrary{shapes,positioning}  %% Clouds and stars
\usetikzlibrary{matrix} %% for matrix
\usepgfplotslibrary{polar} %% for polar plots
\usepgfplotslibrary{fillbetween} %% to shade area between curves in TikZ
\usetkzobj{all}
\usepackage[makeroom]{cancel} %% for strike outs
%\usepackage{mathtools} %% for pretty underbrace % Breaks Ximera
%\usepackage{multicol}
\usepackage{pgffor} %% required for integral for loops



%% http://tex.stackexchange.com/questions/66490/drawing-a-tikz-arc-specifying-the-center
%% Draws beach ball
\tikzset{pics/carc/.style args={#1:#2:#3}{code={\draw[pic actions] (#1:#3) arc(#1:#2:#3);}}}



\usepackage{array}
\setlength{\extrarowheight}{+.1cm}
\newdimen\digitwidth
\settowidth\digitwidth{9}
\def\divrule#1#2{
\noalign{\moveright#1\digitwidth
\vbox{\hrule width#2\digitwidth}}}






\DeclareMathOperator{\arccot}{arccot}
\DeclareMathOperator{\arcsec}{arcsec}
\DeclareMathOperator{\arccsc}{arccsc}

















%%This is to help with formatting on future title pages.
\newenvironment{sectionOutcomes}{}{}


\title{Roots and Radicals}

\begin{document}

\begin{abstract}
separating dimensions
\end{abstract}
\maketitle


While polynomials and rational functions only include terms with positive integer powers, power functions, by themselves, include any real number power.  The noninteger powers belong to a category called \textbf{roots and radicals}.



\begin{definition} \textbf{\textcolor{green!50!black}{Roots}}

$x^{\tfrac{1}{n}}$ is called \textbf{the nth root of x}, where $n$ is a natural number.




\textbf{odd roots}

When $n$ is odd, $x^{\tfrac{1}{n}}$ is defined to be the unique real number, $r$, such that $r^n = x$. The domain is all real numbers.




\textbf{even roots}

When $n$ is even, $x^{\tfrac{1}{n}}$ is defined to be the unique nonnegative real number, $r$, such that $r^n = x$. The domain is all nonnegative real numbers.


\end{definition}



\begin{definition} \textbf{\textcolor{green!50!black}{Radicals}}

An alternate point-of-view uses radical notation rather than exponents.

\[   \sqrt[n]{x} =  x^{\tfrac{1}{n}}     \]

where $\sqrt{ }$ is called the \textbf{radical} symbol.


$\sqrt[n]{x}$ is called the \textbf{nth root of x}.

\end{definition}




\subsection{Algebra Rules}

The nth-root is an exponent, so it follows all of the exponent rules.


\begin{itemize}
\item $a^n \cdot a^m = a^{n+m}$

\item $\frac{a^n}{a^m} = a^{n-m}$

\item $(a^n)^m = a^{n \cdot m}$

\item $a^n \cdot b^n = (a \cdot b)^n$

\item $\frac{a^n}{b^n} = \left(\frac{a}{b}\right)^n$


\end{itemize}



Similar rules for radicals.

\begin{itemize}

\item $\sqrt[n]{a} \cdot \sqrt[n]{b} = \sqrt[n]{a \cdot b}$

\item $\frac{\sqrt[n]{a}}{\sqrt[n]{b}} = \sqrt[n]{\frac{a}{b}}$

\item $\sqrt[m]{\sqrt[n]{a}} = \sqrt[nm]{a}$

\end{itemize}






\subsection{Graphs}



Negative numbers raised to odd powers result in negative numbers, which means every real number has an odd root.  The domain of odd roots or radicals is all real numbers.

\begin{example}


The graph of $y = R(t) = t^{\tfrac{1}{3}} = \sqrt[3]{t}$


\begin{image}
\begin{tikzpicture} 
  \begin{axis}[
            domain=-10:10, ymax=10, xmax=10, ymin=-10, xmin=-10,
            axis lines =center, xlabel=$t$, ylabel=$y$,
            ytick={-10,-8,-6,-4,-2,2,4,6,8,10},
            xtick={-10,-8,-6,-4,-2,2,4,6,8,10},
            ticklabel style={font=\scriptsize},
            every axis y label/.style={at=(current axis.above origin),anchor=south},
            every axis x label/.style={at=(current axis.right of origin),anchor=west},
            axis on top
          ]
          
          \addplot [line width=2, penColor, smooth, samples=200, domain=(0:9),->] {x^(0.333333)};
          \addplot [line width=2, penColor, smooth, samples=200, domain=(-9:0),<-] {-(-x)^(0.333333)};



           

  \end{axis}
\end{tikzpicture}
\end{image}

\end{example}





Negative numbers raised to even powers result in positive numbers, just like positive numbers raised to even power.  This makes going backwards a problem.  



Should $\sqrt{4} = -2$ or $\sqrt{4} = 2$?  When squared, both $-2$ and $2$ give $4$.



We are forced to choose between two even roots.  We pick positive numbers . The domain and range of even roots are both $[0, \infty)$.

\begin{example}


The graph of $y = R(t) = t^{\tfrac{1}{2}} = \sqrt{t} $


\begin{image}
\begin{tikzpicture} 
  \begin{axis}[
            domain=-10:10, ymax=10, xmax=10, ymin=-10, xmin=-10,
            axis lines =center, xlabel=$t$, ylabel=$y$,
            ytick={-10,-8,-6,-4,-2,2,4,6,8,10},
            xtick={-10,-8,-6,-4,-2,2,4,6,8,10},
            ticklabel style={font=\scriptsize},
            every axis y label/.style={at=(current axis.above origin),anchor=south},
            every axis x label/.style={at=(current axis.right of origin),anchor=west},
            axis on top
          ]
          
          \addplot [line width=2, penColor, smooth, samples=200, domain=(0:9),->] {x^(0.5)};
          \addplot[color=penColor,fill=penColor,only marks,mark=*] coordinates{(0,0)};




           

  \end{axis}
\end{tikzpicture}
\end{image}

\end{example}






There are no horizontal asymptotes on these graphs.  The graph keeps moving up or down unbounded.  The root and radical functions are unbounded. Their values tend to $\pm \infty$ as the domain moves out towards $\pm \infty$.











\begin{center}
\textbf{\textcolor{green!50!black}{ooooo=-=-=-=-=-=-=-=-=-=-=-=-=ooOoo=-=-=-=-=-=-=-=-=-=-=-=-=ooooo}} \\

more examples can be found by following this link\\ \link[More Examples of Elementary Functions]{https://ximera.osu.edu/csccmathematics/precalculus1/precalculus1/elementaryLibrary/examples/exampleList}

\end{center}



\end{document}
