\documentclass{ximera}


\graphicspath{
  {./}
  {ximeraTutorial/}
  {basicPhilosophy/}
}

\newcommand{\mooculus}{\textsf{\textbf{MOOC}\textnormal{\textsf{ULUS}}}}

\usepackage{tkz-euclide}\usepackage{tikz}
\usepackage{tikz-cd}
\usetikzlibrary{arrows}
\tikzset{>=stealth,commutative diagrams/.cd,
  arrow style=tikz,diagrams={>=stealth}} %% cool arrow head
\tikzset{shorten <>/.style={ shorten >=#1, shorten <=#1 } } %% allows shorter vectors

\usetikzlibrary{backgrounds} %% for boxes around graphs
\usetikzlibrary{shapes,positioning}  %% Clouds and stars
\usetikzlibrary{matrix} %% for matrix
\usepgfplotslibrary{polar} %% for polar plots
\usepgfplotslibrary{fillbetween} %% to shade area between curves in TikZ
\usetkzobj{all}
\usepackage[makeroom]{cancel} %% for strike outs
%\usepackage{mathtools} %% for pretty underbrace % Breaks Ximera
%\usepackage{multicol}
\usepackage{pgffor} %% required for integral for loops



%% http://tex.stackexchange.com/questions/66490/drawing-a-tikz-arc-specifying-the-center
%% Draws beach ball
\tikzset{pics/carc/.style args={#1:#2:#3}{code={\draw[pic actions] (#1:#3) arc(#1:#2:#3);}}}



\usepackage{array}
\setlength{\extrarowheight}{+.1cm}
\newdimen\digitwidth
\settowidth\digitwidth{9}
\def\divrule#1#2{
\noalign{\moveright#1\digitwidth
\vbox{\hrule width#2\digitwidth}}}






\DeclareMathOperator{\arccot}{arccot}
\DeclareMathOperator{\arcsec}{arcsec}
\DeclareMathOperator{\arccsc}{arccsc}

















%%This is to help with formatting on future title pages.
\newenvironment{sectionOutcomes}{}{}


\outcome{Consider limits as behavior nearer and nearer to a point.}


\title[Break-Ground:]{Limits and velocity}

\begin{document}
\begin{abstract}
Two young mathematicians discuss limits and instantaneous velocity.
\end{abstract}
\maketitle

Check out this dialogue between two calculus students (based on a true
story):

\begin{dialogue}
\item[Devyn] Hey Riley, I've been thinking about limits.
\item[Riley] That is awesome.
\item[Devyn] I know! You know limits remind me of something\dots
  How a GPS or a phone computes velocity!
\item[Riley] Huh.  A GPS can calculate our location.   Then, to compute velocity 
from position, it must look at
  \[
  \frac{\text{change in position}}{\text{change in time}}
  \]
\item[Devyn] And then we study this as the change in time gets closer
  and closer to zero.
\item[Riley] Just like with limits at zero, we can study something by
  looking \textbf{near} a point, but \textbf{not exactly at} a point.
\item[Devyn] O.M.G.\ Life's a rich tapestry.
\item[Riley] Poet, you know it.
\end{dialogue}



Suppose you take a road trip from Columbus Ohio to Urbana-Champaign
Illinois. Moreover, suppose your position is modeled by
\[
s(t) = 36t^2 -4.8t^3 \qquad\text{(miles West of Columbus)} %% note the model is wrong
\]
where $t$ is measured in hours and runs from $0$ to $5$ hours. 


\begin{problem}
  What is the average velocity for the entire trip?
  \begin{hint}
    Remember, 
    \[
    \text{change in distance} = \text{rate}\cdot\text{change in time}.
    \]
  \end{hint}
  \begin{hint}
    So, 
    \[
    \frac{\Delta\text{distance}}{\Delta\text{time}} = \text{rate}.
    \]
  \end{hint}
  \begin{hint}
    So, 
    \[
    \frac{\Delta\text{distance}}{\Delta\text{time}} = \frac{300}{5}.
    \]
  \end{hint}
  \begin{prompt}
    The average velocity is $\answer{60}$ miles per hour.
  \end{prompt}
\end{problem}


\begin{problem}
  Use a calculator to estimate the instantaneous velocity at $t=2$.
  \begin{hint}
    Remember, 
    \[
    \text{change in distance} = \text{rate}\cdot\text{change in time}.
    \]
  \end{hint}
  \begin{hint}
    So, 
    \[
    \frac{\Delta\text{distance}}{\Delta\text{time}} = \text{rate}.
    \]
  \end{hint}
  \begin{hint}
    Compute
    \[
    \frac{36 t^2 -4.8 t^3 -\left(36\cdot 2^2 -4.8\cdot 2^3\right) }{ t-2}
    \]
   for $t$  closer, and closer to $2$.
  \end{hint}
  \begin{prompt}
    The instantaneous velocity, (rounded to the nearest tenth) is $\answer{86.4}$ miles per hour.
  \end{prompt}
\end{problem}


\begin{problem}
  Considering the work above, when we want to compute instantaneous
  velocity, we need to compute
  \[
  \frac{\text{change in position}}{\text{change in time}}
  \]
  when (choose all that apply):
 \begin{selectAll}
    \choice{The ``change in time'' is zero.}
    \choice[correct]{The ``change in time'' gets closer and closer to zero.}
    \choice[correct]{The ``change in time'' approaches zero.}
    \choice{The ``change in time'' is near zero.}
    \choice[correct]{The ``change in time'' goes to zero.}
 \end{selectAll}
\end{problem}


Computing average velocities for smaller, and smaller, values of
$t-2$ as we did above is tedious. Nevertheless, this is exactly
how a GPS determines velocity from position! To avoid these tedious
calculations, we would really like to have a formula.


%\input{../leveledQuestions.tex}


\end{document}
