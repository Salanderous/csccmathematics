\documentclass{ximera}


\graphicspath{
  {./}
  {ximeraTutorial/}
  {basicPhilosophy/}
}

\newcommand{\mooculus}{\textsf{\textbf{MOOC}\textnormal{\textsf{ULUS}}}}

\usepackage{tkz-euclide}\usepackage{tikz}
\usepackage{tikz-cd}
\usetikzlibrary{arrows}
\tikzset{>=stealth,commutative diagrams/.cd,
  arrow style=tikz,diagrams={>=stealth}} %% cool arrow head
\tikzset{shorten <>/.style={ shorten >=#1, shorten <=#1 } } %% allows shorter vectors

\usetikzlibrary{backgrounds} %% for boxes around graphs
\usetikzlibrary{shapes,positioning}  %% Clouds and stars
\usetikzlibrary{matrix} %% for matrix
\usepgfplotslibrary{polar} %% for polar plots
\usepgfplotslibrary{fillbetween} %% to shade area between curves in TikZ
\usetkzobj{all}
\usepackage[makeroom]{cancel} %% for strike outs
%\usepackage{mathtools} %% for pretty underbrace % Breaks Ximera
%\usepackage{multicol}
\usepackage{pgffor} %% required for integral for loops



%% http://tex.stackexchange.com/questions/66490/drawing-a-tikz-arc-specifying-the-center
%% Draws beach ball
\tikzset{pics/carc/.style args={#1:#2:#3}{code={\draw[pic actions] (#1:#3) arc(#1:#2:#3);}}}



\usepackage{array}
\setlength{\extrarowheight}{+.1cm}
\newdimen\digitwidth
\settowidth\digitwidth{9}
\def\divrule#1#2{
\noalign{\moveright#1\digitwidth
\vbox{\hrule width#2\digitwidth}}}






\DeclareMathOperator{\arccot}{arccot}
\DeclareMathOperator{\arcsec}{arcsec}
\DeclareMathOperator{\arccsc}{arccsc}

















%%This is to help with formatting on future title pages.
\newenvironment{sectionOutcomes}{}{}


\author{Jim Talamo}

\outcome{Add and subtract vectors.}

\begin{document}
\begin{exercise}
Suppose that $\vec{u} = \vector{0,-1,3}$ and $\vec{v} =
\vector{-1,2,0}$.  Find a vector
$\vec{w}$  with a positive $y$-component of magnitude $7$ that is parallel to $\vec{u}+2\vec{v}$.

\[
\vec{w} = \vector{\answer{\frac{-14}{\sqrt{22}} }, \answer{ \frac{21}{\sqrt{22}} } , \answer{ \frac{21}{\sqrt{22}} }}
\]

\begin{hint}
To begin, let's find a vector in the same direction as $\vec{u}+2\vec{v}$.  Using the rules of addition and scalar multiplication, we find:

\[
\vec{u}+2\vec{v} = \vector{\answer{-2} , \answer{3} , \answer{3}}
\]

How should we proceed?

\begin{multipleChoice}
\choice{Multiply this result by $7$; that is, $\vec{w} = \vector{-14 , 21, 21}$.}
\choice[correct]{Find the magnitude of $\vector{-2, 3, 3}$ and scale it appropriately if necessary.}
\end{multipleChoice}

We compute:

\[
\left|\vec{u}+2\vec{v}\right| = \sqrt{\left( \answer{-2} \right)^2+\left(  \answer{3} \right)^2+\left(  \answer{3}\right)^2 }  = \sqrt{\answer{22}}
\]
(type the components in the order of $\vec{u}+2\vec{v}$)

A unit vector in the direction of $\vec{w}$ is thus $\frac{\vec{u}+2\vec{v}}{\left|\vec{u}+2\vec{v}\right|}$, so:

\[
\uvec{w} = \vector{\answer{\frac{-2}{\sqrt{22}} }, \answer{ \frac{3}{\sqrt{22}} } , \answer{ \frac{3}{\sqrt{22}} }}
\]

and  $\vec{w} = \answer{7} \uvec{w}$.
\end{hint}

\end{exercise}
\end{document}
