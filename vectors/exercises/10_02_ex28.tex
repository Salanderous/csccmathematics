\documentclass{ximera}


\graphicspath{
  {./}
  {ximeraTutorial/}
  {basicPhilosophy/}
}

\newcommand{\mooculus}{\textsf{\textbf{MOOC}\textnormal{\textsf{ULUS}}}}

\usepackage{tkz-euclide}\usepackage{tikz}
\usepackage{tikz-cd}
\usetikzlibrary{arrows}
\tikzset{>=stealth,commutative diagrams/.cd,
  arrow style=tikz,diagrams={>=stealth}} %% cool arrow head
\tikzset{shorten <>/.style={ shorten >=#1, shorten <=#1 } } %% allows shorter vectors

\usetikzlibrary{backgrounds} %% for boxes around graphs
\usetikzlibrary{shapes,positioning}  %% Clouds and stars
\usetikzlibrary{matrix} %% for matrix
\usepgfplotslibrary{polar} %% for polar plots
\usepgfplotslibrary{fillbetween} %% to shade area between curves in TikZ
\usetkzobj{all}
\usepackage[makeroom]{cancel} %% for strike outs
%\usepackage{mathtools} %% for pretty underbrace % Breaks Ximera
%\usepackage{multicol}
\usepackage{pgffor} %% required for integral for loops



%% http://tex.stackexchange.com/questions/66490/drawing-a-tikz-arc-specifying-the-center
%% Draws beach ball
\tikzset{pics/carc/.style args={#1:#2:#3}{code={\draw[pic actions] (#1:#3) arc(#1:#2:#3);}}}



\usepackage{array}
\setlength{\extrarowheight}{+.1cm}
\newdimen\digitwidth
\settowidth\digitwidth{9}
\def\divrule#1#2{
\noalign{\moveright#1\digitwidth
\vbox{\hrule width#2\digitwidth}}}






\DeclareMathOperator{\arccot}{arccot}
\DeclareMathOperator{\arcsec}{arcsec}
\DeclareMathOperator{\arccsc}{arccsc}

















%%This is to help with formatting on future title pages.
\newenvironment{sectionOutcomes}{}{}


\author{Gregory Hartman \and Bart Snapp}
\license{Creative Commons 3.0 By-NC}
\acknowledgement{https://github.com/APEXCalculus}

\outcome{Use vectors in applied settings.}

\begin{document}
\begin{exercise}
Consider a weight of $100\mathrm{lb}$ hanging from two chains:
  \begin{image}
    \begin{tikzpicture}
      \filldraw[thick,black,fill=gray!30] (-.5,0) -- (.5,0) -- (1,-1) -- (-1,-1)--cycle;
      \draw (0,-.5) node {$100\mathrm{lb}$};
      
      \draw [thick] (-1.5,1.5) -- (2,1.5);
      \clip (-1.5,1.5) rectangle (2,-1.25);
      \draw [thick,rotate=120] (0,0) -- (3,0);
      \draw [thick,rotate=45] (0,0) -- (3,0);
      \draw [dashed] (0,0) -- (0,2);
      \draw [rotate=45,->] (.8,0) arc (0:45:.8);
      \draw [rotate=67] (1,0) node {\scriptsize $30^\circ$};
      \draw [rotate=90,->] (1,0) arc (0:30:1);
      \draw [rotate=105] (1.2,0) node {\scriptsize $30^\circ$};
    \end{tikzpicture}
  \end{image}
  Find the magnitude of the force applied to each chain.
\begin{prompt}
  \[
  \text{Magnitude of the force on the left chain} = \answer{100/\sqrt{3}}\mathrm{lb}
  \]
  \[
  \text{Magnitude of the force on the right chain} = \answer{100/\sqrt{3}}\mathrm{lb}
  \]
\end{prompt}

\end{exercise}
\end{document}
