\documentclass{ximera}


\graphicspath{
  {./}
  {ximeraTutorial/}
  {basicPhilosophy/}
}

\newcommand{\mooculus}{\textsf{\textbf{MOOC}\textnormal{\textsf{ULUS}}}}

\usepackage{tkz-euclide}\usepackage{tikz}
\usepackage{tikz-cd}
\usetikzlibrary{arrows}
\tikzset{>=stealth,commutative diagrams/.cd,
  arrow style=tikz,diagrams={>=stealth}} %% cool arrow head
\tikzset{shorten <>/.style={ shorten >=#1, shorten <=#1 } } %% allows shorter vectors

\usetikzlibrary{backgrounds} %% for boxes around graphs
\usetikzlibrary{shapes,positioning}  %% Clouds and stars
\usetikzlibrary{matrix} %% for matrix
\usepgfplotslibrary{polar} %% for polar plots
\usepgfplotslibrary{fillbetween} %% to shade area between curves in TikZ
\usetkzobj{all}
\usepackage[makeroom]{cancel} %% for strike outs
%\usepackage{mathtools} %% for pretty underbrace % Breaks Ximera
%\usepackage{multicol}
\usepackage{pgffor} %% required for integral for loops



%% http://tex.stackexchange.com/questions/66490/drawing-a-tikz-arc-specifying-the-center
%% Draws beach ball
\tikzset{pics/carc/.style args={#1:#2:#3}{code={\draw[pic actions] (#1:#3) arc(#1:#2:#3);}}}



\usepackage{array}
\setlength{\extrarowheight}{+.1cm}
\newdimen\digitwidth
\settowidth\digitwidth{9}
\def\divrule#1#2{
\noalign{\moveright#1\digitwidth
\vbox{\hrule width#2\digitwidth}}}






\DeclareMathOperator{\arccot}{arccot}
\DeclareMathOperator{\arcsec}{arcsec}
\DeclareMathOperator{\arccsc}{arccsc}

















%%This is to help with formatting on future title pages.
\newenvironment{sectionOutcomes}{}{}




\author{Bart Snapp}


\outcome{State the definition of a vector.}
\outcome{Work with vectors in two or three dimensions. }
\outcome{Multiply vectors by scalars.}
\outcome{Add and subtract vectors.}
\outcome{Calculate the magnitude of a vector.}
\outcome{Find unit vectors.}
\outcome{Use vectors in applied settings.}


\title[Dig-In:]{Vectors}

\begin{document}
\begin{abstract}
  Vectors are lists of numbers that denote direction and magnitude.
\end{abstract}
\maketitle


\section{The idea of vectors}

The most successful textbook that was ever written was
\link[\textit{Euclid's
    Elements}]{https://mathcs.clarku.edu/~djoyce/java/elements/elements.html}. While
you are surely skeptical of this claim, and it is \textit{good} to be
skeptical, consider this: \textit{Euclid's Elements} was used (in
various editions) as a primary mathematics textbook for nearly 2000
years. There are few textbooks (if any) that can share this
claim. However, \textit{Euclid's Elements} does have its shortcomings. 
Euclid defines a point as ``that which has no part.'' Many people 
(including this author) find this to be 
a pretty confusing definition. What does Euclid mean by this statement? 
However, from our modern viewpoint, a point is an
ordered list of numbers, like
\[
(1,1)\quad\text{or}\quad(4,2).
\]
We have grown to see that a point should be thought of as
\textit{location}, and nothing but location. With this definition in mind, it
doesn't really make sense to have operations between points like
addition or subtraction.

When trying to understand the world around us, we are often concerned
with quantities that denote both \textit{direction} and
\textit{magnitude}. We can do this by starting with two points










This object formed by the differences in the values of the coordinates of the points is 
called a \textit{vector}. In the graph above, the vector is
$\overset{\rightharpoonup}{\mathbf{v}}=\left\langle a-c,b-d \right\rangle$. We write vectors typographically in
boldface, decorated with a harpoon (like $\overset{\rightharpoonup}{\mathbf{v}}$ or
$\overset{\rightharpoonup}{\mathbf{w}}$). Other authors may simply use a boldface (like
$\mathbf{v}$ or $\mathbf{w}$) or just a harpoon (like ${\overset{\rightharpoonup}{v}}$
or ${\overset{\rightharpoonup}{w}}$). We often visualize a vector (at least in two and
three dimensions) as an arrow to explicitly show its direction and
magnitude. This visualization leads us to our definition of a vector.































\end{document}
