\documentclass{ximera}


\graphicspath{
  {./}
  {ximeraTutorial/}
  {basicPhilosophy/}
}

\newcommand{\mooculus}{\textsf{\textbf{MOOC}\textnormal{\textsf{ULUS}}}}

\usepackage{tkz-euclide}\usepackage{tikz}
\usepackage{tikz-cd}
\usetikzlibrary{arrows}
\tikzset{>=stealth,commutative diagrams/.cd,
  arrow style=tikz,diagrams={>=stealth}} %% cool arrow head
\tikzset{shorten <>/.style={ shorten >=#1, shorten <=#1 } } %% allows shorter vectors

\usetikzlibrary{backgrounds} %% for boxes around graphs
\usetikzlibrary{shapes,positioning}  %% Clouds and stars
\usetikzlibrary{matrix} %% for matrix
\usepgfplotslibrary{polar} %% for polar plots
\usepgfplotslibrary{fillbetween} %% to shade area between curves in TikZ
\usetkzobj{all}
\usepackage[makeroom]{cancel} %% for strike outs
%\usepackage{mathtools} %% for pretty underbrace % Breaks Ximera
%\usepackage{multicol}
\usepackage{pgffor} %% required for integral for loops



%% http://tex.stackexchange.com/questions/66490/drawing-a-tikz-arc-specifying-the-center
%% Draws beach ball
\tikzset{pics/carc/.style args={#1:#2:#3}{code={\draw[pic actions] (#1:#3) arc(#1:#2:#3);}}}



\usepackage{array}
\setlength{\extrarowheight}{+.1cm}
\newdimen\digitwidth
\settowidth\digitwidth{9}
\def\divrule#1#2{
\noalign{\moveright#1\digitwidth
\vbox{\hrule width#2\digitwidth}}}






\DeclareMathOperator{\arccot}{arccot}
\DeclareMathOperator{\arcsec}{arcsec}
\DeclareMathOperator{\arccsc}{arccsc}

















%%This is to help with formatting on future title pages.
\newenvironment{sectionOutcomes}{}{}


\outcome{Solve optimization problems by finding the appropriate extreme values.}

\title[Break-Ground:]{A mysterious formula}

\begin{document}
\begin{abstract}
Two young mathematicians discuss optimization from an abstract point
of view.
\end{abstract}
\maketitle

Check out this dialogue between two calculus students:

\begin{dialogue}
\item[Devyn] Riley, what do you think is the maximum value of
  \[
  f(x) = \frac{10}{x^2-2.8x+3}?
  \]
\item[Riley] Where did that function come from?
\item[Devyn] It's just some, um, random function.
\item[Riley] Wait, does this have to do with coffee?
\item[Devyn] Um, uh, no?
\item[Riley] Well what interval are we on?
\item[Devyn] Let's say $[0,10]$, I mean there's no way I could possibly drink ten cups of coff\dots
\item[Riley] I knew this was about coffee.
\end{dialogue}

Here Devyn has made a function, that is supposed to record Devyn's
``well-being'' with respect to the number of cups of coffee consumed
in one day.

\begin{problem}
  Graph Devyn's function. Where do you estimate the maximum on the
  interval $[0,10]$ to be?
  \begin{prompt}
    The maximum is at $x=\answer[tolerance=.2]{1.4}$
  \end{prompt}
\end{problem}

\begin{problem}
  If you wanted to argue that this is the (global) maximum value on
  $[0,10]$ without plotting, what arguments could you use?
  \begin{freeResponse}
  \end{freeResponse}
 \end{problem}

%\input{../leveledQuestions.tex}

\end{document}
