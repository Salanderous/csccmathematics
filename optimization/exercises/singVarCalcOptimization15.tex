\documentclass{ximera}

\graphicspath{
  {./}
  {ximeraTutorial/}
  {basicPhilosophy/}
}

\newcommand{\mooculus}{\textsf{\textbf{MOOC}\textnormal{\textsf{ULUS}}}}

\usepackage{tkz-euclide}\usepackage{tikz}
\usepackage{tikz-cd}
\usetikzlibrary{arrows}
\tikzset{>=stealth,commutative diagrams/.cd,
  arrow style=tikz,diagrams={>=stealth}} %% cool arrow head
\tikzset{shorten <>/.style={ shorten >=#1, shorten <=#1 } } %% allows shorter vectors

\usetikzlibrary{backgrounds} %% for boxes around graphs
\usetikzlibrary{shapes,positioning}  %% Clouds and stars
\usetikzlibrary{matrix} %% for matrix
\usepgfplotslibrary{polar} %% for polar plots
\usepgfplotslibrary{fillbetween} %% to shade area between curves in TikZ
\usetkzobj{all}
\usepackage[makeroom]{cancel} %% for strike outs
%\usepackage{mathtools} %% for pretty underbrace % Breaks Ximera
%\usepackage{multicol}
\usepackage{pgffor} %% required for integral for loops



%% http://tex.stackexchange.com/questions/66490/drawing-a-tikz-arc-specifying-the-center
%% Draws beach ball
\tikzset{pics/carc/.style args={#1:#2:#3}{code={\draw[pic actions] (#1:#3) arc(#1:#2:#3);}}}



\usepackage{array}
\setlength{\extrarowheight}{+.1cm}
\newdimen\digitwidth
\settowidth\digitwidth{9}
\def\divrule#1#2{
\noalign{\moveright#1\digitwidth
\vbox{\hrule width#2\digitwidth}}}






\DeclareMathOperator{\arccot}{arccot}
\DeclareMathOperator{\arcsec}{arcsec}
\DeclareMathOperator{\arccsc}{arccsc}

















%%This is to help with formatting on future title pages.
\newenvironment{sectionOutcomes}{}{}

\author{Bart Snapp\and nela Lakos}
\license{Creative Commons 3.0 By-NC}
\acknowledgement{https://www.whitman.edu/mathematics/calculus/}
  \outcome{Interpret an optimization problem as the procedure used to make a system or design as effective or functional as possible.}
  \outcome{Set up an optimization problem by identifying the objective function and appropriate constraints.}
  \outcome{Solve optimization problems by finding the appropriate absolute extremum.}
  \outcome{Solve basic word problems involving maxima or minima.}
\begin{document}
\begin{exercise}

 A trough, shown in the figure, is 64 ft long and its ends have the shape of isosceles triangles whose sides are 2 ft long. 
  \begin{image}
    \begin{tikzpicture}[scale=.5]
      \draw[penColor, very thick] (-2,4) -- (2,3.5) -- (0,0)--cycle;
      \draw[penColor, very thick] (0,0) -- (9,1) -- (11,4.5) --(2,3.5);
      \draw[penColor,very thick] (-2,4) -- (7,5) -- (7.45,4.11);
      \draw[penColor,very thick] (11,4.5) -- (7,5);
      \draw[penColor,very thick,dashed] (7.45,4.11) -- (9,1);
      \node at (9,5.1) {$x$};
      \node at (3,5) {$64$};
      \node at (-1.5,2) {$2$};
      \node at (1.5,1.8) {$2$};
    \end{tikzpicture}
  \end{image}
  
 
  Find  the length of the base of the two isosceles triangles, marked $x$ in the figure, that maximizes the volume.
  \begin{hint}
  Let V be the volume of the trough. Then
  
  V=(Area of the isosceles triangle)$\cdot 64$
  
  Let A be the area of the isosceles triangle. 
   \begin{image}
    \begin{tikzpicture}[scale=.5]
      \draw[penColor,  thick] (-2,4) -- (2,4) -- (0,0)--cycle;
    \draw[penColor2, dotted,thin] (0,0) -- (0,4) --(0,0)--cycle;
      \node at (0,4.2) {$x$};
       \node at (0.2,2.5) {$h$};
      \node at (-1.5,2) {$2$};
      \node at (1.5,1.8) {$2$};
    \end{tikzpicture}
  \end{image}

 Then
  $A=\frac{1}{2}\cdot x\cdot h$,
  
  and
  
  
$A=\frac{1}{2}\cdot x\cdot\sqrt{4-\frac{x^2}{4}}$
  \end{hint}
  \begin{prompt}
  \[
  x = \answer{2\sqrt{2}}
  \]
  \end{prompt}
\end{exercise}
\end{document}
