\documentclass{ximera}


\graphicspath{
  {./}
  {ximeraTutorial/}
  {basicPhilosophy/}
}

\newcommand{\mooculus}{\textsf{\textbf{MOOC}\textnormal{\textsf{ULUS}}}}

\usepackage{tkz-euclide}\usepackage{tikz}
\usepackage{tikz-cd}
\usetikzlibrary{arrows}
\tikzset{>=stealth,commutative diagrams/.cd,
  arrow style=tikz,diagrams={>=stealth}} %% cool arrow head
\tikzset{shorten <>/.style={ shorten >=#1, shorten <=#1 } } %% allows shorter vectors

\usetikzlibrary{backgrounds} %% for boxes around graphs
\usetikzlibrary{shapes,positioning}  %% Clouds and stars
\usetikzlibrary{matrix} %% for matrix
\usepgfplotslibrary{polar} %% for polar plots
\usepgfplotslibrary{fillbetween} %% to shade area between curves in TikZ
\usetkzobj{all}
\usepackage[makeroom]{cancel} %% for strike outs
%\usepackage{mathtools} %% for pretty underbrace % Breaks Ximera
%\usepackage{multicol}
\usepackage{pgffor} %% required for integral for loops



%% http://tex.stackexchange.com/questions/66490/drawing-a-tikz-arc-specifying-the-center
%% Draws beach ball
\tikzset{pics/carc/.style args={#1:#2:#3}{code={\draw[pic actions] (#1:#3) arc(#1:#2:#3);}}}



\usepackage{array}
\setlength{\extrarowheight}{+.1cm}
\newdimen\digitwidth
\settowidth\digitwidth{9}
\def\divrule#1#2{
\noalign{\moveright#1\digitwidth
\vbox{\hrule width#2\digitwidth}}}






\DeclareMathOperator{\arccot}{arccot}
\DeclareMathOperator{\arcsec}{arcsec}
\DeclareMathOperator{\arccsc}{arccsc}

















%%This is to help with formatting on future title pages.
\newenvironment{sectionOutcomes}{}{}


%\outcome{Describe the goals of optimization problems generally.}
%\outcome{Find all local maximums and minimums using the First and Second Derivative tests.}
%\outcome{Identify when we can find an absolute maximum or minimum on an open interval.}
%\outcome{Contrast optimization on open and closed intervals.}
%\outcome{Describe the objective function and constraints in a given optimization problem.}
%\outcome{Solve optimization problems by finding the appropriate extreme values.}

\author{Nela Lakos \and Kyle Parsons}

\begin{document}
\begin{exercise}

A right circular cone is to be made with a fixed slant height of 8 ft.
\begin{image}
\begin{tikzpicture}

\coordinate (c) at (0,0);
\coordinate (t) at (0,3);
\coordinate (r) at (2,0);
\coordinate (l) at (-2,0);

\filldraw[color=white,fill=penColor,fill opacity=0.3] {(c)++(0,-0.02)} circle [x radius = 2.002,y radius=0.3];

\draw[penColor] (l) ++ (0,-0.02) arc[x radius=2.002,y radius=.3,start angle=180,end angle=366];
\draw[dashed,penColor] (l) ++ (0,-0.02) arc[x radius=2.002,y radius=.3,start angle=180,end angle=0];
\draw[penColor] {(l)++(0.005,0)} -- (t) -- node[above,sloped]{\tiny slant height} (r);

\draw[penColor,dashed] (c) -- node[above,sloped]{\tiny height} (t);

\node at (c) [penColor] {\tiny\textbullet};

\end{tikzpicture}
\end{image}

The height that maximizes the volume of the cone is
\[
h = \answer{\frac{8}{\sqrt{3}}}
\]

Hint: The volume of the cone is $\frac{\pi r^2h}{3}$, where $h$ is the height of the cone and $r$ is the radius.

\end{exercise}
\end{document}