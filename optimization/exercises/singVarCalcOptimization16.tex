\documentclass{ximera}

\graphicspath{
  {./}
  {ximeraTutorial/}
  {basicPhilosophy/}
}

\newcommand{\mooculus}{\textsf{\textbf{MOOC}\textnormal{\textsf{ULUS}}}}

\usepackage{tkz-euclide}\usepackage{tikz}
\usepackage{tikz-cd}
\usetikzlibrary{arrows}
\tikzset{>=stealth,commutative diagrams/.cd,
  arrow style=tikz,diagrams={>=stealth}} %% cool arrow head
\tikzset{shorten <>/.style={ shorten >=#1, shorten <=#1 } } %% allows shorter vectors

\usetikzlibrary{backgrounds} %% for boxes around graphs
\usetikzlibrary{shapes,positioning}  %% Clouds and stars
\usetikzlibrary{matrix} %% for matrix
\usepgfplotslibrary{polar} %% for polar plots
\usepgfplotslibrary{fillbetween} %% to shade area between curves in TikZ
\usetkzobj{all}
\usepackage[makeroom]{cancel} %% for strike outs
%\usepackage{mathtools} %% for pretty underbrace % Breaks Ximera
%\usepackage{multicol}
\usepackage{pgffor} %% required for integral for loops



%% http://tex.stackexchange.com/questions/66490/drawing-a-tikz-arc-specifying-the-center
%% Draws beach ball
\tikzset{pics/carc/.style args={#1:#2:#3}{code={\draw[pic actions] (#1:#3) arc(#1:#2:#3);}}}



\usepackage{array}
\setlength{\extrarowheight}{+.1cm}
\newdimen\digitwidth
\settowidth\digitwidth{9}
\def\divrule#1#2{
\noalign{\moveright#1\digitwidth
\vbox{\hrule width#2\digitwidth}}}






\DeclareMathOperator{\arccot}{arccot}
\DeclareMathOperator{\arcsec}{arcsec}
\DeclareMathOperator{\arccsc}{arccsc}

















%%This is to help with formatting on future title pages.
\newenvironment{sectionOutcomes}{}{}

\author{Bart Snapp\and Nela Lakos}
\license{Creative Commons 3.0 By-NC}
\acknowledgement{https://www.whitman.edu/mathematics/calculus/}
  \outcome{Interpret an optimization problem as the procedure used to make a system or design as effective or functional as possible.}
  \outcome{Set up an optimization problem by identifying the objective function and appropriate constraints.}
  \outcome{Solve optimization problems by finding the appropriate absolute extremum.}
  \outcome{Solve basic word problems involving maxima or minima.}
\begin{document}
\begin{exercise}

  A rectangle is inscribed in the ellipse
  \[
  \frac{x^2}{3}+\frac{y^2}{5}=1
  \]
  so that two of its edges are parallel to the $x$-axis, and the other two are parallel to the $y$-axis.
  Find the dimensions of the rectangle with the largest area.
  \begin{hint}
  \begin{image}
\begin{tikzpicture}
\coordinate (c) at (1.495,2.492);
\draw[penColor, thick] (0,2) ellipse (2 and .7);
%\draw[very thick,penColor!20!background] (2,-2) arc (0:180:2 and .7);% top half of ellipse
%\draw[very thick,penColor] (-2,-2) arc (180:360:2 and .7);% bottom half of ellips
\draw[penColor,  thin] (-3,2) -- (3,2);
\draw[penColor,  thin] (0,0.5) -- (0,3.5);
\draw[penColor2,  thin] (-3/2,1.57+2/2.24) -- (3/2,1.57+2/2.24);
\draw[penColor2,  thin] (-3/2,2.43-2/2.24) -- (3/2,2.43-2/2.24);
\draw[penColor2,  thin]  (-3/2,2.43-2/2.24) -- (-3/2,1.57+2/2.24);
\draw[penColor2,  thin]  (3/2,2.43-2/2.24) -- (3/2,1.57+2/2.24);
    \draw [color=penColor,fill=penColor] (1.5,3);  %% closed hole   
    \node [above,penColor] at (0.2,3) {$y$};
 \node [above,penColor] at (3,1.7) {$x$};
  \node [above,penColor] at (1.9,2.45) {$(x,y)$};
\node at (c) [penColor] {\small\textbullet};
\end{tikzpicture}
\end{image}
\end{hint}
\begin{hint}
Area of the rectangle
 $A=2x\cdot 2y$. 
 We will express $A$ as a function of, say, $x$.
Note that $x$ and $y$ satisfy the equation of the ellipse, so
 \[
y^2=5-  \frac{5x^2}{3}
  \]
Since in our figure $y\ge0$, we have 
 \[
y=\sqrt{5-  \frac{5x^2}{3}}
  \]
\end{hint}
\begin{hint}
Now
$A(x)=2x\cdot 2\sqrt{5-  \frac{5x^2}{3}}$.

The domain of $A=\left[0,\answer{\sqrt{3}}\right]$.
\end{hint}
\begin{hint}
We have to find the critical points of $A$.


$A'(x)=4\cdot \sqrt{5-  \frac{5x^2}{3}}+\frac{4x}{2\sqrt{5-  \frac{5x^2}{3}}}\cdot \left(-\frac{10}{3}x\right)$.
If we simplify $A'(x)$, we get

$A'(x)=4\cdot \sqrt{5-  \frac{5x^2}{3}}-\frac{20x^2}{3\sqrt{5-  \frac{5x^2}{3}}}$, 

and if we simplify again, we get

$A'(x)=\frac{60-40x^2}{3\sqrt{5-  \frac{5x^2}{3}}}$

The function $A$ has the only critical point at $x=\sqrt{\frac{\answer{3}}{2}}$.
\end{hint}
\begin{hint}

Since the domain of $A$ is a closed interval $\left[0,\answer{\sqrt{3}}\right]$, 
we have to evaluate the function $A$ at the end points and at the critical point:
 \[
A(0)=0
  \]
  \[
A(\sqrt{3})=0
  \]
  \[
A\left(\sqrt{\frac{\answer{3}}{2}}\right)=4\sqrt{\frac{\answer{3}}{2}}\sqrt{5-  \frac{5}{2}}=2\sqrt{15}.
  \]
  So, the maximum is attained at the critical point.
\end{hint}
  \begin{prompt}
  \begin{align*}
  \text{base}&= \answer{\sqrt{6}}\\
  \text{height} &= \answer{\sqrt{10}}
  \end{align*}
  \end{prompt}
\end{exercise}
\end{document}
