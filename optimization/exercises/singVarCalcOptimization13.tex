\documentclass{ximera}

\graphicspath{
  {./}
  {ximeraTutorial/}
  {basicPhilosophy/}
}

\newcommand{\mooculus}{\textsf{\textbf{MOOC}\textnormal{\textsf{ULUS}}}}

\usepackage{tkz-euclide}\usepackage{tikz}
\usepackage{tikz-cd}
\usetikzlibrary{arrows}
\tikzset{>=stealth,commutative diagrams/.cd,
  arrow style=tikz,diagrams={>=stealth}} %% cool arrow head
\tikzset{shorten <>/.style={ shorten >=#1, shorten <=#1 } } %% allows shorter vectors

\usetikzlibrary{backgrounds} %% for boxes around graphs
\usetikzlibrary{shapes,positioning}  %% Clouds and stars
\usetikzlibrary{matrix} %% for matrix
\usepgfplotslibrary{polar} %% for polar plots
\usepgfplotslibrary{fillbetween} %% to shade area between curves in TikZ
\usetkzobj{all}
\usepackage[makeroom]{cancel} %% for strike outs
%\usepackage{mathtools} %% for pretty underbrace % Breaks Ximera
%\usepackage{multicol}
\usepackage{pgffor} %% required for integral for loops



%% http://tex.stackexchange.com/questions/66490/drawing-a-tikz-arc-specifying-the-center
%% Draws beach ball
\tikzset{pics/carc/.style args={#1:#2:#3}{code={\draw[pic actions] (#1:#3) arc(#1:#2:#3);}}}



\usepackage{array}
\setlength{\extrarowheight}{+.1cm}
\newdimen\digitwidth
\settowidth\digitwidth{9}
\def\divrule#1#2{
\noalign{\moveright#1\digitwidth
\vbox{\hrule width#2\digitwidth}}}






\DeclareMathOperator{\arccot}{arccot}
\DeclareMathOperator{\arcsec}{arcsec}
\DeclareMathOperator{\arccsc}{arccsc}

















%%This is to help with formatting on future title pages.
\newenvironment{sectionOutcomes}{}{}

\author{Bart Snapp\and Nela Lakos}
\license{Creative Commons 3.0 By-NC}
\acknowledgement{https://www.whitman.edu/mathematics/calculus/}
  \outcome{Interpret an optimization problem as the procedure used to make a system or design as effective or functional as possible.}
  \outcome{Set up an optimization problem by identifying the objective function and appropriate constraints.}
  \outcome{Solve optimization problems by finding the appropriate absolute extremum.}
  \outcome{Solve basic word problems involving maxima or minima.}
\begin{document}
\begin{exercise}

  A window is to be built in the shape of a semi-circle over a rectangle:
  \begin{image}
    \begin{tikzpicture}
      \draw[very thick,penColor] (2,0) arc (0:180:2 and 2);% top half of ellipse
      \draw[penColor, very thick] (-2,-2) -- (2,-2);
      \draw[penColor, very thick] (2,-2) -- (2,0);
      \draw[penColor,very thick] (-2,-2) -- (-2,0);
      \draw[penColor,very thick,dashed] (-2,0) -- (2,0);
    \end{tikzpicture}
  \end{image}
  If the outer perimeter is $10$ feet, maximize the area. Round your answer to the closest tenth of a square foot.
  \begin{hint}
  Let's label the picture.
   \begin{image}
    \begin{tikzpicture}
      \draw[very thick,penColor] (2,0) arc (0:180:2 and 2);% top half of ellipse
      \draw[penColor, very thick] (-2,-2) -- (2,-2);
        \draw[penColor2, very thick, dashed] (0,0) -- (0,2);
      \draw[penColor, very thick] (2,-2) -- (2,0);
      \draw[penColor,very thick] (-2,-2) -- (-2,0);
      \draw[penColor,very thick,dashed] (-2,0) -- (2,0);
        \node [below,penColor] at (1,0.35) {$r$};
         \node [below,penColor] at (0.2,1.2) {$r$};
          \node [below,penColor] at (0,-1.6) {$2r$};
           \node [below,penColor] at (-1.7,-0.7) {$x$};
    \end{tikzpicture}
  \end{image}
    \end{hint}
      \begin{hint}
      The outer perimeter, $P$, is given by
      
      $P=2r+2x+r\pi$.
      
      Since we know that $P=10$, we can express $x$ in terms of $r$.
      
      $x=5-r\frac{\pi}{2}-r$.
      
      The area, A , consists of the area of the rectangle + the area of the semicircle:
      
      $A= 2\cdot r\cdot x+\frac{\pi}{2} r^2$
      
        \end{hint}
           \begin{hint}
           We can now express the area  $A$ as a function of $r$:
           
           $A(r)=2\cdot r\cdot (5-r\frac{\pi}{2}-r)+\frac{\pi}{2} r^2=10r-\frac{\pi}{2}r^2-2r^2$.
           
           The domain of A is  the closed interval $\left[0,\frac{10}{2+\pi}\right]$.(Remember: P=10!)
              \end{hint}
                \begin{hint}
                Now we have to find the global maximum of $A$ on its domain.
                First, we have to find the critical points of $A$.
                Since
                
                $A'(r)=10-\pi r-4r$,
                
                it follows that the function $A$ has its only critical point at  $r=\frac{\answer{10}}{4+\pi}$ .
                              \end{hint}
                                \begin{hint}
                                Now, we evaluate $A$ at the end points and at the critical point and then compare the values.
                                
                                $A(0)=0$
                                
                                $A\left(\frac{10}{2+\pi}\right)=\frac{50\pi}{(2+\pi)^2}$
                                
                                $A\left(\frac{\answer{10}}{4+\pi}\right)=\frac{\answer{50}}{4+\pi}\approx 7$
                                
                                This seems complicated. It is better to argue that the derivative 
                                
                                $A'(r)>0$ on $\left(0,\frac{10}{4+\pi}\right)$ and  $A'(r)<0$ on $\left(\frac{10}{4+\pi},\frac{10}{2+\pi}\right)$,
                                
                                 since $A'(r)=10-\pi r-4r=10-r(\pi+4)=-(\pi+4)\left(r-\frac{10}{\pi+4}\right)$.
                                 
                                 
                                 This implies that the global maximum occurs at the critical point.

                                  \end{hint}
  \begin{prompt}
  \[
  A = \answer{7.0} ft^2
  \]
  \end{prompt}
\end{exercise}
\end{document}
