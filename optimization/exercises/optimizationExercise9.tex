\documentclass{ximera}


\graphicspath{
  {./}
  {ximeraTutorial/}
  {basicPhilosophy/}
}

\newcommand{\mooculus}{\textsf{\textbf{MOOC}\textnormal{\textsf{ULUS}}}}

\usepackage{tkz-euclide}\usepackage{tikz}
\usepackage{tikz-cd}
\usetikzlibrary{arrows}
\tikzset{>=stealth,commutative diagrams/.cd,
  arrow style=tikz,diagrams={>=stealth}} %% cool arrow head
\tikzset{shorten <>/.style={ shorten >=#1, shorten <=#1 } } %% allows shorter vectors

\usetikzlibrary{backgrounds} %% for boxes around graphs
\usetikzlibrary{shapes,positioning}  %% Clouds and stars
\usetikzlibrary{matrix} %% for matrix
\usepgfplotslibrary{polar} %% for polar plots
\usepgfplotslibrary{fillbetween} %% to shade area between curves in TikZ
\usetkzobj{all}
\usepackage[makeroom]{cancel} %% for strike outs
%\usepackage{mathtools} %% for pretty underbrace % Breaks Ximera
%\usepackage{multicol}
\usepackage{pgffor} %% required for integral for loops



%% http://tex.stackexchange.com/questions/66490/drawing-a-tikz-arc-specifying-the-center
%% Draws beach ball
\tikzset{pics/carc/.style args={#1:#2:#3}{code={\draw[pic actions] (#1:#3) arc(#1:#2:#3);}}}



\usepackage{array}
\setlength{\extrarowheight}{+.1cm}
\newdimen\digitwidth
\settowidth\digitwidth{9}
\def\divrule#1#2{
\noalign{\moveright#1\digitwidth
\vbox{\hrule width#2\digitwidth}}}






\DeclareMathOperator{\arccot}{arccot}
\DeclareMathOperator{\arcsec}{arcsec}
\DeclareMathOperator{\arccsc}{arccsc}

















%%This is to help with formatting on future title pages.
\newenvironment{sectionOutcomes}{}{}


%\outcome{Describe the goals of optimization problems generally.}
\outcome{Find all local maximums and minimums using the First and Second Derivative tests.}
%\outcome{Identify when we can find an absolute maximum or minimum on an open interval.}
%\outcome{Contrast optimization on open and closed intervals.}
\outcome{Describe the objective function and constraints in a given optimization problem.}
\outcome{Solve optimization problems by finding the appropriate extreme values.}

\author{Nela Lakos \and Kyle Parsons}

\begin{document}
\begin{exercise}

A toy roller coaster has been designed so that the track is in the shape of the graph of $f(x) = x - \sin\left(\frac{\pi x}{5}\right)$ where $x$ and $f(x)$ are in inches.

\begin{image}
 \begin{tikzpicture}
    \begin{axis}[
        xmin=-0.3,xmax=10.3,ymin=-0.3,ymax=10.3,
        clip=true,
        unit vector ratio*=1 1 1,
        axis lines=center,
        grid = major,
        ytick={-20,-18,...,20},
        xtick={-20,-18,...,20},
        xlabel=$x$, ylabel=$y$,
        y tick label style={anchor=west},
        every axis y label/.style={at=(current axis.above origin),anchor=south},
        every axis x label/.style={at=(current axis.right of origin),anchor=west},
      ]
      \addplot[very thick,penColor,domain=0:10,samples=50] plot{x - sin(36*x)};
      
	  \addplot[only marks,mark=*,penColor] coordinates{(0,0) (10,10)};      
      
      \node at (axis cs:3,9) {$y=f(x)$};
      \end{axis}`
  \end{tikzpicture}
\end{image}

The average rate of change of the altitude of the roller coaster on the interval $[0,10]$ is $\answer{1}$.

Select the best interpretation of $f'(a)$ for $0<a<10$.
\begin{multipleChoice}
\choice{$f'(a)$ is the altitude at $a$.}
\choice{$f'(a)$ is the average rate of change of the altitude on the interval $[0,a]$.}
\choice{$f'(a)$ is the velocity of an object at $a$.}
\choice[correct]{$f'(a)$ is the instantaneous rate of change of the altitude at $a$.}
\choice{$f'(a)$ is the slope of the secant line passing through $(0,0)$ and $(10,10)$.}
\end{multipleChoice}

Because $f$ is \wordChoice{\choice[correct]{continuous}\choice{differentiable}} on the interval $[0,10]$ and $f$ is \wordChoice{\choice{continuous}\choice[correct]{differentiable}} on the interval $(0,10)$, $f$ satisfies the conditions of the Mean Value Theorem.

By the Mean Value Theorem, there exists $c$ in $(0,10)$ such that $f'(c) = 1$.  In face this happens twice, when $c_1=\answer{\frac{5}{2}}$ and when $c_2=\answer{\frac{15}{2}}$ (assume $c_1<c_2$).

The steepest point on the roller coaster is $\left(\answer{5},\answer{5}\right)$. (Hint: maximize $f'(x)$ on $(0,10)$.)

The linear approximation, $L$, to $f$ at $a=5$ is 
\[
L(x) = \answer{\left(1+\frac{\pi}{5}\right)(x-5) + 5}.
\]

Using this linear approximation we estimate that $f(7)$ is approximately
\[
f(7) \approx \answer{\left(1+\frac{\pi}{5}\right)2+5}.
\]
This estimate is an \wordChoice{\choice[correct]{overestimate}\choice{underestimate}} because $f$ is \wordChoice{\choice{concave up}\choice[correct]{concave down}} between 5 and 7.

When $x$ changes from $x=5$ to $x+\Delta x=7$ the change in $f$, $\Delta y$ is
\[
\Delta y = \answer{2-\sin\left(\frac{7\pi}{5}\right)}.
\]
We can approximate this change with the differential $\mathrm{d}y$ which is
\[
\mathrm{d}y = \answer{\left(1+\frac{\pi}{5}\right)2}.
\]
(Hint: Check the definition of differentials in the book.)



\end{exercise}
\end{document}