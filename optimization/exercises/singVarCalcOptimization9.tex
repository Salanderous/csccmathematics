\documentclass{ximera}

\graphicspath{
  {./}
  {ximeraTutorial/}
  {basicPhilosophy/}
}

\newcommand{\mooculus}{\textsf{\textbf{MOOC}\textnormal{\textsf{ULUS}}}}

\usepackage{tkz-euclide}\usepackage{tikz}
\usepackage{tikz-cd}
\usetikzlibrary{arrows}
\tikzset{>=stealth,commutative diagrams/.cd,
  arrow style=tikz,diagrams={>=stealth}} %% cool arrow head
\tikzset{shorten <>/.style={ shorten >=#1, shorten <=#1 } } %% allows shorter vectors

\usetikzlibrary{backgrounds} %% for boxes around graphs
\usetikzlibrary{shapes,positioning}  %% Clouds and stars
\usetikzlibrary{matrix} %% for matrix
\usepgfplotslibrary{polar} %% for polar plots
\usepgfplotslibrary{fillbetween} %% to shade area between curves in TikZ
\usetkzobj{all}
\usepackage[makeroom]{cancel} %% for strike outs
%\usepackage{mathtools} %% for pretty underbrace % Breaks Ximera
%\usepackage{multicol}
\usepackage{pgffor} %% required for integral for loops



%% http://tex.stackexchange.com/questions/66490/drawing-a-tikz-arc-specifying-the-center
%% Draws beach ball
\tikzset{pics/carc/.style args={#1:#2:#3}{code={\draw[pic actions] (#1:#3) arc(#1:#2:#3);}}}



\usepackage{array}
\setlength{\extrarowheight}{+.1cm}
\newdimen\digitwidth
\settowidth\digitwidth{9}
\def\divrule#1#2{
\noalign{\moveright#1\digitwidth
\vbox{\hrule width#2\digitwidth}}}






\DeclareMathOperator{\arccot}{arccot}
\DeclareMathOperator{\arcsec}{arcsec}
\DeclareMathOperator{\arccsc}{arccsc}

















%%This is to help with formatting on future title pages.
\newenvironment{sectionOutcomes}{}{}

\author{Bart Snapp\and Nela Lakos}
\license{Creative Commons 3.0 By-NC}
\acknowledgement{https://www.whitman.edu/mathematics/calculus/}
  \outcome{Interpret an optimization problem as the procedure used to make a system or design as effective or functional as possible.}
  \outcome{Set up an optimization problem by identifying the objective function and appropriate constraints.}
  \outcome{Solve optimization problems by finding the appropriate absolute extremum.}
  \outcome{Solve basic word problems involving maxima or minima.}
\begin{document}
\begin{exercise}

  You want to make cylindrical containers to hold 1 liter (1000$\text{cm}^3$) using the
  least amount of construction material.  The side is made from a
  rectangular piece of material, and this can be done with no material
  wasted.  However, the top and bottom are cut from squares of side
  $2r$, so that $2(2r)^2=8r^2$ of material is needed (rather than
  $2\pi r^2$, which is the total area of the top and bottom).  Find
  the dimensions of the container using the least amount of material.
  \begin{hint}
  The total  area of the material spent consists of the area of the lateral side of the cylinder plus the area of two squares of side $2r$.
  
  
  $S= 2\pi r h +8 r^2$

\begin{image}
    \begin{tikzpicture}
      \draw[very thick,penColor] (-1.6,0)  (0,2);% top half of ellipse
      \draw[penColor, very thick] (-1.6,-2) -- (1.6,-2);
      \draw[penColor, very thick] (1.6,-2) -- (1.6,0);
      \draw[penColor,very thick] (-1.6,-2) -- (-1.6,0);
      \draw[penColor,very thick] (-1.6,0) -- (1.6,0);
       \draw[penColor, very thick] (-0.5,-2) -- (-0.5,-3);
      \draw[penColor, very thick] (-0.5,-3) -- (0.5,-3);
      \draw[penColor,very thick] (0.5,-2) -- (0.5,-3);
      \draw[penColor,very thick] (-0.5,1) -- (0.5,1);
       \draw[penColor, very thick] (0.5,0) -- (0.5,1);
      \draw[penColor, very thick] (-0.5,0) -- (-0.5,1);
     \node [below,penColor] at (0.75,0.75) {$2r$};
       \node [below,penColor] at (0,1.5) {$2r$};
      \node [below,penColor] at (-1.4,-0.7) {$h$};
       \node [below,penColor] at (0,-1.4) {$2\cdot r\cdot \pi$};
       
        \draw [fill=gray,opacity=.5,thick](7,1.5) ellipse (1.25 and 0.5);
\draw (5.75,1.5) -- (5.75,-1);
\draw (5.75,-1) arc (180:360:1.25 and 0.5);
\draw [dashed] (5.75,-1) arc (180:360:1.25 and -0.5);
\draw (8.25,1.5) -- (8.25,-1) node[right,pos=.5]{$h$};  
\fill [gray,opacity=0.5] (5.75,1.5) -- (5.75,-1) arc (180:360:1.25 and 0.5) -- (8.25,1.5) arc (0:180:1.25 and -0.5);
\draw[dashed](5.75+1.25,-1) -- (8.25,-1) node[above,pos=.5] {$r$};
    \end{tikzpicture}
  \end{image}
  \end{hint}
  \begin{hint}
  Use the fact that the volume of the cylinder, $V=r^2\cdot\pi\cdot h$ and that $V=1000$.
  Express $h$ in terms of $r$.
  Now you can express $S$ as a function of $r$.
    \end{hint}
   Our objective function $S$ is given by the expression
   
   
    $S(r)=\frac{\answer{2000}}{r}+8r^{\answer{2}}$.
    \begin{hint}
    
    \end{hint}
  \begin{prompt}
  \[
  \text{radius}=\answer{5}\text{cm}\qquad\text{height}=\answer{40/\pi}\text{cm}
  \]
  \end{prompt}
\end{exercise}
\end{document}
