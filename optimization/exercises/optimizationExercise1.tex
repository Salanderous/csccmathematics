\documentclass{ximera}


\graphicspath{
  {./}
  {ximeraTutorial/}
  {basicPhilosophy/}
}

\newcommand{\mooculus}{\textsf{\textbf{MOOC}\textnormal{\textsf{ULUS}}}}

\usepackage{tkz-euclide}\usepackage{tikz}
\usepackage{tikz-cd}
\usetikzlibrary{arrows}
\tikzset{>=stealth,commutative diagrams/.cd,
  arrow style=tikz,diagrams={>=stealth}} %% cool arrow head
\tikzset{shorten <>/.style={ shorten >=#1, shorten <=#1 } } %% allows shorter vectors

\usetikzlibrary{backgrounds} %% for boxes around graphs
\usetikzlibrary{shapes,positioning}  %% Clouds and stars
\usetikzlibrary{matrix} %% for matrix
\usepgfplotslibrary{polar} %% for polar plots
\usepgfplotslibrary{fillbetween} %% to shade area between curves in TikZ
\usetkzobj{all}
\usepackage[makeroom]{cancel} %% for strike outs
%\usepackage{mathtools} %% for pretty underbrace % Breaks Ximera
%\usepackage{multicol}
\usepackage{pgffor} %% required for integral for loops



%% http://tex.stackexchange.com/questions/66490/drawing-a-tikz-arc-specifying-the-center
%% Draws beach ball
\tikzset{pics/carc/.style args={#1:#2:#3}{code={\draw[pic actions] (#1:#3) arc(#1:#2:#3);}}}



\usepackage{array}
\setlength{\extrarowheight}{+.1cm}
\newdimen\digitwidth
\settowidth\digitwidth{9}
\def\divrule#1#2{
\noalign{\moveright#1\digitwidth
\vbox{\hrule width#2\digitwidth}}}






\DeclareMathOperator{\arccot}{arccot}
\DeclareMathOperator{\arcsec}{arcsec}
\DeclareMathOperator{\arccsc}{arccsc}

















%%This is to help with formatting on future title pages.
\newenvironment{sectionOutcomes}{}{}


%\outcome{Describe the goals of optimization problems generally.}
\outcome{Find all local maximums and minimums using the First and Second Derivative tests.}
%\outcome{Identify when we can find an absolute maximum or minimum on an open interval.}
%\outcome{Contrast optimization on open and closed intervals.}
\outcome{Describe the objective function and constraints in a given optimization problem.}
\outcome{Solve optimization problems by finding the appropriate extreme values.}

\author{Nela Lakos \and Kyle Parsons}

\begin{document}
\begin{exercise}

A shipping company is trying to minimize the cost of driving a truck between Chicago and New Orleans.  They know the following things:
\begin{itemize}
\item The trip is 750 miles.
\item Running at 50 mph, the truck gets 4 miles per gallon.
\item For each 1 mph increase in speed above 50 mph the truck loses 1/10 of a mile per gallon in fuel efficiency, and for each 1 mph decrease in speed below 50 mph the truck gains 1/10 of a mile per gallon in fuel efficiency.
\item The team of drivers earns 27 dollars per hour.
\item There is an additional cost of 15 dollars per hour to operate the truck.
\item Diesel fuel for the truck costs \$3.90 per gallon.
\item The truck will be driven at a constant speed for the entirety of the trip.
\end{itemize}

Write a cost function, $C(x)$, for the total cost of driving the truck from Chicago to New Orleans at the \textbf{constant} speed $x$.
\[
C(x) = \answer{750\left(\frac{42}{x}+\frac{39}{90-x}\right)}\,\text{dollars}
\]

The domain of this cost function is $\left(\answer{0},\answer{90}\right)$ in miles per hour.

To minimize this function we look for the critical numbers where $C'(x)=0$.  

\[
C'(x) = \answer{750\left(-\frac{42}{x^2} + \frac{39}{(90-x)^2}\right)}
\]

Solving this equation for zero gives us one critical number, $x_0$, in the domain of $C$ at $x=\answer{45.8}$ miles per hour (round to the nearest tenth of a mile per hour).

To confirm this critical number is a local minimum we use the first derivative test.  For $x<x_0$, $C'(x)$ is \wordChoice{\choice{positive}\choice[correct]{negative}}, and for $x>x_0$, $C'(x)$ is \wordChoice{\choice[correct]{positive}\choice{negative}}.  Then by the first derivative test, $C$ has a local minimum.  Now since $C$ has only one critical number on its domain, and it has a local minimum there, then that local minimum is also the absolute minimum of $C$.

So we conclude that to minimize cost the truck should be driven at $\answer{45.8}$ miles per hour (rounded to the nearest tenth of a mile per hour).

\end{exercise}
\end{document}