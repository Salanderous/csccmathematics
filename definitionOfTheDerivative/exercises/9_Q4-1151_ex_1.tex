\documentclass{ximera}


\graphicspath{
  {./}
  {ximeraTutorial/}
  {basicPhilosophy/}
}

\newcommand{\mooculus}{\textsf{\textbf{MOOC}\textnormal{\textsf{ULUS}}}}

\usepackage{tkz-euclide}\usepackage{tikz}
\usepackage{tikz-cd}
\usetikzlibrary{arrows}
\tikzset{>=stealth,commutative diagrams/.cd,
  arrow style=tikz,diagrams={>=stealth}} %% cool arrow head
\tikzset{shorten <>/.style={ shorten >=#1, shorten <=#1 } } %% allows shorter vectors

\usetikzlibrary{backgrounds} %% for boxes around graphs
\usetikzlibrary{shapes,positioning}  %% Clouds and stars
\usetikzlibrary{matrix} %% for matrix
\usepgfplotslibrary{polar} %% for polar plots
\usepgfplotslibrary{fillbetween} %% to shade area between curves in TikZ
\usetkzobj{all}
\usepackage[makeroom]{cancel} %% for strike outs
%\usepackage{mathtools} %% for pretty underbrace % Breaks Ximera
%\usepackage{multicol}
\usepackage{pgffor} %% required for integral for loops



%% http://tex.stackexchange.com/questions/66490/drawing-a-tikz-arc-specifying-the-center
%% Draws beach ball
\tikzset{pics/carc/.style args={#1:#2:#3}{code={\draw[pic actions] (#1:#3) arc(#1:#2:#3);}}}



\usepackage{array}
\setlength{\extrarowheight}{+.1cm}
\newdimen\digitwidth
\settowidth\digitwidth{9}
\def\divrule#1#2{
\noalign{\moveright#1\digitwidth
\vbox{\hrule width#2\digitwidth}}}






\DeclareMathOperator{\arccot}{arccot}
\DeclareMathOperator{\arcsec}{arcsec}
\DeclareMathOperator{\arccsc}{arccsc}

















%%This is to help with formatting on future title pages.
\newenvironment{sectionOutcomes}{}{}


\outcome{Use limits to find the slope of the tangent line at a point.}
\outcome{Understand the definition of the derivative at a point.}

\begin{document}
\begin{exercise}

Using the \textbf{definition of the derivative}, fill in the blanks:
  \[
  f'(a)=
  \lim_{h\to\answer[format=integer]{0}}
  \frac{f(\answer{a+h})-f(\answer{a})}{\answer{h}}
  \]

  %% \begin{validator}[(first*first*first - second*second*second)==third*(3*a*a+3*a*h+h*h)]
  %% \[
  %% f'(a)=
  %% \lim_{h\to 0}
  %% \frac{f(\answer[id=first]{a+h})-f(\answer[id=second]{a})}{\answer[id=third]{h}}
  %% \]
  %% \end{validator}
  %% TO DO THIS CORRECTLY, SEE https://ximera.osu.edu/course/kisonecat/ximeraSample/sample

\begin{exercise}
  Let $f(x)=\sqrt{5+x^2}$. Use the limit definition of the derivative
  above to find $f'(2)$.  
  \begin{exercise}
    Write
    \[
    f'(2) = \lim_{h\to 0}\frac{f(\answer{2+h}) -f(\answer{2})}{\answer{h}}
    \]
    \begin{exercise}
      Now substitute $f(x) = \sqrt{5+x^2}$ to obtain
      \[
      f'(2) = \lim_{h\to 0} \frac{\answer{\sqrt{5+(2+h)^2}}-3}{h}
      \]
      \begin{exercise}
        To analytically compute this limit, we should multiply
        \[
        \lim_{h\to 0} \frac{\sqrt{\answer{9+4h+h^2}}-3}{h}
        \]
        by the fraction:
        \[
        \frac{\answer{\sqrt{9+4h+h^2}+3}}{\answer{\sqrt{9+4h+h^2}+3}}
        \]
        \begin{exercise}
          Now we have 
          \[
          \lim_{h\to 0} \frac{\answer{4h+h^2}}{h(\sqrt{9+4h+h^2}+3)}
          \]
          so we may cancel a factor of $h$,
          \[
          \lim_{h\to 0} \frac{\answer{4+h}}{\sqrt{9+4h+h^2}+3}
          \]
          and finally see that:
          \[
          f'(2) = \answer{2/3}
          \]
        \end{exercise}
      \end{exercise}
    \end{exercise}
  \end{exercise}
\end{exercise}
\end{exercise}

\end{document}
