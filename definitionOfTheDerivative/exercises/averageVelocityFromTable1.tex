\documentclass{ximera}


\graphicspath{
  {./}
  {ximeraTutorial/}
  {basicPhilosophy/}
}

\newcommand{\mooculus}{\textsf{\textbf{MOOC}\textnormal{\textsf{ULUS}}}}

\usepackage{tkz-euclide}\usepackage{tikz}
\usepackage{tikz-cd}
\usetikzlibrary{arrows}
\tikzset{>=stealth,commutative diagrams/.cd,
  arrow style=tikz,diagrams={>=stealth}} %% cool arrow head
\tikzset{shorten <>/.style={ shorten >=#1, shorten <=#1 } } %% allows shorter vectors

\usetikzlibrary{backgrounds} %% for boxes around graphs
\usetikzlibrary{shapes,positioning}  %% Clouds and stars
\usetikzlibrary{matrix} %% for matrix
\usepgfplotslibrary{polar} %% for polar plots
\usepgfplotslibrary{fillbetween} %% to shade area between curves in TikZ
\usetkzobj{all}
\usepackage[makeroom]{cancel} %% for strike outs
%\usepackage{mathtools} %% for pretty underbrace % Breaks Ximera
%\usepackage{multicol}
\usepackage{pgffor} %% required for integral for loops



%% http://tex.stackexchange.com/questions/66490/drawing-a-tikz-arc-specifying-the-center
%% Draws beach ball
\tikzset{pics/carc/.style args={#1:#2:#3}{code={\draw[pic actions] (#1:#3) arc(#1:#2:#3);}}}



\usepackage{array}
\setlength{\extrarowheight}{+.1cm}
\newdimen\digitwidth
\settowidth\digitwidth{9}
\def\divrule#1#2{
\noalign{\moveright#1\digitwidth
\vbox{\hrule width#2\digitwidth}}}






\DeclareMathOperator{\arccot}{arccot}
\DeclareMathOperator{\arcsec}{arcsec}
\DeclareMathOperator{\arccsc}{arccsc}

















%%This is to help with formatting on future title pages.
\newenvironment{sectionOutcomes}{}{}


\outcome{Compute average velocity.}
\outcome{Approximate instantaneous velocity.}

\author{Nela Lakos \and Kyle Parsons}

\begin{document}
\begin{exercise}

Suppose $s(t)$ is the position (in feet) of an object moving along a line for time $t\geq 0$ (in seconds).  The average velocity of the object on the interval $[a,b]$ is
\[
v_{\text{av}} = \frac{s(b)-\answer{s(a)}}{\answer{b-a}}.
\]

\begin{exercise}

The table below gives the value of $s(t)$ at several times.

\[
\begin{array}{|c|c|c|c|c|c|c|c|}
\hline
t & 0 & 0.5 & 1 & 1.5 & 2 & 2.5 & 3 \\\hline
s(t) & 0 & 22 & 32 & 48 & 54 & 64 & 74 \\\hline
\end{array}
\]

The average velocity of the object on the interval $[1,3]$ is
\[
v_{\text{av}} = \answer{21}\text{ft/s.}
\]

\begin{exercise}

Based on the information in the table above we \wordChoice{\choice{can}\choice[correct]{cannot}} calculate the instantaneous velocity of the object at $t=1$.

\begin{exercise}

The best we can do is calculate the average velocity of the object over the intervals $[0.5,1]$ and $[1,1.5]$.  

The average velocity of the object on $[0.5,1]$ is 
\[
v_{\text{av}} = \answer{20}\text{ft/s.}
\]

The average velocity on the interval $[1,1.5]$ is
\[
v_{\text{av}} = \answer{32}\text{ft/s.}
\]

\end{exercise}
\end{exercise}
\end{exercise}
\end{exercise}
\end{document}