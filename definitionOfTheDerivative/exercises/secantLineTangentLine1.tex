\documentclass{ximera}


\graphicspath{
  {./}
  {ximeraTutorial/}
  {basicPhilosophy/}
}

\newcommand{\mooculus}{\textsf{\textbf{MOOC}\textnormal{\textsf{ULUS}}}}

\usepackage{tkz-euclide}\usepackage{tikz}
\usepackage{tikz-cd}
\usetikzlibrary{arrows}
\tikzset{>=stealth,commutative diagrams/.cd,
  arrow style=tikz,diagrams={>=stealth}} %% cool arrow head
\tikzset{shorten <>/.style={ shorten >=#1, shorten <=#1 } } %% allows shorter vectors

\usetikzlibrary{backgrounds} %% for boxes around graphs
\usetikzlibrary{shapes,positioning}  %% Clouds and stars
\usetikzlibrary{matrix} %% for matrix
\usepgfplotslibrary{polar} %% for polar plots
\usepgfplotslibrary{fillbetween} %% to shade area between curves in TikZ
\usetkzobj{all}
\usepackage[makeroom]{cancel} %% for strike outs
%\usepackage{mathtools} %% for pretty underbrace % Breaks Ximera
%\usepackage{multicol}
\usepackage{pgffor} %% required for integral for loops



%% http://tex.stackexchange.com/questions/66490/drawing-a-tikz-arc-specifying-the-center
%% Draws beach ball
\tikzset{pics/carc/.style args={#1:#2:#3}{code={\draw[pic actions] (#1:#3) arc(#1:#2:#3);}}}



\usepackage{array}
\setlength{\extrarowheight}{+.1cm}
\newdimen\digitwidth
\settowidth\digitwidth{9}
\def\divrule#1#2{
\noalign{\moveright#1\digitwidth
\vbox{\hrule width#2\digitwidth}}}






\DeclareMathOperator{\arccot}{arccot}
\DeclareMathOperator{\arcsec}{arcsec}
\DeclareMathOperator{\arccsc}{arccsc}

















%%This is to help with formatting on future title pages.
\newenvironment{sectionOutcomes}{}{}


\outcome{Compute average velocity.}
\outcome{Compare average and instantaneous velocity.}

\author{Nela Lakos \and Kyle Parsons}

\begin{document}
\begin{exercise}

An object is moving along a horizontal line.  Its position in feet is given by
\[
s(t) = t^2 - 2
\]
where $0\leq t\leq 5$ is in seconds.

Consider the points on the line below.

\begin{tikzpicture}
    \begin{axis}[
        xmin=-4.3,xmax=4.3,ymin=-.1,ymax=.1,
        clip=false,
        unit vector ratio*=1 1 1,
        axis lines=center,
        hide obscured x ticks=false,
        grid = major,
        xtick={-4,-3,...,4},
        hide y axis,
        xlabel=$s$,
        every axis x label/.style={anchor=south}
        ]
        \addplot[only marks,very thick,penColor,mark=*]
        	coordinates{(-2,0) (-1,0) (0,0) (2,0) (4,0)};
	
		\node at (axis cs:-2,0) [penColor,above] {$A$};
		\node at (axis cs:-1,0) [penColor,above] {$B$};
		\node at (axis cs:0,0) [penColor,above] {$C$};
		\node at (axis cs:2,0) [penColor,above] {$D$};
		\node at (axis cs:4,0) [penColor,above] {$E$};
    \end{axis}`
\end{tikzpicture}

The point that corresponds to the position of the particle at $t=1$ is $\answer{B}$.

\begin{exercise}

The average velocity of the object on the interval $[1,3]$ is
\[
v_{\text{av}} = \answer{4}\text{ft/s.}
\]

The average velocity of the object on the interval $[1,t]$ for $t>1$ is
\[
v_{\text{av}} = \answer{t+1}\text{ft/s.}
\]

The average velocity of the object on the interval $[t,1]$ for $0<t<1$ is
\[
v_{\text{av}} = \answer{t+1}\text{ft/s.}
\]

\begin{exercise}

The instantaneous velocity of the object at $t=1$ is
\[
v_{\text{inst}} = \answer{2}\text{ft/s.}
\]

\begin{exercise}

The graph of $s(t)$ is given below.

\begin{tikzpicture}
    \begin{axis}[
        xmin=-0.3,xmax=3.3,ymin=-2.3,ymax=7.3,
        clip=true,
        unit vector ratio*=1 1 1,
        axis lines=center,
        hide obscured x ticks=true,
        grid = major,
        xtick={0,1,...,3},
        ytick={-2,-1,...,7},
        xlabel=$s$,
        ylabel=$t$,
        every axis y label/.style={at=(current axis.above origin),anchor=south},
        every axis x label/.style={at=(current axis.right of origin),anchor=west},
        ]
        \addplot[very thick,penColor,domain=0:3.3,samples=50] plot{x^2-2};
	
		\node at (axis cs:1.2,5.1) [penColor,above] {$s=s(t)$};
    \end{axis}`
\end{tikzpicture}

Assume that $P$ and $A$ are points on the graph of $s(t)$.  Then
\[
P = \left(1,\answer{-1}\right) \text{ and } A = \left(t,\answer{t^2-2}\right).
\]

\begin{exercise}

The secant line through $P$ and $A$ is shown in the figure below.

\begin{tikzpicture}
    \begin{axis}[
        xmin=-0.3,xmax=3.3,ymin=-2.3,ymax=7.3,
        clip=true,
        unit vector ratio*=1 1 1,
        axis lines=center,
        hide obscured x ticks=true,
        grid = major,
        xtick={0,1,...,3},
        ytick={-2,-1,...,7},
        xlabel=$s$,
        ylabel=$t$,
        every axis y label/.style={at=(current axis.above origin),anchor=south},
        every axis x label/.style={at=(current axis.right of origin),anchor=west},
        ]
        \addplot[very thick,penColor,domain=0:3.3,samples=50] plot{x^2-2};
		
		\addplot[very thick,black,only marks,mark=*] coordinates{(0.3,0.3^2-2)};
		\node at (axis cs:0.3,0.3^2-2) [below right] {$A$};
		
		\addplot[very thick,black,only marks,mark=*] coordinates{(1,1^2-2)};
		\node at (axis cs:1,1^2-2) [below right] {$P$};
		
		\addplot[very thick,red,domain=-0.3:3.3] plot{(-1-(.3^2-2))/(1-.3)*(x-1)-1};
		
		\node at (axis cs:1.2,5.1) [penColor,above] {$s=s(t)$};
    \end{axis}`
\end{tikzpicture}

The slope of this secant line for $0<t<1$ is
\[
m_{\text{sec}} = \answer{\frac{1-t^2}{1-t}}.
\]

What is the connection between the slope of this secant line and the average velocity of the object over the interval $[t,1]$?

\begin{multipleChoice}
\choice{$m_{\text{sec}}$ is greater than $v_{\text{av}}$.}
\choice[correct]{$m_{\text{sec}}$ equals $v_{\text{av}}$.}
\choice{$m_{\text{sec}}$ is less than $v_{\text{av}}$.}
\choice{There is no connection between $m_{\text{sec}}$ and $v_{\text{av}}$.}
\end{multipleChoice}

\begin{exercise}

The point $P$ and the tangent line to the graph of $s$ at $P$ are given in the figure below.

\begin{tikzpicture}
    \begin{axis}[
        xmin=-0.3,xmax=3.3,ymin=-2.3,ymax=7.3,
        clip=true,
        unit vector ratio*=1 1 1,
        axis lines=center,
        hide obscured x ticks=true,
        grid = major,
        xtick={0,1,...,3},
        ytick={-2,-1,...,7},
        xlabel=$s$,
        ylabel=$t$,
        every axis y label/.style={at=(current axis.above origin),anchor=south},
        every axis x label/.style={at=(current axis.right of origin),anchor=west},
        ]
        \addplot[very thick,penColor,domain=0:3.3,samples=50] plot{x^2-2};
		
		\addplot[very thick,black,only marks,mark=*] coordinates{(1,1^2-2)};
		\node at (axis cs:1,1^2-2) [below right] {$P$};
		
		\addplot[very thick,red,domain=-0.3:3.3] plot{2*(x-1)-1};
		
		\node at (axis cs:1.2,5.1) [penColor,above] {$s=s(t)$};
    \end{axis}`
\end{tikzpicture}

The slope of the tangent line above is
\[
m_{\text{tan}} = \answer{2}.
\]


\end{exercise}
\end{exercise}
\end{exercise}
\end{exercise}
\end{exercise}
\end{exercise}
\end{document}