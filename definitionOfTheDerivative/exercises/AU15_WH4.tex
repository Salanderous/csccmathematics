\documentclass{ximera}


\graphicspath{
  {./}
  {ximeraTutorial/}
  {basicPhilosophy/}
}

\newcommand{\mooculus}{\textsf{\textbf{MOOC}\textnormal{\textsf{ULUS}}}}

\usepackage{tkz-euclide}\usepackage{tikz}
\usepackage{tikz-cd}
\usetikzlibrary{arrows}
\tikzset{>=stealth,commutative diagrams/.cd,
  arrow style=tikz,diagrams={>=stealth}} %% cool arrow head
\tikzset{shorten <>/.style={ shorten >=#1, shorten <=#1 } } %% allows shorter vectors

\usetikzlibrary{backgrounds} %% for boxes around graphs
\usetikzlibrary{shapes,positioning}  %% Clouds and stars
\usetikzlibrary{matrix} %% for matrix
\usepgfplotslibrary{polar} %% for polar plots
\usepgfplotslibrary{fillbetween} %% to shade area between curves in TikZ
\usetkzobj{all}
\usepackage[makeroom]{cancel} %% for strike outs
%\usepackage{mathtools} %% for pretty underbrace % Breaks Ximera
%\usepackage{multicol}
\usepackage{pgffor} %% required for integral for loops



%% http://tex.stackexchange.com/questions/66490/drawing-a-tikz-arc-specifying-the-center
%% Draws beach ball
\tikzset{pics/carc/.style args={#1:#2:#3}{code={\draw[pic actions] (#1:#3) arc(#1:#2:#3);}}}



\usepackage{array}
\setlength{\extrarowheight}{+.1cm}
\newdimen\digitwidth
\settowidth\digitwidth{9}
\def\divrule#1#2{
\noalign{\moveright#1\digitwidth
\vbox{\hrule width#2\digitwidth}}}






\DeclareMathOperator{\arccot}{arccot}
\DeclareMathOperator{\arcsec}{arcsec}
\DeclareMathOperator{\arccsc}{arccsc}

















%%This is to help with formatting on future title pages.
\newenvironment{sectionOutcomes}{}{}


\outcome{Recognize and distinguish between secant and tangent lines.}
\outcome{Understand the definition of the derivative at a point.}

\begin{document}
An oil tank is to be drained for cleaning. There are $V(t)$ gallons of oil left in the tank $t$ minutes after the draining began, where $V(t)=45(60-t)^2$.
\begin{exercise}
Find the \textbf{average rate} at which oil drains during the first $15$ minutes. 
\[
\answer{-4725}\unit{gal/m}
\]
\begin{exercise}
Find the \textbf{average rate} at which oil drains during the time interval $[10,15]$. 
\[
\answer{-4275}\unit{gal/m}
\]
\begin{exercise}
Find the \textbf{rate} at which oil is draining $15$ minutes after the draining began.
\[
\answer{-4050}\unit{gal/m}
\]
\begin{exercise}
Find the \textbf{average rate}, $AR(t)$, at which oil drains during the time interval $[15+ t,15]$ if $-1< t<0$. 
\[
AR(t)=\answer{45(t-90)}\unit{gal/m}
\]
Now find the \textbf{average rate}, $AR( t)$, at which oil drains during the time interval $[15,15+ t]$ if $0< t<1$. 
\[
AR(t)=\answer{45(t-90)}\unit{gal/m}
\]
\begin{exercise}
Using your result, compute the limit
\[
\lim_{t\to 0}AR(t)=\answer{-4050}\unit{gal/m}
\]
\begin{exercise}
What does this limit $\lim_{t\to 0}AR(t)$ represent?
\begin{multipleChoice}
\choice{The average rate of change at $t=0$.}
\choice{The quantity $V(15)$.}
\choice[correct]{The instantaneous rate of change of $V(t)$ at $t=15$.}
\choice{The derivative of $V(t)$ at $t=0$.}
\end{multipleChoice}
\begin{exercise}
Suppose the oil tank has been completely drained and is now being refilled. Let $F(t)$ denote the gallons of oil in the tank $t$ minutes after the refilling process began. Suppose we are told that the function $F$ is a \textbf{differentiable function} and that the tank is so large that it will take a while to fill. 

Find an expression for the \textbf{average rate}, $\bar{r}(t)$, at which oil flows into the tank in the first $t$ minutes for $t>0$. Express $\bar{r}$ \textbf{symbolically} in terms of the function $F(t)$, the variable $t$ and the quantity $F(0)$.
\begin{multipleChoice}
\choice{$\bar{r}(t)=\frac{F(0)-F(t)}{t}\unit{gal/m}$}
\choice{$\bar{r}(t)=F(0)-F(t)\unit{gal/m}$}
\choice{$\bar{r}(t)=\frac{F(0)}{t}\unit{gal/m}$}
\choice{$\bar{r}(t)=\frac{F(t)}{t}\unit{gal/m}$}
\choice[correct]{$\bar{r}(t)=\frac{F(t)-F(0)}{t}\unit{gal/m}$}
\choice{$\bar{r}(t)=\frac{F(t)}{t}\unit{gal/m}$}
\end{multipleChoice}
%% \begin{hint}
%% While the parameters of the problem have changed, the crux of the exercise has not. Refer to your thinking in the first to come up with a solution. 
%% \end{hint}
\begin{exercise}
Now let $\bar{r}_{\Delta}(t,\tau)$ be the function denoting the \textbf{average rate} at which oil drains in the time interval $[t,\tau]$ where $t$ is greater than $0$ and less than tau---that is, where $0<t<\tau$. Express $\bar{r}_{\Delta}$ \textbf{symbolically} in terms of the function $F$ and the variables $t_0$ and $t_f$.
\begin{multipleChoice}
\choice{$\bar{r}_\Delta(t,\tau)=\frac{F(t)-F(\tau)}{\tau-t}\unit{gal/m}$}
\choice{$\bar{r}_\Delta(t,\tau)=F(t)-F(\tau)\unit{gal/m}$}
\choice{$\bar{r}_\Delta(t,\tau)=\frac{F(t)}{\tau}\unit{gal/m}$}
\choice{$\bar{r}_\Delta(t,\tau)=\frac{F(\tau)}{t}\unit{gal/m}$}
\choice[correct]{$\bar{r}_\Delta(t,\tau)=\frac{F(\tau)-F(t)}{\tau-t}\unit{gal/m}$}
\choice{$\bar{r}_\Delta(t,\tau)=\frac{F(\tau)}{\tau}\unit{gal/m}$}
\end{multipleChoice}
\begin{exercise}
  What does $\lim_{t\to\tau} \bar{r}_\Delta(t,\tau)$ compute?
    \begin{selectAll}
      \choice{The average rate of change of $\bar{r}(\tau)$.}
      \choice{The average rate of change of $V$ at $\tau$.}
      \choice{The instantanous rate of change of $\bar{r}(\tau)$.}
      \choice[correct]{The instantanous rate of change of $V$ at time $\tau$.}
      \choice[correct]{$V'(\tau)$}
      \choice{$\bar{r}'(\tau)$.}
    \end{selectAll}
\end{exercise}
\end{exercise}
\end{exercise}
\end{exercise}
\end{exercise}
\end{exercise}
\end{exercise}
\end{exercise}
\end{exercise}
\end{document}
