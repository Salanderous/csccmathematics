\documentclass{ximera}


\graphicspath{
  {./}
  {ximeraTutorial/}
  {basicPhilosophy/}
}

\newcommand{\mooculus}{\textsf{\textbf{MOOC}\textnormal{\textsf{ULUS}}}}

\usepackage{tkz-euclide}\usepackage{tikz}
\usepackage{tikz-cd}
\usetikzlibrary{arrows}
\tikzset{>=stealth,commutative diagrams/.cd,
  arrow style=tikz,diagrams={>=stealth}} %% cool arrow head
\tikzset{shorten <>/.style={ shorten >=#1, shorten <=#1 } } %% allows shorter vectors

\usetikzlibrary{backgrounds} %% for boxes around graphs
\usetikzlibrary{shapes,positioning}  %% Clouds and stars
\usetikzlibrary{matrix} %% for matrix
\usepgfplotslibrary{polar} %% for polar plots
\usepgfplotslibrary{fillbetween} %% to shade area between curves in TikZ
\usetkzobj{all}
\usepackage[makeroom]{cancel} %% for strike outs
%\usepackage{mathtools} %% for pretty underbrace % Breaks Ximera
%\usepackage{multicol}
\usepackage{pgffor} %% required for integral for loops



%% http://tex.stackexchange.com/questions/66490/drawing-a-tikz-arc-specifying-the-center
%% Draws beach ball
\tikzset{pics/carc/.style args={#1:#2:#3}{code={\draw[pic actions] (#1:#3) arc(#1:#2:#3);}}}



\usepackage{array}
\setlength{\extrarowheight}{+.1cm}
\newdimen\digitwidth
\settowidth\digitwidth{9}
\def\divrule#1#2{
\noalign{\moveright#1\digitwidth
\vbox{\hrule width#2\digitwidth}}}






\DeclareMathOperator{\arccot}{arccot}
\DeclareMathOperator{\arcsec}{arcsec}
\DeclareMathOperator{\arccsc}{arccsc}

















%%This is to help with formatting on future title pages.
\newenvironment{sectionOutcomes}{}{}


\outcome{Understand the definition of the derivative at a point.}
\outcome{Estimate the slope of the tangent line graphically.}

\author{Nela Lakos \and Kyle Parsons}

\begin{document}
\begin{exercise}

Let $s(t) = t^3-2t^2$. The graph of $s$ is given below.

\begin{image}
  \begin{tikzpicture}
    \begin{axis}[
        xmin=-0.3,xmax=2.8,ymin=-1.3,ymax=3.3,
        clip=true,
        unit vector ratio*=1 1 1,
        axis lines=center,
        grid = major,
        ytick={-1,0,...,3},
    xtick={0,1,...,2},
        xlabel=$t$, ylabel=$s$,
        every axis y label/.style={at=(current axis.above origin),anchor=south},
        every axis x label/.style={at=(current axis.right of origin),anchor=west},
      ]
		\addplot[very thick,penColor,domain=-0.3:2.8,samples=50] plot{x^3-2*x^2};
		
		\addplot[only marks,mark=*] coordinates{(1,-1) (2,0)};
		\node at (axis cs:1,-1) [above right] {$A$};
		\node at (axis cs:2,0) [above left] {$B$};
		
%		\addplot[very thick,red,domain=-0.2:5.2] plot{5-x};
%		
		\node at (axis cs:1.5,1.5) {$s=s(t)$};

      \end{axis}`
  \end{tikzpicture}
\end{image}

The points $A$ and $B$ are two points on the graph of $s$.
\[
A=\left(1,\answer{-1}\right)\quad B=\left(2,\answer{0}\right)
\]

\begin{exercise}

The slope of the secant line through the points $A$ and $B$ is
\[
m_{\text{sec}}=\answer{1}.
\]

\begin{exercise}

\begin{image}
  \begin{tikzpicture}
    \begin{axis}[
        xmin=-0.3,xmax=2.8,ymin=-1.8,ymax=3.3,
        clip=true,
        unit vector ratio*=1 1 1,
        axis lines=center,
        grid = major,
        ytick={-1,0,...,3},
    xtick={0,1,...,2},
        xlabel=$t$, ylabel=$s$,
        every axis y label/.style={at=(current axis.above origin),anchor=south},
        every axis x label/.style={at=(current axis.right of origin),anchor=west},
      ]
		\addplot[very thick,penColor,domain=-0.3:2.8,samples=50] plot{x^3-2*x^2};
		
		\addplot[only marks,mark=*] coordinates{(1,-1) (2,0)};
		\node at (axis cs:1,-1) [above right] {$A$};
		\node at (axis cs:2,0) [above left] {$B$};
		
		\addplot[very thick,red,domain=-0.3:1.8] plot{-x};

		\node at (axis cs:1.5,1.5) {$s=s(t)$};

      \end{axis}`
  \end{tikzpicture}
\end{image}

The tangent line to the graph of $s$ at the point $A$ is shown above.  Looking closely at this line we can approximate its slope as
\[
m_{\text{tan}}\approx\answer{-1}.
\]

In terms of the function $s$, the limit we would have to evaluate in order to calculate the slope of the tangent line to $s$ at $t=1$ is
\[
m_{\text{tan}} = \lim_{t\to1}\frac{s(\answer{t})-s(1)}{\answer{t-1}}.
\]
When we substitute in the formula for $s$ we get the following limit which is in terms of $t$:
\[
m_{\text{tan}} = \lim_{t\to1}\frac{\answer{t^3-2t^2+1}}{\answer{t-1}}.
\]

\begin{exercise}

Assume that the function $s(t) = t^3-2t^2$ represents the position of an object at time $t$ moving along a line, where $s$ is measured in feet and $t$ is measured in seconds.  The average velocity of the object over the time interval $[1,2]$ is
\[
v_{\text{av}} = \answer{1} ft/s.
\]

Complete the following table with appropriate average velocities (in $ft/s$).

\[
\begin{array}{|c|c|}
\hline
\text{Time interval} & \text{Average velocity} \\\hline\hline
[1,2] & \answer{1} \\\hline
[1,1.5] & \answer{-0.25} \\\hline
[1,1.1] & \answer{-0.89} \\\hline
[0.5,1] & \answer{-1.25} \\\hline
[0.9,1] & \answer{-1.09} \\\hline
\end{array}
\]

Based on the above values a reasonable conjecture for the instantaneous velocity of the object at $t=1$ is
\[
v_{\text{inst}}\approx{\answer{-1}} ft/s.
\]

In terms of the function $s$, the limit we would have to evaluate in order to calculate the instantaneous velocity of the object at $t=1$ is
\[
v_{\text{inst}} = \lim_{t\to1}\frac{s(\answer{t})-s(1)}{\answer{t-1}}.
\]
Notice that this is exactly the same expression we would need to evaluate in order to calculate the slope of the tangent line to the graph of $s(t)$ at $t=1$.

\end{exercise}
\end{exercise}
\end{exercise}
\end{exercise}
\end{document}