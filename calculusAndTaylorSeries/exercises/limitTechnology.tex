\documentclass{ximera}


\graphicspath{
  {./}
  {ximeraTutorial/}
  {basicPhilosophy/}
}

\newcommand{\mooculus}{\textsf{\textbf{MOOC}\textnormal{\textsf{ULUS}}}}

\usepackage{tkz-euclide}\usepackage{tikz}
\usepackage{tikz-cd}
\usetikzlibrary{arrows}
\tikzset{>=stealth,commutative diagrams/.cd,
  arrow style=tikz,diagrams={>=stealth}} %% cool arrow head
\tikzset{shorten <>/.style={ shorten >=#1, shorten <=#1 } } %% allows shorter vectors

\usetikzlibrary{backgrounds} %% for boxes around graphs
\usetikzlibrary{shapes,positioning}  %% Clouds and stars
\usetikzlibrary{matrix} %% for matrix
\usepgfplotslibrary{polar} %% for polar plots
\usepgfplotslibrary{fillbetween} %% to shade area between curves in TikZ
\usetkzobj{all}
\usepackage[makeroom]{cancel} %% for strike outs
%\usepackage{mathtools} %% for pretty underbrace % Breaks Ximera
%\usepackage{multicol}
\usepackage{pgffor} %% required for integral for loops



%% http://tex.stackexchange.com/questions/66490/drawing-a-tikz-arc-specifying-the-center
%% Draws beach ball
\tikzset{pics/carc/.style args={#1:#2:#3}{code={\draw[pic actions] (#1:#3) arc(#1:#2:#3);}}}



\usepackage{array}
\setlength{\extrarowheight}{+.1cm}
\newdimen\digitwidth
\settowidth\digitwidth{9}
\def\divrule#1#2{
\noalign{\moveright#1\digitwidth
\vbox{\hrule width#2\digitwidth}}}






\DeclareMathOperator{\arccot}{arccot}
\DeclareMathOperator{\arcsec}{arcsec}
\DeclareMathOperator{\arccsc}{arccsc}

















%%This is to help with formatting on future title pages.
\newenvironment{sectionOutcomes}{}{}


\author{Jim Talamo}
\license{Creative Commons 3.0 By-bC}
%Example from stewart

\outcome{}


\begin{document}

\begin{exercise}
The following exercise is meant to be done using technology to perform calculations.

Consider the function $f(x) = \frac{\sin(\tan(x))-\tan(\sin(x))}{\arcsin(\arctan(x))-\arctan(\arcsin(x))}$.

Use a calculator or a computer to complete the table below.  Round your answer to four decimal places.

\begin{tabular}{llll}
$f(.1) = \answer[tolerance=.001]{1.1067359}$ & $f(.01) = \answer[tolerance=.001]{0.989637}$ & $f(.001) = \answer[tolerance=.001]{1.0000}$ & $f(.0001) = \answer{1.00000}$ 
\end{tabular}

Try some other values near $x=0$.  What do you see?

Now, use a calculator or computational software of your choice to graph $y=f(x)$ (A good option is Desmos).  Does it look like $\lim_{x \to 0} f(x)$ exists?

\begin{multipleChoice}
\choice[correct]{I have read the above and drawn the graph.}
\choice{I have not drawn the graph yet.}
\end{multipleChoice}

\begin{exercise}
Use technology to compute $f'(x)$.  The result should be quite unpleasant.  Thankfully, technology is well-equipped to handle Taylor polynomials.  Using a program of your choice, write down the second degree Taylor Polynomial $p_2(x)$ centered at $x=0$ for $f(x)$.

\[
p_2(x) = \answer{1}+\answer{0}x+\answer{\frac{5}{3}}x^2
\]

From this, we see that $\lim_{x \to 0} f(x) = \answer{1}$.
\end{exercise}
\end{exercise}
\end{document}
