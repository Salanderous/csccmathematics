\documentclass{ximera}


\graphicspath{
  {./}
  {ximeraTutorial/}
  {basicPhilosophy/}
}

\newcommand{\mooculus}{\textsf{\textbf{MOOC}\textnormal{\textsf{ULUS}}}}

\usepackage{tkz-euclide}\usepackage{tikz}
\usepackage{tikz-cd}
\usetikzlibrary{arrows}
\tikzset{>=stealth,commutative diagrams/.cd,
  arrow style=tikz,diagrams={>=stealth}} %% cool arrow head
\tikzset{shorten <>/.style={ shorten >=#1, shorten <=#1 } } %% allows shorter vectors

\usetikzlibrary{backgrounds} %% for boxes around graphs
\usetikzlibrary{shapes,positioning}  %% Clouds and stars
\usetikzlibrary{matrix} %% for matrix
\usepgfplotslibrary{polar} %% for polar plots
\usepgfplotslibrary{fillbetween} %% to shade area between curves in TikZ
\usetkzobj{all}
\usepackage[makeroom]{cancel} %% for strike outs
%\usepackage{mathtools} %% for pretty underbrace % Breaks Ximera
%\usepackage{multicol}
\usepackage{pgffor} %% required for integral for loops



%% http://tex.stackexchange.com/questions/66490/drawing-a-tikz-arc-specifying-the-center
%% Draws beach ball
\tikzset{pics/carc/.style args={#1:#2:#3}{code={\draw[pic actions] (#1:#3) arc(#1:#2:#3);}}}



\usepackage{array}
\setlength{\extrarowheight}{+.1cm}
\newdimen\digitwidth
\settowidth\digitwidth{9}
\def\divrule#1#2{
\noalign{\moveright#1\digitwidth
\vbox{\hrule width#2\digitwidth}}}






\DeclareMathOperator{\arccot}{arccot}
\DeclareMathOperator{\arcsec}{arcsec}
\DeclareMathOperator{\arccsc}{arccsc}

















%%This is to help with formatting on future title pages.
\newenvironment{sectionOutcomes}{}{}


\author{Jim Talamo}
\license{Creative Commons 3.0 By-bC}


\outcome{}


\begin{document}

\begin{exercise}
Given that $\ln(1+x) = \sum_{k=1}^{\infty} \frac{(-1)^{k+1}}{k}x^k$, find $ \lim_{x \to 0} \frac{x \sin(x) +2\cos(x)-2}{\ln(1+x^6)}
$

The limit is $\answer{\infty}$.

(Use $\infty$ or $-\infty$ where appropriate or ``DNE'' if the limit otherwise does not exist)

\begin{hint}
We can use a similar procedure as before:

\paragraph{Step 1:} Let's start by writing out the sum of the first several terms in the series centered at $x=0$ (where the limit is to be evaluated) for the numerator and the denominator:

\begin{question}
For the numerator, we use the usual composition rule to find sum of the first few nonzero terms in the series for $x \sin(x) +2\cos(x)-2$:

The sum of first three nonzero terms in the Taylor series centered at $x=0$ for $x \sin(x)$ is $\answer{x^2-\frac{1}{3!}x^4+\frac{1}{5!}x^6} +\ldots$.

The sum of first three nonzero terms in the Taylor series centered at $x=0$ for $2\cos(x)$ is $\answer{2-x^2+\frac{1}{12}x^4}+\ldots$.

Note that the sum here will be valid up to the $x^4$ term only since the series for $2\cos(x)$ has an $x^6$ term that has not been exhibited.  Furthermore, it will be necessary to exhibit this since the $x^2$ terms will cancel when we add these series together!

\begin{question}
The sum of first four nonzero terms in the Taylor series centered at $x=0$ for $2\cos(x)$ is $\answer{2-x^2+\frac{1}{12}x^4-\frac{2}{6!} x^6} + \ldots$

Thus, up to and including the the $x^6$ term:

\[
x \sin(x) +2\cos(x)-2 = \answer{-\frac{1}{12}x^4+\frac{1}{180}x^6} +\ldots
\]

Note that we may need more terms, but we can always exhibit more if necessary.  
\end{question}
\end{question}


\begin{question}
For the denominator, the sum of the first three nonzero terms in the Taylor series centered at $x=0$ can be found using the usual rules as well:

\[
\ln(1+x) = \answer{x-\frac{1}{2}x^2+\frac{1}{3}x^3} +\ldots 
\]

So: $\ln(1+x^6) = \answer{x^6-\frac{1}{2}x^{12}+\frac{1}{3}x^{18}} +\ldots $


\end{question}


\begin{question}
Note that we may need more terms, but we can always exhibit more if necessary.  

\paragraph{Step 2:} Now that we have the series, let's substitute them into the limit:

\[
\lim_{x \to 0} \frac{x \sin(x) -2\cos(x)+2}{\cos(x)-1} = \lim_{x \to 0} \frac{\answer{-\frac{1}{12}x^4+\frac{1}{180}x^6}+\ldots}{ \answer{x^6-\frac{1}{2}x^{12}+\frac{1}{3}x^{18}} +\ldots}
\]

\paragraph{Step 3:} We can proceed by factoring out the dominant terms from the numerator and denominator:

As $x \to 0$, the dominant power of $x$ in the numerator is $x^{\answer{4}}$.

As $x \to 0$, the dominant power of $x$ in the denominator is $x^{\answer{6}}$.

\begin{question}
We can thus write:
\[
L = \lim_{x \to 0} \frac{-\frac{1}{12}x^4+\frac{1}{180}x^6+\ldots}{  x^6-\frac{1}{2}x^{12}+\frac{1}{3}x^{18} +\ldots}  = \lim_{x \to 0} \frac{ x^4 \cdot \left[\answer{-\frac{1}{12}+\frac{1}{180}x^2}+\ldots \right]}{x^6 \cdot \left[\answer{1-\frac{1}{2}x^{6}+\frac{1}{3}x^{12}} +\ldots\right]}
\]

\begin{question}
 Now, since we took out the dominant power of $x$ in both the numerator and denominator separately, the limits of the functions inside of the square brackets will be constant.  Indeed, we can write the limit as:

\[
L = \lim_{x \to 0} \frac{x^4}{x^6} \cdot \frac{\left[-\frac{1}{12}+\frac{1}{180}x^2+\ldots\right]}{\left[1-\frac{1}{2}x^{6}+\frac{1}{3}x^{12} +\ldots \right]}
\]

\paragraph{Step 4:} Simplify the expression above:

Note $\frac{x^4}{x^6} = x^{\answer{-2}} \to \answer{\infty}$ as $x \to 0$, while $\frac{\left[-\frac{1}{12}+\frac{1}{180}x^2+\ldots\right]}{\left[1-\frac{1}{2}x^{6}+\frac{1}{3}x^{12} +\ldots \right]} \to \answer{-\frac{1}{12}}$ as $x \to 0$.  

Thus, $L = \answer{-\infty}$.

\end{question}
\end{question}
\end{question}
\end{hint}
\end{exercise}
\end{document}
