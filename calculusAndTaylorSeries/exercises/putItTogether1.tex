\documentclass{ximera}


\graphicspath{
  {./}
  {ximeraTutorial/}
  {basicPhilosophy/}
}

\newcommand{\mooculus}{\textsf{\textbf{MOOC}\textnormal{\textsf{ULUS}}}}

\usepackage{tkz-euclide}\usepackage{tikz}
\usepackage{tikz-cd}
\usetikzlibrary{arrows}
\tikzset{>=stealth,commutative diagrams/.cd,
  arrow style=tikz,diagrams={>=stealth}} %% cool arrow head
\tikzset{shorten <>/.style={ shorten >=#1, shorten <=#1 } } %% allows shorter vectors

\usetikzlibrary{backgrounds} %% for boxes around graphs
\usetikzlibrary{shapes,positioning}  %% Clouds and stars
\usetikzlibrary{matrix} %% for matrix
\usepgfplotslibrary{polar} %% for polar plots
\usepgfplotslibrary{fillbetween} %% to shade area between curves in TikZ
\usetkzobj{all}
\usepackage[makeroom]{cancel} %% for strike outs
%\usepackage{mathtools} %% for pretty underbrace % Breaks Ximera
%\usepackage{multicol}
\usepackage{pgffor} %% required for integral for loops



%% http://tex.stackexchange.com/questions/66490/drawing-a-tikz-arc-specifying-the-center
%% Draws beach ball
\tikzset{pics/carc/.style args={#1:#2:#3}{code={\draw[pic actions] (#1:#3) arc(#1:#2:#3);}}}



\usepackage{array}
\setlength{\extrarowheight}{+.1cm}
\newdimen\digitwidth
\settowidth\digitwidth{9}
\def\divrule#1#2{
\noalign{\moveright#1\digitwidth
\vbox{\hrule width#2\digitwidth}}}






\DeclareMathOperator{\arccot}{arccot}
\DeclareMathOperator{\arcsec}{arcsec}
\DeclareMathOperator{\arccsc}{arccsc}

















%%This is to help with formatting on future title pages.
\newenvironment{sectionOutcomes}{}{}


\author{Jim Talamo}
\license{Creative Commons 3.0 By-bC}


\outcome{}


\begin{document}

\begin{exercise}
Consider the function $f(x) = \sum_{k=0}^{\infty} \frac{k+5}{16^k} (x-3)^{4k}$

\begin{exercise}
Find $f(3)$.

\[
f(3) = \answer{5}
\]
\begin{hint}
Write out a few terms in the sum.  What happens when you evaluate at $x=3$?
\end{hint}
\end{exercise}

%%%%%%%%%%%%%%%%%%
\begin{exercise}
The radius of convergence for the series is $\answer{2}$.

\begin{hint}
You can either use the Ratio Test or the rules for compositions to find this.
\end{hint}

\end{exercise}
%%%%%%%%%%%%%%%%%%
\begin{exercise}
Select all of the following series that converge:

\begin{selectAll}
\choice{The series represented by $f(-2)$}
\choice{The series represented by $f(0)$}
\choice[correct]{The series represented by $f(2)$}k+5
\choice[correct]{The series represented by $f(4)$}
\choice{The series represented by $f(5)$}
\choice{The series represented by $f(7)$}
\end{selectAll}

\begin{hint}
The radius of convergence is $\answer{2}$ and the series is centered at $x=\answer{3}$.  Thus:

The lefthand endpoint for the interval of convergence is $x=\answer{1}$.

The righthand endpoint for the interval of convergence is $x=\answer{5}$.

Note that when $x=5$, we must check the series represented by $f(5)$ separately since this is at the endpoint of the interval of convergence.  Note:

\[
f(5) = \sum_{k=0}^{\infty}  \frac{k+5}{16^k} (5-3)^{4k} = \sum_{k=0}^{\infty}  \frac{k+5}{16^k} (16)^k = \sum_{k=0}^{\infty} (k+5)
\]

Which convergence test can be immediately applied now?
\end{hint}

\end{exercise}
%%%%%%%%%%%%%%%%%%
\begin{exercise}
Find $f^{(52)}(3)$.

\[
f^{(52)}(3) = \answer{\frac{(18)(52)!}{(16)^{13}}}
\]
\end{exercise}
%%%%%%%%%%%%%%%%%%
\begin{exercise}
Find $\lim_{x \to 0} \frac{f(x)-5}{2x^4-8x^8}$

\[
\lim_{x \to 0} \frac{f(x+3)-5}{2x^4-8x^8} = \answer{\frac{3}{16}}
\]
\end{exercise}


\end{exercise}
\end{document}
