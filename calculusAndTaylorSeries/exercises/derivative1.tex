\documentclass{ximera}


\graphicspath{
  {./}
  {ximeraTutorial/}
  {basicPhilosophy/}
}

\newcommand{\mooculus}{\textsf{\textbf{MOOC}\textnormal{\textsf{ULUS}}}}

\usepackage{tkz-euclide}\usepackage{tikz}
\usepackage{tikz-cd}
\usetikzlibrary{arrows}
\tikzset{>=stealth,commutative diagrams/.cd,
  arrow style=tikz,diagrams={>=stealth}} %% cool arrow head
\tikzset{shorten <>/.style={ shorten >=#1, shorten <=#1 } } %% allows shorter vectors

\usetikzlibrary{backgrounds} %% for boxes around graphs
\usetikzlibrary{shapes,positioning}  %% Clouds and stars
\usetikzlibrary{matrix} %% for matrix
\usepgfplotslibrary{polar} %% for polar plots
\usepgfplotslibrary{fillbetween} %% to shade area between curves in TikZ
\usetkzobj{all}
\usepackage[makeroom]{cancel} %% for strike outs
%\usepackage{mathtools} %% for pretty underbrace % Breaks Ximera
%\usepackage{multicol}
\usepackage{pgffor} %% required for integral for loops



%% http://tex.stackexchange.com/questions/66490/drawing-a-tikz-arc-specifying-the-center
%% Draws beach ball
\tikzset{pics/carc/.style args={#1:#2:#3}{code={\draw[pic actions] (#1:#3) arc(#1:#2:#3);}}}



\usepackage{array}
\setlength{\extrarowheight}{+.1cm}
\newdimen\digitwidth
\settowidth\digitwidth{9}
\def\divrule#1#2{
\noalign{\moveright#1\digitwidth
\vbox{\hrule width#2\digitwidth}}}






\DeclareMathOperator{\arccot}{arccot}
\DeclareMathOperator{\arcsec}{arcsec}
\DeclareMathOperator{\arccsc}{arccsc}

















%%This is to help with formatting on future title pages.
\newenvironment{sectionOutcomes}{}{}


\author{Jim Talamo}
\license{Creative Commons 3.0 By-bC}

\outcome{Compute derivatives of Taylor series}


\begin{document}

\begin{exercise}
Suppose that the Taylor series for a certain function $f(x)$ is given by $f(x) = \sum_{k=0}^{\infty} \frac{3}{k!} x^{2k+3}$.  

Find the Taylor series for $f''(x)$ centered at $x=0$.

\[
f''(x) = \sum_{k=0}^{\infty} \answer{\frac{3(2k+3)(2k+2)}{k!}} \cdot x^{\answer{2k+1}}
\]


\begin{hint}
To find the series for $f''(x)$, we begin by finding the series for $f'(x)$:

\[
f'(x) = \frac{d}{dx} \left[\sum_{k=0}^{\infty}  \frac{3}{k!} x^{2k+3}  \right] = \sum_{k=0}^{\infty}  \frac{d}{dx} \left[ \frac{3}{k!} x^{2k+3}  \right]
\]
(interchange ``$\sum$'' and ``$\frac{d}{dx}$'')

We can interchange the constants and the derivative:

\[
\sum_{k=0}^{\infty}  \frac{d}{dx} \left[\frac{3}{k!} x^{2k+3}\right] = \sum_{k=0}^{\infty} \answer{\frac{3}{k!}} \frac{d}{dx} \left[ x^{2k+3}  \right]
\]

(For each $k$, the expression in front of the power of $x$ is constant, so it can be pulled outside of the derivative)

Now, all we have to do is differentiate powers of $x$.  Indeed:

\[
\sum_{k=0}^{\infty}\frac{3}{k!} \frac{d}{dx} \left[ x^{2k+3}  \right] = \sum_{k=0}^{\infty} \frac{3}{k!} \cdot\answer{(2k+3)x^{2k+2}}
\]

Note that since the coefficient when $k=0$ is nonzero, we should still keep the lower index of summation as $k=0$.  

Now, repeat this process to find the series for $f''(x)$.

\end{hint}

%%%%%%%%%%%%

\begin{exercise}
Find the radius of convergence for $f(x)$.

The radius of convergence is $\answer{\infty}$.

(Use $\infty$ when appropriate)

\begin{exercise}
Suppose now that we want to find the radius of convergence for the series represented by $f''(x)$.  We can do this two ways:

\begin{exercise}
Method 1: Use the Ratio Test.

The limit needed to determine the radius of convergence for $f''(x) = \sum_{k=0}^{\infty} \frac{3(2k+3)(2k+2)}{k!}x^{2k+1}$ is:

\[
L(x) = \lim_{n \to \infty} \frac{3(2(n+1)+3)(2(n+1)+2)x^{2(n+1)}}{(n+1)!} \cdot \frac{n!}{3(2n+3)(2n+2)x^{2n}}
\]
Evaluating this, we find $L(x) = \answer{0}$ and thus the radius of convergence is $\answer{\infty}$.

(Use $\infty$ when appropriate)
\end{exercise}

\begin{exercise}
Method 2: We can note that differentiating does not change the radius of convergence of a series.  Since the radius of convergence for the series represented by $f(x)$ is infinite, we expect the radius of convergence for $f''(x)$ to be $\answer{\infty}$.
\end{exercise}
\end{exercise}
\end{exercise}
\end{exercise}
\end{document}
