\documentclass{ximera}


\graphicspath{
  {./}
  {ximeraTutorial/}
  {basicPhilosophy/}
}

\newcommand{\mooculus}{\textsf{\textbf{MOOC}\textnormal{\textsf{ULUS}}}}

\usepackage{tkz-euclide}\usepackage{tikz}
\usepackage{tikz-cd}
\usetikzlibrary{arrows}
\tikzset{>=stealth,commutative diagrams/.cd,
  arrow style=tikz,diagrams={>=stealth}} %% cool arrow head
\tikzset{shorten <>/.style={ shorten >=#1, shorten <=#1 } } %% allows shorter vectors

\usetikzlibrary{backgrounds} %% for boxes around graphs
\usetikzlibrary{shapes,positioning}  %% Clouds and stars
\usetikzlibrary{matrix} %% for matrix
\usepgfplotslibrary{polar} %% for polar plots
\usepgfplotslibrary{fillbetween} %% to shade area between curves in TikZ
\usetkzobj{all}
\usepackage[makeroom]{cancel} %% for strike outs
%\usepackage{mathtools} %% for pretty underbrace % Breaks Ximera
%\usepackage{multicol}
\usepackage{pgffor} %% required for integral for loops



%% http://tex.stackexchange.com/questions/66490/drawing-a-tikz-arc-specifying-the-center
%% Draws beach ball
\tikzset{pics/carc/.style args={#1:#2:#3}{code={\draw[pic actions] (#1:#3) arc(#1:#2:#3);}}}



\usepackage{array}
\setlength{\extrarowheight}{+.1cm}
\newdimen\digitwidth
\settowidth\digitwidth{9}
\def\divrule#1#2{
\noalign{\moveright#1\digitwidth
\vbox{\hrule width#2\digitwidth}}}






\DeclareMathOperator{\arccot}{arccot}
\DeclareMathOperator{\arcsec}{arcsec}
\DeclareMathOperator{\arccsc}{arccsc}

















%%This is to help with formatting on future title pages.
\newenvironment{sectionOutcomes}{}{}


\author{Jim Talamo}
\license{Creative Commons 3.0 By-bC}

\outcome{Compute derivatives of Taylor series}


\begin{document}

\begin{exercise}
Suppose that the Taylor series for a certain function $f(x)$ is given by $f(x) = \sum_{k=0}^{\infty} \frac{2^k}{k+1} x^k$.  


Suppose that we want to find the third degree Taylor polynomial for $f'(x)$, which we will denote by $p_3(x)$.  We can do this two ways.

\begin{selectAll}
\choice[correct]{write out several terms and differentiate}
\choice[correct]{work in summation notation.}
\end{selectAll}
(select both answers to continue)


Method 1: Work with the terms explicitly.

First, let's write out the sum of the first several nonzero terms in the series for $f(x)$, up to and including the $x^5$ term:

\[
f(x) = \answer{1+x+\frac{4}{3}x^2+2x^3+\frac{16}{5}x^4+\frac{16}{3}x^5}+\ldots
\]

Taking the derivative term-by-term shows:

\[
f'(x) = \answer{1+\frac{8}{3}x+6x^2+\frac{64}{5}x^3+\frac{80}{3}x^4}+\ldots
\]

From this, we can extract the third degree Taylor polynomial for $f'(x)$:

\[
p_3(x) = \answer{1+\frac{8}{3}x+6x^2+\frac{64}{5}x^3}
\] 


%%%%%%%%%%%%%%%%%%%%

Method 2: Work in summation notation.

When taking derivatives of the explicit terms above, you likely differentiated each term, then added the results.  In summation notation, this is represented by the equality.


\[
\frac{d}{dx} \left[\sum_{k=0}^{\infty}  \frac{2^k}{k+1} x^k  \right] = \sum_{k=0}^{\infty}  \frac{d}{dx} \left[\frac{2^k}{k+1} x^k  \right]
\]
(interchange ``$\sum$'' and ``$\frac{d}{dx}$'')

To take the derivative of any of the single explicit terms in the preceding exercise, you likely differentiated the power of $x$ then multiplied by the coefficient in front (i.e. when differentiating $\frac{4}{3}x^2$, you likely noted that the derivative of $x^2$ is $\answer{2x}$, so the derivative of $\frac{4}{3}x^2$ is $\frac{4}{3} \cdot 2x$.  In summation notation, this looks like:

\[
\sum_{k=0}^{\infty}  \frac{d}{dx} \left[\frac{2^k}{k+1} x^k\right] = \sum_{k=0}^{\infty} \answer{\frac{2^k}{k+1}} \frac{d}{dx} \left[ x^k  \right]
\]

(For each $k$, the expression in front of the power of $x$ is constant, so it can be pulled outside of the derivative)

Now, all we have to do is differentiate powers of $x$.  Indeed:

\[
\sum_{k=0}^{\infty}\frac{2^k}{k+1} \frac{d}{dx} \left[ x^k  \right] = \sum_{k=0}^{\infty} \frac{2^k}{k+1} \cdot(\answer{kx^{k-1}})
\]

Note that since the coefficient when $k=0$ is $\answer{0}$, we may freely start the summation at $k=1$.  

From this, we can extract the third degree Taylor polynomial for $f'(x)$ by writing out the terms in the above sum until we have exhibited the sum of the terms up to and including $x^3$.  Doing this gives:

\[
p_3(x) = \answer{1+\frac{8}{3}x+6x^2+\frac{64}{5}x^3}
\] 

Do the results match?

\begin{multipleChoice}
\choice[correct]{Yes}
\choice{No}
\end{multipleChoice}

One of the major points of this exercise is to show that the idea of differentiating a Taylor series term-by-term is something you've been doing for quite a while now.  The notation may seem intimidating at first, but the actual mathematical procedure is nothing new!


\end{exercise}
\end{document}
