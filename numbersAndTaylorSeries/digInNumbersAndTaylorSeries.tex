\documentclass{ximera}


\graphicspath{
  {./}
  {ximeraTutorial/}
  {basicPhilosophy/}
}

\newcommand{\mooculus}{\textsf{\textbf{MOOC}\textnormal{\textsf{ULUS}}}}

\usepackage{tkz-euclide}\usepackage{tikz}
\usepackage{tikz-cd}
\usetikzlibrary{arrows}
\tikzset{>=stealth,commutative diagrams/.cd,
  arrow style=tikz,diagrams={>=stealth}} %% cool arrow head
\tikzset{shorten <>/.style={ shorten >=#1, shorten <=#1 } } %% allows shorter vectors

\usetikzlibrary{backgrounds} %% for boxes around graphs
\usetikzlibrary{shapes,positioning}  %% Clouds and stars
\usetikzlibrary{matrix} %% for matrix
\usepgfplotslibrary{polar} %% for polar plots
\usepgfplotslibrary{fillbetween} %% to shade area between curves in TikZ
\usetkzobj{all}
\usepackage[makeroom]{cancel} %% for strike outs
%\usepackage{mathtools} %% for pretty underbrace % Breaks Ximera
%\usepackage{multicol}
\usepackage{pgffor} %% required for integral for loops



%% http://tex.stackexchange.com/questions/66490/drawing-a-tikz-arc-specifying-the-center
%% Draws beach ball
\tikzset{pics/carc/.style args={#1:#2:#3}{code={\draw[pic actions] (#1:#3) arc(#1:#2:#3);}}}



\usepackage{array}
\setlength{\extrarowheight}{+.1cm}
\newdimen\digitwidth
\settowidth\digitwidth{9}
\def\divrule#1#2{
\noalign{\moveright#1\digitwidth
\vbox{\hrule width#2\digitwidth}}}






\DeclareMathOperator{\arccot}{arccot}
\DeclareMathOperator{\arcsec}{arcsec}
\DeclareMathOperator{\arccsc}{arccsc}

















%%This is to help with formatting on future title pages.
\newenvironment{sectionOutcomes}{}{}


\outcome{Use Taylor series to approximate constants.}
\outcome{Use Taylor series to compute limits.}
\outcome{Use Taylor series to sum series.}

\title[Dig-In:]{Numbers and Taylor series}

\begin{document}
\begin{abstract}
  Taylor series are a computational tool.
\end{abstract}
\maketitle

\section{Using series to approximate special constants}

You probably know that $\pi \approx 3.14159$.  Have you ever wondered
how this kind of approximation is obtained?  There are many ways to
do it, but one way is to use Taylor series!  Here is a plan for
approximating $\pi$ using a series:
\begin{itemize}
\item Find a function which takes a nice number (like $0$, or $1$,
    or $\frac{1}{2}$) as an input and returns something involving
    $\pi$ (like $\pi$, or $\frac{\pi}{2}$, or $1/\pi$) as an output.
  \item Make sure that this function has a Taylor series which we can
    compute easily.
  \item Plug the nice number into the Taylor series: We now have an
    algorithm for approximating $\pi$.
\end{itemize} 
The idea is to use the following fact:
\[
\arctan(1) = \frac{\pi}{4}
\]
If you recall, we found that Taylor series for arctangent already, by
substituting $z=-x^2$ into the geometric series $\frac{1}{1-z}$ to
find the series for $\frac{1}{1+x^2}$, and then integrating this
series to find the series for $\arctan(x)$:
\[
\arctan(x) = x-\frac{x^3}{3}+\frac{x^5}{5}-\frac{x^7}{7}+\cdots
\]
And so 
\[
\frac{\pi}{4} = \arctan(1) = 1-\frac{1}{3}+\frac{1}{5}-\frac{1}{7}+\cdots
\]
Thus
\[
\pi =  4-\frac{4}{3}+\frac{4}{5}-\frac{4}{7}+\cdots
\]
Really cool!

\begin{example}
  How many terms of
  \[
  4-\frac{4}{3}+\frac{4}{5}-\frac{4}{7}+\cdots
  \]
  are required to approximate $\pi$ within $\frac{1}{100}$?
  \begin{explanation}
    Since this is an alternating series, we can use the alternating
    series approximation theorem. In this case the theorem says that
    since
    \[
    \sum_{k=0}^\infty a_k = \sum_{k=0}^\infty \frac{(-1)^k4}{2k+1}
    \]
    We need to find $k$ such that
    \begin{align*}
      |a_{k+1}| &<\frac{1}{100}\\
      \answer[given]{\frac{4}{2(k+1)+1}} &< \frac{1}{100}\\
      \answer[given]{\frac{2k+3}{4}} &> 100\\
      \answer[given]{2k+3} &> 400\\
      \answer[given]{2k} &> 397\\
      \answer[given]{k} &> 397/2
    \end{align*}
    Since $k$ is a whole number, we would need to sum
    $\answer[given]{199}$ terms just to be assured that you've found
    an approximation within $\frac{1}{100}$ of $\pi$! Even then, you
    have to be careful not to accumulate too many rounding errors when
    performing the computations.
  \end{explanation}
\end{example}


Part of the reason the series above converges so slowly is that the
series is not absolutely convergent.  Also $1$ is the right endpoint
of the interval of convergence for $\arctan(x)$: It just barely makes
the cut between convergent and not convergent! There are more advanced
series for $\pi$ which converge much more quickly, for example
Ramanujan's formula:
\[
\frac{1}{\pi} = \frac{2 \sqrt 2}{99^2} \sum_{k=0}^\infty \left(\frac{(4k)!}{(k!)^4} \cdot \frac{26390k+1103}{396^{4k}}\right)
\]
This series computes \textbf{eight} additional decimal places for
$\pi$ with each term of the series. Amazing! What about other
approximations?  You already know one way to compute the number $e$:
As
\[
e = \lim_{n\to \infty} \left( 1+\frac{1}{n}\right)^n.
\]
Now we can also approximate $e$ using series!

\begin{example}
  Use the power series 
  \[
  e^x = 1+x+\frac{x^2}{2!}+\frac{x^3}{3!}+\cdots
  \]
  to approximate $e$ within $1/1000$.
  \begin{explanation}
    We know that
	\[
	e = 1+1+\frac{1}{2!}+\frac{1}{3!}+\frac{1}{4!}+\frac{1}{5!}+\cdots
	\]
        By Taylor's theorem, we know that
        \[
        e^1 = p_n(1) + R_n(1)
        \]
        and since the largest any derivative of $e^x$ is on the
        interval $[0,1]$ is $\answer[given]{e^1}$,
        \[
        R_n(1) < \frac{e^1}{(n+1)!} 
        \]
        since $e^1<3$, we want
        \[
        \frac{3}{(n+1)!} \le \frac{1}{1000}
        \]
        which means (verified with a calculator) $n=\answer[given]{6}$. Hence
        \[
        \sum_{k=0}^{\answer[given]{6}}\frac{1}{k!}
        \]
        approximates $e$ within $1/1000$.
  \end{explanation}
\end{example}

Finally, we'll show how to use Taylor series to obtain arbitrary
precision when dealing with square-roots.

\begin{example}
  Use a Taylor series to approximate $\sqrt[3]{50}$ with four
  terms. Without computing the actual value of $\sqrt[3]{50}$, what
  accuracy can you guarantee?
  \begin{explanation}
    We start by finding a Taylor series for $f(x) = \sqrt[3]{x}$
    centered around $x=64$. We choose $64$ because it is a perfect
    cube that is near $50$. Now we make a table of derivatives going
    up to the $4$th derivative of $f$, since we will need this to
    compute the precision of our estimation.
    \[
    \begin{array}{lcl}
      f(x) = x^{1/3} & \Rightarrow &f(64) = 4\\
      f'(x) = \answer[given]{x^{-2/3}/3} & \Rightarrow & f'(64) = \answer[given]{1/48}\\
      f''(x) = \answer[given]{-2x^{-5/3}/9} &\Rightarrow &f''(64) = \answer[given]{-1/4608}\\
      f'''(x) = \answer[given]{10x^{-8/3}/27} &\Rightarrow &f'''(64) = \answer[given]{5/884736}\\
      f^{(4)}(x) = \answer[given]{-80 x^{-11/3}/81} &\Rightarrow &f^{(4)}(64) = \answer[given]{-5/21233664}
    \end{array}
    \]
    So our Taylor series is
    \[
    4 + \frac{\answer[given]{x-64}}{48} - \frac{(\answer[given]{x-64})^2}{2!\cdot 4608} + \frac{5(\answer[given]{x-64})^3}{3!\cdot 884736}
    \]
    and evaluating at $x=50$ we find:
    \[
    4 + \frac{\answer[given]{50-64}}{48} - \frac{(\answer[given]{50-64})^2}{2!\cdot 4608} + \frac{5(\answer[given]{50-64})^3}{3!\cdot 884736} \approx 3.68448
    \]
    Since this series is alternating, the remainder is bounded by the next term:
    \[
    \left|\frac{5(\answer[given]{50-64})^4}{4!\cdot 21233664}\right| < 0.0004 
    \]
    Hence our estimation is within $0.0004$ of the true value.
  \end{explanation}
\end{example}




\section{Taylor series as a tool to evaluate limits}

Taylor series can be used like L'H\^{o}pital's rule on steroids when
evaluating limits.

First lets see why Taylor's series subsumes L'H\^{o}pital's rule: Say
$f(a) = g(a) = 0$, and we are interested in
\[
\lim_{x \to a}\frac{f(x)}{g(x)}.
\]
Then using Taylor series
\begin{align*}
	\lim_{x \to a} \frac{f(x)}{g(x)} &= \lim_{x \to a} \frac{f(a)+f'(a)(x-a)+\frac{f''(a)}{2!}(x-a)^2+\cdots}{g(a)+g'(a)(x-a)+\frac{g''(a)}{2!}(x-a)^2+\cdots}\\
		&=  \lim_{x \to a} \frac{0+f'(a)(x-a)+\frac{f''(a)}{2!}(x-a)^2+\cdots}{0+g'(a)(x-a)+\frac{g''(a)}{2!}(x-a)^2+\cdots}\\
		&=  \lim_{x \to a} \frac{f'(a)+\frac{f''(a)}{2!}(x-a)+\frac{f^{(3)}(a)}{3!}(x-a)^2+\cdots}{g'(a)+\frac{g''(a)}{2!}(x-a)+\frac{g^{(3)}(a)}{3!}(x-a)^2+\cdots}\\
		&=\frac{f'(a)}{g'(a)}
\end{align*}
As long as $g'(a) \neq 0$.  This is exactly L'H\^{o}pital's rule!
Let's use this in a L'H\^{o}pital's rule situation, without invoking
L'H\^{o}pital's rule directly:

\begin{example}
  Compute:
  \[
  \lim_{x \to 0} \frac{e^x - 1}{x}
  \]
  \begin{explanation}
    Just replace $e^x$ with its Maclaurin series:
    \begin{align*}
      \lim_{x \to 0} \frac{e^x - 1}{x} &= \lim_{x \to 0} \frac{(1+x+\frac{x^2}{2!}+\frac{x^3}{3!}+\cdots) - 1}{x}\\
      &= \lim_{x \to 0} \frac{x+\frac{x^2}{2!} +\answer[given]{\frac{x^3}{3!}}+\cdots}{x}\\
      &=\lim_{x \to 0} 1+\frac{x}{2!}+\answer[given]{\frac{x^2}{3!}}+\cdots\\
      &=\answer[given]{1}
    \end{align*}
  \end{explanation}
\end{example}

We can also use this approach to limit evaluation in cases where
L'H\^{o}pital's rule would need to be applied multiple times.

\begin{example}
  Compute:
  \[
  \lim_{x \to 0} \frac{\sin(x)-x}{x^3\cos(x)} 
  \]
  \begin{explanation}
    Just replace sine and cosine with their Maclaurin series:
    \begin{align*}
      \lim_{x \to 0} \frac{\sin(x)-x}{x^3\cos(x)} &= \lim_{x \to 0} \frac{(x-\frac{x^3}{3!}+\frac{x^5}{5!}-\cdots)-x}{x^3(1-\frac{x^2}{2!}+\frac{x^4}{4!}-\cdots)}\\
      &= \lim_{x \to 0} \frac{-\frac{x^3}{3!}+\frac{x^5}{5!}-\cdots}{x^3 - \frac{x^5}{2!}+\frac{x^7}{4!}-\cdots}\\
      &=\lim_{x \to 0} \frac{-\frac{1}{6}+\frac{x^2}{3!}-\cdots}{1-\frac{x^2}{2!}+\frac{x^4}{4!}-\cdots}\\
      &=\answer[given]{\frac{-1}{6}}
    \end{align*}
  \end{explanation}
\end{example}

It might not seem like Taylor series would be much help evaluating
limits at infinity, since Taylor series are all about approximating a
function close to some given finite point.  It turns out that we can
still use Taylor series to study function behavior at infinity by
transforming the function:
\begin{quote}
  Composing with $\frac{1}{z}$ ``moves infinity to zero,''
\end{quote}
and we can then use Maclaurin series to study the behavior.  Let's see
that in action:

\begin{example}
  Compute:	
  \[
  \lim_{x \to \infty} \left(2x^8\cos\left(\frac{1}{x^2}\right) - x^4-2x^8\right)
  \]
  \begin{explanation}
    Letting $x=\frac{1}{t}$ we have
	\[
	\lim_{x \to \infty} \left(2x^8\cos\left(\frac{1}{x^2}\right) - x^4-2x^8\right)=\lim_{t \to 0} \frac{2\cos\left(t^2\right) - t^4-2}{t^8}
	\]
	You should be able to handle it from here using the Taylor
        series method:
	\begin{align*}
	  \lim_{t \to 0} \frac{2\cos(t^2) - t^4-2}{t^8} &= \lim_{t \to 0} \frac{2(1-\frac{t^4}{2!}+\frac{t^8}{4!}-\frac{t^{12}}{6!}+\cdots) - t^4-2}{t^8}\\
	  &=  \lim_{t \to 0} \frac{\frac{2t^8}{4!}-\frac{2t^{12}}{12!}+\cdots}{t^8}\\
	  &=\lim_{t \to 0} \frac{2}{4!}-\frac{2t^4}{6!} +\cdots\\
	  &=\answer[given]{\frac{1}{12}}
        \end{align*}
        So
	\[
	\lim_{t \to 0} \frac{2\cos(t^2) - t^4-2}{t^8} = \answer[given]{\frac{1}{12}}
	\]
  \end{explanation}	
\end{example}

\section{Evaluating series}

Sometimes we get a series as an answer to some problem (For instance,
in the next section we will find series solutions to differential
equations), but we would really like a \textit{closed form expression}. A \textbf{closed form expression}
is one that can be evaluated in a \textit{finite} number of steps. For example
\[
\underbrace{\sum_{n=0}^\infty x^n}_\text{an infinite sum}  = \underbrace{\frac{1}{1-x}}_{\text{closed form expression}}
\]




This is not always possible, but sometimes if we are insightful we can
manipulate a given series into form where we can recognize it.


\begin{example}
  Compute the sum:
  \[
  \sum_{k=0}^\infty \frac{(-1)^k}{(2n)!\cdot 2^n}
  \]
  \begin{explanation}
    First notice that this series
    \[
    \sum_{k=0}^\infty \frac{(-1)^k}{(2n)!\cdot 2^n} = 1 - \frac{1}{2!\cdot 2} + \frac{1}{(4)!\cdot 2^2}-\frac{1}{(6)!\cdot 2^3} + \cdots
    \]
    looks a bit like the power series for cosine centered at $x=0$:
    \[
    \cos(x) = 1-\frac{x^2}{2!} + \frac{x^4}{4!} - \frac{x^6}{6!}+ \cdots
    \]
    And it looks like if we evaluate $\cos(x)$ at $x =
    \answer[given]{\frac{1}{\sqrt{2}}}$, we'll hit our target series:
    \begin{align*}
      \cos\left(\frac{1}{\sqrt{2}}\right) &=1-\frac{\left(\frac{1}{\sqrt{2}}\right)^2}{2!} + \frac{\left(\frac{1}{\sqrt{2}}\right)^4}{4!} - \frac{\left(\frac{1}{\sqrt{2}}\right)^6}{6!}+ \cdots\\
      &= 1 - \frac{1}{2!\cdot 2} + \frac{1}{(4)!\cdot 2^2}-\frac{1}{(6)!\cdot 2^3} + \cdots\\
      &=\sum_{k=0}^\infty \frac{(-1)^k}{(2n)!\cdot 2^n}.
    \end{align*}
  \end{explanation}
\end{example}


\begin{example}
  Compute the sum:
  \[
  \sum_{n=0}^\infty \frac{(-1)^n \pi^{2n+1}}{2^{2n+1}(2n+1)!}
  \]
  \begin{explanation}
    First notice that this series
    \[
    \sum_{n=0}^\infty \frac{(-1)^n \pi^{2n+1}}{2^{2n+1}(2n+1)!} = \frac{\pi}{2} - \frac{\pi^3}{2^3\cdot 3!}+ \frac{\pi^5}{2^5\cdot 5!} - \frac{\pi^7}{2^7\cdot 7!} + \cdots
    \]
    looks a bit like the power series for sine centered at $x=0$:
    \[
    \sin(x) = x-\frac{x^3}{3!} + \frac{x^5}{5!} - \frac{x^7}{7!}+ \cdots
    \]
    And it looks like if we evaluate $\sin(x)$ at $x =
    \answer[given]{\frac{\pi}{2}}$, we'll hit our target series:
    \begin{align*}
      \sin\left(\frac{\pi}{2}\right) &=\left(\frac{\pi}{2}\right)-\frac{\left(\frac{\pi}{2}\right)^3}{3!} + \frac{\left(\frac{\pi}{2}\right)^5}{5!} - \frac{\left(\frac{\pi}{2}\right)^7}{7!}+ \cdots\\
      &= \frac{\pi}{2} - \frac{\pi^3}{2^3\cdot 3!}+ \frac{\pi^5}{2^5\cdot 5!} - \frac{\pi^7}{2^7\cdot 7!} + \cdots\\
      &=\sum_{n=0}^\infty \frac{(-1)^n \pi^{2n+1}}{2^{2n+1}(2n+1)!}.
    \end{align*}
  \end{explanation}
\end{example}



\begin{example}
  Find a closed form expression for the series:
  \[
  \sum_{n=1}^\infty nx^{n-1}
  \]
  \begin{explanation}
    First notice that this series
    \[
    \sum_{n=1}^\infty nx^{n-1} = 1 + 2x + 3x^2 + 4x^3 + \cdots
    \]
    looks a bit like the power series for $\frac{1}{1-x}$  centered at $x=0$:
    \[
    \frac{1}{1-x} = 1+ x+x^2 + x^3+x^4+\cdots\qquad |x|<1
    \]
    In fact, the series we are interested in is exactly the derivative of this series! Write
    \begin{align*}
    \ddx &1+ x+x^2 + x^3+x^4+\cdots && |x|<1\\
    &=1 + 2x + 3x^2 + 4x^3 + \cdots && |x|<1\\
    &=\sum_{n=1}^\infty nx^{n-1}.
    \end{align*}
    Hence
    \begin{align*}
    \sum_{n=1}^\infty nx^{n-1} &= \ddx \frac{1}{1-x}\\
    &=\answer[given]{\frac{1}{(1-x)^2}}&& |x|<1
    \end{align*}
  \end{explanation}
\end{example}



\end{document}
