\documentclass{ximera}


\graphicspath{
  {./}
  {ximeraTutorial/}
  {basicPhilosophy/}
}

\newcommand{\mooculus}{\textsf{\textbf{MOOC}\textnormal{\textsf{ULUS}}}}

\usepackage{tkz-euclide}\usepackage{tikz}
\usepackage{tikz-cd}
\usetikzlibrary{arrows}
\tikzset{>=stealth,commutative diagrams/.cd,
  arrow style=tikz,diagrams={>=stealth}} %% cool arrow head
\tikzset{shorten <>/.style={ shorten >=#1, shorten <=#1 } } %% allows shorter vectors

\usetikzlibrary{backgrounds} %% for boxes around graphs
\usetikzlibrary{shapes,positioning}  %% Clouds and stars
\usetikzlibrary{matrix} %% for matrix
\usepgfplotslibrary{polar} %% for polar plots
\usepgfplotslibrary{fillbetween} %% to shade area between curves in TikZ
\usetkzobj{all}
\usepackage[makeroom]{cancel} %% for strike outs
%\usepackage{mathtools} %% for pretty underbrace % Breaks Ximera
%\usepackage{multicol}
\usepackage{pgffor} %% required for integral for loops



%% http://tex.stackexchange.com/questions/66490/drawing-a-tikz-arc-specifying-the-center
%% Draws beach ball
\tikzset{pics/carc/.style args={#1:#2:#3}{code={\draw[pic actions] (#1:#3) arc(#1:#2:#3);}}}



\usepackage{array}
\setlength{\extrarowheight}{+.1cm}
\newdimen\digitwidth
\settowidth\digitwidth{9}
\def\divrule#1#2{
\noalign{\moveright#1\digitwidth
\vbox{\hrule width#2\digitwidth}}}






\DeclareMathOperator{\arccot}{arccot}
\DeclareMathOperator{\arcsec}{arcsec}
\DeclareMathOperator{\arccsc}{arccsc}

















%%This is to help with formatting on future title pages.
\newenvironment{sectionOutcomes}{}{}


\outcome{Relate the derivative function to the derivative at a point.}
% Outcome: understanding variables (as bound or unbound)

\title[Break-Ground:]{Wait for the right moment}

\begin{document}
\begin{abstract}
Two young mathematicians discuss derivatives as
functions.
\end{abstract}
\maketitle

Check out this dialogue between two calculus students (based on a true
story):

\begin{dialogue}
\item[Devyn] Riley, I might be a calculus genius.
\item[Riley] Yeah?  Explain this one to me.
\item[Devyn] Let me first ask you a question.  Say you have a function, like 
 $f(x) = x^2$, and you want to know $f'(3)$.  Do
  you plug in the number $3$ before or after you find the derivative?
\item[Riley] Hmmmm. Well, my next step is usually
  \[
  f'(3) = \lim_{h\to 0}\frac{f(3+h)-f(3)}{h}.
  \]
  So I guess before.
\item[Devyn] Aha!  I think you're wasting time. You see I write
  \[
  f'(x) = \lim_{h\to 0}\frac{f(x+h)-f(x)}{h}.
  \]
  and it means that I can look at the derivative of my function at
  \textit{any} point.  So, I plug in the $3$ {\em after} I've found the derivative.
\item[Riley] That does seem like a pretty genius move. But doesn't working
 with $x$, instead of numbers, make all of this more difficult?
\item[Devyn] Not at all. Let's do the problems both ways, at the same time:
  \begin{image}
    \begin{tikzpicture}
      \node at (0,0) {
      $ \underbrace{\begin{aligned}
          f'(3) &= \lim_{h\to 0}\frac{f(3+h)-f(3)}{h}\\
          &= \lim_{h\to 0}\frac{(3+h)^2-9}{h}\\
          &= \lim_{h\to 0}\frac{9+6h+h^2-9}{h}\\
          &= \lim_{h\to 0}\frac{6h+h^2}{h}\\
          &= \lim_{h\to 0}(6+h)\\
          &= 6.
        \end{aligned}}_{\text{plugging in}}$};
      \node at (5,0) {
      $ \underbrace{\begin{aligned}
            f'(x) &= \lim_{h\to 0}\frac{f(x+h)-f(x)}{h}\\
            &= \lim_{h\to 0}\frac{(x+h)^2-x^2}{h}\\
            &= \lim_{h\to 0}\frac{x^2+2xh+h^2-x^2}{h}\\
            &= \lim_{h\to 0}\frac{2xh+h^2}{h}\\
            &= \lim_{h\to 0}(2x+h)\\
            &= 2x,\\
            \text{so }f'(3) &=6.
        \end{aligned}}_{\text{working with $x$}}$};
    \end{tikzpicture}
  \end{image}
\item[Riley] Whoa. So now the derivative is a function. Wait, what's
    its domain? Its range?
\end{dialogue}


\begin{problem}
  Suppose you have a function $f$. Which of the following are true?
  \begin{selectAll}
    \choice{The domain of $f'$ is equal to the domain of $f$.}
    \choice{The range of $f'$ is equal to the range of $f$.}
    \choice[correct]{The domain of $f'$ is a subset of the domain of $f$.}
    \choice{The range of $f'$ is a subset of the range of $f$.}
  \end{selectAll}
\end{problem}

\begin{problem}
Find $g'(2)$ for $g(x) = x^2 + 1$ using both methods described above.
\begin{prompt}
\[
g'(2) = \answer{4}
\]
\end{prompt}
\end{problem}

%\input{../leveledQuestions.tex}

\end{document}
