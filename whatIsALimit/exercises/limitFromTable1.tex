\documentclass{ximera}


\graphicspath{
  {./}
  {ximeraTutorial/}
  {basicPhilosophy/}
}

\newcommand{\mooculus}{\textsf{\textbf{MOOC}\textnormal{\textsf{ULUS}}}}

\usepackage{tkz-euclide}\usepackage{tikz}
\usepackage{tikz-cd}
\usetikzlibrary{arrows}
\tikzset{>=stealth,commutative diagrams/.cd,
  arrow style=tikz,diagrams={>=stealth}} %% cool arrow head
\tikzset{shorten <>/.style={ shorten >=#1, shorten <=#1 } } %% allows shorter vectors

\usetikzlibrary{backgrounds} %% for boxes around graphs
\usetikzlibrary{shapes,positioning}  %% Clouds and stars
\usetikzlibrary{matrix} %% for matrix
\usepgfplotslibrary{polar} %% for polar plots
\usepgfplotslibrary{fillbetween} %% to shade area between curves in TikZ
\usetkzobj{all}
\usepackage[makeroom]{cancel} %% for strike outs
%\usepackage{mathtools} %% for pretty underbrace % Breaks Ximera
%\usepackage{multicol}
\usepackage{pgffor} %% required for integral for loops



%% http://tex.stackexchange.com/questions/66490/drawing-a-tikz-arc-specifying-the-center
%% Draws beach ball
\tikzset{pics/carc/.style args={#1:#2:#3}{code={\draw[pic actions] (#1:#3) arc(#1:#2:#3);}}}



\usepackage{array}
\setlength{\extrarowheight}{+.1cm}
\newdimen\digitwidth
\settowidth\digitwidth{9}
\def\divrule#1#2{
\noalign{\moveright#1\digitwidth
\vbox{\hrule width#2\digitwidth}}}






\DeclareMathOperator{\arccot}{arccot}
\DeclareMathOperator{\arcsec}{arcsec}
\DeclareMathOperator{\arccsc}{arccsc}

















%%This is to help with formatting on future title pages.
\newenvironment{sectionOutcomes}{}{}


\outcome{Consider values of a function at inputs approaching a given point.}
\outcome{Understand the concept of a limit.}
\outcome{Identify when a limit does not exist.}

\author{Nela Lakos \and Kyle Parsons}

\begin{document}
\begin{exercise}

Complete the following table.  Use \textbf{exact} values.

\[
\begin{array}{|c|c|c|c|c|c|c|c|c|}
\hline
x & \frac{1}{3} & \frac{-1}{30} & \frac{1}{300} & \frac{-1}{301} & \frac{1}{1000} & \frac{1}{1001} & \frac{1}{50000} & \frac{-1}{100004}\\\hline
\frac{\pi}{x} & 3\pi & -30\pi & 300\pi & -301\pi & 1000\pi & 1001\pi & 50000\pi & -100004\pi \\\hline
\cos\left(\frac{\pi}{x}\right) & \answer{-1} & \answer{1} & \answer{1} & \answer{-1} & \answer{1} & \answer{-1} & \answer{1} & \answer{1} \\\hline
\end{array}
\]

\begin{exercise}
Based on the table above, make a conjecture about $\lim_{x\to0}\cos\left(\frac{\pi}{x}\right)$.  Does it exist? Explain.
\begin{freeResponse}
It seems like the limit does not exist.  $\cos\left(\frac{\pi}{x}\right)$ doesn't seem to approach 1, -1, or any value in between even when $x$ is very close to 0.  It seems to oscillate in between.
\end{freeResponse}
\end{exercise}

\end{exercise}
\end{document}