\documentclass{ximera}


\graphicspath{
  {./}
  {ximeraTutorial/}
  {basicPhilosophy/}
}

\newcommand{\mooculus}{\textsf{\textbf{MOOC}\textnormal{\textsf{ULUS}}}}

\usepackage{tkz-euclide}\usepackage{tikz}
\usepackage{tikz-cd}
\usetikzlibrary{arrows}
\tikzset{>=stealth,commutative diagrams/.cd,
  arrow style=tikz,diagrams={>=stealth}} %% cool arrow head
\tikzset{shorten <>/.style={ shorten >=#1, shorten <=#1 } } %% allows shorter vectors

\usetikzlibrary{backgrounds} %% for boxes around graphs
\usetikzlibrary{shapes,positioning}  %% Clouds and stars
\usetikzlibrary{matrix} %% for matrix
\usepgfplotslibrary{polar} %% for polar plots
\usepgfplotslibrary{fillbetween} %% to shade area between curves in TikZ
\usetkzobj{all}
\usepackage[makeroom]{cancel} %% for strike outs
%\usepackage{mathtools} %% for pretty underbrace % Breaks Ximera
%\usepackage{multicol}
\usepackage{pgffor} %% required for integral for loops



%% http://tex.stackexchange.com/questions/66490/drawing-a-tikz-arc-specifying-the-center
%% Draws beach ball
\tikzset{pics/carc/.style args={#1:#2:#3}{code={\draw[pic actions] (#1:#3) arc(#1:#2:#3);}}}



\usepackage{array}
\setlength{\extrarowheight}{+.1cm}
\newdimen\digitwidth
\settowidth\digitwidth{9}
\def\divrule#1#2{
\noalign{\moveright#1\digitwidth
\vbox{\hrule width#2\digitwidth}}}






\DeclareMathOperator{\arccot}{arccot}
\DeclareMathOperator{\arcsec}{arcsec}
\DeclareMathOperator{\arccsc}{arccsc}

















%%This is to help with formatting on future title pages.
\newenvironment{sectionOutcomes}{}{}


\outcome{Consider values of a function at inputs approaching a given point.}
\outcome{Understand the concept of a limit.}
\outcome{Calculate limits from a graph (or state that the limit does not exist).}

\author{Nela Lakos \and Kyle Parsons}

\begin{document}
\begin{exercise}

The entire graph of a function $f$ is given below.

\begin{image}
  \begin{tikzpicture}
    \begin{axis}[
        xmin=-5.3,xmax=8.3,ymin=-5.3,ymax=7.3,
        clip=false,
        unit vector ratio*=1 1 1,
        axis lines=center,
        grid = major,
        ytick={-5,-4,...,7},
    xtick={-5,-4,...,8},
        xlabel=$x$, ylabel=$y$,
        every axis y label/.style={at=(current axis.above origin),anchor=south},
        every axis x label/.style={at=(current axis.right of origin),anchor=west},
      ]
      \addplot[very thick,penColor] plot coordinates {(-4,6) (0,2)};
      \addplot[very thick,penColor,domain=0:3,samples=50] {sqrt(4-4*x/3)};
      \addplot[very thick,penColor,domain=3:7,samples=50] {-sqrt(x-3)-2};
       
      \addplot[color=penColor,fill=white,only marks,mark=*] coordinates{(0,2)};  %% open hole
      \addplot[color=penColor,fill=white,only marks,mark=*] coordinates{(3,0)};  %% open hole
       
      \addplot[color=penColor,fill=penColor,only marks,mark=*] coordinates{(-4,6)};  %% closed hole
      \addplot[color=penColor,fill=penColor,only marks,mark=*] coordinates{(0,-1)};  %% closed hole
      \addplot[color=penColor,fill=penColor,only marks,mark=*] coordinates{(3,-2)};  %% closed hole
      \addplot[color=penColor,fill=penColor,only marks,mark=*] coordinates{(7,-4)};  %% closed hole
       
      \node[penColor] at (axis cs:3, 3.5) [penColor] {$y=f(x)$};
      \end{axis}`
  \end{tikzpicture}
\end{image}

The domain of $f$ is $\left[\answer{-4},\answer{7}\right]$.

The range of $f$ (from bottom to top) is $\left[\answer{-4},\answer{-2}\right] \cup \left\{\answer{-1}\right\} \cup \left(\answer{0},\answer{2}\right) \cup \left(\answer{2},\answer{6}\right]$.

Find the following limits, if they exist.  If a limit does not exist, explain why.

\[
\lim_{x\to0}f(x) = \answer{2}
\]
\begin{multipleChoice}
\choice[correct]{The limit does exist.}
\choice{The limit does not exist because $f(0)$ does not exist.}
\choice{The limit does not exist because $f(0)$ is not close to the values of $f$ near 0.}
\choice{The limit does not exist because $\lim_{x\to0^-}f(x)\neq\lim_{x\to0^+}f(x)$.}
\end{multipleChoice}

\noindent\rule[0.5ex]{\linewidth}{.2pt}

\[
\lim_{x\to3^-}f(x) = \answer{0}
\]
\begin{multipleChoice}
\choice[correct]{The limit does exist.}
\choice{The limit does not exist because $f(3)$ does not exist.}
\choice{The limit does not exist because $f(3)$ is not close to the values of $f$ near 3.}
\choice{The limit does not exist because $\lim_{x\to3^-}f(x)\neq\lim_{x\to3^+}f(x)$.}
\end{multipleChoice}

\noindent\rule[0.5ex]{\linewidth}{.2pt}

\[
\lim_{x\to3^+}f(x) = \answer{-2}
\]
\begin{multipleChoice}
\choice[correct]{The limit does exist.}
\choice{The limit does not exist because $f(3)$ does not exist.}
\choice{The limit does not exist because $f(3)$ is not close to the values of $f$ near 3.}
\choice{The limit does not exist because $\lim_{x\to3^-}f(x)\neq\lim_{x\to3^+}f(x)$.}
\end{multipleChoice}

\noindent\rule[0.5ex]{\linewidth}{.2pt}

\[
\lim_{x\to3}f(x) = \answer{DNE}
\]
\begin{multipleChoice}
\choice{The limit does exist.}
\choice{The limit does not exist because $f(3)$ does not exist.}
\choice{The limit does not exist because $f(3)$ is not close to the values of $f$ near 3.}
\choice[correct]{The limit does not exist because $\lim_{x\to3^-}f(x)\neq\lim_{x\to3^+}f(x)$.}
\end{multipleChoice}

Find the following values.

\begin{align*}
f(0) &= \answer{-1}\\
f^{-1}(-2) &= \answer{3}
\end{align*}

\end{exercise}
\end{document}
