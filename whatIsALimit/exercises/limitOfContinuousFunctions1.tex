\documentclass{ximera}


\graphicspath{
  {./}
  {ximeraTutorial/}
  {basicPhilosophy/}
}

\newcommand{\mooculus}{\textsf{\textbf{MOOC}\textnormal{\textsf{ULUS}}}}

\usepackage{tkz-euclide}\usepackage{tikz}
\usepackage{tikz-cd}
\usetikzlibrary{arrows}
\tikzset{>=stealth,commutative diagrams/.cd,
  arrow style=tikz,diagrams={>=stealth}} %% cool arrow head
\tikzset{shorten <>/.style={ shorten >=#1, shorten <=#1 } } %% allows shorter vectors

\usetikzlibrary{backgrounds} %% for boxes around graphs
\usetikzlibrary{shapes,positioning}  %% Clouds and stars
\usetikzlibrary{matrix} %% for matrix
\usepgfplotslibrary{polar} %% for polar plots
\usepgfplotslibrary{fillbetween} %% to shade area between curves in TikZ
\usetkzobj{all}
\usepackage[makeroom]{cancel} %% for strike outs
%\usepackage{mathtools} %% for pretty underbrace % Breaks Ximera
%\usepackage{multicol}
\usepackage{pgffor} %% required for integral for loops



%% http://tex.stackexchange.com/questions/66490/drawing-a-tikz-arc-specifying-the-center
%% Draws beach ball
\tikzset{pics/carc/.style args={#1:#2:#3}{code={\draw[pic actions] (#1:#3) arc(#1:#2:#3);}}}



\usepackage{array}
\setlength{\extrarowheight}{+.1cm}
\newdimen\digitwidth
\settowidth\digitwidth{9}
\def\divrule#1#2{
\noalign{\moveright#1\digitwidth
\vbox{\hrule width#2\digitwidth}}}






\DeclareMathOperator{\arccot}{arccot}
\DeclareMathOperator{\arcsec}{arcsec}
\DeclareMathOperator{\arccsc}{arccsc}

















%%This is to help with formatting on future title pages.
\newenvironment{sectionOutcomes}{}{}



\outcome{Understand the concept of a limit.}
\outcome{Calculate limits of continuous functions.}

\author{Nela Lakos}

\begin{document}
\begin{exercise}


For the given function $f$, evaluate the limit and justify your answer.\\


\begin{enumerate}
 \item $f(x)=x$
 \[
\lim_{x\to 7}f(x) = \answer{7}
\] 
Justification:\\ $f$ is continuous at $a=7$, which implies that
$\lim_{x\to 7}f(x)=f(\answer{7})=\answer{7}$.

\noindent\rule[0.5ex]{\linewidth}{.2pt}

\item $f(x)=\sin{x}$
 \[
\lim_{x\to \frac{\pi}{2}}f(x) = \answer{1}
\] 
Justification:\\ $f$ is continuous at $a=\frac{\pi}{2}$, which implies that\\[1em]
$\lim_{x\to \frac{\pi}{2}}f(x)=f\Bigl(\answer{\frac{\pi}{2}}\Bigr)=\sin{\Bigl(\answer{\frac{\pi}{2}}\Bigr)}=\answer{1}$.

\noindent\rule[0.5ex]{\linewidth}{.2pt}

\item  $f(x)=e^{x}$
 \[
\lim_{x\to 0}f(x) = \answer{1}
\] 
Justification:\\ $f$ is continuous at $a=0$, which implies that\\[1em]
$\lim_{x\to 0}f(x)=f\Bigl(\answer{0}\Bigr)=e^{\answer{0}}=\answer{1}$.

\noindent\rule[0.5ex]{\linewidth}{.2pt}

\item  $f(x)=\ln{x}$
 \[
\lim_{x\to e^{4}}f(x) = \answer{4}
\] 
Justification:\\ $f$ is continuous at $a=e^{4}$, which implies that\\[1em]
$\lim_{x\to e^{4}}f(x)=f\Bigl(\answer{e^{4}}\Bigr)=\ln{\Bigl(\answer{e^{4}}\Bigr)}=\answer{4}$.

\noindent\rule[0.5ex]{\linewidth}{.2pt}

\item $f(x)=\cos{x}$
 \[
\lim_{x\to \frac{2\pi}{3}}f(x) = \answer{-\frac{1}{2}}
\] 
Justification:\\ $f$ is continuous at $a= \frac{2\pi}{3}$, which implies that\\[1em]
$\lim_{x\to \frac{2\pi}{3}}f(x)=f\Bigl(\answer{ \frac{2\pi}{3}}\Bigr)=\cos{\Bigl(\answer{ \frac{2\pi}{3}}\Bigr)}=\answer{-\frac{1}{2}}$.

\noindent\rule[0.5ex]{\linewidth}{.2pt}

\item  $f(x)=x^3$
 \[
\lim_{x\to -2}f(x) = \answer{-8}
\] 
Justification:\\ $f$ is continuous at $a=-2$, which implies that\\[1em]
$\lim_{x\to -2}f(x)=f\Bigl(\answer{ -2}
\Bigr)=\Bigl(\answer{-2 }\Bigr)^3=\answer{-8}$.

\noindent\rule[0.5ex]{\linewidth}{.2pt}
\end{enumerate}
\end{exercise}
\end{document}
