\documentclass{ximera}


\graphicspath{
  {./}
  {ximeraTutorial/}
  {basicPhilosophy/}
}

\newcommand{\mooculus}{\textsf{\textbf{MOOC}\textnormal{\textsf{ULUS}}}}

\usepackage{tkz-euclide}\usepackage{tikz}
\usepackage{tikz-cd}
\usetikzlibrary{arrows}
\tikzset{>=stealth,commutative diagrams/.cd,
  arrow style=tikz,diagrams={>=stealth}} %% cool arrow head
\tikzset{shorten <>/.style={ shorten >=#1, shorten <=#1 } } %% allows shorter vectors

\usetikzlibrary{backgrounds} %% for boxes around graphs
\usetikzlibrary{shapes,positioning}  %% Clouds and stars
\usetikzlibrary{matrix} %% for matrix
\usepgfplotslibrary{polar} %% for polar plots
\usepgfplotslibrary{fillbetween} %% to shade area between curves in TikZ
\usetkzobj{all}
\usepackage[makeroom]{cancel} %% for strike outs
%\usepackage{mathtools} %% for pretty underbrace % Breaks Ximera
%\usepackage{multicol}
\usepackage{pgffor} %% required for integral for loops



%% http://tex.stackexchange.com/questions/66490/drawing-a-tikz-arc-specifying-the-center
%% Draws beach ball
\tikzset{pics/carc/.style args={#1:#2:#3}{code={\draw[pic actions] (#1:#3) arc(#1:#2:#3);}}}



\usepackage{array}
\setlength{\extrarowheight}{+.1cm}
\newdimen\digitwidth
\settowidth\digitwidth{9}
\def\divrule#1#2{
\noalign{\moveright#1\digitwidth
\vbox{\hrule width#2\digitwidth}}}






\DeclareMathOperator{\arccot}{arccot}
\DeclareMathOperator{\arcsec}{arcsec}
\DeclareMathOperator{\arccsc}{arccsc}

















%%This is to help with formatting on future title pages.
\newenvironment{sectionOutcomes}{}{}


\title[Dig-In:]{Derivatives of trigonometric functions}

\outcome{Apply chain rule to relate quantities expressed with different units.}
\outcome{Compute derivatives of trigonometric functions.}

\begin{document}
\begin{abstract}
  We use the chain rule to unleash the derivatives of the
  trigonometric functions.
\end{abstract}
\maketitle

Up until this point of the course we have been ignoring a large class
of functions: Trigonometric functions other than $\sin(x)$. We know
that
\[
\frac{d}{dx} \sin(x) = \cos(x).
\]
Armed with this fact we will discover the derivatives of all of the
standard trigonometric functions.

\begin{theorem}[The derivative of cosine]\index{derivative!of cosine}
\[
\frac{d}{dx} \cos(x) = -\sin(x).
\]
\begin{explanation}
Recall that
\begin{itemize}
\item $\cos(x) = \sin\left(\frac{\pi}{2}-x\right)$, and
\item $\sin(x) = \cos\left(\frac{\pi}{2}-x\right)$.
\end{itemize}
Now
\begin{align*}
\frac{d}{dx} \cos(x) &= \frac{d}{dx} \sin\left(\frac{\pi}{2}-x\right)\\
&=-\cos\left(\frac{\pi}{2}-x\right) \\
&= -\sin(x).
\end{align*}
\end{explanation}
\end{theorem}

\begin{example}
Compute:
\[
\bigg[ \frac{d}{dx} \cos \left( \frac{x^3}{2} \right) \bigg]_{x=\sqrt[3]{\pi}}
\]
\begin{explanation}
Now that we know the derivative of cosine, we may combine this with the
chain rule, so we have that
\[
\frac{d}{dx} \cos \left( \frac{x^3}{2} \right) = \answer[given]{\frac{3 x^2}{2}} \left(- \sin \left( \frac{x^3}{2} \right) \right)
\]
and so
\[
\bigg[ \frac{d}{dx} \cos \left( \frac{x^3}{2} \right) \bigg]_{x=\sqrt[3]{\pi}}
\]
\begin{align*}
  &= \bigg[ \left( \frac{3}{2} x^2 \left(- \sin \left( \frac{x^3}{2}
    \right) \right) \right) \bigg]_{x=\sqrt[3]{\pi}} \\
  &= - \frac{3}{2}(\sqrt[3]{\pi})^2 \sin \left( \frac{\pi}{2} \right) \\
  &= -\frac{3}{2} \pi^{\frac{2}{3}} \cdot \answer[given]{1} \\
  &=\answer[given]{\frac{-3 \pi^{\frac{2}{3}}}{2}}.
\end{align*}
\end{explanation}
\end{example}


Next we have:

\begin{theorem}[The derivative of tangent]\index{derivative!of tangent}
\[
\frac{d}{dx} \tan(x) = \sec^2(x).
\]

\begin{explanation}
We'll rewrite $\tan(x)$ as $\frac{\sin(x)}{\cos(x)}$ and use the
quotient rule. Write with me:
\begin{align*}
\frac{d}{dx}\tan(x) &= \frac{d}{dx}\frac{\sin(x)}{\cos(x)}\\
&=\frac{\cos^2(x) + \answer[given]{\sin^2(x)}}{\cos^2(x)}\\
&=\frac{\answer[given]{1}}{\cos^2(x)}\\
&=\sec^2(x).
\end{align*}
\end{explanation}
\end{theorem}

\begin{example}
Compute:
\[
\frac{d}{dx} \left( \frac{5x \tan(x)}{x^2 - 3} \right)
\]
\begin{explanation}
Applying the quotient rule, and the product rule, and the derivative
of tangent:
\begin{align*}
  \frac{d}{dx} &\left( \frac{5x \tan(x)}{x^2 - 3} \right) \\
  &= \frac{(x^2 - 3) \cdot \frac{d}{dx}(\answer[given]{5x \tan(x)}) - 5x \tan(x) \cdot \frac{d}{dx} (\answer[given]{x^2 - 3})}{(x^2 - 3)^2}  \\
  &= \frac{(x^2 - 3)(5 \tan(x) + 5x \answer[given]{\sec^2(x)}) - 5x \tan(x) \cdot 2x}{(x^2 - 3)^2}  \\
  &= \frac{5(x^2-3)(\tan(x)+x \sec^2(x)) - 10x^2 \tan(x)}{(x^2-3)^2}
\end{align*}
\end{explanation}
\end{example}

Finally, we have:

\begin{theorem}[The derivative of secant]\index{derivative!of secant}
\[
\frac{d}{dx} \sec(x) = \sec(x)\tan(x).
\]


\begin{explanation}
We'll rewrite $\sec(x)$ as $(\cos(x))^{-1}$ and use the power rule and the chain rule. Write
\begin{align*}
\frac{d}{dx} \sec(x) &= \frac{d}{dx}(\cos (x))^{-1}\\
&=-1(\cos(x))^{-2}(\answer[given]{-\sin(x)}) \\
&= \frac{\sin(x)}{\cos^2(x)} \\
&= \frac{1}{\cos(x)} \cdot \frac{\sin(x)}{\cos(x)}  \\
&= \sec(x)\tan(x).
\end{align*}
\end{explanation}
\end{theorem}

The derivatives of the cotangent and cosecant are similar and left as
exercises.  Putting this all together, we have:

\begin{theorem}[The Derivatives of Trigonometric Functions] \hfil
\begin{itemize}
\item $\frac{d}{dx} \sin(x) = \cos(x)$.
\item $\frac{d}{dx} \cos(x) = -\sin(x)$.
\item $\frac{d}{dx} \tan(x) = \sec^2(x)$.
\item $\frac{d}{dx} \sec(x) = \sec(x)\tan(x)$.
\item $\frac{d}{dx} \csc(x) = -\csc(x)\cot(x)$.
\item $\frac{d}{dx} \cot(x) = -\csc^2(x)$.
\end{itemize}
\end{theorem}

\begin{example}
Compute:
\[
\bigg[ {\frac{d}{dx} ( \csc(x) \cot(x) ) \bigg]_{x=\frac{\pi}{3}}
\]
\begin{explanation}
Applying the product rule and the facts above, we know that
\[
\frac{d}{dx} ( \csc(x) \cot(x) ) = - \csc^3(x) - \cot^2(x)\answer[given]{\csc(x)}
\]
and so
\[
\bigg[ \frac{d}{dx} ( \csc(x) \cot(x) ) \bigg]_{x=\frac{\pi}{3}}
\]
\begin{align*}
  &= \bigg[   - \csc^3(x) - \cot^2(x) \answer[given]{\csc(x)} \bigg]_{x=\frac{\pi}{3}}  \\
&= - \frac{8}{3 \sqrt{3}} - \frac{1}{3}\cdot \answer[given]{2/\sqrt{3}}
\end{align*}
\end{explanation}
\end{example}


\begin{warning}
When working with derivatives of trigonometric functions, we suggest
you use \textbf{radians} for angle measure. For example, while
\[
\sin\left((90^\circ\right)^2) = \sin\left(\left(\frac{\pi}{2}\right)^2\right),
\]
one must be careful with derivatives as
\[
\bigg[ \frac{d}{dx} \sin\left(x^2\right) \bigg]_{x=90^\circ} \ne \underbrace{2\cdot 90\cdot \cos(90^2)}_{\text{incorrect}}
\]
Alternatively, one could think of $x^\circ$ as meaning
$\frac{x\cdot\pi}{180}$, as then $90^\circ = \frac{90\cdot\pi}{180} =
\frac{\pi}{2}$. In this case
\[
2\cdot 90^\circ\cdot \cos((90^\circ)^2) = 2\cdot \frac{\pi}{2}\cdot\cos\left(\left(\frac{\pi}{2}\right)^2\right).
\]
\end{warning}


\end{document}
