\documentclass{ximera}


\graphicspath{
  {./}
  {ximeraTutorial/}
  {basicPhilosophy/}
}

\newcommand{\mooculus}{\textsf{\textbf{MOOC}\textnormal{\textsf{ULUS}}}}

\usepackage{tkz-euclide}\usepackage{tikz}
\usepackage{tikz-cd}
\usetikzlibrary{arrows}
\tikzset{>=stealth,commutative diagrams/.cd,
  arrow style=tikz,diagrams={>=stealth}} %% cool arrow head
\tikzset{shorten <>/.style={ shorten >=#1, shorten <=#1 } } %% allows shorter vectors

\usetikzlibrary{backgrounds} %% for boxes around graphs
\usetikzlibrary{shapes,positioning}  %% Clouds and stars
\usetikzlibrary{matrix} %% for matrix
\usepgfplotslibrary{polar} %% for polar plots
\usepgfplotslibrary{fillbetween} %% to shade area between curves in TikZ
\usetkzobj{all}
\usepackage[makeroom]{cancel} %% for strike outs
%\usepackage{mathtools} %% for pretty underbrace % Breaks Ximera
%\usepackage{multicol}
\usepackage{pgffor} %% required for integral for loops



%% http://tex.stackexchange.com/questions/66490/drawing-a-tikz-arc-specifying-the-center
%% Draws beach ball
\tikzset{pics/carc/.style args={#1:#2:#3}{code={\draw[pic actions] (#1:#3) arc(#1:#2:#3);}}}



\usepackage{array}
\setlength{\extrarowheight}{+.1cm}
\newdimen\digitwidth
\settowidth\digitwidth{9}
\def\divrule#1#2{
\noalign{\moveright#1\digitwidth
\vbox{\hrule width#2\digitwidth}}}






\DeclareMathOperator{\arccot}{arccot}
\DeclareMathOperator{\arcsec}{arcsec}
\DeclareMathOperator{\arccsc}{arccsc}

















%%This is to help with formatting on future title pages.
\newenvironment{sectionOutcomes}{}{}


%\outcome{Understand the statement of the Extreme Value Theorem.}
\outcome{Understand the statement of the Mean Value Theorem.}
%\outcome{Sketch pictures to illustrate why the Mean Value Theorem is true.}
\outcome{Determine whether Rolle's Theorem or the Mean Value Theorem can be applied.}
%\outcome{Find the values guaranteed by Rolle's Theorem or the Mean Value Theorem.}
\outcome{Use the Mean Value Theorem to solve word problems.}
%\outcome{Compare and contrast the Intermediate Value Theorem, the Mean Value Theorem, and Rolle's Theorem.}
%\outcome{Identify calculus ideas which are consequences of the Mean Value Theorem.}
%\outcome{Use the Mean Value Theorem to bound the error in linear approximation.}

\author{Nela Lakos \and Kyle Parsons}

\begin{document}


The population (in millions of cells) of a culture of bacteria is given by the function $P$ where $t$ is measured in weeks after some start date.

Below is a partial table of values of $P(t)$.
\[
\begin{array}{|c|c|c|c|c|}
\hline
t & 0 & 1 & 3 & 4\\\hline
P(t) & 0 & 50 & 75 & 80\\\hline
\end{array} 
\]
\begin{exercise}
Compute the average rate of change of the population during the time interval $[0,4]$.
\begin{hint}
Recall, the average rate of change of the population during the time interval $[0,4]$ is given by the expression $\frac{P\left(\answer{4}\right)-P\left(\answer{0}\right)}{4-0}$.
\end{hint}
The average rate of change of the population during the time interval $[0,4]$ is $\answer{20}$ million cells per week.
\end{exercise}
\begin{exercise}
Suppose that  the function $P$ is continuous on $[0,10)$ and differentiable on $(0,10)$. Then  on the interval $[0,4]$ the function $P$  \wordChoice{\choice[correct]{satisfies}\choice{does not satisfy}} the conditions of the Mean Value Theorem so \wordChoice{\choice[correct]{there exists}\choice{there does not necessarily exist}} a $0<c<4$ such that $P'(c)$ equals the above calculated average rate of change of the population.

That means that at the time $t=c$ the \textbf{instantaneous rate of change} of the population is equal to the \textbf{average rate of change} of the population during the time interval $[0,4]$.
\end{exercise}
\end{document}