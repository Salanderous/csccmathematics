\documentclass{ximera}

\graphicspath{
  {./}
  {ximeraTutorial/}
  {basicPhilosophy/}
}

\newcommand{\mooculus}{\textsf{\textbf{MOOC}\textnormal{\textsf{ULUS}}}}

\usepackage{tkz-euclide}\usepackage{tikz}
\usepackage{tikz-cd}
\usetikzlibrary{arrows}
\tikzset{>=stealth,commutative diagrams/.cd,
  arrow style=tikz,diagrams={>=stealth}} %% cool arrow head
\tikzset{shorten <>/.style={ shorten >=#1, shorten <=#1 } } %% allows shorter vectors

\usetikzlibrary{backgrounds} %% for boxes around graphs
\usetikzlibrary{shapes,positioning}  %% Clouds and stars
\usetikzlibrary{matrix} %% for matrix
\usepgfplotslibrary{polar} %% for polar plots
\usepgfplotslibrary{fillbetween} %% to shade area between curves in TikZ
\usetkzobj{all}
\usepackage[makeroom]{cancel} %% for strike outs
%\usepackage{mathtools} %% for pretty underbrace % Breaks Ximera
%\usepackage{multicol}
\usepackage{pgffor} %% required for integral for loops



%% http://tex.stackexchange.com/questions/66490/drawing-a-tikz-arc-specifying-the-center
%% Draws beach ball
\tikzset{pics/carc/.style args={#1:#2:#3}{code={\draw[pic actions] (#1:#3) arc(#1:#2:#3);}}}



\usepackage{array}
\setlength{\extrarowheight}{+.1cm}
\newdimen\digitwidth
\settowidth\digitwidth{9}
\def\divrule#1#2{
\noalign{\moveright#1\digitwidth
\vbox{\hrule width#2\digitwidth}}}






\DeclareMathOperator{\arccot}{arccot}
\DeclareMathOperator{\arcsec}{arcsec}
\DeclareMathOperator{\arccsc}{arccsc}

















%%This is to help with formatting on future title pages.
\newenvironment{sectionOutcomes}{}{}

\author{Steven Gubkin\and Nela Lakos}
\license{Creative Commons 3.0 By-NC}

\outcome{Find the values guaranteed by Rolle's Theorem or the Mean Value Theorem.}
\outcome{Understand the statement of the Mean Value Theorem.}
\outcome{Determine whether Rolle's Theorem or the Mean Value Theorem can be applied.}
\begin{document}
\begin{exercise}

Let $f(x) = Ax^2+Bx+C$.  Find the constant $c$, known to exist by the Mean Value Theorem, for which $f'(c) = \frac{f(b)-f(a)}{b-a}$, where $a$ and $b$ are any two distinct real numbers.  The result is a very interesting property of quadratics.


\begin{hint}
First, substitute: 

\[
f'(c) = \frac{f(b)-f(a)}{b-a}=\frac{Ab^2+Bb+C-(Aa^2+Ba+C)}{b-a}
\]
Then, simplify.
\end{hint}


\begin{hint}
\[
f'(c) = \frac{f(b)-f(a)}{b-a}=\frac{A(b^2-a^2)+B(b-a)}{b-a}=\frac{A(b-a)(b+a)+B(b-a)}{b-a}=A(b+a)+B
\]
\end{hint}

\begin{hint}
Then, we compute $f'(x)$.

\[
f'(x) =2A\answer{x}+B 
\]
\end{hint}

\begin{hint}
Then, we solve the equation

\[
f'(c) =A\left(\answer{b+a}\right)+B
\]
\end{hint}

\begin{hint}
In the equation above we replace $f'(c)$ with $2A\answer{c}+B $. Now, we solve for $c$:

\[
2A(\answer{c})+B =A\left(\answer{b+a}\right)+B
\]
\end{hint}
\begin{prompt}
	$$c = \frac{b+a}{\answer{2}}$$
\end{prompt}

\end{exercise}
\end{document}