\documentclass{ximera}


\graphicspath{
  {./}
  {ximeraTutorial/}
  {basicPhilosophy/}
}

\newcommand{\mooculus}{\textsf{\textbf{MOOC}\textnormal{\textsf{ULUS}}}}

\usepackage{tkz-euclide}\usepackage{tikz}
\usepackage{tikz-cd}
\usetikzlibrary{arrows}
\tikzset{>=stealth,commutative diagrams/.cd,
  arrow style=tikz,diagrams={>=stealth}} %% cool arrow head
\tikzset{shorten <>/.style={ shorten >=#1, shorten <=#1 } } %% allows shorter vectors

\usetikzlibrary{backgrounds} %% for boxes around graphs
\usetikzlibrary{shapes,positioning}  %% Clouds and stars
\usetikzlibrary{matrix} %% for matrix
\usepgfplotslibrary{polar} %% for polar plots
\usepgfplotslibrary{fillbetween} %% to shade area between curves in TikZ
\usetkzobj{all}
\usepackage[makeroom]{cancel} %% for strike outs
%\usepackage{mathtools} %% for pretty underbrace % Breaks Ximera
%\usepackage{multicol}
\usepackage{pgffor} %% required for integral for loops



%% http://tex.stackexchange.com/questions/66490/drawing-a-tikz-arc-specifying-the-center
%% Draws beach ball
\tikzset{pics/carc/.style args={#1:#2:#3}{code={\draw[pic actions] (#1:#3) arc(#1:#2:#3);}}}



\usepackage{array}
\setlength{\extrarowheight}{+.1cm}
\newdimen\digitwidth
\settowidth\digitwidth{9}
\def\divrule#1#2{
\noalign{\moveright#1\digitwidth
\vbox{\hrule width#2\digitwidth}}}






\DeclareMathOperator{\arccot}{arccot}
\DeclareMathOperator{\arcsec}{arcsec}
\DeclareMathOperator{\arccsc}{arccsc}

















%%This is to help with formatting on future title pages.
\newenvironment{sectionOutcomes}{}{}


\outcome{Understand the statement of the Extreme Value Theorem.}
\outcome{Understand the statement of the Mean Value Theorem.}
%\outcome{Sketch pictures to illustrate why the Mean Value Theorem is true.}
\outcome{Determine whether Rolle's Theorem or the Mean Value Theorem can be applied.}
\outcome{Find the values guaranteed by Rolle's Theorem or the Mean Value Theorem.}
%\outcome{Use the Mean Value Theorem to solve word problems.}
%\outcome{Compare and contrast the Intermediate Value Theorem, the Mean Value Theorem, and Rolle's Theorem.}
%\outcome{Identify calculus ideas which are consequences of the Mean Value Theorem.}
%\outcome{Use the Mean Value Theorem to bound the error in linear approximation.}

\author{Nela Lakos \and Kyle Parsons}

\begin{document}



Consider the functions $f_A$, $f_B$, $f_C$, and $f_D$ graphed below.

\resizebox{0.45\textwidth}{!}{
  \begin{tikzpicture}
    \begin{axis}[
        xmin=-.3,xmax=5.3,ymin=-.3,ymax=5.3,
        clip=true,
        unit vector ratio*=1 1 1,
        axis lines=center,
        grid = major,
        ytick={-20,-19,...,20},
    	xtick={-20,-19,...,20},
        xlabel=$x$, ylabel=$y$,
        y tick label style={anchor=west},
        every axis y label/.style={at=(current axis.above origin),anchor=south},
        every axis x label/.style={at=(current axis.right of origin),anchor=west},
      ]
      \addplot[very thick,penColor,domain=1:5] plot{(x-3)^2};
      
      \addplot[penColor,only marks,mark=*] coordinates{(1,4) (5,4)};
            
      \node at (axis cs:1.3,.5) {$y=f_A(x)$};
      \end{axis}`
  \end{tikzpicture}}
\hfill
\resizebox{0.45\textwidth}{!}{
  \begin{tikzpicture}
    \begin{axis}[
        xmin=-.3,xmax=5.3,ymin=-.3,ymax=5.3,
        clip=true,
        unit vector ratio*=1 1 1,
        axis lines=center,
        grid = major,
        ytick={-20,-19,...,20},
    	xtick={-20,-19,...,20},
        xlabel=$x$, ylabel=$y$,
        y tick label style={anchor=west},
        every axis y label/.style={at=(current axis.above origin),anchor=south},
        every axis x label/.style={at=(current axis.right of origin),anchor=west},
      ]
      \addplot[very thick,penColor,domain=1:3] plot{sqrt(2*(3-x))};
      \addplot[very thick,penColor,domain=3:5] plot{sqrt(2*(x-3))};
      
      \addplot[penColor,only marks,mark=*] coordinates{(1,2) (5,2)};
      
      \node at (axis cs:1.3,.5) {$y=f_B(x)$};
      \end{axis}`
  \end{tikzpicture}}

\resizebox{0.45\textwidth}{!}{
  \begin{tikzpicture}
    \begin{axis}[
        xmin=-.3,xmax=5.3,ymin=-.3,ymax=5.3,
        clip=true,
        unit vector ratio*=1 1 1,
        axis lines=center,
        grid = major,
        ytick={-20,-19,...,20},
    	xtick={-20,-19,...,20},
        xlabel=$x$, ylabel=$y$,
        y tick label style={anchor=west},
        every axis y label/.style={at=(current axis.above origin),anchor=south},
        every axis x label/.style={at=(current axis.right of origin),anchor=west},
      ]
      \addplot[very thick,penColor,domain=1:5] plot{5-x};

      \addplot[penColor,only marks,mark=*] coordinates{(1,5) (5,1)};
      \addplot[penColor,fill=white,only marks,mark=*] coordinates{(1,4) (5,0)};
      
      \node at (axis cs:1.3,.5) {$y=f_C(x)$};
      \end{axis}`
  \end{tikzpicture}}
\hfill
\resizebox{0.45\textwidth}{!}{
  \begin{tikzpicture}
    \begin{axis}[
        xmin=-.3,xmax=5.3,ymin=-.3,ymax=5.3,
        clip=true,
        unit vector ratio*=1 1 1,
        axis lines=center,
        grid = major,
        ytick={-20,-19,...,20},
    	xtick={-20,-19,...,20},
        xlabel=$x$, ylabel=$y$,
        y tick label style={anchor=west},
        every axis y label/.style={at=(current axis.above origin),anchor=south},
        every axis x label/.style={at=(current axis.right of origin),anchor=west},
      ]
      \addplot[very thick,penColor,domain=1:5] plot{5-x};

      \addplot[penColor,only marks,mark=*] coordinates{(1,5) (5,0)};
      \addplot[penColor,fill=white,only marks,mark=*] coordinates{(1,4)};
      
      \node at (axis cs:1.3,.5) {$y=f_D(x)$};
      \end{axis}`
  \end{tikzpicture}}
\begin{exercise}
Select all functions that are continuous on $[1,5]$.
\begin{selectAll}
\choice[correct]{$f_A(x)$}
\choice[correct]{$f_B(x)$}
\choice{$f_C(x)$}
\choice{$f_D(x)$}
\end{selectAll}
\begin{exercise}
Select all functions that are differentiable on $(1,5)$.
\begin{selectAll}
\choice[correct]{$f_A(x)$}
\choice{$f_B(x)$}
\choice[correct]{$f_C(x)$}
\choice[correct]{$f_D(x)$}
\end{selectAll}
\begin{exercise}
Answer the following true-false questions.

The Mean Value Theorem \textbf{does} apply to $f_B(x)$ on $[1,5]$, and $c=3$ is the point guaranteed to exist.
\begin{multipleChoice}
\choice{True}
\choice[correct]{False}
\end{multipleChoice}

The Mean Value Theorem \textbf{does not} apply to $f_A(x)$ on $[1,5]$. 
\begin{multipleChoice}
\choice{True}
\choice[correct]{False}
\end{multipleChoice}

The Mean Value Theorem \textbf{does} apply to $f_A(x)$ on $[1,5]$, and $c=3$ is the point guaranteed to exist.
\begin{multipleChoice}
\choice[correct]{True}
\choice{False}
\end{multipleChoice}

Since the function $f_C$ does not satisfy the conditions of the Mean value theorem, there is no point $c$ in $(1,5)$ such that $(f_C)'(c)=\frac{f_C(5)-f_C(1)}{5-1}=\frac{1-5}{5-1}=-1$.
\begin{hint}
Look at the graph of $f_C$ on an open interval $(1,5)$: it is a line of slope $m=\answer{-1}$. Therefore, for every point $c$ in $(1,5)$ it is true that  $(f_C)'(c)=-1=\frac{f_C(5)-f_C(1)}{5-1}$. This may happen or not happen with functions that do not satisfy the conditions of the Mean value theorem (see graph of $f_D$). But, if the conditions are satisfied, the Mean value theorem \textbf{guarantees} that such a point $c$ exists.
\end{hint}
\begin{multipleChoice}
\choice{True}
\choice[correct]{False}
\end{multipleChoice}
$f_C(x)$ attains an absolute minimum on $[1,5]$.
\begin{multipleChoice}
\choice{True}
\choice[correct]{False}
\end{multipleChoice}

$f_D(x)$ attains an absolute minimum on $[1,5]$ at a critical point.
\begin{multipleChoice}
\choice{True}
\choice[correct]{False}
\end{multipleChoice}
\end{exercise}
\end{exercise}
\end{exercise}

\end{document}