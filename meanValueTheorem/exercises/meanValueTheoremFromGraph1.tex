\documentclass{ximera}


\graphicspath{
  {./}
  {ximeraTutorial/}
  {basicPhilosophy/}
}

\newcommand{\mooculus}{\textsf{\textbf{MOOC}\textnormal{\textsf{ULUS}}}}

\usepackage{tkz-euclide}\usepackage{tikz}
\usepackage{tikz-cd}
\usetikzlibrary{arrows}
\tikzset{>=stealth,commutative diagrams/.cd,
  arrow style=tikz,diagrams={>=stealth}} %% cool arrow head
\tikzset{shorten <>/.style={ shorten >=#1, shorten <=#1 } } %% allows shorter vectors

\usetikzlibrary{backgrounds} %% for boxes around graphs
\usetikzlibrary{shapes,positioning}  %% Clouds and stars
\usetikzlibrary{matrix} %% for matrix
\usepgfplotslibrary{polar} %% for polar plots
\usepgfplotslibrary{fillbetween} %% to shade area between curves in TikZ
\usetkzobj{all}
\usepackage[makeroom]{cancel} %% for strike outs
%\usepackage{mathtools} %% for pretty underbrace % Breaks Ximera
%\usepackage{multicol}
\usepackage{pgffor} %% required for integral for loops



%% http://tex.stackexchange.com/questions/66490/drawing-a-tikz-arc-specifying-the-center
%% Draws beach ball
\tikzset{pics/carc/.style args={#1:#2:#3}{code={\draw[pic actions] (#1:#3) arc(#1:#2:#3);}}}



\usepackage{array}
\setlength{\extrarowheight}{+.1cm}
\newdimen\digitwidth
\settowidth\digitwidth{9}
\def\divrule#1#2{
\noalign{\moveright#1\digitwidth
\vbox{\hrule width#2\digitwidth}}}






\DeclareMathOperator{\arccot}{arccot}
\DeclareMathOperator{\arcsec}{arcsec}
\DeclareMathOperator{\arccsc}{arccsc}

















%%This is to help with formatting on future title pages.
\newenvironment{sectionOutcomes}{}{}


\author{Nela Lakos}

\outcome{Global extrema from a graph.}
\outcome{Extreme value theorem.}


\begin{document}


\begin{exercise}
  Let $f$ be a function defined on $[0,4]$. The graph of $f$ is given in the figure below.
 Based on the graph, answer the questions below.
  \begin{image}
  \begin{tikzpicture}
    \begin{axis}[
        xmin=0,xmax=4,ymin=0,ymax=4,
        clip=false,
        unit vector ratio*=1 1 1,
        axis lines=center,
        grid = major,
        ytick={-1,...,4},
	xtick={0,...,4},
        xlabel=$x$, ylabel=$y$,
        every axis y label/.style={at=(current axis.above origin),anchor=south},
        every axis x label/.style={at=(current axis.right of origin),anchor=west},
      ]
     
       \addplot[very thick,penColor,domain=0:4] {(1/4)*x^2};
      \addplot[color=penColor,fill=penColor,only marks,mark=*] coordinates{(0,0)};  %% closed hole  
        \addplot[color=penColor,fill=penColor,only marks,mark=*] coordinates{(4,4)};  %% closed hole  
      \node[penColor] at (axis cs:1,3) [penColor] {$y=f(x)$};
      \end{axis}`
  \end{tikzpicture}
  \end{image}

  Select  the correct statement.  
  \begin{hint}
  Recall, a function $g$ has to be continuous on a closed interval $[a,b]$ and differentiable on an open interval $(a,b)$, so that we can apply the Mean value theorem.
  If the above  conditions are satisfied, the the Mean value theorem (MVT) guarantees that there exists a point $c$ in $(a,b)$ such that $f'(c)=\frac{f(b)-f(a)}{b-a}$. 
    \end{hint}
\begin{selectAll}
\choice[correct]{The function $f$ satisfies the conditions of the MVT on its domain.}
\choice{The function $f$ does not satisfy the conditions of the MVT on its domain, because  $f$ is not continuous on its domain.}
\choice{The function $f$ does not satisfy the conditions of the MVT on its domain, because $f$ is not differentiable on $(0,4)$.}
\end{selectAll}
\end{exercise}
  \begin{exercise}
   \begin{image}
  \begin{tikzpicture}
    \begin{axis}[
        xmin=0,xmax=4,ymin=0,ymax=4,
        clip=false,
        unit vector ratio*=1 1 1,
        axis lines=center,
        grid = major,
        ytick={-1,...,4},
	xtick={0,...,4},
        xlabel=$x$, ylabel=$y$,
        every axis y label/.style={at=(current axis.above origin),anchor=south},
        every axis x label/.style={at=(current axis.right of origin),anchor=west},
      ]
     
       \addplot[very thick,penColor,domain=0:4] {(1/4)*x^2};
      \addplot[very thick, color=red, smooth, domain=(0:4)] {x};
      \addplot[color=penColor,fill=penColor,only marks,mark=*] coordinates{(0,0)};  %% closed hole  
        \addplot[color=penColor,fill=penColor,only marks,mark=*] coordinates{(4,4)};  %% closed hole  
      \node[penColor] at (axis cs:1,3) [penColor] {$y=f(x)$};
      \end{axis}`
  \end{tikzpicture}
  \end{image}

 Let $m$ be the slope of the secant line through the points $(0,f(0))$ and $(4,f(4))$ (see figure above). Select the correct statement.
  
\begin{selectAll}
\choice{$m=4$}
\choice{$m=-4$}
\choice[correct]{$m=1$}
\choice{$m=-1$}
\choice{$m=0$}
\choice{$m=8$}
\choice{$m=-8$}
\choice{$m=2$}
\choice{$m=6$}
\choice{$m=8$}
\end{selectAll}
\end{exercise}
  \begin{exercise}
  Complete the statement below  regarding the function $f$ on the interval $[0,4]$.
  
  
 Select all correct choices.
 (Note, $m$, the slope of the secant line through the points $(0,f(0))$ and $(4,f(4))$, was computed in the previous exercise.)
 
 STATEMENT:
  
  The Mean value theorem guarantees that there exists a point $c$ in the open interval $(0,4)$ such that 
    \begin{selectAll}
\choice{$f'(c)=4$}
\choice{$f(c)=4$}
\choice{$f(c)=m$}
\choice[correct]{$f'(c)=m$}
\choice[correct]{$f'(c)=1$}
\choice{$f(c)=1$}
\choice{$f'(c)=-1$}
\choice{$f(c)=-1$}
\choice[correct]{$f'(c)=\frac{f(4)-f(0)}{4}$}
\choice{$f(c)=\frac{f(4)-f(0)}{4}$}
\end{selectAll}
\end{exercise}

 \begin{exercise}
Find a point $c$ in $(0,4)$, guaranteed by the MVT, such that

 $f'(c)=\frac{f(4)-f(0)}{4-0}$. Select the correct choice.
\begin{hint}
Observe the figure below. It shows that the \textbf{secant line} through the points $(0,f(0))$ and $(4,f(4))$ is \textbf{parallel} to the \textbf{tangent line} to the curve $y=f(x)$ at the point $x=\answer{2}$.
 \begin{image}
  \begin{tikzpicture}
    \begin{axis}[
        xmin=0,xmax=4,ymin=0,ymax=4,
        clip=false,
        unit vector ratio*=1 1 1,
        axis lines=center,
        grid = major,
        ytick={-1,...,4},
	xtick={0,...,4},
        xlabel=$x$, ylabel=$y$,
        every axis y label/.style={at=(current axis.above origin),anchor=south},
        every axis x label/.style={at=(current axis.right of origin),anchor=west},
      ]
     
       \addplot[very thick,penColor,domain=0:4] {(1/4)*x^2};
      \addplot[very thick, color=red, smooth, domain=(0:4)] {x};
       \addplot[very thick, color=red, smooth, domain=(1:4)] {x-1};
      \addplot[color=penColor,fill=penColor,only marks,mark=*] coordinates{(0,0)};  %% closed hole  
        \addplot[color=penColor,fill=penColor,only marks,mark=*] coordinates{(2,1)};  %% closed hole  
        \addplot[color=penColor,fill=penColor,only marks,mark=*] coordinates{(4,4)};  %% closed hole  
      \node[penColor] at (axis cs:1,3) [penColor] {$y=f(x)$};
      \end{axis}`
  \end{tikzpicture}
  \end{image}
\end{hint}
\begin{selectAll}
\choice{$c=1$}
\choice[correct]{$c=2$}
\choice{$c=3$}
\choice{$c=4$}
\end{selectAll}

\end{exercise}
\begin{exercise}
 Find an equation of tangent line to the curve $y=f(x)$ at the point where $x=2$.
 \begin{hint}
 You have enough information to write an equation of a line: you know the coordinates of a point $(c,f(c)$ and the slope of the (tangent) line, $f'(c)$.
  \end{hint}
  Select all correct choices.
  
 
  \begin{selectAll}
\choice{$y=1$}
\choice{$y=2$}
\choice{$y=f'(x)$}
\choice{$y-2=1(x-1)$}
\choice{$y=x+1$}
\choice[correct]{$y-1=1(x-2)$}
\choice[correct]{$y=x-1$}
\choice{$y=f(x)$}
\end{selectAll}
\end{exercise}

\end{document}
