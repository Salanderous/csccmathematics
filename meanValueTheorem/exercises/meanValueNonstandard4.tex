\documentclass{ximera}

\graphicspath{
  {./}
  {ximeraTutorial/}
  {basicPhilosophy/}
}

\newcommand{\mooculus}{\textsf{\textbf{MOOC}\textnormal{\textsf{ULUS}}}}

\usepackage{tkz-euclide}\usepackage{tikz}
\usepackage{tikz-cd}
\usetikzlibrary{arrows}
\tikzset{>=stealth,commutative diagrams/.cd,
  arrow style=tikz,diagrams={>=stealth}} %% cool arrow head
\tikzset{shorten <>/.style={ shorten >=#1, shorten <=#1 } } %% allows shorter vectors

\usetikzlibrary{backgrounds} %% for boxes around graphs
\usetikzlibrary{shapes,positioning}  %% Clouds and stars
\usetikzlibrary{matrix} %% for matrix
\usepgfplotslibrary{polar} %% for polar plots
\usepgfplotslibrary{fillbetween} %% to shade area between curves in TikZ
\usetkzobj{all}
\usepackage[makeroom]{cancel} %% for strike outs
%\usepackage{mathtools} %% for pretty underbrace % Breaks Ximera
%\usepackage{multicol}
\usepackage{pgffor} %% required for integral for loops



%% http://tex.stackexchange.com/questions/66490/drawing-a-tikz-arc-specifying-the-center
%% Draws beach ball
\tikzset{pics/carc/.style args={#1:#2:#3}{code={\draw[pic actions] (#1:#3) arc(#1:#2:#3);}}}



\usepackage{array}
\setlength{\extrarowheight}{+.1cm}
\newdimen\digitwidth
\settowidth\digitwidth{9}
\def\divrule#1#2{
\noalign{\moveright#1\digitwidth
\vbox{\hrule width#2\digitwidth}}}






\DeclareMathOperator{\arccot}{arccot}
\DeclareMathOperator{\arcsec}{arcsec}
\DeclareMathOperator{\arccsc}{arccsc}

















%%This is to help with formatting on future title pages.
\newenvironment{sectionOutcomes}{}{}

\author{Steven Gubkin}
\license{Creative Commons 3.0 By-NC}

\outcome{Understand the statement of the Mean Value Theorem.}
\outcome{Determine whether Rolle's Theorem or the Mean Value Theorem can be applied.}
\outcome{Identify calculus ideas which are consequences of the Mean Value Theorem.}
\begin{document}
\begin{exercise}

We will use the mean value theorem to show that $\sin x<x$ for $x>0$.

First we consider $0<x<2\pi$.  We apply the mean value theorem to $\sin$ on the interval $[0,x]$ to find there exists a $0<c<x$ with 
\[
\frac{\answer{\sin{x}}-\sin{0}}{\answer{x}-0} = \answer{\cos{c}}.
\]
Now we solve for $\sin{x}$ in the above expression to find
\[
\sin{x} = \answer{x\cos{c}}.
\]
Since $0<c<2\pi$ we know that 
\[
\cos{c} < \answer{1}.
\]
Using this bound we find that 
\[
\sin{x} < \answer{x}.
\]

Next for $x=2\pi$, $\sin{x}=\answer{0}$ and so $\sin{x}<x$.

Finally since $\sin$ is periodic with period $\answer{2\pi}$, for $x>2\pi$ there exists some $0<\chi\leq2\pi$ with $\sin{x}=\sin{\chi}$. (Note $\chi$ is entered \verb|\chi|.)  With this in mind
\[
\sin{x}=\sin{\chi}<\answer{\chi}<\answer{x}.
\]

Thus we have shown for all $x>0$ that $\sin{x}<x$.

\end{exercise}
\end{document}