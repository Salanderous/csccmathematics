\documentclass{ximera}


\graphicspath{
  {./}
  {ximeraTutorial/}
  {basicPhilosophy/}
}

\newcommand{\mooculus}{\textsf{\textbf{MOOC}\textnormal{\textsf{ULUS}}}}

\usepackage{tkz-euclide}\usepackage{tikz}
\usepackage{tikz-cd}
\usetikzlibrary{arrows}
\tikzset{>=stealth,commutative diagrams/.cd,
  arrow style=tikz,diagrams={>=stealth}} %% cool arrow head
\tikzset{shorten <>/.style={ shorten >=#1, shorten <=#1 } } %% allows shorter vectors

\usetikzlibrary{backgrounds} %% for boxes around graphs
\usetikzlibrary{shapes,positioning}  %% Clouds and stars
\usetikzlibrary{matrix} %% for matrix
\usepgfplotslibrary{polar} %% for polar plots
\usepgfplotslibrary{fillbetween} %% to shade area between curves in TikZ
\usetkzobj{all}
\usepackage[makeroom]{cancel} %% for strike outs
%\usepackage{mathtools} %% for pretty underbrace % Breaks Ximera
%\usepackage{multicol}
\usepackage{pgffor} %% required for integral for loops



%% http://tex.stackexchange.com/questions/66490/drawing-a-tikz-arc-specifying-the-center
%% Draws beach ball
\tikzset{pics/carc/.style args={#1:#2:#3}{code={\draw[pic actions] (#1:#3) arc(#1:#2:#3);}}}



\usepackage{array}
\setlength{\extrarowheight}{+.1cm}
\newdimen\digitwidth
\settowidth\digitwidth{9}
\def\divrule#1#2{
\noalign{\moveright#1\digitwidth
\vbox{\hrule width#2\digitwidth}}}






\DeclareMathOperator{\arccot}{arccot}
\DeclareMathOperator{\arcsec}{arcsec}
\DeclareMathOperator{\arccsc}{arccsc}

















%%This is to help with formatting on future title pages.
\newenvironment{sectionOutcomes}{}{}


\author{Jim Talamo and Bart Snapp}
\license{Creative Commons 3.0 By-bC}


\outcome{}


\begin{document}
\begin{exercise}

Consider a unit square and do the following: 

\begin{itemize}
\item[1.] Cut the square into two equal regions. Let
  $a_1$ be the area of one of these regions. 
\item[2.] Cut the other region into
  two equal regions and let $a_2$ be the area of one of these
  regions. 
\item[3.] Continue on \textit{ad-infinitium}.  
\end{itemize}

This process can be imagined as shown below:
  \begin{image}[2in]
    \begin{tikzpicture}[scale=3]
      \tkzDefPoint(0,0){A} 
      \tkzDefPoint(1,0){B} 
      \tkzDefPoint(1,1){C}
      \tkzDefPoint(0,1){D}
      \draw[penColor,very thick] (A)--(B)--(C)--(D)--cycle;

      \tkzDefPoint(.5,1){D} 
      \tkzDefPoint(.5,0){E} 
      \draw[penColor,very thick] (D)--(E);

      \tkzDefPoint(.5,.5){F} 
      \tkzDefPoint(1,.5){G} 
      \draw[penColor,very thick] (F)--(G);

      \tkzDefPoint(.75,1){H} 
      \tkzDefPoint(.75,.5){I} 
      \draw[penColor,very thick] (H)--(I);

      \tkzDefPoint(.75,.75){J} 
      \tkzDefPoint(1,.75){K} 
      \draw[penColor,very thick] (J)--(K);

      \tkzDefPoint(.875,.75){L} 
      \tkzDefPoint(.875,1){M} 
      \draw[penColor,very thick] (L)--(M);

      \tkzDefPoint(.875,.875){N} 
      \tkzDefPoint(1,.875){O} 
      \draw[penColor,very thick] (N)--(O);

      \node at (.25,.5) {$a_1$};
      \node at (.75,.25) {$a_2$};
      \node at (.6125,.75) {$a_3$};
      \node at (.875,.6125) {$a_4$};
    \end{tikzpicture}
  \end{image}
  
Define the sequence $\{a_n\}$ by this process by requiring that $a_n$ be the area of one of the regions made during the $n$-th cut.

    From the picture, we know $a_1 = \frac{1}{2}$ and can write out a few more terms:
    
\begin{align*}
a_2&= \answer{\frac{1}{4}} & a_3&= \answer{\frac{1}{8}} & a_4&= \answer{\frac{1}{16}} 
\end{align*}

This series is:
\begin{multipleChoice}
\choice{arithmetic.}
\choice[correct]{geometric.}
\choice{neither arithmetic nor geometric.}
\end{multipleChoice}

\begin{exercise}
The explicit formula is $a_n = \answer[given]{\left(\frac{1}{2}\right)^{n}}$ for $n \geq 1$. 
\end{exercise}

\begin{exercise}
The recursive formula $a_1 = 1/2$ and $a_n =
    \answer[given]{\left(\frac{1}{2}\right)}\cdot a_{n-1}$.
\end{exercise}

\end{exercise}
\end{document}
