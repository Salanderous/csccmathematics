\documentclass{ximera}


\graphicspath{
  {./}
  {ximeraTutorial/}
  {basicPhilosophy/}
}

\newcommand{\mooculus}{\textsf{\textbf{MOOC}\textnormal{\textsf{ULUS}}}}

\usepackage{tkz-euclide}\usepackage{tikz}
\usepackage{tikz-cd}
\usetikzlibrary{arrows}
\tikzset{>=stealth,commutative diagrams/.cd,
  arrow style=tikz,diagrams={>=stealth}} %% cool arrow head
\tikzset{shorten <>/.style={ shorten >=#1, shorten <=#1 } } %% allows shorter vectors

\usetikzlibrary{backgrounds} %% for boxes around graphs
\usetikzlibrary{shapes,positioning}  %% Clouds and stars
\usetikzlibrary{matrix} %% for matrix
\usepgfplotslibrary{polar} %% for polar plots
\usepgfplotslibrary{fillbetween} %% to shade area between curves in TikZ
\usetkzobj{all}
\usepackage[makeroom]{cancel} %% for strike outs
%\usepackage{mathtools} %% for pretty underbrace % Breaks Ximera
%\usepackage{multicol}
\usepackage{pgffor} %% required for integral for loops



%% http://tex.stackexchange.com/questions/66490/drawing-a-tikz-arc-specifying-the-center
%% Draws beach ball
\tikzset{pics/carc/.style args={#1:#2:#3}{code={\draw[pic actions] (#1:#3) arc(#1:#2:#3);}}}



\usepackage{array}
\setlength{\extrarowheight}{+.1cm}
\newdimen\digitwidth
\settowidth\digitwidth{9}
\def\divrule#1#2{
\noalign{\moveright#1\digitwidth
\vbox{\hrule width#2\digitwidth}}}






\DeclareMathOperator{\arccot}{arccot}
\DeclareMathOperator{\arcsec}{arcsec}
\DeclareMathOperator{\arccsc}{arccsc}

















%%This is to help with formatting on future title pages.
\newenvironment{sectionOutcomes}{}{}


\author{Jim Talamo}
\license{Creative Commons 3.0 By-bC}


\outcome{}


\begin{document}
\begin{exercise}

Suppose that $\{a_n\}_{n=1}$ is defined by the rule:

\[
 a_{n+1} = a_n+2(n+1) , \qquad a_1 = 8
\]

Then:

    \begin{align*}
       a_1 &= \answer[given]{8} & 
      a_2 &= \answer[given]{12} & 
      a_3 &= \answer[given]{18} \\
      a_4 &= \answer[given]{26} & 
      a_5 &= \answer[given]{36}  & 
      a_6 &= \answer[given]{48} 
    \end{align*}

\begin{exercise}
Define a new sequence: $\{b_n\}_{n=1}$ by the rule:

\[
b_n = \frac{1}{2}a_{n+2} - a_n
\]
Write out the first five terms in this new sequence.
    
     \begin{align*}
     	b_1 &=  \answer[given]{1}  & 
      	b_2 &=  \answer[given]{1}  & 
	b_3 &= \answer[given]{0}  \\ 
	b_4 &= \answer[given]{-2}  & 
	b_5 &=  \answer[given]{-5}  & 
    \end{align*}
    

    
\end{exercise}

\begin{exercise}
Define another new sequence: $\{c_n\}_{n=1}$ by the rule:

\[
c_n = \frac{b_n}{a_n} 
\]
Write out the first five terms in this new sequence.

     \begin{align*}
     	c_1 &=  \answer[given]{\frac{1}{8}}  & 
      	c_2 &=  \answer[given]{\frac{1}{12}}  & 
	c_3 &= \answer[given]{0}   \\ 
	c_4 &= \answer[given]{-\frac{1}{13}}   & 
	c_5 &= \answer[given]{-\frac{5}{36}}   & 
    \end{align*}
    

\end{exercise}

\begin{exercise}
Define another new sequence: $\{s_n\}_{n=1}$ by the rule:

\[
s_n = \sum_{k=1}^n a_k 
\]
Write out the first five terms in this new sequence.

Starting with $n=1$ we find:

\[      s_1 = a_1 = \answer[given]{8}       \]
      
Continuing in the same way, we find:     
     \begin{align*}
      	s_2 &=  a_1 +a_2 = \answer[given]{20}  \\ 
	s_3 &=  a_1 +a_2 + a_3 = \answer[given]{38}   \\ 
	s_4 &=  a_1 +a_2 +a_3 +a_4= \answer[given]{64}  \\ 
	s_5 &=  a_1 +a_2 +a_3+a_4+a_5= \answer[given]{100}    
    \end{align*}

\end{exercise}   
    
    \end{exercise}
\end{document}
