\documentclass{ximera}


\graphicspath{
  {./}
  {ximeraTutorial/}
  {basicPhilosophy/}
}

\newcommand{\mooculus}{\textsf{\textbf{MOOC}\textnormal{\textsf{ULUS}}}}

\usepackage{tkz-euclide}\usepackage{tikz}
\usepackage{tikz-cd}
\usetikzlibrary{arrows}
\tikzset{>=stealth,commutative diagrams/.cd,
  arrow style=tikz,diagrams={>=stealth}} %% cool arrow head
\tikzset{shorten <>/.style={ shorten >=#1, shorten <=#1 } } %% allows shorter vectors

\usetikzlibrary{backgrounds} %% for boxes around graphs
\usetikzlibrary{shapes,positioning}  %% Clouds and stars
\usetikzlibrary{matrix} %% for matrix
\usepgfplotslibrary{polar} %% for polar plots
\usepgfplotslibrary{fillbetween} %% to shade area between curves in TikZ
\usetkzobj{all}
\usepackage[makeroom]{cancel} %% for strike outs
%\usepackage{mathtools} %% for pretty underbrace % Breaks Ximera
%\usepackage{multicol}
\usepackage{pgffor} %% required for integral for loops



%% http://tex.stackexchange.com/questions/66490/drawing-a-tikz-arc-specifying-the-center
%% Draws beach ball
\tikzset{pics/carc/.style args={#1:#2:#3}{code={\draw[pic actions] (#1:#3) arc(#1:#2:#3);}}}



\usepackage{array}
\setlength{\extrarowheight}{+.1cm}
\newdimen\digitwidth
\settowidth\digitwidth{9}
\def\divrule#1#2{
\noalign{\moveright#1\digitwidth
\vbox{\hrule width#2\digitwidth}}}






\DeclareMathOperator{\arccot}{arccot}
\DeclareMathOperator{\arcsec}{arcsec}
\DeclareMathOperator{\arccsc}{arccsc}

















%%This is to help with formatting on future title pages.
\newenvironment{sectionOutcomes}{}{}


\author{Jim Talamo and Bart Snapp}
\license{Creative Commons 3.0 By-bC}


\outcome{}


\begin{document}
\begin{exercise}

  Suppose you have $100$ dollars in your bank account and you earn
  $2.25\%$ interest per year. Let $m_n$ be the amount of money in your
  account after $n$ years. 
  
  Determine the amount of money you have in your account after years 1, 2, 3, and 4 to 2 decimal places:
  
  \begin{align*}
m_1&= \answer[tolerance=.03]{102.25} &m_2&= \answer[tolerance=.03]{104.55} &m_3&= \answer[tolerance=.03]{106.90} &m_4&= \answer[tolerance=.03]{109.31} &
\end{align*}
  
\begin{exercise}
This series is:
\begin{multipleChoice}
\choice{arithmetic.}
\choice[correct]{geometric.}
\choice{neither arithmetic nor geometric.}
\end{multipleChoice}

\begin{hint}
    After one year, your bank account has $m_1 = 1.0225\cdot 100$ dollars
    in it. To find the amount in each successive year, you multiply
    again by $1.0225$. Hence this is a geometric sequence!
\end{hint}

This is a geometric sequence and can be given by both an explicit and recursive formula:

\begin{exercise}
The explicit formula is $m_n = \answer[given]{100 \cdot (1.0225)^n}$ for $n \geq 0$. 
\end{exercise}

\begin{exercise}
The recursive formula $m_0 = 100$ and $m_n =
    \answer[given]{1.0225}\cdot m_{n-1}$.
\end{exercise}
\end{exercise}
\end{exercise}
\end{document}
