\documentclass{ximera}


\graphicspath{
  {./}
  {ximeraTutorial/}
  {basicPhilosophy/}
}

\newcommand{\mooculus}{\textsf{\textbf{MOOC}\textnormal{\textsf{ULUS}}}}

\usepackage{tkz-euclide}\usepackage{tikz}
\usepackage{tikz-cd}
\usetikzlibrary{arrows}
\tikzset{>=stealth,commutative diagrams/.cd,
  arrow style=tikz,diagrams={>=stealth}} %% cool arrow head
\tikzset{shorten <>/.style={ shorten >=#1, shorten <=#1 } } %% allows shorter vectors

\usetikzlibrary{backgrounds} %% for boxes around graphs
\usetikzlibrary{shapes,positioning}  %% Clouds and stars
\usetikzlibrary{matrix} %% for matrix
\usepgfplotslibrary{polar} %% for polar plots
\usepgfplotslibrary{fillbetween} %% to shade area between curves in TikZ
\usetkzobj{all}
\usepackage[makeroom]{cancel} %% for strike outs
%\usepackage{mathtools} %% for pretty underbrace % Breaks Ximera
%\usepackage{multicol}
\usepackage{pgffor} %% required for integral for loops



%% http://tex.stackexchange.com/questions/66490/drawing-a-tikz-arc-specifying-the-center
%% Draws beach ball
\tikzset{pics/carc/.style args={#1:#2:#3}{code={\draw[pic actions] (#1:#3) arc(#1:#2:#3);}}}



\usepackage{array}
\setlength{\extrarowheight}{+.1cm}
\newdimen\digitwidth
\settowidth\digitwidth{9}
\def\divrule#1#2{
\noalign{\moveright#1\digitwidth
\vbox{\hrule width#2\digitwidth}}}






\DeclareMathOperator{\arccot}{arccot}
\DeclareMathOperator{\arcsec}{arcsec}
\DeclareMathOperator{\arccsc}{arccsc}

















%%This is to help with formatting on future title pages.
\newenvironment{sectionOutcomes}{}{}


\author{Jim Talamo and Bart Snapp}
\license{Creative Commons 3.0 By-bC}


\outcome{}


\begin{document}
\begin{exercise}

Suppose that $\{a_n\}_{n=1}$ is an \emph{arithmetic} sequence, whose first few terms are shown below:

\[
2, 6, 10, 14, 18, \ldots
\]

Note that since we are told that the sequence is arithmetic:

\begin{multipleChoice}
\choice[correct]{The difference between subsequent terms is constant.}
\choice{The ratio of subsequent terms is constant.}
\end{multipleChoice}

\begin{exercise}
In fact, we notice:
  \begin{image}
  \begin{tikzpicture}[node distance=1.5cm]
    \node (a1) {$2$,};
    \node (a2) [right of=a1] {$6$,};
    \node (a3) [right of=a2] {$10$,};
    \node (a4) [right of=a3] {$14$,};
    \node (a5) [right of=a4] {$18$,};
    \node (a6) [right of=a5] {$\ldots$};

    \path[->] (a1) edge [bend left=20] node[above]{$4$} (a2);
    \path[->] (a2) edge [bend left=20] node[above]{$4$} (a3);
    \path[->] (a3) edge [bend left=20] node[above]{$4$} (a4);
    \path[->] (a4) edge [bend left=20] node[above]{$4$} (a5);
    \path[->] (a5) edge [bend left=20] node[above]{$4$} (a6);
  \end{tikzpicture}
  \end{image}
  
  We can describe this sequence explicitly or recursively. In fact:
  
  This sequence is given explicitly by the function $a_n=\answer[given]{4n - 2}$ for $n \geq 1$.
  
  This sequence is given recursively by the rule $a_1 = \answer[given]{2}$ and $a_{n+1} = a_n +
  \answer[given]{4}$. 

\begin{exercise}
What is $a_{505}$ for this sequence?

\[
a_{505}=\answer{2018}
\]

Which rule is easier to use to find this by hand?

\end{exercise}
  
\end{exercise}
\end{exercise}
\end{document}
