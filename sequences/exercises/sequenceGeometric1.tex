\documentclass{ximera}


\graphicspath{
  {./}
  {ximeraTutorial/}
  {basicPhilosophy/}
}

\newcommand{\mooculus}{\textsf{\textbf{MOOC}\textnormal{\textsf{ULUS}}}}

\usepackage{tkz-euclide}\usepackage{tikz}
\usepackage{tikz-cd}
\usetikzlibrary{arrows}
\tikzset{>=stealth,commutative diagrams/.cd,
  arrow style=tikz,diagrams={>=stealth}} %% cool arrow head
\tikzset{shorten <>/.style={ shorten >=#1, shorten <=#1 } } %% allows shorter vectors

\usetikzlibrary{backgrounds} %% for boxes around graphs
\usetikzlibrary{shapes,positioning}  %% Clouds and stars
\usetikzlibrary{matrix} %% for matrix
\usepgfplotslibrary{polar} %% for polar plots
\usepgfplotslibrary{fillbetween} %% to shade area between curves in TikZ
\usetkzobj{all}
\usepackage[makeroom]{cancel} %% for strike outs
%\usepackage{mathtools} %% for pretty underbrace % Breaks Ximera
%\usepackage{multicol}
\usepackage{pgffor} %% required for integral for loops



%% http://tex.stackexchange.com/questions/66490/drawing-a-tikz-arc-specifying-the-center
%% Draws beach ball
\tikzset{pics/carc/.style args={#1:#2:#3}{code={\draw[pic actions] (#1:#3) arc(#1:#2:#3);}}}



\usepackage{array}
\setlength{\extrarowheight}{+.1cm}
\newdimen\digitwidth
\settowidth\digitwidth{9}
\def\divrule#1#2{
\noalign{\moveright#1\digitwidth
\vbox{\hrule width#2\digitwidth}}}






\DeclareMathOperator{\arccot}{arccot}
\DeclareMathOperator{\arcsec}{arcsec}
\DeclareMathOperator{\arccsc}{arccsc}

















%%This is to help with formatting on future title pages.
\newenvironment{sectionOutcomes}{}{}


\author{Jim Talamo and Bart Snapp}
\license{Creative Commons 3.0 By-bC}


\outcome{}


\begin{document}
\begin{exercise}

Suppose that $\{a_n\}_{n=1}$ is an \emph{geometric} sequence, whose first few terms are shown below:

\[
\frac{3}{5}, \frac{-3}{25}, \frac{3}{125}, -\frac{3}{625}, \frac{3}{3125}, \ldots
\]

Note that since we are told that the sequence is geometric:

\begin{multipleChoice}
\choice{The difference between subsequent terms is constant.}
\choice[correct]{The ratio of subsequent terms is constant.}
\end{multipleChoice}

In fact, the ratio between successive terms is $\answer{-\frac{1}{5}}$:

\begin{exercise}

  \begin{image}
    \begin{tikzpicture}[node distance=1.5cm]
    \node (a1) {$\frac{3}{5}$,};
    \node (a2) [right of=a1] {$\frac{-3}{25}$,};
    \node (a3) [right of=a2] {$\frac{3}{125}$,};
    \node (a4) [right of=a3] {$\frac{-3}{625}$,};
    \node (a5) [right of=a4] {$\frac{3}{3125}$,};
    \node (a6) [right of=a5] {$\ldots$};

    \path[->] (a1) edge [bend left=20] node[above] {$\times\frac{-1}{5}$} (a2);
    \path[->] (a2) edge [bend left=20] node[above] {$\times\frac{-1}{5}$} (a3);
    \path[->] (a3) edge [bend left=20] node[above] {$\times\frac{-1}{5}$} (a4);
    \path[->] (a4) edge [bend left=20] node[above] {$\times\frac{-1}{5}$} (a5);
    \path[->] (a5) edge [bend left=20] node[above] {$\times\frac{-1}{5}$} (a6);
  \end{tikzpicture}
  \end{image}
  This sequence can be described both explicitly and recursively.
  
  It is given by the explicit formula $a_n=\answer[given]{\left(\frac{3}{5}\right)\cdot
    \left(\frac{-1}{5}\right)^{n-1}}$ for $n \geq 1$.
    
    It is given by the recursive rule $a_1 = \answer[given]{\frac{3}{5}}$ and
  $a_{n+1} = \answer[given]{\left(\frac{-1}{5}\right)}\cdot a_n$. 
  
\end{exercise}
\end{exercise}
\end{document}
