\documentclass{ximera}


\graphicspath{
  {./}
  {ximeraTutorial/}
  {basicPhilosophy/}
}

\newcommand{\mooculus}{\textsf{\textbf{MOOC}\textnormal{\textsf{ULUS}}}}

\usepackage{tkz-euclide}\usepackage{tikz}
\usepackage{tikz-cd}
\usetikzlibrary{arrows}
\tikzset{>=stealth,commutative diagrams/.cd,
  arrow style=tikz,diagrams={>=stealth}} %% cool arrow head
\tikzset{shorten <>/.style={ shorten >=#1, shorten <=#1 } } %% allows shorter vectors

\usetikzlibrary{backgrounds} %% for boxes around graphs
\usetikzlibrary{shapes,positioning}  %% Clouds and stars
\usetikzlibrary{matrix} %% for matrix
\usepgfplotslibrary{polar} %% for polar plots
\usepgfplotslibrary{fillbetween} %% to shade area between curves in TikZ
\usetkzobj{all}
\usepackage[makeroom]{cancel} %% for strike outs
%\usepackage{mathtools} %% for pretty underbrace % Breaks Ximera
%\usepackage{multicol}
\usepackage{pgffor} %% required for integral for loops



%% http://tex.stackexchange.com/questions/66490/drawing-a-tikz-arc-specifying-the-center
%% Draws beach ball
\tikzset{pics/carc/.style args={#1:#2:#3}{code={\draw[pic actions] (#1:#3) arc(#1:#2:#3);}}}



\usepackage{array}
\setlength{\extrarowheight}{+.1cm}
\newdimen\digitwidth
\settowidth\digitwidth{9}
\def\divrule#1#2{
\noalign{\moveright#1\digitwidth
\vbox{\hrule width#2\digitwidth}}}






\DeclareMathOperator{\arccot}{arccot}
\DeclareMathOperator{\arcsec}{arcsec}
\DeclareMathOperator{\arccsc}{arccsc}

















%%This is to help with formatting on future title pages.
\newenvironment{sectionOutcomes}{}{}

\outcome{Define the cross product.}
\outcome{Compute cross products.}
\outcome{Use cross products in applied settings.}

\title[Dig-In:]{The cross product}

\begin{document}
\begin{abstract}
  The cross product is a special way to multiply two vectors in
  three-dimensional space.
\end{abstract}
\maketitle

There is no useful way to ``multiply'' two vectors and obtain another
vector in $\mathbb{R}^n$ for arbitrary $n$. However, in the special case of
$\mathbb{R}^3$, there is an important multiplication operation called ``the
cross product.''

The cross product is linked inextricably to the determinant, so we
will first introduce the determinant before introducing this new
operation. 

\section{Determinants}

\begin{definition}
  Given a $2\times2$ matrix, the \textbf{determinant} is given by
  \[
  \det
  \begin{bmatrix}
    a & b\\
    c & d
  \end{bmatrix}
  =
  \begin{vmatrix}
    a & b\\
    c & d
  \end{vmatrix}
  = ad -bc.
  \]
\end{definition}
Why would anyone ever be interested in this? Well to start, given nonzero vectors
\[
\overset{\boldsymbol{\rightharpoonup}}{\mathbf{v}} = \left< a,b \rigth> \quad\text{and}\quad \overset{\boldsymbol{\rightharpoonup}}{\mathbf{w}} = \left< c,d \right>
\]
we can form a parallelogram as below.
\begin{image}
\begin{tikzpicture}
  \draw[fill=fill1!50!white,draw=none]
  (0, 0)         % starting point
  -- ++(5,1)     % move along this vector
  -- ++(2,3)     % then along this vector
  -- ++(-5,-1)   % then back along that vector
  -- cycle;      % and back to where you started
  \draw[->,ultra thick,penColor] (0,0) -- (5,1);
  \draw[->,ultra thick,penColor2] (0,0) -- (2,3);
  \draw[->,ultra thick,dashed,penColor2] (5,1) -- (7,4);
  \draw[->,ultra thick,dashed,penColor] (2,3) -- (7,4);
  \node[above,penColor] at (2.5,.5) {$\overset{\boldsymbol{\rightharpoonup}}{\mathbf{v}}$}; %% <a,b>
  \node[below right,penColor2] at (1,1.5) {$\overset{\boldsymbol{\rightharpoonup}}{\mathbf{w}}$}; %% <c,d>
  \draw[->] (.5,.1) arc[radius=.5cm,start angle=11.3,end angle=56.3];
  \node[above right] at (.4,.2) {$\theta$}; 
\end{tikzpicture}
\end{image}
The determinant gives the (signed) area of this parallelogram, where
the sign of the area is given by the sign of the angle $\theta$ (drawn
counterclockwise) between $\overset{\boldsymbol{\rightharpoonup}}{\mathbf{v}}$ and $\overset{\boldsymbol{\rightharpoonup}}{\mathbf{w}}$. To understand why
this is true, make a rectangle around the parallelogram as below.
\begin{image}
\begin{tikzpicture}
  \draw[fill=fill1!50!white,draw=none]
  (0, 0)         % starting point
  -- ++(5,1)     % move along this vector
  -- ++(2,3)     % then along this vector
  -- ++(-5,-1)   % then back along that vector
  -- cycle;      % and back to where you started

  \draw[fill=fill5] (5, 0) -- (7,0) -- (7,1) -- (5,1) -- cycle;
  \draw[fill=fill5] (0, 3) -- (2,3) -- (2,4) -- (0,4) -- cycle;

  \draw[fill=fill3] (0, 0) -- (5,0) -- (5,1) -- cycle;
  \draw[fill=fill3] (2,3) -- (7,4) -- (2,4) -- cycle;

  \draw[fill=fill4] (0, 0) -- (2,3) -- (0,3) -- cycle;
  \draw[fill=fill4] (5, 1) -- (7,1) -- (7,4) -- cycle;
  
  \draw[->,ultra thick,penColor] (0,0) -- (5,1);
  \draw[->,ultra thick,penColor2] (0,0) -- (2,3);
  \draw[->,ultra thick,dashed,penColor2] (5,1) -- (7,4);
  \draw[->,ultra thick,dashed,penColor] (2,3) -- (7,4);

  \draw[decoration={brace,mirror,raise=.1cm},decorate,thin] (0,0)--(5,0);
  \node[below] at (2.5,-.2) {$a$};

  \draw[decoration={brace,mirror,raise=.1cm},decorate,thin] (7,4)--(2,4);
  \node[above] at (4.5,4.2) {$a$};

  \draw[decoration={brace,mirror,raise=.1cm},decorate,thin] (7,0)--(7,1);
  \node[right] at (7.2,.5) {$b$};

  \draw[decoration={brace,mirror,raise=.1cm},decorate,thin] (0,4)--(0,3);
  \node[left] at (-.2,3.5) {$b$};

  \draw[decoration={brace,mirror,raise=.1cm},decorate,thin] (7,1)--(7,4);
  \node[right] at (7.2,2.5) {$d$};

  \draw[decoration={brace,mirror,raise=.1cm},decorate,thin] (0,3)--(0,0);
  \node[left] at (-.2,1.5) {$d$};

  \draw[decoration={brace,mirror,raise=.1cm},decorate,thin] (0,3)--(0,0);
  \node[left] at (-.2,1.5) {$d$};

  \draw[decoration={brace,raise=.1cm},decorate,thin] (0,4)--(2,4);
  \node[above] at (1,4.2) {$c$};

  \draw[decoration={brace,mirror,raise=.1cm},decorate,thin] (5,0)--(7,0);
  \node[below] at (6,-.2) {$c$};
  
  \node[above,penColor] at (2.5,.5) {$\overset{\boldsymbol{\rightharpoonup}}{\mathbf{v}}$}; %% <a,b>
  \node[below right,penColor2] at (1,1.5) {$\overset{\boldsymbol{\rightharpoonup}}{\mathbf{w}}$}; %% <c,d>
\end{tikzpicture}
\end{image}
Now, we see that the area of the parallelogram is
\[
\text{Area of rectangle} - \text{Area of other regions}
\]
and this is
\begin{align*}
  (a+c)(b+d) - \left(cd + ab + 2bc \right)&= ab + ad + bc + cd - cd - ab-2bc\\
  &=  ad - bc\\
  &= \det
  \begin{bmatrix}
    a & b\\
    c & d
  \end{bmatrix}.
\end{align*}




































\end{document}
