\documentclass{ximera}


\graphicspath{
  {./}
  {ximeraTutorial/}
  {basicPhilosophy/}
}

\newcommand{\mooculus}{\textsf{\textbf{MOOC}\textnormal{\textsf{ULUS}}}}

\usepackage{tkz-euclide}\usepackage{tikz}
\usepackage{tikz-cd}
\usetikzlibrary{arrows}
\tikzset{>=stealth,commutative diagrams/.cd,
  arrow style=tikz,diagrams={>=stealth}} %% cool arrow head
\tikzset{shorten <>/.style={ shorten >=#1, shorten <=#1 } } %% allows shorter vectors

\usetikzlibrary{backgrounds} %% for boxes around graphs
\usetikzlibrary{shapes,positioning}  %% Clouds and stars
\usetikzlibrary{matrix} %% for matrix
\usepgfplotslibrary{polar} %% for polar plots
\usepgfplotslibrary{fillbetween} %% to shade area between curves in TikZ
\usetkzobj{all}
\usepackage[makeroom]{cancel} %% for strike outs
%\usepackage{mathtools} %% for pretty underbrace % Breaks Ximera
%\usepackage{multicol}
\usepackage{pgffor} %% required for integral for loops



%% http://tex.stackexchange.com/questions/66490/drawing-a-tikz-arc-specifying-the-center
%% Draws beach ball
\tikzset{pics/carc/.style args={#1:#2:#3}{code={\draw[pic actions] (#1:#3) arc(#1:#2:#3);}}}



\usepackage{array}
\setlength{\extrarowheight}{+.1cm}
\newdimen\digitwidth
\settowidth\digitwidth{9}
\def\divrule#1#2{
\noalign{\moveright#1\digitwidth
\vbox{\hrule width#2\digitwidth}}}






\DeclareMathOperator{\arccot}{arccot}
\DeclareMathOperator{\arcsec}{arcsec}
\DeclareMathOperator{\arccsc}{arccsc}

















%%This is to help with formatting on future title pages.
\newenvironment{sectionOutcomes}{}{}


\outcome{}

\title[Break-Ground:]{Replacing curves with lines}

\begin{document}
\begin{abstract}
Two young mathematicians discuss linear approximation.
\end{abstract}
\maketitle

Check out this dialogue between two calculus students (based on a true
story):



\begin{dialogue}
\item[Devyn] Hmmmm. Riley, I just thought of something\dots
\item[Riley] What is it?
\item[Devyn] When we compute derivatives, we are looking at the slope
  of tangent lines right?
\item[Riley] You know it.
\item[Devyn] Well, I wonder: Instead of studying curves, could we just
  study ``zoomed-in'' lines?
\item[Riley] I'm not sure\dots
\end{dialogue}


You read someplace that
\[
\l(x) = \frac{1}{4}(x-4)+2
\]
is a good approximation for $f(x) = \sqrt{x}$ when $x$ is close to
$4$.

\begin{problem}
  Plot $\l(x)$ and $f(x)$. Explain how this shows that $\ell(x)$ is a
  good approximation when $x$ is close to $4$.
  \begin{prompt}
  \begin{freeResponse}
  \end{freeResponse}
  \end{prompt}
\end{problem}

\begin{problem}
 Explain (if you can) using concepts of calculus to explain why
 $\ell(x)$ is a good approximation for $f(x)$ when $x$ is close to $4$.
 \begin{prompt}
  \begin{freeResponse}
  \end{freeResponse}
 \end{prompt}
\end{problem}




%\input{../leveledQuestions.tex}

\end{document}
