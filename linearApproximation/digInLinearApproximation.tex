\documentclass{ximera}



\graphicspath{
  {./}
  {ximeraTutorial/}
  {basicPhilosophy/}
}

\newcommand{\mooculus}{\textsf{\textbf{MOOC}\textnormal{\textsf{ULUS}}}}

\usepackage{tkz-euclide}\usepackage{tikz}
\usepackage{tikz-cd}
\usetikzlibrary{arrows}
\tikzset{>=stealth,commutative diagrams/.cd,
  arrow style=tikz,diagrams={>=stealth}} %% cool arrow head
\tikzset{shorten <>/.style={ shorten >=#1, shorten <=#1 } } %% allows shorter vectors

\usetikzlibrary{backgrounds} %% for boxes around graphs
\usetikzlibrary{shapes,positioning}  %% Clouds and stars
\usetikzlibrary{matrix} %% for matrix
\usepgfplotslibrary{polar} %% for polar plots
\usepgfplotslibrary{fillbetween} %% to shade area between curves in TikZ
\usetkzobj{all}
\usepackage[makeroom]{cancel} %% for strike outs
%\usepackage{mathtools} %% for pretty underbrace % Breaks Ximera
%\usepackage{multicol}
\usepackage{pgffor} %% required for integral for loops



%% http://tex.stackexchange.com/questions/66490/drawing-a-tikz-arc-specifying-the-center
%% Draws beach ball
\tikzset{pics/carc/.style args={#1:#2:#3}{code={\draw[pic actions] (#1:#3) arc(#1:#2:#3);}}}



\usepackage{array}
\setlength{\extrarowheight}{+.1cm}
\newdimen\digitwidth
\settowidth\digitwidth{9}
\def\divrule#1#2{
\noalign{\moveright#1\digitwidth
\vbox{\hrule width#2\digitwidth}}}






\DeclareMathOperator{\arccot}{arccot}
\DeclareMathOperator{\arcsec}{arcsec}
\DeclareMathOperator{\arccsc}{arccsc}

















%%This is to help with formatting on future title pages.
\newenvironment{sectionOutcomes}{}{}


\outcome{Define linear approximation as an application of the tangent to a curve.}
\outcome{Find the linear approximation to a function at a point and use it to approximate the function value.}
\outcome{Identify when a linear approximation can be used.}
\outcome{Label a graph with the appropriate quantities used in linear approximation.}
\outcome{Find the error of a linear approximation.}
\outcome{Compute differentials.}
%\outcome{Use the second derivative to discuss whether the linear approximation over or underestimates the actual function value.}
\outcome{Contrast the notation and meaning of $dy$ versus $\Delta y$.}
%\outcome{Understand that the error shrinks faster than the displacement in the input.}
\outcome{Justify the chain rule via the composition of linear approximations.}


\title[Dig-In:]{Linear approximation}

\begin{document}
\begin{abstract}
We use a method called ``linear approximation'' to estimate the value
of a (complicated) function at a given point.
%  We derive the constant rule, sum rule, power rule, and product rule. 
\end{abstract}
\maketitle

%\section{Linear approximation}

Given a function, a \textit{linear approximation} is a fancy phrase
for something you already know:
\begin{quote}
  \textbf{The line tangent to the graph of a function at a point is very close to the graph of the function near that point.}
\end{quote}
This tangent line is the graph of a linear function, called  the \textbf{linear approximation}.
\begin{example}
Let $f$ be a function that is differentiable on some interval I that contains the point $a$. The graph of a function $f$  and the line tangent to the curve

 $y=f(x)$ at the point where $x=a$ are given in the figure below.
Find the equation of the tangent line.
 \begin{image}
%\begin{marginfigure}
\begin{tikzpicture}
	\begin{axis}[
            xmin=0,xmax=2,ymin=0,ymax=2,
            axis lines=center,
            ticks=none,
            %width=3in,
            %height=2in,
            unit vector ratio*=1 1 1,
            xlabel=$x$, ylabel=$y$,
            every axis y label/.style={at=(current axis.above origin),anchor=south},
            every axis x label/.style={at=(current axis.right of origin),anchor=west},
          ]        
          \addplot [ thick, penColor, smooth, domain=(0:3)] {3*sqrt(x)-2};
          \addplot [thick, penColor2,smooth] {(3/2)*x+3/2-2};
          \node at (axis cs:1.7,1.5) [penColor] {$y=f(x)$};
          \node at (axis cs:1,1.5) [penColor2] {$y=L(x)$}; 
          \node at (axis cs:1.3,1) [penColor2] {$(a,f(a))$}; 
            \addplot[color=penColor3,fill=penColor3,only marks,mark=*] coordinates{(1,1)};  %% closed hole         
        \end{axis}
\end{tikzpicture}
%\caption{A linear approximation of $f(x) = \sin(x)$ at $x=0$.}
%\label{figure:la sin}
%\end{marginfigure}
\end{image}
First, find the expression for $m$, the slope of the tangent line to the curve $y=f(x)$ at the point $(a,f(a))$.
 Select the correct choice.
 






































 

\end{document}
