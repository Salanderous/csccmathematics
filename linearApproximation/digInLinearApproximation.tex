\documentclass{ximera}



\graphicspath{
  {./}
  {ximeraTutorial/}
  {basicPhilosophy/}
}

\newcommand{\mooculus}{\textsf{\textbf{MOOC}\textnormal{\textsf{ULUS}}}}

\usepackage{tkz-euclide}\usepackage{tikz}
\usepackage{tikz-cd}
\usetikzlibrary{arrows}
\tikzset{>=stealth,commutative diagrams/.cd,
  arrow style=tikz,diagrams={>=stealth}} %% cool arrow head
\tikzset{shorten <>/.style={ shorten >=#1, shorten <=#1 } } %% allows shorter vectors

\usetikzlibrary{backgrounds} %% for boxes around graphs
\usetikzlibrary{shapes,positioning}  %% Clouds and stars
\usetikzlibrary{matrix} %% for matrix
\usepgfplotslibrary{polar} %% for polar plots
\usepgfplotslibrary{fillbetween} %% to shade area between curves in TikZ
\usetkzobj{all}
\usepackage[makeroom]{cancel} %% for strike outs
%\usepackage{mathtools} %% for pretty underbrace % Breaks Ximera
%\usepackage{multicol}
\usepackage{pgffor} %% required for integral for loops



%% http://tex.stackexchange.com/questions/66490/drawing-a-tikz-arc-specifying-the-center
%% Draws beach ball
\tikzset{pics/carc/.style args={#1:#2:#3}{code={\draw[pic actions] (#1:#3) arc(#1:#2:#3);}}}



\usepackage{array}
\setlength{\extrarowheight}{+.1cm}
\newdimen\digitwidth
\settowidth\digitwidth{9}
\def\divrule#1#2{
\noalign{\moveright#1\digitwidth
\vbox{\hrule width#2\digitwidth}}}






\DeclareMathOperator{\arccot}{arccot}
\DeclareMathOperator{\arcsec}{arcsec}
\DeclareMathOperator{\arccsc}{arccsc}

















%%This is to help with formatting on future title pages.
\newenvironment{sectionOutcomes}{}{}


\outcome{Define linear approximation as an application of the tangent to a curve.}
\outcome{Find the linear approximation to a function at a point and use it to approximate the function value.}
\outcome{Identify when a linear approximation can be used.}
\outcome{Label a graph with the appropriate quantities used in linear approximation.}
\outcome{Find the error of a linear approximation.}
\outcome{Compute differentials.}
%\outcome{Use the second derivative to discuss whether the linear approximation over or underestimates the actual function value.}
\outcome{Contrast the notation and meaning of $dy$ versus $\Delta y$.}
%\outcome{Understand that the error shrinks faster than the displacement in the input.}
\outcome{Justify the chain rule via the composition of linear approximations.}


\title[Dig-In:]{Linear approximation}

\begin{document}
\begin{abstract}
We use a method called ``linear approximation'' to estimate the value
of a (complicated) function at a given point.
%  We derive the constant rule, sum rule, power rule, and product rule. 
\end{abstract}
\maketitle

%\section{Linear approximation}

Given a function, a \textit{linear approximation} is a fancy phrase
for something you already know:
\begin{quote}
  \textbf{The line tangent to the graph of a function at a point is very close to the graph of the function near that point.}
\end{quote}
This tangent line is the graph of a linear function, called  the \textbf{linear approximation}.
\begin{example}
Let $f$ be a function that is differentiable on some interval I that contains the point $a$. The graph of a function $f$  and the line tangent to the curve

 $y=f(x)$ at the point where $x=a$ are given in the figure below.
Find the equation of the tangent line.






 \begin{image}
%\begin{marginfigure}
\begin{tikzpicture}
  \begin{axis}[
            xmin=0,xmax=2,ymin=0,ymax=2,
            axis lines=center,
            ticks=none,
            %width=3in,
            %height=2in,
            unit vector ratio*=1 1 1,
            xlabel=$x$, ylabel=$y$,
            every axis y label/.style={at=(current axis.above origin),anchor=south},
            every axis x label/.style={at=(current axis.right of origin),anchor=west},
          ]        
          \addplot [ thick, penColor, smooth, domain=(0:3)] {3*sqrt(x)-2};
          \addplot [thick, penColor2,smooth] {(3/2)*x+3/2-2};
          \node at (axis cs:1.7,1.5) [penColor] {$y=f(x)$};
          \node at (axis cs:1,1.5) [penColor2] {$y=L(x)$}; 
          \node at (axis cs:1.3,1) [penColor2] {$(a,f(a))$}; 
            \addplot[color=penColor3,fill=penColor3,only marks,mark=*] coordinates{(1,1)};  %% closed hole         
        \end{axis}
\end{tikzpicture}
%\caption{A linear approximation of $f(x) = \sin(x)$ at $x=0$.}
%\label{figure:la sin}
%\end{marginfigure}
\end{image}




First, find the expression for $m$, the slope of the tangent line to the curve $y=f(x)$ at the point $(a,f(a))$.
 Select the correct choice.
 \begin{multipleChoice}

 \choice{$m= f(a)$}
  \choice[correct] {$m=f'(a)$}
   \choice{$m= \frac{f(a)-f(0)}{a-0}$}
     \choice{$m= \frac{f(a+h)-f(a)}{h}$}
 \choice{ We don't have enough information to determine the slope.}
  \end{multipleChoice}
Since we know that the point $(a,f(a))$  lies on the tangent line,  we can write an equation of the tangent line. 

\[
y= f'(a)(x-a) +f(a).
\]
Now, we define a function, $L$,  by $L(x)= f'(a)(x-a) +f(a)$. This function is linear and its graph is the line tangent to the curve $y=f(x)$ at the point where $x=a$.
This function deserves a special name.
\end{example}



\begin{definition}\index{linear approximation}
If $f$ is a function differentiable at $x=a$, then a \textbf{linear
  approximation} to the function $f$ at $x=a$ is given by
\[
L(x) = f'(a)(x-a) +f(a).
\]
\end{definition}


Note that the graph of $L$ is just the tangent line to the graph of $f$ at $x=a$.

A linear approximation of $f$ is a ``good'' approximation as long as
$x$ is ``not too far'' from $a$.
%As we see from Figure~\ref{figure:informal-tangent}, 
If one ``zooms in'' on the graph of $f$ sufficiently, then the graphs of $f$ and $L$ 
 are nearly indistinguishable.
 
  As a first example, we
will see how linear approximations allow us to approximate
``difficult'' computations.

\begin{example}
Let $f$ be a function defined by 
\[
f(x) =\sqrt[3]{x}.
\]
Approximate $\sqrt[3]{50}$, using $L$,  a linear approximation to the function $f$ at $a=64$.

\begin{explanation}
To start, write
\[
\frac{d}{dx} f(x) = \frac{d}{Dx} x^{1/3} = \frac{1}{3x^{\answer[given]{2/3}}}.
\]
\begin{align*}
L(x) &= \answer[given]{4}+ \frac{1}{3\cdot 64^{2/3}} (x-64)  \\
&=4+ \frac{1}{\answer[given]{48}} (x-64) \\
&= \frac{x}{48} +\frac{8}{3}.
\end{align*}
\begin{image}
%\begin{marginfigure}
\begin{tikzpicture}
  \begin{axis}[
            xmin=0,xmax=100,ymin=0,ymax=5,
            axis lines=center,
             xtick={0, 20, 40, 50, 64, 80, 90, 100},
        xticklabels={0, 20, 40, 50, 64, 80, 90, 100},
            xlabel=$x$, ylabel=$y$,
            every axis y label/.style={at=(current axis.above origin),anchor=south},
            every axis x label/.style={at=(current axis.right of origin),anchor=west},
          ]        
          \addplot [very thick, penColor, samples=150,smooth,domain=(0:100)] {x^(1/3))};
          \addplot [very thick, penColor2, domain=(0:100)] {x/48+8/3};
          \addplot [textColor,dashed] plot coordinates {(64,0) (64,4)};
          \addplot [textColor,dashed] plot coordinates {(0,4) (64,4)};
          \addplot [textColor,dashed] plot coordinates {(50,0) (50,3.68)};
          \node at (axis cs:20,2.3) [penColor] {$f$};
          \node at (axis cs:20,3.3) [penColor2] {$L$};
          \addplot[color=penColor3,fill=penColor3,only marks,mark=*] coordinates{(64,4)};  %% closed hole     
           \addplot[color=penColor,fill=penColor,only marks,mark=*] coordinates{(50,0)};  %% closed hole  
                  
        \end{axis}
\end{tikzpicture}
%\caption{A linear approximation of $f(x) = \sqrt[3]{x}$ at $x=64$.}
%\label{figure:la sqrt3x}
%\end{marginfigure}
\end{image}
Now we evaluate $L(50) \approx 3.71$ and compare it to
$\sqrt[3]{50}\approx 3.68$.  From this we see that the linear
approximation, while perhaps inexact, is computationally \textbf{easier}
than computing the cube root.
\end{explanation}

What would happen if we chose $a=27$ instead?

Then we would use $L_{27}$, the linear approximation to the function $f$ at $a=27$.
In that case, $L_{27}=f(27)+f'(27)(x-27)=3+\frac{1}{27}(x-27)$.
The graph of  $L_{27}$, together with the graphs of $f$ and $L=L_{64}$ is given in the figure below.
\begin{image}
%\begin{marginfigure}
\begin{tikzpicture}
  \begin{axis}[
            xmin=1,xmax=100,ymin=0,ymax=5,
            axis lines=center,
              xtick={0, 27, 40, 50, 64, 80, 90, 100},
        xticklabels={0, 27, 40, 50, 64, 80, 90, 100},
            xlabel=$x$, ylabel=$y$,
            every axis y label/.style={at=(current axis.above origin),anchor=south},
            every axis x label/.style={at=(current axis.right of origin),anchor=west},
          ]        
          \addplot [very thick, penColor, samples=150,smooth,domain=(0:100)] {x^(1/3))};
          \addplot [very thick, penColor2, domain=(0:100)] {x/48+8/3};
            \addplot [very thick, penColor4, domain=(0:100)] {(x-27)/27+3};
              \addplot [textColor,dashed] plot coordinates {(50,0) (50,3.68)};
          \node at (axis cs:6,1.3) [penColor] {$f$};
          \node at (axis cs:20,3.3) [penColor2] {$L_{64}$};
           \node at (axis cs:6,2.5) [penColor4] {$L_{27}$};
          \addplot[color=penColor3,fill=penColor3,only marks,mark=*] coordinates{(64,4)};  %% closed hole         
            \addplot[color=penColor3,fill=penColor4,only marks,mark=*] coordinates{(27,3)};  %% closed hole    
             \addplot[color=penColor3,fill=penColor,only marks,mark=*] coordinates{(50,0)};  %% closed hole     
        \end{axis}
\end{tikzpicture}
%\caption{A linear approximation of $f(x) = \sqrt[3]{x}$ at $x=64$.}
%\label{figure:la sqrt3x}
%\end{marginfigure}
\end{image}
From the picture we can see that 

 $L_{27}(50)>L_{64}(50)>f(50)$.
 
 So, our choice, $a=64$, was better!
\end{example}
With modern calculators and computing software, it may not appear
necessary to use linear approximations. In fact they are quite
useful. In cases requiring an explicit numerical approximation, they
allow us to get a quick rough estimate which can be used as a
``reality check'' on a more complex calculation. In some complex
calculations involving functions, the linear approximation makes an
otherwise intractable calculation possible, without serious loss of
accuracy.



\begin{example}%\label{exam:linear approximation of sine}
Use a linear approximation of $f(x) =\sin(x)$ at $a=0$ to approximate
$\sin(0.3)$.
\begin{explanation}
To start, write
\[
\frac{d}{dx} f(x) = \answer[given]{\cos(x)},
\]
so our linear approximation is
\begin{align*}
L(x) &= 0+\answer[given]{\cos(0)}\cdot(x-0)\\
&= x.
\end{align*}
\begin{image}
%\begin{marginfigure}
\begin{tikzpicture}
  \begin{axis}[
            xmin=-1.6,xmax=1.6,ymin=-1.5,ymax=1.5,
            axis lines=center,
            xtick={-1.57, 0,0.3, 1.57},
            xticklabels={$-\pi/2$, $0$, 0.3, $\pi/2$},
            ytick={-1,1},
            %ticks=none,
            %width=3in,
            %height=2in,
            unit vector ratio*=1 1 1,
            xlabel=$x$, ylabel=$y$,
            every axis y label/.style={at=(current axis.above origin),anchor=south},
            every axis x label/.style={at=(current axis.right of origin),anchor=west},
          ]        
          \addplot [very thick, penColor, samples=100,smooth, domain=(-1.6:1.6)] {sin(deg(x))};
          \addplot [very thick, penColor2, samples=100,smooth] {x};
          \addplot [textColor,dashed] plot coordinates {(0.3,0) (0.3,0.295)};
          \node at (axis cs:1,.7) [penColor] {$f$};
          \node at (axis cs:1,1.18) [penColor2] {$L$};
          \addplot[color=penColor3,fill=penColor3,only marks,mark=*] coordinates{(0,0)};  %% closed hole   
            \addplot[color=penColor3,fill=penColor,only marks,mark=*] coordinates{(0.3,0)};  %% closed hole           
        \end{axis}
\end{tikzpicture}
%\caption{A linear approximation of $f(x) = \sin(x)$ at $x=0$.}
%\label{figure:la sin}
%\end{marginfigure}
\end{image}
Hence, a linear approximation for $\sin(x)$ at $a=0$ is $L(x) = x$,
and so $L(0.3) = 0.3$.  Comparing this to $\sin(.3) \approx 0.295$,
we see that the approximation is quite good. For this reason, it is common
to approximate $\sin(x)$ with its linear approximation $L(x) = x$
when $x$ is near zero.  
%see Figure~\ref{figure:la sin}.
\end{explanation}
\end{example}



\section{Differentials}

The  graph of a function $f$ and the graph of $L$, the linear approximation of $f$ at $a$, are shown in the figure below.
Also, two quantities, $dx$ and $df$, and a point $P$ are marked in the figure. Look carefully at the figure when answering the questions below.


\begin{image}
%\begin{marginfigure}[0in]
\begin{tikzpicture}
  \begin{axis}[
            xmin=1, xmax=2, range=0:6,ymax=6,ymin=0,
            axis lines =left, xlabel=$x$, ylabel=$y$,
            every axis y label/.style={at=(current axis.above origin),anchor=south},
            every axis x label/.style={at=(current axis.right of origin),anchor=west},
            ticks=none,
            axis on top,
          ]         
          \addplot [draw=black,dashed,->,>=stealth'] plot coordinates {(1.4,10/6) (1.7,10/6)};
    \addplot [draw=black,dashed,->,>=stealth'] plot coordinates {(1.7,10/6) (1.7,10/6 +.3/.36)};
          \addplot [very thick,penColor, smooth,samples=100,domain=(0:1.833)] {-1/(x-2)};
            \addplot [very thick,penColor2, smooth,samples=100,domain=(0:1.833)] {1/0.6+(1/0.36)*(x-1.4)};
  \addplot[color=penColor,fill=penColor,only marks,mark=*] coordinates{(1.7,1/0.3)};  %% closed hole  
   \addplot[color=penColor3,fill=penColor3,only marks,mark=*] coordinates{(1.7,10/6 +.3/.36)};  %% closed hole  
          \addplot[color=penColor3,fill=penColor3,only marks,mark=*] coordinates{(1.4,10/6)};  %% closed hole            
          \node at (axis cs:1.55,1.67) [below] {$dx$};
          \node at (axis cs:1.7,2.08) [right] {$df$};
            \node at (axis cs:1.15,1.85) [below] {$y=f(x)$};
          \node at (axis cs:1.05,0.55) [right] {$y=L(x)$};
           \node at (axis cs:1.24,2.0) [right] {$(a,f(a))$};
            \node at (axis cs:1.478,3.315) [right] {$(x,f(x))$};
              \node at (axis cs:1.68,2.8) [right] {$P$};
        \end{axis}
\end{tikzpicture}
%\caption{While $dy$ and $\d x$ are both variables, $dy$ depends on $\d x$,
  %and approximates how much a function grows after a change of size
  %$\d x$ from a given point.}
%\label{figure:differentials}
%\end{marginfigure}
\end{image}



\begin{question}
  Select all the correct expressions for the quantity $dx$.
   \begin{hint}
     You can see that $x=a+dx$.
    \end{hint}
    \begin{selectAll}
      \choice{$dx=f(x)-f(a)$}
            \choice{$dx=f(x)-L(x)$}
      \choice[correct]{$dx=x-a$}
      \choice{$dx=L(x)-f(a)$}
          \choice{$dx=L(x)-L(a)$}
    \end{selectAll}
   
  \end{question}
  \begin{question}
  Select all the correct expressions for the quantity $df$.
   \begin{hint}
     You can see that $L(x)=f(a)+df$.
    \end{hint}
      \begin{hint}
    Recall: $L(a)=f(a)$.
    \end{hint}
    \begin{selectAll}
       \choice{$df=f(x)-f(a)$}
            \choice{$df=f(x)-L(x)$}
      \choice{$df=x-a$}
      \choice[correct]{$df=L(x)-f(a)$}
          \choice[correct]{$df=L(x)-L(a)$}
         
    \end{selectAll}
   
  \end{question}

 \begin{question}
Based on  the figure and the expression for $L(x)$, select all the correct expressions for $df$.
 \begin{hint}
    Recall: $df=L(x)-f(a)=f(a)+f'(a)(x-a)-f(a)=f'(a)(x-a)=f'(a)dx$.
    \end{hint}
      \begin{selectAll}
      \choice[correct]{ $df=f'(a)(x-a)$}
      \choice[correct]{ $df=f'(a)dx$}
  \choice{$df=f'(x-a)$}
    \choice{$df=f'(x)dx$}
    \choice{$df=f'(x)(x-a)$}
      \end{selectAll}
  \end{question}


 So, we can write $df=f'(a)dx$ and call it a \textbf{differential} of $f$ at $a$. Notice that we can define a differential at any point $x$ of the domain of $f$, provided that $f'(x)$ exists.
We will  do that in our next definition.
\begin{definition}
Let $f$ be a differentiable function, let $x$ be a point in the domain of $f$, and let  $dx$ be some quantity, called a \textbf{differential of $x$}.
We define  $df$, a differential of $f$,  at a point $x$ by 
\[
df=f'(x)\cdot dx
\] 

 Geometrically, differentials can be interpreted via the diagram below.
\begin{image}
%\begin{marginfigure}[0in]
\begin{tikzpicture}
  \begin{axis}[
            xmin=1, xmax=2, range=0:6,ymax=6,ymin=0,
            axis lines =left, xlabel=$x$, ylabel=$y$,
            every axis y label/.style={at=(current axis.above origin),anchor=south},
            every axis x label/.style={at=(current axis.right of origin),anchor=west},
            ticks=none,
            axis on top,
          ]         
          \addplot [draw=black,dashed,->,>=stealth'] plot coordinates {(1.4,10/6) (1.7,10/6)};
    \addplot [draw=black,dashed,->,>=stealth'] plot coordinates {(1.7,10/6) (1.7,10/6 +.3/.36)};
          \addplot [very thick,penColor, smooth,samples=100,domain=(0:1.833)] {-1/(x-2)};
            \addplot [very thick,penColor2, smooth,samples=100,domain=(0:1.833)] {1/0.6+(1/0.36)*(x-1.4)};
  \addplot[color=penColor,fill=penColor,only marks,mark=*] coordinates{(1.7,1/0.3)};  %% closed hole  
   \addplot[color=penColor3,fill=penColor3,only marks,mark=*] coordinates{(1.7,10/6 +.3/.36)};  %% closed hole  
          \addplot[color=penColor3,fill=penColor3,only marks,mark=*] coordinates{(1.4,10/6)};  %% closed hole            
          \node at (axis cs:1.55,1.67) [below] {$dx$};
          \node at (axis cs:1.7,2.08) [right] {$df$};
            \node at (axis cs:1.15,1.85) [below] {$f$};
          \node at (axis cs:1.05,0.55) [right] {$L$};
           \node at (axis cs:1.24,2.0) [right] {$(x,f(x))$};
            \node at (axis cs:1.26,3.6) [right] {$(x+dx,f(x+dx))$};
          
        \end{axis}
\end{tikzpicture}
%\caption{While $dy$ and $\d x$ are both variables, $dy$ depends on $\d x$,
  %and approximates how much a function grows after a change of size
  %$\d x$ from a given point.}
%\label{figure:differentials}
%\end{marginfigure}
\end{image}

Note, it is now the case (by definition!) that 
\[
 f'(x)=\frac{df}{dx}.
\]
\end{definition}
We should not be surprised, since the slope of the tangent line in the figure is $f'(x)$,  and this slope is also given by $\frac{df}{dx}$.
















\end{document}
