\documentclass{ximera}

\graphicspath{
  {./}
  {ximeraTutorial/}
  {basicPhilosophy/}
}

\newcommand{\mooculus}{\textsf{\textbf{MOOC}\textnormal{\textsf{ULUS}}}}

\usepackage{tkz-euclide}\usepackage{tikz}
\usepackage{tikz-cd}
\usetikzlibrary{arrows}
\tikzset{>=stealth,commutative diagrams/.cd,
  arrow style=tikz,diagrams={>=stealth}} %% cool arrow head
\tikzset{shorten <>/.style={ shorten >=#1, shorten <=#1 } } %% allows shorter vectors

\usetikzlibrary{backgrounds} %% for boxes around graphs
\usetikzlibrary{shapes,positioning}  %% Clouds and stars
\usetikzlibrary{matrix} %% for matrix
\usepgfplotslibrary{polar} %% for polar plots
\usepgfplotslibrary{fillbetween} %% to shade area between curves in TikZ
\usetkzobj{all}
\usepackage[makeroom]{cancel} %% for strike outs
%\usepackage{mathtools} %% for pretty underbrace % Breaks Ximera
%\usepackage{multicol}
\usepackage{pgffor} %% required for integral for loops



%% http://tex.stackexchange.com/questions/66490/drawing-a-tikz-arc-specifying-the-center
%% Draws beach ball
\tikzset{pics/carc/.style args={#1:#2:#3}{code={\draw[pic actions] (#1:#3) arc(#1:#2:#3);}}}



\usepackage{array}
\setlength{\extrarowheight}{+.1cm}
\newdimen\digitwidth
\settowidth\digitwidth{9}
\def\divrule#1#2{
\noalign{\moveright#1\digitwidth
\vbox{\hrule width#2\digitwidth}}}






\DeclareMathOperator{\arccot}{arccot}
\DeclareMathOperator{\arcsec}{arcsec}
\DeclareMathOperator{\arccsc}{arccsc}

















%%This is to help with formatting on future title pages.
\newenvironment{sectionOutcomes}{}{}

\author{Steven Gubkin}
\license{Creative Commons 3.0 By-NC}

\outcome{Define linear approximation as an application of the tangent to a curve.}
\outcome{Find the linear approximation to a function at a point and use it to approximate the function value.}
\outcome{Identify when a linear approximation can be used.}

\begin{document}
\begin{exercise}
In Einstein's theory of relativity, we can derive that

$$E(v) = \frac{mc^2}{\sqrt{1-\frac{v^2}{c^2}}}$$

where $E(v)$ is the energy of an object with ``Rest mass'' $m$ and velocity $v$.

Let us analyze this more closely.

First find the linear approximation to the function $f(u) = \frac{mc^2}{\sqrt{1-u}}$ at $u=0$.

Using this approximation, and substituting $u = \frac{v^2}{c^2}$, we can obtain an approximation for $E(v)$ which is valid for small velocities $v$.

The approximation you obtain should have two terms.  One of which is the famous $E = mc^2$ (representing the resting energy) and the other should be the classical kinetic energy of the object.

\begin{prompt}
	The local linearization of $\frac{m c^2}{\sqrt{1-u}}$ at $u=0$ is $\answer{m c^2 +\frac{m c^2}{2}u}$.

	So we get that $E(v) \approx \answer{mc^2+\frac{1}{2}mv^2}$.
\end{prompt}

\end{exercise}
\end{document}
