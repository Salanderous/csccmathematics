\documentclass{ximera}


\graphicspath{
  {./}
  {ximeraTutorial/}
  {basicPhilosophy/}
}

\newcommand{\mooculus}{\textsf{\textbf{MOOC}\textnormal{\textsf{ULUS}}}}

\usepackage{tkz-euclide}\usepackage{tikz}
\usepackage{tikz-cd}
\usetikzlibrary{arrows}
\tikzset{>=stealth,commutative diagrams/.cd,
  arrow style=tikz,diagrams={>=stealth}} %% cool arrow head
\tikzset{shorten <>/.style={ shorten >=#1, shorten <=#1 } } %% allows shorter vectors

\usetikzlibrary{backgrounds} %% for boxes around graphs
\usetikzlibrary{shapes,positioning}  %% Clouds and stars
\usetikzlibrary{matrix} %% for matrix
\usepgfplotslibrary{polar} %% for polar plots
\usepgfplotslibrary{fillbetween} %% to shade area between curves in TikZ
\usetkzobj{all}
\usepackage[makeroom]{cancel} %% for strike outs
%\usepackage{mathtools} %% for pretty underbrace % Breaks Ximera
%\usepackage{multicol}
\usepackage{pgffor} %% required for integral for loops



%% http://tex.stackexchange.com/questions/66490/drawing-a-tikz-arc-specifying-the-center
%% Draws beach ball
\tikzset{pics/carc/.style args={#1:#2:#3}{code={\draw[pic actions] (#1:#3) arc(#1:#2:#3);}}}



\usepackage{array}
\setlength{\extrarowheight}{+.1cm}
\newdimen\digitwidth
\settowidth\digitwidth{9}
\def\divrule#1#2{
\noalign{\moveright#1\digitwidth
\vbox{\hrule width#2\digitwidth}}}






\DeclareMathOperator{\arccot}{arccot}
\DeclareMathOperator{\arcsec}{arcsec}
\DeclareMathOperator{\arccsc}{arccsc}

















%%This is to help with formatting on future title pages.
\newenvironment{sectionOutcomes}{}{}


\outcome{Define linear approximation as an application of the tangent to a curve.}
\outcome{Find the linear approximation to a function at a point and use it to approximate the function value.}
%\outcome{Identify when a linear approximation can be used.}
\outcome{Label a graph with the appropriate quantities used in linear approximation.}
\outcome{Find the error of a linear approximation.}
\outcome{Compute differentials.}
\outcome{Use the second derivative to discuss whether the linear approximation over or underestimates the actual function value.}
%\outcome{Contrast the notation and meaning of \d{y} versus \Delta y.}
%\outcome{Understand that the error shrinks faster than the displacement in the input.}
%\outcome{Justify the chain rule via the composition of linear approximations.}

\author{Nela Lakos \and Kyle Parsons}

\begin{document}
\begin{exercise}

Consider the function $f(x) = \frac{4}{x}$.  The graph below is an attempt to depict $f$ and the linear approximation to $f$ at $a=2$. Is the graph correct?
\begin{multipleChoice}
\choice{Yes}
\choice[correct]{No}
\end{multipleChoice}

\begin{image}
  \begin{tikzpicture}
    \begin{axis}[
        xmin=-0.3,xmax=4.3,ymin=-0.3,ymax=4.3,
        clip=true,
        unit vector ratio*=1 1 1,
        axis lines=center,
        grid = major,
        ytick={-6,-5,...,4},
 	    xtick={-6,-5,...,6},
        xlabel=$x$, ylabel=$y$,
        every axis y label/.style={at=(current axis.above origin),anchor=south},
        every axis x label/.style={at=(current axis.right of origin),anchor=west},
      ]
      \addplot[very thick,penColor,domain=0.1:4.3] plot{4/x};
      \addplot[very thick,red,domain=-0.3:4.3] plot{-(x-2)/2+2};
      \draw[very thick,dashed] (axis cs:2,2) -- (axis cs:2,0);
      
      \node at (axis cs:1.6,3.7) [black] {$y=\frac{4}{x}$};
      \node at (axis cs:0.7,2.2) [black] {$y=L(x)$};
      \end{axis}`
  \end{tikzpicture}
\end{image}

The linear approximation to $f$ at $a=2$ is
\[
L(x) = \answer{4-x}.
\]

Using that linear approximation, we can estimate the value of $\frac{4}{2.8}$ to be about
\[
\frac{4}{2.8} \approx \answer{4-2.8}.
\]

This approximation is an \wordChoice{\choice{overestimate}\choice[correct]{underestimate}} because $f(x)$ is concave \wordChoice{\choice[correct]{up}\choice{down}} and the graph of $L$ lies \wordChoice{\choice{above}\choice[correct]{below}} the graph of $f$ near $a=2$.

The percent error,PE, in an approximation is given by
\[
100\frac{|approximation - exact|}{|exact|},
\]
where the exact value is given by a calculator.  In this case our percent error PE (to the nearest percentage point) in approximating $\frac{4}{2.8}$ is
\[
PE= \answer{16}\%.
\]

When $x$ changes from $a=2$ to $a+\Delta x = 2.8$, the change in $y$ is 
\[
\Delta y = f\left(\answer{2.8}\right) - f\left(\answer{2}\right) = -\frac{4}{\answer{7}}.
\]
Now in this case the \textbf{approximate} change in $y$ is
\[
dy = f'\left(\answer{2}\right) dx = \answer{-0.8}.
\]

The picture below shows the graph of $f$ along with the linear approximation to $f$ at $a=2$.  On this diagram are quantities labeled $A$, $B$, $C$, $D$, and $E$.  Correctly identify them below.

\begin{image}
  \begin{tikzpicture}
    \begin{axis}[
        xmin=-0.3,xmax=4.3,ymin=-.9,ymax=4.3,
        clip=true,
        unit vector ratio*=1 1 1,
        axis lines=center,
        grid = major,
        ytick={-6,-5,...,4},
 	    xtick={-6,-5,...,6},
        xlabel=$x$, ylabel=$y$,
        every axis y label/.style={at=(current axis.above origin),anchor=south},
        every axis x label/.style={at=(current axis.right of origin),anchor=west},
      ]
      \draw[very thick,dashed] (axis cs:2,2) -- (axis cs:2,0);
      \draw[very thick,dashed] (axis cs:2.8,2) -- (axis cs:2.8,0);
      \draw[very thick,dashed] (axis cs:2,1.2) -- (axis cs:2.8,1.2);
      \draw[very thick,dashed] (axis cs:2,10/7) -- (axis cs:2.8,10/7);
      \draw[very thick,dashed] (axis cs:2,2) -- (axis cs:2.8,2);

	  \draw[decorate,decoration={brace,amplitude=3pt,mirror},xshift=-3pt] (axis cs:2,2) -- (axis cs:2,10/7) node[midway,xshift=-10pt]{$A$};
	  \draw[decorate,decoration={brace,amplitude=3pt},xshift=3pt] (axis cs:2.8,2) -- (axis cs:2.8,1.2) node[midway,xshift=10pt]{$B$};
	  \draw[decorate,decoration={brace,amplitude=3pt},yshift=3pt] (axis cs:2,2) -- (axis cs:2.8,2) node[midway,yshift=10pt]{$C$};
	  \node at (axis cs:2,0) [below left,yshift=-2pt] {$D$};
	  \node at (axis cs:2.8,0) [below left,yshift=-2pt] {$E$};
	  \draw (axis cs:2.8,0) -- (axis cs: 2.7,-0.12);
	  \draw (axis cs:2,0) -- (axis cs:1.88,-0.12);
	        
      \addplot[very thick,penColor,domain=0.1:4.3] plot{4/x};
      \addplot[very thick,red,domain=-0.3:4.3] plot{-(x-2)+2};
      
      \node at (axis cs:1.6,3.7) [black] {$y=\frac{4}{x}$};
      \node at (axis cs:0.7,2.2) [black] {$y=L(x)$};
      \end{axis}`
  \end{tikzpicture}
\end{image}

Chose the correct expression for $A$
\begin{multipleChoice}
\choice{$dx$}
\choice{$dy$}
\choice[correct]{$\Delta y$}
\choice{$a$}
\choice{$a+\Delta x$}
\end{multipleChoice}

Chose the correct expression for $B$
\begin{multipleChoice}
\choice{$dx$}
\choice[correct]{$dy$}
\choice{$\Delta y$}
\choice{$a$}
\choice{$a+\Delta x$}
\end{multipleChoice}

Chose the correct expression for $C$
\begin{multipleChoice}
\choice[correct]{$dx$}
\choice{$dy$}
\choice{$\Delta y$}
\choice{$a$}
\choice{$a+\Delta x$}
\end{multipleChoice}

Chose the correct expression for $D$
\begin{multipleChoice}
\choice{$dx$}
\choice{$dy$}
\choice{$\Delta y$}
\choice[correct]{$a$}
\choice{$a+\Delta x$}
\end{multipleChoice}

Chose the correct expression for $E$
\begin{multipleChoice}
\choice{$dx$}
\choice{$dy$}
\choice{$\Delta y$}
\choice{$a$}
\choice[correct]{$a+\Delta x$}
\end{multipleChoice}

\end{exercise}
\end{document}