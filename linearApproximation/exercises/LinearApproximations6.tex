\documentclass{ximera}


\graphicspath{
  {./}
  {ximeraTutorial/}
  {basicPhilosophy/}
}

\newcommand{\mooculus}{\textsf{\textbf{MOOC}\textnormal{\textsf{ULUS}}}}

\usepackage{tkz-euclide}\usepackage{tikz}
\usepackage{tikz-cd}
\usetikzlibrary{arrows}
\tikzset{>=stealth,commutative diagrams/.cd,
  arrow style=tikz,diagrams={>=stealth}} %% cool arrow head
\tikzset{shorten <>/.style={ shorten >=#1, shorten <=#1 } } %% allows shorter vectors

\usetikzlibrary{backgrounds} %% for boxes around graphs
\usetikzlibrary{shapes,positioning}  %% Clouds and stars
\usetikzlibrary{matrix} %% for matrix
\usepgfplotslibrary{polar} %% for polar plots
\usepgfplotslibrary{fillbetween} %% to shade area between curves in TikZ
\usetkzobj{all}
\usepackage[makeroom]{cancel} %% for strike outs
%\usepackage{mathtools} %% for pretty underbrace % Breaks Ximera
%\usepackage{multicol}
\usepackage{pgffor} %% required for integral for loops



%% http://tex.stackexchange.com/questions/66490/drawing-a-tikz-arc-specifying-the-center
%% Draws beach ball
\tikzset{pics/carc/.style args={#1:#2:#3}{code={\draw[pic actions] (#1:#3) arc(#1:#2:#3);}}}



\usepackage{array}
\setlength{\extrarowheight}{+.1cm}
\newdimen\digitwidth
\settowidth\digitwidth{9}
\def\divrule#1#2{
\noalign{\moveright#1\digitwidth
\vbox{\hrule width#2\digitwidth}}}






\DeclareMathOperator{\arccot}{arccot}
\DeclareMathOperator{\arcsec}{arcsec}
\DeclareMathOperator{\arccsc}{arccsc}

















%%This is to help with formatting on future title pages.
\newenvironment{sectionOutcomes}{}{}


\outcome{Define linear approximation as an application of the tangent to a curve.}
\outcome{Find the linear approximation to a function at a point and use it to approximate the function value.}
%\outcome{Identify when a linear approximation can be used.}
%\outcome{Label a graph with the appropriate quantities used in linear approximation.}
\outcome{Find the error of a linear approximation.}
\outcome{Compute differentials.}
%\outcome{Use the second derivative to discuss whether the linear approximation over or underestimates the actual function value.}
%\outcome{Contrast the notation and meaning of \d{y} versus \Delta y.}
%\outcome{Understand that the error shrinks faster than the displacement in the input.}
%\outcome{Justify the chain rule via the composition of linear approximations.}

\author{Nela Lakos \and Kyle Parsons}

\begin{document}
\begin{exercise}

Consider the function $f(x) = \frac{(x+2)^2}{12}$.

In the figure below the graph of $f$, the secant line between $(0,f(0))$ and $(4,f(4))$, and the linear approximation to $f$ at $a=4$ are shown.

\begin{image}
  \begin{tikzpicture}
    \begin{axis}[
        xmin=-.3,xmax=6.3,ymin=-.3,ymax=6.3,
        clip=true,
        unit vector ratio*=1 1 1,
        axis lines=center,
        grid = major,
        ytick={-6,-5,...,6},
    xtick={-6,-5,...,6},
        xlabel=$x$, ylabel=$y$,
        every axis y label/.style={at=(current axis.above origin),anchor=south},
        every axis x label/.style={at=(current axis.right of origin),anchor=west},
      ]
      \addplot[very thick,penColor,domain=-0.3:6.3] plot{(x+2)^2/12};
      \addplot[thick,red,domain=-0.3:6.3] plot{(x-4)+3};
      \addplot[thick,black,domain=-0.3:6.3] plot{2/3*(x-4)+3};
            
      \node at (axis cs: 4.5,5.5) {$y=f(x)$};
      \end{axis}`
  \end{tikzpicture}
\end{image}

The slope of the secant line depicted is
\[
m_{sec} = \answer{\frac{2}{3}}.
\]

The linear approximation $L$ to $f$ at $a=4$ is
\[
L(x) = \answer{(x-4) + 3}.
\]

Using this linear approximation, we can estimate $f$ at $x=4.01$ as
\[
f(4.01) \approx \answer{3.01}.
\]
This estimate is an \wordChoice{\choice[correct]{underestimate}\choice{overestimate}} because the second derivative of $f$ is \wordChoice{\choice{negative}\choice[correct]{positive}} and the graph of $L$ lies \wordChoice{\choice{above}\choice[correct]{below}} the graph of $f$.

When $x$ changes from $a=4$ to $a+\Delta x=4.01$, $f(x)$ changes by $\Delta y$ (rounded to the thousands place) given by
\[
\Delta y = f\left(\answer{4.01}\right) - f\left(\answer{4}\right) = \answer{0.01}.
\]
We can also estimate this change in $f(x)$ as $dy$ given by
\[
dy = f'\left(\answer{4}\right) dx = \answer{0.01}.
\]
We see that our approximation of the change in $f(x)$ is very close to the true change.

\end{exercise}
\end{document}