\documentclass{ximera}


\graphicspath{
  {./}
  {ximeraTutorial/}
  {basicPhilosophy/}
}

\newcommand{\mooculus}{\textsf{\textbf{MOOC}\textnormal{\textsf{ULUS}}}}

\usepackage{tkz-euclide}\usepackage{tikz}
\usepackage{tikz-cd}
\usetikzlibrary{arrows}
\tikzset{>=stealth,commutative diagrams/.cd,
  arrow style=tikz,diagrams={>=stealth}} %% cool arrow head
\tikzset{shorten <>/.style={ shorten >=#1, shorten <=#1 } } %% allows shorter vectors

\usetikzlibrary{backgrounds} %% for boxes around graphs
\usetikzlibrary{shapes,positioning}  %% Clouds and stars
\usetikzlibrary{matrix} %% for matrix
\usepgfplotslibrary{polar} %% for polar plots
\usepgfplotslibrary{fillbetween} %% to shade area between curves in TikZ
\usetkzobj{all}
\usepackage[makeroom]{cancel} %% for strike outs
%\usepackage{mathtools} %% for pretty underbrace % Breaks Ximera
%\usepackage{multicol}
\usepackage{pgffor} %% required for integral for loops



%% http://tex.stackexchange.com/questions/66490/drawing-a-tikz-arc-specifying-the-center
%% Draws beach ball
\tikzset{pics/carc/.style args={#1:#2:#3}{code={\draw[pic actions] (#1:#3) arc(#1:#2:#3);}}}



\usepackage{array}
\setlength{\extrarowheight}{+.1cm}
\newdimen\digitwidth
\settowidth\digitwidth{9}
\def\divrule#1#2{
\noalign{\moveright#1\digitwidth
\vbox{\hrule width#2\digitwidth}}}






\DeclareMathOperator{\arccot}{arccot}
\DeclareMathOperator{\arcsec}{arcsec}
\DeclareMathOperator{\arccsc}{arccsc}

















%%This is to help with formatting on future title pages.
\newenvironment{sectionOutcomes}{}{}


\author{Jason Miller and Jim Talamo}
\license{Creative Commons 3.0 By-NC}


\outcome{Evaluate integrals using integration by parts, including mulitple iterations}


\begin{document}
\begin{exercise}
Determine the definite integral 

\[
\int_0^{\pi} x\sin(x) \cos(x) dx = \answer{-\frac{1}{4} \pi}
\]

\begin{hint}
Since it is generally nice to let $u$ be the polynomial when we have one, let $u = x$ and $\d v = \sin(x) \cos(x)$.  We need to integrate $\sin(x) \cos(x)$, and this can be done at least two different ways.

\begin{itemize}
\item One way is to let $w=\sin(x)$.  Then, $dw = \answer{\cos(x)} dx$ and $\int \sin(x) \cos(x) \d x = \int \answer{w} dw$.  Working through this substitution gives

\[
\int \sin(x) \cos(x) dx = \frac{1}{2}\sin^2(x)+C.
\]
 
(Note: a similar substitution with $w=\cos(x)$ also will work.)

Using integration by parts, the \emph{indefinite} integral $\int x\sin(x) \cos(x) dx$ can be found.

\[
\int x\sin(x) \cos(x) dx =\frac{1}{2}x\sin^2(x) - \int \frac{1}{2} \sin^2(x) dx. 
\]

To compute the integral on the righthand side, the trig identity

\[
\sin^2(x) = \frac{1}{2} - \frac{1}{2} \cos(2x)
\] 

will be helpful.  Once you find the indefinite integral, you may use Fundamental Theorem of Calculus to find $\int_0^{\pi} x\sin(x) \cos(x) dx $.

 \item Another way is to use the identity $\sin(2x) = 2 \sin(x)\cos(x)$ to write $\int  \sin(x)\cos(x) dx = \int \frac{1}{2} \sin(2x) dx = \answer{-\frac{1}{4}\cos(2x)}$.

Using integration by parts, the \emph{indefinite} integral $\int x\sin(x) \cos(x) dx$ can be found.

\[
\int x\sin(x) \cos(x) dx =-\frac{1}{4}x\cos(2x) + \int \frac{1}{4} \cos(2x) dx. 
\]

Once you find the indefinite integral, you may use Fundamental Theorem of Calculus to find $\int_0^{\pi} x\sin(x) \cos(x) dx $.

\end{itemize}
 
\end{hint}


\end{exercise}
\end{document}
