\documentclass{ximera}


\graphicspath{
  {./}
  {ximeraTutorial/}
  {basicPhilosophy/}
}

\newcommand{\mooculus}{\textsf{\textbf{MOOC}\textnormal{\textsf{ULUS}}}}

\usepackage{tkz-euclide}\usepackage{tikz}
\usepackage{tikz-cd}
\usetikzlibrary{arrows}
\tikzset{>=stealth,commutative diagrams/.cd,
  arrow style=tikz,diagrams={>=stealth}} %% cool arrow head
\tikzset{shorten <>/.style={ shorten >=#1, shorten <=#1 } } %% allows shorter vectors

\usetikzlibrary{backgrounds} %% for boxes around graphs
\usetikzlibrary{shapes,positioning}  %% Clouds and stars
\usetikzlibrary{matrix} %% for matrix
\usepgfplotslibrary{polar} %% for polar plots
\usepgfplotslibrary{fillbetween} %% to shade area between curves in TikZ
\usetkzobj{all}
\usepackage[makeroom]{cancel} %% for strike outs
%\usepackage{mathtools} %% for pretty underbrace % Breaks Ximera
%\usepackage{multicol}
\usepackage{pgffor} %% required for integral for loops



%% http://tex.stackexchange.com/questions/66490/drawing-a-tikz-arc-specifying-the-center
%% Draws beach ball
\tikzset{pics/carc/.style args={#1:#2:#3}{code={\draw[pic actions] (#1:#3) arc(#1:#2:#3);}}}



\usepackage{array}
\setlength{\extrarowheight}{+.1cm}
\newdimen\digitwidth
\settowidth\digitwidth{9}
\def\divrule#1#2{
\noalign{\moveright#1\digitwidth
\vbox{\hrule width#2\digitwidth}}}






\DeclareMathOperator{\arccot}{arccot}
\DeclareMathOperator{\arcsec}{arcsec}
\DeclareMathOperator{\arccsc}{arccsc}

















%%This is to help with formatting on future title pages.
\newenvironment{sectionOutcomes}{}{}


\author{Jim Talamo}
\license{Creative Commons 3.0 By-NC}


\outcome{Understand why a constant of integration is not needed when integrating $dv$.}


\begin{document}
\begin{exercise}
This exercise explores why it is not necessary to include a constant when we integrate $dv$. in the context of a specific example.

Consider the indefinite integral:
\[
\int x \sec^2(x) dx 
\]

Suppose that we decide to include a constant of integration in the integration by parts procedure when we integrate $dv$.

Let $u=\answer{x}$ so $du =\answer{1} dx$ and $dv = \sec^2(x) dx$.  If we include a constant of integration, we find $v = \tan(x)+C$.

Thus, using the integration by parts formula:

\[
\int x \sec^2(x) dx = x \left( \tan(x) +C \right) - \int \answer{\tan(x) +C} dx
\]
(Use $C$ for the constant)

\begin{exercise}

Evaluating the integral and simplifying gives:

\[
\int x \sec^2(x) dx = x \tan(x) +Cx - \left(\answer{\ln|\sec(x)| + Cx}\right)
\]
As you can see the terms with $C$ cancel!  It's easy to generalize this argument to ensure that is not necessary to include a constant of integration when integrating $\d v$.

\end{exercise}
\end{exercise}
\end{document}
