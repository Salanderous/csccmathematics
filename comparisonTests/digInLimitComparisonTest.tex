\documentclass{ximera}


\graphicspath{
  {./}
  {ximeraTutorial/}
  {basicPhilosophy/}
}

\newcommand{\mooculus}{\textsf{\textbf{MOOC}\textnormal{\textsf{ULUS}}}}

\usepackage{tkz-euclide}\usepackage{tikz}
\usepackage{tikz-cd}
\usetikzlibrary{arrows}
\tikzset{>=stealth,commutative diagrams/.cd,
  arrow style=tikz,diagrams={>=stealth}} %% cool arrow head
\tikzset{shorten <>/.style={ shorten >=#1, shorten <=#1 } } %% allows shorter vectors

\usetikzlibrary{backgrounds} %% for boxes around graphs
\usetikzlibrary{shapes,positioning}  %% Clouds and stars
\usetikzlibrary{matrix} %% for matrix
\usepgfplotslibrary{polar} %% for polar plots
\usepgfplotslibrary{fillbetween} %% to shade area between curves in TikZ
\usetkzobj{all}
\usepackage[makeroom]{cancel} %% for strike outs
%\usepackage{mathtools} %% for pretty underbrace % Breaks Ximera
%\usepackage{multicol}
\usepackage{pgffor} %% required for integral for loops



%% http://tex.stackexchange.com/questions/66490/drawing-a-tikz-arc-specifying-the-center
%% Draws beach ball
\tikzset{pics/carc/.style args={#1:#2:#3}{code={\draw[pic actions] (#1:#3) arc(#1:#2:#3);}}}



\usepackage{array}
\setlength{\extrarowheight}{+.1cm}
\newdimen\digitwidth
\settowidth\digitwidth{9}
\def\divrule#1#2{
\noalign{\moveright#1\digitwidth
\vbox{\hrule width#2\digitwidth}}}






\DeclareMathOperator{\arccot}{arccot}
\DeclareMathOperator{\arcsec}{arcsec}
\DeclareMathOperator{\arccsc}{arccsc}

















%%This is to help with formatting on future title pages.
\newenvironment{sectionOutcomes}{}{}


\outcome{Use the limit comparison test to determine if a series diverges or converges.}

\title[Dig-In:]{The limit comparison test}

\begin{document}
\begin{abstract}
We compare infinite series to each other using limits.
\end{abstract}
\maketitle

Using the comparison test can be hard, because finding the right
sequence of inequalities is difficult.  The \textit{limit comparison test}
eliminates this part of the method.

\begin{theorem}[The Limit Comparison Test]\index{limit comparison test}
  Let $\sum_{k=0}^\infty a_k$ and $\sum_{k=0}^\infty b_k$ be series with positive
  terms and let
  \[
  \lim_{k \to \infty} \frac{a_k}{b_k} = L.
  \]
  \begin{itemize}
  \item If $0<L<\infty$ then either both series converge, or they both diverge.
  \item If $L=0$ and $\sum b_k$ converges, then $\sum a_k$ converges.
  \item If $L=\infty$ and $\sum b_k$ diverges, then $\sum a_k$ diverges.
  \end{itemize}
\end{theorem}

This theorem should make intuitive sense.
\begin{itemize}
\item If $0<L<\infty$ then we have $a_k \approx L \cdot b_k$ for large
  $k$, so the behavior of the respective series should be the same.
\item If $L=0$ then $a_k$ should be way less than $b_k$.  So if $b_k$
  converges, $a_k$ should also converge by the comparison test.
\item If $L=\infty$, then $a_k$ should be way greater than $b_k$. So
  if $b_k$ diverges, $a_k$ should also diverge by the comparison test.
\end{itemize}

The way we actually use this in practice still involves some
creativity: we have to decide on a ``similar'' series for which we know
the convergence properties.  However, unlike the comparison test,
we can just mechanically take a limit of the ratio of our guess with
our original series, instead of having to ``get our hands dirty''
with inequalities.

\begin{example}
  Is $\sum_{k=1}^\infty \frac{\ln(k)}{k^3}$ convergent or divergent?
  Justify your answer using the limit comparison test.
  \begin{explanation}
    We should expect that this series will converge, because $\ln(k)$
    goes to infinity slower than $k$, so the series is ``no worse''
    than the $p$-series with $p=2$. In the notation of the theorem,
    let
    \[
    a_k = \frac{\ln(k)}{k^3}.
    \]
    We will use the limit comparison test with the series
    \[
    \sum_{k=1}^\infty \frac{1}{k^2},
    \]
    so that
    \[
    b_k = \frac{1}{k^2}.
    \]
    To apply the limit comparison test, examine the limit
    \begin{align*}
    \lim_{k \to \infty} \frac{a_k}{b_k} &= \lim_{k \to \infty} \frac{\ln(k)}{k} \\
    &= 0
    \end{align*}
    Since $\sum_{k=1}^\infty b_k$ is convergent by the
    $p$-series test with $p=2$, then the limit comparison
    test applies, and $\sum_{k=1}^\infty \frac{\ln(k)}{k^3}$
    must also converge.
  \end{explanation}
\end{example}

If the limit comparison test is easier to use than the comparison
test, why do we even have the comparison test?  Sometimes, 
the comparison test is actually more powerful.  The next example
illustrates this idea.

\begin{question}
  Consider $\sum_{k=1}^\infty \frac{1+\sin(k)}{2^k}$.  What happens 
  when you try to use each of the comparison tests with $\sum_{k=1}^\infty \frac{1}{2^k}$?
  \begin{selectAll}
    \choice{The limit comparison test shows that the original series is convergent.}
    \choice{The limit comparison test shows that the original series is divergent.}
    \choice[correct]{The limit comparison test does not apply because the limit in question does not exist.}
    \choice[correct]{The comparison test can be used to show that the original series converges.}
    \choice{The comparison test can be used to show that the original series diverges.}
  \end{selectAll}
  
  \begin{hint}
    $\frac{a_k}{b_k} = 1+\sin(k)$, which does not have a limit as $k
    \to \infty$, so the limit comparison test does not apply.  On the
    other hand, we can see that
    \[
    0<1+\sin(k)<2,
    \]
    so
    \[
    \frac{1+\sin(k)}{2^k} < \frac{2}{2^k},
    \]
    which is a convergent geometric series with $r = \frac{1}{2}<1$.
    Thus the original series converges via the comparison test.
  \end{hint}
\end{question}

Let's pause another moment to consider the task of choosing a test to use 
when analyzing a series for convergence or divergence.  Take a few minutes 
to make a list of all the tests we know so far, and the best situations in which 
to use each of them.


\end{document}


