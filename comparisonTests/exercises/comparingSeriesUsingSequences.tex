\documentclass{ximera}


\graphicspath{
  {./}
  {ximeraTutorial/}
  {basicPhilosophy/}
}

\newcommand{\mooculus}{\textsf{\textbf{MOOC}\textnormal{\textsf{ULUS}}}}

\usepackage{tkz-euclide}\usepackage{tikz}
\usepackage{tikz-cd}
\usetikzlibrary{arrows}
\tikzset{>=stealth,commutative diagrams/.cd,
  arrow style=tikz,diagrams={>=stealth}} %% cool arrow head
\tikzset{shorten <>/.style={ shorten >=#1, shorten <=#1 } } %% allows shorter vectors

\usetikzlibrary{backgrounds} %% for boxes around graphs
\usetikzlibrary{shapes,positioning}  %% Clouds and stars
\usetikzlibrary{matrix} %% for matrix
\usepgfplotslibrary{polar} %% for polar plots
\usepgfplotslibrary{fillbetween} %% to shade area between curves in TikZ
\usetkzobj{all}
\usepackage[makeroom]{cancel} %% for strike outs
%\usepackage{mathtools} %% for pretty underbrace % Breaks Ximera
%\usepackage{multicol}
\usepackage{pgffor} %% required for integral for loops



%% http://tex.stackexchange.com/questions/66490/drawing-a-tikz-arc-specifying-the-center
%% Draws beach ball
\tikzset{pics/carc/.style args={#1:#2:#3}{code={\draw[pic actions] (#1:#3) arc(#1:#2:#3);}}}



\usepackage{array}
\setlength{\extrarowheight}{+.1cm}
\newdimen\digitwidth
\settowidth\digitwidth{9}
\def\divrule#1#2{
\noalign{\moveright#1\digitwidth
\vbox{\hrule width#2\digitwidth}}}






\DeclareMathOperator{\arccot}{arccot}
\DeclareMathOperator{\arcsec}{arcsec}
\DeclareMathOperator{\arccsc}{arccsc}

















%%This is to help with formatting on future title pages.
\newenvironment{sectionOutcomes}{}{}


\author{Jim Talamo}
\license{Creative Commons 3.0 By-bC}


\outcome{}


\begin{document}
\begin{exercise}

One of the primary goals when applying the Comparison Test or the Limit Comparison Test is to compare two given \emph{sequences} and use this comparison to establish whether both \emph{series} either converge or diverge.  

Determine whether the following statement is true or false:

\begin{quote}
Suppose that $\{a_n\}$ and $\{b_n\}$ are both sequences whose terms are positive, and that the terms in the sequences become arbitrarily close together; or, mathematically: $\lim_{n \to \infty} a_n = \lim_{n \to \infty} b_n$.  Then, $\sum_{k=1}^{\infty} a_k$ and $\sum_{k=1}^{\infty} b_k$ either both converge or both diverge.
\end{quote}

\begin{multipleChoice}
\choice{This statement is true.}
\choice[correct]{This statement is false.}
\end{multipleChoice}

There is no reason \emph{a priori} to conclude that we can gain any insight into a \emph{series} by only studying the sequence whose terms we are adding!  

\begin{exercise}
Indeed, the terms in the sequences $\{a_n\}$ given by $a_n =  \frac{1}{2^n}$ and $\{b_n \}$ given by: $b_n=\frac{1}{2^n} + \frac{1}{n}$ are sequences whose terms become arbitrarily close together as $n$ increases since:

\[
a_n-b_n =  \frac{1}{2^n} - \left(\frac{1}{2^n} + \frac{1}{n}\right) = \answer{-\frac{1}{n}}
\]
and thus the difference tends to $0$.  However:

\begin{multipleChoice}
\choice[correct]{$\sum_{k=1}^n a_k$ is a geometric series with $|r| <1$.  It converges.}
\choice{$\sum_{k=1}^n a_k$ is a geometric series with $|r| >1$.  It diverges.}
\end{multipleChoice}

If $\sum_{k=1}^{\infty} b_k$ converges, then $\sum_{k=1}^{\infty} a_k-b_k$ would converge.  However, $\sum_{k=1}^{\infty} a_k-b_k = \sum_{k=1}^{\infty} -\frac{1}{k}$, which: 

\begin{multipleChoice}
\choice[correct]{diverges; it's a $p$-series with $p =1$.}
\choice{converges; it's a $p$-series with $p=1$.}
\end{multipleChoice}

Hence, $\sum_{k=1}^{\infty} b_k$ diverges.  

What conclusion should we draw?

\begin{multipleChoice}
\choice[correct]{We cannot use the fact that the terms in the \emph{sequences} $\{a_n\}$ and $\{b_n\}$ become arbitrarily close to determine if the \emph{series} $\sum_{k=1}^{\infty} a_k$ and $\sum_{k=1}^{\infty} b_k$ either both converge or both diverge.}
\choice{If two sequences $\{a_n\}$ and $\{b_n\}$ have terms that become arbitrarily close, then $\sum_{k=1}^{\infty} a_k$ and $\sum_{k=1}^{\infty} b_k$ will either both converge or both diverge.}
\end{multipleChoice}

Indeed, we have to be very careful to establish how to use terms of \emph{sequences} in order to draw conclusions about the respective sums of their terms.
\begin{exercise}
Recall that if $\{a_n\}$ and $\{b_n\}$ are sequences of nonnegative terms, the limit comparison test tells us that if $\lim_{n \to \infty} \frac{a_n}{b_n} $ is nonzero and finite, then both $\sum_{k=1}^{\infty} a_k$ and $\sum_{k=1}^{\infty} b_k$ either both converge or both diverge.  

A student sees this and makes the following claim:

\begin{quote}
Suppose that $\{a_n\}$ and $\{b_n\}$ are both sequences whose terms are positive and $\lim_{n \to \infty} a_n = \lim_{n \to \infty} b_n$.  Then, we can divide both sides by $\lim_{n \to \infty} b_n$ and write:

\[
\frac{\lim_{n \to \infty} a_n}{\lim_{n \to \infty} b_n} = \frac{\lim_{n \to \infty} b_n}{\lim_{n \to \infty} b_n} =1 \\
\]

Since $\lim_{x \to a} \frac{f(x)}{g(x)} = \frac{\lim_{x \to a} f(x)}{\lim_{x \to a} g(x)}$,  then $\frac{\lim_{n \to \infty} a_n}{\lim_{n \to \infty} b_n} = \lim_{n \to \infty} \frac{a_n}{b_n}$, so using the above:

\[
 \lim_{n \to \infty} \frac{a_n}{b_n} = 1
\]

Hence, by the Limit Comparison Test, $\sum_{k=1}^{\infty} a_k$ and $\sum_{k=1}^{\infty} b_k$ either both converge or both diverge.

\end{quote}
What is wrong with this student's argument?  


\end{exercise}
\end{exercise}
\end{exercise}

\end{document}
