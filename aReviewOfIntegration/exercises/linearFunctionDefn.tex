\documentclass{ximera}

\graphicspath{
  {./}
  {ximeraTutorial/}
  {basicPhilosophy/}
}

\newcommand{\mooculus}{\textsf{\textbf{MOOC}\textnormal{\textsf{ULUS}}}}

\usepackage{tkz-euclide}\usepackage{tikz}
\usepackage{tikz-cd}
\usetikzlibrary{arrows}
\tikzset{>=stealth,commutative diagrams/.cd,
  arrow style=tikz,diagrams={>=stealth}} %% cool arrow head
\tikzset{shorten <>/.style={ shorten >=#1, shorten <=#1 } } %% allows shorter vectors

\usetikzlibrary{backgrounds} %% for boxes around graphs
\usetikzlibrary{shapes,positioning}  %% Clouds and stars
\usetikzlibrary{matrix} %% for matrix
\usepgfplotslibrary{polar} %% for polar plots
\usepgfplotslibrary{fillbetween} %% to shade area between curves in TikZ
\usetkzobj{all}
\usepackage[makeroom]{cancel} %% for strike outs
%\usepackage{mathtools} %% for pretty underbrace % Breaks Ximera
%\usepackage{multicol}
\usepackage{pgffor} %% required for integral for loops



%% http://tex.stackexchange.com/questions/66490/drawing-a-tikz-arc-specifying-the-center
%% Draws beach ball
\tikzset{pics/carc/.style args={#1:#2:#3}{code={\draw[pic actions] (#1:#3) arc(#1:#2:#3);}}}



\usepackage{array}
\setlength{\extrarowheight}{+.1cm}
\newdimen\digitwidth
\settowidth\digitwidth{9}
\def\divrule#1#2{
\noalign{\moveright#1\digitwidth
\vbox{\hrule width#2\digitwidth}}}






\DeclareMathOperator{\arccot}{arccot}
\DeclareMathOperator{\arcsec}{arcsec}
\DeclareMathOperator{\arccsc}{arccsc}

















%%This is to help with formatting on future title pages.
\newenvironment{sectionOutcomes}{}{}

\author{Jim Talamo}
\license{Creative Commons 3.0 By-NC}
\outcome{Determine if a function is linear}
%THIS EXERCISE SHOULD COME BEFORE ``functionsoflinearfunctions", which is designed for students to compute indefinite integrals of functions of linear functions without performing an explicit substitution
\begin{document}
\begin{exercise}
We say that a function $f(x)$ is \emph{linear in x} if we can write $f(x)$ in the form $f(x) = ax+b$ for constants $a$ and $b$.  If a function is not linear, we say that it is \emph{nonlinear in x}.

Determine if the following functions are linear or nonlinear:

For the function $f(x) = 5x+3$, $f(x)$ is \wordChoice{\choice[correct]{linear} \choice{nonlinear}} in $x$.

For the function $f(x) = \sin(2x+1)$, $f(x)$ is \wordChoice{\choice{linear} \choice[correct]{nonlinear}} in $x$.

For the function $f(x) = 2x^2+3$, $f(x)$ is \wordChoice{\choice{linear} \choice[correct]{nonlinear}} in $x$.

For the function $f(x) = (2x+1)^2-4x^2$, $f(x)$ is \wordChoice{\choice[correct]{linear} \choice{nonlinear}} in $x$.

For the function $f(x) = (\sqrt{x}+1)^2-2\sqrt{x}$, $f(x)$ is \wordChoice{\choice[correct]{linear} \choice{nonlinear}} in $x$.

For the function $f(x) = \frac{\sin(x)+5}{3\sin(x)}$, $f(x)$ is \wordChoice{\choice{linear} \choice[correct]{nonlinear}} in $x$.

\end{exercise}
\end{document}