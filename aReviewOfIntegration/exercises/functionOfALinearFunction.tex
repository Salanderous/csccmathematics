\documentclass{ximera}

\graphicspath{
  {./}
  {ximeraTutorial/}
  {basicPhilosophy/}
}

\newcommand{\mooculus}{\textsf{\textbf{MOOC}\textnormal{\textsf{ULUS}}}}

\usepackage{tkz-euclide}\usepackage{tikz}
\usepackage{tikz-cd}
\usetikzlibrary{arrows}
\tikzset{>=stealth,commutative diagrams/.cd,
  arrow style=tikz,diagrams={>=stealth}} %% cool arrow head
\tikzset{shorten <>/.style={ shorten >=#1, shorten <=#1 } } %% allows shorter vectors

\usetikzlibrary{backgrounds} %% for boxes around graphs
\usetikzlibrary{shapes,positioning}  %% Clouds and stars
\usetikzlibrary{matrix} %% for matrix
\usepgfplotslibrary{polar} %% for polar plots
\usepgfplotslibrary{fillbetween} %% to shade area between curves in TikZ
\usetkzobj{all}
\usepackage[makeroom]{cancel} %% for strike outs
%\usepackage{mathtools} %% for pretty underbrace % Breaks Ximera
%\usepackage{multicol}
\usepackage{pgffor} %% required for integral for loops



%% http://tex.stackexchange.com/questions/66490/drawing-a-tikz-arc-specifying-the-center
%% Draws beach ball
\tikzset{pics/carc/.style args={#1:#2:#3}{code={\draw[pic actions] (#1:#3) arc(#1:#2:#3);}}}



\usepackage{array}
\setlength{\extrarowheight}{+.1cm}
\newdimen\digitwidth
\settowidth\digitwidth{9}
\def\divrule#1#2{
\noalign{\moveright#1\digitwidth
\vbox{\hrule width#2\digitwidth}}}






\DeclareMathOperator{\arccot}{arccot}
\DeclareMathOperator{\arcsec}{arcsec}
\DeclareMathOperator{\arccsc}{arccsc}

















%%This is to help with formatting on future title pages.
\newenvironment{sectionOutcomes}{}{}

\author{Jim Talamo}
\license{Creative Commons 3.0 By-NC}
\outcome{Compute indefinite integrals of functions of linear functions without using a substitution}
\begin{document}
\begin{exercise}

Functions of linear functions arise frequently in this course as well as in others.  While an explicit substitution can be performed to calculate indefinite integrals of functions of linear functions, it is very helpful to be able to compute these quickly without carrying out the explicit substitution!  

Compute the following indefinite integrals WITHOUT carrying out an explicit substitution:

\begin{prompt} (Use $C$ for the constant of integration) \end{prompt}

\[\int \sin(2x) \d x = \begin{prompt}\answer{-\frac{1}{2}\cos(2x)+C}\end{prompt}\]

\[\int \frac{2}{3x+4} \d x = \begin{prompt}\answer{\frac{2}{3}\ln|3x+4|+C}\end{prompt}\]

\begin{hint}
Did you remember the absolute value?
\end{hint}

\[\int e^{\frac{x}{4}} \d x = \begin{prompt}\answer{4e^{\frac{x}{4}}+C}\end{prompt}\]

\[\int \frac{2}{\sqrt{4x+5}} \d x = \begin{prompt}\answer{\sqrt{4x+5}+C}\end{prompt}\]

\[\int \frac{1}{9x^2+4} \d x = \begin{prompt}\answer{\frac{1}{6} \arctan\left(\frac{3x}{2}\right)+C}\end{prompt}\]
\[\mbox{Hint: Write $9x^2$ as $(3x)^2$.}\]





\end{exercise}
\end{document}