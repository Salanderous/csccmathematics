\documentclass{ximera}

\graphicspath{
  {./}
  {ximeraTutorial/}
  {basicPhilosophy/}
}

\newcommand{\mooculus}{\textsf{\textbf{MOOC}\textnormal{\textsf{ULUS}}}}

\usepackage{tkz-euclide}\usepackage{tikz}
\usepackage{tikz-cd}
\usetikzlibrary{arrows}
\tikzset{>=stealth,commutative diagrams/.cd,
  arrow style=tikz,diagrams={>=stealth}} %% cool arrow head
\tikzset{shorten <>/.style={ shorten >=#1, shorten <=#1 } } %% allows shorter vectors

\usetikzlibrary{backgrounds} %% for boxes around graphs
\usetikzlibrary{shapes,positioning}  %% Clouds and stars
\usetikzlibrary{matrix} %% for matrix
\usepgfplotslibrary{polar} %% for polar plots
\usepgfplotslibrary{fillbetween} %% to shade area between curves in TikZ
\usetkzobj{all}
\usepackage[makeroom]{cancel} %% for strike outs
%\usepackage{mathtools} %% for pretty underbrace % Breaks Ximera
%\usepackage{multicol}
\usepackage{pgffor} %% required for integral for loops



%% http://tex.stackexchange.com/questions/66490/drawing-a-tikz-arc-specifying-the-center
%% Draws beach ball
\tikzset{pics/carc/.style args={#1:#2:#3}{code={\draw[pic actions] (#1:#3) arc(#1:#2:#3);}}}



\usepackage{array}
\setlength{\extrarowheight}{+.1cm}
\newdimen\digitwidth
\settowidth\digitwidth{9}
\def\divrule#1#2{
\noalign{\moveright#1\digitwidth
\vbox{\hrule width#2\digitwidth}}}






\DeclareMathOperator{\arccot}{arccot}
\DeclareMathOperator{\arcsec}{arcsec}
\DeclareMathOperator{\arccsc}{arccsc}

















%%This is to help with formatting on future title pages.
\newenvironment{sectionOutcomes}{}{}

\author{Jim Talamo}
\license{Creative Commons 3.0 By-NC}
\outcome{Explore the integrals that involve inverse tangents}
\begin{document}
\begin{exercise}
The following exercise gives practice working with the rule: 

\[\int \frac{1}{x^2+a^2} \d x = \frac{1}{a} \arctan\left(\frac{x}{a}\right) +C, \]

which will recur frequently in the course.  Using this formula and the rules for antidifferentiation, find the following:

\begin{prompt} (Use $C$ for the constant of integration) \end{prompt}

\begin{align*}
\int \frac{4}{x^2+1} \d x &= \answer{4\arctan(x)+C}\\
\int \frac{1}{x^2+4} \d x &= \answer{\frac{1}{2}\arctan\left(\frac{x}{2}\right)+C}\\
\int \frac{1}{4x^2+1} \d x &= \answer{\frac{1}{2} \arctan(2x)+C}\\
\end{align*}

Now, try one a little more complicated:

\[ \int \frac{3}{9x^2+16} \d x = \answer{\frac{1}{4}\arctan\left(\frac{3x}{4}\right)+C} \]

\begin{prompt} Use $C$ for the constant of integration. \end{prompt}
\end{exercise}
\end{document}